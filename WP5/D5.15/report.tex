\documentclass{deliverablereport}

\deliverable{hpc}{GAP-HPC-report}
\deliverydate{XX/YY/201Z}
\duedate{31/08/2019 (M48)}
\author{Author names}

\begin{document}
\maketitle
% This will be the abstract, fetched from the github description
\githubissuedescription

% write the report here

% Original list of sections and subsections created from
% https://github.com/OpenDreamKit/OpenDreamKit/issues/113

\section{Developments in the core GAP system}

\subsection{libGAP: allowing 3rd party code to link GAP as a library}

GAP 4.10.0 (November 2018) is the first official GAP release
which provided an experimental way to allow 3rd party code to 
link GAP as a library. It was based on the libGAP code by SageMath, 
but different: while we aim to provide the same functionality, 
we do not rename any symbols, and we do not provide the same API. 

Since then, we improved the robustness of our libGAP implementation
in GAP 4.10.1 (February 2019) and GAP 4.10.2 (June 2019) releases,
and extended its API with new functionality (See Chapter 2 
``Changes between GAP 4.9 and GAP 4.10'' of the 
``GAP - Changes from Earlier Versions'' manual for the detailed
descriptions). This allowed SageMath to drop its custom 
modifications for GAP and use the official, documented and regularly 
tested GAP interface instead, starting from SageMath 8.6 (January 2019).
% See SageTrac ticket 22626 made that happen; it was merged in 
% SageMath 8.6.beta0, so SageMath 8.6 had it but not SageMath 8.5.

\subsection{other changes}

TODO: This section should highlight most important changes from
releases overview at \url{https://www.gap-system.org/Manuals/doc/changes/chap0.html}.
All GAP 4.7 releases were made before the beginning of the project,
so we start from GAP 4.8 here.

% 4.10 https://www.gap-system.org/Manuals/doc/changes/chap2.html

% 4.9 https://www.gap-system.org/Manuals/doc/changes/chap3.html

% 4.8 https://www.gap-system.org/Manuals/doc/changes/chap4.html

% What about making https://www.gap-system.org/Manuals/doc/changes/manual.pdf 
% an electronic appendix to the deliverable?

\section{High-performance computing with GAP}

\subsection{HPC-GAP: multithreaded programming in GAP}

GAP 4.9.1 (May 2018) for the first time included experimental code to 
support multithreaded programming in GAP, dubbed HPC-GAP. The HPC-GAP 
codebases diverged from the original GAP code during its development. 
Unifying the codebases and incorporating the HPC-GAP code back into the 
mainstream GAP version considerably simplified further development of 
HPC-GAP. It also provided users an opportunity to star to experiment 
with HPC-GAP, which comes together with the new manual book called 
``HPC-GAP Reference Manual'' located in the `doc/hpc` directory.

% Release announcement: https://www.gap-system.org/Manuals/doc/changes/chap3.html#X7F52B77B7DBACC17

% HPC-GAP manual: https://www.gap-system.org/Manuals/doc/hpc/chap0.html

% What about making https://www.gap-system.org/Manuals/doc/hpc/manual.pdf 
% an electronic appendix to the deliverable?

\section{meataxe64: high-performance linear algebra over finite fields}

WHO: Steve

This will describe the GAP interface to meataxe64: \url{https://meataxe64.wordpress.com/}.

% TODO: make a release at https://github.com/gap-packages/meataxe64/ and publish it at https://gap-packages.github.io/meataxe64/ 

\section{Packages}

TODO: Introduction - brief background details on what GAP packages are.

\subsection{Package ecosystem}

WHO: Alex

This will describe the growth of the community, the improvements in tests, frequency
of package updates, tools that appeared and helped to package authors. The narrative
will be based recent on GAP posters and talk in May in Manchester, with visualisations
used there.

\subsection{GAP package manager}

WHO: Michael

Describe \url{https://gap-packages.github.io/PackageManager/}


\section{Jupyter interface to GAP}

WHO: Alex, with help from and Michael

TODO: Refer to previous deliverables and describe how their outputs
evolved and any impacts they achieved

JupyterKernel has been reported in DX.Y. Since that, there were
several more releases of the kernel, associated with new GAP
releases, improving its robustness. 

JupyterKernel lead to several follow-up developments, their
functionality will be described below: Francy, JupyterViz,
other work by Nathan Carter (also mention related work on
GroupExplorer).

Have used in in teaching and in explaining how to share 
reproducible experiments that use GAP. Using Binder
allows to try GAP in Jupyter online. Provide details
and consider including an appendix with an example 
demonstrating the setup.

\section{GAP distributions}

WHO: Alex

Main distributions

Regression testing

Windows distribution and installer migrated from Windows 7 to Windows 10

Alternative distributions: Docker, Homebrew, 

Opportunity to try GAP online with Jupyter notebook running on Binder.

Improved testing

Code coverage, various sets of automated Travis builds

This is based on Docker containers for various branches.

\section{Community building, Training and Dissemination}

WHO: Alex

Describe all training events, Software Carpentry lesson, other workshops

\section{Interfaces, WP6}

WHO: Michael

Fix title of the section.

High-level interoperability

Persistent memoisation

\section{A collection of demonstrators}

For example:

Reproducible experiments with Jupyter

``full-stack semigroups''

persistent memoisation

databases

\end{document}

%%% Local Variables:
%%% mode: latex
%%% TeX-master: t
%%% End:

