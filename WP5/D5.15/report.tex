\documentclass{deliverablereport}

\deliverable{hpc}{GAP-HPC-report}
\deliverydate{XX/YY/201Z}
\duedate{31/08/2019 (M48)}
\author{Author names}

\begin{document}
\maketitle
% This will be the abstract, fetched from the github description
\githubissuedescription

% write the report here

% Original list of sections and subsections created from
% https://github.com/OpenDreamKit/OpenDreamKit/issues/113

\section{Developments in the core GAP system}

\subsection{libGAP: allowing 3rd party code to link GAP as a library}

GAP 4.10.0 (November 2018) is the first official GAP release
which provided an experimental way to allow 3rd party code to 
link GAP as a library. It was based on the libGAP code by SageMath, 
but different: while we aim to provide the same functionality, 
we do not rename any symbols, and we do not provide the same API. 

Since then, we improved the robustness of our libGAP implementation
in GAP 4.10.1 (February 2019) and GAP 4.10.2 (June 2019) releases,
and extended its API with new functionality (See Chapter 2 
``Changes between GAP 4.9 and GAP 4.10'' of the 
``GAP - Changes from Earlier Versions'' manual for the detailed
descriptions). This allowed SageMath to drop its custom 
modifications for GAP and use the official, documented and regularly 
tested GAP interface instead, starting from SageMath 8.6 (January 2019).
% See SageTrac ticket 22626 made that happen; it was merged in 
% SageMath 8.6.beta0, so SageMath 8.6 had it but not SageMath 8.5.


%
%\subsection{other changes}
%
% see releases overview at https://www.gap-system.org/Manuals/doc/changes/chap0.html)

% 4.10 https://www.gap-system.org/Manuals/doc/changes/chap2.html

% 4.9 https://www.gap-system.org/Manuals/doc/changes/chap3.html

% 4.8 https://www.gap-system.org/Manuals/doc/changes/chap4.html

% All GAP 4.7 releases were made before the beginning of the project
% so we start from GAP 4.8 here

% What about making https://www.gap-system.org/Manuals/doc/changes/manual.pdf 
% an electronic appendix to the deliverable?

%\section{HPC-GAP}
%
% merging into mainstream GAP releases 
% see https://www.gap-system.org/Manuals/doc/changes/chap3.html#X7F52B77B7DBACC17

% HPC-GAP manual: https://www.gap-system.org/Manuals/doc/hpc/chap0.html

% What about making https://www.gap-system.org/Manuals/doc/hpc/manual.pdf 
% an electronic appendix to the deliverable?

%meataxe64 interface (https://github.com/gap-packages/meataxe64 and https://meataxe64.wordpress.com/)

% TODO: make a release at https://github.com/gap-packages/meataxe64/ and publish it at https://gap-packages.github.io/meataxe64/ 
%
%\section{Packages}
%
% package manager

% Michael to describe https://gap-packages.github.io/PackageManager/
%
% improving the health of the package ecosystem

% Alex to describe and add pictures

% the narrative based on GAP posters and talk in May in Manchester, with visualisations

%\section{Jupyter and derivatives}

% WHO: Alex (help from and Michael?)
%
%JupyterKernel
%
%Francy
%
%JupyterViz
%
%other packages by @nathancarter

% Groups explorer
%
%\section{GAP distributions}

% WHO: Alex

%Docker and other alternative distributions
%
%Improved testing
%
% Code coverage, various sets of automated Travis builds
%
%\section{Events}
%
% WHO: Alex
%
%community building events
%
%training events, Software Carpentry lesson, other workshops
%
%\section{Interfaces, WP6}
%
% WHO: Michael
%
%high-level interoperability
%
%persistent memoisation
%
%\section{A collection of demonstrators}
%
% For example:
%
%"full-stack semigroups"
%
% persistent memoisation
%
% databases

\end{document}

%%% Local Variables:
%%% mode: latex
%%% TeX-master: t
%%% End:

