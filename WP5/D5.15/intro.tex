%% \TODO{Can we make more of semigroups -- it's an example of \ODK
%%   collaboration and of high performance \GAP package used directly for research?
%%   \qquad MT: I added to the paragraph (core-gap.tex:124) with details about this.}

\TODO{Abstract!!}

\section{Introduction}
This report describes the work undertaken in connection with Task~5.2,
  a wide range of developments within the \GAP system to meet the
  general objective of workpackage 5 (``improve the performance of the
  computational components of \ODK''). We have worked on supporting
  use of a range of High Performance Computing environments,
  prioritizing those most accessible to our users, namely multi-core
  systems, adding to our existing support for clusters and cloud
  computing. We have also made fundamental performance improvements
  and improvements to our tools, processes and to the programming
  language, that benefit all \GAP users.

In a system such as \GAP, which is actively developed by a widely
distributed team, only a few of whom are part of the \ODK project, any
developments which are to be useful to users of \GAP (whether as a component of
\ODK or not) need to be maintainable, and
integrate well with the general flow of \GAP development, which itself
has to be nurtured and supported. Because of this, it is neither
possible nor desirable to clearly delineate every development which
is part of Task~5.2 from every one which is not. On the other hand,
the impetus and resources from the \ODK project have meant that \ODK
goals have pervaded a great deal of \GAP's evolution over the last
four years.

In light of this, this report will necessarily include  a broad
survey of all the major developments in \GAP and its package ecosystem
over the period, but we will concentrate on, and highlight, links between
these developments and \ODK goals. Work which is specifically part of
other work packages is reported in appropriate deliverables, but these
are mainly quite focused, leaving a broad range of material for this report.

\subsection{Context and Background}

Recall that \GAP (``Groups, Algorithms, Programming'') is a software system for
computational discrete mathematics, with its origins in finite group
theory. The system has been developed continually since 1986, with the
first public release in 1988. Throughout its life it has been made
available free of charge for use, extension and non-commercial
redistribution. It is widely used by researchers in mathematics,
computer science and other disciplines, and has been cited in
\href{https://www.gap-system.org/Doc/Bib/bib.html}{more than 3000
  research papers}. It is also widely used in teaching.

Thus far, \GAP has
normally been used via a single-threaded text-based ``read--eval--print loop'' user interface, with the
same \GAP language serving both for the interface and for the
implementation of much of the system. A research user 
may use the system purely interactively, as a ``desk calculator'', but
typically will extend the system with a small or large collection of
\GAP functions and complete their research calculations by interacting
with those. One of the goals of \GAP-related work in \ODK is to offer
a range of flexible alternatives to this in line with current
practices in other areas, and more suited for modern high-performance
computers.

A key feature of \GAP today is the rapidly growing ecosystem of
contributed extensions called ``packages'' (145 are redistributed with
\GAP 4.10.2). These packages may: extend mathematical functionality of
the system by implementing new algorithms; provide mathematical
datasets, together with search and access functions or provide
enhanced tools for users and/or developers. Packages can be published
independently of the core \GAP system and each has a separate
authorship. Packages that pass our checks (for example having
documentation and tests and not causing regressions in the core system
or clashes with other packages) can (at the author's request) be
automatically redistributed with \GAP releases. Again at the author's
choice, they may be
submitted for ``refereeing'', a process designed to be analogous to the
refereeing of scientific papers, providing both quality control for
the user and recognition for the author.

For the last decade or more, the majority of the
development work in the \GAP ecosystem has been in packages.
Another area of \GAP development under \ODK is to ensure
that the package system (including the infrastructure used to gather,
test and release packages) scales as the number of packages grows.


Prior to the start of \ODK, a number of initiatives had developed
alternative models for the use of \GAP. \SageMath (\Sage hereafter)
has used \GAP for much of its group theoretic functionality since
2006. One benefit of that was that it allowed the use of \GAP
functionality through \Sage's notebook user interface. In the Framework~6 project
``SCIEnce: Symbolic Computation Infrastructure for Europe'' we
developed a powerful and flexible remote procedure call interface to
\GAP using the \href{https://www.openmath.org/}{OpenMath} data representation format, called
\href{https://www.openmath.org/standard/scscp/}{SCSCP}
(Symbolic Computation Software Composability Protocol) one of
whose roles was to support coarse-grained distributed computation. In
a UK-funded software development project, \HPCGAP, we began the
development of an intrinsically multi-threaded version of \GAP. As
well as new software, these projects gave us insight into the parallel
and high-performance computing needs of our user community, leading
us, for instance, to remain more focused on multi-core servers and
\textit{ad hoc} clusters (such as a mathematical research group or
department might provide) than on ``supercomputers''. Even on these
systems, interactive use rather than ``batch'' computation remains the
norm, and we aim to support that, and enhance it through Jupyter
notebooks rather than force users to a model that does not suit them.

The central thrust of the involvement of \GAP in \ODK is increased
diversity of users and models of use. To this end, in WP3~we have
added support for direct use of \GAP via Jupyter notebooks as an
alternative to our 1980s text-based user interface. In this work package we report
work intended to support \GAP users in performing larger and more
demanding computations. This includes general work on increasing the
performance and robustness of \GAP and its packages, continued
development of our support for shared memory parallel computing via
\HPCGAP and MeatAxe64 and maintenance of our support for distributed
memory parallel computing using the SCSCP package \cite{SCSCP}.
We also report enhancements to the \GAP language and to
\GAP's development and debugging tools and our support for package
developers. More demanding computations requires more complex software
to be written, debugged, optimized and shared, and these changes
enable that.

\subsection{Overview}
This report is divided into sections as follows: 
section \ref{sec:hpc} reports on two major initiatives supporting
parallel computation on multi-core shared memory
systems: \HPCGAP and Meataxe64. Section  \ref{sec:core-gap}
reports a wide range of relevant general developments in the core \GAP
system, highlighting \texttt{libGAP} which gives a much faster and
more reliable link to other systems and the new linguistic reflection
facilities which will provide a basis for future developments in
optimization and automatic parallelization. Section \ref{sec:packages}
similarly surveys relevant developments among the many new or updated packages.
Section \ref{sec:gap-infra} discusses our work aimed at improving
robustness and reliability, both of \GAP and its packages. This
includes both new tool support and new processes. 

