\section{Developments in the Core \GAP System}\label{sec:core-gap}

This Section highlights some of the most important
changes to the core \GAP system released
during the project which contribute to \ODK goals. Other important developments are
included in \GAP packages and described in later sections.  More
detailed descriptions of each major and minor \GAP release, with links
to the corresponding source code changes on GitHub, are contained in
the ``GAP - Changes from Earlier Versions'' document. This document is
redistributed with \GAP and is also available on the \GAP web site at
\url{https://www.gap-system.org/Manuals/doc/changes/chap0.html} and
included as an Appendix~\ref{sec:changes-manual} to this report.

This work includes many contributions from collaborators outside \ODK,
whose authorship can be seen from the version control history.

\subsection{libGAP: Allowing 3rd Party Code to Link \GAP Efficiently as a Library}\label{libgap}

The initial connection between \Sage and \GAP was developed around
2006, and relied on running a full \GAP process with which \Sage
communicated through UNIX pipes, creating \GAP input as text and then parsing
the text which was output from \GAP. This enabled an initial connection, but
introduced considerable overhead converting objects to and from text
representations and passing them between processes, and considerable
unreliability, since \Sage was essentially using \GAP through an
interface designed for humans, rather than programs.

A better solution is to provide a C API enabling \GAP to be integrated
into the \Sage process, and allowing a \GAP session to be started,
objects created, functions called and results retrieved, via C function
calls. This is, however, not a mode of operation for which \GAP was
ever designed, and implementing it in a robust and flexible way
required considerable adaptations to the system, essentially
disentangling the computational engine from the user interface, while retaining,
for example, the ability to dynamically load C code from \GAP
packages into the process that was using \software{libGAP}. Another complication
is the need to make sure that the \GAP memory manager does not delete
objects which the calling process will want in the future.

Based on an initial version developed within the \Sage team, a fully
general-purpose and flexible version was released in \GAP 4.10.0
(Month 39).

Subsequent releases have improved and will continue to improve the
robustness of \software{libGAP}, and to extend its API with new functionality
(For the detailed descriptions, see Chapter 2 
``Changes between \GAP 4.9 and \GAP 4.10'' of the 
``\GAP\,- Changes from Earlier Versions'' manual,
provided in Appendix~\ref{sec:changes-manual}).
This has allowed \Sage to drop its custom 
modifications to \GAP and use an official, documented and regularly 
tested \GAP interface instead, starting from SageMath 8.6 (Month 41).

% See SageTrac ticket 22626 made that happen; it was merged in 
% SageMath 8.6.beta0, so SageMath 8.6 had it but not SageMath 8.5.

\subsection{Linguistic Reflection -- Supporting Transformation and
  Optimisation of \GAP Functions at Runtime}\label{syntaxtree}
The \GAP programming language, in which much of the system is written,
is interpreted and dynamically typed (somewhat like the better-known
Python). It has a unique ``method selection'' system
(\cite{BreuerLinton98}) in which the type of an object reflects
mathematical information learned about it at runtime; method
selection uses the known types of all arguments in order to choose a method.

A compiler for the language (converting \GAP code
into C extensions to the \GAP kernel) has been available for many
years as part of \GAP, but gives surprisingly little speedup on most code, and has not
seen widespread adoption. The main reason for this is that objects'
types must still be checked, and method selection decisions made, at
run-time, even in compiled code.

\Sage, implemented initially in Python, faced many of the same
problems, and addressed them by adopting, and promoting
development of, the \href{http://cython.org}{Cython} project, which
extends the Python language with C-like types and constructions to
allow more effective compilation.

In WP5 the \GAP team have sought a natural and usable way to deliver
similar benefits for the \GAP language (bearing in mind the relative
sizes of the \GAP and Python user and developer communities). After
some experiments and a great deal of discussion across the community,
we have settled on ``Linguistic reflection'', combined with code
annotations (pragmas) as an approach that offered a good return of
performance and other benefits for the effort invested.

The key infrastructural tools for both of these have now been added to
the \GAP language and interpreter. The new \verb|SyntaxTree| function
``unpacks'' the workspace representation of a \GAP function,
converting it into a data structure which can be easily manipulated by
\GAP, and converted back into a new executable \GAP function when
needed. Code annotations in the form of pragmas are retained through
this process, so that the programmer can use them to guide in
program transformation, and different software transformers can use
them to communicate. All of this functionality is publicly available
now in the development version of \GAP on GitHub and will be included
in \GAP 4.11 (expected November 2019).

Future versions will build on this for optimization, OpenMP-style
semi-automatic parallelization and (our original motivating
application) to produce specialized versions of functions which reduce
the amount of run-time method selection required, allowing them to be
executed more quickly and compiled to C more effectively.

As a small example, consider the following code fragment:


\begin{Small}
\begin{verbatim}
foo := function(n, x, y)
    local z,i;
    z := 0;
    for i in [1..n] do
        #% x and y always integers
        z := z+Gcd(x,y);
    od;
    return z;
end;
\end{verbatim}
\end{Small}

The \verb|#%| here denotes a pragma. This can be unpacked to produce
the cumbersome, but easily processed data structure shown in Figure \ref{fig:syntaxtree}.

\begin{figure}[!ht]
  \begin{mdframed}
  \begin{Small}
\begin{verbatim}
gap> t:= SyntaxTree(foo)!.tree;
rec( name := "foo", nams := [ "n", "x", "y", "z", "i" ], narg := 3, 
  nloc := 2, 
  stats := 
    rec( 
      statements := 
        [ 
          rec( lvar := 4, rhs := rec( type := "EXPR_INT", value := 0 ), 
              type := "STAT_ASS_LVAR" ), 
          rec( 
              body := 
                [ 
                  rec( type := "STAT_PRAGMA", 
                      value := "% x and y always integers" ), 
                  rec( lvar := 4, 
                      rhs := 
                        rec( 
                          left := 
                            rec( lvar := 4, type := "EXPR_REF_LVAR" ), 
                          right := 
                            rec( 
                              args := 
                                [ 
                                  rec( lvar := 2, 
                                      type := "EXPR_REF_LVAR" ), 
                                  rec( lvar := 3, 
                                      type := "EXPR_REF_LVAR" ) ], 
                              funcref := 
                                rec( gvar := "Gcd", 
                                  type := "EXPR_REF_GVAR" ), 
                              type := "EXPR_FUNCCALL_2ARGS" ), 
                          type := "EXPR_SUM" ), 
                      type := "STAT_ASS_LVAR" ) ], 
              collection := 
                rec( first := rec( type := "EXPR_INT", value := 1 ), 
                  last := rec( lvar := 1, type := "EXPR_REF_LVAR" ), 
                  type := "EXPR_RANGE" ), type := "STAT_FOR_RANGE2", 
              variable := rec( lvar := 5, type := "EXPR_REF_LVAR" ) ), 
          rec( obj := rec( lvar := 4, type := "EXPR_REF_LVAR" ), 
              type := "STAT_RETURN_OBJ" ) ], type := "STAT_SEQ_STAT3" )
    , type := "EXPR_FUNC", variadic := false )
\end{verbatim}
  \end{Small}
  \end{mdframed}
  \caption{A Full Syntax Tree Data Structure}\label{fig:syntaxtree}
  \end{figure}
  

This data structure contains all of the necessary information for
a simple \GAP program to locate the pragma and, using the information
in it, replace the general function \verb|Gcd| by the specialized
\verb|GcdInt| statically.  Doing this by hand for the moment, we can
produce the session in Figure \ref{fig:using-syntaxtree}. This shows 
the full round trip from one executable \GAP function \verb|foo| to an
optimised one \verb|bar| which completes in approximately one fifth the time.

\begin{figure}[!ht]
  \begin{mdframed}
\begin{Small}
\begin{verbatim}
gap> foo := function(n, x, y)
>     local z,i;
>     z := 0;
>     for i in [1..n] do
>         #% x and y always integers
>         z := z+Gcd(x,y);
>     od;
>     return z;
> end;
function( n, x, y ) ... end
gap> t := SyntaxTree(foo)!.tree;;
gap> u := StructuralCopy(t);;
gap> u.stats.statements[2].body[1].value := "% Gcd specialised to GcdInt";;
gap> u.stats.statements[2].body[2].rhs.right.funcref.gvar := "GcdInt";;
gap> bar := SYNTAX_TREE_CODE(u);;
gap> Print(bar);
function ( n, x, y )
    local z, i;
    z := 0;
    for i in [ 1 .. n ] do
        #% Gcd specialised to GcdInt
        z := z + GcdInt( x, y );
    od;
    return z;
end
gap> foo(10^7, 102334155, 165580141);; time
2245
gap> bar(10^7, 102334155, 165580141);; time;
424
\end{verbatim}
\end{Small}
  \end{mdframed}
  \caption{An example of manual optimisation using pragmas and Syntax
    Trees}\label{fig:using-syntaxtree}
  \end{figure}
  

\textbf{Note:} It is important to understand that this is \emph{not} a
source code level transformation. Automated transformation tools built
on this foundation do
not need to parse or emit \GAP syntax, and all questions of scope,
including bindings in closures, are already resolved. A small planned
extension will add profiling information to the syntax tree when available.

Other applications for this technology include automated
parallelization (in combination with \HPCGAP) and traditional code optimizations.

While we would have preferred to have had more time to exploit these
facilities within the project, it was important to achieve consensus
around a solution that fit the needs of \GAP and its community and
will provide a basis for developments for years to come.

\subsection{Developments in \GAP 4.8 (Month 8)}\label{gap-4.8}

\GAP 4.8 was the first major release of \GAP published
after the start\footnote{In the report we
refer to months of the project, i.e. Month 1 is September 2015, while Month 48 is August 2019.}
of the \ODK project. 
In preparation for the planned integration of the \HPCGAP branch
(supporting \GAP language-level threading) into
the core system, some language extensions were backported to core
\GAP at this stage. 
This helped to unify the codebases of \GAP~4 and \HPCGAP
by allowing \HPCGAP code to be read in a single-threaded
\GAP session. For example, the
\verb|atomic|, \verb|readonly| and \verb|readwrite| keywords used for
controlling access to shared data in \HPCGAP were
added as ``no-ops'' to the regular \GAP language.


In addition, this release also provided support for much enhanced
profiling, essential to all our performance engineering work.
Previously, \GAP\ allowed tracking of the CPU time spent
executing each function, but the new features in this release
reduced this granularity to individual lines of \GAP
code. Related developments also made it easier to track time spent on
individual lines of C code in the \GAP kernel. A further related
feature makes it very easy to measure ``test coverage'' -- the
proportion of the system that has actually been executed by a given
set of tests. The \software{Profiling} package \cite{profiling} by Christopher Jefferson
(USTAN) provided excellent tools for transforming these profiles into
various human-readable forms. These features supported later work on
enhancing both the performance and the reliability of \GAP, through
easier identification of hot-spots and easier measurement of test
quality.

This release also included some language extensions which would allow
users to write more efficient and readable code:

\begin{itemize}
  \item support for \emph{partially
  variadic functions} similar to those allowed in Python, allowing
function expressions like
\verb|function( a, b, c, x... ) ... end;|
which requires at least three arguments \verb|a|, \verb|b| and
\verb|c| and assigns the first three to \verb|a|, \verb|b| and \verb|c|
and then a list containing any remaining ones to \verb|x|.
\item more flexible list indexing, allowing users to install methods
  for multiple indexing (\verb|m[i,j]| instead of \verb|m[i][j]| for
  arrays) and more varied indices (such as \verb|l[-1]| to access a
  doubly-infinite virtual list \verb|l|). As well as being more
  flexible, this is a step towards supporting more efficient array
  implementations in which ``row objects'' (like \verb|m[i]| in
  \verb|m[i][j]|) might not exist.
\end{itemize}

\GAP 4.8 had seven public releases between Month 6 and Month 24, and
was replaced by \GAP 4.9 in Month 33.

\subsection{Developments in \GAP 4.9 (Month 33)}\label{gap-4.9}

\GAP 4.9 continued the restructuring and modernizing of the core \GAP
system, aiming at the greater reliability, flexibility and performance
which would be needed for \GAP users to gain the full advantage of the \ODK ecosystem.
For example, we removed our old home-grown large integer arithmetic
implementation, and committed fully to the state of the art external
library \href{https://gmplib.org/}{\software{GMP}}. This
move continues our cautious programme of exploiting existing free
software in our core system. Our caution here is based on
experience. Moving too quickly in this area in the past has created
problems with the reliability and
portability of \GAP on which our users depend.

This release incorporated a completely
reimplemented build system for \GAP, using more modern tools and
complying with modern expectations. This resolved many issues with the
old system, and made it easier to maintain and extend. For example it
makes it much easier to maintain multiple configurations of \GAP on a
single system from a single copy of the source. Technical
details of the new build system are described in {\tt
  README.buildsys.md} from the \GAP source code. Crucially the new build system
was the final prerequisite for the planned  merge of the \HPCGAP codebase  into the
mainstream \GAP system as a compile-time option (for details, see
Subsection~\ref{hpc-gap}).

An additional benefit of the new build system is that it makes it much
easier to robustly include C or C++ extensions to the \GAP kernel in
contributed packages, including interfaces to many free C and C++
libraries. By making it easy to accelerate bottleneck steps, this
greatly enhanced the overall performance of many packages, while
encouraging integration of existing C libraries in packages complements
our cautious approach mentioned above to including them in the core
system.

One such example is the \software{Semigroups} package \cite{Semigroups} by James D.
Mitchell (USTAN).  This started as a pure \GAP package not requiring compilation,
but a number of key algorithms from it were later converted into C and C++ in
order to achieve the high performance necessary for its use in mathematical
research.  Rather than keeping these algorithms in a simple \GAP kernel
extension, they were made into a separate C++ library called \software{libsemigroups}
\cite{libsemigroups}, which does not rely on \GAP or any of its packages.  This
library is a required component of the \software{Semigroups} package, and provides
high-performance code to compute such objects as presentations, Cayley graphs
and rewriting systems for finite semigroups, as well as information about
semigroup congruences.  Many parts of the library support multi-threading,
meaning that users with multi-core computers can take advantage of parallelism
even when not running \HPCGAP.  Furthermore, the library can be accessed from
Python or \Sage via \href{https://github.com/libsemigroups/libsemigroups-python-bindings}{a special set of Python bindings}
 without
any need for \GAP to be loaded.  The development of these bindings, other work
on \software{libsemigroups}, and the mathematical research that came from it, are
good examples of \ODK collaboration between USTAN and UPSud.
An example of \software{libsemigroups} being called and used through \GAP is shown in
Appendix~\ref{sec:libsemigroups-notebook}.

Another example is the \software{curlInterface} package \cite{curlInterface}
by Christopher Jefferson and Michael Torpey, which first appeared
in \GAP 4.9.3 distribution.  This package allows a user to interact 
with http and https servers on the Internet, using the \software{curl} library, 
and is used by the \software{PackageManager} package
\cite{PackageManager} described in Subsection~\ref{pkg-manager}.

The \GAP~4.9.1 release also incorporated the new \software{ZeroMQInterface}
package
\cite{ZeroMQInterface} by Markus Pfeiffer and Reimer Behrends, which
provides both low-level bindings as well as some higher level interfaces
for the \software{ZeroMQ} message passing library for \GAP and \HPCGAP, 
enabling lightweight distributed computation. 


The release of this package at this time allowed us to include in \GAP~4.9.2 a
further three packages.  Firstly we have the \software{JupyterKernel} package \cite{JupyterKernel} 
by Markus Pfeiffer, developed under WP4 and reported in deliverable D4.7,
providing a so-called kernel for the
\href{https://jupyter.org/}{Jupyter interactive document system}. This package
requires GAP packages \software{IO}, \software{ZeroMQInterface}, \software{json},
and also two other new packages by Markus Pfeiffer called
\software{crypting} and \software{uuid} \cite{crypting,uuid}, all included 
in the \GAP~4.9.2 distribution.

\GAP~4.9.3 also included the \software{datastructures} package \cite{datastructures}
by Markus Pfeiffer, Max Horn, Christopher Jefferson and Steve Linton, which aims 
at providing standard datastructures, consolidating existing code and improving 
on it, in particular in view of \HPCGAP.

The \GAP~4.9.1 release also incorporated many smaller performance improvements. For example,
\GAP now supports named constants, whose value cannot change 
at runtime. While maintaining readability through naming, uses of
these constants can be optimized as a function is parsed, so that they
are at least as fast as using explicit values. Another notable improvement was
to the widely used  sorting functions which now use the best
recently published algorithms and were also made easier to maintain by
restructuring the implementation.

Debugging  and profiling tools were also improved based on experience with
\GAP 4.8. Further changes improved \GAP
usability by allowing more efficient and/or readable code, in line
with other modern programming languages:
\begin{itemize}
  \item a concise syntax for short
    functions with any number of arguments, so that, for example,
    \verb|{a,b} -> a+b| abbreviates \verb|function(a,b) return a+b; end|. 
\item more flexibility in \GAP expression syntax,
  where previously there were unintuitive technical restrictions. For
  example, an entry of the sum of two vectors can now be accessed as
  \verb|x := (v+w)[i]| instead of \verb|u := v+w; x := u[i]|.
  
%%   use of function literals in expressions such \verb|y := f().x;|
%% replacing previously required two-step calculation
%% \verb|y := f();; y := y.x;|.
%% This -- and the release
%%   announcement -- are actually wrong! This syntax always worked, the
%%   example in the PR is \verb|(f()).x| which is different}

\end{itemize}

\GAP 4.9 had three public releases between Months 33 and 37, and
was replaced by \GAP~4.10 in Month 39.

\subsection{Developments in \GAP 4.10 (Month 39)}\label{gap-4.10}

The release of \GAP 4.9 with the new build system allowed the
restructuring of the core \GAP system to continue, and allowed us to
finalize a number of other ongoing changes, leading to \GAP 4.10
after a shorter-than-usual interval. \GAP 4.10 is the current
version of \GAP, with the latest release being \GAP 4.10.2, published
in Month 46, and \GAP 4.11 expected in November 2019.

The most important new feature in \GAP 4.10 was the first public
release of \software{libGAP}: an
option to compile \GAP as a C library instead of a stand-alone
executable. This is very important for many uses of \GAP in \ODK and
was discussed in further detail in Section \ref{libgap} below.

\GAP 4.10 also provides experimental support for using alternative
memory management systems, notably the \href{https://julialang.org/}{Julia} garbage
collector and the \href{http://www.hboehm.info/gc/}{Boehm garbage collector}, in place of the
\software{GASMAN} system which was developed for \GAP 4 in the 1990s.  These
garbage collectors are thread-safe (important for \HPCGAP) and the
option to use the Julia garbage collector will allow tight integration
of \GAP with Julia programs and other systems via Julia. Use of Julia
in this way is an initiative led by the
\href{https://oscar.computeralgebra.de/}{OSCAR} project, developed
within the DFG-funded grant SFB-TRR 195 ``Symbolic Tools in
Mathematics and their Application'' and is seen as complementary to \ODK.

The work on improving debugging and profiling in \GAP has
continued. Profiling now reports memory usage as well as CPU time, and
additional checking options (both internal and by using external tools such as
\href{http://www.valgrind.org/}{\software{valgrind}}) help to detect misuse of memory in the kernel.

Using the improved debugging and profiling facilities we were able to
detect many performance bottlenecks and confirm that new code
genuinely improved performance, as well as detecting many bugs much
earlier in the development process. To alleviate performance
bottlenecks in key applications, we rewrote much of the method
selection (dispatch) code from \GAP to C and gained large speedups by
re-engineering many integer arithmetic functions.

For example, the table below gives the CPU time in milliseconds required
to run the following calculation:

\begin{verbatim}
for i in [1..k] do x:= 1/p^q mod n; od;
\end{verbatim}
\medbreak
\begin{center}
\begin{tabular}{| c | c | c | c | c | c |} 
\hline
p & q    & n          & k        & ms, before & ms, after \\
\hline
3 & 60   & $2^{59}$   & 2000000  & 2364       & 1219 \\
3 & 60   & $2^{60}$   & 2000000  & 2821       & 1292 \\
2 & 6000 & $3^{6000}$ & 6000     & 19358      & 692  \\
\hline
\end{tabular}
\end{center}

%Before
%gap> q:=1/3^60;;n:=2^59;; for i in [1..2000000] do x:= q mod n; od; time;
%2364
%gap> q:=1/3^60;;n:=2^60;; for i in [1..2000000] do x:= q mod n; od; time;
%2821
%gap> q:=1/2^6000;;n:=3^6000;; for i in [1..6000] do x:= (q mod n); od; time;
%19358
%
%After:
%
%gap> q:=1/3^60;;n:=2^59;; for i in [1..2000000] do x:= q mod n; od; time;
%1219
%gap> q:=1/3^60;;n:=2^60;; for i in [1..2000000] do x:= q mod n; od; time;
%1292
%gap> q:=1/2^6000;;n:=3^6000;; for i in [1..6000] do x:= (q mod n); od; time;
%692

%DDrop this next one, a bit technical
%
%Also, \verb|InverseMatMod| with integer modulus has been improved.% #2426
%The following table shows the runtime needed to call
%\verb|x:=InverseMatMod(m,q);| a hundred times for a random unimodular $100 \times 100$
%matrix \verb|m|:
%
%\begin{center}
%\begin{tabular}{| c | c | c |} 
%\hline
%q     & ms, before    & ms, after \\
%\hline
%2     & 277           & 273 \\
%251   & 1564          & 1249 \\
%65537 & out of memory & 1226 \\
%\hline
%\end{tabular}
%\end{center}

%Before:
%
%gap> m2:=RandomUnimodularMat(100);;
%gap> for i in [1..100] do x:=InverseMatMod(m2,2); od; time;
%277
%gap> for i in [1..100] do x:=InverseMatMod(m2,251); od; time;
%1564
%gap> for i in [1..100] do x:=InverseMatMod(m2,65537); od; time;
%Error, reached the pre-set memory limit
%
%After:
%
%gap> m2:=RandomUnimodularMat(100);;
%gap> for i in [1..100] do x:=InverseMatMod(m2,2); od; time;
%273
%gap> for i in [1..100] do x:=InverseMatMod(m2,251); od; time;
%1249
%gap> for i in [1..100] do x:=InverseMatMod(m2,65537); od; time;
%1226

The same improved profiling tools identified other
``hidden'' bottlenecks in \GAP code. For instance \GAP has a
sophisticated system of inference that can trigger 
so-called ``immediate methods'': quick attempts to deduce additional
consequnces from information that becomes known about an object.
In general, these are very effective, and the deduced information can
lead the system to chose better methods for subsequent
calculations. For instance, when a group learns its order, if that
order is finite and odd, then by the celebrated
Feit--Thomson theorem the group must be solvable. Once that
is known, much more efficient methods can be applied for many
subsequent calculations.

While this is a powerful feature in most situations, in some
calculations, a large number of short-lived objects are created,
and the immediate methods use a great deal of time deducing knowledge
about each of them
that will never be used. The profiling developments made this 
easily visible for the first time, and some small adjustments to the
immediate methods produced up to a six-fold speed-up in fundamental
operations such as computing isomorphism groups. For instance, the
run-time for the code fragment below dropped from 130
seconds to 22.


{\Small
\begin{verbatim}
G:=PcGroupCode( 
  741231213963541373679312045151639276850536621925972119311,
  11664);;
IsomorphismGroups(G,PcGroupCode(CodePcGroup(G),Size(G)))<>fail;
\end{verbatim}
}


We have also continued to develop package \software{JupyterKernel} following 
its inclusion in \GAP~4.9 to enhance its usability. The advent of
\href{https://mybinder.org/}{Binder} allowed us to give users an
opportunity to try \GAP Jupyter notebooks online 
and share reproducible \GAP experiments (see Appendix~\ref{sec:repro-gap}).

The inclusion of the \software{JupyterKernel} package and its associated 
dependencies into \GAP~4.9 facilitated the appearance of new packages
extending its graphical capabilities. \GAP 4.10.0 incorporated 
the \software{Francy} package \cite{Francy} by Manuel Martins and
the \software{JupyterViz} package \cite{JupyterViz} by 
Nathan Carter\footnote{Nathan Carter visited USTAN on sabbatical from
Bentley University in 2018--2019, and was employed part time to work
mainly on WP3. In addition to \software{JupyterViz}, he also carried out a major redesign
of his \software{GroupExplorer} software -- a popular teaching resource, 
accompanied by the textbook \cite{Carter-book}, which had been developed
in the previous decade and was in an
urgent need of modernization. The new \software{GroupExplorer} version, available
at \cite{GroupExplorer}, also allows a user to load and run the \GAP code
to construct and explore groups from the \software{GroupExplorer} library
interactively, launching a \GAP session on a \SageMath cell server.
}
which allow users to create and work with charts and graphs. The latter
package also allows the user to produce graphics from the \GAP command line.


\subsection{Developments in \GAP 4.11 (anticipated in November 2019)}\label{gap-4.11}

% Since they are released on GitHub, I see no reason not to also
% highlight changes which are merged and definitely coming out in
% 4.11, if they are relevant SL

% Release notes are collected at https://github.com/gap-system/gap/wiki/GAP-4.11-release-notes

At the time of preparation of this report, the next major release,
\GAP~4.11,
is approaching completion. Almost all of the features to be
included in it (including all the ones highlighted here) are already
publicly  available on GitHub to developers or anyone else interested.

One important and relevant feature of this release will be the ``syntaxtree'' reflection
API for \GAP, allowing \GAP programs to access and modify the internals of \GAP
functions and opening up a range of possibilities for optimization, automatic
parallelization, and more effective compilation into C. This is
discussed in more detail in Section \ref{syntaxtree} above. 

Additional relevant features are continued improvements to debugging and
profiling facilities (for example,
%#2772 
adding support for profiling interpreted code
and 
%#3420
making profiling reports more informative by 
giving more library methods human-readable names;
%#2900 %#1633 
and enhancements to the testing infrastructure designed to make the
test suite more robust and easier to run, and tests more independent
and easier to write).

The \software{libGAP} API has been extended
to allow programs using the system in this way
more convenient access to \GAP objects such as
%#2998, #2999, #3007 
characters, floats, integers
%#3554
and matrices; 
%#3516
the performance of Julia GC integration also improved,
%#3335 
and the memory usage on Windows when running external programs was reduced.

Further performance enhancements included, among others,
%#3579 
speeding up writing to global variables;
%#3566 
optimizing operations involving identity permutations;
%#3231 
speeding up \verb|IsConjugate| for \verb|IsNaturalSymmetricGroup|,
%#2924 
improving performance of \verb|NormalizerViaRadical|,
%#3031 
and of \verb|ConjugacyClasses| for solvable groups. For example, the time
of the following calculation reduces from 110 seconds to 53 seconds:

{\Small
\begin{verbatim}
G:=WreathProduct(CyclicGroup(IsPermGroup,12),
       DihedralGroup(IsPermGroup,12));
ConjugacyGlasses(G);
\end{verbatim}
}
\noindent

% What about making pages for relevant releases in
% https://www.gap-system.org/Manuals/doc/changes/manual.pdf 
% (or print HTML version of the manual into PDF)
% an electronic appendix to the deliverable?
%
% Yes -- let's do that. Add them as appendices.
%
% Indeed, as the ODK README.md says:
%
%  The report shall be self-contained. Indeed, the
%  deliverable will be evaluated based upon its version submitted on
%  the EU portal without retrieving other resources. Links have no
%  legal value, since there is no guarantee that the referenced
%  material will remain unchanged. One may typically want to add
%  relevant material as appendix (e.g. snapshots of software
%  documentation, websites, or other relevant documents); see e.g.
%  [WP5/D5.1/report.tex](WP5/D5.1/report.tex) or
%  [WP2/D2.1/report.tex](WP2/D2.1/report.tex).


%% \comment{tests from 4.7.9 take 3237s in GAP 4.7.9;
%% 4.10.2: total 3039s (150s GC) and 328GB allocated;
%% GAP master: total 2913s (140s GC) and 293GB allocated}

As one example of the general effectiveness of our performance
improvements, the full test suite of \GAP 4.7.9 (current at the
start of the project) which takes 3237 seconds to run on \GAP 4.7.9 on
current hardware, runs unchanged in just over 2900 seconds with a
pre-release version of \GAP 4.11. Many of the most widely used and
important areas of \GAP functionality have seen bigger changes.  

