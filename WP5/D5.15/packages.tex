\section{Additional Developments in \GAP packages} \label{sec:packages}

Other sections of this report describe
a number of \GAP packages such as Meataxe64 and Profiling that were developed, published,
and/or added to the \GAP distribution over the duration of the project.  In this section
we briefly list some other notable new or improved packages that
contribute to our overall goals of making \GAP more efficient, usable,
composable and scalable, and enabling users to solve larger and more
complex problems.

\begin{itemize}

%% \item
%% {\sf datastructures} \cite{datastructures} by Markus Pfeiffer, Max Horn, 
%% Christopher Jefferson and Steve Linton, which provides high
%% performance scalable implementations of modern standard datastructures
%% suitable for use in multi-threaded computations. (added in \GAP~4.9.3).
%% \TODO{is this one already covered in the section on \GAP 4.9?}

\item
{\sf Digraphs} by Jan De Beule, Julius Jonu\v{s}as, James Mitchell,
Michael Torpey and Wilf Wilson \cite{Digraphs}, which provides very efficient C-based
functionality for fundamental computations with directed and
undirected graphs, including those with multiple edges. (added in
\GAP~4.8.2 and repeatedly extended).  These algorithms underpin many
mathematical computations, especially in semigroups and combinatorics.

\item
{\sf ferret} by Christopher Jefferson \cite{ferret}, which provides a C++
reimplementation of Jeffrey Leon's Partition Backtrack framework for
solving graph-isomorphism like problems in permutation groups
(published as a package submitted for the redistribution with
\GAP). This has the effect of accelerating many standard
group-theoretic computations.

\item
{\sf FinInG} by John Bamberg, Anton Betten, Philippe Cara, Jan De
Beule, Michel Lavrauw and Max Neunh\"offer \cite{fining} for computation in Finite
Incidence Geometry (added in \GAP~4.8.2).

\item
{\sf MajoranaAlgebras} by Markus Pfeiffer and Madeleine Whybrow \cite{MajoranaAlgebras}, which
constructs Majorana representations of finite groups (added in
\GAP~4.10.1). This is interesting primarily as an application of high
performance kernels. This whole area of mathematics was inspired by
the Griess algebra used to construct the celebrated ``Monster''
sporadic simple group, which has dimension 196884. Many of the
interesting problems in this area have similarly high dimension and
\HPCGAP and meataxe64 methods have both been used to good effect in
this area.
%
% Possible demo?
%

\item
{\sf matgrp}  by Alexander Hulpke \cite{matgrp}, which provides an interface
to the solvable radical functionality for matrix groups, building on
constructive recognition (added in GAP 4.8.2). This improves the
general capabilities of \GAP in this area, and reduces many key
problems to linear algebra and/or partition backtrack questions, which
can exploit the power of {\sf meataxe64} and/or {\sf ferret}.

\item
{\sf NormalizInterface} by Sebastian Gutsche, Max Horn and
Christof S\"oger \cite{NormalizInterface}, which provides a GAP interface to Normaliz, enabling
direct access to the complete functionality of Normaliz, such as efficient
computations in affine monoids, vector configurations, lattice
polytopes, and rational cones (added in GAP 4.8.2). 

\item
{\sf SCSCP} by Alexander Konovalov and Steve Linton \cite{SCSCP}
migrated to GitHub and had several releases
in Months 18--45, which included improved testing of 
master--worker communication due to the use of
CI tools and new functionality required in WP6 as well as
code refactoring and bugfixes. Its usability for 
coarse-grained distributed memory parallel computation
(see \url{https://github.com/alex-konovalov/scscp-demo}
for a short tutorial) makes it a suitable replacement for the {\sf ParGAP}
package by Gene Cooperman, which has not been redistributed 
with \GAP since \GAP~4.9.

\item
{\sf SemigroupViz} by Nathan Carter \cite{SemigroupViz} is an emerging
package which is based on the {\sf JupyterViz} package \cite{JupyterViz}
and provides tools to visualise egg-box diagrams and Cayley graphs in
GAP. It can be used in Jupyter notebooks or from the GAP command line.


%
% Possible demo?
%

\item
{\sf walrus} by Markus Pfeiffer \cite{walrus}, providing methods for proving 
hyperbolicity of finitely presented groups in polynomial time
(added in GAP 4.10.1). This again is an application area for High
Performance methods in \GAP, requiring some very unusual data
structures and search primitives.

\item
{\sf YangBaxter} by Leandro Vendramin and Alexander Konovalov \cite{YangBaxter}, 
which provides functionality to construct classical and 
skew braces, and also includes a database of classical 
and skew braces of small orders (added in GAP 4.10.1).

\end{itemize}
