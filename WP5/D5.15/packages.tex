\section{Additional Developments in \GAP packages} \label{sec:packages}

Other sections of this report describe
a number of \GAP packages such as Meataxe64 and Profiling that were developed, published,
and/or added to the \GAP distribution over the duration of the project.  In this Section
we briefly list some other notable new or improved packages that
contribute to our overall goals of making \GAP more efficient, usable,
composable and scalable, and enabling users to solve larger and more
complex problems.

\begin{itemize}

\item
{\sf datastructures} \cite{datastructures} by Markus Pfeiffer, Max Horn, 
Christopher Jefferson and Steve Linton, which provides high
performance scalable implementations of modern standard datastructures
suitable for use in multi-threaded computations. (added in GAP 4.9.3).

\item
{\sf Digraphs} by Jan De Beule, Julius Jonu\v{s}as, James Mitchell,
Michael Torpey and Wilf Wilson, which provides very efficient C-based
functionality for fundamental computations
with directed and undirected graphs, including those with multiple edges. (added in GAP 4.8.2 and
repeatedly extended).  These algorithms underpin many mathematical
computations, especially in semigroups and combinatorics.

\item
{\sf ferret} package by Christopher Jefferson, which provides a C++
reimplementation of Jeffery Leon's Partition Backtrack framework for
solving graph-isomorphism like problems in permutation groups
(published as a package submitted for the redistribution with
GAP). This has the effect of accelerating many standard
group-theoretic computations.

\item
{\sf FinInG} package by John Bamberg, Anton Betten, Philippe Cara, Jan De Beule, Michel Lavrauw and Max Neunh\"offer for computation in Finite Incidence Geometry (added in GAP 4.8.2).

\item
{\sf MajoranaAlgebras} by Markus Pfeiffer and Madeleine Whybrow, 
which constructs Majorana representations of finite groups (added in
GAP 4.10.1). This is interesting primarily as an application of high
performance kernels. Many interesting problems here have very high
dimension and \HPCGAP and Meataxe64 methods have both been used in
this area.

\item
{\sf matgrp} package by Alexander Hulpke, which provides an interface
to the solvable radical functionality for matrix groups, building on
constructive recognition (added in GAP 4.8.2). This improves the
general capabilities of \GAP in this area, and reduces many key
problems to linear algebra and/or partition backtrack questions, which
can exploit the power of meataxe64 and/or ferret.

\item
{\sf NormalizInterface} package by Sebastian Gutsche, Max Horn and
Christof S\"oger, which provides a GAP interface to Normaliz, enabling
direct access to the complete functionality of Normaliz, such as efficient
computations in affine monoids, vector configurations, lattice
polytopes, and rational cones (added in GAP 4.8.2). 

\item
{\sf walrus} by Markus Pfeiffer, providing methods for proving 
hyperbolicity of finitely presented groups in polynomial time
(added in GAP 4.10.1). This again is an application area for High
Performance methods in \GAP, requiring some very unusual data
structures and search primitives.

\item
{\sf YangBaxter} by Leandro Vendramin and Alexander Konovalov, 
which provides functionality to construct classical and 
skew braces, and also includes a database of classical 
and skew braces of small orders (added in GAP 4.10.1).

\end{itemize}
