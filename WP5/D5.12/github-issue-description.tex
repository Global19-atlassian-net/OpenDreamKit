\section*{\texorpdfstring{Deliverable description, as taken from Github
issue
\href{https://github.com/OpenDreamKit/OpenDreamKit/issues/110}{\#110} on
2018-08-31}{Deliverable description, as taken from Github issue \#110 on 2018-08-31}}

\begin{itemize}
\tightlist
\item
  \textbf{WP5:}
  \href{https://github.com/OpenDreamKit/OpenDreamKit/tree/master/WP5}{High
  Performance Mathematical Computing}
\item
  \textbf{Lead Institution:} Universite Grenoble Alpes
\item
  \textbf{Due:} 2018-08-31 (month 36)
\item
  \textbf{Nature:} Demonstrator
\item
  \textbf{Task:} T5.3
  (\href{https://github.com/OpenDreamKit/OpenDreamKit/issues/101}{\#101})
  LinBox
\item
  \textbf{Proposal:}
  \href{https://github.com/OpenDreamKit/OpenDreamKit/raw/master/Proposal/proposal-www.pdf}{p.
  52}
\end{itemize}

\section*{Context}

Computational linear algebra is a key tool delivering high computing
throughput to applications requiring large scale compuations. In
numerical computing, dealing with floating point arithmetic and
approximations, a long history of efforts has lead to the design of a
full stack of technology for numerical HPC: from the design of stable
and fast algorithms, to their implementation in standardized libraries
such as LAPACK and BLAS, and their parallelization on shared memory
servers or supercomputers with distributed memory. On the other hand,
computational mathematics relies on linear algebra with exact
arithmetic, i.e. multiprecision integers and rationals, finite fields,
etc. This leads to significant differences in the algorithmic and
implementations approaches. Over the last 20 years, a continuous stream
of research has improved the exact linear algebra algorithmic and
simultaneously, software projects, such as LinBox and fflas-ffpack were
created to deliver, a similar set of kernel linear algebra routines as
LAPACK but of exact arithmetic.

\section*{Goal of the deliverable}

This deliverable aims at taking a major step forward in the advancement
of this technology stack for exact linear algebra: the development of
new application frameworks, new algorithms, their careful implementation
as high performance kernels in a standardized library. As a demonstrator
for the usability of this building block for the development of virtual
research environment, a key outcome of this deliverable is a tight
integration of the libraries LinBox and fflas-ffpack into the software
SageMath.
