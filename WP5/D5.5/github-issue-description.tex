\section*{\texorpdfstring{Deliverable description, as taken from Github
issue
\href{https://github.com/OpenDreamKit/OpenDreamKit/issues/118}{\#118} on
2017-02-27}{Deliverable description, as taken from Github issue \#118 on 2017-02-27}}\label{deliverable-description-as-taken-from-github-issue-118-on-2017-02-27}

\begin{itemize}
\tightlist
\item
  \textbf{WP5:}
  \href{https://github.com/OpenDreamKit/OpenDreamKit/tree/master/WP5}{High
  Performance Mathematical Computing}
\item
  \textbf{Lead Institution:} University of Kaiserslautern
\item
  \textbf{Due:} 2017-02-28 (month 18)
\item
  \textbf{Nature:} Demonstrator
\item
  \textbf{Task:} T5.5
  (\href{https://github.com/OpenDreamKit/OpenDreamKit/issues/103}{\#103}):
  MPIR
\item
  \textbf{Proposal:}
  \href{https://github.com/OpenDreamKit/OpenDreamKit/raw/master/Proposal/proposal-www.pdf}{P.
  52}
\item
  \textbf{\href{https://github.com/OpenDreamKit/OpenDreamKit/raw/master/WP5/D5.5/report-final.pdf}{Final
  report}}
\end{itemize}

\section*{Context and Problem
statement}\label{context-and-problem-statement}

\href{www.mpir.org}{\texttt{MPIR}} is a highly optimised library for
bignum arithmetic forked from \href{https://gmplib.org/}{GMP}. It is a
fundamental building block for many open source mathematical
computational components (\href{http://sagemath.org}{SageMath},
\href{http://flintlib.org/}{FLINT}, \href{http://nemocas.org/}{Nemo},
\href{https://room.eiffel.com/node/407}{Eiffelroom},
\href{https://pypi.python.org/pypi/gmpy2}{GMPY},
\href{http://www.advanpix.com/}{Advanpix}, \href{http://php.net/}{PHP}
and \href{http://wezeku.github.io/Mpir.NET/}{MPIR.net}), and therefore
its fine optimization on a variety of processor architecture is
important for the High Performance aims of OpenDreamKit.

For this deliverable the task was to implement a superoptimizer which
tries valid permutations (i.e., that do not change program behaviour) of
instructions in assembly functions, times each permutation, and chooses
the fastest one. In addition, new MPIR functions for recent processor
architectures were to be written, making use of recently added features
like AVX2 instructions, and be optimized with the super-optimizer where
applicable.

It is usually the case that the difference between assembly optimised
code and C code compiled by an optimising compiler such as GCC is a
factor of 4-12 for bignum arithmetic. But each new processor
microarchitecture requires new assembly language code to be written. One
can use older assembly code, but each new microarchitecture can do
around 20\% better than the previous one if hand optimisation is done.
In addition to that, speedups due to superoptimisation can be anywhere
from 5\% to 100\%.

In MPIR, we are typically comparing superoptimised code that was written
for a previous, but related microarchitecture, and so if the job is done
properly, we expect about 20\% improvement. We see that, and more,
below.

\section*{Work completed}\label{work-completed}

For the first six months of the project, we wrote the \texttt{ajs}
superoptimizer (\url{https://github.com/akruppa/ajs}), based on the
open-source \texttt{AsmJit} library
(\url{https://github.com/asmjit/asmjit}), a complete Just In Time and
remote assembler for C++ language.

For the second six months, we solved several problems with the
\texttt{ajs} superoptimizer, especially erratic timings that had put the
concept in jeopardy, and, with contributions from Jens Nurmann, wrote
and/or optimized a set of core functions for MPIR and some auxiliary
functions used internally (see below).

\section*{\texorpdfstring{The \texttt{ajs}
superoptimizer}{The ajs superoptimizer}}\label{the-ajs-superoptimizer}

The biggest problem with the superoptimizer was the highly erratic
timings it measured for function executions. This made it practically
impossible to have it automatically choose (one of) the fastest
permutations for a given function.

The major problem was that the
\href{https://en.wikipedia.org/wiki/Time_Stamp_Counter}{RDTSC(P)}
instructions no longer count cpu core cycles, but cycles of a
fixed-frequency counter, i.e., elapsed natural time. Due to extensive
clock scaling features of recent cpus, the measured time depended much
more on power saving decisions made by the cpu than on the
(comparatively small) speedup by finding a good permutation. This is
especially true as functions may have to be superoptimized in several
pieces, e.g., separately for lead-in, core loop, and lead-out, to reduce
the search space so that decent permutations are found within acceptable
time.

The solution we used was the RDPMC instruction which provides
low-latency access to performance measurement counters, including the
``second fixed- function counter'' (FFC2) which does, in fact, count cpu
core clock cycles. The problem was enabling access to this counter from
user mode applications, which requires setting some bits in MSR/CR.
Attempts to do so via kernel modules we wrote turned out unreliable as
the kernel disabled the bits again (and my modules killed machines on
multiple occasions).

Eventually an excellent solution to this problem was found in the
jevents library of the pmu-tools
(\url{https://github.com/andikleen/pmu-tools/}) which provides an API to
the perf subsystem of the Linux kernel. This allows enabling RDPMC to
read FFC2 without the kernel spuriously disabling it again.

The resulting timings within one program run were much more stable than
before, usually resulting in the same cycle count for a given function.
The timings still vary by 1 occasionally (very rarely 2 or more); we
have tried to find the source of the remaining variance, but to no
avail.

Another major source of error, but invariant within one super-optimizer
run, was the alignment of the stack, which appears to be randomly chosen
at program start. The writes to the stack (PUSH/CALL) on a function call
could alias (mod 4096) with the measured function's input operands,
causing ``partial address alias'' stalls which inflated execution time
by as much as 10 cycles. This problem was solved by forcing a particular
stack alignment.

Other problems that occurred within \texttt{ajs} and which were solved:

\begin{itemize}
\tightlist
\item
  Jump instruction were always encoded in long form by \texttt{asmjit},
  changing instruction alignment compared to other assemblers. We now
  manually annotate those instructions that require long form; all other
  use short. This requires manual work to annotate and verify the
  resulting instruction encodings.
\item
  Allow new registers introduced with AVX2, and instructions with 4
  operands
\item
  Various fixes and extensions to asm parsing code
\end{itemize}

All in all, fixing the aforementioned problems in \texttt{ajs} consumed
well over 2 months of time on the project. The code to generate
permutations that honour data dependencies is quite powerful; however,
subtle interactions with the cpu hardware made it very time-consuming to
get nearly cycle-accurate timings as we required.

\section*{Optimized functions for
MPIR}\label{optimized-functions-for-mpir}

We now review the functions that have been optimized on various
processor microarchitectures (Intel
\href{https://en.wikipedia.org/wiki/Haswell_(microarchitecture)}{Haswell}
and
\href{https://en.wikipedia.org/wiki/Skylake_(microarchitecture)}{Skylake}
and AMD
\href{https://en.wikipedia.org/wiki/Bulldozer_(microarchitecture)}{Bulldozer}).

Whilst these aren't the most recent architectures from the major chip
manufacturers, they are coming into widespread use around now. Indeed it
is difficult to get access to more recent machines. Naturally access to
the particular architecture is required in order to optimise for it.

For Haswell and Skylake, the following set of core functions was
re-implemented or existing code optimized to take advantage of the
respective micro-architecture: \texttt{add\_n}, \texttt{sub\_n},
\texttt{addmul\_1}, \texttt{submul\_1}, \texttt{addlsh1\_n},
\texttt{sublsh1\_n}, \texttt{com\_n}, \texttt{copyi}, \texttt{copyd},
\texttt{rshift1}, \texttt{lshift1}, \texttt{rshift}, \texttt{lshift},
\texttt{mul\_1}, \texttt{mul\_basecase}.

The only AMD CPU to which we could gain access was a Bulldozer which is
a fairly old and poorly designed microarchitecture; in particular, new
instruction set extensions like AVX2 are so slow on Bulldozer (and
Piledriver) that they are best avoided. This left little room for
optimization, and we opted not to write new code for this outdated cpu,
but to cherry-pick existing code that performs well.

We are very grateful to Jens Nurmann who contributed significant amounts
of code and expertise on AVX2 programming, to Brian Gladman for porting
the new code to the Microsoft Visual C build system, and to William
Stein for granting us access to a Bulldozer machine.

\subsection{Haswell microarchitecture}\label{haswell-microarchitecture}

For Haswell, new AVX2 versions of \texttt{com\_n}, \texttt{copyd},
\texttt{copyi}, \texttt{lshift}, \texttt{lshift1}, \texttt{rshift},
\texttt{rshift1} were written anew and super-optimized.

The \texttt{addmul\_1}, \texttt{submul\_1}, \texttt{mul\_1},
\texttt{mul\_basecase}, and \texttt{sqr\_basecase} functions for Haswell
in the GMP library were copied as these are extremely well optimized
already - we did not think we could produce better in what little time
we had left. Attempts to super-optimize these functions did not find
better code.

Existing \texttt{add\_n}, \texttt{sub\_n}, \texttt{karaadd},
\texttt{karasub}, \texttt{hgcd2} functions were modified for Haswell and
super-optimized, while \texttt{sumdiff\_n} and \texttt{nsumdiff\_n} were
written anew.

To give a summary of the speedups obtained, we include here results
obtained with the \texttt{mpir\_bench} program
(\url{https://github.com/akruppa/mpir_bench_two}). Higher values are
better (function executions per unit time); the apparent slow-down for
size \textless{} 512 GCD is to be investigated.

\begin{longtable}[c]{@{}lll@{}}
\toprule
Program multiply (weight 1.00) & Old & New\tabularnewline
\midrule
\endhead
128 128 & 108222650 & 107111633\tabularnewline
512 512 & 22816149 & 26895874\tabularnewline
8192 8192 & 228124 & 289984\tabularnewline
131072 131072 & 3884 & 5015\tabularnewline
2097152 2097152 & 173 & 203\tabularnewline
128 128 & 108109328 & 107223557\tabularnewline
512 512 & 17689648 & 20384648\tabularnewline
8192 8192 & 155145 & 189057\tabularnewline
131072 131072 & 2771 & 3479\tabularnewline
2097152 2097152 & 118 & 133\tabularnewline
15000 10000 & 80120 & 91788\tabularnewline
20000 10000 & 61030 & 71776\tabularnewline
30000 10000 & 37966 & 42448\tabularnewline
16777216 512 & 501 & 658\tabularnewline
16777216 262144 & 24.6 & 28.7\tabularnewline
\bottomrule
\end{longtable}

\begin{longtable}[c]{@{}lll@{}}
\toprule
Program gcd (weight 0.50) & Old & New\tabularnewline
\midrule
\endhead
128 128 & 3729465 & 3646816\tabularnewline
512 512 & 767983 & 554155\tabularnewline
8192 8192 & 10974 & 15908\tabularnewline
131072 131072 & 175 & 223\tabularnewline
1048576 1048576 & 9.38 & 11.5\tabularnewline
\bottomrule
\end{longtable}

\begin{longtable}[c]{@{}lll@{}}
\toprule
Program gcdext (weight 0.50) & Old & New\tabularnewline
\midrule
\endhead
128 128 & 2628011 & 2036197\tabularnewline
512 512 & 595026 & 451973\tabularnewline
8192 8192 & 7900 & 11192\tabularnewline
131072 131072 & 129 & 171\tabularnewline
1048576 1048576 & 6.04 & 7.94\tabularnewline
\bottomrule
\end{longtable}

The new code can be found in the directory
\url{https://github.com/akruppa/mpir/tree/master/mpn/x86_64/haswell} .

\subsection{Skylake microarchitecture}\label{skylake-microarchitecture}

For Skylake, \texttt{add\_n}, \texttt{sub\_n}, \texttt{mul\_1},
\texttt{add\_err1\_n} and \texttt{sub\_err1\_n} were written anew and
super-optimized. The \texttt{addmul\_1}, \texttt{mul\_basecase} and
\texttt{sqr\_basecase} functions were taken from GMP. The other
functions for Haswell are used as fall-backs.

\begin{longtable}[c]{@{}lll@{}}
\toprule
Program multiply (weight 1.00) & Old & New\tabularnewline
\midrule
\endhead
128 128 & 123326551 & 123312872\tabularnewline
512 512 & 29477397 & 33899135\tabularnewline
8192 8192 & 298474 & 358841\tabularnewline
131072 131072 & 4924 & 6024\tabularnewline
2097152 2097152 & 213 & 246\tabularnewline
128 128 & 123340235 & 123340948\tabularnewline
512 512 & 22551903 & 25322713\tabularnewline
8192 8192 & 208058 & 238204\tabularnewline
131072 131072 & 3497 & 4316\tabularnewline
2097152 2097152 & 142 & 155\tabularnewline
15000 10000 & 104503 & 112647\tabularnewline
20000 10000 & 80121 & 89101\tabularnewline
30000 10000 & 47871 & 54247\tabularnewline
16777216 512 & 611 & 693\tabularnewline
16777216 262144 & 29.1 & 33.6\tabularnewline
\bottomrule
\end{longtable}

\begin{longtable}[c]{@{}lll@{}}
\toprule
Program gcd (weight 0.50) & Old & New\tabularnewline
\midrule
\endhead
128 128 & 4387356 & 4373122\tabularnewline
512 512 & 814864 & 682194\tabularnewline
8192 8192 & 11468 & 18970\tabularnewline
131072 131072 & 208 & 274\tabularnewline
1048576 1048576 & 11.3 & 14.1\tabularnewline
\bottomrule
\end{longtable}

\begin{longtable}[c]{@{}lll@{}}
\toprule
Program gcdext (weight 0.50) & Old & New\tabularnewline
\midrule
\endhead
128 128 & 2750101 & 2562046\tabularnewline
512 512 & 640358 & 557060\tabularnewline
8192 8192 & 8526 & 13743\tabularnewline
131072 131072 & 155 & 212\tabularnewline
1048576 1048576 & 7.50 & 9.83\tabularnewline
\bottomrule
\end{longtable}

The new code can be found in the directory
\url{https://github.com/akruppa/mpir/tree/master/mpn/x86_64/skylake} .

\subsection{Bulldozer
microarchitecture}\label{bulldozer-microarchitecture}

On Bulldozer, the speed gains obtained are much more humble than on
Haswell and Skylake, as relatively few functions were replaced by faster
ones. This microarchitecture is not a profitable target for code
optimization any more.

\begin{longtable}[c]{@{}lll@{}}
\toprule
Program multiply (weight 1.00) & Old & New\tabularnewline
\midrule
\endhead
128 128 & 55322152 & 55550756\tabularnewline
512 512 & 12248577 & 12586138\tabularnewline
8192 8192 & 139406 & 138848\tabularnewline
131072 131072 & 2406 & 2421\tabularnewline
2097152 2097152 & 101 & 105\tabularnewline
128 128 & 55781257 & 51370568\tabularnewline
512 512 & 7690668 & 8710261\tabularnewline
8192 8192 & 90386 & 83592\tabularnewline
131072 131072 & 1587 & 1584\tabularnewline
2097152 2097152 & 64.0 & 65.9\tabularnewline
15000 10000 & 44703 & 45193\tabularnewline
20000 10000 & 33852 & 35294\tabularnewline
30000 10000 & 20000 & 20199\tabularnewline
16777216 512 & 268 & 294\tabularnewline
16777216 262144 & 12.7 & 13.4\tabularnewline
\bottomrule
\end{longtable}

\begin{longtable}[c]{@{}lll@{}}
\toprule
Program gcd (weight 0.50) & Old & New\tabularnewline
\midrule
\endhead
128 128 & 2597029 & 2611829\tabularnewline
512 512 & 284031 & 289573\tabularnewline
8192 8192 & 6800 & 6810\tabularnewline
131072 131072 & 108 & 107\tabularnewline
1048576 1048576 & 5.77 & 5.77\tabularnewline
\bottomrule
\end{longtable}

\begin{longtable}[c]{@{}lll@{}}
\toprule
Program gcdext (weight 0.50) & Old & New\tabularnewline
\midrule
\endhead
128 128 & 1270472 & 1239850\tabularnewline
512 512 & 223972 & 218197\tabularnewline
8192 8192 & 4944 & 4924\tabularnewline
131072 131072 & 78.1 & 78.0\tabularnewline
1048576 1048576 & 3.65 & 3.65\tabularnewline
\bottomrule
\end{longtable}

The new code can be found in the directory
\url{https://github.com/akruppa/mpir/tree/master/mpn/x86_64/bulldozer} .

\section*{Additional work}\label{additional-work}

Since the end of the project, we have added preliminary Broadwell CPU
support. This does not include any superoptimisation at this point.
Broadwell is essentially a revision of Haswell, but with some Skylake
features. We have added processor detection to MPIR and sped up this CPU
by making use of the Haswell code written for this project. Work is
underway to make some of the new Skylake code available to Broadwell
chips, and to write new assembly code for Broadwell. Many thanks to our
volunteers, Jens Nurmann and David Cleaver who have agreed to work on
this.

\section*{Future work}\label{future-work}

The superoptimizer works reasonably reliably now and can be used to
optimize more functions in MPIR and other software projects. At this
stage MPIR is the only project that has made use of the superoptimiser,
however we have already received a support request, so we expect there
to be more use cases soon.

The division and GCD functions in MPIR are worthwhile targets for
additional optimization work.

The new Zen microarchitecture of AMD was released towards the end of our
project, and looks promising for scientific computation. An optimization
effort here would be worthwhile; it will require to get access to such a
machine.

\section*{Source code}\label{source-code}

The \texttt{ajs} superoptimizer can be found at
\url{https://github.com/akruppa/ajs} . The optimized functions for
\texttt{MPIR} are merged into the main \texttt{MPIR} repository at
\url{https://github.com/wbhart/mpir} .

\section*{Testing this code}\label{testing-this-code}

Build instructions for MPIR are as follows:

Download MPIR-3.0.0 from \url{http://mpir.org/}

Note that you also need to have the latest yasm to build MPIR:
\url{http://yasm.tortall.net/}

To build yasm, download the tarball:

\begin{verbatim}
./configure
make
\end{verbatim}

To test MPIR, download the tarball:

\begin{verbatim}
./configure --enable-gmpcompat --with-yasm=/path_to_yasm/yasm
make
make check
\end{verbatim}

A Haswell, Skylake, or Bulldozer CPU is required to test the changes
referred to above.

\subsection{Blog post}\label{blog-post}

I have blogged about this project at
\url{https://wbhart.blogspot.de/2017/02/assembly-superoptimisation-in-mpir.html}
.
