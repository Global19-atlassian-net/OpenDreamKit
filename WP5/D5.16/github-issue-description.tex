\hypertarget{deliverable-description-as-taken-from-github-issue-114-on-2019-08-24}{%
\section*{\texorpdfstring{Deliverable description, as taken from Github
issue
\href{https://github.com/OpenDreamKit/OpenDreamKit/issues/114}{\#114} on
2019-08-24}{Deliverable description, as taken from Github issue \#114 on 2019-08-24}}\label{deliverable-description-as-taken-from-github-issue-114-on-2019-08-24}}

\begin{itemize}
\tightlist
\item
  \textbf{WP5:}
  \href{https://github.com/OpenDreamKit/OpenDreamKit/tree/master/WP5}{High
  Performance Mathematical Computing}
\item
  \textbf{Lead Institution:} CNRS
\item
  \textbf{Due:} 2019-08-31 (month 48)
\item
  \textbf{Nature:} Demonstrator
\item
  \textbf{Task:} T5.1
  (\href{https://github.com/OpenDreamKit/OpenDreamKit/issues/99}{\#99})
\item
  \textbf{Proposal:}
  \href{https://github.com/OpenDreamKit/OpenDreamKit/raw/master/Proposal/proposal-www.pdf}{p.
  51}
\item
  \textbf{\href{https://github.com/OpenDreamKit/OpenDreamKit/raw/master/WP5/D5.16/report-final.pdf}{Report}}
  (\href{https://github.com/OpenDreamKit/OpenDreamKit/raw/master/WP5/D5.16/}{sources})
\end{itemize}

We report on the work carried out during the OpenDreamKit project on the
PARI number theory software to improve the support for parallel
compuptation on a variety of hardware. This entailed devising a generic
parallelisation engine for PARI, and using it to prototype selected
functions (integer factorisation, discrete logarithm, modular
polynomials). And then releasing a PARI suite (libpari, GP, and GP2C)
that fully support parallelisation allowing individual implementations
to scale gracefully between single core / multicore / massively parallel
machines.
