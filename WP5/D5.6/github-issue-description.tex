\section*{\texorpdfstring{Deliverable description, as taken from Github
issue
\href{https://github.com/OpenDreamKit/OpenDreamKit/issues/119}{\#119} on
2017-02-08}{Deliverable description, as taken from Github issue \#119 on 2017-02-08}}\label{deliverable-description-as-taken-from-github-issue-119-on-2017-02-08}

\begin{itemize}
\tightlist
\item
  \textbf{WP5:}
  \href{https://github.com/OpenDreamKit/OpenDreamKit/tree/master/WP5}{High
  Performance Mathematical Computing}
\item
  \textbf{Lead Institution:} University of Kaiserslautern
\item
  \textbf{Due:} 2017-02-28 (month 18)
\item
  \textbf{Nature:} Demonstrator
\end{itemize}

The Quadratic Sieve is an algorithm for factoring integers n = pq, with
n typically in the 15-90 decimal digit range. For this deliverable, we
have implemented a quadratic sieve with the following features:

\begin{itemize}
\tightlist
\item
  Small prime variant
\item
  Large prime variant with in-memory hash table and on-disk
  pseudorelation storage
\item
  Multiple-polynomial self initialisation with Carrier-Wagstaff method
\item
  Polynomial generation using Bradford-Monagan-Percival method
\item
  Cache efficient sieving
\item
  Block-Lanczos linear algebra (adapted from msieve)
\item
  Knuth-Schroeppel multiplier
\item
  Threaded relation sieving (using OpenMP)
\end{itemize}

Some partial progress towards this goal had already been completed
before the OpenDreamKit project began, but the code didn't compile or
run and was in an unusable state.

We have completed the
\href{https://github.com/wbhart/flint2/tree/trunk/qsieve}{implementation}
and the code has already been merged into Flint.

Our implementation is not competitive below 130 bits
(\textasciitilde{}40 digits), due to the
Carrier-Wagstaff/Bradford-Monagan-Percival combination. Here are some
timings for larger factorisations comparing Pari/GP with our
implementation in Flint on one and four cores. There were no significant
improvements for larger numbers of cores.

\begin{longtable}[c]{@{}llll@{}}
\toprule
n (bits) & Pari & Flint & 4 core\tabularnewline
\midrule
\endhead
130 & 0.11 & 0.24 & 0.28\tabularnewline
140 & 0.22 & 0.33 & 0.30\tabularnewline
150 & 0.43 & 0.55 & 0.39\tabularnewline
160 & 0.79 & 0.99 & 0.56\tabularnewline
170 & 1.8 & 1.4 & 0.83\tabularnewline
180 & 3.8 & 3.0 & 1.7\tabularnewline
190 & 8.6 & 4.8 & 2.4\tabularnewline
200 & 17.6 & 10.2 & 6.0\tabularnewline
210 & 38.4 & 20.9 & 13.5\tabularnewline
220 & 67.7 & 33.2 & 15.8\tabularnewline
230 & 199 & 78 & 34\tabularnewline
240 & 257 & 140 & 52\tabularnewline
250 & 444 & 248 & 92\tabularnewline
260 & 1175 & 708 & 283\tabularnewline
270 & 2199 & 1522 & 544\tabularnewline
\bottomrule
\end{longtable}

Possible future projects could include:

\begin{itemize}
\tightlist
\item
  Cache efficient handling of factor base and self-initialisation data
  structures
\item
  PPMPQS method
\item
  Parallelising the file handling
\item
  Writing an efficient self-initialising MPQS for 20-128 bits.
\end{itemize}

The Block Wiedemann algorithm computes kernel vectors of a matrix over a
finite field. Along with the Block Lanczos algorithm it is often used in
the Quadratic Sieve and Number Field sieve over GF2, but also has other
applications over a prime p.

We investigated parallelising the Block Wiedemann algorithm using the
external library M4RI for basic linear algebra over GF2. This required
three components to be written.

\begin{itemize}
\tightlist
\item
  A sparse matrix library over GFp for Flint
\item
  An implementation of the Block Wiedemann algorithm over GF2
\item
  A naive quadratic sieve to generate realistic data
\end{itemize}

These were implemented by Anders Jensen and Alex Best and are available
as
\href{https://github.com/alexjbest/flint2/tree/nmod_sparse_mat/nmod_sparse_mat}{modules
for Flint} and as an \href{https://github.com/alexjbest/bw}{external
implementation} respectively.

Our experiments showed that it was possible to make use of the highly
optimised SIMD arithmetic in the external library M4RI as a form of
parallelism, to speed up the Block Wiedermann algorithm, and that this
outperformed Block Wiedermann using M4RI in threaded mode, at least for
the size and sparsity of problem encountered in the quadratic sieve! The
result was comparable to the Block Lanczos code already used in Flint
for the quadratic sieve.

The sparse matrix library was not the focus of this deliverable, but
will be expanded at a later date and included in Flint. Because of the
additional external dependency on M4RI, the Block Wiedemann
implementation isn't eligible for inclusion in Flint, but the code is
available to members of the OpenDreamKit collaboration. Future work may
see the removal of this external dependency so that it can be merged
into Flint.
