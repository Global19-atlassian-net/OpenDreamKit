\section*{\texorpdfstring{Deliverable description, as taken from Github
issue
\href{https://github.com/OpenDreamKit/OpenDreamKit/issues/119}{\#119} on
2017-02-27}{Deliverable description, as taken from Github issue \#119 on 2017-02-27}}\label{deliverable-description-as-taken-from-github-issue-119-on-2017-02-27}

\begin{itemize}
\tightlist
\item
  \textbf{WP5:}
  \href{https://github.com/OpenDreamKit/OpenDreamKit/tree/master/WP5}{High
  Performance Mathematical Computing}
\item
  \textbf{Lead Institution:} University of Kaiserslautern
\item
  \textbf{Due:} 2017-02-28 (month 18)
\item
  \textbf{Nature:} Demonstrator
\item
  \textbf{Task:} T5.4
  (\href{https://github.com/OpenDreamKit/OpenDreamKit/issues/102}{\#102})
  Singular
\item
  \textbf{Proposal:}
  \href{https://github.com/OpenDreamKit/OpenDreamKit/raw/master/Proposal/proposal-www.pdf}{P.
  52}
\item
  \textbf{\href{https://github.com/OpenDreamKit/OpenDreamKit/raw/master/WP5/D5.6/report-final.pdf}{Final
  report}}
\end{itemize}

\section*{Context}\label{context}

One of the approaches toward OpenDreamKit's aim of High Performance
Mathematical Computing is to explore the addition of very fine grained
parallelism (e.g.~threading or SIMD) to some key computational
components like \href{http://flintlib.org}{Flint} and
\href{http://mpir.org/}{MPIR}. In this deliverable, we tackle two
typical algorithms of computer algebra: integer factorization and block
Wiedemann for finding kernel vectors of matrices over a finite field.

Whilst the threading of the quadratic sieve looks very promising, our
conclusion is that the SIMD speedup that we implemented for the block
Wiedemann algorithm (and the threaded experiments we did) weren't
particularly useful for the quadratic sieve. We expect that only a
number field sieve implementation, with matrices that are truly massive
(millions of rows) would benefit from the kind of speedup we saw.

This doesn't mean that SIMD is not useful in general, it very much is,
see for example
\href{https://github.com/OpenDreamKit/OpenDreamKit/issues/118}{D5.5} and
\href{https://github.com/OpenDreamKit/OpenDreamKit/issues/120}{D5.7}
where SIMD proves to be a big win. SIMD just hasn't proved to be a major
benefit for a project like the quadratic sieve, where only a small part
of the runtime depends on the linear algebra phase of the algorithm.

The other improvements for the quadratic sieve that we describe below,
such as threading the relation sieving where we get a speedup of up to 3
on 4 cores, and the various algorithmic improvements that we describe
below, turned out to be much more important in practice.

\section*{Report on writing a parallel implementation of the quadratic
sieve}\label{report-on-writing-a-parallel-implementation-of-the-quadratic-sieve}

\subsection{Problem description}\label{problem-description}

The Quadratic Sieve is an algorithm for factoring integers n = pq, with
n typically in the 15-90 decimal digit range. For this deliverable, we
have implemented a quadratic sieve with the following features:

\begin{itemize}
\tightlist
\item
  Small prime variant
\item
  Large prime variant with in-memory hash table and on-disk
  pseudorelation storage
\item
  Multiple-polynomial self initialisation with Carrier-Wagstaff method
\item
  Polynomial generation using Bradford-Monagan-Percival method
\item
  Cache efficient sieving
\item
  Block-Lanczos linear algebra (adapted from msieve)
\item
  Knuth-Schroeppel multiplier
\item
  Threaded relation sieving (using OpenMP)
\end{itemize}

\subsubsection{Results}\label{results}

Some partial progress towards this goal had already been completed
before the OpenDreamKit project began, but the code didn't compile or
run and was in an unusable state. This included a partial implementation
of a single large prime variant quadratic sieve implemented by a Google
Summer of Code student, based on a prior version for small integers
only. This was not parallelised and a rough sketch only. It didn't
compile and wasn't correct or complete.

We have completed the
\href{https://github.com/wbhart/flint2/tree/trunk/qsieve}{implementation}
and the code has already been merged into the \texttt{Flint} repository.

Our implementation is not competitive below 130 bits
(\textasciitilde{}40 digits), due to the
Carrier-Wagstaff/Bradford-Monagan-Percival combination. Here are some
timings for larger factorisations comparing \texttt{Pari/GP} with our
implementation in \texttt{Flint} on one and four cores.

In our tests, no significant improvements were observed for larger
numbers of cores, so we report only on the improvements for four cores
here.

\begin{longtable}[c]{@{}llll@{}}
\toprule
n (bits) & Pari & Flint & 4 core\tabularnewline
\midrule
\endhead
130 & 0.11 & 0.24 & 0.28\tabularnewline
140 & 0.22 & 0.33 & 0.30\tabularnewline
150 & 0.43 & 0.55 & 0.39\tabularnewline
160 & 0.79 & 0.99 & 0.56\tabularnewline
170 & 1.8 & 1.4 & 0.83\tabularnewline
180 & 3.8 & 3.0 & 1.7\tabularnewline
190 & 8.6 & 4.8 & 2.4\tabularnewline
200 & 17.6 & 10.2 & 6.0\tabularnewline
210 & 38.4 & 20.9 & 13.5\tabularnewline
220 & 67.7 & 33.2 & 15.8\tabularnewline
230 & 199 & 78 & 34\tabularnewline
240 & 257 & 140 & 52\tabularnewline
250 & 444 & 248 & 92\tabularnewline
260 & 1175 & 708 & 283\tabularnewline
270 & 2199 & 1522 & 544\tabularnewline
\bottomrule
\end{longtable}

\subsection{Future possibilities}\label{future-possibilities}

Possible future projects could include:

\begin{itemize}
\tightlist
\item
  Cache efficient handling of factor base and self-initialisation data
  structures
\item
  PPMPQS method
\item
  Parallelising the file handling
\item
  Writing an efficient self-initialising MPQS for 20-128 bits.
\end{itemize}

\subsubsection{Testing the quadratic sieve
code}\label{testing-the-quadratic-sieve-code}

The \texttt{Flint} repository is available here:
\url{https://github.com/wbhart/flint2}

To build and test the code mentioned above, you must have
\texttt{GMP/MPIR} installed on your machine (refer to your system
documentation for how to do this). Then do:

\begin{verbatim}
git clone https://github.com/wbhart/flint2
cd flint
./configure --with-mpir=/path/to/mpir --with-mpfr=/path/to/mpfr --enable-openmp
export OMP_NUM_THREADS=4
make
make check MOD=qsieve
\end{verbatim}

Full instructions on how to build \texttt{Flint} are available in the
\texttt{Flint} documentation, available at the
\href{http://flintlib.org}{\texttt{Flint} website}.

The quadratic sieve functionality is made available by including
qsieve.h, and the main interface is via the function:

\begin{verbatim}
void qsieve_factor(fmpz_factor_t factors, const fmpz_t n)
\end{verbatim}

\subsection{Blog post}\label{blog-post}

A blog post about the improvements to the quadratic sieve in Flint is
available at
\url{https://wbhart.blogspot.de/2017/02/integer-factorisation-in-flint.html}
.

\section*{Report on parallel block
Wiedemann}\label{report-on-parallel-block-wiedemann}

\subsection{Problem description}\label{problem-description-1}

The Block Wiedemann algorithm computes kernel vectors of a matrix over a
finite field. Along with the Block Lanczos algorithm it is often used in
the Quadratic Sieve and Number Field sieve over GF2, but also has other
applications over a prime p.

\subsection{Work done}\label{work-done}

We investigated parallelising the Block Wiedemann algorithm using the
external library \href{https://bitbucket.org/malb/m4ri}{\texttt{M4RI}}
for basic linear algebra over GF2. This required three components to be
written.

\begin{itemize}
\tightlist
\item
  A sparse matrix library over GFp for \texttt{Flint}
\item
  An implementation of the Block Wiedemann algorithm over GF2
\item
  A naive quadratic sieve to generate realistic data
\end{itemize}

These were implemented by Anders Jensen and Alex Best and are available
as
\href{https://github.com/alexjbest/flint2/tree/nmod_sparse_mat/nmod_sparse_mat}{modules
for Flint} and as an \href{https://github.com/alexjbest/bw}{external
implementation} respectively.

\subsection{Conclusions}\label{conclusions}

Our experiments showed that it was possible to make use of the highly
optimised SIMD arithmetic in the external library \texttt{M4RI} as a
form of parallelism, to speed up the Block Wiedermann algorithm, and
that this outperformed Block Wiedermann using \texttt{M4RI} in threaded
mode, at least for the size and sparsity of problem encountered in the
quadratic sieve! The result was comparable to the Block Lanczos code
already used in Flint for the quadratic sieve.

The sparse matrix library was not the focus of this deliverable, but
will be expanded at a later date and included in Flint. Because of the
additional external dependency on \texttt{M4RI}, the Block Wiedemann
implementation isn't eligible for inclusion in Flint, but the code is
available to members of the OpenDreamKit collaboration. Future work may
see the removal of this external dependency so that it can be merged
into Flint. However, we currently feel that this isn't something that
should be made a priority.

In fact, the quadratic sieve implementation we now have, even in
parallel mode, doesn't have linear algebra as a bottleneck. This is
because of the extremely small factor base sizes that are possible with
the particular implementation of the quadratic sieve that we completed
as part of this project.
