\section*{\texorpdfstring{Deliverable description, as taken from GitHub
issue
\href{https://github.com/OpenDreamKit/OpenDreamKit/issues/116}{\#116} on
2016-09-07}{Deliverable description, as taken from GitHub issue \#116 on 2016-09-07}}\label{deliverable-description-as-taken-from-github-issues-116-on-2016-09-07}

\begin{itemize}
\tightlist
\item
  \textbf{WP5:}
  \href{https://github.com/OpenDreamKit/OpenDreamKit/tree/master/WP5}{High
  Performance Mathematical Computing}
\item
  \textbf{Lead Institution:} University of Sheffield
\item
  \textbf{Due:} 2016-08-31 (month 12)
\item
  \textbf{Nature:} Other
\item
  \textbf{Task:} T5.8
  \href{https://github.com/OpenDreamKit/OpenDreamKit/issues/106}{\#106}
\item
  \textbf{Proposal:}
  \href{https://github.com/OpenDreamKit/OpenDreamKit/raw/master/Proposal/proposal-www.pdf}{p.51}
\item
  \textbf{\href{https://github.com/OpenDreamKit/OpenDreamKit/raw/master/WP5/D5.3/report-final.pdf}{Final
  report}}
\end{itemize}

It is common for academic High Performance Computing (HPC) clusters to
make use of schedulers based on Sun Grid Engine with Son of Grid Engine
as one of the most popular. It is used, for example, on the
institutional HPC systems in the Universities of Sheffield and
Manchester in the UK. It is also used on the regional N8 HPC facility, a
system shared by the 8 most research intensive universities in the North
of England.\\
In this deliverable, we have developed and demonstrated a Sun Grid
Engine notebook spawner for Project Jupyter, allowing users to easily
access Jupyter notebooks on HPC clusters directly from the web-browser.
This development allows users with no background in High Performance
Computing to easily migrate workflows from laptop to HPC cluster,
allowing them to access greater resources with no additional training
required.
