\section*{\texorpdfstring{Deliverable description, as taken from Github
issue's
\href{https://github.com/OpenDreamKit/OpenDreamKit/issues/107}{\#107} on
2016-09-07}{Deliverable description, as taken from Github issue's \#107 on 2016-09-07}}\label{deliverable-description-as-taken-from-github-issues-107-on-2016-09-07}

\begin{itemize}
\tightlist
\item
  \textbf{WP5:}
  \href{https://github.com/OpenDreamKit/OpenDreamKit/tree/master/WP5}{High
  Performance Mathematical Computing}
\item
  \textbf{Lead Institution:} Université Paris-Sud
\item
  \textbf{Due:} 2015-11-30 (month 3)
\item
  \textbf{Submitted to Sage:} 2015-11-30
\item
  \textbf{Merged into Sage:} 2016-04-08
\item
  \textbf{Task}: T5.6
  (\href{https://github.com/OpenDreamKit/OpenDreamKit/issues/104}{\#104})
\item
  \textbf{Nature:} Demonstrator
\item
  \textbf{Proposal:}
  \href{https://github.com/OpenDreamKit/OpenDreamKit/raw/master/Proposal/proposal-www.pdf}{p.51}
\item
  \textbf{\href{https://github.com/OpenDreamKit/OpenDreamKit/raw/master/WP5/D5.1/report-final.pdf}{Final
  report}},
  \textbf{\href{https://github.com/OpenDreamKit/OpenDreamKit/raw/master/WP5/D5.1/slides-final.pdf}{slides}}
\end{itemize}

\href{https://en.wikipedia.org/wiki/MapReduce}{MapReduce} is a classical
programming model for distributed computations where one maps a function
on a large data set and use a reduce function to summarize all the
produced information. A use case that occurs often e.g.~in combinatorics
is to have a data sets that is described by a recursion tree, and is too
big to be expanded in memory. Instances include counting the number of
elements in the data set, or collecting some statistics on them.

A prototype distributed implementation of this programming model had
been written in 2010-2014 for SageMath, using multiple processes on a
single machine and work-stealing for load balancing. In this
deliverable, we have turned this prototype into production code and
integrated it into the SageMath distribution.

See \href{http://trac.sagemath.org/ticket/13580}{Trac Ticket 13580} for
the source code and the discussion about the integration into Sage, as
well as this
\href{https://github.com/OpenDreamKit/OpenDreamKit/blob/master/WP5/T5.6/documentation.pdf}{snapshot
of the documentation}.

This work was presented at the
\href{http://tesson.julien.free.fr/LaMHA/2015/automne.php}{journée du
groupe de travail LaMHA} at Université Pierre et Marie Curie on November
the 26th of 2016. The
\href{https://github.com/OpenDreamKit/OpenDreamKit/raw/master/WP5/T5.6/HPC-Combi.pdf}{slides}
give an overview of the motivations, algorithm, and implementation.
