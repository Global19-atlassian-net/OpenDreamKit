\section*{\texorpdfstring{Deliverable description, as taken from Github
issue
\href{https://github.com/OpenDreamKit/OpenDreamKit/issues/115}{\#115} on
2017-02-08}{Deliverable description, as taken from Github issue \#115 on 2017-02-08}}\label{deliverable-description-as-taken-from-github-issue-115-on-2017-02-08}

\begin{itemize}
\tightlist
\item
  \textbf{WP5:}
  \href{https://github.com/OpenDreamKit/OpenDreamKit/tree/master/WP5}{High
  Performance Mathematical Computing}
\item
  \textbf{Lead Institution:} Université Grenoble Alpes
\item
  \textbf{Due:} 2017-02-28 (month 18)
\item
  \textbf{Nature:} Demonstrator\\
  Task: 5.7
\end{itemize}

%% See page 51 of the
%% \href{https://github.com/OpenDreamKit/OpenDreamKit/raw/master/Proposal/proposal-www.pdf}{proposal}
%% for the full description.

The Python programming language, widely used in the development of
computational mathematics software for its flexibility, yet suffers from the
inefficiencies due its nature of interpreted language. Cython is both an extension of
this language, and a compiler producing compilable C code, delivering
performances of higher standard.
The use of the very widespread library NumPy is a bottlneck in this framework,
as it prevents the user from taking advantage of Cython's improvement.
On the other hand, Pythran is a dedicated Python to C++ compiler. It has a C++
implementation of a major set of the Numpy API, optimized for speed, with the
support of expression templates, that minimize the number of memory transfers
needed to compute complex expression. It also use as best as possible SIMD instructions available modern CPUs when possible.

This deliverable hence proposes to incorporate Pythran's support for Numpy code
wihtin Cython, which consequently will provide high performance numerical linear
algebra to high level mathematical software developpers, especially within the
SageMath software.

As an illustration, the following code, using SIMD instructions, lead to a
speed-up of about 4, when using the new Pythran backend in Cython:
\begin{lstlisting}[language=python]
def sqrt_sum (numpy.ndarray[FLOATTYPE_t, ndim=1] a,
              numpy.ndarray[FLOATTYPE_t, ndim=1] b):
    return numpy.sqrt(numpy.sqrt(a*a+b*b))
\end{lstlisting}
