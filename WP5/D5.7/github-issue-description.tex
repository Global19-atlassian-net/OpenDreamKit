\section*{\texorpdfstring{Deliverable description, as taken from Github
issue
\href{https://github.com/OpenDreamKit/OpenDreamKit/issues/120}{\#120} on
2017-02-28}{Deliverable description, as taken from Github issue \#120 on 2017-02-28}}\label{deliverable-description-as-taken-from-github-issue-120-on-2017-02-28}

\begin{itemize}
\tightlist
\item
  \textbf{WP5:}
  \href{https://github.com/OpenDreamKit/OpenDreamKit/tree/master/WP5}{High
  Performance Mathematical Computing}
\item
  \textbf{Lead Institution:} University of Kaiserslautern
\item
  \textbf{Due:} 2017-02-28 (month 18)
\item
  \textbf{Nature:} Demonstrator
\item
  \textbf{Task:} T5.4
  (\href{https://github.com/OpenDreamKit/OpenDreamKit/issues/102}{\#102}):
  Singular, T5.5
  (\href{https://github.com/OpenDreamKit/OpenDreamKit/issues/103}{\#103}):
  MPIR
\item
  \textbf{Proposal:}
  \href{https://github.com/OpenDreamKit/OpenDreamKit/raw/master/Proposal/proposal-www.pdf}{P.
  52}
\item
  \textbf{\href{https://github.com/OpenDreamKit/OpenDreamKit/raw/master/WP5/D5.7/report-final.pdf}{Final
  report}}
\end{itemize}

\section{Report on parallelising the
FFT}\label{report-on-parallelising-the-fft}

\subsection{Problem statement}\label{problem-statement}

Given two polynomials of length n, the time to multiply them using
classical schoolboy multiplication is O(n\^{}2). But there are numerous
algorithms which can do better. The Karatsuba method already takes time
O(n\^{}log\_2(3)). There are other methods, including Toom-Cook which
slightly improve the exponent.

The Fast Fourier technique allows multiplication of such polynomials in
O(n log(n)) operations. This is a technique that goes back as far as
Gauß, but has seen extensive development since then, with over 800
papers on the method and related techniques, with applications from
signal processing to string search or polynomial and integer arithmetic.

The version of the FFT that is used in \texttt{Flint} and \texttt{MPIR}
is the Schoenhage-Strassen method. Instead of doing a convolution over
the complex numbers, which would make use of imprecise floating point
numbers, which would be subject to rounding error, it makes use of an
exact ring, namely Z/pZ where p = 2\^{}(2\^{}n) + 1. This technique
allows exact multiplication of polynomials and integers with nearly
linear complexity.

In summary, the existing FFT in \texttt{Flint} is used for:

\begin{itemize}
\tightlist
\item
  Large integer multiplication
\item
  Schoenhage-Strassen univariate polynomial multiplication
\item
  Kronecker-Segmentation univariate polynomial multiplication
\end{itemize}

The purpose of this task was to parallelise the FFT in \texttt{Flint}.

Typically, parallelising the FFT algorithm is difficult. However,
\texttt{Flint} makes use of a cache-friendly implementation of the FFT
which uses the Matrix Fourier Algorithm. This breaks one very large FFT
convolution up into many smaller FFT's.

The existing FFT implementation in \texttt{Flint} (and \texttt{MPIR}) is
world class and includes:

\begin{itemize}
\tightlist
\item
  truncated fourier transform
\item
  use of low level GMP/MPIR assembly optimised functions
\item
  square root of 2 trick
\item
  Matrix Fourier Algorithm
\item
  Nussbaumer convolution
\item
  Chinese remainder with naive convolution
\end{itemize}

\subsection{The method}\label{the-method}

In order to thread the FFT in \texttt{Flint}, we used \texttt{OpenMP}.
The level at which we threaded it was at the level of the Matrix Fourier
Algorithm. This involved separating temporary storage that is used
throughout the algorithm, on a per thread level, and then adding
\texttt{OpenMP} primitives to the part of the Matrix Fourier Algorithm
that breaks the FFT into lots of smaller FFTs.

We also threaded the code which splits large integers into FFT
coefficients. Unfortunately it is difficult, or even impossible to fully
parallelise the recombination that happens after the FFT convolution has
run, so this wasn't attempted. However, it is a negligible portion of
the run time.

Fortunately, once the Matrix Fourier Algorithm becomes more efficient
than a single large FFT (due to its cache aware properties), the
threaded version also becomes more efficient than the single-threaded
version. In fact, the tuning crossover was found to be at exactly the
same point! This is an interesting coincidence and made tuning very
easy.

To maximise the benefit of threads, we combine parts of the small inward
FFTs, the relevant pointwise multiplications and parts of the outward
inverse FFTs into combined blocks that each run on a single thread
without interruption. The whole FFT convolution consists of many of
these smaller blocks. This was by design rather than accident!

The algorithm in \texttt{Flint} also combines the truncated Fourier
transform and Matrix Fourier algorithm in such a way that the entire
large FFT breaks down exactly into the smaller threaded blocks discussed
above, with no additional bits that have to be dealt with serially. This
is due to an innovation in the \texttt{Flint} FFT which isn't available
elsewhere. Again, this was a design feature, not an accident. The scope
of this method is exceptionally technical and well beyond the scope of
this report to describe.

In fact, we were able to preserve every single one of the technical
tricks mentioned above in our parallel implementation of the FFT in
\texttt{Flint}.

\subsection{Results}\label{results}

The new code for the threaded Matrix Fourier algorithm has been
implemented as part of this deliverable and merged into the main
\href{https://github.com/wbhart/flint2/tree/trunk/fft}{\texttt{Flint}
repository}.

Here are timings of the new code in \texttt{Flint} on a single core,
versus four and eight cores for various sized integer multiplications on
a 64 bit machine.

\begin{longtable}[c]{@{}llll@{}}
\toprule
limbs & 1 core & 4 core & 8 core\tabularnewline
\midrule
\endhead
114525 & 0.066s & 0.049s & 0.049s\tabularnewline
229725 & 0.14s & 0.11s & 0.11s\tabularnewline
360237 & 0.32s & 0.12s & 0.09s\tabularnewline
721709 & 0.65s & 0.25s & 0.19s\tabularnewline
1245101 & 1.14s & 0.39s & 0.27s\tabularnewline
2492333 & 2.33s & 0.81s & 0.55s\tabularnewline
4587132 & 4.45s & 1.52s & 1.02s\tabularnewline
9178748 & 9.07s & 3.02s & 2.06s\tabularnewline
25947772 & 28.1s & 9.35s & 6.25s\tabularnewline
51908220 & 57.9s & 24.0s & 13.8s\tabularnewline
118997068 & 143s & 48.4s & 33.2s\tabularnewline
238026828 & 309s & 105s & 65.7s\tabularnewline
506425420 & 801s & 241s & 146s\tabularnewline
\bottomrule
\end{longtable}

\subsection{Testing the parallel FFT}\label{testing-the-parallel-fft}

The Flint repository is available
\href{https://github.com/wbhart/flint2}{here}.

To build and test the code mentioned above, you must have
\texttt{GMP}/\texttt{MPIR} and \texttt{MPFR} installed on your machine
(refer to your system documentation for how to do this). Then do:

\begin{verbatim}
git clone https://github.com/wbhart/flint2
cd flint
./configure --with-mpir=/path/to/mpir --with-mpfr=/path/to/mpfr --enable-openmp
export OMP_NUM_THREADS=4
make
make check MOD=fft
\end{verbatim}

Full instructions on how to build \texttt{Flint} are available in the
\texttt{Flint} documentation, available at the
\href{http://flintlib.org/}{\texttt{Flint} website}.

The description of the FFT interface is well beyond the scope of this
documentation, but can be found in the
\href{http://flintlib.org/flint-2.5.pdf}{\texttt{Flint} documentation}
(625 pp.) There is also additional information specific to the FFT in
the Flint FFT
\href{https://github.com/wbhart/flint2/tree/trunk/fft}{README}

\section{Report on writing assembly primitives for the FFT
butterflies}\label{report-on-writing-assembly-primitives-for-the-fft-butterflies}

\subsection{Problem statement}\label{problem-statement-1}

For this deliverable, our task was to improve existing functions or
write new ones to use features of recent microprocessors (esp. AVX2) to
speed up the Schönhage-Strassen FFT butterflies. Such assembly
primitives are provided by the \texttt{MPIR} library.

The main operations used in the FFT butterflies are:

\begin{itemize}
\tightlist
\item
  Compute a+b, a-b for given a,b
\item
  Compute -(a+b), a-b for given a,b
\item
  Bit-wise shifts by varying bit-counts
\item
  Subtraction, and to a lesser extent addition and negation
\end{itemize}

Some of these operations already had assembly primitives available as
part of the \texttt{MPIR} library. However, these were not optimised for
recent architectures using AVX, for example. In this task, we also added
a new assembly primitive, as described below, which is used directly in
the FFT butterflies (where most of the FFT work is actually done).

Each year or two, Intel and AMD release new CPU microarchitectures. The
ones we focused on for this deliverable were Intel Haswell and Skylake
and AMD Bulldozer. These are not the most recent architectures, but they
are coming into widespread use at the present time.

\subsection{Results}\label{results-1}

The microarchitectures for which we optimized the code are mainly Intel
Haswell and Intel Skylake, and to a lesser extent AMD Bulldozer. For
Bulldozer (and Piledriver) it should be noted that the opportunities\\
for optimization are rather limited: the microarchitecture generally
performs poorly, especially in hyper-threading mode, and the AVX
instructions in particular are so slow as to be practically useless. The
newer AMD Steamroller fares better, but we did not have access to one.

For Haswell and Skylake, the \texttt{mpn\_lshift1},
\texttt{mpn\_rshift1}, \texttt{mpn\_lshift}, and \texttt{mpn\_rshift}
have been written anew, using AVX2 instructions which gave a large
speed-up over the previous code. The
\texttt{mpn\_add\_n}/\texttt{mpn\_sub\_n} functions (which are
identical, performance-wise) have been modified from existing code and
optimized according to the respective micro-architecture. An
\texttt{mpn\_sumdiff\_n} (computes a+b, a-b) has been introduced into
\texttt{MPIR}; this function existed for older processors but not for
recent x86\_64.

We are very grateful to Jens Nurmann who contributed significant amounts
of code and expertise on AVX2 programming.

\subsubsection{Haswell
microarchitecture}\label{haswell-microarchitecture}

Timings in cycles per limb:

\begin{longtable}[c]{@{}lll@{}}
\toprule
Function & Old & New\tabularnewline
\midrule
\endhead
mpn\_lshift1 & 1.11 & 0.564\tabularnewline
mpn\_rshift1 & 1.39 & 0.589\tabularnewline
mpn\_lshift & 1.85 & 0.568\tabularnewline
mpn\_rshift & 1.40 & 0.578\tabularnewline
mpn\_add\_n & 1.32 & 1.11\tabularnewline
mpn\_sumdiff\_n & 2.62(1) & 2.42\tabularnewline
mpn\_nsumdiff\_n & 3.23(2) & 2.64\tabularnewline
\bottomrule
\end{longtable}

(1) The sum of the times of \texttt{mpn\_add\_n}, \texttt{mpn\_sub\_n}.

(2) The sum of the times of \texttt{mpn\_add\_n}, \texttt{mpn\_sub\_n},
\texttt{mpn\_neg\_n}.

Timings for the full Schönhage-Strassen large integer multiplication
(\texttt{mpn\_mul\_n}) in seconds:

\begin{longtable}[c]{@{}llll@{}}
\toprule
Limbs & Old & New & Ratio\tabularnewline
\midrule
\endhead
10000 & 0.002399728 & 0.002171788 & 0.91\tabularnewline
100000 & 0.026374851 & 0.022960783 & 0.87\tabularnewline
1000000 & 0.357847841 & 0.302023203 & 0.84\tabularnewline
\bottomrule
\end{longtable}

Note that these timings include the effect of code improvements made for
D5.5
(\href{https://github.com/OpenDreamKit/OpenDreamKit/issues/118}{\#118}),
in particular, better \texttt{mpn\_mul\_basecase} and Karatsuba code.

\subsubsection{Skylake
microarchitecture}\label{skylake-microarchitecture}

Timings in cycles per limb:

\begin{longtable}[c]{@{}lll@{}}
\toprule
Function & Old & New\tabularnewline
\midrule
\endhead
mpn\_lshift1 & 1.01 & 0.601\tabularnewline
mpn\_rshift1 & 1.52 & 0.601\tabularnewline
mpn\_lshift & 2.01 & 0.608\tabularnewline
mpn\_rshift & 1.52 & 0.606\tabularnewline
mpn\_add\_n & 1.22 & 1.04\tabularnewline
mpn\_sumdiff\_n & 2.44(1) & 2.04\tabularnewline
mpn\_nsumdiff\_n & 3.06(2) & 2.32\tabularnewline
\bottomrule
\end{longtable}

Of note here is the speed of mpn\_add\_n/mpn\_sub\_n, at essentially
1c/l for the core loop, optimal both in terms of the data dependency
chain and memory accesses, as Skylake can in theory execute 2 read and 1
write per clock cycle. In practice, presumably the instruction scheduler
falls into a bad pattern after running at 1c/l for a while, and from
then on runs the loop only at \textasciitilde{}1.2c/l. Jens Nurmann
found that inserting a meaningless AVX2 instruction into the core loop
(which does not otherwise use AVX2)\\
breaks up this bad scheduling pattern, allowing these critically
important core functions to run at the optimal speed reliably.

Timings for mpn\_mul\_n in seconds:

\begin{longtable}[c]{@{}llll@{}}
\toprule
Limbs & Old & New & Ratio\tabularnewline
\midrule
\endhead
10000 & 0.002125143 & 0.001711500 & 0.81\tabularnewline
100000 & 0.025231292 & 0.020712453 & 0.82\tabularnewline
1000000 & 0.304166761 & 0.258099884 & 0.85\tabularnewline
\bottomrule
\end{longtable}

\subsubsection{Bulldozer
microarchitecture}\label{bulldozer-microarchitecture}

Much less optimization effort was made for Bulldozer than for Haswell
and Skylake, owing to the age and poor performance of this processor. No
code was written from scratch, but among all the existing
implementations for a given function, the one that ran fastest on
Bulldozer was chosen.

Among those functions that were replaced by faster versions, these three
are relevant to the FFT butterflies:

\begin{longtable}[c]{@{}lll@{}}
\toprule
Function & Old & New\tabularnewline
\midrule
\endhead
com\_n & 1.28 & 0.723\tabularnewline
rshift & 2 & 1.11\tabularnewline
lshift & 2.42 & 1.24\tabularnewline
\bottomrule
\end{longtable}

Timings for mpn\_mul\_n in seconds:

\begin{longtable}[c]{@{}llll@{}}
\toprule
Limbs & Old & New & Ratio\tabularnewline
\midrule
\endhead
10000 & 0.004771156 & 0.004764643 & 1.0\tabularnewline
100000 & 0.054624774 & 0.053038739 & 0.97\tabularnewline
1000000 & 0.651062127 & 0.652278285 & 1.0\tabularnewline
\bottomrule
\end{longtable}

Unfortunately, the improvements to the mpn\_{[}lr{]}shift functions are
barely visible in the integer multiplication benchmark on Bulldozer.

All code written for this deliverable has been committed to Alex
Kruppa's fork of the \texttt{MPIR} repository at
\url{https://github.com/akruppa/mpir} and merged into the main
\texttt{MPIR} repository at \url{https://github.com/wbhart/mpir} and
will be available in the MPIR-3.0.0 release, available at the
\href{http://mpir.org/}{\texttt{MPIR} website}.

Build instructions for \texttt{MPIR} are as follows:

Download MPIR-3.0.0 from: \href{http://mpir.org}{http://mpir.org/}

Note that you also need to have the latest \texttt{Yasm} assembler to
build MPIR: \url{http://yasm.tortall.net/}

To build \texttt{Yasm}, download the tarball:

\begin{verbatim}
./configure
make
\end{verbatim}

To test \texttt{MPIR}, download the tarball:

\begin{verbatim}
./configure --enable-gmpcompat --with-yasm=/path_to_yasm/yasm
make
make check
\end{verbatim}

A Haswell, Skylake, or Bulldozer CPU is required to test the changes
referred to above.

\subsection{Blog post}\label{blog-post}

A blog post about the design of the Flint FFT and the work done for this
project is available at
\small \url{https://wbhart.blogspot.de/2017/02/parallelising-integer-and-polynomial.html}.
