\subsection{Identifiers}

GAP uses a flat namespace: It identifies every operation by a name.

Different declarations whose types do not conflict, may use the same name.
These can be understood as giving two different types to the same operation.

Each operation can have multiple methods.
These do not have identifiers.
But operation references can be refine to a method reference by using the method's documentation string.

\subsection{Object Level}

GAP allows arbitrary run-time representations of mathematical objects (which is natural for efficiency).
These types of the run-time system are called families.
Each GAP object is typed by exactly one family.

An \textbf{object} is
\begin{compactitem}
  \item a literal,
  \item a list of objects,
  \item a function on object,
  \item any other object introduced by a user-declared family.
\end{compactitem}

Objects are typed.
A \textbf{type} is
\begin{compactitem}
  \item a base type (called a \textbf{family})
  \item a predicate subtype of some type by a unary predicate on objects (called a \textbf{filter})
\end{compactitem}

Users can declare new families.
But some families are built-in:
\begin{compactitem}
  \item one base type for each type of built-in literals:
    \begin{compactitem}
      \item cyclotomic numbers (elements of the algebraic closure of the rationals),
      \item booleans,
      \item strings,
    \end{compactitem}
  \item one base type each for several built-in operators that allow forming complex objects
    \begin{compactitem}
      \item homogeneous lists (called \textbf{collections}): lists of objects that have the same family,
      \item arbitrary lists of objects,
      \item functions on objects.
    \end{compactitem}
\end{compactitem}

A \textbf{filter} is a unary predicate on objects.
\begin{compactitem}
  \item the universal filter $\isobj$ (the filter of all objects)
  \item a category $C$,
  \item a property $P$,
  \item a conjunction $F\wedge G$ of filters.
\end{compactitem}
By convention, the names of atomic filters are of the form \lstinline|IsXXX|.

The \textbf{typing relation} is a binary relation between objects and filters.
We write it as $\has{O}{F}$.
It is defined by
\begin{compactitem}
  \item $\has{O}{\isobj}$ always
  \item $\has{O}{C}$ if $O$ was returned by a constructor of category $C$,
  \item $\has{O}{P}$ if evaluating $P$ on $O$ returns \lstinline|true|,
  \item $\has{O}{F\wedge G}$ if $\has{O}{F}$ and $\has{O}{G}$.
\end{compactitem}

For atomic filters $F$, the known state of the relation $\has{O}{F}$ is cached with $O$.
Thus, the type of every object can be inferred as the conjunction of atomic filters that are known to hold.
This type changes dynamically as more properties are evaluated.

\subsection{Declaration Level}

\paragraph{Categories}
A \textbf{category} declaration introduces a definition-less filter.
A category declaration consists of
\begin{compactitem}
  \item a name,
  \item a filter (called the superfilter).
\end{compactitem}
The concrete syntax is \lstinline|DeclareCategory(name: String, superfilter: Filter)|.

Categories can be used to represent the type of models of an abstract specification.
The details of the specification are formulated by declaring operations and properties on the category.

Because categories are definition-less, all categories are created empty.
The objects typed by the category are introduced by declaring constructors.
These are operations whose implementation explicitly marks the returned objects as having the category as a filter.

\paragraph{Operations}
An \textbf{operation} declaration introduces an $n$-ary\footnote{GAP has an implementation restriction of $n\leq 6$.} function on objects.
Operations are softly typed: Each $n$-ary operations provides a list of length $n$ providing the input filter of the respective argument.
Operations may also carry an optional return type, which defaults to $\isobj$ if omitted.\footnote{This is not implemented yet.}

The concrete syntax is
 \lstinline|DeclareOperation(name: String, inputfilters: Filter*, outputfilter: Filter?)|.

An \textbf{attribute} declaration can be seen as a special case of an operation that is unary.
The special treatment of attributes is important only for efficiency reasons: The values of attributes are cached with each object.
The concrete syntax is \lstinline|DeclareAttribute(name: String, inputfilter: Filter, outputfilter: Filter?)|.

A \textbf{property} declaration can be seen as a special case of an attribute that returns a boolean.
The special treatment of properties is important only because properties can be used as filters.
The concrete syntax is \lstinline|DeclareProperty(name: String, inputfilter: Filter)|.

An \textbf{constructor} declaration can be seen as a special case of an operation that returns an object of a given category.
The concrete syntax is \lstinline|DeclareConstructor(name: String, inputfilters: Filter*, outputfilter: Filter?)|.
\footnote{The return argument is not implemented yet.}
\footnote{The current implementation of constructors is somewhat awkward and may be subject to change. Currently, a constructor's first argument is special: It must be the expected return filter (rather than an object). This is used to allow method selection to choose a different method for different special cases.
A more elegant solution would be to allow every operation to declare that some of its arguments must be filters.
This would yield a very nice untyped version of bounded polymorphism, which is routine in typed programming languages.}

Conceptually, all operations are defined.
However, the definiens is declared separately through methods.

\paragraph{Methods}
Every operation can have multiple definitions, which are declared by methods.
A method declaration consists of
\begin{compactitem}
  \item the named of the operation,
  \item the input and output filters,
  \item the actual definition, as a function in the underlying programming language.
\end{compactitem}

The concrete syntax of a method declaration is
\lstinline|InstallMethod(operationname: String, inputfilters: Filter*, outputfilter: Filter?, definition: function)|.

The input and return filters of a method may be more restrictive than the filters used in the operation declaration.
More restrictive input filters can be used to represent overloading of operations or run-time polymorphism.
A more restrictive output filter can be used to indicate a sharper type than required by the operations.

When evaluating the application of an operation to arguments, one method is selected and its definition executed.
If more than one method is found, whose input filters are type the operation arguments, an internal ranking is used to disambiguate.

\subsection{Theory Level}

There is no explicit theory level.
Instead, theories are represented as categories, and theory morphisms as operations, and their relation is a special case of typing.

Therefore, we can treat each source file as a theory.

\subsection{Document Level}

Source files are grouped into folders and \textbf{packages}.
The package bundled with GAP is called the \textbf{library}.