\begin{event}{International Workshop on Software Engineering for
    Science}{se4science2016}{Austin (TX, USA),
    2016-05-16}{USO}{15}{http://se4science.org/workshops/se4science16/}

  \textbf{Main goals.} Spread recommendations to support better
  science in the area of software engineering for computational research.

  \textbf{ODK implication.} The work presented has been created with
  the upcoming Jupyter OOMMF integration in mind, and is of wider
  interest to the OpenDreamKit partners and users. The conference
  attendance was paid from a different grant.

  \textbf{Event summary.} Hans Fangohr delivered a talk on Software
  Engineering for Computational Science, in particular reviewing
  technical and social aspects of a computational science code that
  was developed about 10 years ago. The presentation, and associated
  publication \cite{16FangohrSE4Science} extracted lessons learned from the past and with the aim
  to enable the community to identify potential mistakes sooner. The
  presentation and work provides recommendations to enable better
  science in the field of computational science and engineering; in
  particular focusing on software engineering for computational
  science and research codes.

  The talk was the keynote presentation of the morning session in the
  workshop on Software Engineering for Science (30 minutes).

  \textbf{Demographic.} About 15 people were present, 3 female.

  \textbf{Results and impact.} We reported evidence from the
  effectiveness of particular sofware engineering techniques and
  provided recommendations for future projects (including user
  interface, testing, version control, documentation,
  installation).


\end{event}
