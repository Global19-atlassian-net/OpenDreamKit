\begin{event}{PyCon}{pycon}{Portland (Oregon), 2016-05-28 to 2016-06-02}{PS}{around 3000}{https://us.pycon.org/2016/}

\textbf{Main goals.} PyCon is the biggest python conference in the world. It is the best place to learn
about the python community, open-source tools, new technologies, etc. It is also a good place
to grow a network in the software development community.

\textbf{ODK implication.} Viviane Pons was present at the meeting for the third time in a row,
consolidating her effort to build links between Sage and python communities. In 2015, she had given
a talk and organized a parallel Sage Days event. It was not possible to do so this year but a
Sage sprint was still maintained.

\textbf{Results and impact.} The conference itself was very instructive as usual in thematics such as:
efficient programming, parallel computing, open-source community building, teaching, inclusivity. It
was a good occasion to discuss with other python programmers and introduce the ODK project. The academic
community did not seem as present as it had been in the previous years. In the future, we might want to
target smaller events such as SciPy and EuroScipy.
\end{event}