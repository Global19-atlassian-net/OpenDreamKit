\subsection{Relation to the Work Programme}

% \eucommentary{
% Indicate the work programme topic to which your proposal relates, and
% explain how your proposal addresses the specific challenge and scope
% of that topic, as set out in the work programme.}
\TOWRITE{ALL}{Table is too long. Could be made a little wider (left
  column). Needs fiddling to fit on page}
\enlargethispage{4cm}

Below we explain how the project addresses the specific challenge and
the scope of the topic ``E-infrastructures for Virtual Research
Environments (\VREs)'' under E-Infrastructures-2015 call, as set out in the work program.
\begin{center}
\begin{tabular}{|m{.37\textwidth}|m{.60\textwidth}|}
  \hline
  \textbf{Specific challenge} &
  \textbf{\TheProject contribution} \\\hline
  Empower researchers through development and deployment of service-driven
  digital research environments, services and tools tailored to their
  specific needs. &
  \TheProject will empower researchers in mathematics and applications by
  developing a service-driven tool, based on software, knowledge and data
  integration. Tailored to the researchers' specific needs and workflows,
  the \VREs will support the entire life-cycle of computational work in
  mathematical research. It will improve the productivity within the
  community by investigating better collaboration processes (\WPref{UI}), and
  identifying, sharing and promoting software development best
  practices (Objective~\ref{objective:community},  \ref{objective:social} and \ref{objective:disseminate}).\\\hline
  \VREs should integrate resources across all layers of the e-infrastructure
  (networking, computing, data, software, user interfaces) &

Our \VREs will be assembled from \TheProject components, which include mathematical \textit{software}
(from \WPref{component-architecture} and \WPref{hpc})
\textit{user interface} tools (from \WPref{UI}) and \textit{databases}
(from \WPref{dksbases}). In \WPref{hpc} we will adapt them to interface to
standard HPC and cloud environments, enabling \textit{computing
  resources} to be included in a VRE. Specialised networking is not
usually needed in this area.%
  %% \TheProject will integrate resources across all layers of the
  %% e-infrastructure: software development models, collaborative
  %% tools, data, component architecture, deployment frameworks,
  %% standardization, social aspects
  %% (Objectives~\ref{objective:framework},
  %% \ref{objective:sustainable} and \ref{objective:social},
  %% \WPref{social-aspects}), but also fostering collaboration inside
  %% the community, community enlargement and links with other
  %% scientific communities (Objectives~\ref{objective:community} and
  %% \ref{objective:demo}, \WPref{dissem} and \WPref{UI}).
  \\\hline
  \VREs should foster cross-disciplinary data interoperability. &
  \TheProject will foster a sustainable ecosystem of interoperable source
  components developed by overlapping communities, and data
  interoperability between different fields of mathematics (Objectives \ref{objective:community}, \ref{objective:updates} and \ref{objective:sustainable}).\\\hline
  \VREs should provide functions allowing data citation and promoting data
  sharing and trust. &
  The project will allow an easy, safe and efficient storage, reuse and
  sharing of rich mathematical data, taking account of provenance and
  citability. It will allow data sharing in a semantically sound way (Objectives~\ref{objectives:core}, \ref{objective:community}, \ref{objective:data} and \ref{objective:disseminate}), and
  make software sustainable, reusable and easily accessible (\WPref{component-architecture} and \WPref{dksbases}).\\\hline
  Each \VRE should abstract from the underlying e-infrastructures using
  standardised building blocks and workflows, well documented interfaces,
  in particular regarding APIs, and interoperable components &
  We will use building blocks with a sustainable development model that
  can be seamlessly combined together to build versatile high performance
  \VREs, each tailored to a specific need in pure mathematics and
  applications (Objectives~\ref{objectives:core} and \ref{objective:sustainable}). 
  We will develop and demonstrate (\WPref{dissem},\WPref{component-architecture}) a set of APIs enabling components
  such as database interfaces, computational modules, separate systems
  such as \GAP or \Sage to be flexibly combined
  and run smoothly across a wide range of environments (cloud, local,
  server etc.). Through well defined APIs, we will enable discovery of
  subsystems, functionality, documentation and computational
  resources.\\\hline
  %
  The \VREs proposals should clearly identify and build on requirements from
  real use cases &
  \TheProject will be built on requirements from use cases (\WPref{dissem}),
  including those involving industrial stakeholders. At the end of the
  project, the effectiveness of the \VREs will be demonstrated for a number
  of use cases from different domains (Objectives~\ref{objective:demo} and \ref{objective:disseminate}).\\\hline
  They should re-use tools and services from existing infrastructures and
  projects at national and/or European level as appropriate.  &
  \TheProject project brings together and integrates already existing tools
  and interactive scientific computing environments: \GAP, \Sage, \Linbox,
  \PariGP, \Singular and \Jupyter (\IPython), connected to databases, that will allow a
  huge gain in efficiency and productivity, enabling a large-scale
  collaboration on software, knowledge, and data (Objectives \ref{objective:community}, \ref{objective:updates}, \ref{objective:sustainable} and \ref{objective:disseminate}, \WPref{component-architecture} and \WPref{hpc}).\\\hline
%
Where data are concerned, projects will define the semantics,
ontologies, the \emph{what} metadata, as
well as the best computing models and levels of abstraction (e.g. by
means of open web services) to process the rich semantics at machine
level, as to ensure interoperability. &
We will investigate patterns to share data, ontologies, and semantics
across computational systems, possibly
connected remotely. We will leverage the well established semantics used
in mathematics (categories, type systems) to give powerful
abstractions on computational objects (Objectives~\ref{objectives:core} and \ref{objective:data}, \WPref{component-architecture} and \WPref{dksbases}).\\\hline
\end{tabular}
\end{center}
%\caption{Relation to the work program}
%\end{table}

\clearpage

%%% Local Variables: 
%%% mode: latex
%%% TeX-master: "proposal"
%%% End: 

%  LocalWords:  Programme eucommentary enlargethispage tablehead hline citability Linbox
%  LocalWords:  IPython emph clearpage TOWRITE textwidth textwidth textbf WPref dissem
%  LocalWords:  dksbases Jupyter hpc
