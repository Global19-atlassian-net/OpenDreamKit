\documentclass{deliverablereport}

\duedate{26/04/2017}
\deliverydate{16/04/2017}


\begin{document}
\enlargethispage{4ex}
\maketitle
\githubissuedescription
\tableofcontents\newpage



\section{Explanation of the work carried out by the beneficiaries and Overview of the progress}

%•	Explain the work carried out during the reporting period in line with the Annex 1 to the Grant Agreement. 
%•	Include an overview of the project results towards the objective of the action in line with the structure of the Annex 1 to the Grant Agreement including summary of deliverables and milestones, and a summary of exploitable results and an explanation about how they can/will be exploited.
%(No page limit per workpackage but report shall be concise and readable. Any duplication should be avoided).

\subsection{Objectives}

%List the specific  objectives  for  the  project  as  described  in  section  1.1  of  Part B   and describe  the  work  carried  out  during  the  reporting  period  towards  the  achievement  of  each listed objective. Provide clear and measurable details.

\subsection{Explanation of the work carried per Work Package}
\subsubsection{WorkPackage 1}

%Explain, task per task, the work carried out in WP1 during the reporting period giving details of the work carried out by each beneficiary involved.

As planned in WP1, \site{PS} has been coordinating \ODK.  Most of the
management effort for year 1 has been made in
\longtaskref{management}{project-finance-management} (see
\delivref{management}{infrastructure} and
\delivref{management}{data-plan1}).  A consortium agreement was signed
between partners, stating precise rules about topics such as:
responsibilities, governance, access to results and the background
included. UGent has recently agreed to sign this Consortium Agreement
without any modification to it.

Communication underpins a distributed network of researchers and
software developers. The website for the project has been continuously
updated with new content, and virtually all work in progress is openly
accessible on the Internet to external experts and contributors (for
example through open source software on Github). A modified, more
outward facing version of the webpages is under development and will
transition into a long-term dissemination and communication tool.
The project has a number of mailing lists, which are currently
reviewed to better support increasing dissemination and communication
beyond the consortium.

During year 1, a kick-off meeting was organised in Orsay, followed by
two progress meetings at which partners presented status reports, and
the steering committee got together.  The first progress meeting was
organised in St Andrews (January 2016) and the second one was located
in Bremen (June 2016).  The latter coincided with the interim project
review, planned at month 9, where deliverables due by then were
presented to the Project Officer and Reviewers. The \ODK\ project was
granted the grade 3 out of 4 for this interim review: ``Good progress
(the project has achieved most of its objectives and technical goals
for the period with relatively minor deviations)''.

Due to the length of the first reporting period (18 months), the
\site{PS} administration decided to organise an internal and interim
breakdown of costs. This exercise aimed at raising potential questions
from partners early on and to make sure partners do follow the EC
rules for the eligibility of costs.

More information on
\longtaskref{management}{project-quality-management} can be found in
Section 4 of this document: Quality assurance plan.
Task \longtaskref{management}{project-innovation-management} will provide
further detail and will be available at month 18.


\subsubsection{WorkPackage 2}
%idem

As planned in \longtaskref{dissem}{dissemination-communication} and
\longtaskref{dissem}{dissemination}, 14 meetings, developer and
training workshops have been organized and co-organized by \ODK during
year 1, and complemented by many presentations and activities in
external events.  Many more are being prepared, including the first
Women in Sage workshop in Europe and three major training conferences
(tentatively at CIRM, Dagstuhl, and ICMS); \ODK and \ODK related work
is regularly presented at conferences (see the report for
\longdelivref{dissem}{workshops-1}).

Two additional workshops have been delivered for the micromagnetic user community
in which the prototypes of the Jupyter and Python interface to the
micromagnetic community's widely used OOMMF simulation code has been demonstrated,
taught and feedback from the users sought (part of
\longtaskref{dissem}{dissemination-of-oommf-nb-virtual-environment})

\ODK\ is also working on its visibility and communication strategy. The
\longdelivref{dissem}{press-release-1} was delivered and a page for the
\href{https://github.com/OpenDreamKit/OpenDreamKit/blob/master/Communication/eInfra-Booklet/ODK.md}{E-infrastructure
  booklet} was written jointly by \ODK members. After one year, we have a clearer
understanding of what is needed by the project.  We are working on a new organization for
the website where day to day activities would be more visible through our blog. Posts
include reports on conferences, workshops, new features and emerging technologies (as part
of \delivref{dissem}{workshops-1}).


\subsubsection{WorkPackage 3}
%idem
The first task of this workpackage is to improve the portability of
computational components
\longtaskref{component-architecture}{portability}. A particular
challenge is the portability of \Sage\ (and therefore all its
dependencies) on Windows, which has remained elusive for a decade,
despite many efforts of the community. We are happy to report that, in
particular thanks to months of intensive and expert work by our
recruit Erik Bray at \site{PS}, this challenge is about to be
tackled, almost one year before the expected delivery time.

Task \taskref{component-architecture}{interface-systems} on interfaces
between mathematical systems is progressing as expected. Experimental
work on a semantic interface between \GAP and \Sage
(\delivref{component-architecture}{semantic-interface-sage-gap}, due
on month 36) has started during the joint GAP-Sage days, and a working
prototype is already available. The current prototype uses \emph{ad
  hoc} language mechanisms to transfer the semantics from one system
to the other; these mechanisms will be replaced with a generic API
(Application programming Interface) once the MitM (Math-in-the-Middle)
approach developed in WP6 will be mature enough. Meanwhile, a purely
technical piece of the puzzle has been already achieved as part of
\delivref{component-architecture}{scscp-sage}, bringing support for the SCSCP
protocol\footnote{SCSCP, the
  \href{http://www.symbolic-computing.org/science/index.php/SCSCP}{Symbolic
    Computation Software Composability Protocol} is a web protocol
  based on OpenMath enabling remote procedure calls across computer
  algebra systems}
to the Python ecosystem, and thus to \Sage and its subsystems. This
is instrumental for supporting the MitM approach.

After \longdelivref{component-architecture}{virtual-machines} was
delivered, a focused workshop in March (Sage Days 77) also triggered
much work and progress on the packaging side
(\taskref{component-architecture}{mod-packaging}), both by \ODK
participants and the community. There is now good hope to have proper
packages for \Sage (and its dependencies) on the Debian distribution
in the coming months, a feature that has been desperately longed for
for over a decade.  The workshop was also the occasion to clarify the
modularization, packaging, and distribution needs and
challenges. Internal notes on the progress made have been taken in the
\href{https://wiki.sagemath.org/days77/packaging}{\Sage wiki}, and a
\href{https://groups.google.com/forum/#!forum/sage-packaging}{mailing
  list specifically dedicated to packaging \Sage} has been created.

Task \longtaskref{component-architecture}{oommf-python-interface} has
been completed and is available online on
\href{https://github.com/joommf/oommfc}{github}, and through the
Python packaging index.

\subsubsection{WorkPackage 4}
%idem
The first task for this workpackage is to enable the use of Jupyter as
uniform notebook interface for the relevant computational components
\taskref{UI}{ipython-kernels}. This is well under way for most
components. Progress was particularly fast for \Sage thanks to a very
active involvement of the community; this will enable, in the coming
months, a systematic transition from the legacy \Sage notebook system
to \Jupyter; this is a particularly important achievement: beside all
the benefits of a uniform and actively developed interface for the
user, outsourcing the maintenance of the notebook interface will save
the \Sage community much needed resources.

A new \Jupyter package, nbdime, was created for
\delivref{UI}{jupyter-collab} enabling easier collaboration on
notebooks via version control systems such as git.  This project was
presented at the major Scientific Python conferences SciPy US in July and EuroSciPy in August, and has been
met with enthusiasm from the scientific Python community for its
prospect of solving a longstanding difficulty in working with
notebooks.  Work has begun on a new package, nbval, for
\delivref{UI}{jupyter-test}, which will integrate the above nbdime
package for delivering testable, reproducible notebooks via
traditional software development testing practices.

The JupyterHub package has received updates and further development,
specifically a Services extension point, which enables shared
workspaces for collaboration, a step on the path toward real-time
collaboration for \delivref{UI}{jupyter-live-collab}.

Active structured documents are a common need with many use cases, and
as many potential solutions. Requirements and venues for
collaborations were explored through discussions between participants,
in particular at the occasion of
\href{https://wiki.sagemath.org/days77/}{Sage Days 77} workshop (see
the
\href{https://wiki.sagemath.org/days77/live-structured-documents}{notes}),
and June's ODK meeting in Bremen. The findings were reported in
\delivref{UI}{adstex}. Sage Days 77 was also the occasion to bootstrap
the long term work of
\href{https://wiki.sagemath.org/days77/documentation}{refactoring the
  Sage documentation build system} (\delivref{UI}{sage-sphinx}) in
collaboration with a Sphinx developer.

One deliverable, \delivref{UI}{pari-python-lib1}, was delayed by a
couple months due to unforeseen technical difficulties, but with no
impact on the rest of the project.


\subsubsection{WorkPackage 5}
%idem
After \delivref{hpc}{sage-paral-tree} was delivered,
\longtaskref{hpc}{pythran} is making good progress towards the
development of Pythran and its interaction with~\Sage.
%
More precisely, Pythran's typing system (\delivref{hpc}{pythran-typing}) has
been improved in two ways: first, the compiler now more accurately tracks the
identifier $\leftrightarrow$ value binding, which in turns makes it possible to
generate strongly typed code for a wider class of Python kernels.  Second, an
unsound type checker for Pythran has been developed. It provides human-readable
error report when a type error is detected at compile time, when a cryptic
internal error was previously reported. Both algorithms have been extensively
detailed in separated blog posts and the resulting implementation is part of
the official Pythran 0.8.0 release
(\href{https://github.com/OpenDreamKit/OpenDreamKit/issues/117}{details}).

The start of Deliverable~\delivref{hpc}{pythran-sage} has been delayed from
Month 12 to Month 18 due to the difficulty of hiring an engineer for the task.
It is now making good progress.

Deliverable~\longdelivref{hpc}{MPIRsuperoptimiser} is making
progress but has hit a major blocker: it requires to use a precise
clock cycle counter, for which a kernel module has been proposed in
this deliverable. However a bug in the Linux kernel seem to
automatically disable these counters. It has been reported upstream
but the long delay to have a patch incorporated into the kernel will
impact the delivery of this deliverable. There is no dependency to
this deliverable, and the problem is now well understood and its
solution is underway
(\href{https://github.com/OpenDreamKit/OpenDreamKit/issues/118}{details}).

As mentioned in the initial proposal, work on
\taskref{hpc}{hpc-linbox} (LinBox) is just starting in September 2016 instead
of December 2015, as the members of this group had already too many PM involved in other
projects. However this will not not have any impact on the workplan as the PM will be doubled
in the three first trimesters of year 4.

A first workshop on HPC will be organized in March 2017 in Grenoble,
France. A smaller workshop may possibly be organized in December 2016,
to gather participants involved in the development and Pythran.

\subsubsection{WorkPackage 6}
%idem
In a series of workshops (September 2015 in Paris, January 2016 in
St. Andrews, June 2016 in Bremen, and July 2016 in Bia{\l}ystok) the
participants working on \WPref{dksbases} met and discussed the topic
of integrating the \pn systems into a mathematical VRE toolkit.  Key
results were
\begin{compactitem}[\bf R1.]
\item the observation that \emph{knowledge-aware interoperability of
    software and database-systems is the most critical objective} for
  \WPref{dksbases} in the \pn project.
\item the consensus that this can be achieved by \emph{aligning the
    mathematical knowledge underlying the various systems}.
\end{compactitem}
This requires explicitly representing the three aspects of math VREs
-- Data (D), Knowledge (K), and Software (S) -- and basing
computational services and inter-system communication on a joint
\DKS-base. These results are engrained in the ``Math-in-the-Middle''
(MitM) paradigm~\cite{DehKohKon:iop16}, which gives a representational
basis for specification-based interoperability of mathematical
software systems -- so that they can be integrated in a VRE
toolkit. In the MitM paradigm, the mathematical knowledge underlying
the VREs (K) and the the interface of the for each system (S) are
represented as modular theory graphs in the OMDoc/MMT format. For the
data aspect (D) we have extended the concept of OMDoc/MMT theories to
``virtual theories'' that allow the practical management of possibly
infinite theories, see~\cite{ODK-D6.2} for details.

A side effect of the \textbf{R1.} is that the verification aspects
anticipated in the proposal are non-critical to the \pn project. In
particular the value of the exemplary verification of an LMFDB
algorithm in \taskref{dksbases}{data-LMFDB} and
deliverable~\delivref{dksbases}{lfmverif} seems highly questionable.

Correspondingly we have refined the notion of ``triformal theories''
coined in the proposal into the concept of ``\DKS theory graphs'',
which can be formalized and implemented without the extension of
OMDoc/MMT for ``biformal theories'' anticipated in the proposal.

Through the concerted effort of the WP6 participants, we have been
able to implement this design into prototypical \DKS base patterned
after the MitM paradigm with virtual theories, generating interface
theory graphs for the \GAP and \Sage systems and integrating the
\LMFDB system via the MitM codec architecture described
in~\cite{ODK-D6.2}. Based on this, we were able to generically
integrate \GAP, \Sage, and \LMFDB via the standardised SCSCP
protocol~\cite{HorRoz:ossp09} -- essentially remote procedure calls
with OpenMath Objects. This case study shows the feasibility of the
initial design of \DKS-bases; further investigations and the
integration of additional systems will determine the practicability.

\subsubsection{WorkPackage 7}
%idem
Ursula Martin has stepped down in anticipation of her upcoming
retirement, and Dmitrii Pasechnik has become the lead PI for this work
package.  This somewhat slowed down the takeoff of this work package,
but the important deliverables are well on track, if not ahead of
schedule. Nevertheless, the consortium may need to request for a new
amendment after the Review of the 1st Reporting Period to clarify the
situation.

As planned, the work was focused on bootstrapping
\longtaskref{social-aspects}{social-input} and
\longtaskref{social-aspects}{isocial-decisionmaking} whose early
outcomes will nurture the design of \ODK's VREs in other work
packages.

For \taskref{social-aspects}{social-input},
\longdelivref{social-aspects}{social-datareport} analysing in
particular the state of affairs in our model system \Sage, is well on
track to be delivered at Month 18, with Part I ready.

For \longtaskref{social-aspects}{isocial-decisionmaking},
\delivref{social-aspects}{social-tracaddon} is largely ready and is to
be tested on the system \Sage; a paper
\cite{Pavlou:2016:MCI:2936924.2936934} forming a part of
\delivref{social-aspects}{social-gametheoretic} has been published.

Finally, early work was done for
\taskref{social-aspects}{social-output} on connections of
reproducibility, crowdsourcing, and a VRE as a mean to test and
control the former: a paper \cite{2016arXiv160100181C} analysing a
concrete well-established area of combinatorics in this respect, and
describing the implementation of the corresponding meta-database in
\Sage system has been published.

\subsection{Impact}
%Include in this section whether the information on section 2.1 of the DoA  (how your project will contribute to the expected impacts) is still relevant or needs to be updated. Include further details in the latter case

\subsection{Infrastructures}
%If access to research infrastructures has been provided under the grant please include access provision activities

\subsubsection{Trans-national Access Activities (TA)}

%Provide for the set of TA Work Packages, the integrated information described below.

% - Description of the publicity concerning the new opportunities for access 
%In  the  first  periodic  report  describe  the  measures  taken  to  publicise  to  research  teams 
%throughout Europe the opportunities for access open to them under the Grant Agreement. In 
%the following periodic reports indicate only additional measures and changes.

% - Description of the selection procedure
%In  the  first  periodic  report,  describe  the  procedure  used  to  select  users:  organisation  of  the 
%Selection Panel, any additional selection criteria employed by the Selection Panel, measures 
%to promote equal opportunities, etc. Specify if feedback is given to rejected applicants and in 
%which form. In the following periodic reports indicate only changes to the existing procedure. 

%The list of the Selection Panel members should be maintained and update when necessary in 
%order to prove that the panel is composed following the conditions indicated in Article 16.1 of 
%the GA. The Commission reserves the right to request this list at any time.

%Indicate number, date and venue (if not carried out remotely) of the meetings of the Selection 
%panel during the reporting period. 

%Provide integrated information on the selection of user projects and on the scientific output of 
%supported users. In particular indicate the number of eligible User projects submitted in the 
%reporting period and the number of the selected ones taking into account only calls for which 
%the  selection  has  been  completed  in  the  reporting  period.  Indicate  also  the  number  of  user 
%projects,  started  and  supported  in  the  reporting  period,  which  have  a  majority  of  users  not 
%working in an EU or associated country


% - Description of the Trans-national Access activity 
%Give an overview of the user-projectsand users supported in the reporting period indicating 
%their number, their scientific fields and other relevant information you may want to highlight. 
%You should maintain the list of the user-projects for which costs have been incurred in the 
%reporting period.  A user-project can run over more than one reporting period. In this case it 
%should be inserted in the list of each concerned reporting period. 
%The list of user-projects must include, for each user-project, the acronym, objectives, as well 
%as  the  amount  of  access  granted  to  it  on  each  installation  used  by  the  user-project  in  the 
%reporting period. When the user-project is completed in the reporting period the list should 
%also include a short description of the work carried out. The Commission reserves the right to 
%request this list at any time.

%In addition you must fill the following tables (in Part A to be filled in the IT tool):

% - List  of  users:   Researchers  who  have  access  to  research  infrastructures/installations 
%(one or more) through Union support under the grant either in person (through visit) or 
%through remote access; 

% - Research  infrastructures  made  accessible  to  all  researchers  in  Europe  and  beyond 
%through  EU  support  and  summary  of  trans-national  access  provision  per  installation 
%per  reporting  period  indicate  for  each  installation  providing  trans-national  access 
%under  the  project  the  quantity  of  access  actually  provided  in  the  Reporting  Period 
%(expressed in the unit of access defined in Annex 1 for that specific installation). 


%    *Scientific output of the users at the facilities:
%Give highlights of important research results from the user-projects supported under the grant 
%agreement.  Indicate  the  number  and  the  type  of  publications  derived  by  user-projects 
%supported under the grant taking into  account only  publications that acknowledge the support 
%of this EU grant. 
%You  should  maintain  a  list  of  publications  that  have  appeared  in  journals  (or  conference 
%proceedings) during the  reporting period and are resulting from work  carried out under the 
%Trans-national  Access  activity.  List  only  publications  that  acknowledge  the  support  of  the 
%European Community. For each publication indicate: the acronyms of the user-projects that 
%have  led  to  the  publication  itself,  the  authors,  the  title,  the  year  of  publication,  the  type  of 
%publication  (Article  in  journal,  Publication  in  conference  proceeding/workshop, 
%Book/Monograph, Chapters in book, Thesis/dissertation, whether it has been peer-reviewed or 
%not,  the  DoI  (Digital  Object  Identifier),  the  publication  references,  and  whether  the 
%publication  is  available  under  Open  Access  or  not.  The  Commission  reserves  the  right  to 
%request this list at any time. 


%    *User meetings
%If any user meetings have been organised in the reporting period, indicate for each of them the 
%date,  the  venue,  the  number  of  users  attending  the  meeting  and  the  overall  number  of 
%attendees.


\subsubsection{Virtual Access Activities (VA)}

%Provide for the set of VA Work Packages, the integrated information described below..

%Provide statistics on the virtual access in the period by each installation, including quantity, 
%geographical distribution of users and, whenever possible, information/statistics on scientific 
%outcomes (publications, patents, etc.) acknowledging the use of the infrastructure.

%As indicated in Art. 16.2, the access providers must have the virtual access services assessed 
%periodically by a board composed of international experts in the field, at least half of whom 
%must  be  independent  from  the  beneficiaries.  In  the  first  periodic  report,  describe  how  the 
%virtual  access  providers  will  comply  with  this  obligation.  In  the  following  periodic  reports 
%indicate only changes to the existing procedure. 

%When an assessment is scheduled under the reporting period, the assessment report must be 
%submitted as deliverable


\section{Update of the plan for exploitation and dissemination of result (if applicable)}
%Include  in  this  section  whether  the  plan  for  exploitation  and  dissemination  of  results  as 
%described in the DoA needs to be updated and give details. 



\section{4. Follow-up of recommendations and comments from previous review(s) (if applicable)}
%Include in this section the list of recommendations and comments from previous reviews and 
%give information on how they have been followed up


\section{Deviations from Annex 1 (if applicable)}
%Explain  the  reasons  for  deviations  from  the  DoA,  the  consequences  and  the  proposed 
%corrective actions

\subsection{Tasks}
%Include  explanations  for  tasks  not  fully  implemented,  critical  objectives  not  fully  achieved 
%and/or  not  being  on  schedule.  Explain  also  the  impact  on  other  tasks  on  the  available 
%resources and the planning


\subsection{Use of resources}
%Include explanations on deviations of the use of resources between actual and planned use of 
%resources in Annex 1, especially related to person-months per work package.

\subsection{Unforeseen subcontracting (if applicable)}

%Specify in this section: 
%a)  the work (the tasks) performed by a subcontractor which may cover only a limited part of 
%the project;
%b)  explanation of the circumstances which  caused  the need for  a subcontract, taking into 
%account the specific characteristics of the project;
%c)  the  confirmation  that  the  subcontractor  has  been  selected  ensuring  the  best  value  for 
%money or, if appropriate, the lowest price and avoiding any conflict of interests

\subsection{Unforeseen use of in kind contribution from third party against payment or free 
of charges (if applicable)}

%Specify in this section:
%d)  the identity of the third party;
%e)  the resources made available by the third party respectively  against payment or free of 
%charges 
%f)  explanation  of  the  circumstances  which  caused   the  need  for  using  these  resources  for 
%carrying out the work. 

\end{document}

%%% Local Variables:
%%% mode: latex
%%% TeX-master: t
%%% End:

%  LocalWords:  maketitle githubissuedescription newpage newcommand xspace Jupyter dissem
%  LocalWords:  tableofcontents visualizations composability itemize analyzed taskref hpc
%  LocalWords:  dissemination-of-oommf-nb-virtual-environment taskref dissem taskref pn
%  LocalWords:  dissemination-of-oommf-nb-workshops dissem ibook taskref taskref taskref
%  LocalWords:  oommf-python-interface oommf-py-ipython-attributes taskref oommf-nb-ve
%  LocalWords:  oommf-tutorial-and-documentation taskref oommf-nb-evaluation taskrefs
%  LocalWords:  delivref pythran-typing sage-paral-tree subsubsection organized Dagstuhl
%  LocalWords:  co-organized organization modularization ipython-kernels nbdime Pythran
%  LocalWords:  jupyter-collab ystok WPref dksbases compactitem emph WPtref DehKohKon
%  LocalWords:  iop16 textbf tasktref lfmverif triformal formalized biformal ossp09 Dima
%  LocalWords:  hline Marijan Pilorget Pierrick Kruppa Dehaye Dehaye's Dehaye's Alnaes
%  LocalWords:  Konovalov Hinsen github printbibliography
