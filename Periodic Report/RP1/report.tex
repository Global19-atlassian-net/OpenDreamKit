\documentclass{deliverablereport}

\duedate{26/04/2017}
\deliverydate{16/04/2017}


\begin{document}
\enlargethispage{4ex}
\maketitle
\githubissuedescription
\tableofcontents\newpage



\section{Explanation of the work carried out by the beneficiaries and Overview of the progress}

%•	Explain the work carried out during the reporting period in line with the Annex 1 to the Grant Agreement. 
%•	Include an overview of the project results towards the objective of the action in line with the structure of the Annex 1 to the Grant Agreement including summary of deliverables and milestones, and a summary of exploitable results and an explanation about how they can/will be exploited.
%(No page limit per workpackage but report shall be concise and readable. Any duplication should be avoided).

\subsection{Objectives}

%List the specific  objectives  for  the  project  as  described  in  section  1.1  of  Part B   and describe  the  work  carried  out  during  the  reporting  period  towards  the  achievement  of  each listed objective. Provide clear and measurable details.

\subsection{Explanation of the work carried per Work Package}
\subsubsection{WorkPackage 1}

%Explain, task per task, the work carried out in WP1 during the reporting period giving details of the work carried out by each beneficiary involved.

\subsubsection{WorkPackage 2}
%idem

\subsubsection{WorkPackage 3}
%idem

\subsubsection{WorkPackage 4}
%idem

\subsubsection{WorkPackage 5}
%idem

\subsubsection{WorkPackage 6}
%idem

\subsubsection{WorkPackage 7}
%idem


\subsection{Impact}
%Include in this section whether the information on section 2.1 of the DoA  (how your project will contribute to the expected impacts) is still relevant or needs to be updated. Include further details in the latter case

\subsection{Infrastructures}
%If access to research infrastructures has been provided under the grant please include access provision activities

\subsubsection{Trans-national Access Activities (TA)}

%Provide for the set of TA Work Packages, the integrated information described below.

% - Description of the publicity concerning the new opportunities for access 
%In  the  first  periodic  report  describe  the  measures  taken  to  publicise  to  research  teams 
%throughout Europe the opportunities for access open to them under the Grant Agreement. In 
%the following periodic reports indicate only additional measures and changes.

% - Description of the selection procedure
%In  the  first  periodic  report,  describe  the  procedure  used  to  select  users:  organisation  of  the 
%Selection Panel, any additional selection criteria employed by the Selection Panel, measures 
%to promote equal opportunities, etc. Specify if feedback is given to rejected applicants and in 
%which form. In the following periodic reports indicate only changes to the existing procedure. 

%The list of the Selection Panel members should be maintained and update when necessary in 
%order to prove that the panel is composed following the conditions indicated in Article 16.1 of 
%the GA. The Commission reserves the right to request this list at any time.

%Indicate number, date and venue (if not carried out remotely) of the meetings of the Selection 
%panel during the reporting period. 

%Provide integrated information on the selection of user projects and on the scientific output of 
%supported users. In particular indicate the number of eligible User projects submitted in the 
%reporting period and the number of the selected ones taking into account only calls for which 
%the  selection  has  been  completed  in  the  reporting  period.  Indicate  also  the  number  of  user 
%projects,  started  and  supported  in  the  reporting  period,  which  have  a  majority  of  users  not 
%working in an EU or associated country


% - Description of the Trans-national Access activity 
%Give an overview of the user-projectsand users supported in the reporting period indicating 
%their number, their scientific fields and other relevant information you may want to highlight. 
%You should maintain the list of the user-projects for which costs have been incurred in the 
%reporting period.  A user-project can run over more than one reporting period. In this case it 
%should be inserted in the list of each concerned reporting period. 
%The list of user-projects must include, for each user-project, the acronym, objectives, as well 
%as  the  amount  of  access  granted  to  it  on  each  installation  used  by  the  user-project  in  the 
%reporting period. When the user-project is completed in the reporting period the list should 
%also include a short description of the work carried out. The Commission reserves the right to 
%request this list at any time.

%In addition you must fill the following tables (in Part A to be filled in the IT tool):

% - List  of  users:   Researchers  who  have  access  to  research  infrastructures/installations 
%(one or more) through Union support under the grant either in person (through visit) or 
%through remote access; 

% - Research  infrastructures  made  accessible  to  all  researchers  in  Europe  and  beyond 
%through  EU  support  and  summary  of  trans-national  access  provision  per  installation 
%per  reporting  period  indicate  for  each  installation  providing  trans-national  access 
%under  the  project  the  quantity  of  access  actually  provided  in  the  Reporting  Period 
%(expressed in the unit of access defined in Annex 1 for that specific installation). 


%    *Scientific output of the users at the facilities:
%Give highlights of important research results from the user-projects supported under the grant 
%agreement.  Indicate  the  number  and  the  type  of  publications  derived  by  user-projects 
%supported under the grant taking into  account only  publications that acknowledge the support 
%of this EU grant. 
%You  should  maintain  a  list  of  publications  that  have  appeared  in  journals  (or  conference 
%proceedings) during the  reporting period and are resulting from work  carried out under the 
%Trans-national  Access  activity.  List  only  publications  that  acknowledge  the  support  of  the 
%European Community. For each publication indicate: the acronyms of the user-projects that 
%have  led  to  the  publication  itself,  the  authors,  the  title,  the  year  of  publication,  the  type  of 
%publication  (Article  in  journal,  Publication  in  conference  proceeding/workshop, 
%Book/Monograph, Chapters in book, Thesis/dissertation, whether it has been peer-reviewed or 
%not,  the  DoI  (Digital  Object  Identifier),  the  publication  references,  and  whether  the 
%publication  is  available  under  Open  Access  or  not.  The  Commission  reserves  the  right  to 
%request this list at any time. 


%    *User meetings
%If any user meetings have been organised in the reporting period, indicate for each of them the 
%date,  the  venue,  the  number  of  users  attending  the  meeting  and  the  overall  number  of 
%attendees.


\subsubsection{Virtual Access Activities (VA)}

%Provide for the set of VA Work Packages, the integrated information described below..

%Provide statistics on the virtual access in the period by each installation, including quantity, 
%geographical distribution of users and, whenever possible, information/statistics on scientific 
%outcomes (publications, patents, etc.) acknowledging the use of the infrastructure.

%As indicated in Art. 16.2, the access providers must have the virtual access services assessed 
%periodically by a board composed of international experts in the field, at least half of whom 
%must  be  independent  from  the  beneficiaries.  In  the  first  periodic  report,  describe  how  the 
%virtual  access  providers  will  comply  with  this  obligation.  In  the  following  periodic  reports 
%indicate only changes to the existing procedure. 

%When an assessment is scheduled under the reporting period, the assessment report must be 
%submitted as deliverable


\section{Update of the plan for exploitation and dissemination of result (if applicable)}
%Include  in  this  section  whether  the  plan  for  exploitation  and  dissemination  of  results  as 
%described in the DoA needs to be updated and give details. 



\section{4. Follow-up of recommendations and comments from previous review(s) (if applicable)}
%Include in this section the list of recommendations and comments from previous reviews and 
%give information on how they have been followed up


\section{Deviations from Annex 1 (if applicable)}
%Explain  the  reasons  for  deviations  from  the  DoA,  the  consequences  and  the  proposed 
%corrective actions

\subsection{Tasks}
%Include  explanations  for  tasks  not  fully  implemented,  critical  objectives  not  fully  achieved 
%and/or  not  being  on  schedule.  Explain  also  the  impact  on  other  tasks  on  the  available 
%resources and the planning


\subsection{Use of resources}
%Include explanations on deviations of the use of resources between actual and planned use of 
%resources in Annex 1, especially related to person-months per work package.

\subsection{Unforeseen subcontracting (if applicable)}

%Specify in this section: 
%a)  the work (the tasks) performed by a subcontractor which may cover only a limited part of 
%the project;
%b)  explanation of the circumstances which  caused  the need for  a subcontract, taking into 
%account the specific characteristics of the project;
%c)  the  confirmation  that  the  subcontractor  has  been  selected  ensuring  the  best  value  for 
%money or, if appropriate, the lowest price and avoiding any conflict of interests

\subsection{Unforeseen use of in kind contribution from third party against payment or free 
of charges (if applicable)}

%Specify in this section:
%d)  the identity of the third party;
%e)  the resources made available by the third party respectively  against payment or free of 
%charges 
%f)  explanation  of  the  circumstances  which  caused   the  need  for  using  these  resources  for 
%carrying out the work. 

\end{document}

%%% Local Variables:
%%% mode: latex
%%% TeX-master: t
%%% End:

%  LocalWords:  maketitle githubissuedescription newpage newcommand xspace Jupyter dissem
%  LocalWords:  tableofcontents visualizations composability itemize analyzed taskref hpc
%  LocalWords:  dissemination-of-oommf-nb-virtual-environment taskref dissem taskref pn
%  LocalWords:  dissemination-of-oommf-nb-workshops dissem ibook taskref taskref taskref
%  LocalWords:  oommf-python-interface oommf-py-ipython-attributes taskref oommf-nb-ve
%  LocalWords:  oommf-tutorial-and-documentation taskref oommf-nb-evaluation taskrefs
%  LocalWords:  delivref pythran-typing sage-paral-tree subsubsection organized Dagstuhl
%  LocalWords:  co-organized organization modularization ipython-kernels nbdime Pythran
%  LocalWords:  jupyter-collab ystok WPref dksbases compactitem emph WPtref DehKohKon
%  LocalWords:  iop16 textbf tasktref lfmverif triformal formalized biformal ossp09 Dima
%  LocalWords:  hline Marijan Pilorget Pierrick Kruppa Dehaye Dehaye's Dehaye's Alnaes
%  LocalWords:  Konovalov Hinsen github printbibliography
