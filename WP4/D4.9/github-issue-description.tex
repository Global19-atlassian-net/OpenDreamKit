\section*{\texorpdfstring{Deliverable description, as taken from Github
issue
\href{https://github.com/OpenDreamKit/OpenDreamKit/issues/97}{\#97} on
2017-02-27}{Deliverable description, as taken from Github issue \#97 on 2017-02-27}}\label{deliverable-description-as-taken-from-github-issue-97-on-2017-02-27}

\begin{itemize}
\tightlist
\item
  \textbf{WP4:}
  \href{https://github.com/OpenDreamKit/OpenDreamKit/tree/master/WP4}{User
  Interfaces}
\item
  \textbf{Lead Institution:} Jacobs University Bremen
\item
  \textbf{Due:} 2017-02-28 (month 18)
\item
  \textbf{Nature:} Demonstrator
\item
  \textbf{Task:} T4.6
  (\href{https://github.com/OpenDreamKit/OpenDreamKit/issues/74}{\#74})
\item
  \textbf{Proposal:}
  \href{https://github.com/OpenDreamKit/OpenDreamKit/raw/master/Proposal/proposal-www.pdf}{p.~48}
\item
  \textbf{\href{https://github.com/OpenDreamKit/OpenDreamKit/raw/master/WP4/D4.9/report-final.pdf}{Final
  report}}
\end{itemize}

One of the most prominent features of a virtual research environment
(VRE) is a unified user interface (UI). There are two complementary
approaches that can serve as a basis for OpenDreamKit's mathematical VRE
UI: computational notebooks and active structured documents. The former
allows for mathematical text around the computation cells of a
real-eval-print loop of a mathematical computational system and the
latter makes semantically annotated documents active.

In D4.2 ``Active/Structured Documents Requirements and existing
Solutions''
(\href{https://github.com/OpenDreamKit/OpenDreamKit/issues/91}{\#91}) we
reported on two systems in the OpenDreamKit project which follow
respectively those two approaches: \href{http://jupyter.org}{Jupyter} --
a notebook server for various computational systems -- and
\href{http://MathHub.info}{MathHub.info} -- a platform for active
mathematical documents. We identified commonalities and differences and
developed a vision for integrating their functionalities.

As a first step into this direction we explore in this deliverable the
requirements of integrating in-situ (i.e.~in-document) computation -- a
forte and indeed the raison-d'être of notebooks -- into conventional,
narrative-structured mathematical documents. We present, analyze, and
classify examples for in-situ computation and explore -- in particular
in a \href{http://MathHub.info}{MathHub.info} based prototype -- how the
active documents technology has to be extended to accommodate this
functionality as a semantic service.
