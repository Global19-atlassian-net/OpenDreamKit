\begin{abstract}
  One of the most prominent features of a virtual research environment (VRE) is a unified
  user interface. The \pn approach is to create a mathematical VRE by integrating various
  pre-existing mathematical software systems. There are two approaches that can serve as a
  basis for the \pn UI: ``computational notebooks'' and ``active documents''. The former
  allow mathematical text around the computation cells of a real-eval-print loop of a
  mathematical software system and the latter make semantically annotated documents
  semantic. 

  We report on two systems in the \pn project: Jupyter -- a notebook server for various
  kernels, and MathHub.info -- a platform for active mathematical documents. We identify
  commonalities and differences and develop a vision for integrating their
  functionalitities.
\end{abstract}

%%% Local Variables:
%%% mode: latex
%%% TeX-master: "report"
%%% End:
