\section{Information Architecture}\label{sec:infarch}

For the \pn project we have developed a general architecture for in-situ computation and
tested it on the use cases \ref{sec:ex:units} to \ref{sec:ex:hyp} above, which involve
computation with math objects -- aka. ``mathematical formlae''\footnote{The other use
  cases are possible in this architecture. Use case~\ref{sec:ex:docstruct} has been
  realized before -- on a similar information architecture but different implemenation --
  and is currently being re-developed for the current version of the \mmt API.} in active
documents.

Active Documents are represented in the \omdoc/\mmt
format~\cite{Kohlhase:OMDoc1.2,uniformal:on,Iancu:phd} which combines document- and
content strutures and thus allows to formalize both sides of active documents -- the lower
part in Figure~\ref{fig:activedocs}; see~\cite{ODK-D4.2} for an overview and comparison to
other \pn systems. The active document player is implemented in the MathHub
system~\cite{MathHub:on} -- see~\cite{ODK-D4.3} for an introduction in the context of the
\pn project, which relies on the \mmt API~\cite{Rabe:MAGMS13,uniformal:on} for
content/knowledge management services. In particular, the \mmt API can communicate with
many of the \pn systems via the SCSCP remote procedure call
standard~\cite{HHKLRAT:SCSCP10}; see~\cite{ODK-D3.3} for an account of the exact state of
affairs. We will use this facility for the actual computations.

To understand the realization of in-situ computation in active documents, we will first
look at the information resources involved and only in the next section give go into
concrete implementation details. The latter may change over the course of the \pn project
while the former will only be refined with more experience.

Active documents have \textbf{math objects} encoded in
MathML~\cite{CarlisleEd:MathML3:base}: In \omdoc/\mmt-based active documents documents
contain presentation MathMl and are crosslinked -- parallel markup -- to content MathML
representations in he content commons; see Figure~\ref{fig:activedocs}. In any case, the
objects which contain identifiers -- e.g. $E$, $m$, and $c$ in (\ref{f:emc2}), which are
introduced (\textbf{declared}) in the text or taken for granted because they have
definitions in the content commons that is assumed to be known by the reader;
see~\cite{WolGriKoh:udc11} for an experimental account of the linguistic side. These
declarations and definitions induce a context that \emph{gives meaning to the math
  objects} by explaining or binding these identifiers. For in-situ computation this
context -- we call it the \textbf{computation context} -- must be identified and passed to
the compute engine.

Our computation context is similar in spirit to the document and user contexts Rikko
Verrijzer identifies in~\cite{Verrijzer:cimdpm15} for MathDox educational
documents~\cite{mathdox:on}. His notions focus mainly on user modeling whereas our's are
techically more generic , since our active documents have more elaborated document and
content structures.

To get a feeling for the situation, consider Figure~\ref{fig:emc-adp} which instantiates
the abstract diagram from Figure~\ref{fig:activedocs} to the situation in our running
example: Einstein's energy/mass equivalence. We have the two parts: the document commons
with (a slightly rephrased fragment of) the document in Figure~\ref{fig:emc2-wikipedia}
on the left upper corner and the content comment on the other side of the dashed line. 

The latter is encoded as an OMDoc/MMT theory graph -- see~\cite{RabKoh:WSMSML13} for
technical details and~\cite{DehKohKon:iop16,ODK-D6.2} for an account of the applications
in the \pn project. All relevant concepts are grouped in named theories (the boxes with
rounded corners), which introduce symbols and their properties e.g. the definition of the
unit Joule in \textsf{energy} or the size of the speed of light in
\textsf{lightspeed}. These theories are connected by theory morphisms -- only inclusions
$S\mmtar{include}T$ which make all declarations (symbols and properties) of $S$ visible in
$T$ -- and give an object-oriented, modular regime of formalizing mathematical knowledte.

\begin{figure}\centering
  {\footnotesize\documentclass{standalone}
\usepackage{amsfonts}
\usepackage{tikz}
\usetikzlibrary{tikzmark}
\usetikzlibrary{mmt}
\usetikzlibrary{docicon}
\begin{document}
\def\cn#1{\mathsf{#1}}
\def\bT{\mathbb{T}}
\def\bL{\mathbb{L}}
\def\bM{\mathbb{M}}
\def\bE{\mathbb{E}}
\begin{tikzpicture}[xscale=2,yscale=2,remember picture]
  \tikzstyle{doc}=[draw,color=black,shape=document]

  \node[thy] (m) at (0,0) {$\mmtthy{mass}{\subnode{cma}{\ensuremath\bM}:\cn{type},g:\bM,kg:\bM,\ldots}{}$}; 
  \node[thy] (l) at (2,0) {$\mmtthy{length}{\bL:\cn{type},m:\bL,\ldots}{}$};
  \node[thy] (t) at (4,0) {$\mmtthy{time}{\bT:\cn{type}, s:\bT,h:\bT,\ldots}{}$}; 
  \node[thy] (e) at (2,1) {$\mmtthy{energy}{\subnode{cen}{\ensuremath\bE}:\cn{type},J:\bE,\ldots}{1J=1\frac{kg\cdot m^2}{s^2}}$};
  \node[thy] (c) at (4,1) {$\mmtthy{lightspeed}{\subnode{clv}{\ensuremath{c}}:\bL/\bT}{1c= 299 792 458\frac{m}{s^2}}$}; 
  \node[thy] (emc2) at (4,2) {$\mmtthy{EME}{E:\bE,m:\bM}{E=mc^2}$}; 
  \draw[include] (t) -- (e);
  \draw[include] (l) -- (e);
  \draw[include] (m) -- (e);
  \draw[include] (t) -- (c); 
  \draw[include] (l) -- (c); 
  \draw[include] (c) -- (emc2); 
  \draw[include] (e) -- (emc2); 

  \node[doc,thick] (snip) at (-.3,1.8) 
  {\begin{minipage}{3.8cm}
      \[E=mc^2\] where \fbox{$E$ is the \subnode{den}{Energy}}, \fbox{$m$ is the \subnode{dma}{Mass}}, and
      \fbox{\subnode{cls}{$c$} the \subnode{dlv}{speed of light}}.
    \end{minipage}};
  \draw[dashed,->] (den) -- (cen);
  \draw[dashed,->] (dma) -- (cma); 
  \draw[dashed,->] (dlv) to[bend left=25] (clv); 
\end{tikzpicture}
\end{document}

%%% Local Variables:
%%% mode: latex
%%% TeX-master: t
%%% End:
}
  \caption{$E=mc^2$ as an Active Document}\label{fig:emc-adp}
\end{figure}

Documents are marked up in terms of its document, statement, and phrase structure in the
ADP. In particular, we mark up
\begin{enumerate}
\item the sectioning structure -- omitted in our running example, 
\item statements -- the assertion for $E=mc^2$ coincides with the whole document
  \dociconprototype{D} in Figure~\ref{fig:emc-adp}, and
\item the phrase structure -- here declarations are shown as boxes and technical terms as
  dashed boxes.
\end{enumerate}
Finally, the marked up structures in the document commons are cross linked to the content
commons to create \textbf{parallel markup}\footnote{The idea of ``parallel markup'' has
  been pioneered by the MathML format~\cite{CarlisleEd:MathML3:base}, which uses it to
  connect equivalent sub-formulae in presentation and content Markup, and~\cite{Iancu:phd}
  generalizes it to all levels in the OMDoc/MMT setting.} at all levels. We see three
dashed arrows: two connect the technical terms ``Energy'' and ``Mass'' in the dashed boxes
to the corresponding concepts in the content tree and one that connects the whole
declaration ``$c$ is the speed of light'' to the corresponding declaration in the theory
\textsf{ligthspeed}. The dotted arrow on the top of Figure~\ref{fig:emc-adp} represents
still another parallel alignment relation, it is the ``home theory'' relation, which makes
all concepts from a theory -- the \textbf{home theory}; here \textsf{EME} -- in a document
fragment. All parallel markup relations must be refinements of this relation to be
well-justified in OMDoc/MMT.

The crucial observation is that together, the home theory relation, its refinements, and
the document markup give a notion of context for the computation. The (required)
declarations for the local identifiers $E$ and $m$ in \dociconprototype{D}, and the
(optional) declaration for the identifier $c$ inherited form the theory
\textsf{lightspeed} give meaning to $E=mc^2$ and also determine what can be
computed. Instantiating the \emph{variables} (locally declared identifiers) $m$ and $E$
give rise to the computations in Section~\ref{sec:equations}, whereas the ``replacing''
the \emph{constant} (an identifier inherited from a theory) $c$ with a different value the
hypothetical calculations from section~\ref{sec:hyp}. We can even predict the grade of
hypotheticality by the inheritance distance in the content theory graph.

Formally, the \textbf{computation context} of a formula comes in the form of
\textbf{declarations}, i.e. triples of the form $c:\tau=\delta$, where $c$ is the name of
the declared identifier, (optionally) $\tau$ its type $\tau$, and (again optionally) a
definiens $\delta$. In our running example, the context of $E=mc^2$ consists of three
declarations: $E:\bbE$, $m:\bbM$, and $c:\bbL/\bbT=299 792 458$ (leaving out units for the
moment). It can be inferred from the information in Figure~\ref{fig:emc-adp}; but let us
make the active document fragment more explicit. 

Listing~\ref{lst:emc-adfragment}\ednote{MK: I am sure that MMT generates something
  slightly similar, update when we know} shows the HTML5 representation for the active
document fragment in Figure~\ref{fig:emc-adp}. It shows the (presentation) MathML
representation of $E=mc^2$, followed by the text part in which the declarations are marked
up with \lstinline|span| elements of class \lstinline|o_declaration| for the variable
declarations and \lstinline|o_symbol| for the symbol declaration. These carry the
OMDoc/MMT attributes \lstinline|o:for| and \lstinline|o:scope|. The former points to the
identifier being declared, and the latter points at the element in which this declaration
is active. In this case, since the scope is the other \lstinline|<p>| element the
declarations govern the identifiers in the displayed equation $E=mc^2$. 

\begin{lstlisting}[label=lst:emc-adfragment,caption=Native Markup for an Active Document Fragment, 
language=HTML,basicstyle=\footnotesize\sf,mathescape,morekeywords={mi,mo,msup,math,mrow}]
<p id="p" xmlns:o="http://omdoc.org/ns">
 <math display="block" xref="http://mathhub.info/phys/famous?equations?eme">
  <mi>E</mi>
  <mo>=</mo>
  <mrow>
   <mi>m</mi>
   <mo>&invisibletimes;</mo>
   <msup><mi>c</mi><mn>2</mn></msup>
  </mrow>
 <math>
 where 
 <span class="o_declaration" o:for="#E" o:scope="#p">
  <math><mi id="E">E</mi></math> is the 
  <span class="term" xref="http://mathhub.info/units/SI/complex?energy?$\bbE$">energy</span>
 </span>,
 <span class="o_declaration" o:for="#m" o:scope="#p">
  <math><mi id="m">m</mi></math> is the 
  <span class="term" xref="http://mathhub.info/units/SI/dimensions?mass?$\bbM$">mass</span>
 </span>, and 
 <span class="o_symbol" o:for="#c" o:scope="#p">
  <math><mi id="c" xref="http://mathhub.info/phys/const/basic?lightspeed?c">c</mi></math> the 
  <span class="term" xref="http://mathhub.info/phys/const/basic?lightspeed?lightspeed">speed of light</span>
 </span>.
</p>
\end{lstlisting}

\begin{wrapfigure}r{7.5cm}\vspace*{-1em}
\begin{lstlisting}[language=XML,morekeywords={constant,type,apply,math,ci,cn,csymbol},basicstyle=\footnotesize\sf]
<constant name="eme">
 <type>
  <math>
   <apply><eq/>
    <ci>E</ci>
    <apply><times/>
     <ci>m</ci>
     <apply><power/>
      <csymbol cd="lightspeed">c</csymbol>
      <cn>2</cn>
     </apply>
    </apply>
   </apply>
  </math>     
 </type>
</constant>
\end{lstlisting}
\caption{$E=mc^2$ in Content MathML}\label{fig:emc-content}\vspace*{-2em}
\end{wrapfigure}
Now let us have a look at how this enables computation: The displayed equation $E=mc^2$
linked into the content commons via its \lstinline|xref| attribute, which points to the
(constant) declaration in Figure~\ref{fig:emc-content}. The \lstinline|constant| element
combines the system name \lstinline|eme| with the statement of the equation in the
\lstinline|type| element\footnote{This is a consequence of using the Curry/Howard
  isomorphism at work; we have elided the details of the type here.}. As the equation is
represented as the ``operator tree'' in content MathML, it is fully disambiguated
functionally and can therefore directly be computed with in a computational engine
(e.g. the MMT system itself or a computer algebra system like GAP or SageMath) after
instantiation of the variables with concrete values.


\subsection{Information Architecture for Unit Conversion}
For the automatic conversion of units, we assume that the document contains formulas
in MathML with cross references between its presentational part and its content MathML. 
The content MathML needs to be annotated with the semantics of the expression, which
includes the information about the present units. 


%%% Local Variables:
%%% mode: latex
%%% TeX-master: "report"
%%% End:

%  LocalWords:  sec:infarch oldpart Verrijzer:cimdpm15 mathdox:on fig:emc-adp textsf emph
%  LocalWords:  fig:activedocs fig:emc2-wikipedia RabKoh:WSMSML13 mmtar formalizing mrow
%  LocalWords:  DehKohKon:iop16,ODK-D6.2 centering emc-adp dociconprototype textbf msup
%  LocalWords:  Iancu:phd ligthspeed sec:equations sec:hyp hypotheticality lstlisting
%  LocalWords:  sf,mathescape,morekeywords mi,mo,msup,math,mrow invisibletimes msup o:for
%  LocalWords:  xref lst:emc-adfragment lstinline lstinline lst:emc-adfragment,caption
%  LocalWords:  xmlns:o wrapfigure vspace constant,type,apply,math,ci,csymbol csymbol
%  LocalWords:  csymbol fig:emc-content
