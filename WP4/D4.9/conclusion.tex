\section{Conclusion and Future Work}\label{sec:concl}
 
We have presented a general framework for in-situ computation in active documents. This is
a contribution towards using mathematical documents -- the traditional form mathematicians
interact with mathematical knowledge and computations -- as a user interface for a
mathematical virtual research environments. This is also a step towards integrating the
two main UI frameworks under investigation in the \pn project: Jupyter notebooks and
active documents -- see~\cite{ODK-D4.2} -- at a conceptual level. The system is
prototypical at the moment, but can already be embedded into active documents via a
Javascript framework and is ready for use in the \pn project. The user interface and SCSCP
connections are quite fresh and need substantial testing and optimizations.

In the current state of the system , we have concentrated on in-situ computation with
MathML formulae, which covers the first three use cases from Section~\ref{sec:examples},
the main problem with extending this method to the to the other ones is flexiformalizing
the data and knowledge at the document and structure levels and implementing the narration
composition engines. \ednote{integration with Jupyter (as an interactive computation
  shell. This mens that Jupyter needs to be made aware of the computation context.)}

\subsection*{Acknowledgements}
This report has profited from discussions with Dennis M\"uller and Ulrich Rabenstein from
the KWARC group in Erlangen. Many of the underlying concepts have evolved in discussion of
the first author with Dr. Mihnea Iancu.

%%% Local Variables:
%%% mode: latex
%%% TeX-master: "report"
%%% End:

%  LocalWords:  sec:concl Mihnea
