\section*{\texorpdfstring{Deliverable description, as taken from Github
issue
\href{https://github.com/OpenDreamKit/OpenDreamKit/issues/94}{\#94} on
2017-02-28}{Deliverable description, as taken from Github issue \#94 on 2017-02-28}}\label{deliverable-description-as-taken-from-github-issue-94-on-2017-02-28}

\begin{itemize}
\tightlist
\item
  \textbf{WP4:}
  \href{https://github.com/OpenDreamKit/OpenDreamKit/tree/master/WP4}{User
  Interfaces}
\item
  \textbf{Lead Institution:} Université Paris-Sud
\item
  \textbf{Due:} 2016-08-31 (month 12)
\item
  \textbf{Nature:} Demonstrator
\item
  \textbf{Tasks:} T4.1
  (\href{https://github.com/OpenDreamKit/OpenDreamKit/issues/69}{\#69}):
  Uniform notebook interface for all interactive components, T4.6
  (\href{https://github.com/OpenDreamKit/OpenDreamKit/issues/74}{\#74})
  Structured documents
\item
  \textbf{Proposal:}
  \href{https://github.com/OpenDreamKit/OpenDreamKit/raw/master/Proposal/proposal-www.pdf}{p.~48}
\item
  \textbf{\href{https://github.com/OpenDreamKit/OpenDreamKit/raw/master/WP4/D4.5/report-final.pdf}{Final
  report}}
\end{itemize}

The \href{https://jupyter.org}{Jupyter Notebook} is a web application
that enables the creation and sharing of executable documents which
contain live code, equations, visualizations and explanatory text.
Thanks to a modular design, Jupyter can be used with any computational
system that provides a so-called
\href{https://jupyter.readthedocs.io/en/latest/projects/kernels.html}{\emph{Jupyter
kernel}} implementing the
\href{https://jupyter-client.readthedocs.io/en/latest/}{\emph{Jupyter
messaging protocol}} to communicate with the notebook. OpenDreamKit
therefore promotes the Jupyter notebook as user interface of choice, in
particular since it is particularly suitable for building modular web
based Virtual Research Environments.

A notebook interface is such a vital integrative component that,
predating Jupyter, the open source mathematical system SageMath had
developed as early as 2005 its own solution, which we refer to here as
the legacy SageMath notebook. Development was fast tracked to ensure its
availability to allow the project to move forward, and it served as
major source of inspiration for Jupyter. Meanwhile, moving at a fast
pace thanks to its much larger community, Jupyter had, by 2014,
basically caught up with the legacy SageMath notebook in terms of
functionality.

Building on top of D4.4
(\href{https://github.com/OpenDreamKit/OpenDreamKit/issues/93}{\#93}):
Basic Jupyter interface for GAP, PARI/GP, SageMath, Singular, the goal
of this deliverable was therefore to help phase out the legacy SageMath
notebook in favour of Jupyter. Outsourcing this key but non disciplinary
component is an important step toward the sustainability of ODK's
ecosystem (Objective 5).

Since 2014, a lot of work was put into this by the SageMath community,
and in particular by Volker Braun. Recently, this work has been
continued thanks to ODK. It included two aspects: ensuring that Jupyter
included all important features of the legacy SageMath notebook, and
enabling a smooth migration path for users:

\begin{itemize}
\tightlist
\item
  \(\checkmark\) \href{http://trac.sagemath.org/ticket/20690}{\#20690},
  \href{http://trac.sagemath.org/ticket/22458}{\#22458}: Live
  documentation @fcayre, @nthiery, @videlec
\item
  {[} {]} Interacts/widgets: full support and backward compatibility
  with the legacy SageNB
  \href{https://trac.sagemath.org/ticket/21267}{\#21267} (closed
  installation issues:
  \href{https://trac.sagemath.org/ticket/21260}{\#21260}
  \href{https://trac.sagemath.org/ticket/21261}{\#21261}
  \href{https://trac.sagemath.org/ticket/20218}{\#20218}
  \href{https://trac.sagemath.org/ticket/21256}{\#21256}) @jdemeyer
\item
  \(\checkmark\) \href{https://trac.sagemath.org/ticket/19877}{\#19877}:
  Conversion of legacy notebooks to Jupyter notebooks @vbraun, @videlec,
  @marcinofulus
\item
  \(\checkmark\) \href{https://trac.sagemath.org/ticket/19740}{\#19740}:
  Migration wizard, as a web application
\end{itemize}

Future work:

\begin{itemize}
\tightlist
\item
  {[} {]} WYSIWYG editor for the markdown cells, such as
  \href{https://www.tinymce.com/}{TinyMCE} (see
  \href{https://groups.google.com/d/msg/sage-devel/t11JSxxCgpw/BR0Bt638AgAJ}{this
  post} for motivation)
\item
  {[} {]} {[}\#20316{]}(\url{https://trac.sagemath.org/ticket/20316}):
  Add button to export SageNB notebooks to Jupyter
\end{itemize}
