\documentclass{deliverablereport}
\usepackage{pdflscape}

\deliverable{UI}{ipython-kernel-sage}
\deliverydate{28/02/2017}
\duedate{01/09/2016 (M12)}
\author{Florent Cayr\'e, Jeroen Demeyer, Nicolas M. Thi\'ery}

\begin{document}
\maketitle
\githubissuedescription
\tableofcontents

\section{Highlight of some of the main features implemented by ODK}

An important use case of the notebook is \emph{interactive functions}:
these allow the user to control the input of a function
using \emph{widgets} such as sliders and text boxes.
See Figure~\ref{fig-interact-sagenb} for an example.
This way, the user can easily investigate how some function changes
when the input changes.
Once the interact is created, it can be used by people having no experience
at all with Sage or Python.
Interacts have been implemented independently in the Sage notebook
and in Jupyter (package \texttt{ipywidgets}).
For the conversion of Sage notebooks to Jupyter to be useful,
also interacts should work the same way.
This has been implemented in the Jupyter kernel for Sage:
see Figure~\ref{fig-interact-jupyter} for the same example as
Figure~\ref{fig-interact-sagenb}, this time in Jupyter.

A handy feature of the legacy \Sage notebook is \emph{live documentation}.
In the Sage notebook, the documentation of Sage is ``live'':
the examples from the documentation become editable notebook cells.
This way, the user can run those examples and experiment with them.
This has now been implemented also in the Jupyter notebook using
the Thebe package from O'Reilly Media.
Thebe allows embedding Jupyter notebook cells in arbitrary webpages
and this has been used to turn the Sage documentation live when viewed
through the Jupyter notebook.
See Figure~\ref{fig-live-doc} for an example.

The default notebook application in \Sage is now
\texttt{sagenb\_export}.  This small web application, built on top of
Jupyter, acts as migration wizard which can convert legacy \Sage
notebooks to Jupyter. It also has buttons to run either the legacy
\Sage notebook server or the new Jupyter notebook server. See
Figure~\ref{fig-export}.

\clearpage
\appendix
\begin{landscape}
\section{Screenshots}

\newcommand{\screenshot}[2]{
\begin{figure}[ht]
  \includegraphics[width=1.4\textwidth,trim={0 0 1cm 1px},clip]{#1}
  \caption{#2}
\end{figure}}

\screenshot{interact-sagenb.png}{\label{fig-interact-sagenb}
  An example of an interact in the legacy Sage notebook.}
\screenshot{interact-jupyter.png}{\label{fig-interact-jupyter}
  The same example interact, converted to Jupyter using \texttt{sagenb\_export}.
  Note the identical functionality, despite a different visual interface.}
\screenshot{live-doc.png}{\label{fig-live-doc}
  Live documentation in the Jupyter notebook:
  the user can edit the examples in the documentation and run them
  using the ``Run Again'' buttons.}
\screenshot{export.png}{\label{fig-export}
  The \texttt{sagenb\_export} application, showing
  buttons to run the Sage or Jupyter notebook and a list of Sage notebooks
  to convert to Jupyter.}
\end{landscape}
\clearpage
\end{document}

%%% Local Variables:
%%% mode: latex
%%% TeX-master: t
%%% End:
