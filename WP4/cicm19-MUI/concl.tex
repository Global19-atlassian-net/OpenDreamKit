
We have presented an integration of two interaction paradigms in mathematical software systems: document-based and computation-oriented interactions. Concretely, we have implemented an integration of three systems:
Jupyter for computation/experimentation in notebooks and MathHub for interactive mathematical documents as well as MMT for describing the semantics of the knowledge contained in the former.
We have evaluated the reach of the evaluation in several case studies.

\begin{oldpart}{TW: re-do this completly}
 The resulting system serves as an intermediate step towards a full Virtual Research Environment as envisioned in the OpenDreamKit project. While this report focuses on the integration of documents and notebooks in a unified system, other work in the project has built other components of the envisioned environment: the Math-in-the-Middle ontology \cite{KohMuePfe:kbimss17}\ednote{maybe cite Dennis' system/resources paper on MitM if it gets written} and the instantiation of the MitM framework with concrete mathematical computation and database systems ~\cite{WieKohRab:vtuimkb17}.
Future work will focus on merging and scaling up these development strands.
\end{oldpart}

%%% Local Variables:
%%% mode: latex
%%% mode: visual-line
%%% fill-column: 5000
%%% TeX-master: "paper"
%%% End:

%  LocalWords:  ednote Jupyter Jupyter MitM-based completly KohMuePfe:kbimss17 WieKohRab:vtuimkb17
