
We have presented an integration of two interaction paradigms in mathematical software systems: document-based and computation-oriented interactions. Concretely, we have implemented an integration of three systems:
Jupyter for computation/experimentation in notebooks and MathHub for interactive mathematical documents as well as MMT for describing the semantics of the knowledge contained in the former.
We have evaluated the reach of the evaluation in several case studies.

% \begin{oldpart}{TW: re-do this completly}
%  The resulting system serves as an intermediate step towards a full Virtual Research Environment as envisioned in the OpenDreamKit project. While this report focuses on the integration of documents and notebooks in a unified system, other work in the project has built other components of the envisioned environment: the Math-in-the-Middle ontology \cite{KohMuePfe:kbimss17}\ednote{maybe cite Dennis' system/resources paper on MitM if it gets written} and the instantiation of the MitM framework with concrete mathematical computation and database systems ~\cite{WieKohRab:vtuimkb17}.
% Future work will focus on merging and scaling up these development strands.
% \end{oldpart}

Even though the work presented in this paper lays the  foundation towards an integration of the static/dynamic paradigms for the interaction with mathematical knowledge, at lot of practical enhancements remain for future work.
We sketch the most important ones here:

\paragraph{Deeper MathHub/Jupyter Integration} e.g., using Jupyter simply as a JavaScript library in MathHub.
This would have been preferable to the current iFrame-based integration, but is infeasible because Jupyter is primarily designed as a monolithic system.
Recent versions of Jupyter are working towards a Jupyter-as-a-module design, so we leave deep integration to future work.

\paragraph{IDE Support for Documents with Active Computation}
Currently, the semantic documents like the one in Figure~\ref{fig:conversionHTML} have to be manually extended by the pertinent semantic annotations. An extension of the \sTeX framework would allow authors e.g., of educational documents to directly manage the annotations in the {\LaTeX} sources. 

\paragraph{REPL Cells/Documents as first-class citizens in MMT}
We already use the notebook-to-MMT-document isomorphism in our system. A first-class model of REPL cells in MMT -- this will need a considerable language design effort -- would allow to strengthen this isomorphism and refactor our system.
We expect that first-class REPL cells in MMT would allow enhanced IDE support for MMT notebooks.  

\paragraph{More Flexible Active Computation} The current widget is relatively inflexible in terms of the objects it allows to change for computation.
In principle, all variables and constants from the context could be used.
We will need more user experience to generalize our current design.

\paragraph{TGView/Notebook Integration} The MMT Jupyter kernel is fundamentally co-dependent on the background theory graph. Therefore we want to explore an integration of the TGView graph viewers~\cite{MarKohRab:tsddvtg19} into the Jupyter front-end.
Both Jupyter and TGView are based on REACT.JS, so this should be feasible

\paragraph{Mathematical Search on Notebooks} Last, but not least, we want to extend mathematical search on MathHub to Jupyter notebooks by extending the MathWebSearch harvester accordingly.

%%% Local Variables:
%%% mode: latex
%%% mode: visual-line
%%% fill-column: 5000
%%% TeX-master: "paper"
%%% End:

%  LocalWords:  ednote Jupyter Jupyter MitM-based completly KohMuePfe:kbimss17 WieKohRab:vtuimkb17 oldpart MarKohRab:tsddvtg19
