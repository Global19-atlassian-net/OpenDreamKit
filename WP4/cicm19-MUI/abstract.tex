Mathematical software systems offer two major paradigms for interacting with mathematical knowledge.
One is to write static files with semantically annotated representations that define mathematical knowledge and can be compiled into documents (pdf, html, etc.), and the other is to dynamically build mathematical objects in interactive read-eval-print loops (REPL) such as notebooks.
Many author-facing interfaces offer both features in some way.
However, reader-facing interfaces usually show only one or the other.

In this paper we present an integration of the approaches in the context of the MMT system.
Firstly, we present a Jupyter kernel for MMT which provides web-ready REPL functionality for MMT.
Secondly, we integrate the resulting Jupyter notebooks into MathHub, a web-based frontend for mathematical documents.
This allows users to context-sensitively open a Jupyter notebook as a dynamic subdocument anywhere inside a static MathHub document.
Vice versa, any such highly interactive and often ephemeral notebook can be saved persistently in the MathHub backend at which point it becomes available as a static document.
We also show how Jupyter widgets can be deeply integrated with the MMT knowledge management facilities to give semantics-aware interaction facilities.

%%% Local Variables:
%%% mode: latex
%%% mode: visual-line
%%% fill-column: 5000
%%% TeX-master: "paper"
%%% End:

%  LocalWords:  Jupyter MitM-based textbf textbf textbf
