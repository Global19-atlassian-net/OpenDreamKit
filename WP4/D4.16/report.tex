\documentclass{deliverablereport}

\deliverable{UI}{ipython-advanced-interacts}
\deliverydate{31/08/2018}
\duedate{31/08/2018 (M36)}
\author{Odile Benassy and Nicolas M. Thiéry}

\begin{document}
\maketitle
% This will be the abstract, fetched from the github description
\githubissuedescription

\section{Introduction}

The \href{https://jupyter.org}{Jupyter Notebook} is a web application
that enables the creation and sharing of executable documents
containing live code, equations, visualizations and explanatory text.
Reaching far beyond the standard
\href{https://en.wikipedia.org/wiki/Read-eval-print_loop}{REPL}
interaction (Read-Eval-Print Loop), a key feature of Jupyter are its
\href{http://jupyter.org/widgets}{Interactive widgets} which enable
real time interactive data visualizations; the Jupyter community has
developed a large array of widgets for interactive 2D and 3D
visualization of data in the form of charts, maps, tables, etc; see
e.g. \delivref{UI}{vis3d} for ODK's contribution to 3D visualization.
Furthermore, widgets can be \emph{composed} to build rich
applications, with all the usual UI components (e.g. menus, sliders,
or layout control).

Hence, the Jupyter stack provides a very flexible environment catering
for use cases ranging from a novice users typing just a few commands
or browsing interactive documents to more advanced users authoring
rich interactive applications for their fellows.

The question we are tackling in this report is how this technology, --
and specifically widgets -- can be leveraged for pure mathematics. The
unique challenge comes from the huge variety of mathematical objects
that the user may want to visualize and manipulate, some even coming
with several natural ways to be represented.

\TODO{search on wikipedia for a nice collection of pictures: a
  partition, a polyomino, an aztec diamond, a monomial ideal, a graph,
  a poset, a formula, a curve, ...}.

We therefore can't hope to provide hand crafted solutions for each
situation; instead we need to devise a toolbox of generic solutions
from which users can easily derive specialized visualizations for
their own pet objects.

We pursue two directions.

In the first one, we explore the development of single object
visualization widgets. As area, we chose combinatorics, to some extent
out of personal interest and expertise, but more importantly because
devising good representations -- mental images -- of discrete objects
is at the heart of research in combinatorics. In section~\ref{grid},
we report on the implementation in SageMath of a generic widget for
objects that admit a representation as a collection of cells on a 2D
grid, and specializations for typical objects such as partitions,
tableaux, aztec diamonds, mazes.

\TODO{screenshot(s) of widgets displaying all of the above}





Another natural use case in combinatorics -- or more generally
discrete maths -- are objects that admit a graph-like representation
(trees, graphs, lattices of subgroups, crystals, posets, discrete
markov chains, to name a few). This use case is being explored by the
authors of the GAP package
\href{https://github.com/mcmartins/francy}{francy} under the
supervision(?) of ODK's member Markus Pfeiffer. Although ODK's
contribution in this direction is minimal at this stage, we briefly
report on this use case to put things in perspective, and also because


\section{State of the art}



\section{A generic widget for objects with a 2D boxed representation}
\label{grid}

This use case combined several advantages:
\begin{itemize}
\item The UI part was relatively straightforward, for example
  requiring no Javascript side extension;
\item Little mathematical background was required; this, together with
  the previous point, made for a smooth learning curve for our
  Research Software Engineer;
\item It was a low hanging fruit with a large coverage;
\item The end result is immediately useful to colleagues, enabling
  early feedback from users;
\item There is a large variety of potential specializations, each with
  it's own quirks and specifics.
\end{itemize}
Hence, this use case provided a unique challenge: all the difficulty
resided in the the design of a generic solution that encapsulates as
mush of the technicalities as possible, enabling users to specialize
the generic solution for their own pet objects with little expertise.


\TODO{Design}




The code is distributed as a Sage package
\href{https://github.com/sagemath/sage-combinat-widgets/}{sage-combinat-widgets}
which is meant to develop beyond this initial seed, in particular by
attracting external developers, and presumably be integrated
progressively into Sage.


\section{francy: an Interactive Discrete Mathematics Framework for GAP}



\section{Sage-explorer}
\label{sage-explorer}

\end{document}

%%% Local Variables:
%%% mode: latex
%%% TeX-master: t
%%% End:

