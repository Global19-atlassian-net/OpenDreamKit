\section*{\texorpdfstring{Deliverable description, as taken from Github
issue
\href{https://github.com/OpenDreamKit/OpenDreamKit/issues/90}{\#90} on
2018-09-01}{Deliverable description, as taken from Github issue \#90 on 2018-09-01}}

\begin{itemize}
\tightlist
\item
  \textbf{WP4:}
  \href{https://github.com/OpenDreamKit/OpenDreamKit/tree/master/WP4}{User
  Interfaces}
\item
  \textbf{Lead Institution:} Université Paris-Sud
\item
  \textbf{Due:} 2018-08-31 (month 36)
\item
  \textbf{Nature:} Demonstrator
\item
  \textbf{Task:} T4.5
  (\href{https://github.com/OpenDreamKit/OpenDreamKit/issues/73}{\#73})
\item
  \textbf{Proposal}:
  \href{https://github.com/OpenDreamKit/OpenDreamKit/raw/master/Proposal/proposal-www.pdf}{p.
  48}
\item
  \textbf{\href{https://github.com/OpenDreamKit/OpenDreamKit/blob/master/WP4/D4.16/report-final.pdf}{Final
  report}}
\end{itemize}

The \href{https://jupyter.org}{Jupyter Notebook} is a web application
that enables the creation and sharing of executable documents containing
live code, equations, visualizations, and explanatory text. Key features
include Jupyter widgets which enable interactive visualization and can
be composed to create full-featured interactive applications.

In this report, we explore the potential of Jupyter widgets for (pure)
mathematics. The unique challenge comes from the huge variety of
mathematical objects that the user may want to visualize and interact
with, and the variety of graphical representations.

Over the course of four PM's, we pursued two main directions: the
interactive visualization and edition of combinatorial objects
(\href{https://github.com/sagemath/sage-combinat-widgets}{sage-combinat-widgets})
and the interactive exploration of Sage features based on a display of
mathematical objects enriched with contextual semantic information
(\href{https://github.com/sagemath/sage-explorer}{sage-combinat-explorer}).
