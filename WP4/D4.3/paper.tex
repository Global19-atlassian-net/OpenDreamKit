\documentclass{llncs}
\usepackage[show]{ed}
\usepackage{stex-logo}
%\usepackage{calbf}
%\usepackage{amstext,amssymb}
\usepackage{xspace}
\usepackage{tikz}
\usetikzlibrary{positioning,mmt,shapes.geometric}

\usepackage{mdframed}
\newenvironment{boxedquote}{\begin{mdframed}[leftmargin=1cm,rightmargin=1cm]}{\end{mdframed}}

\usepackage{wrapfig,paralist}
\usepackage[hyperref,style=alphabetic,backend=bibtex]{biblatex}
% \bibliography{paper-bib}
\addbibresource{kwarcpubs.bib}
\addbibresource{extpubs.bib}
\addbibresource{kwarccrossrefs.bib}
\addbibresource{extcrossrefs.bib}
\usepackage{bibtweaks}

\pagestyle{plain}

\def\defemph#1{\textbf{#1}}
\def\defeq{:=}
\def\omdoc{\textsf{OMDoc}\xspace}
\def\mmt{\textsf{MMT}\xspace}
\def\sys{\textsf{MathHub.info}\xspace}
\usepackage{hyperref}

\title{System Deascription: \sys}
\author{Mihnea  Iancu, Constantin Jucovschi, Michael  Kohlhase, Tom Wiesing}
\institute{
  Computer Science, Jacobs University Bremen, \url{http://kwarc.info}
}

\begin{document}
\maketitle
\begin{abstract}
  We present the \sys system, a development environment for active mathematical documents
  and an archive for flexiformal mathematics. It offers a rich interface for reading,
  writing, and interacting with mathematical documents and knowledge. The core of the \sys
  system is an archive for flexiformal mathematical documents and libraries in the
  \omdoc/\mmt format. Content can be authored or archived in the source format of the
  respective system, is versioned in GIT repositories, and transformed into \omdoc/\mmt
  for machine-support and further into HTML5 for reading and interaction.
\end{abstract}

\section{Introduction}\label{sec:intro}

As the field of Mathematical Knowledge Management (MKM) and Digital Mathematical Libraries
(DML) mature, we need to shift attention from experimenting with semantic services on
small and practice data sets to supporting management of large corpora of mathematical
knowledge and documents. In the past, MKM and DML have latched onto existing
corpora/libraries ranging from digitized mathematical articles (e.g.
EuDML~\cite{EuDML:on}) over semi-structured representations generated from {\LaTeX}
sources~\cite{StaKoh:tlcspx10} to fully formal theorem prover libraries, e.g. the Mizar
Mathematical Library~\cite{MizarKB:on}.  But so far management support for and semantic
services deployed on such libraries have been essentially insular, existing cross-library
methods remain experiments, and have not been implemented into usable systems.

This insularity also applies to the work in the KWARC group. For instance, we have
\begin{inparaenum}[\em i\rm)]
\item designed a cross-library representation language: \omdoc (\underline{O}pen
  \underline{M}athematical \underline{Doc}uments),~\cite{Kohlhase:OMDoc1.2},
\item developed a meta-logical framework \mmt (A \underline{M}odule System for
  \underline{M}athematical \underline{T}heories)~\cite{RabKoh:WSMSML13,CodHorKoh:palai11})
  that allows to represent the logical languages underlying the theorem prover libraries,
  and relate them to each other by logic morphisms,
\item built libraries of formalized mathematics in \omdoc/\mmt either by manually creating content (e.g. LATIN~\cite{CodHorKoh:palai11}) or by implementing transformations from existing theorem prover libraries (e.g. Mizar Mathematical Library~\cite{IanKohRabUrb:tmmliotaa13}), and 
\item have used \omdoc as the basis for active mathematical documents in the Planetary
  system~\cite{Kohlhase:ppte12}.
\end{inparaenum}
But we have not integrated all of these or made them available to mathematicians in one
comprehensive environment.

To change this situation -- and to provide a realistic base for our own cross-library
research and development efforts -- we have started work to realize a universal archiving
solution for formal and informal mathematical corpora/libraries.  

We present the \sys system and its design goals in this paper. \sys must satisfy two
conflicting goals: On the one hand, it must be so generic that it is open to all logics
and implementations; on the other hand, it must be aware of the semantics of the
formalized content so that it can offer meaningful services. These services must be
independent of both the formal system and the implementation used to produce the library
and offer a uniform high-level interface for both users and machines to access the
combined library.

We claim that \sys will resolve two major bottlenecks in the current state of the art. It
will provide a permanent archiving solution that not all systems and user communities can
afford to maintain separately. And it will establish a standardized and open library
format that serves as a catalyst for comparison and thus evolution of systems.

Concretely, we see three ways the formal methods and mathematical knowledge management
communities can benefit from \sys:
\begin{inparaenum}[\em i\rm)]
\item users can view formerly disparate developments in a common, neutral framework and
  compare them,
\item system developers can import libraries from other logical systems to extend the
  reach of formalizations and avoid duplicate development
\item the existence of a library management system (and importable content) can lower the
  entry hurdle for developing new logic-based systems.
\end{inparaenum}
In the next section we present the current system architecture and realization, and
Section~\ref{sec:concl} concludes the paper.

\section{System Architecture and Realization}\label{sec:arch}

\sys is realized as an instance of the Planetary System~\cite{Kohlhase:ppte12}, which we have substantially extended in the course of the work reported here. 

The system architecture has three main components: 
\begin{inparaenum}[\em i\rm)]
 \item a versioned \emph{data store} holding the source documents
 \item a \emph{semantic service provider} that imports the source documents and provides services for them 
 \item and a \emph{frontend} that makes the sources and the semantic services available to users.
\end{inparaenum}
Specifically, we use the GitLab repository manager~\cite{GitLab:on} as the data store, the \mmt API as the semantic service provider and Drupal as the frontend. 

Figure~\ref{fig:arch} shows the detailed architecture.

\begin{figure}[ht]\centering
\begin{tikzpicture}
  \pgfdeclareimage[width=1cm]{user}{user}
  \pgfdeclareimage[width=1cm]{author}{author}
  \tikzstyle{system} = [rectangle, draw, fill=blue!20, text width=1cm, text centered,
                                    rounded corners, minimum height=.8cm,shade, 
                                    top color=white, bottom color=blue!20]
   \tikzstyle{database} = [cylinder,cylinder uses custom fill,
      cylinder body fill=yellow!50,cylinder end fill=yellow!50,
      shape border rotate=90,
      aspect=0.25,draw]
\node (user) {\pgfuseimage{user}}; 
\node[system,right=1.5cm of user] (browser) {Browser}; 
\node[system,right=1.3cm of browser] (drupal) {Drupal};
\node[system,above right=0cm and 1.4cm of drupal] (mmt) {MMT};
\node[system,below right=0cm and 1.4cm of drupal] (gl) {GitLab};
\node[database,below left=-.3cm and 1.5cm of gl] (lib) {library};
\node[below right=-.6cm and 1.6cm of gl] (author) {\pgfuseimage{author}}; 
\node[below right=-.5cm and .6cm of mmt] (conv) {\begin{tabular}{l}\footnotesize convert to\\ OMDoc/MMT\end{tabular}};
\draw[<-,thick] (mmt) -- node[left]{load} (gl);
\draw[<->,dotted] (user) -- node[above]{read} node[below]{interact} (browser);
\draw[->,thick] (browser) -- node[above]{REST} (drupal);
\draw[->,thick] (browser) to[loop above] node [above] (jobad) {JOBAD} (browser); 
\draw[->,thick] (gl) to[loop left,out=20,in=45,looseness=14] (gl); 
\draw[<->,dashed] (jobad) -- (mmt);
\draw[<->,dashed] (conv) -- (mmt);
\draw[<-,thick] (drupal) -- node[above]{present}(mmt); 
\draw[->,thick] (drupal) -- node[above]{edit}(gl); 
\draw[->,dotted] (author) -- node[above]{local} node[below]{edit} (gl);
\draw[->,dotted] (lib) -- node[below]{import} (gl);
\end{tikzpicture}
\caption{The \sys Architecture}\label{fig:arch}
\end{figure}
In this setup, Drupal serves as a container management system\footnote{Drupal and similar
  systems self-describe as content management systems, but they actually only manage the
  documents without changing their internal structure.} that supplies uniform theming,
user management, discussion forums, etc. GitLab on the other hand, provides versioned
storage of the content documents, and organizes them into repositories owned by users and
groups. 
The advantage of this setup is that we can combine two methods for accessing the
contents of \sys: 
\begin{inparaenum}[\em i\rm)]
\item an online, web-based editing/interaction workflow for the casual user, in the spirit
  of the Planetary system and
\item (new) an offline editing/authoring workflow based on a GIT working copy.
\end{inparaenum}
The latter is important for power authors and for bulk editing jobs. A user can fork or
pull the relevant repositories from GitLab, edit them and submit them back to MathHub
either via a pull request to the repository masters or a direct commit/push. As the
content is usually highly networked and distributed across multiple GIT repositories, we
have developed a command line tool \texttt{lmh} (\underline{l}ocal
\underline{M}ath\underline{H}ub) that manages working copies across repository
borders. 

In the web-based system, semantic services (notation-based, presentation, definition lookup,
relational navigation, dependency management, etc.) are provided by \mmt and are made available to the user, primarily by dedicated JOBAD~\cite{GLR:WebSvcActMathDoc09} modules.
Note that even though the active document functionalities and semantic editing
support in MathHub.info are based on \omdoc/\mmt representation of the content, the
authors interact with the content in the source format. Both of these representations are versioned in
GitLab and are converted into \omdoc/\mmt by custom transformers. \texttt{lmh} also
supports running these transformers locally and previewing HTML5 renderings of the
generated \omdoc/\mmt.

In order to to deal with flexiformal mathematical content in \omdoc, we have also extended the \mmt API, which was previously restricted to fully formal content. In the extended \mmt API, each \mmt service works whenever it is theoretically applicable (e.g. type checking when there exists type information, change management when there is dependency information, etc.).


% \section{Content \& Licensing}\label{sec:orga}
% \sys is intended both as a portal for active mathematical documents as well as a
% development environment and archive of flexiformal mathematics ranging from fully formal
% libraries of theorem provers to informal documents preloaded with explicit structural and
% semantic annotations. We are currently hosting a test set of formal and informal
% mathematical content to develop and evaluate system functionality; concretely:
% \begin{inparaenum}[\em i\rm)]
% \item the SMGloM termbase with ca. 1500 small \sTeX files containing definitions of
%   mathematical terminology and notation definitions.
% \item ca. 6500 files with \sTeX-encoded teaching materials (slides, course notes,
%   problems, and solutions) in Computer Science,
% \item the LATIN logic atlas with ca. 1000 meta-theories and logic morphisms, 
% \item the Mizar Mathematical Library of ca. 1000 articles with ca. 50.000 theorems,
%   definitions, and proofs, and
% \item a part of the HOL Light Library with 22 theories and over 2800 declarations.
% \end{inparaenum}
% 
% In the future we want to open the portal up for user-supplied content, eliciting documents
% from mathematics and nearby disciplines. We anticipate that \sys may be attractive to
% authors because it 
% \begin{inparaenum}[\em i\rm)]
% \item offers free private repositories
% \item allows to transform math papers into hosted, searchable HTML5 documents
% \item allows to add interactivity by semantic annotations in a stepwise fashion.
% \end{inparaenum}
% But ultimately, we are interested in a communal resource, in which the document sources
% are available for inspection, re-use and semantic analysis. Therefore the free private
% repositories are in what we call \defemph{public escrow}, i.e. they are private as long as
% the user actively requests them to be (e.g. during development and submission). If an
% author or author group fails to renew the public escrow, the sources are published under
% copyleft license of the user's choice. We hope that with this measure we will be able to
% enlarge the corpus of flexiformal documents available for research purposes.

\section{Conclusion}\label{sec:concl}

\sys is deployed at \url{http://mathhub.info} and has reached a state, where it can be
used for initial experiments and resources, but has not been scaled much beyond 10\,000
documents and a couple of dozens or users and repositories yet.

Specifically, we are currently hosting a test set of formal and informal
 mathematical content to develop and evaluate system functionality; concretely:
 \begin{inparaenum}[\em i\rm)]
 \item the SMGloM termbase with ca. 1500 small \sTeX files containing definitions of
   mathematical terminology and notation definitions.
 \item ca. 6500 files with \sTeX-encoded teaching materials (slides, course notes,
   problems, and solutions) in Computer Science,
 \item the LATIN logic atlas with ca. 1000 meta-theories and logic morphisms, 
 \item the Mizar Mathematical Library of ca. 1000 articles with ca. 50.000 theorems,
   definitions, and proofs, and
 \item a part of the HOL Light Library with 22 theories and over 2800 declarations.
 \end{inparaenum}
Already now, it is unique
in its class in that it gives a unified interface to multiple theorem prover libraries
together with linguistic and educational resources. Now that the ground work has been
laid, we anticipate the rapid integration of new semantic services, editing support and
new content.

\paragraph{Acknowledgements} This work has been partially funded by DFG under Grant KO
2428/13-1. The authors acknowledge that the MathHub system builds on a long series of
experiments in system integration in the KWARC group and that the design and
implementation would not have been possible without substantial discussions in the group.

\printbibliography
\end{document}

%%% Local Variables: 
%%% mode: latex
%%% TeX-master: t
%%% End: 

% LocalWords:  maketitle KohDavGin pswads11 planetmath tntbase concl emph mkm05 lt92 ge
% LocalWords:  printbibliography newpart ednote hlt08 btc07 Lawvere thy-grph tn biform tp
% LocalWords:  seq ldots vdots noindent subtheory tikzpicture tdots thygraph cn funsat le
% LocalWords:  cdots ts1 ts2 tsdots tsn realm-mk realmref textbf realmref circ funsattac
% LocalWords:  compactenum highlevel wrapfigure vspace yscale textsf KohIan redge rcedge
% LocalWords:  ssmk12 defemph subseteq tpc12 RabKoh hookrightarrow xscale leadsto atp ci2
%  LocalWords:  KarMer ftg79 ttg59 slcirc circsl scriptscriptstyle invsl defeq mmtthy rm
%  LocalWords:  flatthy boxedquote varphi varphi bigraph mmtar medskip includeleft nmmtar
%  LocalWords:  pviewleft Kleisli mpd11 inparaenum overline lsl esl mapsto mapsto mapsto
%  LocalWords:  StaKoh tlcspx10 ppte12 Rabe omdoc orga mmt athematical uments odular gl
%  LocalWords:  CodHorKoh palai11 pgfdeclareimage tikzstyle pgfuseimage broswer Broswer
%  LocalWords:  jobad footnotesize texttt lmh ocal ath ub SMGloM termbase
