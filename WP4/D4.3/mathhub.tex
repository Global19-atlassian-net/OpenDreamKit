\section{The \sys System/Portal}\label{sec:mathhub}

\subsection{System Architecture and Realization}\label{sec:arch}

\sys is realized as an instance of the Planetary System~\cite{Kohlhase:ppte12}, which we
have substantially extended in the course of the work reported here.

The system architecture has three main components: 
\begin{compactenum}[\em i\rm)]
 \item a versioned \emph{data store} holding the source documents
 \item a \emph{semantic service provider} that imports the source documents and provides services for them 
 \item and a \emph{frontend} that makes the sources and the semantic services available to users.
\end{compactenum}
Specifically, we use the GitLab repository manager~\cite{GitLab:on} as the data store, the
\mmt API~\cite{Rabe:MAGMS13,uniformal:on} as the semantic service provider and build
system, and Drupal~\cite{drupal:on} as the user-pointing frontend. Figure~\ref{fig:arch}
shows the overall architecture of the \sys system.

\begin{figure}[ht]\centering
\begin{tikzpicture}
  \pgfdeclareimage[width=1cm]{user}{user}
  \pgfdeclareimage[width=1cm]{author}{author}
  \tikzstyle{system} = [rectangle, draw, fill=blue!20, text width=1cm, text centered,
                                    rounded corners, minimum height=1.6cm,shade, 
                                    top color=white, bottom color=blue!20,
                                    minimum width=2.2cm]
   \tikzstyle{database} = [cylinder,cylinder uses custom fill,
      cylinder body fill=yellow!50,cylinder end fill=yellow!50,
      shape border rotate=90,
      aspect=0.25,draw]
\node (user) {\pgfuseimage{user}}; 
\node[system,right=1.5cm of user] (browser) {Browser}; 
\node[system,right=1.5cm of browser] (drupal) {Drupal};
\node[system,above right=0cm and 1.6cm of drupal] (mmt) {MMT API}; 
\node[system,right=1cm of mmt] (build) {MMT build}; 
\node[system,below right=0cm and 1.6cm of drupal] (gl) {GitLab};
\node[database,below left=-.6cm and 3cm of gl] (lib) {library};
\node[below right=-1.2cm and 1.5cm of gl] (author) {\pgfuseimage{author}}; 
\draw[<-,thick] (mmt) -- node[left]{load} (gl);
\draw[<->,dotted] (user) -- node[above]{read} node[below]{interact} (browser);
\draw[->,thick] (browser) -- node[above]{REST} (drupal);
\draw[->,thick] (browser) to[loop above] node [above] (jobad) {JOBAD} (browser); 
\draw[<->,dashed] (jobad) -- (mmt);
\draw[<-,thick] (drupal) -- node[above]{present}(mmt); 
\draw[->,thick] (drupal) -- node[above]{edit}(gl); 
\draw[->,dotted] (author) -- node[above]{local} node[below]{edit} (gl);
\draw[->,dotted] (lib) -- node[below]{import} (gl);
\draw[<->,thick] (gl) -- (build);
\end{tikzpicture}
\caption{The \sys Architecture}\label{fig:arch}
\end{figure}

In this setup, Drupal serves as a container management system\footnote{Drupal and similar
  systems self-describe as content management systems, but they actually only manage the
  documents without changing their internal structure.} that supplies uniform theming,
user management, discussion forums, etc. GitLab on the other hand, provides versioned
storage of the content documents, and organizes them into repositories owned by users and
groups (math archives).

The \mmt API provides semantic services (notation-based, presentation, definition lookup,
relational navigation, dependency management, etc.) to the user, primarily by dedicated
JOBAD~\cite{GLR:WebSvcActMathDoc09} modules. Note that even though the active document 
functionalities and semantic editing support in \sys are based on \omdoc/\mmt
representation of the content, the authors interact with the content in the source
format. Both of these representations are versioned in GitLab and are converted into
\omdoc/\mmt by an \mmt-based build system. 

In order to deal with flexiformal mathematical content in \omdoc, we have also extended
the \mmt API, which was previously restricted to fully formal content. In the extended
\mmt API, each \mmt service works whenever it is theoretically applicable (e.g. type
checking when there exists type information, change management when there is dependency
information, etc.).

\subsection{Deployment and Content}

\sys is deployed at \url{http://mathhub.info} and has reached a state where it can be
used for the \pn project, but has not been scaled yet much beyond 10\,000 documents and a
couple dozens of users and repositories.

We are currently hosting a test set of formal and informal mathematical
content to develop and evaluate system functionality, specifically consisting of:
\begin{compactenum}[\em i\rm)]
\item the SMGloM termbase with ca. 1500 small \sTeX files containing definitions of
  mathematical terminology and notation definitions.
\item ca. 6500 files with \sTeX-encoded teaching materials (slides, course notes,
  problems, and solutions) in Computer Science,
\item the LATIN logic atlas with ca. 1000 meta-theories and logic morphisms,
\item the Mizar Mathematical Library of ca. 1000 articles with ca. 50.000 theorems,
  definitions, and proofs, and
\item a part of the HOL Light Library with 22 theories and over 2800 declarations.
\end{compactenum}

Already now, this set of archives is unique in its class in that it gives a
unified interface to multiple theorem prover libraries together with linguistic
and educational resources. Now that the ground work has been laid, we anticipate
the rapid integration of new semantic services, editing support and new content.


%%% Local Variables: 
%%% mode: latex
%%% TeX-master: "report"
%%% End: 
