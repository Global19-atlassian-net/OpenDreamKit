\section{The MathHub Portal}\label{sec:mathhub}

\subsection{ System Architecture and Realization}\label{sec:arch}

\sys is realized as an instance of the Planetary System~\cite{Kohlhase:ppte12}, which we have substantially extended in the course of the work reported here. 

The system architecture has three main components: 
\begin{inparaenum}[\em i\rm)]
 \item a versioned \emph{data store} holding the source documents
 \item a \emph{semantic service provider} that imports the source documents and provides services for them 
 \item and a \emph{frontend} that makes the sources and the semantic services available to users.
\end{inparaenum}
Specifically, we use the GitLab repository manager~\cite{GitLab:on} as the data store, the \mmt API as the semantic service provider and Drupal as the frontend. 

Figure~\ref{fig:arch} shows the detailed architecture.

\begin{figure}[ht]\centering
\begin{tikzpicture}
  \pgfdeclareimage[width=1cm]{user}{user}
  \pgfdeclareimage[width=1cm]{author}{author}
  \tikzstyle{system} = [rectangle, draw, fill=blue!20, text width=1cm, text centered,
                                    rounded corners, minimum height=.8cm,shade, 
                                    top color=white, bottom color=blue!20]
   \tikzstyle{database} = [cylinder,cylinder uses custom fill,
      cylinder body fill=yellow!50,cylinder end fill=yellow!50,
      shape border rotate=90,
      aspect=0.25,draw]
\node (user) {\pgfuseimage{user}}; 
\node[system,right=1.5cm of user] (browser) {Browser}; 
\node[system,right=1.3cm of browser] (drupal) {Drupal};
\node[system,above right=0cm and 1.4cm of drupal] (mmt) {MMT};
\node[system,below right=0cm and 1.4cm of drupal] (gl) {GitLab};
\node[database,below left=-.3cm and 1.5cm of gl] (lib) {library};
\node[below right=-.6cm and 1.6cm of gl] (author) {\pgfuseimage{author}}; 
\node[below right=-.5cm and .6cm of mmt] (conv) {\begin{tabular}{l}\footnotesize convert to\\ OMDoc/MMT\end{tabular}};
\draw[<-,thick] (mmt) -- node[left]{load} (gl);
\draw[<->,dotted] (user) -- node[above]{read} node[below]{interact} (browser);
\draw[->,thick] (browser) -- node[above]{REST} (drupal);
\draw[->,thick] (browser) to[loop above] node [above] (jobad) {JOBAD} (browser); 
\draw[->,thick] (gl) to[loop left,out=20,in=45,looseness=14] (gl); 
\draw[<->,dashed] (jobad) -- (mmt);
\draw[<->,dashed] (conv) -- (mmt);
\draw[<-,thick] (drupal) -- node[above]{present}(mmt); 
\draw[->,thick] (drupal) -- node[above]{edit}(gl); 
\draw[->,dotted] (author) -- node[above]{local} node[below]{edit} (gl);
\draw[->,dotted] (lib) -- node[below]{import} (gl);
\end{tikzpicture}
\caption{The \sys Architecture}\label{fig:arch}
\end{figure}
In this setup, Drupal serves as a container management system\footnote{Drupal and similar
  systems self-describe as content management systems, but they actually only manage the
  documents without changing their internal structure.} that supplies uniform theming,
user management, discussion forums, etc. GitLab on the other hand, provides versioned
storage of the content documents, and organizes them into repositories owned by users and
groups. 
The advantage of this setup is that we can combine two methods for accessing the
contents of \sys: 
\begin{inparaenum}[\em i\rm)]
\item an online, web-based editing/interaction workflow for the casual user, in the spirit
  of the Planetary system and
\item (new) an offline editing/authoring workflow based on a GIT working copy.
\end{inparaenum}
The latter is important for power authors and for bulk editing jobs. A user can fork or
pull the relevant repositories from GitLab, edit them and submit them back to MathHub
either via a pull request to the repository masters or a direct commit/push. As the
content is usually highly networked and distributed across multiple GIT repositories, we
have developed a command line tool \texttt{lmh} (\underline{l}ocal
\underline{M}ath\underline{H}ub) that manages working copies across repository
borders. 

In the web-based system, semantic services (notation-based, presentation, definition
lookup, relational navigation, dependency management, etc.) are provided by \mmt and are
made available to the user, primarily by dedicated JOBAD~\cite{GLR:WebSvcActMathDoc09}
modules.  Note that even though the active document functionalities and semantic editing
support in MathHub.info are based on \omdoc/\mmt representation of the content, the
authors interact with the content in the source format. Both of these representations are
versioned in GitLab and are converted into \omdoc/\mmt by custom
transformers. \texttt{lmh} also supports running these transformers locally and previewing
HTML5 renderings of the generated \omdoc/\mmt.

In order to to deal with flexiformal mathematical content in \omdoc, we have also extended
the \mmt API, which was previously restricted to fully formal content. In the extended
\mmt API, each \mmt service works whenever it is theoretically applicable (e.g. type
checking when there exists type information, change management when there is dependency
information, etc.).

\subsection{Deployment and Content}

\sys is deployed at \url{http://mathhub.info} and has reached a state, where it can be
used for initial experiments and resources, but has not been scaled much beyond 10\,000
documents and a couple of dozens or users and repositories yet.

Specifically, we are currently hosting a test set of formal and informal
 mathematical content to develop and evaluate system functionality; concretely:
 \begin{inparaenum}[\em i\rm)]
 \item the SMGloM termbase with ca. 1500 small \sTeX files containing definitions of
   mathematical terminology and notation definitions.
 \item ca. 6500 files with \sTeX-encoded teaching materials (slides, course notes,
   problems, and solutions) in Computer Science,
 \item the LATIN logic atlas with ca. 1000 meta-theories and logic morphisms, 
 \item the Mizar Mathematical Library of ca. 1000 articles with ca. 50.000 theorems,
   definitions, and proofs, and
 \item a part of the HOL Light Library with 22 theories and over 2800 declarations.
 \end{inparaenum}
 Already now, it is unique in its class in that it gives a unified interface to multiple
 theorem prover libraries together with linguistic and educational resources. Now that the
 ground work has been laid, we anticipate the rapid integration of new semantic services,
 editing support and new content.

%%% Local Variables: 
%%% mode: latex
%%% TeX-master: report
%%% End: 

% LocalWords:  maketitle KohDavGin pswads11 planetmath tntbase concl emph mkm05 lt92 ge
% LocalWords:  printbibliography newpart ednote hlt08 btc07 Lawvere thy-grph tn biform tp
% LocalWords:  seq ldots vdots noindent subtheory tikzpicture tdots thygraph cn funsat le
% LocalWords:  cdots ts1 ts2 tsdots tsn realm-mk realmref textbf realmref circ funsattac
% LocalWords:  compactenum highlevel wrapfigure vspace yscale textsf KohIan redge rcedge
% LocalWords:  ssmk12 defemph subseteq tpc12 RabKoh hookrightarrow xscale leadsto atp ci2
%  LocalWords:  KarMer ftg79 ttg59 slcirc circsl scriptscriptstyle invsl defeq mmtthy rm
%  LocalWords:  flatthy boxedquote varphi varphi bigraph mmtar medskip includeleft nmmtar
%  LocalWords:  pviewleft Kleisli mpd11 inparaenum overline lsl esl mapsto mapsto mapsto
%  LocalWords:  StaKoh tlcspx10 ppte12 Rabe omdoc orga mmt athematical uments odular gl
%  LocalWords:  CodHorKoh palai11 pgfdeclareimage tikzstyle pgfuseimage broswer Broswer
%  LocalWords:  jobad footnotesize texttt lmh ocal ath ub SMGloM termbase
