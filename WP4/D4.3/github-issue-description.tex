\section*{\texorpdfstring{Deliverable description, as taken from Github
issue
\href{https://github.com/OpenDreamKit/OpenDreamKit/issues/92}{\#92} on
2017-02-28}{Deliverable description, as taken from Github issue \#92 on 2017-02-28}}\label{deliverable-description-as-taken-from-github-issue-92-on-2017-02-28}

\begin{itemize}
\tightlist
\item
  \textbf{WP4:}
  \href{https://github.com/OpenDreamKit/OpenDreamKit/tree/master/WP4}{User
  Interfaces}
\item
  \textbf{Lead Institution:} Friedrich-Alexander Universität
  Erlangen/Nürnberg (FAU), formerly Jacobs Universität Bremen
\item
  \textbf{Due:} 2017-02-28 (Month 18)
\item
  \textbf{Nature:} Demonstrator
\item
  \textbf{Task:} T4.7
  (\href{https://github.com/OpenDreamKit/OpenDreamKit/issues/75}{\#75}):
  Active Documents Portal, T4.8
  (\href{https://github.com/OpenDreamKit/OpenDreamKit/issues/76}{\#76}):
  Visualisation system for 3D data in web-notebook
\item
  \textbf{Proposal:}
  \href{https://github.com/OpenDreamKit/OpenDreamKit/raw/master/Proposal/proposal-www.pdf}{p.~48}
\item
  \href{https://github.com/OpenDreamKit/OpenDreamKit/raw/master/WP4/D4.3/report-final.pdf}{\textbf{Final
  report}}
\end{itemize}

One of the most prominent features of a virtual research environment
(VRE) is a unified user interface. The OpenDreamKit approach is to
create a mathematical VRE by integrating various pre-existing
mathematical software systems. There are two approaches that can serve
as a basis for the OpenDreamKit UI: \texttt{computational\ notebooks}
and \texttt{active\ documents}. The former allows for mathematical text
around the computation cells of a real-eval-print loop of a mathematical
software system and the latter makes semantically annotated documents
active.

MathHub is a portal for active mathematical documents ranging from
formal libraries of theorem provers to informal -- but rigorous --
mathematical documents lightly marked up by preserving LaTeX markup. In
the OpenDreamKit project MathHub acts as:

\begin{itemize}
\item
  A portal and management system for theory-graph structured active
  documents, i.e.~documents that use the semantic structure of the
  document and the knowledge context it links to to render semantic
  services embedded in the document -- it becomes active,
  i.e.~interactive, reactive - see D4.2
  (\href{https://github.com/OpenDreamKit/OpenDreamKit/issues/91}{\#91}).
\item
  The repository for the Math-in-the-Middle (MitM) ontology, see D6.2
  (\href{https://github.com/OpenDreamKit/OpenDreamKit/issues/136}{\#136}).
  This ontology is used as a basis for interoperability of the
  mathematical software systems that make up the OpenDreamKit VRE
  toolkit, which is a crucial concern for work package
  \href{https://github.com/OpenDreamKit//OpenDreamKit/tree/master/WP6}{WP6}.
\end{itemize}

As the authoring, maintenance, and curation of theory-structured
mathematical ontologies and the transfer of mathematical knowledge via
active documents are an important part of the OpenDreamKit VRE toolkit,
the editing facilities in MathHub play a great role for the project.

This report discusses the main design decisions of the editing
facilities in MathHub; they can be assessed at
\url{http://mathhub.info}.
