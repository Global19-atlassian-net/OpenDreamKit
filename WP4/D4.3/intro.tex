\section{Introduction}\label{sec:intro}

\sys is a portal for active mathematical documents ranging from formal libraries of
theorem provers to informal -- but rigorous -- mathematical documents lightly marked up by
preserving {\LaTeX} markup. In the \pn project \sys acts as: 
\begin{itemize}
\item A portal and management system for theory-graph structured active documents,
  i.e. documents that use the semantic structure of the document and the knowledge context
  it links to to render semantic services embedded in the document -- it becomes
  \emph{active}, i.e. interactive, reactive -- see~\cite{ODK-D4.3}, self-adapting --
  see~\cite{ODK-D4.2}.
\item The repository for the Math-in-the-Middle (MitM) ontology
  (see~\cite{DehKohKon:iop16,ODK-D6.2}). This ontology is used as a basis for
  interoperability of the mathematical software systems that make up the \pn VRE toolkit,
  which is a crucial concern for work package \textbf{WP6}.
\end{itemize}
As the authoring, maintenance and curation of theory-structured mathematical ontologies
and the transfer of mathematical knowledge via active documents are an important part of
the \pn VRE toolkit, the editing facilities in \sys play a great role for the
project. This report discusses the main design decisions of the editing facilities in
\sys; they can be assessed at \url{http://mathhub.info}.

In Section~\ref{sec:mathhub} we briefly present the \sys system as a basis and in
Section~\ref{sec:editing} we drill in on the editing-specific system
features. Section~\ref{sec:concl} concludes the report. 

%%% Local Variables:
%%% mode: latex
%%% TeX-master: "report"
%%% End:

%  LocalWords:  sec:intro DehKohKon:iop16,ODK-D6.2 sec:mathhub sec:editing sec:concl emph
%  LocalWords:  itemize textbf
