A mathematical Virtual Research Environment (VRE) needs to support two kinds of user interface paradigms.
Firstly, mathematical \emph{documents} have been very successful for presenting mathematical knowledge.
While there have been efforts to make them modular and interactive, they predominantly remain in the mode of archiving and transporting knowledge in Mathematics.
Secondly, \emph{notebook} interfaces focus on REPL (Read/Eval/Print Loop) interaction leading to documents consisting of a sequence of computational cells within which the mathematical discourse is interspersed in the form of rich comments.
%\footnote {A ``literate programming'' version of notebooks which gives mathematical discourse structural precedence is possible in principle but has not been supported consistently at the system level.}

In this report we present a prototypical integration of Jupyter Notebooks into the MathHub document hosting platform.
In addition to versioned persistent storage, MathHub offers an interface for reading, writing, and interacting with mathematical documents.
Jupyter offers a uniform interface to the computation facilities of the OpenDreamKit VRE toolkit in the form of a read-eval-print loop (REPL).
The tension between these systems has previously been explored in OpenDreamKit Deliverable D4.2~\cite{ODK-D4.2}, and the concept of in-document computation in OpenDreamKit Deliverable D4.9~\cite{ODK-D4.9}.
In both cases, the integration was incomplete, since it lacked a full integration of the underlying knowledge management and computation services.

Generally, the integration of MathHub and Jupyter consists of two challenges:
\begin{inparaenum}[\em a\rm )]
\item the integration of the document paradigms and user interfaces and
\item the integration of the knowledge management and computation services.
\end{inparaenum}
The latter requires defining the semantics of the mathematical knowledge maintained in the user interfaces, and both Jupyter and MathHub are parametric in this semantics.
In Jupyter, a separate kernel must be provided for each concrete language, in particular there are separate kernels for all computation systems used in OpenDreamKit.
In MathHub, the determination of the semantics is delegated to the MMT system.
Moreover, MMT is also used in OpenDreamKit as the center of the Math-in-the-Middle based system integration (see OpenDreamKit Deliverable D6.5~\cite{ODK-D6.5}).

% The first step is setting up a Jupyter server, which currently runs on \url{http://juypter.mathhub.info}. \ednote{KA: maybe show picture of it?}
% For this server, we have developed a custom kernel, that forwards the input entered into the Jupyter notebook to the MMT backend.
% This then processes said input and sends the response back to the Jupyter frontend via the kernel.
% We will cover the implementation of the Jupyter kernel and the MMT-backend, later in this report.

This reports describes progress in both integration challenges.

In Section~\ref{sec:mmt-jp}, for the integration of services, we present an MMT kernel for Jupyter.
This not only makes the MMT functionality available at the Jupyter level, but also deeply integrates Jupyter widgets with the MMT Scala level.
Widgets are a key Jupyter feature that reaches far beyond the standard REPL interaction.
For instance, the Jupyter community has developed a large array of widgets for interactive 2D and 3D visualization of data in the form of charts, maps, tables, etc.

In Section~\ref{sec:nb-mh}, for the integration of document paradigms, we first show how to extend MathHub with a Jupyter server that allows viewing notebooks stored in MathHub.
Then we present a MathHub feature that allows using interactive, ephemeral Jupyter Notebooks as subdocuments of static mathematical documents, e.g., HTML pages generated from scientific articles.

In Section~\ref{sec:mitm-nb}, we present two cases studies that evaluate our results: in-document computing facilities in active documents and a knowledge-based specification dialog for modeling and simulation.
Section~\ref{sec:concl} concludes the report.

\paragraph{Acknowledgements}
The authors gratefully acknowledge the support of the Jupyter team and in particular the advice of Benjamin Ragan-Kelly.
The MoSIS system was developed in collaboration with Theresa Pollinger~\cite{PolKohKoe:kacse18}. 

%%% Local Variables:
%%% mode: latex
%%% mode: visual-line
%%% fill-column: 5000
%%% TeX-master: "report"
%%% End:

%  LocalWords:  emph Jupyter inparaenum ednote sec:mmt-jp visualization sec:nb-mh sec:mitm-nb sec:concl PolKohKoe:kacse18
