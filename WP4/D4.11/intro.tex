In this report we present a prototypical integration of the Jupyter notebooks into the MathHub.info portal for active mathematical documents and a versioned hosting system for flexiformal mathematics.
MathHub.info offers a rich interface for reading, writing, and interacting with mathematical documents and knowledge. Jupyter offers a uniform interface to the computation facilities of the OpenDreamKit VRE toolkit in the form of a read-eval-print loop (REPL).

An mathematical Virtual Research Environment (VRE) needs both kinds of interface functionality: mathematical documents have been very successful for presenting mathematical knowledge, and while there have been efforts to make them modular and interactive remain the dominant mode of archiving and transporting knowledge in Mathematics.
Notebook interfaces which also use the document metaphor at the surface, but concentrate on the REPL interaction and are largely a sequence of computational interaction cells where mathematical discourse largely appears in the form of ``extended comments''.
A ``literate programming'' version of notebooks which gives mathematical discourse structural precedence is possible in principle, but has not been supported consistently at the system level.
This tension and trade-off has been explored in OpenDreamKit Deliverable D4.9~\cite{ODK-D4.2}, and the concept of in-document computation in OpenDreamKit Deliverable D4.9~\cite{ODK-D4.9}.
In both cases, the integration was incomplete, since it lacked a full integration of the back-ends. 

Generally, the integration of MathHub and Jupyter consists of two parts: the integration of
\begin{inparaenum}[\em a\rm )]
\item user interfaces (as reported previously) and
\item the knowledge/computation management facilities. 
\end{inparaenum}
Here we report on progress in both: we for the backend integration we present an MMT kernel for Jupyter (knowledge management and Math-in-the-Middle-based distributed computation (see OpenDreamKit Deliverable D6.5~\cite{ODK-D6.5})).
We also show how the new Jupyter widgets can be deeply integrated with the MMT knowledge management facilities to give semantics-aware interaction facilities, extending the front-end capabilities of MathHub/Jupyter Notebooks by semantic widgets driven the the MMT in-document knowledge management services.
 
We show and evaluate the integration on two case studies: in-document computing facilities in active documents and a knowledge-based specification dialog for modeling and simulation. 

This report is structured as follows. In Section~\ref{sec:mmt-jp} we report on the MathHub/Jupyter integration at the system level: a Jupyter server as part of the MathHub system and a MMT kernel for Jupyter. Section~\ref{sec:nb-mh} presents the integration of Jupyter Notebooks as active documents in the (new) MathHub front-end, and Section~\ref{sec:mitm-nb} presents the two case studies. Section~\ref{sec:concl} concludes the report and discusses future work.


The goal of this report is to integrate Jupyter notebooks into MathHub
and make them compatible with MMT, in a way that we can conveniently use 
MMT syntax in these notebooks and also a little bit of extra functionality
like e.g. the Jupyter widgets. The first step is setting up a Jupyter server,
which currently runs on \url{http://juypter.mathhub.info}. \ednote{KA: maybe show picture of it?}
For this server, we have developed a custom kernel, that forwards the input 
entered into the Jupyter notebook to the MMT backend. This then processes 
said input and sends the response back to the Jupyter frontend via the kernel.
We will cover the implementation of the Jupyter kernel and the MMT-backend,
later in this report.


\paragraph{Acknowledgements} The authors gratefully acknowledge the support of the Jupyter team and in particular the advice of Benjamin Ragan-Kelly. Also, the input of Theresa Pollinger and her work on the MoSIS system~\cite{PolKohKoe:kacse18} has shaped our perception of the integration reported here. 

%%% Local Variables:
%%% mode: latex
%%% mode: visual-line
%%% fill-column: 5000
%%% TeX-master: "report"
%%% End:
