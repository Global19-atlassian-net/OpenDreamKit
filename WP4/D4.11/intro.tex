
\paragraph{Motivation}
%In this report we present a prototypical integration of the Jupyter notebooks into the MathHub.info portal for active mathematical documents and a versioned hosting system for flexiformal mathematics.
Jupyter notebooks offer a uniform interface to computation systems, including the ones involved in the OpenDreamKit project.
It is intended to serve as the frontend component of the virtual research environment (VRE) built by OpenDreamKit.
MathHub.info offers a scalable versioned project management system for mathematical documents.
It is intended to serve as the backend component of the OpenDreamKit VRE.
MMT provides a formal system for representing the mathematical knowledge involved in the OpenDreamKit project.
It is intended to serve as the semantic integration layer of the OpenDreamKit VRE.

A mathematical (VRE) needs all of these functionalities.
Notably, while Jupyter notebooks (which also use the document metaphor at the surface) concentrate on the REPL interaction and are largely a sequence of computational commands, MathHub supports large structured mathematical documents.
The former is great for experimentation and sharing of snippets. The latter is critical for archiving and distributing bodies of mathematical knowledge.
%A ``literate programming'' version of notebooks which gives mathematical discourse structural precedence is possible in principle, but has not been supported consistently at the system level.
This tension and trade-off has been explored in OpenDreamKit Deliverable D4.9~\cite{ODK-D4.2}, and the concept of in-document computation in OpenDreamKit Deliverable D4.9~\cite{ODK-D4.9}.
In both cases, the integration was incomplete, since it lacked a full integration of the back-ends.

Both Jupyter and MathHub lack a mathematical semantics layer.
The former uses computation as the definition of the semantics.
The latter relies on human understanding.
But a VRE aiming at the integration of systems requires such a semantics layer in order to safeguard the transport of knowledge among systems, among users, or between systems and users.
Therefore, OpenDreamKit has developed the Math-in-the-Middle approach, where MMT is used to formalize the relevant mathematical background knowledge that is implicit in the systems' source code and the users' minds.

Therefore, OpenDreamKit integrates these three components.
Generally, this consists of two parts:
\begin{inparaenum}[\em a\rm )]
\item user interfaces (as reported previously) in the frontend
\item knowledge management and computation facilities in the backend.
\end{inparaenum}
Here we report on progress in both: for the backend integration we present an MMT kernel for Jupyter (knowledge management and Math-in-the-Middle-based distributed computation (see OpenDreamKit Deliverable D6.5~\cite{ODK-D6.5})).
For the frontend, we show how Jupyter widgets can be deeply integrated with the data structures maintained by MMT (even though this requires bridging between two programming languages), which allows us to extend MathHub/Jupyter notebooks with semantics aware widgets driven by the MMT in-document knowledge management services.

We demonstrate and evaluate our integration on two case studies: in-document computing facilities in active documents and a knowledge-based specification dialog for modeling and simulation.

\paragraph{Overview}
Section~\ref{sec:mmt-jp} describes our MMT kernel for Jupyter.
Section~\ref{sec:nb-mh} presents the integration of Jupyter Notebooks in the MathHub front-end.
Section~\ref{sec:mitm-nb} presents our case studies.
Section~\ref{sec:concl} concludes the report and discusses future work.

\paragraph{Acknowledgements}
The authors gratefully acknowledge the support of the Jupyter team and in particular the advice of Benjamin Ragan-Kelly.
Theresa Pollinger worked on the MoSIS system~\cite{PolKohKoe:kacse18}, which has shaped our perception of the integration reported here, and which we sketch at the end. 

%%% Local Variables:
%%% mode: latex
%%% mode: visual-line
%%% fill-column: 5000
%%% TeX-master: "report"
%%% End:
