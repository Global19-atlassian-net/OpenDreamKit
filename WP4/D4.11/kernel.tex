\ednote{KA: ask Tom about the server}

We have designed and implemented a Jupyter kernel for MMT.
We describe its interface in Section~\ref{sec:kernel:syntax} and the implementation in Section~\ref{sec:kernel:impl}.

\subsection{Interface}\label{sec:kernel:syntax}

MMT differs from typical languages using Jupyter in that it does not only or even primarily perform computation but also handles symbolic expressions with uninterpreted function symbols.
To adequately handle these subtleties, we systematically specified a new interface for interacting with MMT in Jupyter-style.

\paragraph{Input}
The possible inputs excepted by the MMT kernel are divided into three groups.
\begin{itemize}
\item Global management commands allow displaying and deleting all current sessions.
 In practice, these commands are typically not available to users, which should only have access to their own session.
\item Local management commands allow starting, quitting, and restarting the current session. These are the main commands issued by the frontend in response to user action.
\item Content commands are the mathematically relevant commands and described below.
\end{itemize}

The content commands are divided into two groups:
\begin{itemize}
 \item Write-commands send new content to the backend in order to build MMT documents step by step.
   The backend maintains one MMT document for each session, and any write command changes that document.
 \item Read-commands retrieve information from the backend without changing the session's document.
   These include lookups (both in the session document and in any other accessible document) or computations.
\end{itemize}

A write-command typically consists of a single MMT declaration roughly corresponding to a line in a typical MMT source file.
However, the nesting of declarations is very important in MMT (contrary to many programming language kernel where nesting is optional).
Therefore, all declarations that may contain other declarations (most importantly all MMT documents and theories) are split into parts as follows: the header, the list of nested declarations, and a special end-of-nesting marker. These are communicated in separate write-commands.
The MMT kernel maintains the current scope as an MMT URI of an MMT document or theory and updates it on every write-commands that opens or closes one of them.
This ensures that all nested declarations are parsed and interpreted in the right scope.

For example, the following sequence of commands builds two nested theories where the inner one refers to a type declared in the outer one:
\begin{itemize}
\item \texttt{theory Test: ur:?LF =}
\item \texttt{a: type}
\item \texttt{theory Test2: ur:?LF =}
\item \texttt{c:a}
\item \texttt{end}
\item \texttt{end}
\end{itemize}
\ednote{@Kai: make this prettier and sync it with the example for which you add a screenshot later}

A special write-command is \texttt{eval T}.
It interprets \texttt{T} in the current scope, infers it type \texttt{A}, computes it, and then adds the declaration \texttt{resI:A=T} to the current theory, where \texttt{I} is a running counter of unnamed declarations.
This corresponds most closely to the REPL functionality in typical Jupyter kernels.

While write-commands correspond closely to the available types of MMT declarations, the set of read-commands is extensible.
For example, the commands \texttt{get U} where \texttt{U} is any MMT URI returns the MMT declaration of that URI.

\paragraph{Output}
The kernel returns the following kinds of return messages:
\begin{itemize}
\item Admin messages are strings returned in response to session management commands.
\item New-element messages return the declaration that was added by a write-command.
\item Existing-element messages return the declaration that was retrieved by a \texttt{get} command.
\end{itemize}
Like read-commands, the set of output messages is extensible.

The new-element and existing-element messages initially return the declaration in MMT's abstract syntax.
And a post-processing layer specific to Jupyter renders them in HTML presentation.
That way the core kernel functionality can be reused easily in other frontends.

\subsection{Implementation}\label{sec:kernel:impl}

The Jupyter infrastructure heavily relies on Python, especially when it comes to Jupyter widgets.
Therefore, it makes sense to implement our kernel in Python.
However, actually executing the user's commands requires a strong integration with the MMT implementation, which uses Scala.
That made it advisable to implement all Jupyter-specific functionality, especially the communication and management, in Python, while all mathematically relevant intelligence remained in MMT.

This is a generally difficult problem.
After some experiments with different solutions (e.g., HTTP communication) and discussion within the OpenDreamKit community, we identified the Py4j library as the best choice.
This is a Python-JVM bridge that allows seamless interaction between Python and any language (such as Scala) that compiles to the JVM.
Thus, our Python kernel can call MMT code directly.
Valuable Py4j features include callbacks from MMT to Python, shared memory (by treating pointers to JVM objects as Python values), and synchronized garbage collection.
That makes our kernel very robust against bit rot and allows benefiting from future improvements to the MMT backend.

Py4j is only JVM-specific, not Scala-specific.
That means that some Scala-specific constructs are not readily exposed to Python.
For example, both Python and Scala allow magic methods for treating any object as a function, but the JVM does not; moreover, the magic method is called \texttt{\_\_call\_\_} in Python and \texttt{apply} in Scala.
Similarly, Scala collections like lists are not automatically seen as their counterparts in Python.
Therefore, we wrote a Python module (which is distributed with MMT) that performs the bureaucracy of matching up Python and Scala.

\ednote{@Kai: add a few details about how the communication works; e.g., by describing how one of the commands from the previous section is handled}


%%% Local Variables:
%%% mode: latex
%%% mode: visual-line
%%% fill-column: 5000
%%% TeX-master: "none"
%%% End:
