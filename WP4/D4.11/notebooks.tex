We have installed a Jupyter server as a component of our MathHub system.
It provides a normal Jupyter server except for additionally supporting our MMT kernel.
It is available at the URL \url{jupyter.mathhub.info}.\ednote{@Kai: check this}

We have added Jupyter notebooks as a new document type in MathHub.info\footnote{As the MathHub front-end is currently undergoing major re-write.
  The old MathHub interface was based on Drupal, which led to major system vulnerabilities and therefore maintenance hassles, because Drupal was targeted by hackers.
  We are currently working on a docker-based orchestration of services with a React.JS based front-end in the general spirit of the OpenDreamKit VRE toolkit; see \url{https://github.com/MathHubInfo/}. The new system can be accessed at \url{http://new.mathhub.info}.}; see Figure~\ref{fig:mathhub-NB} for an example. In the MathHub front-end, documents are displayed with multiple tabs:
\begin{compactenum}
\item \textsf{view} gives a preview of the notebook, essentially the computation cells without output, pre-rendered for served static cally without a kernel.\ednote{MK: we should implement this; I am not sure what the best way is for this. I guess a build target based on \texttt{https://github.com/jendas1/jupyter-notebook-quick-look}.}
\item \textsf{run/edit} starts the respective notebook  on \url{jupyter.mathhub.info} for running and editing; any changes to the notebook are committed to the Git repository on  \url{gl.mathhub.info} that hosts the notebook. 
\item \textsf{metadata} (this is the tab open in Figure~\ref{fig:mathhub-NB}), shows the metadata provided by the Jupyter kernel and the repository. 
\item \textsf{source} provides access to the document source; here simply a link to \url{gl.mathhub.info}
\item \textsf{statistics} shows statistical information about the notebook, its ephemeral document (see Figure~\ref{fig:test_theory} and the discussion there), and the connections into the background theory graph. 
\item \textsf{graph} links to the a display of the graphs (document graph, declaration graph, and dependency graph) induced by the document and the background knowledge via TGView, a canvas-based in-browser visualizer for knowledge graph information~\cite{RupKohMue:fitgv17}.
\end{compactenum}
This integration combines the interactive features of the Jupyter server with the knowledge management facilities on MathHub.info. In the future, we plan to integrate the notebook diff/patch \textsf{nbdime} developed in OpenDreamKit to extend the kwowledge management facilities. 

\begin{figure}[ht]\centering
  \fbox{\includegraphics[width=13cm]{NB-MathHub}}
  \caption{A Jupyter Notebook in MathHub.info (Metadata)}\label{fig:mathhub-NB}
\end{figure}

A Jupyter Notebook additionally has a special button appears that allows users to open the notebook in the associated Jupyter server.\ednote{MK@KA: does this use the running MMT process on MathHub.info? That would give the kernel access to the whole MathHub universe --> describe that as a feature. KA: No, currently it uses its own MMT}
\ednote{@Kai, @Tom: check this, do the implementation}

\ednote{KA: write some demo notebooks stored in gl.mathhub.info, including the one corresponding to the example in the previous section}

\ednote{KA: show a screenshot of the example notebook from the previous section in jupyter.mathhub.info}


%%% Local Variables:
%%% mode: latex
%%% mode: visual-line
%%% fill-column: 5000
%%% TeX-master: "report"
%%% End:

%  LocalWords:  Jupyter ednote
