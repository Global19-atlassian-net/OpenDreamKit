A comprehensive distributed virtual research environment (VRE) for mathematics as envisioned in the OpenDreamKit project must integrate many disparate existing systems.
Often these are developed independently, for different reasons and by different communities.
In the absence of a joint vision for an overarching VRE, they tend to remain isolated and incompatible.

The OpenDreamKit project has supplied such a joint vision, and this report considers three of these systems: the Jupyter user interface system, the MathHub document hosting system, and the MMT knowledge representation system.
Jupyter offers a uniform interface to the computation facilities of OpenDreamKit systems in the form of a Read-Eval-Print Loop.
MathHub offers versioned persistent storage of mathematical documents based on git.
MMT provides semantics-aware knowledge management for mathematical objects and as such serves as the center of Math-in-the-Middle (MitM) infrastructure developed in OpenDreamKit (see D6.5).
Our ultimate goal is to integrate these three standalone systems as functional units of a VRE: respectively, its frontend, backend, and semantic kernel.

Concretely, we present two practical steps towards this goal.
Firstly, we have designed and implemented a Jupyter kernel for MMT.
That allows using Jupyter as a frontend for MitM operations in MMT.
We also show how Jupyter widgets can be deeply integrated with the MMT knowledge management facilities to give semantics-aware interaction facilities.
Secondly, we have designed how to add a Jupyter server to MathHub.
That allows storing Jupyter Notebooks as MathHub documents.
We also show how to dynamically employ the highly interactive and often ephemeral Jupyter Notebooks as subdocuments of other mathematical documents such as static HTML pages generated from scientific articles.

We evaluate the integration in two case studies.
The first one serves as a stepping stone towards the OpenDreamKit VRE: MitM-based computation via MMT using a Jupyter Notebook that is maintained by MathHub.
The second one applies our results to a concrete problem outside of mathematics: a knowledge-based specification dialog for modeling and simulation.