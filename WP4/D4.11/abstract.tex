\Ednote{NT: Please expand: what are the different components involved (MathHub, Jupyter, MMT), what problem was tackled in this deliverable, and how it fits within the general aims/objectives/strategy of ODK; what as achieved; it should be reasonably comprehensible by any mathematician.}
\Ednote{NT: move this to the github description.}

In this report we present a prototypical integration of the Jupyter notebooks into the MathHub.info portal for active mathematical documents and a versioned hosting system for flexiformal mathematics.
MathHub.info offers a rich interface for reading, writing, and interacting with mathematical documents and knowledge.
Jupyter offers a uniform interface to the computation facilities of the OpenDreamKit VRE in the form of a Read-Eval-Print Loop (REPL). 

After specifying an integration of active documents with REPL facilities (Jupyter) in D4.2 and exploring in-document computation in D4.9 we report on the integration of the back-ends: providing an MMT kernel for Jupyter (knowledge management and Math-in-the-Middle-based distributed computation (see D6.5)).
We show how the new Jupyter widgets can be deeply integrated with the MMT knowledge management facilities to give semantics-aware interaction facilities.

We show and evaluate the integration in two case studies: MitM-based user virtual research environments in Jupyter and a knowledge-based specification dialog for modeling and simulation. 

%%% Local Variables:
%%% mode: latex
%%% mode: visual-line
%%% fill-column: 5000
%%% TeX-master: "report"
%%% End:

%  LocalWords:  Jupyter flexiformal MitM-based
