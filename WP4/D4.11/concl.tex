We have presented an integration of the two front-end components in the OpenDreamKit project:
Jupyter for computation/experimentation and MathHub.info for interactive mathematical documents.
The prerequisite for this integration is progress at the back-end: mainly a MMT kernel for Jupyter:
MMT is the system that drives semantic services in MathHub.info, and the new kernel now makes a tight integration possible at the front-end levels.
We have evaluated the reach of the evaluation on two case studies: A document with in-document computation and MMT-based interview application for knowledge-based acquisition of configurations for simulation software.

The next steps will be to explore the potential of the integration of computational and knowledge-centered approaches to “doing mathematics” and in integrating mathematical data sources like LMFDB.
In practice this will mean developing Jupyter Notebooks that employ the MMT kernel for knowledge management, e.g. from the Math-in-the-Middle (MitM see ~\cite{ODK-D6.8}) ontology, or integrating MMT as the mediator in MitM-based integration paradigm developed in WP6 of the OpenDreamKit Project – see~\cite{ODK-D6.5}.

%%% Local Variables:
%%% mode: latex
%%% mode: visual-line
%%% fill-column: 5000
%%% TeX-master: "report"
%%% End:

%  LocalWords:  ednote Jupyter Jupyter MitM-based
