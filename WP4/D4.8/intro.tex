\section{Introduction}\label{sec:intro}

\ednote{MK: more intro blabla about narration and computation -- can also use this for the
  one of the tetrapod edges.}

We have proposed ``Active Structured Documents'' as a natural interface for interacting
with mathematical knowledge for the working mathematician and thus as a UI component for
the \pn Virtual Math Research Environment Toolkit~\cite{ODK-D4.2}.

In a nutshell, active documents are documents which make aspects of the meaning of their
contents explicit enough that it becomes machine-actionable in a document player that
delivers services -- in our case for computation -- that can be triggered 

In~\cite{KohDavGin:psewads11} we present a system for active documents following the
``active document paradigm'' which defines Active Documents as semantically annotated
documents associated with a content commons that holds the corresponding background
ontologies. An \textit{Active Document Player} embeds user-visible, interactive services
like program execution, computation, visualization, navigation, information aggregation
and information retrieval to make documents executable. We call this framework the Active
Documents Paradigm (ADP; see Figure~\ref{fig:activedocs}), since documents can also
actively adapt to user preferences and environment rather than only executing services
upon user request.
\begin{figure}[ht]\centering
  \documentclass{standalone}
% \usepackage[mh]{mikoslides}
% % this file defines root path local repository
\defpath{MathHub}{/Users/kohlhase/localmh/MathHub}
\mhcurrentrepos{MiKoMH/talks}
\libinput{WApersons}
% we also set the base URI for the LaTeXML transformation
\baseURI[\MathHub{}]{https://mathhub.info/MiKoMH/talks}

% \libinput{preamble}
\usepackage{tikz}
\usetikzlibrary{positioning}
\usetikzlibrary{shapes.geometric}
\usetikzlibrary{shadows}
\usetikzlibrary{patterns}
\usetikzlibrary{arrows}
\usetikzlibrary{backgrounds}
\usetikzlibrary{mmt}
\usetikzlibrary{tikzmark}
\usetikzlibrary{decorations,decorations.markings,decorations.text,decorations.pathmorphing}
\begin{document}
\begin{tikzpicture}
[inner sep=1.5mm, node distance=2cm,outer sep = 0pt,
place/.style={circle,draw=blue!50,fill=blue!20,thick,outer sep = 0pt},
transition/.style={regular polygon,regular polygon sides=4,draw=blue!50,fill=blue!20,thick},
dblock/.style={rectangle, draw, fill=blue!20, 
    text width=4em, text centered, minimum height=3.5em},
wblock/.style={rectangle, draw, fill=blue!20, rounded corners,
    text width=5em, text centered, minimum height=3.5em,outer sep=4pt},
scale=1.4]

%% the diagram, which is also defines the origin
\node (center) at (0,0) [place] {};
\node (middleleft) at (-1.5,0) [transition] {};
\node (middleright) at (1.5,0) [transition] {};

\node (righttop) at ( 1,1) [place] {};
\node (lefttop) at (-1,1) [place] {};b

\node (leftbottom) at (-1,-1) [place] {};
\node (rightbottom) at (1,-1) [place] {};

%% labels
\coordinate (A) at (-2.5,5);
\coordinate (B) at (-2.5,-1.5);
\coordinate [label=left:{Document Commons}] (C) at (-3,4.8);
\coordinate [label=right:{Content Commons}] (D) at (-2,4.8);
\node[text width=4em] at (2.5,1.2) {Content Objects};

%% lines

\draw[-] (center) -- (lefttop);
\draw[-] (center) -- (rightbottom);
\draw[-] (center) -- (leftbottom);
\draw[-] (center) -- (righttop);
\draw[-] (lefttop) -- (righttop);
\draw[-] (leftbottom) -- (rightbottom);
\draw[-] (leftbottom) -- (middleleft);
\draw[-] (middleleft) -- (lefttop);
\draw[-] (middleright) -- (righttop);
\draw[-] (middleright) -- (rightbottom);
\draw[->,dashed] (lefttop.north) to [bend left=45] (righttop.north);
\draw[->,dashed] (leftbottom) -- (lefttop);

\draw[-,dashed] (A) -- (B);

%% active documents and player
\node (activedocs) at (-5,0) [dblock] {Semantic Documents};
\node at (-4.6,0.4) [dblock] {Semantic Documents};
\node at (-4.8,0.2) [dblock] {Semantic Documents};
\node at (-5,0) [dblock] {Semantic Documents};

\node (player) at (-2.5,3.2) [wblock] {Active Document Player};
\node[draw,left=3cm of player] (user) { User };
\draw[<->,draw,dashed] (player) -- node[above]{view} node[below]{interact} (user);

\draw[<->,draw,thick] (-1.5,0) -- (-3.5,0);
\draw[<->,draw,thick] (-4.5,1.1) -- (-3.5,2.5);
\draw[<->,draw,thick] (-.3,1.1) -- (-1.5,2.5);
\end{tikzpicture}
\end{document}

%%% Local Variables:
%%% mode: latex
%%% TeX-master: t
%%% End:

  \caption{Active Documents}\label{fig:activedocs} 
\end{figure} 

The ADP is implemented in the Active Documents Portal MathWeb.info~\cite{MathHub:on}
building on standard components as an instance of the Planetary system
\cite{Kohlhase:ppte12}.

In the long run, we propose to integrate active structured documents with the Jupyter
notebooks, and as a first step into this direction we explore the requirements of
integrating \emph{in-situ} (i.e. in-document) \emph{computation} -- a forte and the indeed
the reason-d'etre of notebooks -- into conventional, narrative-structured mathematical
documents; Section~\ref{sec:examples} presents, analyzes, and classifies examples for
in-situ computation.  We also explore how the active documents technology has to be
extended to accomodate this functionality as a semantic service -- see
Section~\ref{sec:infarch} for details.\ednote{MK: also an implementation section? We have
  to put the screenshots somewhere} Finally Section~\ref{sec:concl} concludes the report
and gives directions of future research and development.

%%% Local Variables:
%%% mode: latex
%%% TeX-master: "report"
%%% End:

%  LocalWords:  sec:intro KohDavGin:psewads11 textit visualization fig:activedocs Jupyter
%  LocalWords:  centering MathHub:on Kohlhase:ppte12 emph emph reason-d'etre sec:examples
%  LocalWords:  analyzes sec:infarch sec:concl
