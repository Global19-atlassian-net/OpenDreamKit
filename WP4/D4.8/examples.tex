\section{Examples}
In the following we will look at some examples to get a feeling for the applications
\subsection{Unit Conversions}\label{sec:units}
\ednote{Ulrich has some text on this}
\subsection{Equations}
We would like to play with equations in documents, e.g. the famous $E=mc^2$, where $E$ is
the energy, $m$ is the mass, and $c$ is the speed of light. There we might be interested
to see what the energy equivalent of 1 gram of matter might be. So we would like to just
right click on the $m$, instantiate it to $1g$ and simplify the expression. Conversely, we
might want to know how many grams of matter it would take to drive from Erlangen to
Paris. So we would instantiate $E$ with $122 kWh$ and solve $122kWh=mc^2$ for
$m$. Actually since we end up with $10^-{???}$ grams we would directly convert the
quantity expression to \ednote{continue}.

\subsection{Computation with Proofs}
\begin{itemize}
  \item calling automated theorem provers on a goal in a document
  \item extending the level of explanation by doing that on a subgoal or deepening the
    level of explanation. E.g. from ``obviously'' to a full proof.  
  \end{itemize}

\subsection{Playing with global/local values}
What would be paper look like if the speed of light were $88 mph$. 

\subsection{Updating Values to current or historical values}
spreadsheets (a well-understood form of active documents) can already do that. 
\begin{itemize}
\item global warming papers with newer models or data
\item Wolfram alpha: ``does it snow in hell?''
\end{itemize}




%%% Local Variables:
%%% mode: latex
%%% TeX-master: t
%%% End:
