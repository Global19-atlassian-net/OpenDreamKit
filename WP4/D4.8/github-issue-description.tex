\section*{\texorpdfstring{Deliverable description, as taken from Github
issue
\href{https://github.com/OpenDreamKit/OpenDreamKit/issues/98}{\#98} on
2017-02-13}{Deliverable description, as taken from Github issue \#98 on 2017-02-13}}\label{deliverable-description-as-taken-from-github-issue-98-on-2017-02-13}

\begin{itemize}
\tightlist
\item
  \textbf{WP4:}
  \href{https://github.com/OpenDreamKit/OpenDreamKit/tree/master/WP4}{User
  Interfaces}
\item
  \textbf{Lead Institution:} Simula Research Laboratory
\item
  \textbf{Due:} 2017-02-28 (month 18)
\item
  \textbf{Nature:} Other
\item
  \textbf{Task:} T4.3
  (\href{https://github.com/OpenDreamKit/OpenDreamKit/issues/71}{\#71})
\item
  \textbf{Final Report:} in progress
\end{itemize}

The \href{https://jupyter.org}{Jupyter Notebook} is a web application
that enables the creation and sharing of executable documents that
contains live code, equations, visualizations and explanatory text.
Thanks to a modular design, Jupyter can be used with any computational
system that provides a so-called
\href{https://jupyter.readthedocs.io/en/latest/projects/kernels.html}{\emph{Jupyter
kernel}} implementing the
\href{https://jupyter-client.readthedocs.io/en/latest/}{\emph{Jupyter
messaging protocol}} to communicate with the notebook. OpenDreamKit
therefore promotes the Jupyter notebook as user interface of choice, in
particular since it is particularly suitable for building modular web
based Virtual Research Environments.

This deliverable aims at enabling testing of Jupyter notebooks, with a
good balance of convenience and configurability to address the range of
possible ways to validate noteboooks. Testing is integral to ODK's goals
of enabling reproducible practices in computational math and science,\\
and this work enables validating notebooks as documentation and
communication products,\\
extending the scope of testing beyond traditional software.

Accomplishments:

\begin{itemize}
\tightlist
\item
  \(\checkmark\) develop
  \href{https://github.com/computationalmodelling/nbval}{nbval} package
  for testing notebooks
\item
  \(\checkmark\) allow multiple testing modes, ranging from lax
  error-checking to strict output comparison
\item
  \(\checkmark\) enable normalizing output for comparison of transient
  values such as memory addresses and dates
\item
  \(\checkmark\) integrate ODK deliverable 4.6 (nbdime,
  \href{https://github.com/OpenDreamKit/OpenDreamKit/issues/95}{\#95})
  for displaying changes between notebooks when they differ
\end{itemize}
