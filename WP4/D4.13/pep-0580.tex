\section{PEP: 580}
\label{pep580}

\begin{description}
\item[Title] The C call protocol
\item[Author] Jeroen Demeyer \textless{}\href{mailto:J.Demeyer@UGent.be}{\nolinkurl{J.Demeyer@UGent.be}}\textgreater{}
\item[Status] Draft
\item[Type] Standards Track
\item[URL] \url{https://www.python.org/dev/peps/pep-0580/}
\item[Content-Type] text/x-rst
\item[Created] 14-Jun-2018
\item[Python-Version] 3.8
\item[Post-History] 20-Jun-2018, 22-Jun-2018, 16-Jul-2018
\end{description}

\subsection{Abstract}

A new "C call" protocol is proposed. It is meant for classes
representing functions or methods which need to implement fast calling.
The goal is to generalize existing optimizations for built-in functions
to arbitrary extension types.

In the reference implementation, this new protocol is used for the
existing classes \texttt{builtin\_function\_or\_method} and
\texttt{method\_descriptor}. However, in the future, more classes may
implement it.

\textbf{NOTE}: This PEP deals only with the Python/C API, it does not
affect the Python language or standard library.

\subsection{Motivation}

Currently, the Python bytecode interpreter has various optimizations for
calling instances of \texttt{builtin\_function\_or\_method},
\texttt{method\_descriptor}, \texttt{method} and \texttt{function}.
However, none of these classes is subclassable. Therefore, these
optimizations are not available to user-defined extension types.

If this PEP is implemented, then the checks for
\texttt{builtin\_function\_or\_method} and \texttt{method\_descriptor}
could be replaced by simply checking for and using the C call protocol.
This simplifies existing code.

We also design the C call protocol such that it can easily be extended
with new features in the future.

For more background and motivation, see PEP 579.

\subsection{Basic idea}

Currently, CPython has multiple optimizations for fast calling for a few
specific function classes. Calling instances of these classes using a
plain \texttt{tp\_call} is slower than using the optimizations. The
basic idea of this PEP is to allow user-defined extension types (not
Python classes) to use these optimizations also, both as caller and as
callee.

The existing class \texttt{builtin\_function\_or\_method} and a few
others use a \texttt{PyMethodDef} structure for describing the
underlying C function and its signature. The first concrete change is
that this is replaced by a new structure \texttt{PyCCallDef}. This
stores some of the same information as a \texttt{PyMethodDef}, but with
one important addition: the "parent" of the function (the class or
module where it is defined). Note that \texttt{PyMethodDef} arrays are
still used to construct functions/methods but no longer for calling
them.

Second, we want that every class can use such a \texttt{PyCCallDef} for
optimizing calls, so the \texttt{PyTypeObject} structure gains a
\texttt{tp\_ccalloffset} field giving an offset to a
\texttt{PyCCallDef\ *} in the object structure and a flag
\texttt{Py\_TPFLAGS\_HAVE\_CCALL} indicating that
\texttt{tp\_ccalloffset} is valid.

Third, since we want to deal efficiently with unbound and bound methods
too (as opposed to only plain functions), we need to handle
\texttt{\_\_self\_\_} too: after the \texttt{PyCCallDef\ *} in the
object structure, there is a \texttt{PyObject\ *self} field. These two
fields together are referred to as a \texttt{PyCCallRoot} structure.

The new protocol for efficiently calling objects using these new
structures is called the "C call protocol".

\subsection{New data structures}

The \texttt{PyTypeObject} structure gains a new field
\texttt{Py\_ssize\_t\ tp\_ccalloffset} and a new flag
\texttt{Py\_TPFLAGS\_HAVE\_CCALL}. If this flag is set, then
\texttt{tp\_ccalloffset} is assumed to be a valid offset inside the
object structure (similar to \texttt{tp\_weaklistoffset}). It must be a
strictly positive integer. At that offset, a \texttt{PyCCallRoot}
structure appears:

\begin{verbatim}
typedef struct {
    PyCCallDef *cr_ccall;
    PyObject   *cr_self;     /* __self__ argument for methods */
} PyCCallRoot;
\end{verbatim}

The \texttt{PyCCallDef} structure contains everything needed to describe
how the function can be called:

\begin{verbatim}
typedef struct {
    uint32_t  cc_flags;
    PyCFunc   cc_func;    /* C function to call */
    PyObject *cc_parent;  /* class or module */
} PyCCallDef;
\end{verbatim}

The reason for putting \texttt{\_\_self\_\_} outside of
\texttt{PyCCallDef} is that \texttt{PyCCallDef} is not meant to be
changed after creating the function. A single \texttt{PyCCallDef} can be
shared by an unbound method and multiple bound methods. This wouldn't
work if we would put \texttt{\_\_self\_\_} inside that structure.

\textbf{NOTE}: unlike \texttt{tp\_dictoffset} we do not allow negative
numbers for \texttt{tp\_ccalloffset} to mean counting from the end.
There does not seem to be a use case for it and it would only complicate
the implementation.

\subsubsection{Parent}

The \texttt{cc\_parent} field (accessed for example by a
\texttt{\_\_parent\_\_} or \texttt{\_\_objclass\_\_} descriptor from
Python code) can be any Python object. For methods of extension types,
this is set to the class. For functions of modules, this is set to the
module. However, custom classes are free to set \texttt{cc\_parent} to
whatever they want, it can also be \texttt{NULL}. It is only used by the
C call protocol if the \texttt{CCALL\_OBJCLASS} flag is set.

The parent serves multiple purposes: for methods of extension types
(more precisely, when the flag \texttt{CCALL\_OBJCLASS} is set), it is
used for type checks like the following:

\begin{verbatim}
>>> list.append({}, "x")
Traceback (most recent call last):
  File "<stdin>", line 1, in <module>
TypeError: descriptor 'append' requires a 'list' object but received a 'dict'
\end{verbatim}

PEP 573 specifies that every function should have access to the module
in which it is defined. It is recommended to use the parent for this,
which works directly for functions of a module. For methods, this works
indirectly through the class, assuming that the class has a pointer to
the module.

The parent would also typically be used to implement
\texttt{\_\_qualname\_\_}. The new C API function
\texttt{PyCCall\_GenericGetQualname()} does exactly that.

\textbf{NOTE}: for functions of modules, the parent is exactly the same
as \texttt{\_\_self\_\_}. However, using \texttt{\_\_self\_\_} for the
module is a quirk of the current implementation: in the future, we want
to allow functions which use \texttt{\_\_self\_\_} in the normal way,
for implementing methods. Such functions can still use
\texttt{cc\_parent} instead to refer to the module.

\subsubsection{Using tp\_print}

We propose to replace the existing unused field \texttt{tp\_print} by
\texttt{tp\_ccalloffset}. Since \texttt{Py\_TPFLAGS\_HAVE\_CCALL} would
\emph{not} be added to \texttt{Py\_TPFLAGS\_DEFAULT}, this ensures full
backwards compatibility for existing extension modules setting
\texttt{tp\_print}. It also means that we can require that
\texttt{tp\_ccalloffset} is a valid offset when
\texttt{Py\_TPFLAGS\_HAVE\_CCALL} is specified: we do not need to check
\texttt{tp\_ccalloffset\ !=\ 0}. In future Python versions, we may
decide that \texttt{tp\_print} becomes \texttt{tp\_ccalloffset}
unconditionally, drop the \texttt{Py\_TPFLAGS\_HAVE\_CCALL} flag and
instead check for \texttt{tp\_ccalloffset\ !=\ 0}.

\subsection{The C call protocol}

We say that a class implements the C call protocol if it has the
\texttt{Py\_TPFLAGS\_HAVE\_CCALL} flag set (as explained above, it must
then set \texttt{tp\_ccalloffset\ \textgreater{}\ 0}). Such a class must
implement \texttt{\_\_call\_\_} as described in this subsection (in
practice, this just means setting \texttt{tp\_call} to
\texttt{PyCCall\_Call}).

The \texttt{cc\_func} field is a C function pointer, which plays the
same role as the existing \texttt{ml\_meth} field of
\texttt{PyMethodDef}. Its precise signature depends on flags. Below are
the possible values for \texttt{cc\_flags\ \&\ CCALL\_SIGNATURE}
together with the arguments that the C function takes. The return value
is always \texttt{PyObject\ *}. The following are analogous to the
existing \texttt{PyMethodDef} signature flags:

\begin{itemize}
\tightlist
\item
  \texttt{CCALL\_VARARGS}:
  \texttt{cc\_func(PyObject\ *self,\ PyObject\ *args)}
\item
  \texttt{CCALL\_VARARGS\ \textbar{}\ CCALL\_KEYWORDS}:
  \texttt{cc\_func(PyObject\ *self,\ PyObject\ *args,\ PyObject\ *kwds)}
\item
  \texttt{CCALL\_FASTCALL}:
  \texttt{cc\_func(PyObject\ *self,\ PyObject\ *const\ *args,\ Py\_ssize\_t\ nargs)}
\item
  \texttt{CCALL\_FASTCALL\ \textbar{}\ CCALL\_KEYWORDS}:
  \texttt{cc\_func(PyObject\ *self,\ PyObject\ *const\ *args,\ Py\_ssize\_t\ nargs,\ PyObject\ *kwnames)}
  (\texttt{kwnames} is either \texttt{NULL} or a non-empty tuple of
  keyword names)
\item
  \texttt{CCALL\_NOARGS}:
  \texttt{cc\_func(PyObject\ *self,\ PyObject\ *unused)} (second
  argument is always \texttt{NULL})
\item
  \texttt{CCALL\_O}: \texttt{cc\_func(PyObject\ *self,\ PyObject\ *arg)}
\end{itemize}

The flag \texttt{CCALL\_FUNCARG} may be combined with any of these. If
so, the C function takes an additional argument as first argument before
\texttt{self}. This argument is used to pass the function object (the
\texttt{self} in \texttt{\_\_call\_\_} but see NOTE below). For example,
we have the following signature:

\begin{itemize}
\tightlist
\item
  \texttt{CCALL\_FUNCARG\ \textbar{}\ CCALL\_VARARGS}:
  \texttt{cc\_func(PyObject\ *func,\ PyObject\ *self,\ PyObject\ *args)}
\end{itemize}

One exception is \texttt{CCALL\_FUNCARG\ \textbar{}\ CCALL\_NOARGS}: the
\texttt{unused} argument is dropped, so the signature becomes

\begin{itemize}
\tightlist
\item
  \texttt{CCALL\_FUNCARG\ \textbar{}\ CCALL\_NOARGS}:
  \texttt{cc\_func(PyObject\ *func,\ PyObject\ *self)}
\end{itemize}

\textbf{NOTE}: in the case of bound methods, it is currently unspecified
whether the "function object" in the paragraph above refers to the bound
method or the original function (which is wrapped by the bound method).
In the reference implementation, the bound method is passed. In the
future, this may change to the wrapped function. Despite this ambiguity,
the implementation of bound methods guarantees that
\texttt{PyCCall\_CCALLDEF(func)} points to the \texttt{PyCCallDef} of
the original function.

\textbf{NOTE}: unlike the existing \texttt{METH\_...} flags, the
\texttt{CCALL\_...} constants do not necessarily represent single bits.
So checking \texttt{(cc\_flags\ \&\ CCALL\_VARARGS)\ ==\ 0} is not a
valid way for checking the signature. There are also no guarantees of
binary compatibility between Python versions for these flags.

\subsubsection{\texorpdfstring{Checking
\_\_\href{name__\%20attribute}{objclass}}{Checking \_\_objclass}}

If the \texttt{CCALL\_OBJCLASS} flag is set and if \texttt{cr\_self} is
NULL (this is the case for unbound methods of extension types), then a
type check is done: the function must be called with at least one
positional argument and the first (typically called \texttt{self}) must
be an instance of \texttt{cc\_parent} (which must be a class). If not, a
\texttt{TypeError} is raised.

\subsubsection{Self slicing}

If \texttt{cr\_self} is not NULL or if the flag \texttt{CCALL\_SELFARG}
is not set in \texttt{cc\_flags}, then the argument passed as
\texttt{self} is simply \texttt{cr\_self}.

If \texttt{cr\_self} is NULL and the flag \texttt{CCALL\_SELFARG} is
set, then the first positional argument is removed from \texttt{args}
and instead passed as first argument to the C function. Effectively, the
first positional argument is treated as \texttt{\_\_self\_\_}. If there
are no positional arguments, \texttt{TypeError} is raised.

This process is called "self slicing" and a function is said to have
self slicing if \texttt{cr\_self} is NULL and \texttt{CCALL\_SELFARG} is
set.

Note that a \texttt{CCALL\_NOARGS} function with self slicing
effectively has one argument, namely \texttt{self}. Analogously, a
\texttt{CCALL\_O} function with self slicing has two arguments.

\subsubsection{Descriptor behavior}

Classes supporting the C call protocol must implement the descriptor
protocol in a specific way. This is required for an efficient
implementation of bound methods: it allows sharing the
\texttt{PyCCallDef} structure between bound and unbound methods. It is
also needed for a correct implementation of
\texttt{\_PyObject\_GetMethod} which is used by the
\texttt{LOAD\_METHOD}/\texttt{CALL\_METHOD} optimization. First of all,
if \texttt{func} supports the C call protocol, then
\texttt{func.\_\_set\_\_} must not be implemented.

Second, \texttt{func.\_\_get\_\_} must behave as follows:

\begin{itemize}
\tightlist
\item
  If \texttt{cr\_self} is not NULL, then \texttt{\_\_get\_\_} must be a
  no-op in the sense that
  \texttt{func.\_\_get\_\_(obj,\ cls)(*args,\ **kwds)} behaves exactly
  the same as \texttt{func(*args,\ **kwds)}. It is also allowed for
  \texttt{\_\_get\_\_} to be not implemented at all.
\item
  If \texttt{cr\_self} is NULL, then
  \texttt{func.\_\_get\_\_(obj,\ cls)(*args,\ **kwds)} (with
  \texttt{obj} not None) must be equivalent to
  \texttt{func(obj,\ *args,\ **kwds)}. In particular,
  \texttt{\_\_get\_\_} must be implemented in this case. Note that this
  is unrelated to self slicing: \texttt{obj} may be passed as
  \texttt{self} argument to the C function or it may be the first
  positional argument.
\item
  If \texttt{cr\_self} is NULL, then
  \texttt{func.\_\_get\_\_(None,\ cls)(*args,\ **kwds)} must be
  equivalent to \texttt{func(*args,\ **kwds)}.
\end{itemize}

There are no restrictions on the object
\texttt{func.\_\_get\_\_(obj,\ cls)}. The latter is not required to
implement the C call protocol for example. It only specifies what
\texttt{func.\_\_get\_\_(obj,\ cls).\_\_call\_\_} does.

For classes that do not care about \texttt{\_\_self\_\_} and
\texttt{\_\_get\_\_} at all, the easiest solution is to assign
\texttt{cr\_self\ =\ Py\_None} (or any other non-NULL value).

\begin{center}\rule{0.5\linewidth}{\linethickness}\end{center}

The C call protocol requires that the function has a
\texttt{\_\_name\_\_} attribute which is of type \texttt{str} (not a
subclass).

Furthermore, this must be idempotent in the sense that getting the
\texttt{\_\_name\_\_} attribute twice in a row must return exactly the
same Python object. This implies that it cannot be a temporary object,
it must be stored somewhere. This is required because
\texttt{PyEval\_GetFuncName} uses a borrowed reference to the
\texttt{\_\_name\_\_} attribute.

\subsubsection{Generic API functions}

This subsection lists the new public API functions or macros dealing with
the C call protocol.

\begin{itemize}
\tightlist
\item
  \texttt{int\ PyCCall\_Check(PyObject\ *op)}: return true if
  \texttt{op} implements the C call protocol.
\end{itemize}

All the functions and macros below apply to any instance supporting the
C call protocol. In other words, \texttt{PyCCall\_Check(func)} must be
true.

\begin{itemize}
\tightlist
\item
  \texttt{PyObject\ *PyCCall\_Call(PyObject\ *func,\ PyObject\ *args,\ PyObject\ *kwds)}:
  call \texttt{func} with positional arguments \texttt{args} and keyword
  arguments \texttt{kwds} (\texttt{kwds} may be NULL). This function is
  meant to be put in the \texttt{tp\_call} slot.
\item
  \texttt{PyObject\ *PyCCall\_FASTCALL(PyObject\ *func,\ PyObject\ *const\ *args,\ Py\_ssize\_t\ nargs,\ PyObject\ *kwds)}:
  call \texttt{func} with \texttt{nargs} positional arguments given by
  \texttt{args{[}0{]}}, \ldots{}, \texttt{args{[}nargs-1{]}}. The
  parameter \texttt{kwds} can be NULL (no keyword arguments), a dict
  with \texttt{name:value} items or a tuple with keyword names. In the
  latter case, the keyword values are stored in the \texttt{args} array,
  starting at \texttt{args{[}nargs{]}}.
\end{itemize}

Macros to access the \texttt{PyCCallRoot} and \texttt{PyCCallDef}
structures:

\begin{itemize}
\tightlist
\item
  \texttt{PyCCallRoot\ *PyCCall\_CCALLROOT(PyObject\ *func)}: pointer to
  the \texttt{PyCCallRoot} structure inside \texttt{func}.
\item
  \texttt{PyCCallDef\ *PyCCall\_CCALLDEF(PyObject\ *func)}: shorthand
  for \texttt{PyCCall\_CCALLROOT(func)-\textgreater{}cr\_ccall}.
\item
  \texttt{PyCCallDef\ *PyCCall\_FLAGS(PyObject\ *func)}: shorthand for
  \texttt{PyCCall\_CCALLROOT(func)-\textgreater{}cr\_ccall-\textgreater{}cc\_flags}.
\item
  \texttt{PyObject\ *PyCCall\_SELF(PyOject\ *func)}: shorthand for
  \texttt{PyCCall\_CCALLROOT(func)-\textgreater{}cr\_self}.
\end{itemize}

Generic getters, meant to be put into the \texttt{tp\_getset} array:

\begin{itemize}
\tightlist
\item
  \texttt{PyObject\ *PyCCall\_GenericGetParent(PyObject\ *func,\ void\ *closure)}:
  return \texttt{cc\_parent}. Raise \texttt{AttributeError} if
  \texttt{cc\_parent} is NULL.
\item
  \texttt{PyObject\ *PyCCall\_GenericGetQualname(PyObject\ *func,\ void\ *closure)}:
  return a string suitable for using as \texttt{\_\_qualname\_\_}. This
  uses the \texttt{\_\_qualname\_\_} of \texttt{cc\_parent} if possible.
  It also uses the \texttt{\_\_name\_\_} attribute.
\end{itemize}

\subsubsection{Profiling}

The profiling events \texttt{c\_call}, \texttt{c\_return} and
\texttt{c\_exception} are only generated when calling actual instances
of \texttt{builtin\_function\_or\_method} or
\texttt{method\_descriptor}. This is done for simplicity and also for
backwards compatibility (such that the profile function does not receive
objects that it does not recognize). In a future PEP, we may extend
C-level profiling to arbitrary classes implementing the C call protocol.

\subsection{Changes to built-in functions and methods}

The reference implementation of this PEP changes the existing classes
\texttt{builtin\_function\_or\_method} and \texttt{method\_descriptor}
to use the C call protocol. In fact, those two classes are almost
merged: the implementation becomes very similar, but they remain
separate classes (mostly for backwards compatibility). The
\texttt{PyCCallDef} structure is simply stored as part of the object
structure. Both classes use \texttt{PyCFunctionObject} as object
structure. This is the new layout for both classes:

\begin{verbatim}
typedef struct {
    PyObject_HEAD
    PyCCallDef  *m_ccall;
    PyObject    *m_self;         /* Passed as 'self' arg to the C function */
    PyCCallDef   _ccalldef;      /* Storage for m_ccall */
    PyObject    *m_name;         /* __name__; str object (not NULL) */
    PyObject    *m_module;       /* __module__; can be anything */
    const char  *m_doc;          /* __text_signature__ and __doc__ */
    PyObject    *m_weakreflist;  /* List of weak references */
} PyCFunctionObject;
\end{verbatim}

For functions of a module and for unbound methods of extension types,
\texttt{m\_ccall} points to the \texttt{\_ccalldef} field. For bound
methods, \texttt{m\_ccall} points to the \texttt{PyCCallDef} of the
unbound method.

\textbf{NOTE}: the new layout of \texttt{method\_descriptor} changes it
such that it no longer starts with \texttt{PyDescr\_COMMON}. This is
purely an implementation detail and it should cause few (if any)
compatibility problems.

\subsubsection{C API functions}

The following function is added (also to the stable ABI\footnote{Löwis,
  PEP 384 -- Defining a Stable ABI,
  \url{https://www.python.org/dev/peps/pep-0384/}}):

\begin{itemize}
\tightlist
\item
  \texttt{PyObject\ *\ PyCFunction\_ClsNew(PyTypeObject\ *cls,\ PyMethodDef\ *ml,\ PyObject\ *self,\ PyObject\ *module,\ PyObject\ *parent)}:
  create a new object with object structure \texttt{PyCFunctionObject}
  and class \texttt{cls}. The entries of the \texttt{PyMethodDef}
  structure are used to construct the new object, but the pointer to the
  \texttt{PyMethodDef} structure is not stored. The flags for the C call
  protocol are automatically determined in terms of
  \texttt{ml-\textgreater{}ml\_flags}, \texttt{self} and
  \texttt{parent}.
\end{itemize}

The existing functions \texttt{PyCFunction\_New},
\texttt{PyCFunction\_NewEx} and \texttt{PyDescr\_NewMethod} are
implemented in terms of \texttt{PyCFunction\_ClsNew}.

The undocumented functions \texttt{PyCFunction\_GetFlags} and
\texttt{PyCFunction\_GET\_FLAGS} are removed because it would be
non-trivial to support them in a backwards-compatible way.

\subsection{Inheritance}

Extension types inherit the type flag \texttt{Py\_TPFLAGS\_HAVE\_CCALL}
and the value \texttt{tp\_ccalloffset} from the base class, provided
that they implement \texttt{tp\_call} and \texttt{tp\_descr\_get} the
same way as the base class. Heap types never inherit the C call protocol
because that would not be safe (heap types can be changed dynamically).

\subsection{Performance}

This PEP should not impact the performance of existing code (in the
positive or negative sense). It is meant to allow efficient new code to
be written, not to make existing code faster.

Here are a few pointers to the \texttt{python-dev} mailing list where
performance improvements are discussed:

\begin{itemize}
\tightlist
\item
  \url{https://mail.python.org/pipermail/python-dev/2018-July/154571.html}
\item
  \url{https://mail.python.org/pipermail/python-dev/2018-July/154740.html}
\item
  \url{https://mail.python.org/pipermail/python-dev/2018-July/154775.html}
\end{itemize}

\subsection{Stable ABI}

None of the functions, structures or constants dealing with the C call
protocol are added to the stable ABI\footnote{Löwis, PEP 384 -- Defining
  a Stable ABI, \url{https://www.python.org/dev/peps/pep-0384/}}.

There are two reasons for this: first of all, the most useful feature of
the C call protocol is probably the \texttt{METH\_FASTCALL} calling
convention. Given that this is not even part of the public API (see also
PEP 579, issue 6), it would be strange to add anything else from the C
call protocol to the stable ABI.

Second, we want the C call protocol to be extensible in the future. By
not adding anything to the stable ABI, we are free to do that without
restrictions.

\subsection{Backwards compatibility}

There is no difference at all for the Python interface, nor for the
documented C API (in the sense that all functions remain supported with
the same functionality).

The removed function \texttt{PyCFunction\_GetFlags}, is officially part
of the stable ABI\footnote{Löwis, PEP 384 -- Defining a Stable ABI,
  \url{https://www.python.org/dev/peps/pep-0384/}}. However, this is
probably an oversight: first of all, it is not even documented. Second,
the flag \texttt{METH\_FASTCALL} is not part of the stable ABI but it is
very common (because of Argument Clinic). So, if one cannot support
\texttt{METH\_FASTCALL}, it is hard to imagine a use case for
\texttt{PyCFunction\_GetFlags}. The fact that
\texttt{PyCFunction\_GET\_FLAGS} and \texttt{PyCFunction\_GetFlags} are
not used at all by CPython outside of \texttt{Objects/call.c} further
shows that these functions are not particularly useful.

Concluding: the only potential breakage is with C code which accesses
the internals of \texttt{PyCFunctionObject} and
\texttt{PyMethodDescrObject}. We expect very few problems because of
this.

\subsection{Rationale}

\subsubsection{Why is this better than PEP 575?}

One of the major complaints of PEP 575 was that is was coupling
functionality (the calling and introspection protocol) with the class
hierarchy: a class could only benefit from the new features if it was a
subclass of \texttt{base\_function}. It may be difficult for existing
classes to do that because they may have other constraints on the layout
of the C object structure, coming from an existing base class or
implementation details. For example, \texttt{functools.lru\_cache}
cannot implement PEP 575 as-is.

It also complicated the implementation precisely because changes were
needed both in the implementation details and in the class hierarchy.

The current PEP does not have these problems.

\subsubsection{Why store the function pointer in the instance?}

The actual information needed for calling an object is stored in the
instance (in the \texttt{PyCCallDef} structure) instead of the class.
This is different from the \texttt{tp\_call} slot or earlier attempts at
implementing a \texttt{tp\_fastcall} slot\footnote{Add tp\_fastcall to
  PyTypeObject: support FASTCALL calling convention for all callable
  objects, \url{https://bugs.python.org/issue29259}}.

The main use case is built-in functions and methods. For those, the C
function to be called does depend on the instance.

Note that the current protocol makes it easy to support the case where
the same C function is called for all instances: just use a single
static \texttt{PyCCallDef} structure for every instance.

\subsubsection{Why CCALL\_OBJCLASS?}

The flag \texttt{CCALL\_OBJCLASS} is meant to support various cases
where the class of a \texttt{self} argument must be checked, such as:

\begin{verbatim}
>>> list.append({}, None)
Traceback (most recent call last):
  File "<stdin>", line 1, in <module>
TypeError: append() requires a 'list' object but received a 'dict'

>>> list.__len__({})
Traceback (most recent call last):
  File "<stdin>", line 1, in <module>
TypeError: descriptor '__len__' requires a 'list' object but received a 'dict'

>>> float.__dict__["fromhex"](list, "0xff")
Traceback (most recent call last):
  File "<stdin>", line 1, in <module>
TypeError: descriptor 'fromhex' for type 'float' doesn't apply to type 'list'
\end{verbatim}

In the reference implementation, only the first of these uses the new
code. The other examples show that these kind of checks appear in
multiple places, so it makes sense to add generic support for them.

\subsubsection{Why CCALL\_SELFARG?}

The flag \texttt{CCALL\_SELFARG} and the concept of self slicing are
needed to support methods: the C function should not care whether it is
called as unbound method or as bound method. In both cases, there should
be a \texttt{self} argument and this is simply the first positional
argument of an unbound method call.

For example, \texttt{list.append} is a \texttt{METH\_O} method. Both the
calls \texttt{list.append({[}{]},\ 42)} and \texttt{{[}{]}.append(42)}
should translate to the C call \texttt{list\_append({[}{]},\ 42)}.

Thanks to the proposed C call protocol, we can support this in such a
way that both the unbound and the bound method share a
\texttt{PyCCallDef} structure (with the \texttt{CCALL\_SELFARG} flag
set).

Concluding, \texttt{CCALL\_SELFARG} has two advantages: there is no
extra layer of indirection for calling and constructing bound methods
does not require setting up a \texttt{PyCCallDef} structure.

\subsubsection{Replacing tp\_print}

We repurpose \texttt{tp\_print} as \texttt{tp\_ccalloffset} because this
makes it easier for external projects to backport the C call protocol to
earlier Python versions. In particular, the Cython project has shown
interest in doing that (see
\url{https://mail.python.org/pipermail/python-dev/2018-June/153927.html}).

\subsection{Alternative suggestions}

PEP 576 is an alternative approach to solving the same problem as this
PEP. See
\url{https://mail.python.org/pipermail/python-dev/2018-July/154238.html}
for comments on the difference between PEP 576 and PEP 580.

\subsection{Reference implementation}

The reference implementation can be found at
\url{https://github.com/jdemeyer/cpython/tree/pep580}

\subsection{Copyright}

This document has been placed in the public domain.
