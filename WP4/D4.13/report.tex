\documentclass{deliverablereport}

\deliverable{UI}{ipython-kernels}
\deliverydate{31/08/2018}
\duedate{31/08/2017 (M24)}
\author{Jeroen Demeyer}

\begin{document}
\maketitle
\tableofcontents

%%%%%%%%%%%%%%%%%%%%%%%%%%%%%%%%%%%%%%%%%%%%%%%%%%%%%%%%%%%%%%%%%%%%%%%%

\section{Introduction}

Sage uses Sphinx to build its documentation.
This is a standard tool which is used by many different Python projects.
It automatically produces organised documentation from docstrings
(strings in the code containing documentation).

For various reasons, Sage uses a very customised version of Sphinx.
This is maintainable in the long term.
The huge size of the Sage documentation also gives
some problems: building the documentation requires a lot of time and memory.

Broadly speaking, the two main goals of this deliverable are
making Sage use a standard Sphinx setup
and making it more efficient in terms of resource usage.

Work on this deliverable is progressing quite slowly.
For a large part, this is because many projects are involved:
Sage, Sphinx, Docutils, Pygments, Cython and Python.

For that reason, we plan to continue working on it until month 48.
There is manpower available at Ghent University and Universit\'e Paris-Sud to do that.
This report gives the current progress
as well as further work to do in the final year of the OpenDreamKit project.

\section{Removing Sage customisations}

Sage uses a highly customised version of Sphinx.
The word ``customised'' can mean any of the following:
it may be an actual patch to the source code,
a monkey-patch (a patch at runtime),
a fork of some plug-in or just a highly specialised configuration.

One of the goals of this deliverable is to remove these
customisations, allowing Sage to use a standard Sphinx.
This can be achieved in two ways:
either patch upstream Sphinx to make the feature standard,
or change Sage (or some other project like Cython or Python)
to remove the need for the customisation.
We now describe the customisations in more detail.
This list is not exhaustive, there are many more small
customisations.

\subsection{Support for Cython functions}

Sage uses a combination of Python and Cython modules.
Cython is a programming language which mostly uses the Python syntax,
but which is compiled to C instead of interpreted like Python.
By default, functions in Cython become Python \emph{built-in} functions.
These are a special kind of optimised functions, implemented in C.

Unfortunately, built-in functions do not work well with Sphinx.
In particular, Sphinx cannot determine the arguments of such functions.
Sage and Cython have various hacks to make it work anyway.
These are quite ugly, so it would not make much sense to support those in Sphinx.

Fixing this properly requires adding support in Python
for fast user-defined function classes.
For user-defined function classes, Cython can add whatever attributes
are needed to support Sphinx.
The problem is that such user-defined function classes
are necessarily slower than built-in functions.
This is not acceptable for Sage where speed is very important.
In June 2018, after a few months of discussions with the Python community,
we submitted PEP 580, a Python Enhancement Proposal to fix this.
Unfortunately, the process of getting a PEP accepted is quite slow:
at the time of writing this report, the PEP had not been accepted (nor rejected).

If PEP 580 is accepted, Cython can be changed to produce
functions which are equally fast as built-in functions
and which support all documentation
features that standard Python functions support.

\subsection{Build the documentation in pieces}

The Sage reference manual is not built as one monolithic document,
but in several pieces:
one for each mathematical topic such as combinatorics or matrices.
The are two reasons for this: it makes it easy to build the various pieces
in parallel and it lowers the memory requirements.
Unfortunately, these pieces are not completely independent:
we still want to end up with one consistent document,
so things like the index and references need to merged.
This is implemented in the Sage \texttt{multidocs} extension for Sphinx.

Since it would be simpler to build the reference manual as one document,
we should work on removing the disadvantages mentioned here.
In particular, the parallellisation should no longer be an issue
(see next section).
Once the disadvantages are gone, we can just build the reference manual
as a single document again.
This is future work.

\subsection{Parallel documentation build}

One of the advantages of building the reference manual in pieces
(see previous section) is that we can build those pieces in parallel.
However, Sphinx itself now supports building in parallel
(this was not the case at the time when Sage started its own parallel docbuilder).

So instead of rolling our own parallel docbuilder,
we should use the Sphinx parallel builder.
Given that the feature already exists, this is easy to do.
But it can only be done after we have removed the existing
splitting and parallel machinery.

\subsection{Custom syntax highlighting}

This is about a relatively small patch to Pygments
(a package for syntax highlighting).
The patch allows to highlight \texttt{sage:} the same way
as standard Python \texttt{>>>} prompts.

No work has been done towards removing this patch,
but it should be relatively simple.
We should investigate whether we can customize Pygments without
patching it.
Alternatively, support for Sage should be pushed upstream to Pygments.

\section{High resource usage}

Building the full Sage documentation takes a lot of resources:
it needs about 3 gigabytes of memory and it needs about 1.5 hours
on an ordinary computer to build the documentation from scratch.
This makes it the most resource-intensive part of the
Sage distribution build.
Because of this, Sage users are often suggested to skip building
the Sage documentation.

We have investigated where the high memory usage comes from.
It turns out that it is a combination of factors
and that there is not a single obvious cause.
Some of it is caching of data which is no longer needed,
which is easy to fix.
Some of it is data structures using basic Python primitives such as a \texttt{dict}
instead of more specialized (and space-efficient) structures.

The Sage customizations to Sphinx
(in particular the splitting in pieces and parallellization)
may cloud this analysis somewhat,
as they might make things better or worse.
Therefore, it would make most sense to tackle this problem after
removing those customizations.

\end{document}
