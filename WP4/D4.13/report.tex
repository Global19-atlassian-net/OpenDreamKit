\documentclass{deliverablereport}

\deliverable{UI}{sage-sphinx}
\deliverydate{31/08/2018}
\duedate{31/08/2017 (M24)}
\author{Jeroen Demeyer}

\begin{document}
\maketitle
\tableofcontents

%%%%%%%%%%%%%%%%%%%%%%%%%%%%%%%%%%%%%%%%%%%%%%%%%%%%%%%%%%%%%%%%%%%%%%%%

\section{Introduction}

The \Sage computational system, like \Python and many other Python
projects, uses the \Sphinx documentation system. Similar in principle
to \lstinline{javadoc} or \lstinline{doxygen}, it generates structured
documentation in various formats (HTML, PDF, ...).
The documentation is generated either from \texttt{.rst} files
or from so-called \emph{docstrings}, that is snippets of documentation
interspersed in the code it documents.

One specific challenge is scaling to the sheer size of the \Sage
documentation (13k pages just for the reference manual): building it
requires a lot of time and memory. In addition to those particularly
stringent needs and a long legacy, \Sage was (at the time of writing
the proposal) using an outdated version of \Sphinx, crippled by many
layers of customisation and adaptations that had accumulated over the
years. This was becoming impossible to maintain.

\taskref{UI}{sage-sphinx} tackles this situation with, broadly
speaking, two main goals:
\begin{enumerate}
\item a critically needed deep refactorisation to
  foster sustainability by outsourcing back to
  \Sphinx all generic aspects (parallel compilation, index generation,
  etc);
\item optimizing resource usage for the documentation build.
\end{enumerate}

Although a lot has already been achieved, progress has been
significantly slower than originally planned for. In large part, this
is because many upstream projects are involved: \Sphinx, \Docutils,
\Pygments, \Cython, and \Python. We are thus dependent on their own
independent schedule whenever an upstream contribution is needed. Such
a situation was foreseen in our Risk Management Plan and its impact on
the overall project is negligible: on the one hand, no other task
depends on \taskref{UI}{sage-sphinx}; on the other hand, it was easy to
shuffle around some of our internal work plan, and reserve FTEs
for this task during OpenDreamKit's final year and at Ghent University
and Université Paris-Sud. Also, knowing that this was a long term
endeavour and that the timeline was not critical, we decided on
several occasions to invest in long term solutions that would benefit
not only \Sage but the community as a whole (see \S\,\ref{cython}).

We report here on the current achievements, and detail the work plan for the
final year. Formally speaking, this deliverable report concerns solely the
first goal; however, as the two goals are closely interrelated, reporting on
both is required to explain the big picture.

\section{Removing \Sage customisations}

\Sage uses a highly customised version of \Sphinx.
The word ``customised'' can mean any of the following:
an actual patch to the source code,
a monkey-patch (a patch at runtime),
a fork of some plug-in or just a highly specialised configuration.

One of the goals of this deliverable is to remove these
customisations, allowing \Sage to use a standard \Sphinx.
This can be achieved in two ways:
either patch upstream \Sphinx to make the feature standard,
or change \Sage (or some other project like \Cython or \Python)
to remove the need for the customisation.

This has multiple advantages:
first of all, having less custom code makes \Sage more maintainable.
Second, it helps packaging of \Sage
(see \taskref{component-architecture}{mod-packaging}):
packagers know how to deal with \Sphinx
but the non-standard \Sage stuff makes it more difficult for them.
Third, if we push fixes upstream, then everybody (not just \Sage)
can benefit from them.

\subsection{General clean up and minor fixes}

We did quite a lot of minor clean up.
This is hard to list in detail.
In particular, various fixes were made to the build system for the documentation,
including using a more standard Sphinx configuration, as well as
removing a several irrelevant customisations to \texttt{autodoc},
the extension for generating documentation from docstrings.
We also made building the \Sage documentation a lot more reliable:
in the past, it often failed to build after an upgrade
but this is no longer the case.

\subsection{Support for Cython functions}\label{cython}

\Sage uses a combination of Python and Cython modules.
Cython is a programming language which mostly uses the Python syntax,
but which is compiled to C instead of interpreted like Python.
By default, functions in Cython become Python \emph{built-in} functions.
These are a special kind of optimised function, implemented in C.

Unfortunately, built-in functions do not work well with \Sphinx, as they
do not have all the needed introspection capabilities.
In particular, \Sphinx cannot determine the input arguments of such functions.
Since the arguments are an important part of the documentation
of a function, this is a big problem.
\Sage and \Cython have various hacks to make it work anyway.
These are quite ugly, so it would not make much sense to support those in \Sphinx.

Fixing this properly requires adding support in \Python
for fast user-defined function classes.
For user-defined function classes, \Cython can add whatever attributes
are needed to support \Sphinx and other use cases that require deep introspection
of functions.
The problem is that such user-defined function classes
are necessarily slower than built-in functions.
This is not acceptable for \Sage where speed is very important.
In June 2018, we submitted PEP 580, a Python Enhancement Proposal to fix this.
At the May 2018 OpenDreamKit workshop in Cernay,
we already discussed an earlier variant of the PEP
with the core \Cython developers and they were very positive about it.
Both before and after submitting the PEP,
we discussed it online with the Python community,
leading to substantial improvements.
Unfortunately, the process of getting a PEP accepted is quite slow:
at the time of writing this report, the PEP had not been accepted (nor rejected).

If PEP 580 is accepted, \Cython can be changed to produce
functions which are equally fast as built-in functions
and which support all documentation
features that standard Python functions support.

\subsection{Build the documentation in pieces}\label{pieces}

The \Sage reference manual is not built as one monolithic document,
but in several pieces:
one for each mathematical topic such as combinatorics or matrices.
The original motivations were twofold: it makes it easy to build the various pieces
in parallel and it lowers the memory requirements.
Unfortunately, these pieces are not completely independent:
we still want to end up with one consistent document,
so things like the index and references need to merged.
This is implemented in the \Sage \texttt{multidocs} extension for \Sphinx.

Since it would be simpler to build the reference manual as a single document,
we should work on removing the disadvantages mentioned here.
In particular, the parallellisation (see \S\,\ref{parallel})
should then no longer be an issue.
Once the disadvantages are gone, we can just build the reference manual
as a single document again.
This is future work.

\subsection{Parallel documentation build}\label{parallel}

One of the advantages of building the reference manual in pieces
(see \S\,\ref{pieces}) is that we can build those pieces in parallel.
However, \Sphinx itself now supports building in parallel.
This was not the case at the time when \Sage
implemented its own parallel docbuilder.

So instead of rolling our own parallel docbuilder,
we should use the \Sphinx parallel builder.
Given that the feature already exists, this is easy to do.
But it can only be done after we have removed the existing
splitting and parallel machinery.

At the Sage Days 77 workshop organized by OpenDreamKit,
we managed to create a proof-of-concept implementation
to show that this can indeed be done.

\subsection{Custom syntax highlighting}

This is about a relatively small patch to \Pygments
(a package for syntax highlighting).
The patch allows to highlight \texttt{sage:} the same way
as standard Python \texttt{>>>} prompts.

No work has been done towards removing this patch,
but it should be rather simple.
We should investigate whether we can customize \Pygments without
patching it.
Alternatively, support for \Sage should be pushed upstream to \Pygments.

\section{Upgrading \Sphinx}

In tandem with removing customisations, it is obvious that we
should use the most recent release of \Sphinx.
So we are indeed regularly upgrading the version of \Sphinx
that is packaged with \Sage.
This is not as trivial as it sounds:
some of the custom code integrates tightly with \Sphinx
and must be regularly changed after an upgrade of \Sphinx.
So, allowing smoother upgrades is yet another reason for
removing the customisations mentioned in the previous section.

\section{High resource usage}

Building the full \Sage documentation takes a lot of resources:
it needs about 3 gigabytes of memory and about 1.5 hours
on an averge personal computer to build the documentation from scratch.
This makes it the most resource-intensive part of
building the complete \Sage computational system.
Because of this, \Sage users who wish to build from source are often
suggested to skip building the \Sage documentation.
It is also a problem for automatic builds in continuous integration
setups, which typically only have limited resources.

At the Sage Days 77 workshop organized by OpenDreamKit,
we have investigated where the high memory usage comes from.
We did this together with Robert Lehmann, a \Sphinx developer.
It turns out that it is a combination of factors
and that there is not a single obvious cause.
Some of it is caching of data which is no longer needed,
which is easy to fix.
Some of it is data structures using basic Python primitives such as a \texttt{dict}
instead of more specialized (and space-efficient) structures.

The \Sage customizations to \Sphinx
(in particular what is mentioned in \S\,\ref{pieces} and \S\,\ref{parallel})
may cloud this analysis somewhat
as they might make things better or worse.
Therefore, it would make most sense to tackle this problem after
removing or at least reducing those customizations.

Fixing this will need to be tackled at the level
of \Docutils or \Sphinx.
Therefore, this will benefit all projects using \Sphinx.

\section{Related work}

We created the package ThebeLab, which uses the Jupyter protocol
to turn the static examples in the documentation live:
the user can edit them and see the output change.
See \delivref{UI}{ipython-kernels}, where this is reported in detail.

It is beneficial for \Sage packages to use the same theme
and configuration for their documentation as \Sage.
We simplified the process by implementing the \verb/sage-package/ utility
(which contains among other things a copy of the Sphinx configuration for \Sage).
The usage of this is illustrated in \verb/sage_sample/,
a template for \Sage packages.
We are investigating how to avoid the aforementioned copy.
In any case, it is our goal to use as much as possible a standard \Sphinx setup,
so this should be limited to some basic configuration and style options.

Finally, we mention two solutions to turn documentation into \Jupyter notebooks:
there is a stand-alone \verb/rst2ipynb/ program for converting a single
\verb/.rst/ file.
There is also a \Sphinx plugin \verb/sphinxcontrib-jupyter/ to which OpenDreamKit
contributed.
This integrates directly with \Sphinx, operating on a collection
of documents instead of a single document.
As such, it supports crosslinks, which makes it a lot more suitable
for converting the Sage manuals, which consist of many documents.
We are currently in the process of adding support for this
in the Sage docbuilder, see \url{https://trac.sagemath.org/ticket/25909}

\end{document}
