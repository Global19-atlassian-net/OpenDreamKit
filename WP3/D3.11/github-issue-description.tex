\hypertarget{deliverable-description-as-taken-from-github-issue-60-on-2019-09-04}{%
\section*{\texorpdfstring{Deliverable description, as taken from Github
issue
\href{https://github.com/OpenDreamKit/OpenDreamKit/issues/60}{\#60} on
2019-09-04}{Deliverable description, as taken from Github issue \#60 on 2019-09-04}}\label{deliverable-description-as-taken-from-github-issue-60-on-2019-09-04}}

\begin{itemize}
\tightlist
\item
  \textbf{WP3:}
  \href{https://github.com/OpenDreamKit/OpenDreamKit/tree/master/WP3}{Component
  Architecture}
\item
  \textbf{Lead Institution:} Univ. Grenoble Alpes
\item
  \textbf{Due:} 2019-08-31 (month 48)
\item
  \textbf{Nature:} Other
\item
  \textbf{Task:} T3.5
  (\href{https://github.com/OpenDreamKit/OpenDreamKit/issues/54}{\#54})
\item
  \textbf{Proposal:}
  \href{https://github.com/OpenDreamKit/OpenDreamKit/raw/master/Proposal/proposal-www.pdf}{p.
  42}
\item
  \textbf{\href{https://github.com/OpenDreamKit/OpenDreamKit/raw/master/WP3/D3.11/report-final.pdf}{Final
  report}}
  (\href{https://github.com/OpenDreamKit/OpenDreamKit/raw/master/WP3/D3.11/}{sources})
\end{itemize}

The primary use of computational mathematics software is to perform
experimental mathematics, for example for testing a conjecture on as
many as possible instances of size as large as possible. In this
perspective, users seek for computational efficiency and the ability to
harness the power of a variety of modern architectures. This is
particularly relevant in the context of the OpenDreamKit Virtual
Research Environment toolkit which is meant to reduce entry barriers by
providing a uniform user experience from multicore personal computers
-\/- a most common use case -\/- to high-end servers or even clusters.
Hence, in the realm of this project, we use the term High Performance
Computing (HPC) in a broad sense, covering all the above architectures
with appropriate parallel paradigms (SIMD, multiprocessing, distributed
computing, etc).

Work Package 5 has resulted in either enabling or drastically enhancing
the high performance capabilities of several computational mathematics
systems, namely \texttt{Singular} (D5.13
\href{https://github.com/OpenDreamKit/OpenDreamKit/issues/111}{\#111}),
\texttt{GAP} (D5.15
\href{https://github.com/OpenDreamKit/OpenDreamKit/issues/113}{\#113})
and \texttt{PARI} (D5.16,
\href{https://github.com/OpenDreamKit/OpenDreamKit/issues/114}{\#114}),
or of the dedicated library \texttt{LinBox} (D5.12
\href{https://github.com/OpenDreamKit/OpenDreamKit/issues/110}{\#110},
D5.14
\href{https://github.com/OpenDreamKit/OpenDreamKit/issues/112}{\#112}).

Bringing further HPC to a general purpose computational mathematics
system such as \texttt{SageMath} is particularly challenging; indeed,
they need to cover a broad -\/- if not exhaustive -\/- range of
features, in a high level and user friendly environment, with
competitive performance. To achieve this, they are composed from the
ground up as integrated systems that take advantage of existing highly
tuned dedicated libraries or sub-systems such as aforementioned.

Were report here on the exploratory work carried out in Task 3.5
(\href{https://github.com/OpenDreamKit/OpenDreamKit/issues/54}{\#54}) to
expose HPC capabilities of components to the end-user level of an
integrated system such as \texttt{SageMath}.

Our first test bed is the \texttt{LinBox} library. Its multicore
parallelism features have been successfully integrated in \texttt{Sage},
with a simple API letting the user control the desired level of
parallelism. We demonstrate the efficiency of the composition with
experiments. Going beyond expectations, the outcome has been integrated
in the next production release of \texttt{SageMath}, hence immediately
benefiting thousands of users.

We proceed by detailing the unique challenges posed by each of the
\texttt{Singular}, \texttt{PARI}, and \texttt{GAP} systems. The common
cause is that they were created decades ago as standalone systems,
represent hundreds of man-years of development, and were only recently
redesigned to be usable as a parallel libraries. Some successes were
nevertheless obtained in experimental setups and pathways to production
are discussed.

We conclude with lessons learned at the occasion of this work and
through expertise sharing within and beyond OpenDreamKit: levels of
integration one may wish for when composing parallel computational
mathematics software. and challenges such integration would raise.
