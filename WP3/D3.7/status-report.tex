\hypertarget{status-report}{%
\section{Status report}\label{status-report}}

One of the main tasks of WP3 is concerned with portability of the
software components of OpenDreamKit. Most components are developed in
UNIX environments (Linux and/or OS X), and porting them to other
platforms, most importantly the Windows operating system, is usually a
challenge.

Although UNIX-like systems are popular among open source software
developers and some academics, the desktop and laptop market share of
Windows computers is estimated to be more than 75\% and is an important
source of potential users, especially students.

Some OpenDreamKit components have been supporting Windows for a long
time. However Windows support for SageMath, the largest of them, has
been elusive for more than 10 years. This is particularly challenging,
not so much because of the Sage Python library (which has some, but
relatively little system-specific code). Rather, the challenge is in
porting all of SageMath's 150+ standard dependencies, and ensuring that
they integrate well on Windows, with full Sage's test suite fully passing.

Thus, finally bringing Windows support to the SageMath distribution has
the added benefit of producing working Windows versions of all the CAS's
and other software SageMath depends on, such as GAP, Singular, etc.

We are happy to report that, thanks to OpenDreamKit's efforts, and in
particular thanks to Erik Bray's work at Université Paris-Sud, starting from
version 8.0 of SageMath, released on July 21, 2017, a native Windows 64 bit
installer based on Cygwin has been made available for all users to
download\footnote{\url{https://www.sagemath.org/download-windows.html}}.
This is now the recommended way to install SageMath on Windows
platforms.

Originally, the deliverable also had the goal of delivering an installer
for Windows 32-bit based on Cygwin. Unfortunately, this goal was not
achieved due to technical obstacles, that we shall briefly explain at
the end of this deliverable. Fortunately, the market share for Windows
32-bit is ever shrinking, and we estimate that the lack of a 32-bit
installer has a very low impact on SageMath's user base.

\hypertarget{description-of-the-achievements}{%
\subsection{Description of the
achievements}\label{description-of-the-achievements}}

As of SageMath version 8.0, Sage is available for 64-bit versions of
Windows 7 and up. It can be downloaded through the SageMath website, and
up-to-date installation instructions are being developed at the SageMath
wiki\footnote{\url{https://wiki.sagemath.org/SageWindows}}.

The installer contains all software and documentation making up the standard
Sage distribution, all libraries needed for Cygwin support, a Bash shell,
numerous standard UNIX command-line utilities, and the MinTTY terminal
emulator, which is generally more user-friendly and better suited for
Cygwin/UNIX software than the standard Windows console.

It is distributed in the form of a single-file executable installer, with a
familiar install wizard interface. The download size of the installer comes in
at just under a gigabyte, but unpacks to more than 4.5 GB in version 8.0.

Because of the large number of files comprising the complete SageMath
distribution, and the heavy compression of the installer, installation can take
a fair amount of time even on a recent system. On an Intel i7 laptop it takes
about ten minutes, but results will vary. Fortunately, after nearly a year of
field testing, this has not yet been a source of complaints.  In future updates
to the installer we may also be able to reduce the installation time/size by
making certain large components (e.g.~offline documentation) optional.

The installer includes three desktop and/or start menu shortcuts:

\begin{enumerate}
\def\labelenumi{\arabic{enumi})}
\item
  The shortcut titled just ``SageMath 8.0'' launches the standard Sage
  command prompt in a text-based console. In general it integrates well
  enough with the Windows shell to launch files with the default viewer
  for those file types. For example, plots are saved to files and
  displayed automatically with the default image viewer registered on
  the computer.
\item
  ``SageMath Shell'' runs a Bash shell with the environment set up to
  run software in the Sage distribution. More advanced users, or users
  who wish to directly use other software included in the Sage
  distribution (e.g.~GAP, Singular) without going through the Sage
  interface.
\item
  ``SageMath Notebook'' starts a Jupyter Notebook server with Sage
  configured as the default kernel and, where possible, opens the
  Notebook interface in the user's browser.
\end{enumerate}

\hypertarget{technical-details}{%
\subsection{Technical details}\label{technical-details}}

The main challenge with porting Sage to Windows/Cygwin has relatively
little to do with the Sage library itself, which is written almost
entirely in Python/Cython and involves relatively few system interfaces.
Rather, most of the effort has gone into build and portability issues
with Sage's more than 150 dependencies.

The majority of issues have been build-related issues. Runtime issues
are less common, as many of Sage's dependencies are primarily
mathematical, numerical code. Another reason is that, although there are
some anomalous cases, Cygwin's emulation of POSIX interfaces is good
enough that most existing code just works as-is. However, because
applications built in Cygwin are native Windows applications and DLLs,
there are Windows-specific subtleties that come up when building some
non-trivial software. So most of the challenge has been getting all of
Sage's dependencies building cleanly on Cygwin, and then maintaining
that support (as the maintainers of most of these dependencies are not
themselves testing against Cygwin regularly).

In fact, maintenance was the most difficult aspect of the Cygwin port
(and this is one of the main reasons past efforts failed--without a
sustained effort it was not possible to keep up with the pace of Sage
development). The critical component that was missing for creating a
sustainable Cygwin port of Sage was a \emph{patchbot} for Cygwin. The
Sage developers maintain a floatilla of "patchbots"--computers
running a number of different OS and hardware platforms that perform
continuous integration testing of all proposed software changes to Sage.
The patchbots are able to catch changes that break Sage before they are
merged into the main development branch. Without a patchbot testing
changes on Cygwin, there was no way to stop changes from being merged
that broke Cygwin. With some effort Erik Bray managed to get a Windows
VM with Cygwin running reliably on Université Paris-Sud's OpenStack
infrastructure, that could run a Cygwin patchbot for Sage. By continuing
to monitor this patchbot the SageMath community can now receive prior
warning if/when a change will break the Cygwin port. We expect this will
impact only a small number of changes--in particular those that update
one of Sage's dependencies.

In so doing we are, indirectly, providing continuous integration on
Cygwin for SageMaths's many dependencies--something most of those
projects do not have the resources to do on their own. So this should be
considered a service to the open source software community at large.

\hypertarget{alternatives}{%
\subsubsection{Alternatives}\label{alternatives}}

There are a few possible routes to supporting Sage on Windows, of which
Cygwin is just one. For example, before restarting work on the Cygwin
port Erik Bray experimented with a solution that would run Sage on
Windows using Docker. This approach ``worked'', but was still fairly
clumsy and error-prone. It was showcased at the first project review in
April 2017, and was advertised on SageMath's website as a temporary
alternate installation method for some time before the release of
SageMath 8.0.

Another approach, which was looked at in the early efforts to port Sage
to Windows, would be to get Sage and all its dependencies building with
the standard Microsoft toolchain (MSVC, etc.). This would mean both
porting the code to work natively on Windows, using the MSVC runtime, as
well as developing build systems compatible with MSVC. There was a time
when, remarkably, many of Sage's dependencies did meet these
requirements. But since then the number of dependencies has grown too
much, and Sage itself become too dependent on the GNU toolchain, that
this would be an almost impossible undertaking with available resources.

A middle ground between MSVC and Cygwin would be to build Sage using the
MinGW toolchain, which is a port of GNU build tools (including binutils,
gcc, make, autoconf, etc.) as well as some other common UNIX tools like
the Bash shell to Windows. Unlike Cygwin, MinGW does not provide
emulation of POSIX or Linux system APIs. This would actually be the
preferred approach, and with enough time and resources it could probably
work. However, it would still require a significant amount of work to
port some of SageMath's more non-trivial dependencies, such as GAP and
Singular, to work on Windows without some POSIX emulation. Now that the
Cygwin port has been completed, a MinGW port seems more feasible.

Finally, a note on the Windows Subsystem for Linux (WSL), which debuted shortly
after the start of OpenDreamKit. The WSL is a new effort by Microsoft to allow
running executables built for Linux directly on Windows, with full support from
the Windows kernel for emulation of Linux system calls. Basically, it aims to
provide all the functionality of Cygwin, but with full support from the kernel,
and the ability to run Linux binaries directly without having to recompile
them. This is a very promising development. Thus, the question is asked if Sage
can run in this environment, and experiments suggest that it works, albeit
imperfectly (the WSL is still under active development by Microsoft and some of
its interfaces are less robust than others).

A detailed account of WSL was already given in deliverable D2.3
(September 2016). In short:
\begin{enumerate}
\def\labelenumi{\arabic{enumi})}
\item The WSL is currently only intended as a developer tool: there is no way
    to package Windows software for end users such that it uses the WSL
    transparently, and users must install Microsoft development tools in order
    to use WSL.

\item It is only available on recent updates of Windows 10--it will never be
    available on older Windows versions.
\end{enumerate}

So to reach the most users, and provide the most hassle-free user experience,
the WSL is not currently a solution.  However, it may still prove useful for
developers as a way to do Sage development on Windows. And in the future it may
be the easiest way to install UNIX-based software on Windows as well,
especially if Microsoft ever expands the scope of its supported use cases.

\hypertarget{port-for-32-bits-architectures}{%
\subsection{Port for 32 bits
architectures}\label{port-for-32-bits-architectures}}

Until 2013 the only available version of Cygwin was for 32-bit
architectures. The original goal of the deliverable was to make
Cygwin-based installers both for 32- and 64-bit architectures. After
getting Sage working on 64-bit Cygwin, when it came time to test on
32-bit Cygwin we hit some significant snags.

The main problem is that 32-bit Windows applications have a user address
space limited to just 2 GB. This is in fact not enough to fit all of
SageMath into memory at once. With some care, such as reserving address
space for the most likely to be used (especially simultaneously)
libraries in Sage, we can work around this problem for the average user.
But the result may still not be 100\% stable.

It becomes a valid question whether this is worth the effort. There are
unfortunately few publicly available statistics on the current market
share of 64-bit versus 32-bit Windows versions among desktop users. Very
few new desktops and laptops sold anymore to the consumer market include
32-bit OSes, but it is still not too uncommon to find on some older,
lower-end laptops. In particular, some laptops sold not too long ago
with Windows 7 were 32-bit. According to Net Market Share\footnote{\url{https://www.netmarketshare.com/operating-system-market-share.aspx?qprid=10\&qpcustomd=0}},
as of writing Windows 7 still makes up nearly 50\% of all desktop
operating system installments. This still does not tell us about 32-bit
versus 64-bit. The popular (12.5 million concurrent users) Steam PC
gaming platform publishes the results of their usage statistics
survey\footnote{\url{http://store.steampowered.com/hwsurvey/}}, which as
of writing shows barely over 5\% of users with 32-bit versions of
Windows. However, computer gamers are not likely to be representative of
the overall market, being more likely to upgrade their software and
hardware.

Given the lack of demand for 32-bit versions of SageMath on Windows, the
continually shrinking market share of such systems, and the availability of
legacy installation methods for them, we have decided not to pursue this effort
further. Should a serious use case for such a port eventually arise, we may
have to reconsider our decision.

\hypertarget{conclusion}{%
\subsection{Conclusion}\label{conclusion}}

Focusing on Cygwin for porting Sage to Windows was definitely the right
way to go. It took only a few months in the summer of 2016 to get the
vast majority of the work done. The rest was just a question of keeping
up with changes to Sage and fixing more bugs. Now, however, enough
issues have been addressed that the Windows version has remained fairly
stable, even in the face of ongoing updates to Sage.

Porting more of Sage's dependencies to build with MinGW and without
Cygwin might still be a worthwhile effort, as Cygwin adds some overhead
in a few areas, but if we had started with that it would have been too
much effort.

In the near future, however, the priority needs to be improvements to
user experience of the Windows Installer. In particular, a better
solution is needed for installing Sage's optional packages on Windows.
And an improved experience for using Sage in the Jupyter Notebook, such
that the Notebook server can run in the background as a Windows Service,
would be desirable. This feature would not be specific to Sage either, and
could benefit all users of the Jupyter Notebook on Windows.
