\documentclass{deliverablereport}

\deliverable{component-architecture}{multiplatform-buildbot}
\deliverydate{31/08/2018}
\duedate{31/08/2018 (M36)}
\author{Erik Bray, et. al.}

\usepackage{graphicx}
\usepackage{subcaption}

\begin{document}
\maketitle

\hypertarget{introduction}{%
\section{Introduction}\label{introduction}}

In this report we look at what some OpenDreamKit-affiliated projects have
achieved in the areas of continuous integration and multi-platform building and
testing.

Continuous integration (CI) in software development is a process whereby work
performed by one or more developers on a software project is regularly merged
together into a single, central software repository (referred to as the
'mainline'), and the software built and tested with success or failure of the
build reported quickly back to the developers of the project.  This helps to
ensure that individual developers' changes do not conflict with each other or
otherwise "break the build", and provides rapid feedback when breaking changes
are introduced into the mainline.  Both the process, and the associated tools
(e.g. automated continuous integration servers) are an essential part of the
day-to-day work of developers on those projects that use it.

Modern CI requires server infrastructure--at the very least one machine which
is always running that both performs software builds and reports the results of
those builds back to developers so that are kept regularly up-to-date on the
"health" of the build.  For some projects-- especially those that support
multiple software platforms--continuous integration infrastructure can involve
a whole fleet of hardware systems, each of which perform builds and tests of
the software and report results back to a "master server" which collates them
into a single multi-platform build report for developers to examine.
Unsurprisingly, as the CI needs of a project grow, so to does the size of its
CI infrastructure, and the time, financial resources, and expertise required
to maintain it.

The \Sage project, being quite large both in terms of number of contributors
and in terms of overall code base (and by extension the length of time required
to build the software and run its test suite) has non-trivial CI needs, and to
address this it has, over time, amassed a small multi-platform fleet of build
machines as part of its "buildbot" infrastructure, as well as expertise needed
to maintain that infrastructure.  One of the original aims of deliverable was
to see if other projects under the OpenDreamKit umbrella could benefit from
using Sage's buildbots, and thus achieve better multi-platform CI.
Additionally, we would look into widening the set of platforms supported by
Sage's buildbot infrastructure--in particular adding Windows builds to coincide
with the Sage's newfound Windows support \TODO{reference D3.7 here}.

In practice, the needs of the OpenDreamKit community as a whole with respect to
multi-platform CI, and in particular the need for a "common infrastructure" for
CI, did align with our original expectations, for reasons that are enumerated
in the following section.  Nevertheless, significant achievements were made by
OpenDreamKit projects in the area of CI, and there are lessons learned that we
are communicating through this report in the hopes of future cross-polination
on the subject.  Our experiences have also taught us that although there is not
a one-size-fits-all solution to CI, there remains a clear need in the community
for easier access to multi-platform build and development infrastructure,
epecially for non-free operating systems such as Windows an Mac OSX.

\hypertarget{changes-to-deliverable}{%
\section{Changes in scope of the deliverable}\label{changes-to-deliverable}}

% * Explosion in the last 5+ years in free CI services and integration into
%   issue trackers (GitHub especially); fewer infrastructure requirements
%   for most projects in this regard.
%
% * Different projects have different needs in this area; most can get by
%   with advisory CI provided by free services.  For some, especially larger
%   projects such as Sage and GAP, it is still useful to have additional
%   CI for checking integrations between multiple changes across a larger
%   matrix of platforms
%
% * Some projects simply have fewer cross-platform continuous integration
%   needs; especially true of projects that are mostly numerical in nature
%   such as linbox
%
% * Other projects can largely achieve multi-platform testing by making
%   their project available in conda-forge, and reusing the conda-forge
%   CI architecture [Julian may wish to say something about this, being
%   more knowledgeable about it.]
%
% * Note issues with obtaining necessary hardware and software licenses for
%   CI/build on non-free platforms: Windows, and ESPECIALLY OSX.
%
% Note: maybe Sage (specifically due to sage-the-distribution) is kind of
% special among all other projects.
%
% Also worth noting that Sage provides implicit cross-platform build and and
% testing for most OpenDreamKit projects and a large amount of other
% mathematical software (dozens of packages).  While Sage's test suite does not
% provide full test coverage for its dependent packages, it does exercise most
% of them extensively.  In particular, Sage tests all its dependencies at the
% system integration level (such as interaction between multiple processes,
% and running multi-threaded computations); these are the areas where the most
% platform-specific challenges arise, as opposed to purely numerical and
% mathematical code.  In fact, the process of porting Sage to Windows, and
% running Sage's tests on Windows \TODO{Reference D3.7} has led to the
% discovery and fixing of these kinds of system-level issues in software such
% as \GAP, ECL, and \PariGP among others.\TODO{Add links to those issues?}


\hypertarget{project-reports}{%
\section{Continuous integration achievements of OpenDreamKit projects}\label{project-reports}}

\subsection{SageMath}

% Julian should remark on his efforts here: In particular what problems
% motivated his efforts, and how he thinks this will improve things in the
% future.  Maybe make note here about producing Docker images as build
% artifacts (TODO: confirm if GAP is doing this too), and further how
% mybinder.org turns out to be a useful development tool in conjunction
% with Dockerized builds.  Not sure how much to say about that here, or
% leave it to the conclusion.

\subsection{GAP}

% Via Alex [can he write a couple paragraphs about this?]:
% I will be ready to add a section with the overview of the current state of
% continuous integration in GAP. In addition to the private Jenkins CI instance
% that we use for wrapping and testing release candidates, checking for GAP
% package updates and testing new versions of GAP packages, we now use Travis
% CI to run a number of tests for the main GAP repositories at
% https://travis-ci.org/gap-system/. These include not only CI test for the
% main GAP development repository, but also package integration tests that use
% a set of Docker containers build in various settings
% (https://hub.docker.com/r/gapsystem/). We have a standard CI setup for GAP
% packages, which package authors may adopt and customise. GAP packages using
% Travis CI in the gap-packages VO can be seen at
% https://travis-ci.org/gap-packages/ (some more at
% https://gap-packages.github.io/). Also, GAP and packages use CodeCov to
% measure code coverage: see https://codecov.io/gh/gap-system/gap for GAP and
% https://codecov.io/gh/gap-packages/ for GAP packages.
% 
% @embray
% > Much of the context to this deliverable is about multi-platform testing and
% support. What aspect of our work addresses this? The Travis-CI builds for GAP
% appear to be entirely, or at least almost entirely for Linux (@alex-konovalov
% can clarify). I don't see any OSX builds, and Travis does not support
% Windows.
% Good point about Windows - forgot to mention that we also use AppVeyor:
% https://ci.appveyor.com/project/gap-system/gap

% About Linux vs macOS: package integration tests (all but one badges at
% https://github.com/gap-system/gap-distribution) use Docker container, hence
% use Linux. The core system tests used both Linux and macOS builds in the
% past, but then were disabled because of performance problems:

% https://github.com/gap-system/gap/blob/c8104cd056833f60c9f40efcb81b37b9cec76198/.travis.yml#L66

% We have all three systems - Linux, Windows and macOS - available as Jenkins
% nodes in St Andrews, so we find this to be satisfactory for the moment.


\subsection{Singular}


% Maybe ??

\subsection{PARI}

% Maybe ??

\hypertarget{best-practices}{%
\section{Lessons learned and best practices}\label{best-practices}}
\end{document}

%%% Local Variables:
%%% mode: latex
%%% TeX-master: t
%%% End:
