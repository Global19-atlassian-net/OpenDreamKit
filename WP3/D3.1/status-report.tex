\section{Status report}\label{status-report}

We have created virtual images and containers for most of the relevant
components of OpenDreamKit. The following projects have been dealt
separately:

\begin{itemize}
\tightlist
\item
  MathHub, (containers hosted at \url{https://hub.docker.com/r/kwarc/})
\item
  GAP (containers hosted at \url{https://hub.docker.com/u/gapsystem/})
\item
  OOMMF (virtual image hosted at
  \url{https://github.com/fangohr/virtualmicromagnetics})
\item
  SageMath, PARI/GP, Singular, etc. (containers hosted at
  \url{https://hub.docker.com/u/sagemath/})
\end{itemize}

We report below on each of these components

\subsection{MathHub}\label{mathhub}

The MathHub team at Jacobs University maintains two Docker containers
for MathHub:

\begin{itemize}
\tightlist
\item
  \href{https://hub.docker.com/r/kwarc/mathhub/}{MathHub} (Dockerfile
  source \url{https://github.com/KWARC/mathhub_docker})
\item
  \href{https://hub.docker.com/r/kwarc/localmh/}{LocalMH} (Dockerfile
  source \url{https://hub.docker.com/r/kwarc/localmh/})
\end{itemize}

The first container is for a MathHub instance, like the one at
\url{https://mathhub.info/}. The second container is for the LocalMH
tool, for offline authoring of MathHub installations.

Both containers are manually built. Build automation via Docker Hub's
\href{https://docs.docker.com/docker-hub/builds/}{automated build
service} is a potential improvement that will be studied in a next step.

\subsection{GAP system}\label{gap-system}

The GAP team maintains two Docker containers:

\begin{itemize}
\tightlist
\item
  \href{https://hub.docker.com/r/gapsystem/gap-container/}{gap-container}
  (Automated build from \url{https://github.com/gap-system/gap-container}).
\item
  \href{https://hub.docker.com/r/gapsystem/gap-docker/}{gap-docker}
  (Automated build from \url{https://github.com/gap-system/gap-docker}).
\end{itemize}

The first is a minimalistic build containing exclusively the core GAP
functionality, without any optional GAP packages (user-contributed
extensions, redistributed with GAP). The second is a full GAP install,
containing all packages redistributed with GAP. Because some of the
packages have non-trivial dependencies on third-party software,
gap-docker depends on a
\href{https://hub.docker.com/r/gapsystem/gap-docker-base/}{gap-docker-base
container} which provides these dependencies. gap-docker-base is not
built automatically because of the time and memory restrictions of
DockerHub automated builds.

The GAP Docker container is suggested as one of the alternative ways to
install GAP: \url{http://www.gap-system.org/Download/alternatives.html}.

As a demo application to demonstrate the use of GAP Docker container in
the cloud, we run an SCSCP server for the number of groups of order
\(n\). This service uses GAP Docker container running on a Microsoft
Azure virtual machine. Usage details could be found at
\url{https://github.com/alex-konovalov/gnu}.

GAP is also bundled in the SageMath containers described below.

\subsection{OOMMF}\label{oommf}

The \href{http://fangohr.github.io/virtualmicromagnetics/}{Virtual
Micromagnetics} package is designed to be run through Vagrant
provisioned by VirtualBox. The virtual image is hosted at
\url{https://atlas.hashicorp.com/virtualmicromagnetics/boxes/full}.

The Virtual Micromagnetics developers are considering Docker as an
alternative provider for the OOMMF component.

\subsection{SageMath}\label{sagemath}

SageMath bundles most of OpenDreamKit components. The community
maintains several Docker containers for use and development of SageMath
at \url{https://hub.docker.com/u/sagemath/}. We describe below the work that
has been done in OpenDreamKit regarding those containers.

For SageMath we have created a repository for issue tracking and
development of build scripts for SageMath docker images at
\href{https://github.com/sagemath/docker-images}{sagemath/docker-images}
(not to be confused with
\href{https://github.com/sagemath/docker/}{sagemath/docker} which we
have agreed will be superseded by the former). The repository currently
includes build recipes for four Docker images: * sagemath/sagemath -
This will always provide a build of the most recent released version of
SageMath (or previous versions through the use of image tags).

\begin{itemize}
\item
  sagemath/sagemath-develop - This provides a build of the most recent
  develop branch of SageMath from the git repository at the time of
  building. This image will need to be updated automatically on a
  regular basis via an automated mechanism (more on that later).
\item
  sagemath/sagemath-jupyter - This is the same as the sagemath/sagemath
  image, but automatically starts the Jupyter notebook with sage when it
  is run (the base sagemath/sagemath image drops into the sage command
  line by default).
\item
  sagemath/sagemath-patchbot - This is the same as the
  sagemath/sagemath-develop image, but automatically runs the sage
  patchbot by default. There are some currently unresolved (but likely
  minor) issues with running the patchbot in this container.
\end{itemize}

We have found Docker containers to be an effective way to distribute a
consistent installation of Sage across multiple platforms--Docker
support Linux, Mac OSX, and Windows (with some caveats; more on that
below). Running Sage in a Docker container provides a consistent
experience to the user, regardless what the host OS is, and is generally
lighter-weight and more transparent to the user than a full VM. When
using the sage command line interface (or any other command line program
running in the container) the user interacts with it directly through
their host OS's terminal emulator. The container can also run headless,
such as when running the notebook, in which case the user can connect
their web browser to the notebook server as though running directly on
localhost.

We are also exploring the use of Docker on Windows as a way to provide a
simple one-click installer for SageMath on Windows as near-term solution
to \#66 until and unless Sage can be built and run natively on Windows.
This in turn has driven improvements to the sagemath Docker images. The
installer for Sage would install Docker and its dependencies (if not
already installed), and would include the sagemath/sagemath image (or
optionally the -develop version) and launcher shortcuts for starting the
command-line and notebook interfaces. This installer is being developed
in \href{https://github.com/embray/sage-windows}{embray/sage-windows}
and will have a prototype ready soon.

The current Docker images for SageMath are based on an Ubuntu 15.10
(Wily Werewolf) base image, though it would be possible to reuse most of
the existing build scripts to build SageMath images based on other Linux
distributions. This could be useful for Sage development, but less
interesting to end-users (see sagemath/docker-images\#12).

The Docker Hub service for hosting Docker images does provide an
\href{https://docs.docker.com/docker-hub/builds/}{Automated Build
service} that could in principle be used to automatically build new
sagemath/sagemath-develop images (as well as sagemath/sagemath when
there is a new release). It can monitor changes to the repository
storing the image build recipes, and could also be configured to
re-build an image upon push to another repository (such as sagemath
itself). Leveraging this service would free us of the overhead of
maintaining our own build infrastructure for images. However, the Docker
Hub Automated Build service does have resource restrictions that may
prove too limiting for building SageMath:

\begin{itemize}
\tightlist
\item
  1 CPU
\item
  2 GB of RAM
\item
  30 GB disk space
\item
  2 hour limit to build time
\end{itemize}

The disk space is enough, and the RAM is also enough given only 1 CPU is
provided. However with only one CPU it may be difficult to get the build
time down to 2 hours, but we have not tried this yet. I (@embray)
believe there are several actions that can be taken to improve build
time of Sage. In particular, a good many of its dependencies are
available, with the correct versions, pre-built from Ubuntu's APT
repository (excluding packages that require special patches from Sage
that affect runtime behavior). Improving the Sage build process to allow
more reliance of system packages where possible will significantly cut
down build time and be beneficial well beyond the case of building
Docker images.

\subsubsection{Caveats:}\label{caveats}

\begin{itemize}
\item
  Disk space: The Docker images for SageMath are quite large. Originally
  the sagemath/sagemath image weighed in at 7.361 GB. It was possible to
  reduce this to 1.988 GB by relying more on system packages and
  cleaning up artifacts of the sage build process at the end of the
  image build. For the sagemath/sagemath-develop image we decided to
  leave those artifacts intact, so running this image gives access to a
  sage source tree as it would appear immediately after running
  \texttt{make}. It's possible a few other reductions can be found (such
  as uninstalling build dependencies once the build is complete), but
  there's no escaping the fact that these Docker images will be a large
  download. Fortunately, Docker images can be exported to a \texttt{tar}
  archive (and further compressed with the compression method of one's
  choice), for distribution on USB drives and the like.
\item
  Connecting to a notebook or other network interface running in the
  container does, however, require manual setup of port forwarding when
  running the Docker container. The command line option for this is
  relatively simple to communicate, however, and wrapper scripts /
  shortcuts may also be provided to users.
\item
  Although the use of Docker containers provides a more or less uniform
  experience across host OS's, Docker does not technically run natively
  on Windows or OSX, as the way it works relies heavily on features
  specific to the Linux kernel. As such, running Docker on Windows or
  OSX does actually require running a Linux VM in the first place. The
  \href{https://www.docker.com/products/docker-toolbox}{Docker Toolbox},
  which installs Docker on Windows and OSX, also installs VirtualBox and
  a very small Linux VM, designed specially for Docker, which includes
  the bare minimum to run Docker and not much else. Because of its
  specialized purpose, this VM adds very little overhead compared to a
  more general purpose VM.
\end{itemize}

Docker Toolbox includes a program that makes management of this VM easy
(\href{https://docs.docker.com/machine/}{docker-machine}), and because
the VM runs headless users will never see it. The shortcuts installed by
Docker Toolbox run a script that ensures the ``boot2docker'' VM is
running, and thus allowing all other \texttt{docker} commands to work.
So with a few exceptions users never need to be aware of this underlying
virtualization layer:

\begin{itemize}
\item
  With Docker it is relatively easy to mount existing directories on the
  host machines as ``volumes'' accessible from the Docker container (see
  \href{https://docs.docker.com/engine/userguide/containers/dockervolumes/}{``Manage
  data in containers''}). There are a few downsides to this:

  \begin{itemize}
  \item
    Although any number of directories can be mounted, they can only be
    mounted when a container is first created, so one has to think ahead
    and ensure that all data one will need to access within the
    container will be accessible (if in doubt, one could mount their
    entire filesystem I suppose). In principle it is possible to add new
    mounts from within a container, but not in a way that is friendly to
    novices (requires using the \texttt{mount} command).
  \item
    On Windows and OSX, because of the two layers of virtualization, any
    local directory that one wants to mount within a Docker container
    needs to also first be mounted in the boot2docker VM that Docker
    itself runs in, so it's a two step process and probably hard to
    explain to novices. On the bright side, boot2docker
    \emph{automatically} mounts the user's home directory, so this is
    immediately available. And in many cases all the files a user cares
    about should be in their home directory, but that is of course often
    not the case as well. Another possible ``bright side'' is that
    VirtualBox includes a command-line interface for controlling
    VMs--\href{https://www.virtualbox.org/manual/ch08.html}{\texttt{VBoxManage}}.
    Although not novice-friendly, this does enable writing
    novice-friendly interfaces for mounting additional directories if
    they need to, and I (@embray) am working on such interfaces in the
    Windows installer for Sage (and maybe, hopefully, upstream to Docker
    as well).
  \end{itemize}
\item
  Networking has the same problem that directory mounting has--it is
  fairly easy, when starting a new Docker container, to set up which TCP
  ports should be forwarded from the Docker container to ports on the
  host machine. However, this is harder to do after a container has
  already been started, and one might have to start a new container if
  they don't forward the ports properly. Providing users with shortcuts
  that do this automatically helps. Windows and OSX have the same
  problem with ports needing to be forwarded on the boot2docker VM as
  well, though this can also be managed with \texttt{VBoxManage}.
\item
  Probably the biggest hurdle to setting up Docker on Windows and OSX is
  the requirement for
  \href{https://en.wikipedia.org/wiki/Hardware-assisted_virtualization}{Hardware
  Assisted Virtualization} (HAV) to be enabled on the user's system.
  This is not so much a limitation of Docker as it is a limitation of
  VirtualBox--it can't run the boot2docker VM (or any 64-bit VM) without
  HAV. The default HAV settings vary from system to system, though it is
  often disabled by default as a security measure. Some systems provide
  a tool to change this and other BIOS settings from within Windows, but
  more likely users will have to change this setting from their BIOS
  configuration menu. This is a support nightmare since so many BIOS
  menus are different and there is no standard. That said, most BIOS
  menus have improved substantially in the last decade, so I think
  helping users with this is less hassle than some other options. But it
  is certainly discouraging to tell novices ``Now to install Sage you
  have to reboot your computer, hold down ESC, or DEL, or F10, or
  Ctrl-C, or Fn-Ctrl-Break, etc. etc. and then rummage around in this
  plain text curses interface until you find something that says
  something something about''virtualization" or maybe ``VT-X'' and make
  sure that's enabled. Then press F10 or F8 or S or etc. etc. to save
  the settings and reboot. Make sure not to change the time!"

  I think it's worth trying this out in at least one workshop to see how
  bad it is in practice though. The broader Docker community may also
  have some useful experience with this, as this is currently the
  \emph{only} way to get Docker on Windows.

  A possible workaround is to provide a 32-bit version of boot2docker,
  which is theoretically possible (though it will run slower
  \emph{without} HAV). But this is not officially supported by the
  Docker project, and isn't likely to be (docker/docker\#136).
\end{itemize}
