\documentclass{deliverablereport}

\deliverable{component-architecture}{sage-distribution}
\deliverydate{31/08/2019}
\duedate{31/08/2019 (M48)}
\author{
  Nicolas M. Thiéry,
  Viviane Pons,
  Florent Hivert,
  Samuel Lelièvre,
  Erik Bray,
  Loïc Gouarin,
  Florian Rabe,
  Vincent Delecroix,
  Vincent Klein,
  Clément Pernet,
  Jean-Guillaume Dumas,
  Daniel Schultz,
  Dmitrii Pasechnik
  Steve Linton,
  Alexander Konovalov,
  Markus Pfeiffer,
  Michael Torpey,
  Luca De Feo,
  John E. Cremona,
  Olivier Cayrol,
  Julien Cristau,
  Jeroen Demeyer,
  \dots
}

\begin{document}
\maketitle
% This will be the abstract, fetched from the github description
\githubissuedescription

\section{Package repositories in Linux}

Operating systems offer several ways to install and/or update
software. %
Since long, Microsoft Windows has accustomed its users with the
``installer'' pattern: obtain an executable, by downloading it or by
running it from external media (e.g., a CD or DVD), which takes care
of installing the software in the appropriate locations.

At the same time, Linux \emph{distributions} have privileged a
different software delivery mechanism: official (and unofficial)
\emph{repositories}, collections of installable \emph{packages} that
software called a \emph{package manager} can retrieve (typically via
download) and install in the appropriate location. %
Thanks to ever increasing Internet speeds, the package repository
model has gained in popularity, and has notably been replicated by the
``app store'' concept of mobile OSes, and more recently Windows 10.

While in the installer model the responsibility for packaging and
distributing software mostly falls on the software vendor, the package
repository model allows for a variety of policies: in the Linux space,
it is customary for distributions to have official repositories, where
software are packaged by select members of the community, along with
unofficial repositories where packages can be contributed by anyone.

\subsection{Package ecosystems in Linux}

Over time, several different package formats have developed for Linux,
with associated package managers. %
Linux distributions tend to support chiefly one specific package
format, eventually providing tools for manipulating packages in other
formats. %
There is usually a distinction between \emph{source packages} and
\emph{binary packages}: the former contain source code, and are
eventually used to build the latter. %
Most Linux distributions provide software to end users in the form of
binary packages (Gentoo is a notable exception); in the rest of this
document we will mostly refer to binary packages.

The two most popular package formats are \texttt{.deb} and
\texttt{.rpm}, respectively created by the Debian and Red Hat
distributions. %
Nowadays, each of them serves as the main format for several popular
distributions; Table~\ref{tab:pkg-fmt} summarizes the principal Linux
distributions with their package format.

\begin{table}
  \centering
  \begin{tabular}{l | p{0.5\textwidth} | l}
    Package format & Distributions & Package manager\\
    \hline
    \texttt{.rpm} & RHEL, Fedora, CentOS, OpenMandriva, OpenSUSE, \dots & yum\\
    \texttt{.deb} & Debian, Ubuntu, Mint, Knoppix, Raspbian, \dots & apt \\
    Portage & Gentoo, Chrome OS, \dots & emerge\\
    \texttt{PKGBUILD} & Arch, Manjaro, \dots & pacman\\
    Anaconda & \emph{OS-agnostic} & conda
  \end{tabular}
  \caption{Most popular Linux distributions and package formats}
  \label{tab:pkg-fmt}
\end{table}

One of the main prerogatives of a Linux distribution is to choose what
packages to make available through its official repositories, prepare
them, and ensure that they are compatible with one another. %
Different distributions that use the same package format may or may
not share the same packages. %
For example, Debian and Ubuntu have mostly disjoint package sets,
prepared independently by the respective communities; however some
packages in Ubuntu, notably those that pertain to \ODK, are provided
by the DebianScience team%
\footnote{\url{https://wiki.debian.org/DebianScience}.}.

\subsection{Other packaging paradigms}

Besides ``classic'' Linux package systems, several other paradigms
have emerged over the years.

Most programming languages nowadays come with an official repository
where source code or pre-compiled code can be stored in the form of
packages, and retrieved using a language-specific package manager,
regardless of the OS of the user. %
Examples of this paradigm are the Python Package Index (PyPI)%
\footnote{\url{https://pypi.org/}} %
for Python, or the Comprehensive R Archive Network (CRAN)%
\footnote{\url{https://cran.r-project.org/}} %
for R. %
While language-specific repositories are mainly targeted at code
written in the respective programming language, every system also
offers facilities to install dependencies written in other programming
languages, such as pre-compiled C libraries, thus blurring the
boundary between the OS package manager and the programming language
one. %
Table~\ref{tab:pkg-lang} summarizes the official package repository
and package manager for some popular programming languages relevant to
\ODK.

\begin{table}
  \centering
  \begin{tabular}{l | l | l}
    Programming language & Official repository & Package managers\\
    \hline
    Python & PyPI & pip\\
    Julia & General & Pkg2\\
    R & CRAN & install\\
    Perl & CPAN & PPM\\
    JavaScript & NPM & npm \\
    \emph{any} & Conda-forge & conda
  \end{tabular}
  \caption{Some programming languages and their official package systems}
  \label{tab:pkg-lang}
\end{table}

\emph{Agnostic} package systems push the concept one step further, by
being unrelated to both an OS and a programming language. %
Currently, one of the most popular package systems is Anaconda, which
consists of an open-source package manager (conda), and a
community-led package repository (Conda-forge%
\footnote{\url{https://conda-forge.org/}}), %
backed by a commercial company named Anaconda. %
Originally born as a package system for Python, it has evolved into a
OS-and-language-agnostic package system targeted at scientific
computing and data science. %
Centered around JupyterLab and the Jupyter notebook for its user
interface, Anaconda offers a one-stop solution to install all
programming languages, libraries and tools to work on scientific
projects.

Being primarily targeted at developers, rather than end users, both
language-specific and language-agnostic package repositories are
usually open to contributions from anyone at any time.%
\footnote{In some cases, like for Anaconda, a separate curated package
  repository may be offered to paying customers.} %
Unlike OS package repositories, they do not aim at having a full set
of software all compatible with one another at any given time, but
rather at having all versions (including bleeding edge) of any given
software installable at all time. %
To better cater to development workflows, these package managers
usually provide an isolation mechanism (often called
\emph{environments}) that permits to have several versions of the same
software installed and running at the same time. %
This way, even if two software have conflicting dependencies, they can
be installed at the same time by having two separate environments for
them.

While potentially more demanding in resources, the installation model
based on isolated environments allows more flexibility, and is
especially interesting for \emph{reproducible} software builds. %
It has thus been replicated in the OS space by some Linux
distributions, in particular NixOS%
\footnote{\url{https://nixos.org/}} %
and GUIX%
\footnote{\url{https://guix.gnu.org/}}. %
A more extreme mechanism uses \emph{containerization} to isolate
software so that distinct environments do not even share system
resources such as file-systems or process tables. %
This is the approach, taken for example by Red Hat's Container Linux%
\footnote{\url{https://coreos.com/}} %
(formerly CoreOS), which is primarily targeted at cloud infrastructures.

\subsection{Relevance for \ODK}

The software made by \ODK is primarily available as source code and
precompiled binaries from the respective project web pages. %
Typically, software projects (as opposed to libraries) also provide
executable installers from their web pages, thus installation by the
installer pattern is always an option for the end user, and in some
cases (e.g., installation on Windows) even the preferred one
(see~\delivref{component-architecture}{portability-cygwin}).

Packaging software for package repositories is rarely the
responsibility of the software developer, and indeed, outside a few
exceptional cases, \ODK software is not packaged by \ODK members. %
While not under our direct responsibility, it is nevertheless of
primary importance that our software is made available through as many
package repositories as possible. %
Indeed, while most end-users (Windows users in particular) who install
software on their personal computers are perfectly happy with the
installer pattern, \ODK's goal is to reach a much larger audience:
shared server and cloud infrastructures, the vast majority of which
runs under Linux, are extremely reliant on package managers
(both OS-specific and OS-agnostic) for installing software.

This deliverable describes the efforts made by \ODK to have all its
software components packaged for the official repositories of all
major Linux distributions. %
While not precisely within the scope of this deliverable, we also
report on packaging for OS-agnostic repositories, and on the impact on
software design.

A different packaging issue is having a package system \emph{for} a
\ODK software, in the same vein of language-specific package
systems. %
For example, \GAP has developed a package format, a package manager,
and maintains an official repository, so that users can publish and
share their code written in \GAP with other users. %
Since this aspect of packaging is inextricably linked with the former,
we also report on it.


\section{Linux packages for \ODK components}

The goal of this deliverable was to have official packages for all
major Linux distributions for all \ODK components. %
Before the start of \ODK, availability varied greatly among
distributions and software. %
To begin, it is necessary to define what is meant by ``major''
distribution; in an inevitably biased way, we chose to target the
following distributions:

\begin{itemize}
\item Debian, Fedora: for their popularity among advanced end-users
  and system administrators;
\item Ubuntu: for its popularity among beginners and casual users, as well as
  in the server and cloud space;
\item Arch, Gentoo: for their popularity among power-users.
\end{itemize}

Being the lone project depending upon all other \ODK components, \Sage
is automatically the most challenging to package. %
Hence, packagers for \Sage are usually, though not always, also
responsible for packaging most \ODK components within a
distribution. %
Table~\ref{tab:maintainers} summarizes the currently active package
maintainers for each of the distributions above; with the exception of
Bill Allombert, and some occasional members of the Debian Science
group, all package maintainers are external to the \ODK project.

\begin{table}
  \centering
  \begin{tabular}{p{0.5\textwidth} | p{0.5\textwidth}}
    Distribution & Maintainers\\
    \hline
    Debian, Ubuntu & Bill Allombert, Debian Science\\
    Fedora & Paulo César Pereira de Andrade \emph{et al.}\\
    Arch & Antonio Rojas, Felix Yan\\
    Gentoo & François Bissey \emph{et al.}\\
    Conda-forge & Isuru Fernando \emph{et al.}
  \end{tabular}
  \caption{Package maintainers for various distributions}
  \label{tab:maintainers}
\end{table}

With the exception of Debian and Ubuntu, all distributions above have
a history of regularly packaging \Sage (and thus all its dependencies)
since at least 2013, well before the start of \ODK. %
Hence, the strategy adopted for this deliverable was to have \Sage
packaged for Debian, in collaboration with the Debian Science team,
with all \Sage dependencies and packages for Ubuntu automatically
following from this work.

Debian has a two year long \emph{release cycle}, each cycle beginning at the
middle of odd numbered years. %
This means that every two years the community decides on which
packages should be included in the next stable release, at which point
a \emph{freeze} happens and packages within a release can only be
updated to fix bugs and security issues. %
``Missing the train'' for one release cycle means that a software
cannot enter Debian for the next two years. %
Table~\ref{tab:debian} summarizes the Debian releases that happened
during the 2015-2019 period.

\begin{table}
  \centering
  \begin{tabular}{c c c}
    Codename & Number & Date\\
    \hline
    Jessie & 8 & April 26th, 2015\\
    Stretch & 9 & June 17th, 2017\\
    Buster & 10 & July 6th, 2019
  \end{tabular}
  \caption{Debian releases during the \ODK project}
  \label{tab:debian}
\end{table}

Ubuntu has a similar two year long \emph{long-term-support (LTS)}
release cycle, but also a shorter six-month cycle for quicker
releases, thus allowing more chances to release a package.

\subsection{Status before \ODK}

At the start of \ODK, only packages for \PariGP and \GAP were
available in Debian/Ubuntu, thanks to the continued efforts of Bill
Allombert, a long time member of the Debian community. %

Information on the packages available at the time are summarized in
Table~\ref{tab:odk-2015}; for succinctness, we only list the four
core computer algebra systems (CAS) distributed by \ODK, leaving libraries out
of the picture.%
\footnote{Libraries typically have fewer dependencies and are easier to
  package.} %
To give an idea of the complexity of the task, we list the number of
packages in the dependency tree of each software (i.e., the total
number of packages that would get installed on a fresh Debian system
with no other software than the package manger); we do the same for
the number of \emph{dependent} packages (packages that depend on the
software).

\begin{table}
  \centering
  \begin{tabular}{l | c | c | c | c | c}
    Software & Stable version & Version in Debian & Maintainer & Dependencies & Dependents \\
    \hline
    \GAP & 4.7.8 & 4r7p5-2 & Bill Allombert & 13 & 9\\
    \PariGP & 2.7.4 & 2.7.2-1 & Bill Allombert & 36 & 6\\
    \Sage & 6.8 & --- & --- & --- & ---\\
    \Singular & 4.0.1 & --- & --- & --- & ---\\
  \end{tabular}
  \caption{\ODK packages in Debian Jessie, September 1, 2015. Source
    \url{http://snapshot.debian.org/}.}
  \label{tab:odk-2015}
\end{table}

\Sage (version 3.0 at the time) had been packaged in Debian for the last time
in 2010, but fell off the official distribution due to incompatible
dependencies. %
Ubuntu used to have an unofficial \Sage package, also installing
\Singular and other dependencies alongside, but this package offered
little in terms of compatibility and integration with the rest of the
system.

\subsection{Current status}

The first occasion to have all \ODK components packaged for Debian was offered
by the \emph{freeze} for the Debian 9.0 (codenamed "Stretch") release in 2017. %
Thanks to the intense collaboration between \ODK and the Debian
Science team, the occasion was not missed, thus achieving the goal of
the deliverable 24 months in advance of the projected deadline. %
Table~\ref{tab:odk-2017} summarizes information on the packages
available in Debian Stretch. %
The same packages were added to the official Ubuntu repositories at
the same time.

\begin{table}
  \centering
  \begin{tabular}{l | c | c | c | c | c}
    Software  & Stable version & Version in Debian & Maintainer & Dependencies & Dependents \\
    \hline
    \GAP      & 4.8.8 & 4r8p6-2         & Bill Allombert &  14 & 20\\
    \PariGP   & 2.9.3 & 2.9.1-1         & Bill Allombert &  39 &  9\\
    \Sage     &   8.0 & 7.4-9           & Debian Science & 653 &  3\\
    \Singular & 4.0.3 & 1:4.0.3-p3+ds-5 & Debian Science &  21 &  4\\
  \end{tabular}
  \caption{\ODK packages in Debian Stretch, August 31, 2017. Source
    \url{http://snapshot.debian.org/}.}
  \label{tab:odk-2017}
\end{table}

The next Debian release cycle came to completion in July 2019; all
\ODK packages were updated and improved for the occasion, with a
marked simplification of the dependency tree. %
Table~\ref{tab:odk-2019} summarizes information on the packages
available in Debian 10 (codenamed "Buster").

\begin{table}
  \centering
  \begin{tabular}{l | c | c | c | c | c}
    Software  & Stable version & Version in Debian & Maintainer & Dependencies & Dependents \\
    \hline
    \GAP      & 4.10.2 & 4r10p0-7        & Bill Allombert &  15 & 13\\
    \PariGP   & 2.11.2 & 2.11.1-2        & Bill Allombert &  41 &  9\\
    \Sage     &    8.8 & 8.6-6           & Debian Science & 593 &  3\\
    \Singular &  4.1.2 & 1:4.1.1-p2+ds-3 & Debian Science &  23 &  4\\
  \end{tabular}
  \caption{\ODK packages in Debian Buster, August 29, 2019. Source
    \url{http://snapshot.debian.org/}.}
  \label{tab:odk-2019}
\end{table}

\subsection{Beyond Linux packages}

Having achieved the goals of the deliverable largely in advance, we
took the occasion to explore other package repositories. %
At the \ODK workshop in Cernay (see~\delivref{dissem}{workshops-4}),
besides experiments with GUIX and Nix packages, a sprint took place to
have \Sage packaged for Conda-forge. %
At the time of writing, the latest \Sage 8.8 and all its dependencies
are available on Conda-forge for the Linux and MacOS platforms. %
This packaging effort is still experimental, and will only become
stable after \Sage is officially migrated to Python 3 (see below).

\section{Work accomplished}


\section{The future}


\end{document}

%%% Local Variables:
%%% mode: latex
%%% TeX-master: t
%%% End:
