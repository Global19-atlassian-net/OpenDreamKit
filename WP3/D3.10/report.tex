\documentclass{deliverablereport}

\hypersetup{
  colorlinks=true,
  urlcolor=cyan,
  linkcolor=blue,
  citecolor=blue,
}

\deliverable{component-architecture}{sage-distribution}
\deliverydate{05/09/2019}
\duedate{31/08/2019 (M48)}
\author{Luca De Feo \emph{et al.}}

\begin{document}
\maketitle
% This will be the abstract, fetched from the github description
\githubissuedescription
\tableofcontents
\clearpage

\section{Package repositories in Linux}

Operating systems offer several ways to install and/or update
software. %
For a long time, Microsoft Windows has accustomed its users to the
``installer'' pattern: obtain an executable, by downloading it or by
running it from external media (e.g., a CD or DVD), which takes care
of installing the software in the appropriate locations.

At the same time, Linux\footnote{In this report, we use the name Linux for short;
  technically this is just the name of the system kernel, and the full
  operating system is called GNU/Linux.}
\emph{distributions} have privileged a
different software delivery mechanism: official (and unofficial)
\emph{repositories}, collections of installable \emph{packages} that
software called a \emph{package manager} can retrieve (typically via
download) and install in the appropriate location. %
Thanks to ever increasing Internet speeds, the package repository
model has gained in popularity, and has notably been replicated by the
``app store'' concept of mobile OSes, and more recently Windows 10.

While in the installer model the responsibility for packaging and
distributing software mostly falls on the software vendor, the package
repository model allows for a variety of policies: in the Linux space,
it is customary for distributions to have official repositories, where
software are packaged by select members of the community, along with
unofficial repositories where packages can be contributed by anyone.

\subsection{Package ecosystems in Linux}

Over time, several different package formats have developed for Linux,
with associated package managers. %
Linux distributions tend to support chiefly one specific package
format, eventually providing tools for manipulating packages in other
formats. %
There is usually a distinction between \emph{source packages} and
\emph{binary packages}: the former contain source code, and are
eventually used to build the latter. %
Most Linux distributions provide software to end users in the form of
binary packages (Gentoo is a notable exception); in the rest of this
document we will mostly refer to binary packages.

The two most popular package formats are \texttt{.deb} and
\texttt{.rpm}, respectively created by the Debian and Red Hat
distributions. %
Nowadays, each of them serves as the main format for several popular
distributions; Table~\ref{tab:pkg-fmt} summarizes the principal Linux
distributions with their package format.

\begin{table}[hb]
  \centering
  \begin{tabular}{l | p{0.5\textwidth} | l}
    Package format & Distributions & Package manager\\
    \hline
    \texttt{.rpm} & RHEL, Fedora, CentOS, OpenMandriva, OpenSUSE, \dots & yum\\
    \texttt{.deb} & Debian, Ubuntu, Mint, Knoppix, Raspbian, \dots & apt \\
    Portage & Gentoo, Chrome OS, \dots & emerge\\
    \texttt{PKGBUILD} & Arch, Manjaro, \dots & pacman\\
    Conda & \emph{OS-agnostic}, user-level & conda
  \end{tabular}
  \caption{Most popular Linux distributions and package formats}
  \label{tab:pkg-fmt}
\end{table}

One of the main prerogatives of a Linux distribution is to choose what
packages to make available through its official repositories, prepare
them, and ensure that they are compatible with one another. %
Different distributions that use the same package format may or may
not share the same packages. %
For example, Debian and Ubuntu have mostly disjoint package sets,
prepared independently by the respective communities; however some
packages in Ubuntu, notably those that pertain to \ODK, are provided
by the DebianScience team%
\footnote{\url{https://wiki.debian.org/DebianScience}.}.

\subsection{Other packaging paradigms}

Besides ``classic'' Linux package systems, several other paradigms
have emerged over the years.

Most programming languages nowadays come with an official repository
where source code or pre-compiled code can be stored in the form of
packages, and retrieved using a language-specific package manager,
regardless of the OS of the user. %
Examples of this paradigm are the Python Package Index (PyPI)%
\footnote{\url{https://pypi.org/}} %
for Python, or the Comprehensive R Archive Network (CRAN)%
\footnote{\url{https://cran.r-project.org/}} %
for R. %
While language-specific repositories are mainly targeted at code
written in the respective programming language, every system also
offers facilities to install dependencies written in other programming
languages, such as pre-compiled C libraries, thus blurring the
boundary between the OS package manager and the programming language
one. %
Table~\ref{tab:pkg-lang} summarizes the official package repository
and package manager for some popular programming languages relevant to
\ODK.

\begin{table}
  \centering
  \begin{tabular}{l | l | l}
    Programming language & Official repository & Package managers\\
    \hline
    Python & PyPI & pip\\
    Julia & General & Pkg\\
    R & CRAN & install\\
    Perl & CPAN & PPM\\
    JavaScript & NPM & npm \\
    \emph{any} & Conda-forge & conda
  \end{tabular}
  \caption{Some programming languages and their official package systems}
  \label{tab:pkg-lang}
\end{table}

\emph{Agnostic} package systems push the concept one step further, by
being unrelated to both an OS and a programming language. %
Currently, one of the most popular package systems is Conda, which
consists of an open-source package manager (conda), and a
community-led package repository (Conda-forge%
\footnote{\url{https://conda-forge.org/}}), %
backed by a commercial company named Anaconda. %
Originally born as a package system for Python, it has evolved into an
OS-and-language-agnostic package system targeted at scientific
computing and data science. %
Centered around JupyterLab and the Jupyter notebook for its user
interface, Conda offers a one-stop solution to install all
programming languages, libraries and tools to work on scientific
projects.

Being primarily targeted at developers, rather than end users, both
language-specific and language-agnostic package repositories are
usually open to contributions from anyone at any time.%
\footnote{In some cases, for example Conda, a separate curated package
  repository may be offered to paying customers.} %
Unlike OS package repositories, they do not aim at having a full set
of software all compatible with one another at any given time, but
rather at having all versions (including bleeding edge) of any given
software installable at all time. %
To better cater to development workflows, these package managers
usually provide an isolation mechanism (often called
\emph{environments}) that permits to have several versions of the same
software installed and running at the same time. %
This way, even if two software systems have conflicting dependencies, they can
be installed at the same time by having two separate environments for
them.

While potentially more demanding in resources, the installation model
based on isolated environments allows more flexibility, and is
especially interesting for \emph{reproducible} software builds. %
It has thus been replicated in the OS space by some Linux
distributions, in particular NixOS%
\footnote{\url{https://nixos.org/}} %
and GUIX%
\footnote{\url{https://guix.gnu.org/}}. %
A more extreme mechanism uses \emph{containerization} to isolate
software so that distinct environments do not even share system
resources such as file-systems or process tables. %
This is the approach, taken for example by Red Hat's Container Linux%
\footnote{\url{https://coreos.com/}} %
(formerly CoreOS), which is primarily targeted at cloud infrastructures.

\subsection{Relevance for \ODK}

The software made by \ODK is primarily available as source code and
precompiled binaries from the respective project web pages. %
Typically, software projects (as opposed to libraries) also provide
executable installers from their web pages, thus installation by the
installer pattern is always an option for the end user, and in some
cases (e.g., installation on Windows) even the preferred one
(see~\longdelivref{component-architecture}{portability-cygwin}).

Packaging software for package repositories is rarely the
responsibility of the software developer, and indeed, outside a few
exceptional cases, \ODK software is not packaged by \ODK members. %
While not our direct responsibility, it is nevertheless of
primary importance that our software is made available through as many
package repositories as possible. %
Indeed, while many end-users (Windows users in particular) who install
software on their personal computers are perfectly happy with the
installer pattern,
the usual practice for Linux end-users and even more system
administrators is to install
software through package managers. \ODK's goal is also to reach a much larger
audience through shared servers and cloud infrastructures; the vast majority of them
run under Linux and are extremely reliant on package managers
(both OS-specific and OS-agnostic) for installing software.


This deliverable describes the efforts made by \ODK to have all its
software components packaged for the official repositories of all
major Linux distributions. %
While not precisely within the scope of this deliverable, we also
report on packaging for OS-agnostic repositories, and on the impact on
software design.

A different packaging issue is having a package system \emph{for}
\ODK software, in the same vein as language-specific package
systems. %
For example, \GAP has developed a package format, a package manager,
and maintains an official repository, so that users can publish and
share their code written in \GAP with other users. %
Since this aspect of packaging is inextricably linked with the former,
we also report on it.


\section{Linux packages for \ODK components}

The goal of this deliverable was to have official packages for all
major Linux distributions for all \ODK components. %
Before the start of \ODK, availability varied greatly among
distributions and software. %
To begin, it is necessary to define what is meant by ``major''
distribution; in an inevitably biased way, we chose to target the
following distributions:

\begin{itemize}
\item Debian, Fedora: for their popularity among advanced end-users
  and system administrators;
\item Ubuntu: for its popularity among beginners and casual users, as well as
  in the server and cloud space;
\item Arch, Gentoo: for their popularity among power-users.
\end{itemize}

Being the lone project depending upon all other \ODK components, \Sage
is automatically the most challenging to package. %
Hence, packagers for \Sage are usually, though not always, also
responsible for packaging most \ODK components within a
distribution. %
Table~\ref{tab:maintainers} summarizes the currently active package
maintainers for each of the distributions above; with the exception of
Bill Allombert, and some occasional members of the Debian Science
group, all package maintainers are external to the \ODK project.

\begin{table}
  \centering
  \begin{tabular}{p{0.5\textwidth} | p{0.5\textwidth}}
    Distribution & Maintainers\\
    \hline
    Debian, Ubuntu & Bill Allombert, Debian Science\\
    Fedora & Paulo César Pereira de Andrade \emph{et al.}\\
    Arch & Antonio Rojas, Felix Yan\\
    Gentoo & François Bissey \emph{et al.}\\
    Conda-forge & Isuru Fernando \emph{et al.}
  \end{tabular}
  \caption{Package maintainers for various distributions}
  \label{tab:maintainers}
\end{table}

With the exception of Debian and Ubuntu, all distributions above have
a history of regularly packaging \Sage (and thus all its dependencies)
since at least 2013, well before the start of \ODK. %
Hence, the strategy adopted for this deliverable was to have \Sage
packaged for Debian, in collaboration with the Debian Science team,
with all \Sage dependencies and packages for Ubuntu automatically
following from this work.

Debian has a two year long \emph{release cycle}, each cycle beginning at the
middle of odd numbered years. %
This means that every two years the community decides on which
packages should be included in the next stable release, at which point
a \emph{freeze} happens and packages within a release can only be
updated to fix bugs and security issues. %
``Missing the train'' for one release cycle means that software
cannot enter Debian for the next two years. %
Table~\ref{tab:debian} summarizes the Debian releases that happened
during the 2015-2019 period.

\begin{table}[ht]
  \centering
  \begin{tabular}{c c c}
    Codename & Number & Date\\
    \hline
    Jessie & 8 & April 26th, 2015\\
    Stretch & 9 & June 17th, 2017\\
    Buster & 10 & July 6th, 2019
  \end{tabular}
  \caption{Debian releases during the \ODK project}
  \label{tab:debian}
\end{table}

Ubuntu has a similar two year long \emph{long-term-support (LTS)}
release cycle, but also a shorter six-month cycle for quicker
releases, thus allowing more chances to release a package.

\subsection{Status before \ODK}

At the start of \ODK, only packages for \PariGP and \GAP were
available in Debian/Ubuntu, thanks to the continued efforts of Bill
Allombert, a long time member of the Debian community. %

Information on the packages available at the end of 2015 are summarized in
Table~\ref{tab:odk-2015}; for succinctness, we only list the four
core computer algebra systems (CAS) distributed by \ODK, leaving libraries out
of the picture.%
\footnote{Libraries typically have fewer dependencies and are easier to
  package.} %
To give an idea of the complexity of the task, we list the number of
packages in the dependency tree of each software package (i.e., the total
number of packages that would get installed on a fresh Debian system
with no other software than the package manger); we do the same for
the number of \emph{dependent} packages (packages that depend on the
software).

\begin{table}
  \centering
  \begin{tabular}{l | c | c | c | c | c}
    Software & Stable version & Version in Debian & Maintainer & Dependencies & Dependents \\
    \hline
    \GAP & 4.7.8 & 4r7p5-2 & Bill Allombert & 13 & 9\\
    \PariGP & 2.7.4 & 2.7.2-1 & Bill Allombert & 36 & 6\\
    \Sage & 6.8 & --- & --- & --- & ---\\
    \Singular & 4.0.1 & --- & --- & --- & ---\\
  \end{tabular}
  \caption{\ODK packages in Debian Jessie, September 1, 2015. Source
    \url{http://snapshot.debian.org/}.}
  \label{tab:odk-2015}
\end{table}

\Sage (version 3.0 at the time) had been packaged in Debian for the last time
in 2010, but fell out of the official distribution due to incompatible
dependencies. %
Ubuntu users also had access to a \Sage package contributed by
Jan Groenewald from AIMS on an unofficial channel. This package also installed
\Singular and other dependencies alongside; however, for the sake of
simplicity it was monolithic and offered
little in terms of compatibility and integration with the rest of the
system.

\subsection{Current status}

The first occasion to have all \ODK components packaged for Debian was offered
by the \emph{freeze} for the Debian 9.0 (codenamed "Stretch") release in 2017. %
Thanks to the intense collaboration between \ODK and the Debian
Science team, the occasion was not missed, thus achieving the goal of
the deliverable 24 months in advance of the projected deadline. %
Table~\ref{tab:odk-2017} summarizes information on the packages
available in Debian Stretch. %
The same packages were added to the official Ubuntu repositories at
the same time.

\begin{table}
  \centering
  \begin{tabular}{l | c | c | c | c | c}
    Software  & Stable version & Version in Debian & Maintainer & Dependencies & Dependents \\
    \hline
    \GAP      & 4.8.8 & 4r8p6-2         & Bill Allombert &  14 & 20\\
    \PariGP   & 2.9.3 & 2.9.1-1         & Bill Allombert &  39 &  9\\
    \Sage     &   8.0 & 7.4-9           & Debian Science & 653 &  3\\
    \Singular & 4.0.3 & 1:4.0.3-p3+ds-5 & Debian Science &  21 &  4\\
  \end{tabular}
  \caption{\ODK packages in Debian Stretch, August 31, 2017. Source
    \url{http://snapshot.debian.org/}.}
  \label{tab:odk-2017}
\end{table}

The next Debian release cycle came to completion in July 2019; all
\ODK packages were updated and improved for the occasion, requiring
tight collaboration between the involved communities, with a
marked simplification of the dependency tree. %
Table~\ref{tab:odk-2019} summarizes information on the packages
available in Debian 10 (codenamed "Buster").

\begin{table}
  \centering
  \begin{tabular}{l | c | c | c | c | c}
    Software  & Stable version & Version in Debian & Maintainer & Dependencies & Dependents \\
    \hline
    \GAP      & 4.10.2 & 4r10p0-7        & Bill Allombert &  15 & 13\\
    \PariGP   & 2.11.2 & 2.11.1-2        & Bill Allombert &  41 &  9\\
    \Sage     &    8.8 & 8.6-6           & Debian Science & 593 &  3\\
    \Singular &  4.1.2 & 1:4.1.1-p2+ds-3 & Debian Science &  23 &  4\\
  \end{tabular}
  \caption{\ODK packages in Debian Buster, August 29, 2019. Source
    \url{http://snapshot.debian.org/}.}
  \label{tab:odk-2019}
\end{table}

\subsection{Beyond Linux packages}

Having achieved the goals of the deliverable largely in advance, we
took the occasion to explore other package repositories. %
At the \ODK workshop in Cernay (see~\delivref{dissem}{workshops-4}),
besides experiments with GUIX and Nix packages, a sprint took place to
have \Sage packaged for Conda-forge. %
At the time of writing, the latest \Sage 8.8 and all its dependencies
are available on Conda-forge for the Linux and MacOS platforms. %
This packaging effort is still experimental, and will only become
stable after \Sage is officially migrated to Python 3 (see below).

In addition, and according to our development best practices, most of
the new software developed by OpenDreamKit has been packaged for e.g.
PyPI if not Conda; this includes for example JOOMMF/UberMag
(\longdelivref{dissem}{oommfnb-vre-deliver}), nbval, nbdime
(\longdelivref{UI}{jupyter-test}), pypersist
(\longdelivref{dksbases}{persistent-memoization}), etc.

\section{Work accomplished}

If a piece of software can be installed on a system, then, in principle, it can
be packaged for it. %
And, indeed, a \Sage package for Ubuntu had been distributed through
an unofficial channel for many years.

The difficulty is not packaging software, as much as packaging it
\emph{properly}. %
Indeed, a package system tries to fulfill several goals:
\begin{itemize}
\item Code reuse: by having a separate package for a component shared
  by several software, the component can be installed only once and
  reused many times. This obviously saves disk space and, in the case
  of shared libraries, it also saves memory and cpu since the library
  needs to only be loaded once in shared memory.

\item Compatibility: different software may depend on different
  versions of the same component, and break when incompatible versions
  are used. By ensuring that all software within a release is mutually
  compatible, Linux distributions provide a smooth experience for the
  user.

  Nevertheless, sometimes it is impossible to satisfy all dependent
  software with a single version of a component. In this case, it is
  not unusual for Linux distributions to provide two different
  versions of the same component installable side-by-side (e.g.,
  Python 2 and Python 3 are both available in all Linux distributions
  at the time of writing). While this approach is in tension with the
  goal of code reuse, it is sometimes the most pragmatic one.
  
\item Flexibility: software may rely on two components to do the
  same job, using indifferently one or the other according to which is
  available in the system. By having separate packages, users can
  freely choose which component they prefer to have installed, and even
  change their mind later, without breaking dependent software.
\end{itemize}

It is clear that, the higher the number of dependencies, the higher
the complexity of packaging a software is. %
In the case of \ODK, a single software package may interact with hundreds of
packages, as illustrated in
Tables~\ref{tab:odk-2015}--\ref{tab:odk-2019}. %

The legacy unofficial \Sage package for Ubuntu,
installing all of \Sage and its dependencies in a handful of huge
packages, was not playing by the rules, which explains why it was not
included in the official repositorites. %
Furthermore, the strong traditions of the Debian community (length of
the release cycle, large quantity of scientific software already
packaged, reluctance to host alternative/modified versions of
packages), exacerbated the problem and made packaging \Sage and its
dependencies extremely challenging.

The successful packaging of all \ODK components in Debian/Ubuntu was
thus the result of several actions undertaken by the \ODK members in
concert with the concerned communities. %
We list below the most important actions.

\begin{description}
\item[Workshops dedicated to packaging] A series of workshops
  specifically targeted at packaging was organized in Cernay (near
  Paris):
  \begin{itemize}
  \item \Sage days 77, April 4--8, 2016, see \delivref{dissem}{workshops-1};
  \item \Sage days 85, April 30 -- May 4, 2018, see \delivref{dissem}{workshops-3};
  \item \Sage days 101, June 17--21, 2019, see \delivref{dissem}{workshops-4}.
  \end{itemize}
  They were the occasion to gather the various actors involved in
  packaging \ODK components, exchange ideas, run experiments, and
  sprint on development. %
  A mailing list \texttt{sage-packaging@googlegroups.com} was created
  to keep exchanging on issues related to packaging after the
  workshops.

\item[Limiting \emph{patches}] When a project depends on an external
  component, it may need, in some instances, to \emph{patch} the
  external dependency, i.e., create a slightly modified version of the
  official component that interfaces better with the project. %
  Several different reasons may lead a project to patch a dependency:
  \begin{itemize}
  \item A bug is found in the dependency, but its developers either
    disagree with the proposed fix, or have a timeline for releasing
    it that is incompatible with that of the dependent project.
  \item The dependency makes assumptions about its environment (file
    paths, system resources, \dots) that are unsatisfied within the
    dependent project.
  \item Two or more dependencies are incompatible, but the dependent
    project needs all; a patch to one of the dependencies may help
    them coexist.
  \end{itemize}

  While patches may be unavoidable, they complicate the work for
  package maintainers: which version of a component should be included
  in the official repository, the original or the patched one? %
  In some cases, it may be necessary to include both, which negates
  the benefits cited previously. %
  For this reason, Linux distributions usually discourage projects
  patching dependencies, keeping that prerogative to themselves. %

  \Sage, owing to its large number of dependencies, has traditionally
  behaved very badly in this respect, maintaining a large set of
  patches. %
  An great amount of work has gone into \emph{pushing patches
    upstream} (i.e., having them accepted in the original project)
  whenever possible, and in removing useless or avoidable patches.
  
\item[Updating dependencies] As dependencies evolve, it is necessary
  to ensure that the dependent project is as compatible as possible
  with the latest version,
  failing which package maintainers would be faced with the dilemma
  of updating the dependency and thus breaking the dependent, or
  keeping the dependency at an obsolete version.

  While in most cases updating a dependency requires little to no
  work, major version changes may require considerable effort. %
  Notable examples for \Sage were the upgrade of \Singular from
  version 3 to 4, and the upgrade of Python from version 2 to 3. %
  The former was completed in mid-2016, one year and a half after the
  initial release of \Singular 4; the latter is still underway (see
  next section).
  
\item[Modularization of \ODK software] Identifying and cutting out
  meaningful components from a large piece of software can also help
  make packaging easier; delegating work even more.

  With the help of \ODK, the various software projects undertook a
  broad effort in modularizing their code base, leading to a
  considerable reduction in duplicated efforts and improved code
  quality. %

  Important examples are the adoption of Jupyter as unified user
  interface, which in the case of \Sage paved the way to the
  deprecation of the old \texttt{sagenb} notebook. %
  The development of \libGAP (see~\longdelivref{hpc}{GAP-HPC-report}) let
  \Sage depend on an officially supported C API to \GAP, rather than
  on fragile bindings developed internally. %
  Finally, spinning the projects CySignals and CyPari off mainline
  \Sage simplified the process of packaging it, and allowed other
  projects outside \ODK to depend on them without having to import the
  full \Sage. Work was also carried out to reduce customization
  of the Sphinx documentation system through upstream contributions
  and low-level contributions to Python and Cython to standardize
  their introspection API, leading to simplified Sphinx's logic
  (see~\longdelivref{UI}{sage-sphinx}).

\item[Providing alternate workflows for user-contributed code]
  A different issue, mostly relevant to full-fledged computer
  algebra systems, is the packaging of user-contributed code
  \emph{for} the system. %
  Indeed, there is a growing demand for workflows that facilitate easy
  sharing of code developed by the users of a system, and this demand can
  hardly be fulfilled by a distribution's package repository. %
  In the same vein of language-specific package repositories, the need
  was felt to have repositories for \Sage and \GAP.%
  \footnote{Due to its smaller and more specialized user base, a
    repository for \PariGP seems unjustified for the moment.} %
  
  \GAP already had a package system before the start of \ODK, but this
  was the occasion to improve the existing tools and automatize the
  workflows for the users (see~\longdelivref{hpc}{GAP-HPC-report}).

  \Sage used to encourage its users to directly contribute code to the
  mainline, however this could be a frustrating experience for
  beginners, who had to learn the coding standards of \Sage and go
  through the standard review process; it also led in some cases to
  \emph{bit-rotting}: old unmaintained code in \Sage with no designated
  maintainer. %
  Some \Sage developers advocated distributing user code in the form
  of SPKGs, the internal format \Sage uses to pack dependencies. %
  Although this is a valid method, the responsibility of hosting the
  SPKG still falls on the user, and the available tools are rather
  limited. %
  
  Lately, a consensus has developed in the community on encouraging
  users to use the standard Python packaging tools, and host their
  packages on PyPI, the main repository for Python projects. %
  Support for installing packages from PyPI has been integrated in
  \Sage via the \texttt{sage -pip} command, and documentation and
  example packages%
  \footnote{\url{https://github.com/sagemath/sage_sample}.}%
  have been developed to guide the user in the process.

  While these improvements do not directly impact packaging for Linux
  distributions, the added possibilities for the end-users enable
  faster growth of the projects, at no additional expense for package
  maintainers.
\end{description}

\section{The future}

The quantity of effort expended to have software properly packaged raises
the question of its sustainability. %
Indeed, many tasks are recurring and need to be iterated at each new
release cycle: updating dependencies, integrating patches upstream,
maintaining interfaces, \dots %
While the end of \ODK will be felt, we argue that the efforts undertaken
until now will not be lost, and will greatly simplify the workflow of
package maintainers.  %
Indeed, the need for some of the actions undertaken had been felt for
a long time, and \ODK offered the occasion to address many
long-standing issues. %
With \ODK coming to an end, many foundational issues have already been
solved, good practices have been taken on by the community, and only recurring tasks need to be dealt with for updating
packages. Last, but not least, the community had feared for the
longest time that proper packaging of such complex software was out of
reach, and this psychological barrier has been taken down.

Among the major foundational issues still open is the migration of
\Sage to Python 3.%
\footnote{Python 2 reaches its end of life in 2020. After that point,
  it will become nearly impossible to package a Python 2 version of
  \Sage for any system.} %
Fortunately, thanks to a titanic effort involving hundreds of entries
in the issue tracker,%
\footnote{\url{https://trac.sagemath.org/ticket/26212}.}  with the
help of members of \ODK,%
\footnote{We are grateful to F. Chapoton for leading this effort.}
\Sage is now compatible with Python 3, and only a few months away from
officially migrating from Python 2.

The move to Python 3 will also be the occasion to advance on several
projects that had stalled because of the limitations of Python 2. %
In particular, packaging for Conda-forge will be simplified after the
shift.%, and it will be easier to build Windows-compatible Conda packages
Conda packages will also enable easier building of JupyterHub and
Binder-compatible images, further stimulating the development of
workflows based on \ODK components.

Not directly related to packaging, the move to Python 3 will also
facilitate gradual typing through type annotations, which will pave
the way to more static analysis (for more robustness and possibly
performance) and will facilitate semantic integration of systems as
explored in WP6.
\end{document}

%%% Local Variables:
%%% mode: latex
%%% TeX-master: t
%%% End:
