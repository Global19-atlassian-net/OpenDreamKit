\hypertarget{status-report}{%
\section{Status report}\label{status-report}}

One of the tasks for WP3 is improving the development workflows of mathematical
software in order to make them more accessible to contributors at all levels,
from experts on those projects, as well as expert developers coming from other
projects, to students and non-programmers (e.g.,~technical writers who might
wish to contribute documentation improvements).

For this deliverable we focused on the \Sage project, and making it easer for
newcomers to the project to provide contributions.  The existing Sage
development community already has a well-established workflow and culture
surrounding its existing development tools--in particular the source code
hosting and issue tracking service \Trac.  This works well for experienced
Sage developers, but also creates certain barriers to contribution, especially
from would-be contributors who are new to open source development.  Therefore
we sought solutions to better integrate Sage's \Trac-based workflow with more
modern and better known development platforms such as \GitHub and \GitLab,
without immediately breaking the Sage project's existing workflow.
Nevertheless, we believe that this will also open the doors to long-term
improvements in Sage's development workflow, as the advantages of using these
new tools become more apparent.  Part of this work also serves as a
prerequisite to improvements in Sage's continuous integration practices, which
are being explored more in
\longdelivref{component-architecture}{multiplatform-buildbot}.

An aspect that Sage (and indeed any open source mathematics software) prides
itself on over closed-source alternatives is the ability to peer inside the
software to see how it works.  It is not a black box and one does not have to
take its results for granted--it is possible to look at the source code in
order to understand how a particular algorithm works, or even modify and
improve it.  To this end we sought ways to more tightly couple Sage's API
documentation to its source code--and through access to the source code,
encourage contribution of improvements.

To these ends, this deliverable includes the following improvements to Sage's
development platform and documentation:
\begin{enumerate}
\item Enable users with existing GitHub credentials to log into Sage's
    Trac site.
\item Create a presence for Sage on the GitLab collaborative source code
    sharing service, including the ability to accept patches to Sage's
    documentation and source code in the form of {\em merge requests}
    (a.k.a.~{\em pull requests}).
\item Link \GitLab merge requests to tickets in Sage's Trac issue tracker in
    a manner that fully integrates with Sage's existing development workflow.
\item Create a direct path from Sage's online API documentation to the source
    code, and from the source code directly an interface to edit the source
    code and submit a patch (as a merge request) all without the user having to
    leave their web browser.
\end{enumerate}



\hypertarget{changes-to-deliverable}{%
\subsection{Changes to the deliverable\label{changes-to-deliverable}}}

This deliverable was originally titled "Integration between SageMathCloud and
Sage's TRAC server".  The vision behind this came from the fact that \SMC (now
\cocalc) provides a new way to get started on Sage development.  Compiling Sage
and its dependencies requires a significant amount of computing power, and also
a number of development software prerequisites that can be hard to set up
initially, especially in the context of a one or two day workshop.  \cocalc
provides an environment that is already suited to building and running Sage in
a cloud computing environment, not limited by the user's local machine.

So the thought was that users ought to be able to do development on \Sage
directly in a \cocalc environment, and have a way to post any new code they
wrote directly to Sage's Trac system (hereafter referred to as Sage-Trac).

It should be said, however, that the existing command-line development tools
for Sage (e.g.~the community-maintained {\tt git trac} utility) already provide
integration with \Trac in a manner independent of what computing environment
the developer is working in.  So it was unclear exactly what the advantage
would have been to adding features to \cocalc solely for the purpose of
integrating with Sage-Trac, especially since those features would be very
specific to Sage and would not integrate well into the direction \cocalc has
moved in, as a more general-purpose computing environment.

It was also unclear how many people were actually using \cocalc as their
primary platform for Sage development, while it is certain that most of the
core Sage developers are not doing so.  So it seemed a better focus of our
efforts to instead to open Sage's development workflow up to the use of more
modern development platforms (see the next section for discussion of how we are
integrating \GitLab into the workflow).  We expect this will have benefits in
the future--more modern development tools integrate well with services like
\GitHub or \GitLab out-of-the-box, whereas \Trac, while an excellent system, is
less and less popular, requiring extra work on the part of the Sage community
to integrate into newer development tools.

We have also had some discussion about how to integrate issue tracking and
source code contributions into VREs in general (not just \cocalc), and while no
consensus has yet emerged as to what form this would take, it would most likely
integrate best with modern code-sharing services like \GitHub and \GitLab.


\hypertarget{description-of-the-achievements}{%
\subsection{Description of the
achievements}\label{description-of-the-achievements}}

\hypertarget{trac-github-login}{%
\subsubsection{Login to Trac via GitHub usernames}\label{trac-github-login}}
A perennial problem in managing the Sage community's self-hosted \Trac site has
been dealing with user registration and authentication.  As with many
bug-tracking sites, Sage-Trac requires users to register and log into the site
in order to post bugs and feature requests.

Allowing anyone on the internet to register an account--not to mention allowing
anonymous users to post to the site--regularly opened it up to a flood of spam.
Even adding CAPTCHAs to the registration process was not fool-proof, as
CAPTCHAs are being defeated more and more by either machine learning or cheap
manual labor.  Therefore, for several years the Sage community has resorted
to a process of manual user registration: potential users of the site would
send an e-mail to a dedicated address with a manually written application form,
and eventually a site administrator--if the application appeared genuine--would
approve the application and assign the user a username and temporary password,
again over e-mail.

This process has worked, and does not generally leave anybody out.  But it is
still a slow, and labor-intensive process.  One would not be surprised if it
has discouraged potential bug-reports, with users deciding it was not worth
their time, or forgetting about the issue while waiting for their account
registration to be approved.

Therefore, we sought to make logging into Sage-Trac a more painless and
instantaneous process.

\GitHub, with its over 40 million users as of writing, is the largest source
code hosting site in the world.  Many people, even who are not full-time
software developers, have a GitHub account if they have worked at all
peripherally with open source software (if nothing else, in order to submit bug
reports to projects).  By making it possible to log into Sage-Trac with one's
existing GitHub account, this opens Sage development up instantly to any of
those 40+ million users without additional effort.  To date, this has not
contributed any spam to Sage-Trac either, as we are able to leverage GitHub's
own spam management.

We still also allow existing accounts to log in, and maintain the manual
registration system so that users who do not already have, or prefer not to
obtain GitHub accounts may be registered.  However, in the four months since
launching this new feature on February 27, 2018, nearly one fourth (31 out of
130) of the users who submitted new tickets were logged into the site using
GitHub.  Of the other 130, only a handful were new manual registrations, while
the rest were using accounts they already had on the system before February 27.
We expect the number of contributions from GitHub users to grow, and will
consider adding the ability to log in via other external authentication
providers, such as \GitLab and Google.


\hypertarget{gitlab-trac-integration}{%
\subsubsection{\GitLab-\Trac integration}\label{gitlab-trac-integration}}

While making it easier for new contributors to \Sage to log into its \Trac site
eliminates one barrier to contribution, another barrier is learning its
somewhat unique development and source code contribution workflow.  While this
workflow is well-documented, and custom tooling is provided to make it easier
(e.g.~the {\tt git trac} command for posting Git branches directly to Trac
from the command line), it is still not an entirely familiar workflow for
anyone new to Sage--experienced developers and novices alike.

Novices in particular--especially those who became involved in open source
development within the last five years--will tend to be more familar with what
is known as the "GitHub
Workflow"\footnote{\url{https://guides.github.com/introduction/flow/}} as it
was popularized through use of GitHub.  Thus, it would make contributing to
Sage more accessible if were possible to do via a \GitHub-like
workflow.\footnote{In fact, Sage's existing process is not that different from
the GitHub workflow, but small differences in semantics and user experience
make this less immediately obvious.}  As we will see in the next section, this
also makes it easier to go directly from browsing Sage's documentation to
making small source code and documentation contributions.

In order to accomplish this, we might have used a mirror of Sage's Git
repository on GitHub (indeed, such a mirror already exists).  However, for this
purpose we chose instead to focus on GitLab--a competitor to GitHub that
provides a similar product, but with an {\em open core} business model, meaning
that the software that runs GitLab is provided for free under an open source
license, but the company maintains optional "enterprise" features under a
separate proprietary license.  A few reasons we chose GitLab for this purpose
over GitHub include:

\begin{enumerate}
\item Many members of Sage's community have strongly-held political objections
    to relying on closed-source commercial software for development, and would
    be reluctant to adopt GitHub into their workflow (the aforementioned GitHub
    login for Trac being acceptable due only to it being optional).  We hope
    that GitLab, itself being {\em mostly} open source, will be a more
    politically viable alternative. Sage itself being an open source project,
    we are happy to work hand-in-hand with other open source projects.
\item GitLab is both a cloud-based repository hosting service (in the form of
    GitLab.com, as well as the GitLab software itself.  Although for now we
    will host a Sage project on GitLab.com, should the need arise for any
    reason we can easily self-host the GitLab {\em software} and transfer the
    Sage project to a self-hosted GitLab service. Indeed, the Sage community
    has long prided itself on being almost entirely independent of commercial
    products for hosting its development tools, and while using GitLab.com is
    convenient (less time spent on server maintenance that could be spent
    improving Sage itself), it is comforting to know that the possibility
    exists.
\item GitLab.com allows users to log in with their GitHub usernames, Google
    accounts, and other third-party authentication providers (in addition to
    registration directly with GitLab).  Thus, although GitHub has a wider
    existing user base, all users with GitHub accounts can log into GitLab
    without significant effort.
\item GitLab has some features that are not present on GitHub.  In particular,
    GitLab has a powerful built-in continuous integration and deployment
    platform that we have found well-suited for Sage.  Our efforts in using
    GitLab's continuous integration/deployment tools will be described more
    in \longdelivref{component-architecture}{multiplatform-buildbot}.
\item Although GitLab the company is currently based out of San Francisco
    (albeit with only one employee at its San Francisco office) the project
    was started in Europe (Ukraine) and a large portion of its all-remote
    employees are based in Europe.
\end{enumerate}

For these reasons, among others, we have set up a new
mirror{\footnote{\url{https://gitlab.com/sagemath/sage}} of Sage's Git
repository on GitLab, with the ability to accept contributions in the form of
{\em merge requests}, which are the same as what GitHub calls "pull requests".

Note: The choice of GitLab for hosting a Sage project may appear to stand in
contrast with the previous section of this deliverable, which added
authentication to Sage-Trac via {\em GitHub}.  However, as noted above, GitLab
also allows authentication with GitHub accounts, taking advantage of GitHub's
existing wide user base.  Therefore, we focused primarily on GitHub-based
authentication.  In the future we would also like to add the ability for users
to authenticate with Sage-Trac via GitLab, if there is demand for it.

The challenge in opening up Sage to accepting contributions via GitLab was
doing so in a way that did not overly disrupt Sage's existing Trac-based
development workflow.  To do this, we implemented a plug-in for Sage-Trac that
uses GitLab's API to automatically open a {\em ticket} on Sage-Trac for each
merge request opened on GitLab.  That way, Sage developers watching Sage-Trac
for new contributions are notified even without having to go to GitLab.  The
proposed code changes in the merge request--which are hosted on GitLab--are
also automatically synchronized to Sage's main Git repository, so that all of
Sage's existing maintenance tools continue to work as normal even for
contributions that came originally through GitLab.  When a ticket is {\em
closed} on Sage-Trac (i.e.~the changes were accepted, or rejected) the
corresponding merge request on GitLab is also automatically closed.  In this
way, veteran Sage developers may, if they wish, accept contributions through
GitLab without ever actually having to go to GitLab.

That said, we hope that most of the Sage development community {\em will}
consider using GitLab.  In particular, we believe that the more modern code
review tools provided by GitLab will prove too compelling to ignore.  There
is a small risk associated with this: Many code change tickets generate 
discussion--sometimes a significant amount--in the form of code review comments
or sometimes even debate.  If an issue is tracked on both Sage-Trac and on
GitLab, there is a risk of that discussion becoming fragmented between the two
sites and difficult to follow.  This can be mitigated either through policy
(e.g.~keep discussion of a single issue on one site or the other, but not
both), or in the form of a technical solution (e.g.~enhance the existing
synchronization plug-in to synchronize discussions between the two sites; this
is non-trivial, however).

Either way, there is prior art for this process of adopting a modern code
sharing site while maintaining a legacy issue tracking system.  Most notably,
the CPython project (the reference implementation of the Python language and
its standard library--what is typically referred to as just "Python") switched
to a similar model beginning in February 2017: Issues are still tracked
primarily on the project's legacy bug tracking system--which carries in it a
wealth of historical knowledge and lingering open issues--while bug fixes and
new features can be submitted as pull requests on GitHub which are
automatically linked to their corresponding issues on the legacy system.  Since
then, over
6500\footnote{\url{https://github.com/python/cpython/pulls?utf8=\%E2\%9C\%93\&q=is\%3Apr+is\%3Aclosed+is\%3Amerged}}
merge requests have been accepted through GitHub.  In an informal
query\footnote{\url{https://mail.python.org/pipermail/python-dev/2018-February/152200.html}}
to the CPython core developers one year after switching to the new hybrid
workflow, its primary architect, Brett Cannon, asked how the new workflow was
working and the response was overwhelmingly positive.  Frequently cited as a
benefit were the several bots that were created by the community to help manage
pull request submissions.  Many of these bots would be useful for Sage as well,
and although they were created to run on GitHub, the GitLab API is
similar-enough that these bots could be ported to GitLab with minimal effort.
There was also a general interest (including by Python's creator Guido van
Rossum) to move forward with migrating all issues to GitHub, and setting the
legacy bug tracker to read-only; the details of this migration, however, remain
a subject of debate.\footnote{\url{https://lwn.net/Articles/754779/}}  We
believe a similar outcome for Sage's partial migration to GitLab would be
desirable, but it remains to be seen how well it will be adopted by the Sage
community.


\hypertarget{conclusion}{%
\subsection{Conclusion}\label{conclusion}}
