\section*{\texorpdfstring{Deliverable description, as taken from Github
issue
\href{https://github.com/OpenDreamKit/OpenDreamKit/issues/61}{\#61} on
2017-02-28}{Deliverable description, as taken from Github issue \#61 on 2017-02-28}}\label{deliverable-description-as-taken-from-github-issue-61-on-2017-02-28}

\begin{itemize}
\tightlist
\item
  \textbf{WP3:}
  \href{https://github.com/OpenDreamKit/OpenDreamKit/tree/master/WP3}{Component
  Architecture}
\item
  \textbf{Lead Institution:} Université Paris-Sud
\item
  \textbf{Due:} 2016-02-28 (month 18; originally month 6)
\item
  \textbf{Nature:} Report
\item
  \textbf{Task:} T3.6
  (\href{https://github.com/OpenDreamKit/OpenDreamKit/issues/55}{\#55}):
  Document and modularise SageMathCloud's codebase
\item
  \textbf{Proposal:}
  \href{https://github.com/OpenDreamKit/OpenDreamKit/raw/master/Proposal/proposal-www.pdf}{p.~43}
\item
  \textbf{\href{https://github.com/OpenDreamKit/OpenDreamKit/raw/master/WP3/D3.2/report-final.pdf}{Final
  report}}
\end{itemize}

SageMathCloud (SMC) is both an open source software project
(\url{https://github.com/sagemathinc/smc}) and an online instance of
that software (hosted at \url{https://cloud.sagemath.com/}) that
provides an interactive, collaborative environment for teaching and
research in science, technology, engineering, and mathematics. See
e.g.~D2.3
(\href{https://github.com/OpenDreamKit/OpenDreamKit/issues/43}{\#43})
for a description of this emergent technology.

SMC predates OpenDreamKit (ODK) and acts as prototypical example of VRE
that can be built from the ecosystem OpenDreamKit aims at fostering. In
particular, SMC is one of the main channels through which some of the
most important technologies of ODK, such as Jupyter and SageMath, are
distributed on-line. It makes it a good mean to distribute some of the
newly developed ODK features, like the new Jupyter kernels of D4.4
(\href{https://github.com/OpenDreamKit/OpenDreamKit/issues/93}{\#93}).
It is very probable that many users will benefit from some of the ODK
new developments through SMC. Reciprocally, the inner technologies of
SMC are of special interest to ODK developers: they show advanced uses
of cutting-edge web technologies and explore new leads that could
inspire the work we do in ODK.

For all these reasons, is has been planned since the beginning for ODK
to collaborate actively with SMC. For example, in D2.4
(\href{https://github.com/OpenDreamKit/OpenDreamKit/issues/44}{\#44}),
we have developed a short course for educators who wish to adopt SMC and
related ODK technology in order to enhance their teaching.

In this deliverable, we start exploring the main layers of SMC's backend
code and give a general overview of its functioning. The material we
have produced can directly help the platform attract more developers.
One of the expected follow-up is an easy install for a local version of
SMC especially designed for development which could be part of upcoming
D3.5
(\href{https://github.com/OpenDreamKit/OpenDreamKit/issues/63}{\#63}).
The long term goal however is to understand the extent of a full install
of a SMC instance on a server or cluster: How hard is it? What is the
total cost of ownership? Is it a viable solution for institutions of
various scales to deploy and run a local SMC instance?
