\section{Context}

There is a fast growing open source ecosystem around the Jupyter [1]
technology for interactive data science and scientific computing; the
flagship is the Jupyter notebook: web-based narrative documents mixing
live computations, equations rich texts and medias, interactive
visualization, etc.

Although rapidly evolving, this is a mature ecosystem, based on open
and acclaimed technologies (e.g. docker/kubernetes, ...), and already
used by hundreds of thousand of people, as much in academia as in the
industry (e.g. the Bloomberg company is actively contributing to
Jupyter to support its internal needs).

\section{Aim}

At this stage, a low hanging fruit with high return value would be to
leverage the access to this technology to academics in Europe
(possibly beyond) by *deploying, hosting, and maintaining* Jupyter
based web services like JupyterHub, tmpnb, or mybinder. All of them
are TRL-8 level, with instances already deployed. Some may require
customization or some specific development for integration in the
EGI/EUDAT/... infrastructure (authentication and data sharing,
container orchestration, ...).

In fact, the need for leveraging the access to such services is so
pressing that we would be very interested in actually starting a
collaboration as soon as possible, without waiting for the application
to be written / accepted.

Some more details are available in our Use Case description [2] we had
presented at the "Design your E-Infrastructure workshop" in Krakow.

\section{OpenDreamKit's involvement}

The OpenDreamKit EInfra-9 project aims at fostering a flexible toolbox
from which researchers in computational mathematics can easily build
and deploy VRE's taylored to their specific needs. Jupyter (together
with SageMathCloud) is its backbone for all the collaborative
workspaces, user interfaces, and web-services aspects. OpenDreamKit is
particularly involved in the collaborative, reproducibility, and ease
of deployment features. We effectively have on board many of the
academic sites in Europe involved in the Jupyter project. Since
OpenDreamKit focuses on the software aspects, our actions would
complement each other well with a EInfra-12 (A) proposal.

In this context, I see OpenDreamKit as a mere umbrella over a
collection of great people. If it would be useful for the EGI/EUDAT
proposal to use the OpenDreamKit "brand", that's great. If it's best
for the would be consortium to instead build a specific "Jupyter
group" mixing people from within OpenDreamKit and outside, that's very
fine too.  OpenDreamKit's spirit is strongly rooted on a «by users,
for users» approach, and our main interest for participating is that
we critically need the outcome.

\section{Proposed services}

\subsection{tmpnb}

Description: provides anonymous users with individual, on-demand Jupyter
notebook servers, which are later culled. See https://github.com/jupyter/tmpnb
for more details.

Rationale: tmpnb supports demonstrations and easy experimentation in
Jupyter notebooks.

\subsection{mybinder}

Description: builds a container for a GitHub repository containing Jupyter
notebooks and an environment description (such as a Dockerfile), then starts a
cloud server where the user may interact with the notebooks. See http://mybinder.org/
for more details.

Rationale: mybinder enables trivial to use deployment and sharing of
Jupyter notebooks; fosters dissemination and reproducible research,
and connects computational resources with interesting use cases.
Its success proves that it has identified just the right service for
a critical need.

Issue: too successful and lacking computational resources

Goal: power up mybinder with additional computational resources to
leverage its use. Option 1: run an alternative instance of mybinder.
Better option: add a button "Run on EGI cloud" on the official
mybinder instance, enabling anyone with appropriate credentials to use
the EGI cloud to run the container.

% Not sure where to put this:
Enable easy sharing of large datasets to be used by notebooks in mybinder. Git
and GitHub, which are used to share notebooks and descriptions of the
environment to run them, are not optimised for large files.

\subsection{jupyterhub}

Description: gives authenticated users access to private, persistent Jupyter
notebook servers. See https://jupyterhub.readthedocs.io/ for more details.

Rationale: JupyterHub allows centralised management of Jupyter notebook servers
for a group of users. For instance, a lecturer may run a JupyterHub instance for
students taking their course, removing the need for students to install software
locally.

Goal: Run an instance of JupyterHub enabling anyone with appropriate
credentials to run jupyter notebooks on the EGI cloud.

\section{Relevance to EINFRA-12 (A) challenges}

% 1. The operation of a federated European data and distributed computing
% infrastructure for research and education communities will optimise the access
% to IT equipment and services

\subsection{Optimising access}

Integrating Jupyter notebooks into the proposed infrastructure would facilitate
access to IT resources for relatively small computational tasks with rapid
feedback to the user, in addition to the large batch jobs which are commonly
run on shared compute resources. In particular, these technologies encourage
open sharing of computational research methods, and reproducing and building on
published work.

% 2. All European researchers and educators are in equal footing to access
% essential resources

% 3. Partnerships with industrial and private partners

\subsection{Industry partnerships}

The Jupyter project already works with a number of industry partners, including
Google, Microsoft, Bloomberg and O'Reilly. This proposal would build on these
relationships.

% 4. Train people in research and academic organisations

% 5. Avoid the locking-in to particular hardware or software platforms

\subsection{Avoiding lock-in}

All of the software involved is not only open source, but built around open
protocols and file formats, allowing for easy interoperation.

% 6. More scientific communities will use storage and computing infrastructures
% with state-of-the-art services

\subsection{More scientific communities}

By presenting code in a more appealing format, Jupyter is helping to expand
the use programming beyond traditionally numerical fields, to become more
commonplace among biologists and social scientists, among others. Connecting
this interface to large scale computational resources would allow a diverse set
of communities to tackle interesting problems.

% 7. The open nature of the infrastructure will allow scientists, educators and
% students to improve the service quality

\subsection{Open infrastructure}

All of the Jupyter code is developed in the open, using the popular GitHub
code sharing site. The Jupyter community regularly receives improvements from
people using the software.

% 8. Increase the incentives for scientific discovery and collaboration across
% disciplinary and geographical boundaries. It will further develop the European
% economic innovation capacity and provide stability to the e-infrastructure.

\section{Information on innovation, dissemination and exploitation}
