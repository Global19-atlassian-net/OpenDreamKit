\section{Contact information}
\begin{itemize}
\item Name of applicant: Hans Fangohr
\item E-mail address: fangohr@soton.ac.uk
  \item Organisation: This application is sent under the "umbrella" of the
E-Infra9 project OpenDreamKit. We are proposing a series of services
around the Jupyter (previously IPython) project which can be grouped
as a WorkPackage itself or as tasks according to what the future
E-INFRA12 project consortium will prefer. The legal bodies that would
enter the consortium if this Expression of interest is accepted are:
\begin{itemize}
\item University of Southampton, United Kingdom
\item Simula XXX Min, can you complete, please
\end{itemize}
\end{itemize}
\section{Service and Activity descriptions}

There is a fast growing open source ecosystem around the Jupyter [1]
technology for interactive data science and scientific computing; the
flagship is the Jupyter notebook: web-based narrative documents mixing
live computations, equations rich texts and medias, interactive
visualization, etc.

Although rapidly evolving, this is a mature ecosystem, based on open
and acclaimed technologies (e.g. docker/kubernetes, ...), and already
used by hundreds of thousand of people, as much in academia as the
industry. High profile industrial users of, and contributors to, the
Jupyter ecostsystem include Bloomberg [2] and Microsoft [3].

\textbf{Aim}

At this stage, a low hanging fruit with high return value would be to
leverage the access to this technology to academics in Europe
(possibly beyond) by *deploying, hosting, and maintaining* Jupyter
based web services like JupyterHub, tmpnb, or mybinder. All of them
are TRL-8 level, with instances already deployed. Some may require
customization or some specific development for integration in the
EGI/EUDAT/INDIGO infrastructure (authentication and data sharing,
container orchestration, ...).

In fact, the need for leveraging the access to such services is so
pressing that we would be very interested in actually starting a
collaboration as soon as possible, without waiting for the application
to be written / accepted.

Some more details are available in our Use Case description [2] we had
presented at the "Design your E-Infrastructure workshop" in Krakow.

\textbf{OpenDreamKit's involvement}

The OpenDreamKit EInfra-9 project aims at fostering a flexible toolbox
from which researchers in computational mathematics can easily build
and deploy VRE's taylored to their specific needs. Jupyter (together
with SageMathCloud) is its backbone for all the collaborative
workspaces, user interfaces, and web-services aspects. OpenDreamKit is
particularly involved in the collaborative, reproducibility, and ease
of deployment features. We effectively have on board many of the
academic sites in Europe involved in the Jupyter project. Since
OpenDreamKit focuses on the software aspects, our actions would
complement each other well with a EInfra-12 (A) proposal.

Like the Jupyter project, OpenDreamKit's spirit is strongly rooted on
a «by users, for users» approach, and our main interest for
participating in the EGI/EUDAT/INDIGO consortium is that we critically
need the outcome.

\section{Proposed services}

\subsection{tmpnb}

Description: provides anonymous users with individual, on-demand Jupyter
notebook servers, which are later culled. See https://github.com/jupyter/tmpnb
for more details.

Rationale: tmpnb supports demonstrations and easy experimentation in
Jupyter notebooks.


\subsection{mybinder}

Description: builds a container for a GitHub repository containing
Jupyter notebooks and an environment description (such as a
Dockerfile), then starts a cloud server where the user may interact
with the notebooks. See http://mybinder.org/ for more details.

Rationale: mybinder enables trivial to use deployment and sharing of
Jupyter notebooks; fosters dissemination and reproducible research,
and connects computational resources with interesting use cases.
Its success proves that it has identified just the right service for
a critical need.

Issue: too successful and lacking computational resources

Goal: power up mybinder with additional computational resources to
leverage its use. Option 1: run an alternative instance of mybinder.
Better option: add a button "Run on EGI cloud" on the official
mybinder instance, enabling anyone with appropriate credentials to use
the EGI cloud to run the container.

% Not sure where to put this:
Enable easy sharing of large datasets to be used by notebooks in mybinder. Git
and GitHub, which are used to share notebooks and descriptions of the
environment to run them, are not optimised for large files.

\subsection{jupyterhub}

Description: gives authenticated users access to private, persistent Jupyter
notebook servers. See https://jupyterhub.readthedocs.io/ for more details.

Rationale: JupyterHub allows centralised management of Jupyter notebook servers
for a group of users. For instance, a lecturer may run a JupyterHub instance for
students taking their course, removing the need for students to install software
locally.

Goal: Run an instance of JupyterHub enabling anyone with appropriate
credentials to run jupyter notebooks on the EGI cloud.

\subsection{Service table}

There is a fast growing open source ecosystem around the
Jupyter [1] technology for interactive data science and scientific computing;
the flagship is the Jupyter notebook: web-based narrative documents mixing live
computations, equations rich texts and medias, interactive visualization, etc.
Although rapidly evolving, this is a mature ecosystem, based on open
and acclaimed technologies (e.g. docker/kubernetes, ...), and already used
by hundreds of thousand of people, as much in academia as the industry.
High profile industrial users of, and contributors to, the Jupyter ecostsystem
include Bloomberg [2] and Microsoft [3].

\begin{tabular}{|p{5cm}|p{9cm}|}
\hline
 & SERVICE OVERVIEW\\
\\\hline
Thematic Service Name&Jupyter e-Infrastructure\\
\\\hline
Service description&Jupyter notebooks are a popular tool for sharing
computational workflows, combining code with narrative description and results,
including graphical output and interactive elements. The Jupyter project
includes a number of services for sharing and running notebooks.
\\\hline
Service provider&Software is developed by the Jupyter project, under the
umbrella of the NumFOCUS foundation. This proposal is to integrate software from
the Jupyter project as services running on EU infrastructure.
\\\hline
Service catalogue&-\\
\\\hline
Value&Enables trivial to use deployment and sharing of
Jupyter notebooks; fosters dissemination and reproducible research,
and connects computational resources with interesting use cases.
\\\hline
Current TLR&The software pieces
are TRL-8 level, with instances already deployed. Some may require
customization or some specific development for integration in the
EGI/EUDAT/INDIGO infrastructure (authentication and data sharing,
container orchestration, ...).  ????\\
\\\hline
Access policy&Wide access\\
\\\hline
Terms of use&Would be developed as part of the integration work.
\\\hline
User groups and scientific disciplines served&All research
disciplines that require computation or data analysis. The service
will address three main use cases:-
\begin{itemize}
\item Users who wish to reproduce and build upon published
  computatonal workflows.
\item Users who require frictionless access to greater computational
  resources; facilitating migration from Desktop to cloud computing.
\item Educators who wish to provide a user-friendly, stable training
  environment for those studying computation and data science.
\end{itemize}\\
\\\hline
Service business model&TODO\\
\\\hline
\end{tabular}

\begin{tabular}{|p{2cm}|p{8cm}|p{4cm}|}
&SERVICE ARCHITECTURE&
\\\hline
Name&Description, standards, resource capacity&Provider (if appointed)
\\\hline
mybinder&Builds a container for a GitHub repository containing
Jupyter notebooks and an environment description (such as a
Dockerfile), then starts a cloud server where the user may interact
with the notebooks. See http://mybinder.org/ for more details.
mybinder enables trivial to use deployment and sharing of
Jupyter notebooks; fosters dissemination and reproducible research,
and connects computational resources with interesting use cases.
Its success proves that it has identified just the right service for
a critical need.&todo
\\\hline
JupyterHub&gives authenticated users access to private, persistent Jupyter
notebook servers. See https://jupyterhub.readthedocs.io/ for more details.
JupyterHub allows centralised management of Jupyter notebook servers
for a group of users. For instance, a lecturer may run a JupyterHub instance for
students taking their course, removing the need for students to install software
locally.&todo
\\\hline
\end{tabular}

\begin{tabular}{|p{7cm}|p{7cm}|}
&SERVICE INTEGRATION WITH GENERIC E-INFRASTRUCURES
\\\hline
Integration activity and concerned service components&At this stage, a low hanging fruit with high return value would be to
leverage the access to this technology to academics in Europe
(possibly beyond) by *deploying, hosting, and maintaining* Jupyter
based web services like JupyterHub, tmpnb, or mybinder, and integrating these
with scalable data storage, computational processing resources, and common
authentication mechanisms.
\\\hline
Overall necessary effort (Person-Months) and timeline&TODO
\\\hline
List of requested service components&Compute, storage, data, authentication
\\\hline
\end{tabular}
\\\\
\begin{tabular}{|p{7cm}|l|}
  &Infrastructure integration
  \\\hline
  Description of infrastructure integration activities relevant to the proposed thematic service (to be planned in the project)&TODO
  \\\hline
\end{tabular}

\begin{tabular}{|p{7cm}|l|}
  &Training
  \\\hline
  Description of training activities relevant to the proposed service (to be planned in the project)&TODO
  \\\hline
\end{tabular}

\begin{longtable}{|p{7cm}|p{7cm}|}
\hline
 & SERVICE OVERVIEW\\
\\\hline
Thematic Service Name&Simulagora\\
\\\hline
Service description&
%
Simulagora is the Web platform supporting the
https://www.simulagora.com web site, which enable users to launch
scientific computations to on-demand deployed cloud computation
resources. The proposal aims at installing Simulagora on EIG cloud to
make it available to its users.
\\
\\\hline
Service provider&Logilab\\
\\\hline
Service catalogue&\\
%
Not Applicable.

\\\hline
Value&
%
Simulagora features improve users' experience dramatically compared to
classical HPC batch managers:
\begin {itemize}
\item no software installation required (a recent browser is enough)
\item users can choose between several managed OS versions
\item users have administrator privileges on the computing machine
\item users can connect to the computing
\item computations are reproducible as everything is stored, from the
  OS itself to the input data and programs used to perform them
\item users can collaborate, share data, code and studies
\item users can use a programmable API to automate the launch of
  several computations (e.g. a parametric study or a design of
  experiment)
\end {itemize}
%
As a result, users reduce the maintenance tasks of their software and
hardware, decrease the time they need to get their computations done
and analyse them, even more so if they need to collaborate with other
people.
\\
\\\hline
Current TLR&
%
Simulagora has been in production for 2 years and used by French
companies (from SMEs like Fluidyn and Phimeca to big companies like
EDF and SNCF) for very different use cases, proving its scalability
and reliability.

\\
\\\hline
Access policy&
%
The access policy would preferably be Policy-based, typically granting
access to any academic in Europe (to be negociated).

\\
\\\hline
Terms of use&TODO\\
\\\hline
%
The service is intendend to all scientific people who need to perform
computations on HPC-like infrastructures and usually use a batch
manager for this purpose. The typical workloads of Simulagora are
Computation Fluid Dynamics (OpenFOAM) or Finite Element problems in
thermics, mechanics or electromagnetics (using Code\_ASTER, FENICS,
...), using Open Source solvers.

\\
\\\hline
Service business model&
%
The initial integration of Simulagora in the EGI cloud platforms can
be estimated to 3 man.months. Its maintenance costs during the project
can vary from 1 to 3 man.month per year depending on the specific
software the users would eventually ask for.

\\
\\\hline
\end{longtable}


\begin{tabular}{|p{7cm}|p{7cm}|}
&SERVICE INTEGRATION WITH GENERIC E-INFRASTRUCURES
\\\hline
Integration activity and concerned service components&TODO
\\\hline
Overall necessary effort (Person-Months) and timeline&TODO
\\\hline
List of requested service components&TODO
\\\hline
\end{tabular}

\begin{tabular}{|l|l|l|}
&SERVICE ARCHITECTURE&
\\\hline
Name&Description, standards, resource capacity&Provider (if appointed)
\\\hline
mybinder&todo&todo
\\\hline
JupyterHub&todo&todo
\\\hline
\end{tabular}

\begin{tabular}{|p{7cm}|l|}
  &Infrastructure integration
  \\\hline
  Description of infrastructure integration activities relevant to the proposed thematic service (to be planned in the project)&TODO
  \\\hline
\end{tabular}

\begin{tabular}{|p{7cm}|l|}
  &Training
  \\\hline
  Description of training activities relevant to the proposed service (to be planned in the project)&TODO
  \\\hline
\end{tabular}


\section{Relevance to EINFRA-12 (A) challenges}

% 1. The operation of a federated European data and distributed computing
% infrastructure for research and education communities will optimise the access
% to IT equipment and services

\subsection{Optimising access}

Integrating Jupyter notebooks into the proposed infrastructure would facilitate
access to IT resources for relatively small computational tasks with rapid
feedback to the user, in addition to the large batch jobs which are commonly
run on shared compute resources. In particular, these technologies encourage
open sharing of computational research methods, and reproducing and building on
published work.

% 2. All European researchers and educators are in equal footing to access
% essential resources

\subsection{Equal access across Europe}

Providing these services on pan-European infrastructure, could enable any
European institution to, for instance, teach a course using hosted Jupyter
notebooks, whereas the cost of such an activity may otherwise limit it to richer
institutions.

% Not sure if this is a good thing to mention - we don't have anything to show
% on i18n yet
As part of this proposal, we would like to work on internationalisation within
Jupyter, and provide a framework for people to contribute to translating the
interface into their own language.

% 3. Partnerships with industrial and private partners

\subsection{Industry partnerships}

The Jupyter project already works with a number of industry partners, including
Google, Microsoft, Bloomberg and O'Reilly. This proposal would build on these
relationships.

% 4. Train people in research and academic organisations

\subsection{Training}

% Not sure if this means using it to support training, or training people to use
% it

We will provide training in using the service through online tutorials
and documentation, through video walk throughs and recorded
presentations, through attendance and delivery of talks, tutorials and
workshops at conferences that are attended by the communities (for
example from PyCon, SciPy, Supercomputing, ...), through engagement
and direct work with organisations that are likely to act as
multipliers of the knowledge, such as the software carpentry and
Southampton's Centre for Doctoral Training in Next Generation
Computational Modelling.


% 5. Avoid the locking-in to particular hardware or software platforms

\subsection{Avoiding lock-in}

All of the software involved is not only open source, but built around open
protocols and file formats, allowing for easy interoperation. In particular,
support for multiple programming languages is modular, relying on 'kernels'
which speak a common protocol. Kernels for a wide range of languages have
already been developed outside the Jupyter project, demonstrating the
effectiveness of this modularity.

% 6. More scientific communities will use storage and computing infrastructures
% with state-of-the-art services

\subsection{More scientific communities}

By presenting code in a more appealing format, Jupyter is helping to expand
the use programming beyond traditionally numerical fields, to become more
commonplace among biologists and social scientists, among others. Connecting
this interface to large scale computational resources would allow a diverse set
of communities to tackle interesting problems.

% 7. The open nature of the infrastructure will allow scientists, educators and
% students to improve the service quality

\subsection{Open infrastructure}

All of the Jupyter code is developed in the open, using the popular GitHub
code sharing site. The Jupyter community regularly receives improvements from
people using the software.

\subsection{Incentives, collaboration, innovation capacity, \ldots}
% 8. Increase the incentives for scientific discovery and collaboration across
% disciplinary and geographical boundaries. It will further develop the European
% economic innovation capacity and provide stability to the e-infrastructure.
Jupyter notebooks provide hugely effective communication of all
details of a computing or data centric study, as each notebook is
executable and can (in principle) repeat its study by being
re-executed. As such, each Jupyter Notebook directly supports
collaboration of groups beyond geographic boundaries where
person-to-person meetings to exchange details of a calculation are
difficult to arrange. The notebook increases research effectiveness,
in quantity and quality, and thus naturally accelerates economic
innovation across many disciplines: academic and commercial
communities from high-energy research to the financial sector have
embraced the Notebook as their tool of choice. The emergence of the
notebook as the de-facto standard computing environment is likely to
contribute to stabilising the eco-system of computational tools; an
indication for this demand and emerging standardisation is that
Github.com has already started to provide rendering of notebook files.




\section{Information on innovation, dissemination and exploitation}

\subsection{Innovation}

Jupyter notebooks provide a step change in efficiency in carrying out
computational studies (be it based on data or computation) through
full integration of the following steps into a single document:
assumptions, code/data, results, post-processing, analysis,
visualisation, interpretation and conclusions. This cuts down on the
time required to carry out a full computational study: previously, all
of the above steps had to be carried out using distinct tools and
environments, and eventually put together in a report manually. Such
studies are a core activity in research in most field, including the
development of research roadmaps, exploration of adventurous ideas and
systematic evaluation of computational technologies with lower
TRLs. The widespread accessibility of Jupyter Notebooks, as proposed
here, fosters innovation through lowering the effort of exploring
innovative ideas and technologies.


\subsection{Dissemination}

Jupyter notebooks are an evolutionary step up from traditional
academic papers.  Whereas traditional papers merely advertise how
computational research was performed, Jupyter notebooks allow the
reader to completely reproduce, critique and build upon it on their
own computational infrastructure.  Combining a narrative description
of the research along with executable code and data, they form
complete research objects that fully encapsulate the research. As
such, they are an unrivaled vehicle for the dissemination of
computational and data-centic methods.

Providing services to host and share notebooks will improve
dissemination at all scales, making it easier for anyone interested in
a result to re-run the analysis, and inspect every single step of the
work (which is often not fully documented in traditional academic
publications).

\subsection{Exploitation}

The Jupyter Notebook accelerates research and impact in two ways: the
quantity of studies we can carry out using notebooks is greater than
without them (see section 'Innovation'), thus providing significant
additional value to any computational and data-centric research
activity. In addition to this quantitative improvement (more research
done per investment), there is also a significant improvement in the
quality of research and dissemination: as the published notebooks
contain every step of the computational study, they can be fully
investigated and exploited by other groups, for example by taking a
published notebook as the starting point of a new study, not having to
spend anytime to reproduce published results at the beginning of a new
project.

Availability of the access to the notebook thus improves the rate and
quality of research that can be carried out, and is a methodology
improvement to benefit research in all areas, leading to more
effective exploitation of research investment.
