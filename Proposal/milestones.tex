\eucommentary{Milestones means control points in the project that help to chart progress. Milestones may
correspond to the completion of a key deliverable, allowing the next phase of the work to begin.
They may also be needed at intermediary points so that, if problems have arisen, corrective
measures can be taken. A milestone may be a critical decision point in the project where, for
example, the consortium must decide which of several technologies to adopt for further
development.}


By nature and design the project consists of a large number of loosely
coupled tasks, especially so in the first stages. We have therefore
chosen to schedule two first milestones that track general progress on
the project. Later milestones track and showcase how the achievements
in certain collections of tasks in the work packages come together to
produce high level user-visible features, in particular in the context
of the Virtual Research Environment.

The milestones have been scheduled at the occasion of the project
yearly meetings, where they can be discussed in detail, tracking the
progress in each work package through status reports on the tasks and
deliverables and take corrective measures, where necessary, and
critical decisions regarding further plans. The later milestones
coincide as well with the formal project reviews for demonstration,
assessment and discussion with the reviewers.
%
We envisage that this
setup will give the project the vital coherence in spite of the broad
interdisciplinary mix of various backgrounds of the participants.

\paragraph{General Milestones}

\begin{milestones}
  \milestone[id=startup,month=12,
  verif={Completed all corresponding deliverables and reported the progress in the 2nd Project meeting report.}]
  {Startup}
  {By this milestone we will have carried out the requirements study, design and prototype implementations and started community building activities.}

  \milestone[id=proto1,month=24,
  verif={Completed all corresponding deliverables and reported the progress in the 4th Project meeting report.}]
  {Implementations}
  {By this milestone we will have constructed first fully functional interface implementations and released enhanced versions of \TheProject components, and train early adopters of \TheProject.}

\end{milestones}

\paragraph{Milestone for WP 3}

\begin{milestones}
  % original delivery date proposal is M36 but milestone is linked to D3.10 which is planned for M48...
  \milestone[id=component-architecture-distribution,month=42,
  verif={Completed all corresponding deliverables and reported the progress in the 6th Project meeting report.}]
  {ODK's computational components available on major platforms}
  {User story: users shall be able to easily install ODK's
    computational components on the three major platforms (Windows,
    Mac, Linux) via their standard distribution channels.}

\end{milestones}

\paragraph{Milestones for WP4}

\begin{milestones}
  \milestone[id=UI-vre-prototype,month=36,
  verif={Completed all corresponding deliverables; public online demonstrator.}]
  {Prototype VRE for mathematical researchers}
  {
  % note: delivref doesn't work here
  User story: a group of mathematical researchers with access to
  common computational resources, such as a shared lab computer or
  cloud servers, shall be able to deploy a prototype VRE with
  \JupyterHub, integrating \ODK components.
  The Jupyter kernels for mathematical software developed as part of \ODK
  make computational mathematical components accessible in a \Jupyter
  environment, enabling a Jupyter-based deployment of the relevant
  tools for the researchers.
  The process of working on notebooks is greatly improved by review tools
  developed as part of WP4,
  enabling researchers to collaborate to some degree
  in a shared computational environment.
  }
  \milestone[id=UI-vre,month=48,
  verif={Completed all corresponding deliverables; public online demonstrator}]
  {Collaborative VRE for mathematical researchers and beyond}
  {
  The prototype VRE shall be extended with improved ease of deployment, new
  functionality such as interactive 3D visualization and real-time
  collaboration, enabling researchers to collaborate productively in a shared
  computational environment. Finally, integrating notebooks and semantic
  knowledge into a publication / knowledge system enable a continuous process
  of leveraging \ODK components from research to publication.
  }
\end{milestones}

\paragraph{Milestone for WP 5}

\begin{milestones}
  \milestone[id=hpc-prototype, month=36,
    verif={Completed all corresponding deliverables; public online demonstrator.}]
  {Seamless use of parallel computing architecture in the VRE (proof of concept)}
  {
    User story: Astrid wants to run compute intensive routines
    involving both dense linear algebra and combinatorics. She has
    access through a JupyterHub-based VRE to a high end multi-core
    machine which includes a vanilla \Sage installation. She
    automatically benefits from the HPC features of the underlying
    specialized libraries (\Linbox, ...). This is a proof of concept
    of the overall framework to integrate the HPC advances of
    specialized libraries into a general purpose VRE.
    %
    It will prepare the final integration of a broader set of such
    parallel features for the end of the project.
    % deliverables \longdelivref{hpc}{LinBox-algo},
    % \delivref{hpc}{sage-HPCcombi} for combinatorics.
    %last deliverable~\delivref{hpc}{LinBox-distributed}.
    \TODO{Relation with WP3 deliverable hpc-configure?}
  }
\end{milestones}


\paragraph{Milestones for WP 6}

\begin{milestones}
  \milestone[id=dksbases-interop1,month=36,
    verif={Public online demonstrators showcasing selected case study examples}]
  {First Math-In-The-Middle-based interoperability prototype}

  {User story: thanks to a fully functional prototype integrating of
    at least the systems \GAP, \Sage, \Singular, and \LMFDB via the
    \SCSCP Protocol, end users shall be able to run calculations
    involving any combination of those systems from any of them. This
    prototype will be the basis for integration work for additional
    systems and the user interface from WP4.}

  \milestone[id=dksbases-interop2,month=42,
    verif={Public online demonstrators showcasing selected case study examples}]
  {Second Math-In-The-Middle-based interoperability prototype}
  {The goal of this milestone is to take into account all the operational 
    experiences with the first prototype and add more systems and integrate some
    of the UI components from WP4. The experiences with the preparation of 
    this prototype will allow us to estimate the joining costs of adding a system 
    to the OpenDreamKit VRE toolkit, which is an important measure of the 
    flexibility of the Math-In-the-Middle approach.}
\end{milestones}

%%% Local Variables:
%%% mode: latex
%%% TeX-master: "proposal"
%%% End:

%  LocalWords:  verif ldots
