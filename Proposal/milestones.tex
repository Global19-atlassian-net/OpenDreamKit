\eucommentary{Milestones means control points in the project that help to chart progress. Milestones may
correspond to the completion of a key deliverable, allowing the next phase of the work to begin.
They may also be needed at intermediary points so that, if problems have arisen, corrective
measures can be taken. A milestone may be a critical decision point in the project where, for
example, the consortium must decide which of several technologies to adopt for further
development.}

The work in the \TheProject project is structured by four milestones, which could be
briefly characterised as: starting up and building prototypes; moving from prototypes to
fully functional implementations; further engagement with the community and producing
research outputs; evaluation and final releases. They coincide with the project meetings
held at the end of each year of the project (four other meetings will be held in the
middle of each year).  Given the nature of the project, with a large number of essentially
independent tasks, there is no need for milestones attached to specific collections of
tasks or deliverables.  Given that the meetings are the main face-to-face interaction
points in the project, we have chosen to schedule the milestones for these events, where
they can be discussed in detail, tracking the progress in each work package through status
reports on the tasks and deliverables and take corrective measures, where necessary, and
critical decisions regarding further plans.  We envisage that this setup will give the
project the vital coherence in spite of the broad interdisciplinary mix of various
backgrounds of the participants.

\paragraph{General Milestones}

\begin{milestones}
  \milestone[id=startup,month=12,
  verif={Completed all corresponding deliverables and reported the progress in the 2nd Project meeting report.}]
  {Startup}
  {By Milestone 1 we will have carried out the requirements study, design and prototype implementations and started community building activities.}

  \milestone[id=proto1,month=24,
  verif={Completed all corresponding deliverables and reported the progress in the 4th Project meeting report.}]
  {Implementations}
  {By Milestone 2 we will have constructed first fully functional interface implementations and released enhanced versions of \TheProject components, and train early adopters of \TheProject.}

  \milestone[id=community,month=36,
  verif={Completed all corresponding deliverables and reported the progress in the 6th Project meeting report.}]
  {Community/ Experiments}
  {By Milestone 3 we will have gathered and evaluated feedback on \TheProject software and established the portfolio of experiments produced with \TheProject through further engaging with the community.}

  \milestone[id=eval,month=48,
  verif={Completed all corresponding deliverables and reported the progress in the 8th Project meeting report.}]
  {Evaluation}
  {By Milestone 4 we will have released final versions of all \TheProject components and completed the project evaluation.}
\end{milestones}

\paragraph{Milestones for WP 4}
We propose two milestones:

\begin{milestones}
  \milestone[id=WP4prototype,month=36,
    verif={Prototype VRE for mathematical researchers}]
  {Prototype VRE for mathematical researchers}
  {
  % note: delivref doesn't work here
  User story: A group of mathematical researchers with access to
  common computational resources, such as a shared lab computer or
  cloud servers, shall be able to deploy a prototype VRE with
  \JupyterHub, integrating \ODK components.
  The Jupyter kernels for mathematical software developed as part of \ODK
  make computational mathematical components accessible in a \Jupyter
  environment, enabling a Jupyter-based deployment of the relevant
  tools for the researchers.
  The process of working on notebooks is greatly improved by review tools
  developed as part of WP4,
  enabling researchers to collaborate to some degree
  in a shared computational environment.
  }
  \milestone[id=WP4collaborative,month=48,
  verif={Collaborative VRE for mathematical researchers}]
  {Collaborative VRE for mathematical researchers}
  {
  The prototype VRE shall be extended with improved ease of deployment, new
  functionality such as interactive 3D visualization and real-time
  collaboration, enabling researchers to collaborate productively in a shared
  computational environment. Finally, integrating notebooks and semantic
  knowledge into a publication / knowledge system enable a continuous process
  of leveraging \ODK components from research to publication.
  }
\end{milestones}

\paragraph{Milestones for WP 6}

\begin{milestones}
  \milestone[id=WP6interop1,month=36,
  verif={Demonstrator Online Public, works on selected case study examples}]
  {First MitM-based interoperability prototype (GAP, SageMath, LMFDB)}
  {We intend to present a fully functional prototype of the integration of at least the
    systems GAP, SageMath, and LMFDB via the SCSCP Protocol at the second review 
    meeting. This prototype will be the basis for additional integration work for 
    additional systems and the use interface from WP4.}
\milestone[id=WP6interop2,month=42,   verif={Demonstrator Online Public, works on selected case study examples}]
  {Second MitM-based interoperability prototype}
  {The goal of this milestone is to take into account all the operational 
    experiences with the first prototype and add more systems and integrate some
    of the UI components from The experiences with the preparation of 
    this prototype will allow us to estimate the joining costs of adding a system 
    to the OpenDreamKit VRE toolkit, which is an important measure of the 
    flexibility of the MitM approach.}
\end{milestones}

%%% Local Variables:
%%% mode: latex
%%% TeX-master: "proposal"
%%% End:

%  LocalWords:  verif ldots
