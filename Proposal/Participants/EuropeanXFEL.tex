\begin{sitedescription}{XFEL}
  \label{sitedescription:xfel}

% PIC:
% see: http://ec.europa.eu/research/participants/portal/desktop/en/orga

% See ../proposal.tex, section Members of the Consortium for a
% complete description of what should go there

  The European X-Ray Free-Electron Laser Facility GmbH is a limited
  liability company under German law. At present, 11 countries are
  supporting European XFEL through cash and in-kind contributions:
  Denmark, France, Germany, Hungary, Italy, Poland, Russia, Slovakia,
  Spain, Sweden, and Switzerland. The company is in charge of the
  construction and operation of the European XFEL, a 3.4$\,$km long X-ray
  free-electron laser facility extending from Hamburg to the
  neighbouring town of Schenefeld in the German federal state of
  Schleswig-Holstein. Civil construction started in early 2009; the
  beginning of user operation is planned for 2017. With its repetition
  rate of 27,000 pulses per second and a peak brilliance a billion
  times higher than that of the best synchrotron X-ray radiation
  sources, it is expected that the European XFEL will enable the
  investigation of still open scientific problems in a variety of
  disciplines (physics, structural biology, chemistry, planetary
  science, study of matter under extreme conditions and many others).

  European XFEL, a landmark on the ESFRI Roadmap, is a single site
  X-ray research infrastructure. When operational, 3~beamlines and
  6~experiments will be available for scientific users. The SASE1
  beamline comprises the instruments Single Particles, clusters, and
  Biomolecules and Serial Femtosecond Crystallography (SPB/SFX) and
  Femtosecond X-ray Experiments (FXE), SASE 2 includes Materials
  Imaging and Dynamics (MID) and High Energy Density Science (HED) and
  SASE3 Small Quantum Systems (SQS), and Spectroscopy \& Coherent
  Scattering (SCS).


\medskip In the context of this proposal, Hans Fangohr has long
standing experience in high performance computer simulation to advance
science and engineering, and the education of researchers in the
most effective pursuit of computational science.

European XFEL is using IPython and the Jupyter
Notebook as core utilities in their large scale experiment control,
data capture and data analysis.

\subsubsection*{Curriculum vitae}

% Curriculum of the personnel at this institution
%
\begin{participant}[PM=3,type=leadPI,gender=male]{Hans Fangohr}
  Hans Fangohr is Professor of Computational Modelling at the
  University of Southampton until August 2017 (3PM), then Senior Data
  Analysis Scientist at the European XFEL GmbH (3PM). He has studied Physics with
  specialisation in Computer Science and Applied Mathematics, gained
  his PhD in High Performance Computing (2002) in computer science and
  has since worked on the development of computational tools and
  application of those in interdisciplinary projects in science and engineering.

  He heads the University's interdisciplinary Computational Modelling
  Group (\url{http://cmg.soton.ac.uk}) at Southampton, and has more than 100
  publications on development of computational methods and applied
  computer simulation in magnetism, superconductivity, biochemistry,
  astrophysics and aircraft design.

  In 2013, he has attracted \EUR{5}m from the UK's Engineering and
  Physical Sciences Research Council (EPSRC) together with additional
  moneys from industry and his University of Southampton to fund the
  \EUR{12}m Centre for Doctoral Training in Next Generation
  Computational Modelling
  (\href{http://ngcm.soton.ac.uk}{ngcm.soton.ac.uk}) in the UK. This
  flagship activity will train about 75 PhD students (10 to 20
  starting every year, first cohort started in September 2014) in the
  state-of-the-art and best-practice in computational modelling, the
  programming of existing and emerging parallel hardware and to apply
  these skills and tools to carry out PhD research projects across a range of
  topics from Science and Engineering. The centre has chosen \IPython
  as a key tool to deliver this teaching, document and communicate
  computational exploration and drive reproducible computation to push
  for excellent computational science.

  Hans Fangohr has led the development of the Open Source \software{Nmag}
  software (\url{http://nmag.soton.ac.uk}), which provides a finite-element
  micromagnetic simulation suite to a community of material
  scientists, engineers and physicists who research magnetic
  nanostructures in academia and industry. He has designed the package
  in 2005 so that it has an \IPython-compatible \Python interface, to
  make the workflow of using the simulation package as accessible as
  possible to scientists without substantial computational
  background. He has extensive experience in micromagnetic simulation
  tool development and use, and due to this an outstanding understanding of the
  requirements for computational workflows in this micromagnetic
  research community.

  He has deep interest in excellence and innovation in learning and
  teaching. He has been awarded the prestigious Vice Chancellor's
  teaching award ($\pounds 1000$) three times (in 2006, 2010, 2013)
  for initiating and realising three separate innovations in the
  university's teaching delivery of computational engineering, and has
  been voted ``best lecturer'' and ``funniest lecturer'' of the year
  by the students. Other Universities in the UK and elsewhere have
  adopted his teaching methods and materials. He has attracted grants
  to further develop learning and teaching activities, and given
  invited talks at international meetings on efficient learning and
  teaching of computational methods.

  Hans Fangohr is chairing the UK's national Scientific Advisory
  Committee for High Performance Computing.
\end{participant}

%%% Local Variables:
%%% mode: latex
%%% TeX-master: "../proposal"
%%% End:

% \begin{participant}[PM=2,type=PI,gender=male]{Ian Hawke}
%
Ian Hawke is a lecturer in Applied Mathematics at the University of
Southampton and a co-director of the \EUR{12}m EPSRC Centre for Doctoral
Training in Next Generation Computational Modelling. An expert in
nonlinear simulations of relativistic matter and numerical techniques,
he has taught numerical methods in many contexts for ten years. He has
worked on \IPython (\Jupyter{}) Notebooks in education, particularly as an
instructor on the ``Practical Numerical Methods in \Python'' MOOC,
which builds on other open technologies including \software{OpenEdX} and GitHub. The
initial author of the ``\software{Whisky}'' relativistic hydrodynamics code, he has
been a contributor to and maintainer of a range of projects used
across the numerical relativity community, including the 
\software{Einstein Toolkit}, the \software{Cactus} infrastructure and 
the \software{Carpet} mesh refinement
code. His recent research has concentrated on numerical methods for
relativistic matter beyond ideal fluids, including modelling sharp
transitions and surfaces, relativistic elasticity, and the first
nonlinear simulations of relativistic multifluids.
%
\end{participant}

%%% Local Variables:
%%% mode: latex
%%% TeX-master: "../proposal"
%%% End:


\begin{participant}[type=R, PM=24,gender=male]{Marijan Beg}
  Dr Marijan Beg is a post-doctoral researcher with experience in
  computational science, IPython and the Jupyter Notebook and micromagnetic
  simulations. He is working under the leadership of and together with Hans
  Fangohr at Southampton until August 2017 (16PM), and then moving to
  continue the work at European XFEL GmbH from September 2017 onwards (24PM).
\end{participant}

% %\input{CVs/First.Last.tex}
%
\subsubsection*{Publications, products, achievements}

\begin{compactenum}
\item Open Source micromagnetic simulation framework Nmag,
  \href{http://nmag.soton.ac.uk}{http://nmag.soton.ac.uk}, Thomas
  Fischbacher, Matteo Franchin, Giuliano Bordignon, Hans Fangohr: \emph{
A Systematic Approach to Multiphysics Extensions of Finite-Element-Based Micromagnetic Simulations: Nmag
IEEE Transactions on Magnetics \textbf{43}, 6, 2896-2898 (2007)}
\item Other open source contributions to the micromagnetic simulation
  community: OVF2VTK, higher order anisotropy extensions to OOMMF,
  OVF2MFM, summarised at
  \href{http://www.southampton.ac.uk/~fangohr/software/index.html}{http://www.southampton.ac.uk/~fangohr/software/index.html}
\item H. Fangohr.
\emph{A Comparison of \software{C}, \Matlab and \Python as Teaching Languages in Engineering}
Lecture Notes on Computational Science \textbf{3039}, 1210-1217 (2004)
\end{compactenum}

\end{sitedescription}



%KEY-MORE-TODOS



%%% Local Variables:
%%% mode: latex
%%% TeX-master: "../proposal"
%%% End:

%  LocalWords:  sitedescription Programme organisations programmes Centres subsubsection
%  LocalWords:  micromagnetic Nmag Fischbacher Franchin Bordignon Fangohr emph textbf
%  LocalWords:  Multiphysics summarised Iridis TFlops Modelling
