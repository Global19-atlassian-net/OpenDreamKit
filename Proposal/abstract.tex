\begin{abstract}
  \TheProject will deliver a flexible toolkit enabling research groups to set up Virtual
  Research Environments, customised to meet the varied needs of research projects in pure
  mathematics and applications, and supporting the full research life-cycle from
  exploration, through proof and publication, to archival and sharing of data and code.

  \TheProject will be built out of a sustainable ecosystem of community-developed open
  software, databases, and services, including popular tools such as \Linbox, \MPIR,
  \Sage(sagemath.org), \GAP, \PariGP, LMFDB, and \Singular. We will extend the \Jupyter
  Notebook environment to provide a flexible user interface. By improving and unifying
  existing building blocks, \TheProject will maximise both sustainability and impact, with
  beneficiaries extending to scientific computing, physics, chemistry, biology and more,
  and including researchers, teachers, and industrial practitioners.

  We will define a novel component-based VRE architecture and adapt existing mathematical
  software, databases, and user interface components to work well within it on varied
  platforms.  Interfaces to standard HPC and grid services will be built in.  Our
  architecture will be informed by recent research into the sociology of mathematical
  collaboration, so as to properly support actual research practice. The ease of set up,
  adaptability and global impact will be demonstrated in a variety of demonstrator VREs.

  We will ourselves study the social challenges associated with large-scale open source
  code development and publications based on executable documents, to ensure
  sustainability.

  \TheProject will be conducted by a Europe-wide steered by demand collaboration,
  including leading mathematicians, computational researchers, and software developers
  with a long track record of delivering innovative open source software solutions for
  their respective communities. All produced code and tools will be open source.
\end{abstract}

%%% Local Variables:
%%% mode: latex
%%% TeX-master: "proposal"
%%% End:
