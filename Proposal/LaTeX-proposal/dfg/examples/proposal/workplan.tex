\section{Objectives and Work Programme}\label{sec:workplan}

\subsection{Anticipated Total Duration of the Project}\label{sec:duration}

\begin{todo}{from the proposal template}
Please state
\begin{itemize}
 \item the project's intended duration 1 and how long DFG funds will be necessary,
 \item for ongoing projects: since when the project has been active.
\end{itemize}
\end{todo}

\subsection{Objectives}\label{sec:objectives}

\begin{objective}[id=firstobj,title=Supporting Authors]
  This is the first objective, after all we have to write proposals all the time, and we
  would rather spend time on research. 
\end{objective}

\begin{objective}[id=secondobj,title=Supporting Reviewers]
  They are only human too, so let's have a heart for them as well. 
\end{objective}


\subsection{Work Programme Including Proposed Research Methods}\label{sec:wawp}

%%%%%%%%%%%%%%%%%%%%%%%%%%%%%%%%%%%%%%%%%%%%%%%%%%%%%%%%%%%%%%%%%%%%%%%%%%%%%%%%%
\LaTeX is the best document markup language, it can even be used for literate
programming~\cite{DK:LP,Lamport:ladps94,Knuth:ttb84}

\begin{todo}{from the proposal template}
 review the state of the art in the and your own contribution to it; probably you want to
  divide this into subsubsections. 
\end{todo}

\begin{todo}{from the proposal template}
For each applicant

Please give a detailed account of the steps planned during the proposed funding pe-
riod. (For experimental projects, a schedule detailing all planned experiments should
be provided.)

The quality of the work programme is critical to the success of a funding proposal. The
work programme should clearly state how much funding will be requested, why the
funds are needed, and how they will be used, providing details on individual items
where applicable.

Please provide a detailed description of the methods that you plan to use in the project:
What methods are already available? What methods need to be developed? What as-
sistance is needed from outside your own group/institute?
Please list all cited publications pertaining to the description of your work programme
in your bibliography under section 3.
\end{todo}

The project is organized around \pdatacount{all}{wa} large-scale work areas which correspond
to the objectives formulated above. These are subdivided into \pdatacount{all}{wp} work
packages, which we summarize in Figure~\ref{fig:wplist}. Work area
\WAref{mansubsus} will run over the whole project\ednote{come up with a better
  example, this is still oriented towards an EU project} duration of {\pn}. All
{\pdatacount{systems}{wp}} work packages in {\WAref{systems}} will and have to be
covered simultaneously in order to benefit from design-implementation-application feedback
loops.

\wpfig

\begin{workplan}
\begin{workarea}[id=mansubsus,title={Management, Support \& Sustainability}, short=Management]
  This work-group corresponds to Objective \OBJref{firstobj} and has two work packages:
  one for management proper ({\WPref{management}}), and one each for
  dissemination ({\WPref{dissem}})
   
  This work group ensures the dissemination and creation of the periodic integrative
  reports containing the periodic Project Management Report, the Project Management
  Handbook, an Knowledge Dissemination Plan ({\WPref{management}}), the Proceedings of the
  Annual {\pn} Summer School as well as non-public Dissemination and Exploitation plans
  ({\WPref{dissem}}), as well as a report of the {\pn} project milestones.
   
\begin{workpackage}[id=management,lead=jacu,
  title=Project Management,
 jacuRM=2,jacuRAM=8,pcgRM=2]
  Based on the ``Bewilligungsbescheid'' of the DFG, and based on the financial and
  administrative data agreed, the project manager will carry out the overall project
  management, including administrative management.  A project quality handbook will be
  defined, and a {\pn} help-desk for answering questions about the format (first
  project-internal, and after month 12 public) will be established. The project management
  will consist of the following tasks
\begin{tasklist} 
\begin{task}[id=foo,wphases=0-3,requires=\taskin{t1}{dissem}]
  To perform the administrative, scientific/technical, and financial management of the
  project 
\end{task}
\begin{task}[wphases=13-17!.5]
  To co-ordinate the contacts with the DFG and other funding bodies, building on the
  results in \taskref{management}{foo}
\end{task}
\begin{task}
  To control quality and timing of project results and to resolve conflicts
\end{task}
\begin{task}
  To set up inter-project communication rules and mechanisms
\end{task}
\end{tasklist}

\end{workpackage}
 
\begin{workpackage}[id=dissem,lead=pcg,
 title=Dissemination and Exploitation,
pcgRM=8,jacuRAM=2] 
Much of the activity of a project involves small groups of nodes in joint work. This work
 package is set up to ensure their best wide-scale integration, communication, and
 synergetic presentation of the results. Clearly identified means of dissemination of
 work-in-progress as well as final results will serve the effectiveness of work within the
 project and steadily improve the visibility and usage of the emerging semantic services.


 The work package members set up events for dissemination of the research and
 work-in-progress results for researchers (workshops and summer schools), and for industry
 (trade fairs). An in-depth evaluation will be undertaken of the response of test-users.
 
 \begin{tasklist}
  \begin{task}[id=t1,wphases=6-7]
    sdfkj
  \end{task}
  \begin{task}[wphases=12-13]
    sdflkjsdf
  \end{task}
  \begin{task}[wphases=18-19]
    sdflkjsdf
  \end{task}
 \begin{task}[wphases=22-24] 
 \end{task}
\end{tasklist}

Within two months of the start of the project, a project website will go live. This
website will have two areas: a members' area and a public area.\ldots
\end{workpackage}
\end{workarea}
 

\begin{workarea}[id=systems,title={System Development}]
  This workarea does not correspond to \OBJtref{secondobj}, but it has two work packages:
  one for the development of the {\LaTeX} class ({\WPref{class}}), and for the
  proposal template ({\WPref{temple}})

  This work group coordinates the system development.

\begin{workpackage}[id=class,lead=jacu,
                    title=A LaTeX class for EU Proposals,short=Class,
                   jacuRM=12,jacuRAM=8,pcgRM=12,pcgRAM=2]
We plan to develop a {\LaTeX} class for marking up EU Proposals

We will follow strict software design principles, first comes a
requirements analys, then \ldots
\begin{tasklist}
  \begin{task}[wphases=0-2]
    sdfsdf
  \end{task}
  \begin{task}[wphases=4-8]
    sdfsdf
  \end{task}
  \begin{task}[id=t3,wphases=10-14]
    sdfsdf
  \end{task}
  \begin{task}[wphases=20-24]
    sdfsdfd
  \end{task}
\end{tasklist}
\end{workpackage} 

\begin{workpackage}[id=temple,lead=pcg,
  title= Proposal Template,short=Template,jacuRM=12]

We plan to develop a template file for {\pn} proposals

We abstract an example from existing proposals
\begin{tasklist}
  \begin{task}[wphases=6-12]
    sdfdsf 
  \end{task}
  \begin{task}[id=temple2,wphases=18-24,requires=\taskin{t3}{class}]
    sdfsdf
  \end{task} 
\end{tasklist}
\end{workpackage}

\begin{workpackage}[id=workphase,title=A work package without tasks,
  wphases=0-4!.5]
  
  And finally, a work package without tasks, so we can see the effect on the gantt chart
  in fig~\ref{fig:gantt}.
\end{workpackage}
\end{workarea}
\end{workplan} 

\ganttchart[draft,xscale=.45] 

\subsection{Data Handling}\label{sec:data}

The \pn project will not systematically produce researchdata. All project results will be
published for at least $x$ years at our archive at \url{http://example.org}.

\subsection{-- 2.7 (Other Information / Explanations on the Proposed Investigations / Information on Scientific and Financial Involvement of International
  Cooperation Partners) \qquad \sf n/a}


%%% Local Variables: 
%%% mode: LaTeX
%%% TeX-master: "proposal"
%%% End: 

% LocalWords:  workplan.tex wplist dfgcount wa mansubsus duratio ipower wpfig
% LocalWords:  ganttchart xscale workplan workarea pdataref dissem workpackage foo
% LocalWords:  tasklist taskin taskref sdfkj sdflkjsdf sdfsdf sdfsdfd sdfdsf pn
% LocalWords:  firstobj secondobj pdatacount WAref ednote OBJref pcgRM pcg
% LocalWords:  ldots OBJtref workphase
