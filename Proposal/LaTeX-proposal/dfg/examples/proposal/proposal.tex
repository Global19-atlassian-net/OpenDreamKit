% The main file for developing the proposal. 
% Variants with different class options are 
% - submit.tex (no draft stuff, no ednotes, no revision information)
% - public.tex (like submit.tex, but no financials either) 
\providecommand{\classoptions}{,keys} % to be overwritten in variants
\documentclass[RAM,gitinfo\classoptions]{dfgproposal}
\addbibresource{../lib/dummy}
\usepackage[utf8]{inputenc}
%%% institutions
\WAinstitution[id=PS,
        countryshort=FR,
        acronym=UPSud]
        {Universit\'e Paris-Sud}

\WAinstitution[id=LL,
        countryshort=FR,
        acronym=Logilab]
        {Logilab}

\WAinstitution[id=UV,
        countryshort=FR,
        acronym=UVSQ]
        {Universit\'e de Versailles Saint-Quentin}

\WAinstitution[id=UJF,
        countryshort=FR,
        acronym=UGA]
        {Universit\'e Grenoble Alpes}

\WAinstitution[id=UB,
        countryshort=FR,
        acronym=CNRS]
        {CNRS}

\WAinstitution[id=UO,
        countryshort=UK,
        acronym=UOXF]
        {University of Oxford}

\WAinstitution[id=USH,
        countryshort=UK,
        acronym=USFD]
        {University of Sheffield}

\WAinstitution[id=USO,
        countryshort=UK,
        acronym=SOUTHAMPTON]
        {University of Southampton}

\WAinstitution[id=SA,
        countryshort=UK,
        acronym=USTAN]
        {University of St Andrews}

\WAinstitution[id=UW,
        countryshort=UK,
        acronym=UWarwick]
        {University of Warwick}

\WAinstitution[id=JU,
        countryshort=DE,
        acronym=JacobsUni]
        {Jacobs University Bremen}

\WAinstitution[id=FAU,
        countryshort=DE,
        acronym=FAU]
        {Friedrich-Alexander Universit\"at Erlangen/N\"urnberg}

\WAinstitution[id=UK,
        countryshort=DE,
        acronym=UNIKL]
        {University of Kaiserslautern}

\WAinstitution[id=US,
        countryshort=PL,
        acronym=USlaski]
        {University of Silesia}

\WAinstitution[id=ZH,
        countryshort=CH,
        acronym=UZH]
        {Universit\"{a}t Z\"{u}rich}

\WAinstitution[id=SR,
        countryshort=NO,
        acronym=Simula]
        {Simula Research Laboratory}

\WAinstitution[id=UG,
	countryshort=BE,
	acronym=UGent]
	{Ghent University}

\WAinstitution[id=XFEL,
        countryshort=DE,
        acronym=XFEL]
        {European XFEL GmbH}

% \WAinstitution[id=UWS,
%         countryshort=US,
%         acronym=UWS]
%         {University of Washington at Seattle}

% \WAperson[id=miko, 
%            personaltitle=Prof. Dr.,
%            birthdate=13. September 1964,
%            academictitle=Professor of Computer Science,
%            affiliation=jacu,
%            department=case,
%            privaddress=None of your business,
%            privtel=that neither,
%            email=m.kohlhase@jacobs-university.de,
%            workaddress={Campus Ring 1, 28757 Bremen},
%            worktel=+49 421 200 3140,
%            worktelfax=+49 421 200 3140/493140,
%            workfax=+49 421 200 493140]
%            {Michael Kohlhase}

\WAperson[id=thiery,
           personaltitle=Prof. ,
           birthdate=28 Mai 1973,
           academictitle=Professor of Computer Science,
           affiliation=PS,
           department=Laboratoire de Recherche en Informatique,
           privaddress=None of your business,
           privtel=that neither,
           email=Nicolas.Thiery@u-psud.fr,
           workaddress={Campus Ring 1, 28757 Bremen},
           %worktel=+33 6 77 90 32 79,
           worktelfax=+33 6 77 90 32 79,
           %workfax=N/A
           ]
           {Nicolas M. Thiéry}

%%% Local Variables: 
%%% mode: latex
%%% TeX-master: "proposal"
%%% End: 

% LocalWords:  WAperson miko personaltitle academictitle privaddress privtel Sud
% LocalWords:  workaddress worktel workfax gc worktelfax pcg pcsa WAinstitution
% LocalWords:  shortname partof streetaddress townzip countryshort efo 3kd89
% LocalWords:  jacobs-logo.png Seefahrtstrasse Kruislann Montparnasse Universit
% LocalWords:  baz Westerfield

% temporary fix for biblatex in TeXLive 2016, 
% see  http://tex.stackexchange.com/questions/311426/bibliography-error-use-of-blxbblverbaddi-doesnt-match-its-definition-ve
\makeatletter\def\blx@maxline{77}\makeatother

\begin{document}

\begin{center}\color{red}\huge
  This mock proposal is just an example for \texttt{dfgproposal.cls} it reflects the 
  current DFG template valid from October 2011.
\end{center}

\urldef{\gcpubs}\url{http://www.pcg.phony/~gc/pubs.html}
\urldef{\mikopubs}\url{http://kwarc.info/kohlhase/publications.html}
\begin{proposal}[PI=miko,PI=gc,site=jacu,site=pcg,
  thema=Intelligentes Schreiben von Antr\"agen,
  acronym={iPoWr},
  acrolong={\underline{I}ntelligent} {\underline{P}r\underline{o}posal} {\underline{Wr}iting},
  title=\pn: \protect\pnlong,
  totalduration=3 years,
  since=1. Feb 2009,
  start=1. Feb. 2010,
  months=24,
  pcgRM=36, pcgRAM=36, jacuRM=36, jacuRAM=36,
  discipline=Computer Science, 
  areas=Knowledge Management]

\begin{Zusammenfassung}
  \begin{todo}{in das Elan System kopieren}
    Fassen Sie die relevanten Projektziele allgemeinverst''andlich in maximal 3000 Zeichen
    (keine Sonderzeichen) zusammen
  \end{todo}
  Das Schreiben von Antr"agen ist ein kollaborativer Prozess in dem Betr"age von mehreren
  Personen integriert werden mu"ussen. Ein ASCII-basiertes Format wie {\LaTeX} erlaubt die
  Koordination dieses Prozesses mittels Versionsverwaltungssystemen wie
  Subversion. Dadurch k''onnen sich die Antragsteller auf Inhalte konzentrieren anstatt
  auf die Mechanik der Dokumentenverwaltung.
\end{Zusammenfassung}

\begin{Summary}
  \begin{todo}{copy into the Elan system}
    Summarize the relevant goals of the proposed project in generally intelligible
    terms. Do not use more than 3000 characters, no special characters allowed.
  \end{todo}
  Writing grant proposals is a collaborative effort that requires the integration of
  contributions from many individuals. The use of an ASCII-based format like
  {\LaTeX}~\cite{Lamport:ladps94} allows to coordinate the process via a source code
  control system like Subversion, allowing the proposal writing team to concentrate on the
  contents rather than the mechanics of wrangling with text fragments and revisions.
\end{Summary}

% It is often good to separate the top-level sections into separate files. 
% Especially in collaborative proposals. We do this here. 
\section{State of the Art and Preliminary Work}\label{sec:state}
\begin{todo}{from the proposal guidelines}
  For new proposals please explain briefly and precisely the state of the art in your
  field in its direct relationship to your project. This description should make clear in
  which context you situate your own research and in what areas you intend to make a
  unique, innovative, promising contribution. This description must be concise and
  understand- able without referring to additional literature.

  For renewal proposals, please report on your previous work. This report should also be
  understandable without referring to additional literature.

  To illustrate and enhance your presentation you may refer to your own and others’ pub-
  lications. Indicate whenever you are referring to other researchers’ work.  Please list
  all cited publications in your bibliography under section 3. This reference list is not
  consid- ered your list of publications. Note that reviewers are not required to read any
  of the works you cite. This also applies to review sessions that are held on site. In
  this case, manuscripts and publications that provide more information on the progress
  reports and are published up to the review panel’s meeting may be made available at the
  meeting to enable reviewers to read through the information. Reviews will be based only
  on the text of the actual proposal.
\end{todo}

\subsection{List of Project-Related Publications}\label{sec:projpapers}

\begin{todo}{from the proposal template}
  Please include a list of own publications that are related to the proposed project. It
  serves as an important basis for assessing your proposal. The number of publications to
  cite here is determined as follows:
  \begin{compactdesc}
    \item[Single applicant] two publications per year of the funding duration
    \item[Multiple applicants] three publications per year of the funding duration
    \end{compactdesc}
    These rules refer to the proposed funding duration for new proposals and the completed
    duration for renewal proposals.
    
    If you are submitting a proposal to the DFG for the first time and have therefore not
    published in the proposed research area, please list the up to five most important
    publications so far.
\end{todo}

\subsubsection{Peer-Reviewed Articles}\label{sec:peer-rev}

\dfgprojpapers[articles,books,confpapers]{Kohlhase:pdpl10,Lamport:ladps94,Knuth:ttb84,KohDavGin:psewads11,Lange:OpenMathCDLinkedData10,providemore}
\ednote{Anmerkung Jens: Ein nützliches Feature wäre hier, wenn das Paket eine (eventuell
  über Optionen der Dokumentklasse unterdrückbare) Warnung ausgeben würde, wenn zu viele
  Publikationen entsprechend DFG-Richtlinien angegeben werden. Die Anzahl ist sehr eng
  begrenzt.}

\subsubsection{Other Articles\qquad None.}
\subsubsection{Patents\qquad None.}

%%% Local Variables: 
%%% mode: LaTeX
%%% TeX-master: "proposal"
%%% End: 

% LocalWords:  subsubsections dfgprojpapers pdpl10 providemore compactdesc
% LocalWords:  ourpubs nociteprolist KohKoh ccbssmt09 KohRabZho tmlmrsca10
% LocalWords:  Hutter09 sifemp09

\begin{todo}{from the proposal template}
\begin{enumerate}
\item Describe the overall strategy of the work plan\ednote{Maximum length – one page}
\item Show the timing of the different WPs and their components (Gantt chart or similar).
\end{enumerate}
\end{todo}
\begin{figure}
  \caption{Work package dependencies}
  \label{fig:wp-deps}
\end{figure}

\ganttchart[draft,xscale=.45] 

%%% Local Variables: 
%%% mode: LaTeX
%%% TeX-master: "propB"
%%% End: 

% LocalWords:  workplan.tex ednote wp-deps ganttchart xscale


\section{Bibliography Concerning the State of the Art, the Research objectives, and the
  Work Programme}\label{sec:bib}

\begin{todo}{from the proposal template}
In this bibliography, list only the works you cite in your presentation of the state of the
art, the research objectives, and the work programme. This bibliography is not the list
of publications. Non-published works must be included with the proposal.
\end{todo}
\printbibliography[heading=empty,notcategory=featured]
% the following will not become part of the public proposal after all most of this is
% technical or confidential.
%\begin{private}
\section{Requested Modules/Funds}\label{sec:funds}

For each applicant, we apply for funding within the Basic Module.

\subsection{Funding for Staff}\label{sec:positions}

\subsubsection{Research Staff}\label{sec:positions:research}

We apply for the following positions. All run over the entire duration of the proposed project.

\paragraph*{Non-doctoral staff}\ednote{compute amount in elan and copy here}

One doctoral researcher for 2 years at $100 \%$ for Michael Kohlhase.

One doctoral researcher for 2 years at $100 \%$ for Florian Rabe.

%\paragraph*{Postdoctoral staff}
%\ednote{postdoctoral researcher and comparable}

\paragraph*{Other research assistants}\ednote{students with BSc.}

One student with BSc. for 2 years at $100 \%$ for Michael Kohlhase.

One student with BSc. for 2 years at $100 \%$ for Florian Rabe.

\subsubsection{Non-Academic Staff \qquad None.}

\subsubsection{Student Assistants \qquad None.}

\subsection{Funding for Direct Project Costs}\label{sec;funds:direct}

\subsubsection{Equipment up to 10,000 \texteuro, Software and Consumables}

None.  PC will cover the workspace, computing needs, and consumables for its staff as part
of the basic support.

\subsubsection{Travel Expenses}\label{sec:travel}

\begin{oldpart}{rework}
  The travel budget shall cover:
  \begin{itemize}
  \item visits to external collaborators. We expect two international visits. We estimate
    that each visit will be most effective, if the junior researchers can spend about 3
    weeks with the partners. Thus we estimate 2500 {\texteuro} per visit.
  \item visits to national conferences to disseminate the results of {\pn}. We expect
    one visit for each year for each of the three researchers. (3 x 3 x 1000 {\texteuro})
  \item visits to international conferences to disseminate the results of {\pn}. These
    are in particular the International Joint Conference on Document Engineering (DocEng)
    and the Tech User Group Meeting (TUG). We expect one visit for each proposed
    researcher and for each year. (3 x 3 x 1500 {\texteuro})
  \end{itemize}

  This sums up to a total amount of 32.500 {\texteuro} for travel expenses for the whole
  funding period of three years which is split into 16.250 {\texteuro} for each institute
  (PC and Jacobs University).
\end{oldpart}

\subsubsection{Visiting Researchers}\label{sec:funds:visiting}

Total expenses \textbf{10.200 \texteuro}
\medskip

As explained in Section~\ref{sec:travel}, we expect 5 incoming research visits.  Assuming
an average duration of 3 weeks, we estimate the cost of one visit at 600 {\texteuro} for
traveling and 70 {\texteuro} per night for accommodation, amounting to 2040 \texteuro per
visit.

\subsubsection*{-- 4.1.2.6, 4.1.3 (Expenses for Laboratory Animals / Other Costs
/ Project Related Publication Expenses / Instrumentation) \sf\qquad n/a}

% \subsubsection{Expenses for Laboratory Animals} None.
% \subsubsection{Other Costs} None.
% \subsubsection{Project-Related Publication Expenses} None.
% \subsection{Funding for Instrumentation} None.

%%% Local Variables: 
%%% mode: LaTeX
%%% TeX-master: "proposal"
%%% End: 

% LocalWords:  ipower texteuro

\section{Project Requirements}\label{sec:requirements}

\subsection{Employment Status Information}\label{sec:req:employment-status}

\begin{todo}{from the proposal template}
  For each applicant, state the last name, first name, and employment status (including
  duration of contract and funding body, if on a fixed-term contract).
\end{todo}

\subsection{First-time Proposal Data}\label{sec:req:first}

\begin{todo}{from the proposal template}
  Only if applicable: Last name, first name of first-time applicant.

  If this is your first proposal, reviewers will consider this fact when assessing your
  pro- posal. Previous proposals for research fellowships, publication funding, travel
  allow- ances, or funding for scientific networks are not considered first proposals. If
  you are submitting a “first-time proposal” and it is part of a joint proposal, please
  note that your independent project must be distinct from the other projects.

  If you have already submitted a proposal as an applicant for a research grant and have
  received a letter informing you of the funding decision, or if you have led an independ-
  ent junior research group or project in a Collaborative Research Centre or Research
  Unit, you are no longer eligible to submit a “first proposal”. If you have submitted a
  “first-time proposal” and it was rejected, you may resubmit the application, in revised
  form, as a first-time proposal for the same project.
\end{todo}

\subsection{Composition of the Project Group}\label{sec:req:group}

\begin{todo}{from the proposal template}
  List only those individuals who will work on the project but will not be paid out of the
  project funds. State each person’s name, academic title, employment status, and type of
  funding.

  Please list separately the individuals paid by your institution and those paid using
  other third-party funding (including fellowships).
\end{todo}

\begin{sitedescription}{jacu}
  The KWARC (Knowledge Adaptation and Reasoning for Content) research group headed by
  Michael Kohlhase for has the following members
  \begin{compactdesc}
  \item[Dr. N.N.] is the \ldots She has a background in\ldots.
  \end{compactdesc}
  Additionally, the group has attracted about 10 undergraduate and master's students that
  actively take part in the project work and various aspects of research.
\end{sitedescription}

\begin{sitedescription}{pcg}
  Power Consulting GmbH is the leading provider of semantic document solutions. Dr. Senior
  Researcher leads an applied research group consisting of
  \begin{compactdesc}
  \item[Dr. N.N.] is the \ldots She has a background in\ldots.
  \end{compactdesc}
  The group has access to seven programming slaves specializing in web development and
  document transformation techniques
\end{sitedescription}


\subsection{Cooperation with other Researchers}\label{sec:coop}

\subsubsection{Planned Cooperations}\label{sec:coop:planned}
\begin{todo}{from the proposal template}
  Researchers with whom you have agreed to cooperate on this project
\end{todo}
\begin{compactdesc}
\item[Prof. Dr. Super Akquisiteur (Uni Paderborn)] knows exactly what to do to get funding
  with DFG, we will interview him closely and integrate all his intuitions into the {\pn}
  templates.
\item[Prof. Dr. Habe Nichts (Uni Hinterpfuiteufel)] has never gotten a grant proposal
  through with DFG, we will try to avoid his mistakes.
\item[Dr. Sach Bearbeiter (DFG)] will consult with the DFG requirements to be met in the
  proposals.
\item[Dr. Donald Knuth (Stanford University)] is so surprised that we want to do grant
  proposals in {\TeX/\LaTeX} that he will help us with any problems we have in coding in
  this wonderful programming language.
\end{compactdesc}

\subsubsection{Scientific Collaborations in the past Three Years}\label{sec:past-coop}

\begin{todo}{from the proposal template}
  Researchers with whom you have collaborated scientifically within the past three years

  This information will assist the DFG’s Head Office in avoiding potential conflicts of
  in- terest during the review process.
\end{todo}

\subsection{Scientific Equipment}\label{sec:req:equipment}

Jacobs University provides laptops or desktop workstations for all academic
employees. Great Consulting GmbH. is rolling in money anyways and has all of the latest
gadgets.


\subsection{Project-Relevant Interests in Commercial Enterprises\qquad n/a}

%%% Local Variables: 
%%% mode: LaTeX
%%% TeX-master: "proposal"
%%% End: 

% LocalWords:  Durchf uhrung subsubsection ipower Hinterpfuiteufel Sach Aktivit
% LocalWords:  Erkl arungen


\section{Additional Information}\label{sec:additional}

Funding proposal XYZ-83282 has been submitted prior to this proposal on related topic XYZ.
\end{proposal}
\end{document}
 
%%% Local Variables: 
%%% mode: LaTeX
%%% TeX-PDF-mode:t
%%% TeX-master: t
%%% End: 

% LocalWords:  empty bibflorian systems rabe institutions modal historical pub
% LocalWords:  kwarc till formalsafe miko gc ipower ipowerlong Antr agen Beitr

% LocalWords:  acrolong intellegible kollaboratives koh arenten ussen Proze pcg
% LocalWords:  Versionsmanagementsystem textsc unterst utzt konzentieren stex
% LocalWords:  mechanik workplan thispagestyle newpage Principcal cvpubsmiko pn
% LocalWords:  ourpubs zusammenfassung printbibliography pubspage ntelligent
% LocalWords:  iting pnlong
