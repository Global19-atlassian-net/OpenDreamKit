
\TOWRITE{ALL}{Proofread 3.1 work plan (except for the work packages themselves) pass 2}

\eucommentary{Please provide the following:\\
\begin{compactitem}
\item
brief presentation of the overall structure of the work plan;
\item
timing of the different work packages and their components (Gantt chart or similar);
\item
detailed work description, i.e.:
\begin{compactitem}
\item
a description of each work package (table 3.1a);
\item
a list of work packages (table 3.1b);
\item
a list of major deliverables (table 3.1c);
\end{compactitem}
\item
graphical presentation of the components showing how they inter-relate (Pert chart or similar).
\end{compactitem}
}

\subsubsection{Overall Structure of the Work Plan}\label{sec:workplan-structure}
\ifgrantagreement
The
\else
As shown in Figure~\ref{fig:wplist}, the
\fi
work plan is broken down into
seven work packages: \WPref{component-architecture} about components,
\WPref{UI} for user interfaces, \WPref{hpc} for parallelisation of the
components, \WPref{dksbases} for databases and finally
\WPref{social-aspects} for social aspects. This is complemented by the
the usual management and dissemination work packages
(\WPref{management}) and (\WPref{dissem}). The Gantt chart on
Page~\pageref{fig:gantt} illustrates the timeline for the various
tasks for these work packages.%, including inter-task dependencies.

\ifgrantagreement\else
%\makeatletter\wp@total@RM{management}\makeatother
\wpfigstyle{\footnotesize\def\tabcolsep{3.5pt}}
%\wpfig[pages,type,start,end]
{\wpfig}
\fi
%\newpage
\subsubsection{How the Work Packages will Achieve the Project Objectives}
\label{sssec:how_the_work_packages_will_achieve}

% (Section~\ref{sect:objectives},page~\pageref{sect:objectives})

The following table recalls the objectives of \TheProject and lists
the work packages that contribute to achieving each of them.

\begin{center}
\begin{tabular}{|l|l|l|}\hline
\textbf{Objective} & \textbf{Purpose} & \textbf{WPs} \\\hline \hline
Objective 1
 & Develop and standardise math soft and data for VRE
 & \WPref{component-architecture},  \WPref{UI}, \WPref{hpc}, \WPref{dksbases} \\\hline
Objective 2
 & Develop core VRE components
 & \WPref{component-architecture}, \WPref{UI}, \WPref{hpc}, \WPref{dksbases} \\\hline
Objective 3
 & Bring together communities
 & \WPref{dissem}, \WPref{component-architecture} \\\hline
Objective 4
 & Update a range of softwares
 & \WPref{component-architecture}, \WPref{hpc} \\\hline
Objective 5
 & Foster a sustainable ecosystem
 & \WPref{component-architecture}, \WPref{UI}, \WPref{hpc}, \WPref{dksbases} \\\hline
Objective 6
 & Explore social aspects
 & Cancelled \\\hline
Objective 7
 & Identify and extend ontologies
 & \WPref{dksbases} \\\hline
Objective 8
 & Effectiveness of the VRE
 & \WPref{dissem} \\\hline
Objective 9
 & Effective dissemination
 & \WPref{dissem} \\\hline
\end{tabular}
\end{center}

\TOWRITE{ALL This next section is freshly rewritten to be more
  detailed. It doesn't show or explain dependencies between WPs or
  anything like that, which would be nice, but would take too
  long. Anyway please check}

\paragraph{Work Programme for Objective 1: }

\taskref{component-architecture}{interface-systems} (Interfaces
between Systems) directly addresses the core of objective 1, making
existing systems compatible with one another in mathematically sound
ways. Other tasks in \WPref{component-architecture} (component
architecture) support this, by making components more portable and
easier to deploy (\taskref{component-architecture}{mod-packaging}:
Modularisation and Packaging;
\taskref{component-architecture}{portability}). \taskref{component-architecture}{extract-smc}
will bring us the benefit of lessons learned and components built for
\SMC. \taskref{dksbases}{data-design} deals with the data-centric
aspects of the interfaces. Additionally elements of \WPref{UI} (user interface) and \WPref{hpc} (HPC)
will also contribute to the framework with user interface pluggability
and interfaces optimised for HPC.

\paragraph{Work Programme for Objective 2: }

We have identified a need for a number of new core components for
\TheProject and planned their construction at appropriate stages of
various workpackages. A new adapter infrastructure is part of
\taskref{component-architecture}{interface-systems}; new virtual
appliances will be built in
\taskref{component-architecture}{portability}; new components will be
extracted from \SMC in  \taskref{component-architecture}{extract-smc};
new documentation components will be developed in
\taskref{UI}{sage-sphinx} and \taskref{UI}{dynamic-inspect}; new
mathematical software will be developed in \taskref{hpc}{hpc-combi}
and new database tools in \taskref{dksbases}{data-memo} and \taskref{dksbases}{mws}.


\paragraph{Work Programme for Objective 3: }

Representatives of a number of communities have already come together
simply to prepare this proposal, and the whole project will work to
bring them together. Specifically developers of many systems  will be brought together to complete
work package~\WPref{component-architecture},
especially~\taskref{component-architecture}{interface-systems}.
Bringing broader communities together is the core purpose of
work package~\WPref{dissem}, which includes workshops, web sites,
demonstrator packages and outreach activities.

\paragraph{Work Programme for Objective 4: }

The concept of this project is centred on leveraging the communities
vast investment in existign open source software systems, and wherever
possible we will proceed by extending and updating existing software components.
In work package \WPref{component-architecture} we will address
portability (\taskref{component-architecture}{portability} and
modularity (\taskref{component-architecture}{extract-smc}) and also
adapt the components to use the new interfaces being designed in
\taskref{component-architecture}{interface-systems}. Work package
\WPref{hpc} is largely about updating software for performance, while
workpackage \WPref{UI} deals with adaptation of UI components and of
other systems to work with them.

\paragraph{Work Programme for Objective 5: }

A number of tasks relate to developing promoting and supporting
sustainable models for collaborative software development. On a
practical level \taskref{component-architecture}{workflow} will adress
processes and technologies, \taskref{dksbases}{data-memo} concerns
collaborative accumulation of data. On a personal level, much of
\WPref{dissem} deals with ensuring a wide and committed user/developer
community.
% Finally in~\taskref{social-aspects}{isocial-decisionmaking}
% we will actually conduct research into the social mechanisms of
% collaborative software development, and lessons from this research
% will be embedded into the structures we leave behind.

\paragraph{Work Programme for Objective 6: }

Objective 6 was covered by a dedicated work package \WPref{social-aspects} on social aspects.
It ranged from analysis of the needs with~\taskref{social-aspects}{social-input} to
evaluation with~\taskref{social-aspects}{isocial-decisionmaking}. That objective and dedicated
WP were cancelled.

\paragraph{Work Programme for Objective 7: }

Objective 7 is addressed directly by \WPref{dksbases}, which deals with data
and its meaning.

\paragraph{Work Programme for Objective 8: }

Producing and evaluating systems that demonstrate our achievements is
a key feature of this project, and this work in embedded throughout
the project. The integration and publicisation of these demonstrators
is key to \WPref{dissem} (dissemination) especially later in the
project, while their formal evaluation is found in \WPref{social-aspects}.

\paragraph{Work Programme for Objective 9: }

Dissemination is the heart of~\WPref{dissem}.
Members of \TheProject will organise workshops within \taskref{dissem}{dissemination}
or \taskref{dissem}{project-intro} as well as less formal meetings
with interested groups. In addition, we will follow open software development
processes throughout the project, so that our work is immediately
available to any interested party. We will announce important
developments or releases through our own web pages and the
established channels of the component systems. Our scientific findings
will be published in the open scientific literature and announced at
scientific meetings and conferences in the usual way and reported in
annual project reports.


\subsubsection{Work Plan Timing: GANTT Chart showing Task Dependencies and Information
  Flows}

Since \TheProject consists mainly in improving independent tools and
integrating them into a VRE, its tasks are fairly independent from each
other, which is reflected by the GANTT chart in Figure~\ref{fig:gantt}

\gantttaskchart[draft,xscale=.33,yscale=.33,milestones]

\ifgrantagreement\else
\newpage
\subsubsection{Deliverables}\label{sec:deliverables}
\inputdelivs{9.3cm}
\fi

\newpage
\subsubsection{Milestones}\label{sec:milestones}
\subsection{List of Milestones}\label{sec:milestones}

\begin{todo}{from the proposal template}
  Milestones are control points where decisions are needed with regard to the next stage
  of the project. For example, a milestone may occur when a major result has been
  achieved, if its successful attainment is a requirement for the next phase of
  work. Another example would be a point when the consortium must decide which of several
  technologies to adopt for further development.

  Means of verification: Show how you will confirm that the milestone has been
  attained. Refer to indicators if appropriate. For examples: a laboratory prototype
  completed and running flawlessly, software released and validated by a user group, field
  survey complete and data quality validated.
\end{todo}


The work in the {\pn} project is structured by seven milestones, which coincide with the
project meetings in summer and fall.  Since the meetings are the main face-to-face
interaction points in the project, it is suitable to schedule the milestones for these
events, where they can be discussed in detail. We envision that this setup will give the
project the vital coherence in spite of the broad mix of disciplinary backgrounds of the
participants.\ednote{maybe automate the milestones}

\begin{milestones}
  \milestone[id=kickoff,verif=Inspection,month=1]
    {Initial Infrastructure}
    {Set up the organizational infrastructure, in particular: Web Presence, project TRAC,\ldots}
  \milestone[id=consensus,verif=Inspection,month=24]{Consensus} {Reach Consensus on the
    way the project goes}
  \milestone[id=exploitation,verif=Inspection,month=36]{Exploitation}{The exploitation
    plan should be clear so that we can start on this in the last year.}
  \milestone[id=final,verif=Inspection,month=48]{Final Results}{all is done}
\end{milestones}

%%% Local Variables: 
%%% mode: latex
%%% TeX-master: "propB"
%%% End: 

% LocalWords:  pn ednote verif ldots


% ---------------------------------------------------------------------------
% Include Work package descriptions
% ---------------------------------------------------------------------------

\newpage
\subsubsection{Work Package Descriptions}\label{sec:workpackages}
%%% work package style may be broken -- fix this!!

\ifgrantagreement
\begingroup
% Note: in the grant agreement, The workpackage description must not appear.
% Yet we want to compile them to get all the metadata right
% Current hack: compile them anyway, reset the page number
% appropriately, and remove them a posteriori with pdftk. We set the
% color to red to make it more visible in case we forget to remove
% them.
% See grantagreement rule in the Makefile
\newcounter{savepage}
\setcounter{savepage}{\value{page}}
\color{red} % To make sure we indeed remove the pages
\fi

\enlargethispage{1cm}

%% Local WP number counter - should possibly be global and hidden?
\begin{workplan}
%\TOWRITE{NT}{Proofread WP 1 Management pass 1}
\TOWRITE{ALL}{Proofread WP 1 Management pass 2}
\begin{draft}
\TOWRITE{PS (Work Package Lead)}{For WP leaders, please check the following (remove items
once completed)}
\begin{verbatim}
- [ ] have all the tasks in this Work Package a lead institution?
- [ ] have all deliverables in the WP a lead institution?
- [ ] do all tasks list all sites involved in them? 
- [ ] does the table of sites and their PM efforts match lists of sites for each task?
      (each site from the table is listed in all relevant tasks, and no site is listed
      only in the table or only at some task)
\end{verbatim}
\end{draft}

\begin{workpackage}[id=management,type=MGT,wphases=0-48!.2,swsites,
  title=Project Management,short=Management,
  lead=PS,
  PSRM=27,SARM=2, % Should be 26.5 for PS
  USORM=1,LLRM=2,UVRM=2,UJFRM=2,UBRM=2,UORM=2, USHRM=2,
  UWRM=2, JURM=1, UKRM=2, USRM=2, ZHRM=1, SRRM=2, UGRM=1, FAURM=1, XFELRM=1] % Should be 1.5 for UG
\begin{wpobjectives}
%  The objectives of this work package are to undertake all project management activities,
%  including:
%  \begin{compactitem}
%  \item monitoring the overall progress of the project and the use of
%    resources;
%  \item ensuring the timely production of deliverables and other
%    project outputs;
%  \item reporting to the European Commission on financial matters;
%  \item preparing for and attending the annual project review
%    meetings; and
%  \item managing the project Advisory Board.
%  \end{compactitem}

Establish and maintain an effective contract, project, and operational management
approach, ensuring (i) an effective and timely implementation of the project, (ii) quality control
of the results, (iii) risk and innovation management of the project as a whole as well as (iv)
timely and necessary interaction with the EC and other interested parties.

  % The objective of  is to undertake all project management
  % activities, including setting up joint infrastructure, organizing
  % meetings, and producing overview reports.
\end{wpobjectives}

\begin{wpdescription}
The project will be managed by UPS, which has extensive experience in
administering and leading EU funded and national projects. The
coordinator together with the WP leaders, will be responsible for
monitoring WP status, coordination of work plan updates and annual
internal progress reports. The project management structure and roles
of partners in the consortium are presented in Section~\ref{sect:mgt}.
\end{wpdescription}

\begin{tasklist}
\begin{task}[title=Project and financial management,
  id=project-finance-management,lead=PS,PM=33,wphases={0-48!.2,0-3,10-12,22-24,34-36,42-48},
  partners={LL,UV,UJF,UB,UO,USH,USO,SA,UW,JU,UK,US,ZH,SR,UG,FAU,XFEL},issue=14]
  The task includes the following activities
  \begin{compactitem}
  \item Preparation and Distribution of the
    Consortium Agreement;
  \item Setting up the project website, intranet and
    communication procedures for effective communication;
  \item
    Organisation of project review and progress meetings;
  \item
    Establishment and maintenance of external contacts (with the EC,
    other relevant national / EU projects, other academic and
    industrial stakeholders) to organise transfer of knowledge,
    present and promote project results;
  \item Progress and Financial Reporting to the EC;
  \item Data and IPR Management will be managed in accordance with
    agreed rules stated in the Consortium Agreement and in accordance
    with the Data Management Plans
    (\delivref{management}{infrastructure}, \delivref{management}{data-plan1}, \delivref{management}{data-plan2}).
  \end{compactitem}
\end{task}

\begin{task}[title=Quality assurance and risk management,id=project-quality-management,
  wphases=6-48!.3,
  lead=PS,PM=15,partners={LL,UV,UJF,UB,UO,USH,USO,SA,UW,JU,UK,US,ZH,SR,UG,FAU,XFEL},issue=15]
  A quality assurance plan will be established to ensure coherent and
  sufficient quality of the work and results. The plan will be
  developed by UPS, involving all partners, and will be followed up
  regularly. In addition, the project coordinator with support from
  the coordination team and quality review board will establish and
  review annually a risk management plan and self-assessment to ensure
  that technical barriers / potential risks are identified and
  corrective measures are put into place on time
  (\delivref{management}{ipr}).
\end{task}

\begin{task}[title=Innovation management,wphases=6-48!.2,
  id=project-innovation-management,lead=PS,PM=10,
  partners={LL,UV,UJF,UB,UO,USH,USO,SA,UW,JU,UK,US,ZH,SR,UG,FAU,XFEL},issue=16] One of the
  most important criteria for success for the OpenDreamKit project is
  to bring the project results into use. Therefore, exploitation
  routes will be sought whenever possible. In order to create a common
  understanding within the Consortium of how we can best shepherd an
  idea all the way from conception to realisation and
  exploitation, the Coordinator will be responsible for the
  preparation and realisation of an Innovation Plan. This plan will assure that
  research activities meet the required milestones and produce outputs 
  fully aligned with the project objectives.  All
  research activities will go through an initial process where the
  exploitation opportunity is identified along with the main
  stakeholders for the exploitation opportunity and an IP owner.
\end{task}
\end{tasklist}

%
%  This workpackage will perform all the activities related to monitoring of progress
%  towards the project milestones shown on Page~\pageref{sec:milestones} and the
%  deliverables listed on Page~\pageref{sec:deliverables}, assuring the quality of the
%  deliverables, ensuring the collation and distribution of the required reports,
%  questionnaires and deliverables including the annual reports to the European Commission,
%  arranging project management meetings, tracking the project budget in terms of
%  expenditure and person-months, obtaining financial certificates as required, convening
%  project management meetings, ensuring that important project documents such as the
%  project contract and the consortium agreement are properly maintained and amended as
%  necessary, ensuring that contractual details are complied with, monitoring compliance
%  with the grant agreement, preparing for the annual review meetings, and reviewing
%  research results against the aims and objectives of the project. It also involves
%  managing and supporting the project Advisory Board, including supporting attendance at
%  project meetings, convening Advisory Board meetings, and obtaining feedback on the
%  project direction and results.


\begin{wpdelivs}
\begin{wpdeliv}[due=1,miles=startup,id=infrastructure,dissem=PU,nature=DEC,lead=PS,issue=17]
  {Basic project infrastructure (websites, wikis, issue trackers, mailing lists, repositories)}
\end{wpdeliv}
\begin{wpdeliv}[due=6,miles=startup,id=data-plan1,dissem=PU,nature=R,lead=PS,issue=18]
  {Data Management Plan V1}
\end{wpdeliv}
\begin{wpdeliv}[due=12,miles=startup,lead=PS,id=ipr,dissem=CO,nature=R,issue=19]
  {Internal Progress Reports year 1, including risk management and quality assurance plan}
\end{wpdeliv}
\begin{wpdeliv}[due=18,miles=proto1,lead=PS,id=tickets,dissem=CO,nature=R,issue=20]
  {Innovation Management Plan v1}
\end{wpdeliv}
\begin{wpdeliv}[due=36,miles=community,lead=PS,id=ipr2,dissem=CO,nature=R,issue=21]
  {Internal Progress Reports year 2, including risk management and quality assurance plan}
\end{wpdeliv}
\begin{wpdeliv}[due=36,miles=community,id=data-plan2,dissem=PU,nature=R,lead=PS,issue=22]
  {Data Management Plan V2}
\end{wpdeliv}
\begin{wpdeliv}[due=45,lead=PS,miles=eval,id=tickets,dissem=CO,nature=R,issue=23]
  {Innovation Management Plan v2}
\end{wpdeliv}

%\begin{wpdeliv}[due=12,id=periodic-rep-1,dissem=PU,nature=OTHER]{First Annual Report (first year)}
% \end{wpdeliv}
%\begin{wpdeliv}[due=24,id=periodic-rep-2,dissem=PU,nature=OTHER]{Project Annual Report (second year)}
% \end{wpdeliv}
%\begin{wpdeliv}[due=36,id=periodic-rep-3,dissem=PU,nature=OTHER]{Project Annual Report (third year)}
% \end{wpdeliv}
%\begin{wpdeliv}[due=48,id=periodic-rep-4,dissem=PU,nature=OTHER]{Project Annual Report (fourth year)}
% \end{wpdeliv}

%Final report is not part of the deliveries.
%\begin{wpdeliv}[due=48,id=final-mgt-rep,dissem=PU,nature=OTHER]{Project Final Report}
% \end{wpdeliv}
\end{wpdelivs}
\end{workpackage}
%%% Local Variables: 
%%% mode: latex
%%% TeX-master: "../proposal"
%%% End: 

%  LocalWords:  workpackage wphases wpobjectives wpdescription pageref wpdelivs wpdeliv
%  LocalWords:  dissem mailinglists swrepository final-mgt-rep compactitem swsites ipr
%  LocalWords:  TOWRITE tasklist delivref
\newpage
\TOWRITE{ALL}{Proofread WP 2 Dissemination pass 2}

% This work package must start in month 1. Not sure what the
% appropriate load factor should be, and why that is important?
\begin{workpackage}[id=dissem,wphases=0-48!.5,
  short={Community Building/Dissemination},
  title={Community Building, Training, Dissemination, Exploitation, and Outreach},
  lead=PS,
  PSRM=9,
  SARM=18,
  USORM=1, %16 in total
  XFELRM=21, % moved 6PM from WP7 (15 + 6 = 21)
  USHRM=46,  % moved 24 PM from WP7 (22+24 = 46)
  USRM=30,
  UVRM=2,
  UKRM=2,
  UBRM=19,
  SRRM=2,
  LLRM=6,
  UGRM=1
]

\begin{wpobjectives}
  The objective of this work package is to further develop the community at the
  European scale, foster cross team collaboration, spread the
  expertise, and engage the greater community to participate in the
  definition and refinement of the requirements, and the implementation and use of the
  produced solutions. This includes:
  \begin{compactitem}
  \item ensuring awareness of the results in the user community;
  \item engaging cross communities discussions to foster scientific collaboration and conjoint development;
  \item spreading the expertise through workshops and training sessions;
  \item providing training for partners inside the project, the
    community engaging with contributions to the project, and
    end-users of \TheProject
  \item reviewing emerging technologies,
  \item develop demonstrators,
  \item defining individual exploitation plans; and,
  \item managing existing and new intellectual property.
  \end{compactitem}
\end{wpobjectives}

\begin{wpdescription}
  We will organise regular open workshops (e.g. Sage Days, PARI Days,
  summer schools, etc.); some of them will be focused on development
  and coding sprints, and others on training. This is also an occasion
  to organise cross community workshops like Sage-Jupyter days.

  Some of the networking activities and internal training will come
  from short- to long-term mutual visits by participants, to
  collaborate on specific features. A typical such visit would bring
  together a \Jupyter developer with a GAP developer for a couple of
  days to implement a first prototype of a notebook interface to GAP.

  A number of demonstrators will be developed in the project and
  disseminated in different ways.

  We will also participate in the concertation activities,
  consultations and other meetings and events of the European
  E-Infrastructure projects.

  All the code, documents, test and build infrastructure produced as
  part of the project will be made available as open source.
  Open access to all publications resulting from the project will be ensured.

  This work package will complement and lean on a parallel COST
  network which role is to build and engage a greater community.
\end{wpdescription}

\begin{tasklist}

\begin{task}[title=Dissemination and Communication activities, lead=PS, partners={SA}, id=dissemination-communication, PM=11, wphases=0-48, issue=24 ]
  This task comprises all forms of direct dissemination and public
  communication activities such as press releases, creation of the
  project web-site (\delivref{management}{infrastructure}) including visitor analysis and monitoring tools,
  scientific and technical publications, outreach activities
  (seminars, keynote talks, media interviews, press releases),
  promotion through social media (e.g. Twitter, Facebook, LinkedIn),
  creation of advertisement materials such as flyers, posters, and
  electronic feeds as well as their distribution. We will use standard
  community building technology such as mailing lists, Wikis and
  Forums, to ensure dissemination and engagement of the community to
  support this. We will also generate press releases at appropriate
  moments (\delivref{dissem}{press-release-1}, \delivref{dissem}{press-release-2}). %, making use of the public relation support services in the respective institutions.
  %At least two press releases will be
  %generated in the course of the project.


  % News articles will be produced by experienced professional staff
  % at relevant partners including ... and communicated to local,
  % national and international media, as appropriate.
\end{task}

\begin{task}[title=Training and training portal,
id=training-portal,lead=PS,PM=1,wphases={0-48!.1,0-1},issue=25]
Training is at the heart of this project: through our open source
approach, networking activities, workshops, demonstrators
(\localtaskref{index-librorum-salvificorum}), interactive books (such
as in \taskref{UI}{oommf-tutorial-and-documentation} and
\localtaskref{ibook}), and training for teachers and trainers
(\localtaskref{project-intro}), we have firmly integrated the training
aspect into the core of our project plan.

Each of these activities will create dedicated webpages hosting the
material to make it accessible to public as wide as possible ---
in line with our philosophy for software, we believe in the benefits of
sharing the material and maximising the value of the financial
investment into this project.

In this task, we create a central \TheProject training portal that
serves as an inclusive point of entry to explore available training
materials. This will be hosted on the projects website (\localtaskref{dissemination-communication}).
\end{task}

\begin{task}[title=Community Building: Development Workshops, lead=PS,PM=24, partners={UB,UK,SR,SA,USH,UG}, id=devel-workshops, wphases=0-48, issue=26]
  We will organise development workshops all throughout the
  project. The aim of these workshops is to bring together developers
  from the different communities to design and implement some key
  aspects of \TheProject such as user interface, and documentation and
  to ensure cross compatibility. These meetings will gather not only
  participants of \TheProject but also members of the different
  communities involved. Bringing talented people together is the best
  way to make actual progress on the different aspects of \TheProject
  and to kick-start new challenges. It is also a way to work within
  the communities we're reaching in and include them in the discussions
  and development. It fosters collaboration between scientists and
  developers from different backgrounds to build tools that are needed
  by all.

  Each workshop will be aimed at a specific software component (\Sage,
  \GAP, \SMC, \IPython, \Singular, etc.) or be joint meeting between
  different communities to improve interoperability and joint
  developments. We are planning to have 4 or 5 such events per
  year. The currently anticipated list of workshops includes:

\begin{compactitem}
\item One \Sage workshop per year in Cernay France (near Orsay) where
  similar gatherings took place before. One of them will be a
  Sage-Sphinx day, dedicated to documentation.

\item One Atelier \Pari in Bordeaux per year. The team in Bordeaux has
  a great experience in organizing this kind of \Pari events.

\item Two \Singular workshops and two \GAP-\Singular workshops in Kaiserslautern
  over the four years.

\item Two workshops dedicated to high performance mathematical
  computing in relations with \WPref{hpc}. One of them should be in
  Grenoble and the second one in Bordeaux to foster the work with
  \Pari towards \taskref{hpc}{hpc-pari}.

\item Two Data Science workshops which use \TheProject to develop effective machine learning and data modelling practice organised by Sheffield.

\item A joint meeting on the topics of \SMC and \Jupyter in Simula in
  relation with \WPref{UI}.

\item A joint event between \GAP, \Sage, and \Singular in ICMS,
  Edinburgh.

\item A joint \Jupyter and \Sage event in Orsay.

\item A joint \LMFDB and \Sage event in Warwick to work towards
  \WPref{dksbases}.

\end{compactitem}

Yearly reports on the impact of these workshops on the community (\delivref{dissem}{workshops-1},
\delivref{dissem}{workshops-2}, \delivref{dissem}{workshops-3},
\delivref{dissem}{workshops-4}) will be delivered.
\end{task}


\begin{task}[title=Reviewing emerging technologies, id=tech-review, lead=PS, partners={SA,USO,USH,US,UV,UB,SR,XFEL},PM=10, wphases=0-48,issue=27]
  In this task, we will produce periodic reviews (\delivref{dissem}{techno}) of emerging
  technologies and relevant developments elsewhere, and implications
  for our plans, taking into account input from the communities. This
  include the review of standard components and service for storage
  and sharing, computational resources, authentication, package
  management, etc. This may further include negotiating access or
  shared development when appropriate. This information will be fed to
  the other work packages, in particular Work
  Package~\WPref{component-architecture} Component Architecture.
\end{task}


\begin{task}[title=Dissemination: reaching towards users and fostering diversity, lead=PS,PM=12, partners={UB,USH,SA}, id=dissemination, wphases=0-48, issue=28]
  As lead developers of \TheProject, most of us consider themselves as
  both scientists and developers. We have experience in reducing the
  gap between those two worlds. We organise training workshops such
  as Sage-days to promote our tools and bring more users and
  developers from the scientific world. On the other hand, we often
  attend and present to more development-oriented gatherings like
  PyCon and SciPy to interact with engineers and foster
  collaboration.

  The aim of this task is to exploit this winning strategy within
  \TheProject. Three events should be organised in the spirit of
  Sage-days to gather and train more users and foster scientific
  development around \TheProject. These conferences usually welcome
  around 50 participants and have a big impact on the scientific
  community. One of them will be at CIRM in Marseille, another one at
  ICMS in Edinburgh and a third one probably in Dagstuhl, Germany. In
  the same spirit, we will also have training sessions organised
  within the universities (Orsay and Grenoble). We will also run a
  series of 4 workshops in developing countries, especially in Africa and
  South America. Some of these workshops will be connected to CIMPA
  schools.  The CIMPA is an international organisation based in Nice
  (France) that promote research in mathematics in developing
  countries. It organises each year around 20 schools.

  The under-representation of women in the scientific world is even
  more visible if we intersect science with software
  development. As we know, we have many talented women in our
  community, and we will organise some events targeted at women in the
  spirit of the "Women in Sage" days that happened many times in the
  US already. We are planning to have two of them in Orsay and at least
  one in Oxford where "Women in CS" days already take place on a regular basis.

  Apart from these different events, we will also be present at
  important events of both our scientific community (international
  mathematical conferences such as FPSAC for combinatorics) and the
  python / open-source software development community: PyCon, SciPy,
  EuroPython, etc. The material we develop for presentation at these
  events will be made publicly available.
\end{task}


%Mike Croucher and Neil Lawrence,Sheffield
\begin{task}[title=Introduce \TheProject to Researchers and Teachers, id=project-intro,lead=USH,wphases=6-44,PM=20,partners={USO,XFEL}, issue=29]
  In this task, we will develop and deliver materials that will introduce \TheProject to
  potential users---both researchers and teachers. Our initial focus will be on teachers,
  but as the results from \WPref{social-aspects} become available we will deploy them with
  researchers, both local to the University of Sheffield and across the wider
  computational biology and machine learning fields.

  We will develop a `taster' seminar (1-2 hours) and follow-up short course
  (1-2 days) on \TheProject for researchers and lecturers in all
  disciplines \delivref{dissem}{short-course}. At Sheffield, this will
  be added to the set of courses that are offered as part of IT
  Services' research support department. As such, it could potentially
  reach all disciplines. It will also be made publicly available for
  widespread dissemination and collaborative modification.

  Elements of this work will also be integrated with the GP Summer
  Schools and Roadshows (\url{http://ml.dcs.shef.ac.uk/gpss/}). The
  Summer School is now in its fourth edition (over 140 students
  educated). The Roadshows have taken place in Uganda, Colombia and in
  2015 they are scheduled for Italy, Australia and Kenya. The Kenya
  school will be the first to have more of a `data science' focus that
  we think will be particularly appropriate for dissemination of
  \TheProject (\delivref{dissem}{datascience-course}).

  These seminars and short courses will also be used to identify
  potential collaborators who are interested in utilising \TheProject
  immediately. We will act as consultants to these collaborators in
  two ways:

  We will work with lecturers at Sheffield to introduce \TheProject to
  various disciplines via the production of interactive lecture notes
  (\delivref{dissem}{lecture-notes}). The focus for the student here
  will not necessarily be on programming but rather on interaction
  with the subject matter via use of \TheProject. Interactive lecture
  notes are an area where commercial vendors such as MapleSoft and
  Wolfram Research are spending a lot of time and money developing
  material. We will provide technical and programming expertise to
  lecturers---helping them to develop the interactive part of notes
  while they provide the subject material.

We will work as consultants with researchers at Sheffield to
  introduce \TheProject to their workflow. Any projects that
  successfully do this will be promoted as case studies for
  \TheProject.

We have a collaboration with a lecturer from \href{https://en.wikipedia.org/wiki/Parthenope_University_of_Naples}{Partenope,
  University of Naples} who would like us to deliver our short course to lecturers there
  and to assist in the migration of some of their courses to \TheProject technologies.

We will use these experiences of migrating courses at Sheffield and Naples to \TheProject technologies
to develop a web template that allows the straightforward development of course websites from a
combination of GitHub and a folder-full of Jupyter notebooks.

The integration of Sun Grid Engine and Project Jupyter, done as part of \TheProject, has led to some educators considering using the notebook to introduce various aspects of High
Performance Computing. This includes GPU, OpenMP and MPI
  programming. \textbf{We have emerging collaborations with
    \href{https://notebooks.azure.com/}{Microsoft},
    \href{http://alces-flight.com/}{AlcesFlight}, and
    \href{http://gpucomputing.shef.ac.uk/}{Sheffield's GPU community}
    which will help further this aspect of \TheProject dissemination into the
    HPC community}.

The development of nbval and nbdime within \TheProject is
leading to some researchers switching to using the notebook for all
of their software documentation and tutorials. This is because
nbval makes it possible to ensure that such documentation will
always work as development of code progresses.
Using nbval allows the documentation to become part of
the formal testing framework.

Building on the work of nbval and nbdime,
we will deliver a workshop on reproducible science using \TheProject at
the UK national Research Software Engineering conference in 2017.

We will work with \href{http://rse.shef.ac.uk/blog/sheffield-code-firstgirls/}{Sheffield Code First: girls}, which introduces programming to women across the UK,
  to incorporate \TheProject into their curriculum.
This potentially has great impact for women in STEM subjects.

 By the end of the
  project we will have produced a series of interactive notebook
  demonstrators \localdelivref{notebook-repo} of \TheProject with a
  particular focus on computational biology, data science and machine
  learning. These notebooks will expand the use of VREs in these
  domains, appealing to researchers used to the domains of
  Bioconductor and MATLAB. We will make use of live notebook posters
  (\delivref{social-aspects}{social-poster}) and commenting systems
  (\delivref{social-aspects}{jupyter-comment}). These interactive notebooks
  will be provided in a public repository
  (\delivref{dissem}{notebook-repo}).
\end{task}

\begin{task}[id=dissemination-of-oommf-nb-virtual-environment,
  title=Open source dissemination of micromagnetic VRE,
  lead=XFEL,PM=4,partners={SR,USH,PS},wphases=28-32,issue=30]
  % 3 months person time + 1 months investigator time
  Tasks \taskref{UI}{oommf-py-ipython-attributes} and
  \taskref{UI}{oommf-tutorial-and-documentation} provide the
  micromagnetic VRE demonstrator
  (\ref{sec:introduction-micromagnetic-vre-demonstrator}) built on top
  of \TheProject.  In this task, we set up of the infrastructure
  to encourage
  and invite code contributions from the micromagnetic community to
  both code and created notebooks, while automating quality control
  and maintaining trust effectively (\delivref{dissem}{oommfnb-vre-deliver}).

  The source code of the micromagnetic VRE will be made available as
  open source on public repository hosting sites (such as
  GitHub/Bitbucket), and announced to the community via appropriate
  mailing lists and other means. We will set up a publicly accessible
  Jenkins/Travis continuous integration (CI) system to (i) run
  regression tests (from
  \taskref{component-architecture}{oommf-python-interface} and
  \taskref{UI}{oommf-py-ipython-attributes}) routinely when the
  micromagnetic VRE code or underlying OOMMF core code changes, (ii)
  re-execute notebooks (from
  \taskref{UI}{oommf-tutorial-and-documentation}) and use them as
  regression tests (using the outcome of task
  \taskref{UI}{notebook-verification}), and (iii) re-build
  downloadable installation files and virtual machine images.
  %This set
  %up will test user-contributions automatically.

  %versions (\delivref{dissem}{oommfnb-source-and-testing-setup}).

\end{task}

\begin{task}[title=Micromagnetic VRE dissemination workshops,
id=dissemination-of-oommf-nb-workshops,lead=XFEL,PM=6,partners={USO},wphases=18-44!0.46,issue=31] % funny ratio here, that's okay (HF, MB)

  % 3 months person time, 2 months investigator time

  We will run a series of workshops
  (\delivref{dissem}{oommfnb-vre-deliver}) during the evenings of 4
  major international meetings on magnetism
  research\footnote{Anticipated most significant international
    meetings in the appropriate time frame are 61st Conference on
    Magnetism and Magnetic Materials (MMM2016), October 31-November 4,
    2016, New Orleans, Louisiana; 62nd Conference on Magnetism and
    Magnetic Materials (MMM 2017), November 6-10, 2017, Pittsburgh,
    Pennsylvania; 21st International Conference on Magnetism (ICM
    2018), July 16–20, 2018, San Francisco, California; 14th Joint
    MMM-Intermag Conference (MMM2019), January 14-18, 2019,
    Washington, DC). Each of those meetings is one week long, and
    serves as a focal point of networking for the european and
    international research community. Other training events have been
    held in the past at these conferences and were well attended.} to
  disseminate the micromagnetic virtual research environment
  (Sect.~\ref{sec:introduction-micromagnetic-vre-demonstrator} and
  \localtaskref{dissemination-of-oommf-nb-virtual-environment}) in the
  micromagnetic community. Each conference attracts around 1500
  participants, and we expect at least 30 for our workshops at every
  event. Depending on demand, multiple workshops will be given per
  conference.

  The taught material will include (i) use of the \Jupyter-based
  micromagnetic VRE, and an (ii) introduction to the standard
  techniques for contributing to open source software (version
  control, pull requests, testing frameworks) to foster excellence in
  computational science and to make the micromagnetic VRE project
  self-sustaining as quickly as possible. In addition, all teaching
  materials, including videos, will be made available on a website.

  For each workshop, we request \euro{500} room hire at the magnetism
  conference location and the travel expenses for two teachers from
  Southampton to attend the one week international conference,
  totalling (\euro{500} + 2x\euro{2200}=\euro{4900}) per
  workshop. There are no other costs.

  We will use the micromagnetic VRE demonstrator
  (\taskref{UI}{oommf-tutorial-and-documentation}), its dissemination
  workshops (\taskref{dissem}{dissemination-of-oommf-nb-workshops})
  and interactions with its users and contributors in the
  micromagnetic community to evaluate, reflect and report on the
  project, taking into account technical and social aspects. A survey
  will be developed and used to gather user input and feedback on
  usefulness of the provided capabilities, with particular focus on
  the capabilities of the micromagnetic VRE to (i) enable new and
  better science, to (ii) allow to make progress effectively, to (iii)
  carry out computational science reproducibly, to (iv)
  collaboratively enable trust and to (v) become a self-sustained
  project from community contributions. Amongst other channels, we
  will target attendees of the micromagnetic VRE dissemination
  workshops (\taskref{dissem}{dissemination-of-oommf-nb-workshops}) to
  gather data.

  All results and insights will be summarised in a public document and
  reported at appropriate workshops and conferences to share the
  lessons learned from this \Jupyter-based VRE for micromagnetics. We
  will create a manuscript for journal publication, summarising the
  demonstrator project and this evaluation. An important point of this
  publication is to provide a reference that can be cited by
  publications making use of the new micromagnetic VRE, to allow
  tracking of uptake and development of this VRE beyond the life time
  of this H2020 project.
\end{task}

\begin{task}[title=Demonstrator: Interactive books,
id=ibook,lead=US,partners={USO,XFEL},PM=42,wphases={0-46,40-46},issue=32]
  % 2x12 _ 3x 3 months for students
  % 6 months Southampton, ibook4: maths for engineering, in months 40-46
One of the important elements of VREs is a common flexible writing format which
enables the creation of large structured documents. There are many
known solutions to that problem, but they usually compromise the
interactivity of the notebook interface and typesetting quality of desktop
publishing software like LaTeX.

Recently, a few approaches tried to bring both interactivity and the
typographic features. The modestly tagged markup language
\href{http://hplgit.github.io/doconce/doc/web/}{DocOnce}
targets the problem of reusability of the document source code for
producing traditional LaTeX-based printed books, IPython notebooks, Sphinx
documents (with Sage cells), and many other formats. MathBook XML
is a lightweight XML application for authors of scientific articles,
textbooks and monographs extensively using Sage cells for
interactive elements. The Sphinx documentation software has been
successfully applied for creation of interactive books containing Sage
cells. Additional interactivity is offered using the \href{http://runestoneinteractive.org}{Runestone tools}.

The technical aspects of format for interactive publications is a
subject of the task ``Structured documents'' in
\taskref{UI}{structdocs}. In this task we will demonstrate usability
of the results of \taskref{UI}{structdocs} in creation of scientific
textbooks. Three interactive books will be created:

\begin{compactitem}
\item Nonlinear Processes in Biology (\delivref{dissem}{ibook1})
\item Linear Algebra (\delivref{dissem}{ibook2})
\item Computational Mathematics for Engineering  (\delivref{dissem}{ibook3c})  % removed ibook4 to reduce number of deliverables as requested
                                                                               % point to ibook 3c instead.
\item Problems in Physics with Sage/Python (\delivref{dissem}{ibook3c})
\end{compactitem}

The choice of those particular topics has been made for the sake of
maximal diversity. The ``Nonlinear Processes in Biology'' will heavily
use numerical solution of ODEs and PDEs, the Linear Algebra book will
be a classical mathematical textbook, while the next book is
targetting the engineering community.  The last example will focus
mostly on collaborative editing and modularity of content which is
produced using VRE technologies. We will demonstrate the power of
symbolic algebra where appropriate throughout.  All books are natively
designed within the VRE, pushing the next generation of learning
methodology out into multiple communities.

The main research aspect for this task will be to integrate modern
computational tools in classical scientific topics and explore how
the VRE environment can accelerate the development and produce electronic
documents with significantly enhanced pedagogy and learning effectiveness.

In particular we will answer following questions:
\begin{compactitem}
\item When is a fully interactive worksheet required and when is
  a textbook with executable code cells sufficient?
\item How to assemble a classical monograph by reusing independently working
  building block of text and code?
\item What are best tools and practices for using a single source for
  producing printed and electronic (interactive) textbooks?
\item How to collaboratively write reusable course material?
\item How can we facilitate automatic testing of all code examples, plots, etc?
\item How can students can benefit from using VRE?
\end{compactitem}
\end{task}

\begin{task}[title=Demonstrator: Computational mathematics resources indexing service,
id=index-librorum-salvificorum,lead=UV,PM=2,partners={UB},wphases=20-25,issue=33] Beyond official documentation and
  tutorials, users of mathematical software and VREs learn from a wide
  array of sources: university courses, Q\&\ A sites, web searches,
  etc.  A simple web search on any major software component yields
  dozens of non-official tutorials and how-tos in many different
  languages. However, search engines mostly miss the relevant
  metadata: how does one find a tutorial on linear algebra in \PariGP,
  written at an undegraduate level, in French or Spanish?

This need has been felt by most communities at some point, and each
has come up with its own solution: most software components (e.g.,
\GAP, \PariGP, \Sage, \dots) simply link material from their official
page; \Sage has a wiki (\url{http://wiki.sagemath.org/}) referencing
additional resources, and used to host a large number of tutorial
worksheets on \url{http://sagenb.org/}; the recent introduction of
public projects in \SMC is sparking approximately the same phenomenon
that had previously happened with \url{http://sagenb.org/}; \IPython
host the Notebook Viewer service (\url{http://nbviewer.ipython.org/}),
which renders (without hosting) community-made notebooks; and teaching
institutions host or link their own collections of pedagogical
resources.

These collections are usually incomplete, limited in scope, hard to
search, and difficult to keep up-to-date.  What the community needs is a
community-curated, searchable, metadata-driven, multilingual, platform
agnostic indexing service whose goal is to reference and rank all the
community generated knowledge around a software component or VRE.

The goal of this task is to create the tool powering such service, and
to host a (free) community-curated index for \TheProject related
resources as a demonstrator (\delivref{dissem}{ils-tool}).

\end{task}




\end{tasklist}



%Raw material:
%\begin{compactitem}
%\item Documentation improvements: overview, cross links, overview of
%  recent improvements
%\item Thematic tutorials
%\item Collections of pedagogical documents\\
%  E.g. a complete collection of interactive class notes with computer
%  lab projects for the ``Algèbre et Calcul formel'' option of the
%  French math aggregation (starting from 2014-2015, only open-source
%  systems will be supported, and Sage is a major player).
%  % See http://nicolas.thiery.name/Enseignement/Agregation/ as a starter
%  % Math labs with Sage for first year students in France (L1): http://math.univ-lyon1.fr/~omarguin/
%\item Localization of the Sage user interface and key documents in
%  various European languages.
%\item Distribution of the documents either in the main distribution of
%  Sage or through the online repository (see collaborative tools).
%\item Massive online introduction course to Sage, drawing on the sage tutorial/notebooks.
%Could be ``First year Sage course in a box''.
%\item Taking the opportunity of Python courses to propose Sage as a natural extension
%for mathematics; an example is French's
%% The url macro eats the accented letters.
%% It doesn't just eat it, it pukes it back!
%``Classes pr\'eparatoires''
%%\footnote{\url{http://en.wikipedia.org/wiki/Classe_préparatoire_aux_grandes_écoles}},
%where Python has been recently selected as the language to learn programming\footnote{See
%the ``Annexe'' at
%\url{http://www.education.gouv.fr/pid25535/bulletin_officiel.html?cid_bo=71586}}.
%%\item \TODO{please expand!}
%\end{compactitem}

% Jeroen: About teaching: in Gent, Sage is already integrated in the
% courses (maybe you can add this, don't know if it's relevant)
% starting in the first year. It's good for the students because it
% helps in 2 ways: it helps them to understand the mathematics better
% and it helps them to learn basic down-to-earth programming (they
% also have a programming course in Java but that contains a lot of
% theory about complicated class structures)
% Same thing in Orsay
% More python centered but same in UZH
% We have also Sage @ Silesia from 1st semester (physics)

\begin{wpdelivs}
  \begin{wpdeliv}[due=6,miles=startup,id=press-release-1,dissem=PU,nature=DEC,lead=PS,issue=34, status=delivered]{Starting press release}\end{wpdeliv}
  \begin{wpdeliv}[due=12,miles=startup,id=workshops-1,dissem=PU,nature=R,lead=PS,issue=42, status=delivered]{Community building: Impact of development workshops, dissemination and training activities, year 1}\end{wpdeliv}
%  \begin{wpdeliv}[due=12,miles=startup,id=ibook3a,dissem=PU,nature=DEM,lead=US]{Demonstrator: Problems in Physics with Sage v1} \end{wpdeliv}
  \begin{wpdeliv}[due=12,miles=startup,id=techno,dissem=PU,nature=R,lead=PS,issue=43, status=delivered]{Review on emerging technologies} \end{wpdeliv}
 \begin{wpdeliv}[due=18,miles=proto1,id=short-course,dissem=PU,nature=DEC,lead=USH,issue=44, status=delivered]{A short course for lecturers on using \TheProject for delivering mathematical education.}\end{wpdeliv}
  \begin{wpdeliv}[due=24,miles=proto1,id=datascience-course,dissem=PU,nature=DEC,lead=USH,issue=45]{Course material on using \TheProject in data science and education}
  This deliverable was merged into \localdelivref{IntroODK}.
  \end{wpdeliv}
  \begin{wpdeliv}[due=24,miles=proto1,id=workshops-2,dissem=PU,nature=R,lead=PS,issue=46,status=canceled]
    {Community building: Impact of development workshops, dissemination and training activities, year 2}
    This deliverable was merged into~\localdelivref{workshops-3}.
  \end{wpdeliv}
  \begin{wpdeliv}[due=24,miles=proto1,id=ils-tool,dissem=PU,nature=DEM,lead=UV,issue=47]{Community-curated indexing tool (open source)} \end{wpdeliv}
% merged with ils-tool  \begin{wpdeliv}[due=24,miles=proto1,id=ils-service,dissem=PU,nature=DEM,lead=UV]{Community-curated indexing service for \TheProject} \end{wpdeliv}
  \begin{wpdeliv}[due=24,miles=UI-vre-prototype,id=ibook2,dissem=PU,nature=DEM,lead=US,issue=48,status=canceled]
    {Demonstrator: Linear Algebra - interactive book}
    This deliverable was merged into~\localdelivref{ibook1}.
  \end{wpdeliv}
% merged with workshop reports  \begin{wpdeliv}[due=24,miles=proto1,id=dissem-1,dissem=PU,nature=R,lead=PS]{Impact of dissemination and training activities, years 1 and 2}\end{wpdeliv}
%  \begin{wpdeliv}[due=30,miles=community,id=ibook3b,dissem=PU,nature=DEM,lead=US]{Demonstrator: Problems in Physics with Sage v2} \end{wpdeliv}
  \begin{wpdeliv}[due=36,id=ibook1,miles=UI-vre-prototype,dissem=PU,nature=DEM,lead=US,issue=49]
    {Demonstrator: interactive books on Linear Algebra and Nonlinear Processes in Biology}
  \end{wpdeliv}
  \begin{wpdeliv}[due=36,id=lecture-notes,miles=UI-vre,dissem=PU,nature=DEM,lead=USH,issue=35,status=canceled]
    {Demonstrator: Interactive lecture notes and marking systems based on \TheProject}
    This deliverable was merged into \localdelivref{IntroODK}.
  \end{wpdeliv}
  \begin{wpdeliv}[due=36,id=workshops-3,dissem=PU,miles=eval,nature=R,lead=PS,issue=36]
    {Community building: Impact of development workshops, dissemination and training activities, year 2 and 3}
  \end{wpdeliv}
  \begin{wpdeliv}[due=44,miles=eval,id=notebook-repo,dissem=PU,nature=DEM,lead=USH,issue=37,status=canceled]
    {Demonstrator: Repository of interactive Notebooks in Machine Learning and Computational Biology based on \TheProject.}
    This deliverable was merged into \localdelivref{IntroODK}.
  \end{wpdeliv}
  \begin{wpdeliv}[due=48,miles=UI-vre,id=oommfnb-vre-deliver,dissem=PU,nature=OTHER,lead=XFEL,issue=38]{Micromagnetic VRE completed and online} \end{wpdeliv}
%  \begin{wpdeliv}[due=47,miles=eval,id=ibook3c,dissem=PU,nature=DEM,lead=US]{Demonstrator: Problems in Physics with Sage v3} \end{wpdeliv}
%  \begin{wpdeliv}[due=47,miles=eval,id=ibook4,dissem=PU,nature=DEM,lead=USO]{Demonstrator: Computational Mathematics for Engineering} \end{wpdeliv}
% Combine the above demonstrators into one Deliverable
  \begin{wpdeliv}[due=47,miles=eval,id=ibook3c,dissem=PU,nature=DEM,lead=US,issue=39]{Demonstrators: Problems in Physics with Sage, Computational Mathematics for Engineering}
  \end{wpdeliv}

  \begin{wpdeliv}[due=48,miles=eval,id=workshops-4,dissem=PU,nature=R,lead=PS,issue=40]{Community building: Impact of development workshops, dissemination and training activities, year 4}\end{wpdeliv}
% merged with workshop reports  \begin{wpdeliv}[due=48,id=dissem-2,dissem=PU,nature=R,lead=PS]{Impact of dissemination and training activities, years 3 and 4}\end{wpdeliv}
  \begin{wpdeliv}[due=48,miles=eval,id=press-release-2,dissem=PU,nature=DEC,lead=PS,issue=41]{Ending press release}\end{wpdeliv}
% The deliverable below is a result of merging datascience-course, lecture-notes and notebook-repo
\begin{wpdeliv}[due=48,miles=UI-vre,id=IntroODK,dissem=PU,nature=DEC,lead=USH,issue=250]{Introduce OpenDreamKit to Researchers and Teachers as laid out in Task 2.6}\end{wpdeliv}
\end{wpdelivs}
\end{workpackage}

%%% Local Variables:
%%% mode: latex
%%% TeX-master: "../proposal"
%%% End:

%  LocalWords:  workpackage dissem wphases wpobjectives wpdescription tasklist WPref nmag
%  LocalWords:  delivref linkedin organisation finalpressrelease organise wpdelivs github
%  LocalWords:  wpdeliv dissemination-of-oommf-nb-virtual-environment OOMMFNB taskref
%  LocalWords:  oommf-python-interface oommf-tutorial-and-documentation mumag magpar
%  LocalWords:  mumax micromagnum micromagnetic oommf-py-ipython-attributes summarising
%  LocalWords:  dissemination-of-oommf-nb-workshops localtaskref MMM-Intermag Fangohr
%  LocalWords:  maximise sagecell structdocs Algèbre Calcul formel eparatoires Annexe
%  LocalWords:  Jeroen
\newpage
\TOWRITE{ALL}{Proofread WP 3 Component Architecture pass 2}
\begin{draft}
\TOWRITE{UV (Work Package Lead)}{For WP leaders, please check the following (remove items
once completed)}
\begin{verbatim}
- [X] have all the tasks in this Work Package a lead institution?
- [X] have all deliverables in the WP a lead institution?
- [X] do all tasks list all sites involved in them?
- [X] does the table of sites and their PM efforts match lists of sites for each task?
      (each site from the table is listed in all relevant tasks, and no site is listed
      only in the table or only at some task)
\end{verbatim}
\end{draft}

\begin{workpackage}[id=component-architecture,wphases=0-48!.5,
  title=Component Architecture,lead=UV,
  PSRM=46,UVRM=8,SARM=16, USORM=6, UORM=4, LLRM=14, UJFRM=6, UGRM=14]
  % PS: Full time dev: 48 PM, NT: 4 PM

  \begin{wpobjectives}
    The objective of this work package is to develop and demonstrate a
    set of APIs that enable components, such as database interfaces,
    computational modules, separate systems such as \GAP or \Sage, to
    be flexibly combined and run smoothly across a wide range of
    environments (such as Cloud-based, local, and server environments).
  \end{wpobjectives}

  \begin{wpdescription}
    This Work Package focuses on the structure of the components that make
    up a mathematical software and their interactions. Such components
    can be separate modules inside a unique software, or separate
    softwares interacting through library calls and/or through APIs
    (e.g.: web APIs). When combined together, they make up a full VRE.

    The architecture of these software components must be:
    \begin{compactitem}
    \item \textbf{Portable}, to support a wide range of platforms
      (mobile, desktops, cloud, \dots).
    \item \textbf{Modular}, so to ease installing, building, testing,
      and remixing.
    \item \textbf{Flexible}, so to adapt to different use cases:
      personal computation, HPC, parallel platforms, \dots
    \item \textbf{Open}, in the sense of \emph{open source}, but also
      in the sense of clearly documented and open to
      the user who wants to understand its underpinnings and/or
      contribute to it. Indeed we must not forget that the working
      mathematicians and other users need to know what algorithms the software is
      going to run to solve a given problem.
    \end{compactitem}
  \end{wpdescription}

  \begin{tasklist}
  \begin{task}[id=portability,title=Portability,lead=UV,PM=35,partners={PS,UG,UO},wphases=0-48,issue=50]
    In order to achieve maximum availability and accessibility,
    mathematical software must be developed and tested for a wide range
    of computer architectures and operating systems.  However most of
    open source development happens in POSIX environments (usually
    Linux or OS X), and almost exclusively on x86 platforms.  The vast
    majority of the developers of mathematical software does not have
    the expertise, nor the access to appropriate hardware and software, to insure
    appropriate testing and porting of components.  The best
    incarnation of this issue is the involved installation procedure
    for \Sage on Windows, a major adoption barrier and common source of
    complaints by end-users.
    Other urgent tasks include porting \Sage to \Python 3, 
    and porting \Sage to the primary OSX C/C++ compiler \clang.
    The latter is important for porting of \Sage to Conda, see
    \taskref{component-architecture}{mod-packaging}.

    In this task we will address the common needs of the community in
    terms of portability layers, building and testing infrastructure.

    \begin{compactitem}
    \item Best practices adopted by the larger open source community
      will be investigated and leveraged, and existing expertise will
      be shared between the component developers.
    \item Windows being largely dominant in the desktop/laptop market,
      a specific focus will be placed on the port of \Sage, and
      therefore all the components included in its distribution (in
      particular \PariGP, \GAP, \Singular, \Linbox) to this platform
      (\delivref{component-architecture}{portability-cygwin}).
    \item The deployment of a common infrastructure for multi-platform
      continuous integration (testing, building and distribution) will
      be addressed
      (\delivref{component-architecture}{multiplatform-buildbot}).
    \item Porting \Sage to \Python 3.  As
    several crucial components of \Sage will cease to be available for \Python 2 in the coming
    2 to 3 years, this is very urgent, although a considerable amount of work on this has already
    been done.
    \item Porting \Sage to \clang; as the MacOS vendor (Apple) adds more and more
    Objective C-style code to the system headers, more and more \Sage components need to be
    built with \clang rather than with \gcc. A full port of \Sage to \clang is thus very
    desirable, and a considerable amount of work in this direction has been done.
    \item Porting \Sage to more \Fortran compilers gets urgent, especially as new
    high-performance \Fortran compilers become available.
    \end{compactitem}

      % Jean-Pierre:
      % Should we mention port to non-x86-64 archs and non-Linuces?
      % 
      % For CPUs:
      % - I guess at least ARM and ppc64 (IBM POWER*) really make sense.
      % - Sparc is less convincing though the latest sparc CPUs
      % are muche more interesting for math computation as the
      % previous ones, e.g. the GMP folk specifically added assembly
      % for them in their latest release.
      % - Itanium is dead, but it can help discovering bugs as any non
      % standard archs.
      % - Supporting any of these would mean buying (potentially very
      % expensive) hardware.
      % 
      % For OSes?
      % - Should we mention OS X which is a pain at each new release?
      % - A BSD variant would be interesting, let's say FreeBSD which
      % is basically (almost) already supported
      % - Solaris? and/or OpenIndiana? Interesting if we mention sparc...
      % - Windows is already included below, my opinion is:
      % * provide live USB, VMs and Cygwin32 first as these three are
      % basically already working solutions
      % * go Cygwin64 as it is still POSIX
      % * explorate a MinGW solution, at least GAP and PARI should be
      % problematic
      % * try to use MSVC
  \end{task}

  \begin{task}[title=Interfaces between systems,id=interface-systems,lead=PS,PM=22,partners={UV,UO,SA,UG},wphases=0-36,issue=51]
    In this task we will investigate patterns to share data,
    ontologies, and semantics across computational systems, possibly
    connected remotely.  We will leverage the well established
    semantics used in mathematics (categories, type systems, \dots) to
    give powerful abstractions on computational objects.
    
    We will build upon the work already done in the EU FP6 project
    26133 ``SCIEnce -- Symbolic Computation Infrastructure for
    Europe'' (\url{http://www.symbolic-computing.org/}) on the Symbolic Computation
    Software Composability Protocol (SCSCP). SCSCP is a remote
    procedure call protocol by which a computer algebra system (CAS)
    may offer services to a variety of possible clients, including
    e.g.  another CAS running on the same computer system or remotely;
    another instance of the same CAS (in a parallel computing
    context); a simplistic SCSCP client
    (e.g. C/C++/Python/etc. program) with a minimal SCSCP support
    needed for a particular application; a Web server which passes on
    the same services as Web services, etc.  A distinctive feature of
    the protocol is that both instructions and data are represented in
    the OpenMath format (\url{http://www.openmath.org/}; previously
    supported by the EU JEM Thematic Network; EU project 24969
    ``ESPRIT'' and other projects); moreover, OpenMath support is not
    limited by existing official OpenMath content dictionaries --
    private encodings may be easily embedded into SCSCP messages.
    
    SCSCP is already supported by a number of computer algebra
    systems, including \GAP, Macaulay2, Maple, TRIP and others. We
    will extend support for SCSCP to other relevant systems involved
    in \TheProject (\delivref{component-architecture}{scscp-sage}).
    Through its API, we will enable discovery of subsystems,
    functionality, documentation and computational resources. The user
    interfaces shall be enabled to automatically choose the best
    available algorithms and resources to perform a required
    computation, as well as clearly and intuitively present the
    available choices to the expert user.

    As a first concrete test bed, we will consider the \Sage interface
    to \GAP, or more precisely \libGAP.  Like most \Sage interfaces,
    this uses the now classical \emph{handle} design pattern, whereby
    one can manipulate from \Sage an object created and stored in
    \GAP, through a \emph{handle} (a.k.a. \emph{remote objects}).  By
    mapping \GAP's categories to \Sage's categories, in
    \delivref{component-architecture}{semantic-interface-sage-gap} we
    will:
    \begin{compactitem}
    \item Implement a modular infrastructure for adapters, based on
      SCSCP, in order to let the implementation of adapters scale to a
      large variety of objects.
    \item Refactor the existing adapters, using this infrastructure to
      generalise their features. This step by itself will provide
      adapters for larger categories like semigroups or monoids.
    \item Merge the adapters into the handles, so that a handle to a
      \GAP group will \emph{automatically} behave like a native \Sage
      group.
      % This will remove much back-conversion burden from the
      % delegating methods
    \end{compactitem}
    A specific challenge will be performance; indeed low level method
    adapters, e.g. for arithmetic, need to be compiled when most of
    the interface infrastructure is dynamic by nature.

    % When different algorithms are available, some of them coded in
    % \Sage, some of them in \GAP, the interface shall offer an easily
    % navigable interface for the expert user to choose among them.
  \end{task}

  \begin{task}[title=Modularisation and packaging,id=mod-packaging,lead=UV,PM=28,partners={PS,LL,UG,UO},wphases=0-48,issue=52]
    % TODO: logilab can contribute to producing VM images
    % using its proven worflow based on saltstack and packer.io PM=2
    % TODO: logilab can help to define the metadata to be provided
    % by software authors to facilitate the packaging of their
    % components. Beware not the reinvent the wheel^H packaging system.
    % PM=2
    % TODO: if needed, logilab can develop a package index similar
    % to PyPI using CubicWeb: html UI for browsing and web service
    % for registration. Maybe an instance of PyPI is enough? PM=6
    % TODO: logilab can help with debian packages PM=6
    In this task we will investigate best practices for composing,
    sharing and interfacing computational components and data for
    connected mathematical systems.

    We will start with a comparative study of the practices adopted in
    various open source projects, both inside and outside of
    \TheProject. This will include reviewing non-mathematical systems,
    e.g.: operating systems, platforms, web frameworks, cloud and HPC
    infrastructures.  In particular, pushed by cloud computing,
    containerisation \cite{Docker} and virtualisation
    \cite{Virtualbox} have become a major trend for distributing
    complex software, thanks to their ease of installation and
    configuration. We shall experiment with these technologies by
    building and distributing virtual appliances for the major
    components of \TheProject
    (\delivref{component-architecture}{virtual-machines}).

    Once the initial study will have identified the present
    shortcomings, we will promote a new generation of mathematical
    software that is capable of scaling to large code bases, large
    datasets, and massively distributed infrastructures. This task
    also needs to consider the results of work
    package~\WPref{social-aspects} on social issues regarding
    distributed development, community management, acknowledging
    contributions, etc.

    As an example, \Sage has a long history of integrating and
    distributing large mathematical libraries/software as a whole,
    with relatively few attention given to defining and exposing
    interfaces. Component re-usability is not a main focus for the
    \Sage community, at the same time the non-standard and relatively
    underused package system discourages writing and maintaining
    autonomous libraries. These factors have contributed to make the
    \Sage distribution what is usually described as a ``monolith''
    (\Sage library code alone, not counting included libraries, makes
    up for 1.5M lines of code and documentation), hard to distribute,
    to maintain, to port, and to develop with.

    On the other hand, \GAP has been distributing
    community-developed ``\GAP packages'' for a long time, but faces
    now fragmentation issues, at the code and at the community
    level. The rudimentary package system adds more technical
    difficulties to \GAP's development model.

    Both models reach the limits of their scalability, and a synthesis
    is very much needed.  Our first experiment will be to enhance
    \Sage's package system
    (\delivref{component-architecture}{sage-repository}), enough to
    support an open repository of user-contributed code, in the same
    spirit of modern systems such as \Julia
    (\url{http://pkg.julialang.org/}),  PyPI
    (\url{https://pypi.python.org/}), and \href{https://conda.io/docs/}{Conda}.
    Once \emph{internal} packaging
    has been dealt with, the route will be paved to further modularise
    the \Sage distribution, and make sure that the major Linux
    distributions have standard packages for it
    (\delivref{component-architecture}{sage-distribution}).
    
    It also becomes urgent to work on a link between \Julia and \Sage,
    in particular as a major component of \Sage, \Singular, is moving towards
    use of \Julia as its main language, as well as completely new computer
    algebra systems written in \Julia, such as \href{http://nemocas.org/}{Nemo},
    get released---it goes without saying that \Julia is gaining in popularity in the
    scientific computing community very rapidly in the past few years.
    As well, \Julia has been chosen as one of
    the languages of choice by the large German DFG grant ``TRR 195:
    Symbolic Tools in Mathematics and their Application'' 2017-2021, and there
    is an overlap between \ODK and the recepients of the latter.

  \end{task}

\begin{task}[id=simulagora-dev,title=Simulagora integration,PM=4,lead=LL,wphases=0-48,issue=53]
  To deliver every six month a new Simulagora VM image containing all the software
  components released over the period. The goal is to prove that the project is
  improving the component architecture by measuring the time it takes to
  integrate them.
\end{task}


  \begin{task}[title=Component architecture for High Performance Computing and Parallelism,id=component-for-HPC,PM=12,wphases=36-48,lead=UJF,partners=SA,issue=54]
    As in all other areas of science, properly supporting massively
    parallel architecture is a major challenge. Many of the
    computational components have already gone a long way in this
    direction, and further work will be carried out in
    WorkPackage~\WPref{hpc}.

    In this task we will investigate and implement
    parallelism-friendly ways of combining components together, so
    that calling components can benefit from the parallelism features
    of called components, with self-adaptation to the environment and
    cooperative sharing of resources. We will use \Sage and its
    components as a test-bed, by producing an HPC-enabled distribution
    (\delivref{component-architecture}{hpc-configure}).
  \end{task}

  \begin{task}[title=Document and modularise \SMC's codebase,id=extract-smc,lead=PS,PM=10,partners={UV,UG},wphases=0-24,issue=55]
    From its inception in 2013, \SMC\ (see Section~\ref{linked-projects}, page~\pageref{sec:SMC-page}) 
    has quickly developed into a full
    featured VRE.  Because of the tight, partly closed source
    development cycles, \SMC's codebase has quickly grown in size,
    with its documentation not always keeping the pace. As a result,
    it is at the moment very hard for a newcomer to set up a clone
    service of \SMC just from its sources.

    Now that \SMC is
    \href{https://twitter.com/sagemath/status/544939872294014977}{fully
      open source}, we need to go through its codebase, understand and
    document it
    (\delivref{component-architecture}{smc-documentation}), isolate
    components that might be reused by other software (e.g.:
    \Jupyter), and make it as portable as possible.

    The ultimate goal of this task is to produce a \emph{personal}
    version of \SMC, to be shipped along with \Sage, that a user can
    run on his own personal computer
    (\delivref{component-architecture}{personal-smc}).
  \end{task}

  \begin{task}[title=Improving the development workflow in mathematical software,id=workflow,lead=UV,PM=10,partners={PS,LL,UG},wphases=6-24,issue=56]
    Truly open software must enable any actor to easily contribute his
    work to the community. Be it an experienced developer, or a
    student. Be it for a major software component or for a piece of
    translation. All the systems involved in \TheProject have
    developed their own workflows for contributing back, but those are
    almost exclusively geared toward experienced developers working on
    large components. When these workflows eventually reach their
    scalability limits, software development stagnates and major
    features are delayed. A well known example is given by \Sage's TRAC
    server, where tickets can stay in ``needs review'' state for a
    long time before entering the code base.  \emph{Upstream} bug
    reporting and fixing is another major factor of slow development.

    This task will seek new ways of accepting contributions to
    mathematical software in a scalable way. For example we will
    experiment with integrating bug reporting and contributing
    features right in the VRE (e.g., in \SMC:
    \delivref{component-architecture}{smc-trac}).).

    % TODO: logilab would like to enhance the existing forges to publish
    % linked open data and ease the sharing of information about
    % package/version/issues across systems. See for example
    % https://packages.qa.debian.org/p/python-defaults.html and the link to RDF
    % meta-data on the right
    % https://packages.qa.debian.org/p/python-defaults.ttl
    % see also https://wiki.debian.org/RDF
    % PM=6 up to 12
  \end{task}


\begin{task}[lead=USO,id=oommf-python-interface,title=Python interface for OOMMF micromagnetic simulation library,PM=6,wphases=7-13,partners={SA},issue=57]
  % 6 person months
  In this task, we create a Python interface
  for the open source Object
  Oriented MicroMagnetic Framework (OOMMF \cite{OOMMF-url}).
  %which is the most widely used micromagnetic simulation package
  %\cite{OOMMF-citations-url}. 
  As a result, the OOMMF library will be fully accessible and usable
  from a Python interface and become a component in the
  Python/\Jupyter eco system of computational tools and in
  \TheProject. We make use of this component architecture in
  \taskref{UI}{oommf-py-ipython-attributes} to build the micromagnetic
  VRE demonstrator
  (Sect.~\ref{sec:introduction-micromagnetic-vre-demonstrator}).

  In more detail, we will first identify the best option for interfacing
  from Python to OOMMF core (C++) routines. The technical options
  include CTypes, Cython, Swig, and Boost-Python, all with particular
  advantages/disadvantages. Following analysis of the current OOMMF
  code layout and compilation model, we will use the most suitable
  tool, bearing in mind our ambition not to modify the OOMMF code so
  that the python interface we create remains functional and
  maintainable with minimal effort while the OOMMF core code is
  developed further by the OOMMF authors. 
  The interface will expose the C++ objects in Python, providing
  an architecture component that provides full access to OOMMF's
  raw capabilities. For clarity, we will refer to this interface as
  \texttt{OOMMF-py-raw}. %Creation of this \texttt{OOMMF-py-raw} is
  %technically doable as OOMMF had been written allowing to do this
  %from Tcl. The \texttt{OOMMF-py-raw} library for Python provides
  %access to the OOMMF functionality but requires some care when being
  %used.

  Secondly, we will create a user friendly Python library
  \texttt{OOMMF-py} that combines the \texttt{OOMMF-py-raw}
  capabilities in an object orientated and safe-to-use Python library
  targeting researchers in the magnetic materials community. We will
  follow the design of the well-received high level Python interface
  in the Nmag micromagnetic simulation package \cite{Fischbacher2007a}
  interface \cite{Nmag-url}. Unit and regression tests for both
  component interfaces \texttt{OOMMF-py-raw} and \texttt{OOMMF-py} are
  simultanously developed.

  %Once this is completed, several new features will be available to
  %OOMMF users: (i) ability to drive OOMMF from Python, (ii)
  %computational steering, and (iii) combination of OOMMF simulation
  %with the existing Python eco-system of computational tools.

  %Can remove the next paragraph if we are pushed for space.

  %We illustrate the advantage of (iii) through an example: to solve
  %the micromagnetic standard problem 3
  %\cite{Micromagnetic-Standardproblem-3}, traditionally multiple OOMMF
  %simulation runs would have to be conducted, and for each of those a
  %new configuration file as to be written. Between these the size of
  %the simulated geometry needs to be modified until two particular
  %values of energy are the same. Given the new interface developed in
  %this work package, this whole process can be replaced by one Python
  %script that creates multiple OOMMF simulations, combined with a root
  %finding method for the automatic iterative determination of the
  %required simulation geometry.

  %For all tasks relating to
  %\OOMMFNB, documentation and tests are created simultaneously with
  %the code. All codes, tests and documentation will be made available as open source.

  %We anticipate to start this task \localtaskref{oommf-python-interface}
  %in month 4, leading to deliverable \delivref{UI}{oommf-py}.
\end{task}





\end{tasklist}

  \begin{wpdelivs}
    % \begin{wpdeliv}[due=6,miles=startup,id=portability-cygwin32,dissem=PU,nature=OTHER,lead=UV]
    %   {One-click install \Sage distribution for Windows with Cygwin 32bits}
    %   % JPF: this should take a few months of work
    %   % This 32bits version would work right away on Windows 64 bits with
    %   % Cygwin 32 bits; more work would be required for a version working on
    %   % a 64 bits of Cygwin.
    %   % JPF: I agree.
    % \end{wpdeliv}%


    \begin{wpdeliv}[due=6,miles=startup,id=virtual-machines,dissem=PU,nature=OTHER,lead=UV,issue=58, status=submitted]
      {Virtual images and containers} Creation and distribution of
      preconfigured cloud oriented virtual machines/containers
      (e.g. Docker images) for \PariGP, \Sage, \SMC, \dots
      % Requires: licenses
      \TOWRITE{??}{Make this deliverable shorter.}
    \end{wpdeliv}
    \begin{wpdeliv}[due=18,miles=startup,id=smc-documentation,dissem=PU,nature=R,lead=PS,issue=61, status=submitted]
      {Understand and document \SMC backend code.}
    \end{wpdeliv}%

    \begin{wpdeliv}[due=12,miles=startup,id=scscp-sage,dissem=PU,nature=OTHER,lead=SA,issue=62, status=submitted]
      {Support for the \href{http://www.symbolic-computing.org/}{SCSCP} interface protocol
        in all relevant components (\Sage, \GAP, etc.) distribution}
    \end{wpdeliv}

    \begin{wpdeliv}[due=24,miles=proto1,id=personal-smc,dissem=PU,nature=OTHER,lead=PS,issue=63]
      {\emph{Personal} \SMC: single user version of \SMC distributed
        with \Sage.}
    \end{wpdeliv}%

    \begin{wpdeliv}[due=24,miles=proto1,id=smc-trac,dissem=PU,nature=OTHER,lead=UV,issue=64]
      {Integration between \SMC and \Sage's TRAC server}
    \end{wpdeliv}
    \begin{wpdeliv}[due=24,miles=proto1,id=sage-repository,dissem=PU,nature=OTHER,lead=UV,issue=65]
      {Open package repository for \Sage} Refactor \Sage package and
      build system to support community-contributed packages
      installable via the web.
    \end{wpdeliv}

    \begin{wpdeliv}[due=24,miles=proto1,id=portability-cygwin,dissem=PU,nature=OTHER,lead=PS,issue=66]
      {One-click install \Sage distribution for Windows with Cygwin 32bits and 64bits}

      % Participants involved: Paris Sud, Kaiserslautern, Saint Andrews, Bordeaux
      % Comments on this by Bill Hart
      % The big problems you will have on Windows 64 on Cygwin include:
      % 
      % * anything with assembly language -- the ABI is different on Windows, so
      % it'll need rewriting, or you can incur a performance penalty by using
      % generic C fallback code
      % * the memory allocator on Windows is not so great
      % * bugs exposed due to being on a different platform, e.g. segfaults due to
      % off-by-one errors that were masked by the granularity of malloc on Linux
      % * build issues, due to identifying Cygwin and using the correct header
      % files, which are often different on Cygwin than linux
      % * issues with PATH vs LD_LIBRARY_PATH
      % * Windows has a case insensitive file system
      % * EOL issues
      % * Windows is not able to rapidly create and delete files, which some
      % libraries (esp. test code) calls for
      % * memory limitations (many people using Windows are using laptops with
      % limited memory, only a portion of which is realistically available to
      % Cygwin)
      % * autotools versions that don't support Windows (usually autotools has a
      % release that is used in all the distributions, which doesn't work correctly
      % on Windows, and this is followed up by a version which has all the Windows
      % patches)
      % * building takes forever on Windows. Mingw2 has now gotten parallel build
      % working on Windows and the speed is within a factor of 5 of Linux. But I'm
      % not sure the improvements have propagated to Cygwin yet.
      % * Cygwin 64 is new, contains quite a few bugs still, and things keep
      % changing with every version as they try to get things right.
      % * Although projects will likely accept patches for Windows, they are less
      % likely to maintain support themselves. I would like to think Singular would
      % be an exception to this. And obviously flint and MPIR work on Windows (even
      % with MSVC as of the next version of flint -- or now if you use our bleeding
      % edge repo version).
      % 
      % Comments by Jean-Pierre on some of the above and mor:
      % * first things first: I already completely built Sage on Cygwin64, though it
      % was surely not completely functional.
      % * assembly: that's right, note that as far as Sage and it's dependencies are
      % concerned, only a few of them actually use assembler, and yes all of them
      % provide fallback generic C code IIRC
      % * PATH vs LD_...: basically the same problem as for Cygwin32, so it's already
      % been taken care of for the Cygwin32 port
      % case issue: not a problem IIRC
      % * EOL issues: I don't thing so, Cygwin is POSIX like
      % * autotools issues: most of Sage dependencies are now updated, I used to track
      % the few problematic ones in 2013
      % * time to build: not so long, sure longer than on a POWER7 machine, but I do
      % it on a usual x86_64 laptop running Debian within a Windows VM in a few hours!
      % what we actually really need is patch/build bots to test on Cygwin 32/64!
      % * upstream cooperation: I agree Windows is often a low priority issue, but
      % most teams have welcomed my Cygwin patches
    \end{wpdeliv}

    \begin{wpdeliv}[due=36,miles=community,id=multiplatform-buildbot,dissem=PP,nature=DEM,lead=UV,issue=67]
      {Continuous integration platform for multi-platform build/test.}

      \Sage's \emph{buildbot} is a x86\_64/Linux specific platform for
      continuous building and testing. We will investigate the
      possibility of evolving it towards a multi-platform tool, and
      opening it to other mathematical software.
    \end{wpdeliv}%
    \begin{wpdeliv}[due=36,miles=community,id=semantic-interface-sage-gap,dissem=PU,nature=OTHER,lead=PS,issue=68]
      {Semantic-aware \Sage interface to \GAP.}
    \end{wpdeliv}


    \begin{wpdeliv}[due=48,miles=eval,id=sage-distribution,dissem=PU,nature=OTHER,lead=UV,issue=59]
      {Packaging for major Linux distributions} Make sure that \Sage and
      all the components it depends on (including \GAP,
      Linbox, \PariGP, Singular, \dots) have standard packages in the
      main Linux distributions.
    \end{wpdeliv}

    % lmonade has very similar objectives but uses the gentoo prefix whereas Linux distributions use very different packaging systems:
    % \begin{compactitem}
    % \item gentoo prefix (gentoo)
    % \item pacman (arch),
    % \item yum (redhat),
    % \item apt (debian),
    % \item easy\_install
    % \item Python index packaging (pip)
    % \end{compactitem}}
    \begin{wpdeliv}[due=48,miles=eval,id=hpc-configure,dissem=PU,nature=OTHER,lead=UJF,issue=60]
      {HPC enabled \Sage distribution}
    \end{wpdeliv}

  \end{wpdelivs}

% \begin{verbatim}
% Raw material:

% Component Architecture
% ----------------------

% Recomputation connection belongs here?

% Collaboration with unreliable (or restricted!) networking connections
% (peer-to-peer, opportunistic syncing, 3rd world). This is technically
% interesting, and gets in support for non-networked working. Not sure
% if it belongs here or not.

% - Security concerns

% Goal: Fostering collaborations/integration between components in an open source ecosystem
% =============================================================================

% - How to make systems "cooperate" rather than "predate each other".
% - E.g. reduce the version issues

% - Foster collaboration with upstream libraries by sharing the
%   development and maintenance of the interfaces, typically as
%   standalone upstream Python bindings (e.g. py-Singular).

% - How to make it easy to develop simultaneously two interdependent
%   components (e.g. Sage+Singular)

% - Foster communication

% - Social aspect:
%   Credit, Citations, Recognition, Funding

% Documentation system
% ====================

% In which package?

% Improvements to Sphinx

% Sage heavily customises the Sphinx documentation system, hacking deep
% in it in some cases, with quite some duplication in some cases.
% Refactor the whole thing, generalizing and contributing back upstream
% as much as possible (e.g. parallel compilation).
% \end{verbatim}

\end{workpackage}

%%% Local Variables: 
%%% mode: latex
%%% TeX-master: "../proposal"
%%% End: 

%  LocalWords:  workpackage wphases wpobjectives wpdescription textbf emph tasklist archs
%  LocalWords:  Linbox delivref portability-cygwin delivref Sparc
%  LocalWords:  non-Linuces sparc muche Itanium OSes VMs Cygwin32 Cygwin64 explorate hpc
%  LocalWords:  Composability Macaulay2 WPref deployment-distrib wpdelivs wpdeliv dissem
%  LocalWords:  Sud segfaults autotools autotools Mingw2 mor multiplatform-buildbot Nmag
%  LocalWords:  buildbot scscp-sage lmonade gentoo pacman redhat debian smc-trac logilab
%  LocalWords:  compactitem worflow saltstack packer.io simulagora-dev Simulagora Jupyter
%  LocalWords:  extract-smc smc-documentation personal-smc TOWRITE oommf-python-interface
%  LocalWords:  micromagnetic oommf-py taskref oommf-py-ipython-attributes CTypes Cython
%  LocalWords:  introduction-micromagnetic-vre-demonstrator texttt OOMMF-py-raw texttt
%  LocalWords:  OOMMF-py-raw simultanously Micromagnetic-Standardproblem-3 OOMMFNB
%  LocalWords:  localtaskref Virtualbox
\newpage
\TOWRITE{ALL}{Proofread WP 4 User Interfaces pass 2}
\begin{draft}
%\begin{verbatim}
%- [ ] do all tasks list all sites involved in them?
%- [ ] does the table of sites and their PM efforts match lists of sites for each task?
%      (each site from the table is listed in all relevant tasks, and no site is listed only in the table or only at some task)
%\end{verbatim}
%fixed: \TODO{D4.14 and D4.15 are not referenced in any task}
\end{draft}

\begin{workpackage}[id=UI,wphases=0-48,
  title=User Interfaces,
  lead=SR,
  PSRM=12,  % Sage-Jupyter interface, sphinx documentation dynamic documentation and exploration system
  UVRM=2,   % Sage-Jupyter interface
  JURM=3,  % Jacobs: active documents
  FAURM=9, % active documents
  USHRM=6, % Supporting reproducible data science and sharing of models
  LLRM=12, % Help on several computer-centered tasks, dynamic SparQL in notebooks
  SARM=18, % GAP
  UKRM=2, % Singular
  UBRM=28,  % Pari
  USORM=16, % Southampton, micromagnetic VRE some contribution (1 month) to 3d visualisation
  SRRM=28,
  UGRM=14,
  USRM=4, % University of Silesia, 3d without subcontracting
  swsites]    % rotate partner logos so that table fits on page.

\begin{wpobjectives}
  The objective of this work package is to provide modern, robust,
  and flexible user interfaces for computation, supporting real-time
  sharing, integration with collaborative problem-solving,
  multilingual documents, paper writing and publication, links to
  databases, etc.
\end{wpobjectives}

\begin{wpdescription}
  Project \Jupyter (formerly \IPython notebook) provides a browser
  based approach to constructing executable documents which comprise
  of code (in multiple languages), mathematics, text, and diagrams (see
  Section~\ref{sec:jupyter}). The
  framework is an ideal portal through which \VREs can be operated. In
  this work package, we will add new functionality to the \Jupyter
  notebook that fosters excellence in computational science and
  research. In particular, we will push towards reproducible and
  effective science by allowing structured documents (such as reports,
  books, theses) from notebooks, and by allowing those notebooks to be
  re-executed as self-contained regression tests. We will unify the
  notebook infrastructure used in \Sage with \Jupyter, push forward
  dynamic documentation exploration capabilities, and work towards
  concurrent multi-user editing of notebooks. We will also develop
  exemplar \Jupyter notebooks for education and research
  (e.g. \taskref{dissem}{ibook}).

  To demonstrate the power of the \TheProject environment to
  accelerate computational science, deliver better value for money and
  make computational science more robust, we will put together a
  micromagnetic
  \VRE(\ref{sec:introduction-micromagnetic-vre-demonstrator}) as a
  demonstrator.

\end{wpdescription}

\begin{tasklist}
\begin{task}[title=Uniform notebook interface for all interactive
  components,id=ipython-kernels,lead=PS, partners={SR,UK,USH,USO,LL,SA,UV,UG},
  PM=24, wphases=0-36,issue=69]
  In this task, we will implement \Jupyter interfaces for the
  interactive computation components of \TheProject, including \GAP,
  \PariGP, \Sage, and Singular. A first release
  \localdelivref{ipython-kernels-basic} will focus on basic functionality,
  and a second release \localdelivref{ipython-kernels} will cover advanced
  features like 3D graphics or transparent documentation browsing (as
  live worksheets whenever relevant).

  % Note from William: my student Andrew Ohana just mostly did
  % something like that for IPython, but then stopped.  Anyway, it's
  % very do-able based on a summer project from another student and a
  % bunch of work I did with THREE.js for SMC.

  One of our objectives is to ensure the sustainability of the project
  (Objective~\ref{objective:sustainable}). The current \Sage notebook
  interface was developed alongside that of \Jupyter, but with
  slightly different goals. A notebook interface for \Sage is a vital
  integrative component, and development was fast tracked to ensure
  its availability to allow the project to move forward. However,
  \Jupyter, whilst it initially proceeded more slowly, has a larger
  developer base and has now caught up with the \Sage notebook in
  terms of functionality. In line with
  Objective~\ref{objective:sustainable} \Sage will now phase out its
  own notebook and switch focus to the \Jupyter notebook, outsourcing
  this key but non disciplinary component.

  % In charge: Jupyter dev + dev in Orsay + community?
  The \Sage and \Jupyter convergence \localdelivref{ipython-kernel-sage} will
  require:
  \begin{compactitem}
  \item Robust migration path and tools for \Sage worksheets,
  \item Support for math, 2D, and interactive 3D scene visualisation,
    % \item Bundling of the \Jupyter notebook and its dependencies within
    %   the Sage distribution. DONE
  \item Import and export of ReST documents, with full support for
    \Sage's specific roles (math, ...),
  \item Support for remote \Sage kernel, typically on the cloud, or
    running with a different Python version (\Sage as a library),
  \item A migration path for interactive widgets implemented with
    \Sage's \texttt{@interact} functionality.
  \end{compactitem}

  Joint meetings and visits between the developers of \Jupyter and of
  the computing components will be a key component of this task.

\end{task}

\begin{task}[id=notebook-collab,title=Notebook improvements for collaboration,lead=SR, partners={PS,USH,JU,FAU,USO,LL}, PM=20, wphases=0-24, issue=70]
  In this task, we will further improve tools for collaboration
  between authors of shared \Jupyter notebooks and draw from the
  experience of collaboration as set in Simulagora, SageMathCloud,
  etc.

  Version control tools, such as Git and Mercurial, have become an
  integral part of open and collaborative science and
  software. Version control tools allow proposed changes to be
  reviewed (`diffing') and resolve conflicts through combination of
  changes (`merging'). \Jupyter notebook documents are stored in text
  files as JSON formatted data. This makes them well suited to
  tracking in version control, but the JSON structure can make diffing
  and merging difficult. We will deploy tools to provide better
  support for visual diffing and merging of Notebook documents. These
  tools will be integrated into existing version control workflows
  \localdelivref{jupyter-collab}. The MathHub.info system already has
  a distributed Git-based versioning system, which can serve as an
  entry point here.

  Given the interactive nature of \Jupyter notebooks, live
  collaboration, where multiple authors work on the document
  simultaneously (like in Google Docs), is particularly
  desirable. However, there are particular challenges for
  collaborative editing of \emph{executable} documents. The potential
  for \emph{shared execution} adds both value and challenge to the
  live collaboration. Some attempts have been made to deal with live
  collaborative sessions (e.g. \SMC, Colaboratory) but so far these
  have been outside the core \Jupyter project. In this task we will
  explore different models of single-notebook collaboration, including
  shared or separate execution \localdelivref{jupyter-collab}. We will
  consider not only indicating authorship, but which author
  triggered which execution, and explore other challenges.  Various
  avenues for live session collaboration will be explored for
  integration into \Jupyter itself
  \localdelivref{jupyter-live-collab}.
\end{task}

\begin{task}[id=notebook-verification,title=Reproducible Notebooks,lead=SR, partners={PS,USO}, PM=4, wphases=12-24, issue=71]
  In this task, we will develop tools that allow re-execution
  notebook documents with automated regression testing. The computed
  output will be compared against the stored output, and deviations
  reported as assertion errors.

  Notebooks are used in a variety of contexts, like training and
  teaching material (tutorials, documentation, books) or computer
  experimentation logbooks, where reproducibility is
  critical. Reproducibility dictates that the notebooks should remain
  functional and correct in the long run, even when the underlying
  computational software or infrastructure changes over time or across
  platforms.

  This task is a critical component of reproducibility, allowing the
  notebook author to get an immediate notice when, e.g., a backward
  incompatible change occurs. It becomes even possible to anticipate
  such situations upstream by including important notebooks directly
  in the automated test suite of the computational software, giving an
  easy way for casual users to contribute regression tests.

  Technically speaking, \Jupyter notebooks store outputs as rich
  mime-type structures, with JSON metadata. Using this metadata, it
  will be possible to express expectations of output, allowing more
  flexible and powerful tests than direct text comparison
  \localdelivref{jupyter-test}.  Prior work has been done in \Sage for
  ReST files, e.g. \lstinline{sage -t notebook.rst}, and this model
  will be extended to notebooks.
\end{task}

\begin{task}[id=sage-sphinx, title=Refactor \Sage's \Sphinx documentation system, lead=PS,PM=6, partners={SR,UV,UG}, wphases=0-36, issue=72]
  \Sage, like \Python and many other \Python based projects, uses the
  \Sphinx documentation system. Due to particularly stringent needs,
  many layers of customisation and adaptations have accumulated over
  the years, in particular for proper scaling to the sheer size of the
  Sage documentation (13k pages just for the reference manual).

  A deep refactorisation (\localdelivref{sage-sphinx}) is critically
  needed to get rid of multiple duplication, and foster sustainability
  by outsourcing back to \Sphinx all generic aspects (parallel
  compilation, index generation, ...).
  \TOWRITE{VP}{Viviane, this seems a little short, can we provide a little more detail of what the refactorisation will involve?}
  % In charge: dev in Orsay or Logilab + visit of Sphinx dev  + FH
\end{task}

\begin{task}[id=dynamic-inspect,title=Dynamic documentation and exploration system,lead=PS, partners={SR,USO,UV,LL,UG}, PM=6, wphases=0-12, issue=73]
  Introspection has become a critical tool in interactive computation,
  allowing user to explore, on the fly, the properties and
  capabilities of the objects under manipulation. This challenge
  becomes particularly acute in systems like \Sage where large parts
  of the class hierarchy is built dynamically, and static
  documentation builders like \Sphinx cannot anymore render all the
  available information.

  In this task, we will investigate how to further enhance the user
  experience. This will include:
  \begin{compactitem}
  \item On the fly generation of Javadoc style documentation, through
    introspection, allowing e.g. the exploration of the class
    hierarchy, available methods, etc.
  \item Widgets based on the HTML5 and web component standards to display
    graphical views of the results of SPARQL queries, as well as populating data
    structures with the results of such queries,
  \item \localdelivref{ipython-advanced-interacts} (Month 36)
    Exploratory support for semantic-aware interactive widgets
    providing views on objects of the underlying computational or
    database components. Preliminary steps are demonstrated in the
    \texttt{Larch Environment} project (see demo video on
    \url{http://www.larchenvironment.com/}) and
    \software{sage-explorer}
    (\url{https://github.com/jbandlow/sage-explorer}). The ultimate
    aim would be to automatically generate \LMFDB-style interfaces.
  \end{compactitem}
  Whenever possible, those features will be implemented generically
  for any computation kernel by extending the \Jupyter protocol with
  introspection and documentation queries.
  % In charge: \Jupyter dev + dev in Orsay + NT?
\end{task}

\begin{task}[title=Structured documents,id=structdocs,
  lead=JU,PM=22,partners={SR,USH,LL,FAU},wphases=0-24,issue=74]
  \Jupyter notebooks consist of a sequence of cells that contain
  either text or a program (see Section~\ref{sec:jupyter}). Complex
  documents, such as books, articles or reports, require a richer
  description that covers the the structure of the document and the
  semantics of its elements. This task will investigate this problem
  and try to find a way to write these documents exploiting the
  breakthroughs achieved in the other tasks to this workpackage.

  Several technical complementary options can be explored:
  \begin{compactitem}
  \item MathHub.info is a portal for reading and interacting with
    ``active documents'' (i.e. documents that have an additional
    semantic layer that supports semantic services like definition
    lookup, type-inference, unit conversion,\ldots)
  \item \Jupyter notebooks are essentially ``programs with documentation'' and lack the
    semantical structure needed by complex documents.
  \item sTeX is a semantic variant of LaTeX that can be transformed into OMDoc/MMT, which
    is the native knowledge representation format for active documents and
    machine-actionable knowledge about math and symbolic programs.
  \end{compactitem}

  After gathering the needs and the requirements for the writing of
  complex documents in the mathematical field, we will study these
  designs and build a solution that meets the expectations
  (\localdelivref{adstex}). The implementation will be achieved
  through an iterative process that incrementally improves existing
  software solutions, making them interoperable and synergistic.
  Results of this convergence will be reported
  in~\localdelivref{adcomp}, \localdelivref{ipython-kernel-sage} and
  \localdelivref{jupyter-import} and used in \taskref{dissem}{ibook}.
\end{task}

\begin{task}[id=mathhub,title=Active Documents Portal,lead=FAU,PM=12,partners={JU},
  wphases=12-36!.5,issue=75]
  We will extend the existing \url{http://mathhub.info} system to a
  portal for interacting with active/structured documents (see
  \localtaskref{structdocs}) and releasing the portal initially for
  internal use in the \TheProject and later for general
  use. \url{MathHub.info} already provides very basic sTeX editing and
  versioning. In \TheProject we will extend it on the computational
  side based on the integrated format from
  \localtaskref{structdocs}. The resulting portal will be made
  available to the consortium as~\localdelivref{mathhub-editing} and
  would be used for semantically enhanced code documentation and
  knowledge representation (see \WPref{dksbases}).
\end{task}

\begin{task}[title=Visualisation system for 3D data in web-notebook
,id=vis3d,lead=SR, partners={US,PS,USO}, PM=13, wphases=0-24, issue=76]
\TOWRITE{MRK,HPL}{wphases does not agree with PM. (13 vs 24}
%12 months from Simular,
% 1 month from Southampton for testing in the micromagnetic VRE demonstrator

The \Jupyter notebook provides an attractive environment for building
user interfaces for research. However, the current support for inline
visualisation is limited to curve plots and 2D scalar fields. Many
scientific simulations need visualisation of 3D scalar and vector
fields, as shown in Figure~\ref{fig:3d-plots}.  Experimentations in
low dimensional topology and differential geometry also relies on good
drawing capabilities
(e.g. \href{http://www.math.uic.edu/t3m/SnapPy/}{SnapPy} or
\href{http://sagemanifolds.obspm.fr/}{SageManifolds} based on \IPython
and \Sage). The amount of data can be tremendous, especially in
time-dependent problems computed in a distributed fashion over
large-scale computational clusters. Interactive inspection of such
simulations can be a valuable tool which accelerates
research. However, for inspection, one does not need to transfer and
gather the full dataset at each time step---getting selected computed
fields on user request or preprocessing certain quantities like cross
sections with some predefined frequency will mostly suffice.

In this task we will first investigate available technologies for fast
in-browser visualisation of the typical structures to be displayed
(isosurfaces, streamlines, vector fields, cross sections, etc.).
There are several existing solutions which could provide basis for
further development. One of the best known and most advanced is
\href{http://threejs.org/}{three.js} which provides a basis for 3D
visualisation in a web browser. Three.js is WebGL based, but also
provides canvas based rendering for system which do not support
WebGL. It has already been experimentally deployed in Sage Cell Server
and SMC projects. Other promising technologies include visualisation
libraries using exclusively OpenGL. They can be deployed in browser
based systems by using of the WebGL API (which is a restricted subset
of the regular OpenGL API). This can be accomplished by visualisation
executed purely on the GPU. The \href{http://vispy.org/}{VisPy} and
\href{http://glumpy.github.io/}{glumpy} projects have found GPU-only
solutions for common visualisation objects (lines, arrows, markers,
text, iso-lines, iso-surfaces, text, etc) where data does not exit the
GPU. The VisPy project already offers an experimental interface with
the \Jupyter notebook that could be extended to cope with our
specifications. Through this tight collaboration with the authors,
\TheProject could benefit from both dedicated and state-of-the art
visualisation techniques.

The \href{http://www.math.uic.edu/t3m/SnapPy/}{SnapPy} and
\href{http://sagemanifolds.obspm.fr/}{SageManifolds} projects will be
considered for deployment of tools we develop (see associated
deliverable \localdelivref{vis3d}).
\end{task}


\begin{task}[title=Visualisation of 3D fluid dynamics data in web-notebook
,id=cfd-vis,lead=SR, partners={US,PS,USO},PM=5,wphases=12-36,issue=77]

We propose to let computational fluid dynamics (CFD) be a driving
application for the development of 3D visualisation in \Jupyter
notebooks (\taskref{UI}{vis3d}) since CFD is one of the most demanding
cases of scientific visualisation. The same time this task
(with deliverable \localdelivref{cfd-vis}) will be
a demonstrator for (\taskref{UI}{vis3d}).

Successfully handling CFD makes the tool immediately applicable to a
range of other fields such as heat transfer, electromagnetics,
material science, and 3D algebraic structures in
mathematics. Simulations would be initialised inside the notebook and
executed on HPC clusters. This approach will significantly lower the
threshold for using parallel computing codes that can be hard to
install correctly on local workstations (see also \WPref{hpc}). Such
use cases with 3D visualisation will greatly extend the potential
applications of the \Jupyter notebook concept throughout science and
engineering.

As an example code for a 3D live web notebook with fluid dynamics
simulations, we will use the Lattice Boltzmann solver which is under
development at the University of Silesia:
\href{http://sailfish.us.edu.pl/}{Sailfish}.  This code is an advanced
Lattice Boltzmann solver designed from the ground up for distributed
systems of GPU compute clusters. It is implemented predominantly in
Python, and it uses run-time code generation techniques to
automatically build optimised code for CUDA and OpenCL devices. Since
running Sailfish requires specialised hardware, it is reasonable to
use it on dedicated HPC installations.
\end{task}

\begin{task}[lead=UB,title=Common option system for various displays
  in Sage,id=Sage-display,PM=12,wphases=0-24,issue=78]
  \TOWRITE{CNRS}{There are no deliverables associated with this task
    that is listed at 12 person months. Perhaps some explanation of
    the challenges of the task would also help.}

  Given a mathematical object, it often has various possible
  representations on a computer. From raw text to \LaTeX, from simple
  2d picture to a complicated 3d animation.

  In this task, we will provide a uniform option system for displaying
  an object within \Sage (raw text, \LaTeX, tikz, matplotlib, jmol,
  tachyon, \ldots). We will implement some of the missing display and
  will benefit of the work done in \taskref{UI}{cfd-vis}.
\end{task}

\begin{task}[lead=USO,title=Case study: micromagnetic VRE built from
  \TheProject,id=oommf-py-ipython-attributes,PM=6,partners={SR,USH},wphases=13-19,issue=79]
  % 6 person months
  In this task, we use the \TheProject architecture to assemble a
  virtual research environment software tailored for the large
  micromagnetic research community
  (see Section \ref{sec:introduction-micromagnetic-vre-demonstrator}).

  The micromagnetic VRE will be based on the \Jupyter notebook, the
  Python interface to the micromagnetic simulation library OOMMF
  (\taskref{component-architecture}{oommf-python-interface}),
  and the additional features added to \Jupyter in this work
  package.

  The \Jupyter notebook environment allows to host, execute and
  document the Python-based OOMMF simulation in an executable
  document. In this interactive environment, objects can be displayed
  using various representations, including, for example, textual
  representation (i.e. strings), bitmap images and SVG (vector
  graphics) files. We will create functionality so that magnetisation
  vector field objects can be presented as a rendered 3d and 2d-view
  of the magnetisation field (Figure~\ref{fig:3d-plots}), and similar
  features for scalar fields such as field components and energies for
  static and time dependent data (linking to
  \localtaskref{cfd-vis}). This allows computational steering and the
  interactive exploration of the behavior of magnetic nanostructures.

  Beyond that, the \Jupyter Widgets allow the creation of graphical
  user interface (GUI) elements, and we will generate code to display
  these widgets on demand to (i) set up micromagnetic simulations
  using a GUI, and (ii) assist in common post-processing simulation
  results. Recent pilot work has shown that it is possible to make
  \Jupyter Widgets interact with the Python interpreter session and
  this allows to activate a GUI-like (widget based) interface when
  desired but to quickly return to the interpreter prompt, taking
  forward the results (data) from the GUI session
  \cite{IPython-widget-GUI-demo-youtube-2014} and providing a
  continuous path from scripting to GUI. Having the ability to mix and
  match GUI-based and command driven analysis combines the best of
  both approaches, caters for users' preferences, and provides
  significant additional value.
\end{task}

\begin{task}[lead=UB,title=Python/Cython bindings for \Pari,PM=16,id=pari-python,partners={UB,UG},wphases=0-48,issue=80]
  \Pari is a C-library and GP is its standalone interpreter. Partial
  Python/Cython bindings are provided by Sage. There is also a not
  more developed library \software{cypari}.

  The task aims to develop an independent Python/Cython library that
  would provide bindings for \PariGP and which would tightly be
  developed within the \PariGP team.

  Firstly, starting from \Sage and \software{cypari}, we will provide a standalone \Pari bindings in Python
  and integrate it in \Sage (\localdelivref{pari-python-lib1}). Secondly, different optimisation
  will be provided for a tighter communication between \Pari and \Python.
  \begin{compactitem}
  \item Use the declaration files of \software{GP} to automatise \Cython declarations.
  \item Replace copy from the \Pari stack by direct pointers.
  \item More tests and documentation.
  \item Integrate the parallelisation features from \taskref{hpc}{hpc-pari} within \Python.
  \item Implement a crossed bug report system between \Sage and \Pari (using
  results of \taskref{social-aspects}{isocial-decisionmaking}).
  \end{compactitem}
  The deliverable for this second step is \localdelivref{pari-python-lib2}. For this task we might
  require expertise of some \Sage, \Pari or \Cython developers (Jeroen Demeyer, Peter Bruin).
\end{task}

\begin{task}[lead=USO,title=Demonstrator: micromagnetic VRE notebooks,
  id=oommf-tutorial-and-documentation,PM=6,partners={SR,PS},wphases=19-25,issue=81]
  % 5 person months + 1 month co-investigator [Ian Hawke's experience]
  The purpose of the micromagnetic \VRE
  (\localtaskref{oommf-py-ipython-attributes}) is to enable excellent
  computational research. To maximise the value of this grant's
  investment for the community, we will not carry out micromagnetic
  research but instead produce a set of executable notebooks using the
  micromagnetic \VRE to demonstrate its power and applicability.

  We will create executable notebook documents
  within the micromagnetic \VRE
  including (i) a new tutorial on computational micromagnetics with
  OOMMF, (ii) the complete documentation of the \texttt{OOMMF-Py}
  library (\taskref{component-architecture}{oommf-python-interface}),
  and (iii) a set of typical micromagnetic case studies. The tutorial,
  in terms of content, will take guidance from the tutorial provided
  for Nmag \cite{Nmag-tutorial-url} and will introduce the additional
  features of the \Jupyter-driven micromagnetic \VRE. We expect this
  substantial and executable documentation of the micromagnetic \VRE to
  become the standard resource that introduces researchers to
  computational micromagnetics, in particular through the online
  portal (\localtaskref{oommf-nb-ve}).

  %% This block is about the benefits of using the notebook. It should
  %% go somewhere else in more generic form:
  %The output of this activity will deliver multiple benefits:
  %providing a systematic introduction to \texttt{OOMMF-py} suitable for both
  %those users (i) new to micromagnetic modelling and those (ii) new to
  %the \texttt{OOMMF-py} interface. Because the documentation is developed in an
  %\Jupyter notebook, the documents are executable. For new learners
  %this is a great simplification because they can skip through the
  %given document and execute the given examples there and then: at the
  %moment, this is a process of manually writing a script, or locating
  %it in the directory structure of files, then executing this,
  %subsequently opening and processing the data files, etc. In the new
  %model, this end-to-end simulation will start from specifying the
  %material parameters in the notebook (all of this is given in the
  %tutorial), to running the simulation in the notebook to processing
  %of computed data while the simulation runs (or subsequently) in the
  %notebook; thus providing one virtual research environment, with all
  %the associated benefits of making best use of the scientist's time
  %using the tool and environment.
\end{task}

\begin{task}[lead=USO,id=oommf-nb-ve,title=Online portal for
  micromagnetic VRE demonstrator,PM=3,partners={SR,FAU},wphases=25-28,issue=82]
  % 3 person months
  Recently, a TeMPorary \Jupyter NoteBook (TMPNB) has been made
  available (at \href{http://tmpnb.org}{http://tmpnb.org}) that allows
  anybody to open this URL and use their very own \Jupyter notebook
  for quick calculations and tests online. We will provide such a
  portal which provides the
  micromagnetic \VRE for anonymous use. This service allows users to
  execute the demonstrator tutorial and documentation notebooks
  (\localtaskref{oommf-tutorial-and-documentation}) and run the
  calculations in real time on the web server, without having to
  install any software on their own machine.  This web service will be
  based on Docker \cite{Docker} virtualisation technology and we will
  make available the scripts to create VirtualBox \cite{Virtualbox}
  images, and Docker containers. The same virtual machine images can
  also be used for Cloud hosted computing services.

  %{HF}{Do we need the resource request here? Or should it
  %just be in resources.tex: Either works, in the resources file there
  %is only the total sum mentioned and a link to here. So no
  %duplication of information, and the particular machine is maybe
  %better explained here. I'll comment this out to 'resolve' it.}

  We request \euro{6000} to purchase a machine to provide these
  services (shared memory, 64 cores, 128GB RAM, Solid-state drive (SDD)
  to make the system more responsive).
  %This machine will also support
  %the regression testing and continuous integration (see task
  %\taskref{dissem}{dissemination-of-oommf-nb-virtual-environment}).
  %Setup and
  %maintenance of the machine is part of this work task.
\end{task}

\end{tasklist}

\begin{wpdelivs}
  \begin{wpdeliv}[due=6,id=pari-python-lib1,dissem=PU,nature=OTHER,lead=UB,issue=83]
	  {Python/Cython bindings for \Pari and its integration in Sage}
  \end{wpdeliv}
  \begin{wpdeliv}[id=adstex,due=9,miles=startup,nature=R,dissem=PU,lead=JU,issue=91]
    {Active/Structured Documents Requirements and existing Solutions} Presenting sTeX and
    \Jupyter to the consortium, comparing and evaluating as stepping stones.
  \end{wpdeliv}
    \begin{wpdeliv}[id=mathhub-editing,due=18,miles=startup,nature=DEM,dissem=PU,lead=FAU,issue=92]
      {Distributed, Collaborative, Versioned Editing of Active Documents in MathHub.info}
    \end{wpdeliv}
  \begin{wpdeliv}[due=12,miles=proto1,id=ipython-kernels-basic,dissem=PU,nature=OTHER,lead=PS,issue=93]
      {Basic \Jupyter interface for GAP, \PariGP, \Sage, Singular}
  \end{wpdeliv}
  \begin{wpdeliv}[due=12,id=ipython-kernel-sage,miles=startup,dissem=PU,nature=DEM,lead=PS,issue=94]
      {\Sage notebook / \Jupyter notebook convergence}
  \end{wpdeliv}
  \begin{wpdeliv}[due=12,id=jupyter-collab,miles=startup,dissem=PU,nature=OTHER,lead=SR,issue=95]
      {Tools for collaborating on notebooks via version-control}
  \end{wpdeliv}
  \begin{wpdeliv}[due=14,id=ipython-kernels,miles=startup,dissem=PU,nature=OTHER,lead=PS,issue=96]
      {Full featured \Jupyter interface for GAP, \PariGP, Singular}
  \end{wpdeliv}
  \begin{wpdeliv}[id=adcomp,due=18,miles=proto1,nature=DEM,dissem=PU,lead=JU,issue=97]
    {In-place computation in active documents (context/computation)}
  \end{wpdeliv}
  \begin{wpdeliv}[due=18,miles=proto1,id=jupyter-test,dissem=PU,nature=OTHER,lead=SR,issue=98]
      {Facilities for running notebooks as verification tests}
  \end{wpdeliv}
  \begin{wpdeliv}[due=36,miles=proto1,id=pari-python-lib2,dissem=PU,nature=OTHER,lead=UB,issue=84]
	  {Second version of the \Pari Python/Cython bindings}
  \end{wpdeliv}
    \begin{wpdeliv}[id=jupyter-import,due=24,miles=proto1,nature=DEM,dissem=PU,lead=FAU,issue=85]
      {Notebook Import into MathHub.info (interactive display)}
    \end{wpdeliv}

  \begin{wpdeliv}[due=24,id=vis3d,miles=proto1,dissem=PU,nature=OTHER,lead=SR,issue=86]
      {\Jupyter extension for 3D visualisation}
  \end{wpdeliv}
  \begin{wpdeliv}[due=24,miles=proto1,id=sage-sphinx,dissem=PU,nature=OTHER,lead=UG,issue=87]
      {Refactorisation of \Sage's \Sphinx documentation system}
  \end{wpdeliv}
  \begin{wpdeliv}[due=36,miles=community,id=cfd-vis,dissem=PU,nature=OTHER,lead=SR,issue=88]
      {Computational Fluid dynamics visualisation in web notebook}
  \end{wpdeliv}
  \begin{wpdeliv}[due=36,miles=community,id=jupyter-live-collab,dissem=PU,nature=OTHER,lead=SR,issue=89]
      {Exploratory support for live notebook collaboration}
  \end{wpdeliv}
  \begin{wpdeliv}[due=36,id=ipython-advanced-interacts,dissem=PU,nature=DEM,lead=PS,issue=90]
      {Exploratory support for semantic-aware interactive widgets providing views on objects
      represented and or in databases}
  \end{wpdeliv}
% communication with live computing process
% post simulation data analysis module
% visualization of vector and scalar fields
% editor for geometry and boundary conditions  on regular meshes


  % Shared \Jupyter sessions embedded in voice-over-IP or
  % teleconference calls or reciprocally.
  %
  % NOTE: This task is probably outdated by appear.in which makes
  % video-conferencing in the browser trivial
  %
  % \delivref{ipython-collaborative}
  % Eugen Dedu:
  % I think such a module can be thought of as a screen-capturing
  % module, i.e. allow Ekiga to capture the screen of a Sage user (this
  % is currently not possible).  This is not a difficult task to do.
  % Julien Puydt: ekiga can do that since something like 2008 with my
  % experimental gstreamer plugin, and I shall be able to present
  % interesting sample code to the ekiga-devel mailing-list in something
  % like two-three weeks (after I'm done with my students), which will
  % hopefully be part of the next version.
  %
  % But as Nicolas noted in his answer, some kind of interactive session
  % where people can share a sage session would be better.
  %
  % I think the feature decomposes in the following pieces:
  % - IPython should have a way to share sessions between several
  % participants using an open and standard protocol ;
  % - ekiga should implement it.
  %
  % In my opinion ekiga, because of its dependency on ptlib and opal
  % libraries and the use of complex protocols like SIP and H323, needs
  % highly technical people.  Students cannot help much, but engineers
  % are appropriate.
  \end{wpdelivs}
\end{workpackage}

%%% Local Variables:
%%% mode: latex
%%% TeX-master: "../proposal.tex"
%%% End:

%  LocalWords:  workpackage wphases Jupyter OOMMFNB wpobjectives wpdescription TOWRITE
%  LocalWords:  Paderborn IPython KBase Hackathon Quantopian Logilab Enthought Authorea
%  LocalWords:  emph Jupyther nanostructures tasklist delivref THREE.js texttt diffing
%  LocalWords:  notebook-collab jupyter-collab Colaboratory jupyter-live-collab Javadoc
%  LocalWords:  notebook.rst Knowls structdocs localtaskref Needs.rst CTypes Cython Nmag
%  LocalWords:  oommf-python-interface OOMMF-py-raw micromagnetic oommf-py magnetisation
%  LocalWords:  Micromagnetic-Standardproblem-3 oommf-py-ipython-attributes vispy taskref
%  LocalWords:  oommf-nb IPython-widget-GUI-demo-youtube-2014 Dedu
%  LocalWords:  oommf-tutorial-and-documentation modelling micromagnetics oommf-nb-ve mws
%  LocalWords:  TeMPorary oommf-nb-virtual Virtualbox Cloudhosted dissem
%  LocalWords:  dissemination-of-oommf-nb-virtual-environment wpdelivs wpdeliv Eugen tikz
%  LocalWords:  Ekiga Puydt gstreamer ekiga-devel ptlib compactitem refactorization numpy
%  LocalWords:  cfd-vis paraview ldots electromagnetics isosurfaces notebooksearch adstex
%  LocalWords:  cassearch simulagora Simulagora mathhub-editing adcomp nbad-search glumpy
%  LocalWords:  swsites visualisation introduction-micromagnetic-vre-demonstrator mathhub
%  LocalWords:  matplotlib jmol maximise virtualisation localdelivref refactorisation
%  LocalWords:  WPref dksbases Simular initialised
\newpage
\TOWRITE{ALL}{Proofread WP 5 High Performance Computing pass 2}



\begin{workpackage}[id=hpc,wphases=0-48,
  short=High Performance Math. Computing,% for Figure 5.
  title=High Performance Mathematical Computing,
  lead=UJF,
  USHRM=16, % Jupyter-SGE integration
  PSRM=6,   % HPC for Combinatorics 
  LLRM=12,  % Pythran
  SARM=18, % GAP
  UKRM=60, % Singular
  UBRM=36,  % Pari + HPC for Combinatorics + Sage integration
  UJFRM=52] % Linbox / Pythran
  

\begin{wpobjectives}
  The objective of this work package is to improve the performance of
  the computational components of \TheProject, in particular on
  massively parallel architectures. This includes notably:
  \begin{compactitem}
  \item Fine grained High Performance Computing on many-cores architectures.
  \item Coarse grained or embarrassingly parallel computing on grids or on the cloud.
  \item Compilation of high level interpreted code to optimised parallel native code.
  \item Develop novel HPC infrastructure in the context of combinatorics.
  \end{compactitem}
  A key aspect will be to foster further sharing expertise and best
  practices between computational components.
\end{wpobjectives}

\begin{wpdescription}
  As in all other areas of science, properly supporting massively
  parallel architectures is a major challenge. Many of the
  computational components in \TheProject have already gone a long way
  in this direction. For example, parallel versions
  of the \GAP kernel for
  a range of architectures were developed during the 2009-2013 EPSRC
  ``HPCGAP'' project. The expertise
  gained there was then transferred to the ongoing \Singular-HPC
  project, in particular through the rehiring of one of the developers
  of HPC-GAP. The French ANR HPAC project (2012-2015) has also widely contributed to design
  parallel exact linear algebra kernels  which is a core component for most HPC
  applications. The \Linbox library, used in sage, has benefited from this
  experience on the multi-core processing aspects. 

  In this work package, we will build on this momentum to further implement HPC support in
  the component Tasks~\localtaskref{hpc-pari} (\Pari),\localtaskref{hpc-gap} (\GAP),
  \localtaskref{hpc-linbox} (\Linbox), \localtaskref{hpc-mpir}(\MPIR) and
  \localtaskref{hpc-singular} (\Singular).
  
  Many of the computational components of \TheProject, notably \Sage
  and \GAP use a high level
  interpreted language for their library. Performance is achieved by
  rewriting or compiling critical sections into a lower-level
  language. \Sage uses
  the \Cython \Python-to-C compiler; \GAP has some more basic support.
  In Tasks~\localtaskref{hpc-gap} and \localtaskref{pythran}, we will also boost performance by
  further developing and applying such compilation tools, allowing the
  application programmer to retain their high level approach.

\end{wpdescription}
\begin{tasklist}
\begin{task}[title=PARI,id=hpc-pari,PM=20, lead=UB,wphases=0-48!0.5,issue=99]
  \Pari is a C library mainly oriented toward arithmetic and number theory.
  
  It currently supports both POSIX threads or MPI but lacks interfaces for
  parallelism. More precisely, it should be easier from an external package
  or software (e.g. Sage) to better exploit \Pari parallel features.

  On the other hand, most basic algorithms in the PARI library (e.g. integer
  factorisation) are currently implemented using only one core. To
  make better use of multi-core architecture and more generally parallel
  architectures, we will devise a generic parallelisation machinery
  to allow individual implementations to scale gracefully between single
  core / multicore / massively parallel machines while avoiding code
  duplication.

  The deliverables for this task are \localdelivref{pari-hpc1}
  and \localdelivref{pari-hpc2}.
\end{task}


\begin{task}[title=GAP,id=hpc-gap,PM=18, lead=SA, wphases={0-48!0.375}, issue=100]
  Thanks to the HPCGAP project, almost the full functionality of \GAP
  can be safely run on multicore architectures, and there is support for
  simple but effective parallel programming, with protection from most
  of the more serious pitfalls that can trouble the novice parallel
  programmer. Experimental versions of \GAP also exist for a number of
  distributed-memory architectures.


In this task we will continue to develop the \GAP infrastructure to
offer performance improvements to real end users on a wide range of
modern hardware. This will be achieved by a number of synergistic
developments in the system

\begin{compactitem}
\item a library of parallel algorithms for algebraic computation. This
  will include general purpose skeletons applicable to many problems;
  distributed data structures suitable for orchestrating distributed
  memory computations; and implementations of specific parallel algorithms for key
  mathematical tasks.  Target skeletons include irregular parallel
  maps and folds; transitive closure operations, especially orbits of
  group actions and chain reduction (as used in Gaussian
  elimination). Data structure targets include distributed task
  queue, hash table and array structures.Specific algorithm targets include linear algebra over
  finite fields; randomised search algorithms in general and matrix group
  recognition in particular and algorithms for analysing the structure
  of groups given by a finite presentation
\item interfaces between \GAP and standard cloud and HPC
  infrastructure. This work will be based on \taskref{component-architecture}{component-for-HPC}.
  At the moment \GAP is designed for interactive use or use as a local
  (lacking resource control or security) SCSCP server. Data is taken
  from local files or fixed URLs. We will develop
  interfaces that make it easy to run \GAP jobs through standard batch
  queue environments; enable SCSCP servers to take advantage of
  widely available authentication and resource control frameworks in
  cloud and HPC settings, and access resources through standard
  discovery and allocation mechanisms.

\item adaptation of the cython and/or pythran technology to allow the
  performance of critical \GAP language subroutines to be increased
  close to that of C code without the programmer cost of a full C implementation.
\end{compactitem}
The deliverables for this task is  \delivref{hpc}{GAP-HPC-report}, a
report on all the \GAP-related activities of this workpackage,
including links to the software which will by then have been
publically released.

\end{task}


\begin{task}[title=Linbox,PM=40,id=hpc-linbox, lead=UJF,wphases={3-48!0.37,12-36},issue=101]
Most intensive mathematical computations rely heavily on exact linear kernels
and their ability to harness parallel computers, grids or clusters. The \Linbox
library, already delivers high sequential efficiency for mathematical software
such as \Sage. The parallelisation of the library for multi-core architectures
has been initiated in the A.N.R. HPAC project and successfully set the building
blocks for high performance algebraic computing. 
The task here is to  address the remaining challenges for the use of such
kernels through a general audience mathematical software, such as \Sage.
A first aspect focuses on code design and domain specific languages allowing to
expose an abstraction of the parallel infrastructure and the parallel features
of the code through the stack of libraries, and support the
composition of parallel routines. This will be addressed in
deliverable~\localdelivref{LinBox-DSL}. More generally the second aspect, addressed
in deliverable~\localdelivref{LinBox-algo} concerns the 
development of new parallel algorithms and implementations, that are still
barriers in the development of High performance mathematical
applications. Lastly, the third part, in
deliverable~\localdelivref{LinBox-distributed}, addresses the specificities of  distributed
computing, with a close focus on communications and heterogeneous infrastructures.

  % \begin{compactitem}
  % \item Large scale parallelization of core computations
  %   \begin{compactitem}
  %   \item Higher levels algorithms
  %   \item Shared memory parallelization
  %   \end{compactitem}
  % \item Closer integration of existing parallel features in \Sage
  %   \begin{compactitem}
  %   \item DSL, runtime
  %   \item resource management, etc
  %   \end{compactitem}
  % \end{compactitem}
  % \TOWRITE{JGD/CP}{Task around HPC/parallelism/perf in Linbox}

\end{task}

\begin{task}[title=Singular,lead=UK, PM=47, id=hpc-singular,wphases=0-48!0.9,issue=102]
% Researcher: 46PM -> 2-48
% Management (Wolfram): 1PM
%  \label{task:hpc-singular}
  The unique challenge of parallelizing Singular has been that it is a decades-old
  project, with a codebase exceeding 300,000 lines of code and an enormous existing
  investment of development effort. This makes a wholesale manual rewrite or reengineering
  approach infeasible.

  We therefore use a multi-pronged approach: First, we have created automated
  source-to-source translation tools that take existing C/C++ code as input and generate
  thread safe code as output. Secondly we are also adding facilities to the C/C++ code and
  the Singular interpreter to safely access shared memory. These facilities ensure in
  particular that common pitfalls of concurrent programming, such as data races and
  deadlocks, cannot occur. For this, we leverage approaches that have already been
  successfully used for HPC-GAP and whose principles are well-understood.

  To supplement the above existing work, we propose to add very fine-grained parallelism
  to some key components of Singular. These include writing a multi-threaded
  implementation of the Singular multivariate polynomial arithmetic, of the main quadratic
  sieve implementation for integer factoring and parallelisation of the FFT based integer
  and dense polynomial multiplication algorithm. These key components are used extensively
  for Singular's overall workload, including in the Groebner basis engine and polynomial
  subsystems. Performance increases through fine-grained parallelisation of key components
  such as these will provide extensive benefits to virtually all users of Singular on
  multi-core machines.
  Output are \localdelivref{QS-linalg}, \localdelivref{FFT} and~\localdelivref{singular-polyarith}.
\end{task}

\begin{task}[title=\MPIR,id=hpc-mpir, lead=UK,PM=13,wphases=6-18,issue=103]
% Researcher: 12PM
% Wolfram (Management): 1PM
\MPIR (Multiple Precision Integers and Rationals) is the core library in \Sage
for bignum arithmetic. It is used extensively by a majority of the core C/C++
libraries in \Sage, and by \Sage directly via \Cython. \MPIR is a fork of the 
GMP (Gnu Multi-precision) library, with many independent implementations of the
core algorithms (including a faster FFT and division code, better 
superoptimisation on some common 64 bit processors and native MSVC support). 
It consists of around 250,000 lines of code, much of which is assembly 
primitives and very low level, highly optimised C code.

Maintenance of \MPIR is not merely a matter of updating the build system.
Rather, every time a new processor is released by AMD, Intel, Sparc or ARM,
significant development has to be invested in hand-optimising and then
superoptimising assembly code for the new processors. This gives up to a 12x
performance increase over optimised C code, due to the specialised nature of
bignum arithmetic, which is in some sense a worst case for C compilers. Indeed
without continuous effort, \MPIR would not even run on new operating systems and
processors, let alone run fast. This is a unique problem that assembly libraries
have.

As a successful and key component of \Sage, we believe it is time to invest in
maintenance and improvement of \MPIR by hiring an assembly expert who can work
full time on the project after \MPIR's lead assembly expert sadly passed
away recently. Significant challenges exist, such as
optimising for SIMD instruction sets. Without investment into maintenance,
assembly superoptimisation, processor support, fat binary support, etc. this key
component of \Sage will fall behind, to the detriment of \Sage as a whole and the
numerous other standalone libraries that make use of \MPIR.

Output is \localdelivref{MPIRsuperoptimiser} and a contribution to \localdelivref{FFT}.
\end{task}

\begin{task}[title=HPC infrastructure for combinatorics,id=hpc-combi,PM=26,lead=PS,partners={UB},wphases={0-6!0.3,12-36!0.5},issue=104]
  Several members of the projects are experts in combinatorics a field where
  Sage is clearly a world leader~\cite{Sage-Combinat}. This particular research
  field has several specific features which makes it interesting from the HPC
  point of view.

  The most important feature is that the goal of research is mostly to
  design and to understand properties of algorithms. As a consequence,
  much more often that in other field, the researcher needs to
  program. However, this is not his ultimate goal so the programming
  environment must be very expressive for fast prototyping.

  At the same time, the problems often require relatively large
  computations; algorithms of exponential complexity are extremely
  common, and combinatorial explosion is the main obstacle to many
  experiments. Hence the programming environment must make no
  compromise on efficiency.

  Finally, embarrassingly parallel problem are extremely common, and
  more often than not problems that are not embarrassingly parallel
  reduce to the exploration of a large tree. Hence the programming
  environment must minimise the extra work needed to get from a serial
  program to a parallel one in these simple situations.

  Through this task we will provide a concrete, practical, and highly
  demanding use cases for the infrastructure developed in this work
  package. In particular, they will serve to evaluate the benefits of
  tasks~\localtaskref{pythran},
  \taskref{component-architecture}{mod-packaging}, and
  \taskref{component-architecture}{component-for-HPC}.
  In particular, we will provide a mixed C/Python implementation that
  will be integrated within Sage and replace most of the Sage-combinat
  code~(deliverable~\localdelivref{sage-paral-tree} 
  and \localdelivref{sage-HPCcombi}).

  In a second and more exploratory direction, some
  experiments~\cite{FromentinHivert} shows that the large tree exploration
  problem is very easily solved in C++ using the new Intel
  \texttt{Cilk++}~\cite{CilkIntel,CilkRefman} technology (See for example:\\
  \href{https://github.com/jfromentin/nsgtree}{https://github.com/jfromentin/nsgtree}
  and
  \href{https://github.com/hivert/IVMPG}{https://github.com/hivert/IVMPG}). We
  would like to explore the possibility to interface smoothly \Pythran,
  \Cython, and \texttt{Cilk++} (deliverable~\localdelivref{cython-pythran-cilk}).
\end{task}

\begin{task}[title=Pythran,id=pythran,lead=LL,partners={UJF},PM=24, wphases=0-24, issue=105]
  \Cython (a fork of \texttt{Pyrex}) is an extended-\Python to C
  compiler that has received significant contributions from \Sage
  developers, and is a thriving project of its own.

  \Pythran and \Cython are similar in spirit but have complementary feature
  sets: \Pythran can heavily optimise high level \Numpy constructs and \Cython
  has broader \Python support. In this task, we will investigate the
  opportunity and feasibility of a convergence between \Cython and \Pythran.
  \begin{compactitem}
    \item depending on the code at hand, one strategy or the other would be automatically selected.
    \item The optimised runtime of \Pythran can be used through \Cython.
  \end{compactitem}
  An effort will be made to improve more and more the parallelism in the
  \Pythran runtime.

  This work will be achieved through a close collaboration between the \Pythran
  developers hired for \TheProject and \Cython developers involved in the \Sage
  project. It should quicken \Sage execution time at least on \Numpy centric
  codes, while not putting an extra burden on the developers.  Preliminary
  discussions with the \Cython community have already taken place and received a
  very favorable feedback.

  Adding \Pythran support in \Sage will be done not only for \Sage code but also
  for \Sage users code thanks to compilation facilities in the notebook interface.
  Output is \localdelivref{pythran-sage}.

%  In a similar perspective, testing and improving the integration between
%  \software{mpi4py} and \Pythran could provide an efficient toolchain for HPC
%  while keeping full backward compatibility with pure \Python code. This will
%  required a continuous integration of \Pythran to ensure its capabilities
%  \localdelivref{pythran}.

  Internally, \Sage uses \Cython for compiling the critical sections of
  its libraries. In this task, we will explore opportunities to
  benefit from \Pythran compilation within the \Sage library to benefit
  \Pythran compile time optimisation. A specific challenge is that the \Sage
  library uses quite heavily object-oriented programming.

  A first step to support object-oriented programming will be to make
  \Pythran type inference more accurate, which will also improve error
  feedback provided for the user. Output is \localdelivref{pythran-typing}.
\end{task}

%Stuart Mumford, Neil Lawrence and Mike Croucher, Sheffield
\begin{task}[title=Sun Grid Engine Integration in Project \Jupyter Hub, lead=USH,id=hpc-jupyter,PM=12,wphases=0-12,issue=106]
The Sun Grid Engine is a commonly used High Performance Computing
scheduler. It is used, for example, on the institutional HPC systems
in both Sheffield and Manchester Universities as well as the regional
N8 HPC facility, a system shared by the 8 most research intensive
universities in the North of England. In this task, we will develop
and demonstrate a Sun Grid Engine notebook spawner for Project
\Jupyter, allowing users to access \Jupyter notebooks on the HPC
cluster. This will enable the interactive analysis of output data
products on the cluster where they were generated and are stored, via
a user-friendly web interface \localdelivref{SGE-jupyter}.
\end{task}
\end{tasklist}

\begin{wpdelivs}
  \begin{wpdeliv}[due=3,miles=startup,id=sage-paral-tree,dissem=PU,nature=DEM,lead=PS,issue=107, status=submitted]
      {Turn the Python prototypes for tree exploration into production code, integrate to \Sage.}
\end{wpdeliv}
  \begin{wpdeliv}[due=18,miles=startup,id=pythran-sage,dissem=PU,nature=DEM,lead=UJF,issue=115, status=submitted]
      {Facility to compile \Pythran compliant user kernels and \Sage code and automatically
       take advantage of multi-cores and SIMD instruction units in \Cython}
  \end{wpdeliv}
%  \begin{wpdeliv}[due=6,id=QS-sieving,dissem=PU,nature=DEM,lead=UK]
 %     {Parallelise the relation sieving component of the Quadratic Sieve}
%  \end{wpdeliv}
  \begin{wpdeliv}[due=12,miles=startup,id=SGE-jupyter,dissem=PU,nature=OTHER,lead=USH,issue=116, status=submitted]
      {Sun Grid Engine support for Project \Jupyter Hub}
  \end{wpdeliv}
  \begin{wpdeliv}[due=18,miles=startup,id=pythran-typing,dissem=PU,nature=DEM, lead=LL,issue=117, status=submitted]
      {Make \Pythran typing better to improve error information.}
  \end{wpdeliv}
  \begin{wpdeliv}[due=18,miles=proto1,id=MPIRsuperoptimiser,dissem=PU,nature=DEM,lead=UK,issue=118, status=submitted, blog=https://wbhart.blogspot.fr/2017/02/assembly-superoptimisation-in-mpir.html]
      {Write an assembly superoptimiser supporting AVX and upcoming Intel processor extensions for the \MPIR library and optimise MPIR for modern processors.}
\end{wpdeliv}
  \begin{wpdeliv}[due=18,miles=proto1,id=QS-linalg,dissem=PU,nature=DEM,lead=UK,issue=119,
    status=submitted, blog={https://wbhart.blogspot.fr/2017/02/integer-factorisation-in-flint.html}]
      {Parallelise the relation sieving component of the Quadratic Sieve and implement a parallel version of Block-Wiederman linear algebra over GF2 and implement large prime variants}
  \end{wpdeliv}
  \begin{wpdeliv}[due=18,miles=proto1,id=FFT,dissem=PU,nature=DEM, lead=UK,issue=120, status=submitted, blog=https://wbhart.blogspot.fr/2017/02/parallelising-integer-and-polynomial.html]
    {Take advantage of multiple cores in the matrix Fourier Algorithm component of the FFT for integer and polynomial arithmetic,and include assembly primitives for SIMD processor instructions (e.g. AVX, etc.), especially in the FFT butterflies.}
\end{wpdeliv}
  %% \begin{wpdeliv}[due=18,miles=proto1,id=GAP-hpc-report,dissem=PU,nature=R,lead=SA]
  %%   {Report on development of designs for the \GAP developments --
  %%     parallel library, interacts to standard infrastructure and
  %%     \Cython-like extensions }
  %% \end{wpdeliv}
  \begin{wpdeliv}[due=24,miles=proto1,id=cython-pythran-cilk,dissem=PU,nature=DEM,lead=PS,issue=121]
      {Explore the possibility to interface smoothly \Pythran, \Cython and \texttt{Cilk++}.}
\end{wpdeliv}
  \begin{wpdeliv}[due=24,miles=proto1,id=LinBox-DSL,dissem=PU,nature=R,lead=UJF,issue=122]
    {Library design and domain specific language exposing \Linbox parallel features to \Sage}
  \end{wpdeliv}
  \begin{wpdeliv}[due=24,id=pari-hpc1,dissem=PU,nature=DEM,lead=UB,issue=108]
  {Devise a generic parallelisation engine for \Pari and use it to prototype selected functions (integer factorisation,
  discrete logarithm, modular polynomials)}
  \end{wpdeliv}
  \begin{wpdeliv}[due=36,miles=community,id=sage-HPCcombi,dissem=PU,nature=DEM,lead=UB,issue=109]
      {Refactor and Optimise the existing combinatorics \Sage code using the new developed \Pythran and \Cython features.}
  \end{wpdeliv}
%   \begin{wpdeliv}[due=24,id=MPIRfat,dissem=PU,nature=DEM,lead=UK]
%       {Provide FAT binary support for all new x86\_64 processors, build system
%         maintenance and improvements to tuning, profiling and testing subsystems
%       for the \MPIR library.}
% \end{wpdeliv}
%  \begin{wpdeliv}[due=36,id=singular-polymul,dissem=PU,nature=DEM,lead=UK]
%      {Parallelise the Singular sparse polynomial multiplication algorithms}
%\end{wpdeliv}
  \begin{wpdeliv}[due=36,miles=community,id=LinBox-algo,dissem=PU,nature=DEM, lead=UJF,issue=110]
    {Exact linear algebra algorithms and implementations. Library maintenance and close integration
      in mathematical software for \Linbox library}
  \end{wpdeliv}
  % \begin{wpdeliv}[due=48,id=MPIRprocessors,dissem=PU,nature=DEM,lead=UK]
  %     {Ongoing support of Intel, AMD, ARM, Sparc processors and new Operating
  %       System versions for the \MPIR library.}
  % \end{wpdeliv}
  % \begin{wpdeliv}[due=36,id=pari-hpc2,dissem=PU,nature=R,lead=UB]
  % {Identify the list of all the functions that could benefit from the generic parallelization engine}
  % \end{wpdeliv}
  %% \begin{wpdeliv}[due=47,miles=eval,id=GAP-software-final,dissem=PU,nature=OTHER,lead=SA]
  %%     {Implementations of the \GAP developments, ready for release}
  %% \end{wpdeliv}
  \begin{wpdeliv}[due=48,miles=eval,id=singular-polyarith,dissem=PU,nature=DEM, lead=UK,issue=111]
      {Parallelise the Singular sparse polynomial multiplication algorithms and
        provide parallel versions of the Singular sparse polynomial division and GCD algorithms.}
\end{wpdeliv}
  \begin{wpdeliv}[due=48,miles=eval,id=LinBox-distributed,dissem=PU,nature=DEM, lead=UJF,issue=112]
    {Implementations of exact linear algebra algorithms on distributed memory et heterogenous
      architectures: clusters and accelerators. Solving large linear systems
      over the rationals is the target application.} 
  \end{wpdeliv}
  \begin{wpdeliv}[due=48,miles=eval,id=GAP-HPC-report,dissem=PU,nature=R,lead=SA,issue=113]
      {Final report and evaluation of all the \GAP developments.}
  \end{wpdeliv}
  \begin{wpdeliv}[due=48,id=pari-hpc2,dissem=PU,nature=DEM,lead=UB,issue=114]
  {\Pari suite release (\libpari, \GP and \GPtoC) that fully support parallelisation
   allowing individual implementations to scale gracefully between single
   core / multicore / massively parallel machines.}
  \end{wpdeliv}

\end{wpdelivs}
\end{workpackage}
%%% Local Variables:
%%% mode: latex
%%% TeX-master: "../proposal"
%%% End:


%  LocalWords:  workpackage hpc wphases Pythran Linbox wpobjectives wpdescription texttt
%  LocalWords:  localtaskref Cython tasklist TOWRITE parallelisation Groebner bignum Cilk
%  LocalWords:  superoptimisation optimised Sparc hand-optimising superoptimising Numpy
%  LocalWords:  specialised optimising hpc-combi deployment-distrib delivref mpi4py
%  LocalWords:  optimisations wpdelivs wpdeliv dissem LinBox-algo Parallelise QS-linalg
%  LocalWords:  Block-Wiederman singular-polymul singular-polyarith MPIRsuperoptimiser
%  LocalWords:  superoptimiser MPIRprocessors MPIRfat HPCcombi compactitem taskref
%  LocalWords:  localdelivref optimisation Jupyter hpc-mpir Sage-Combinat FromentinHivert
%  LocalWords:  CilkRefman sage-paral-tree randomised analysing
\newpage
\begin{workpackage}[id=dksbases,%wphases=1-48!.5,
  title=Data/Knowledge/Software-Bases,lead=FAU,
  ZHRM=12,JURM=12,FAURM=34,UWRM=25,SARM=10,LLRM=2,PSRM=37]

\begin{wpobjectives}
  The ultimate purpose of a mathematical VRE is to create \emph{data} ($\mathcal{D}$; see
  Section~\ref{sec:innovation}), \emph{knowledge} ($\mathcal{K}$), and \emph{software} ($\mathcal{S}$)
  by modeling world situations, computing mathematical objects, and running
  computational experiments. To be effective a VRE needs an infrastructure that supports
  the creation, management, access, and dissemination of \DKS-Structures.  All
  the systems considered in this proposal (\GAP, \Sage, \Pari, \Singular, OEIS, \texttt{arXiv.org},
  \ldots) already include data, knowledge, and software modules as part of their regular
  distribution, but not in a form that is interoperable between systems, severely limiting
  the usefulness of the systems and results. The objectives of this work package are
\begin{compactenum}
\item to design metadata and representation formats for trans-system $\mathcal{DKS}$
  structures as a basis for a math VRE, 
\item implement interfaces to existing systems for interoperability and compatibility with
  the RE, and
\item implement a joint \DKS infrastructure for, searches, documentation, traceability,
  versioning, provenance, visualisation and native dissemination of \TheProject results
  (the latter three together with \WPref{UI}).
\end{compactenum}
Concretely we will design and build an infrastructure that would make it easy for either
individual mathematicians or a distributed collaboration to manage and use such
interlinked mathematical data. This work would provide part of the backend to \WPref{UI},
and would draw on previous work with the \LMFDB and \FindStat (which will be treated as
prototypes for our purposes, to serve as exemplars to other projects) and in return will
substantially enhance their capabilities.

User prerequisites should be kept to a minimum (depending on contributors' and users'
needs and goals), and in particular would not require any background in databases to
contribute new data or perform queries.
\end{wpobjectives}

\begin{wpdescription}
  We need ways to represent \DKS in the same representational system, make the \DKS
  structures explicit and therefore machine-manageable and -- since current
  computational/experimental mathematics involve quite extensive \DKS -- we need a new
  kind of ``database'', which we will call Mathematical Data/Knowledge/Software-base
  (\textbf{\DKS base}), and which we will build in this work package.

  The starting points for this unification effort will be the system-oriented data bases
  for $\mathcal{D}$, the OMDoc (\underline{O}pen \underline{M}athematical
  \underline{Doc}uments) framework~\cite{Kohlhase:OMDoc1.2} for $\mathcal{K}$.
  OMDoc/MMT~\cite{RabKoh:WSMSML13} is a representation format for mathematical documents
  and knowledge that incorporates a metalogical framework to be foundation-independent,
  which allows interoperability between various ontologies/foundations of mathematics. For
  the integration of $\mathcal{K}$ and $\mathcal{S}$ we will build on the notion of
  \emph{biform theories} developed by Carette/Farmer~\cite{Farmer:btc07} and extended to
  OMDoc/MMT by \site{JU} in~\cite{KohManRab:aumftg13}. In this setup, the programming
  language serves as a foundation, just as ZFC set theory might for mathematical
  knowledge. Complex relationships between mathematical objects, interpretations of the
  underlying languages, and unit transformations are modeled in a graph of theories and
  theory morphisms.

  The complexity of mathematical \DKS structures is on vivid display in the
  \emph{L-functions and Modular Forms database} project (\LMFDB): while the general shape
  of the functional equation of an $L$-function is dependent on a lot of theoretical
  knowledge, it also requires parameter data and the coefficients of the associated
  Dirichlet series. Once this is obtained, highly optimised (and heavily parallelizable)
  algorithms can be run to compute values of this function.
\end{wpdescription}

\begin{tasklist}
\begin{task}[title={Survey of existing \DKS bases, Formulation of requirements},
  id=data-assessment,lead=ZH,partners={JU,SA,UW,US},wphases=0-3,PM=4,issue=123]
  In this task, we will survey existing databases, the technology used to implement them,
  how they were linked to the rest of the existing infrastructure and the functionalities
  offered. We will also select additional external data and projects to add to this
  effort, aiming to maximise the impact of our work.

  We will organise a workshop associated to this task (see
  \taskref{dissem}{devel-workshops}). Results will be communicated in
  \localdelivref{design}.
\end{task}

\begin{task}[title=Triform Theories in OMDoc/MMT,id=data-triform,
  lead=JU,partners={ZH},PM=12,wphases=0-12,issue=124]
  Work here would extend OMDoc/MMT biform theories along the data axis, which will require
  a specialised but integrated treatment. This integration will serve as a theoretical
  basis informing the design of a \DKS base in \localtaskref{data-design}.

  The results are reported in \localdelivref{dkstheories}.
\end{task}

\begin{task}[id=data-design,lead=JU,partners={ZH,US,SA,UW,LL,FAU},wphases={6-12,15-18!.33},PM=12,
  title={\DKS Base Design},issue=125]
  Ontologies are the canonical method used to implement databases that require significant
  data interchange. However, because of the extreme reification present in mathematics
  (relations between objects themselves become objects of study), there are specific
  obstacles compared to the usual semantic web model of publishing.

  Drawing on semantic web/Linked Open Data experience of the \site{LL} group, specialised
  to mathematics through the OMDoc/MMT work of the Bremen group, we will design a
  decentralised infrastructure for \TheProject. This infrastructure would allow modular
  collaborations, through decentralised hosting of data without the need to merge
  everything centrally.
  
  The initial design of the \DKS base in \TheProject is reported in
 \localdelivref{design}. Conversion issues are covered in \localtaskref{data-foundationCAS}.
\end{task}

\begin{task}[title=Computational Foundation for Python/Sage,
  id=data-foundationCAS,lead=FAU,partners={ZH,SA,JU},PM=9,wphases=6-18!.66,issue=126]
  In the OMDoc/MMT world a foundation is a logical base language that gives the formal
  meaning to all objects represented/formalised in it. We have created a very initial
  computational foundation for the programming language Scala and implemented it in the MMT API. This can be used
  to execute (or verify) computations directly in OMDoc/MMT and thus forms the basis for
  various integration tasks for OMDoc/MMT biform theories that integrate Scala
  computations. Here we propose to develop a somewhat more complete computational
  foundation for Python and/or parts of Sage (coverage to be determined). Bi/Triform
  theories come in three parts:
  \begin{compactitem}
  \item \emph{syntax}: what operators/types are there, how do they nest,
  \item \emph{computation}: what does the computation relation look like (sometimes called
    operational semantics). The declarative semantics of a computational foundation can be
    given as an OMDoc/MMT theory morphism into another foundation (e.g. a set theory);
  \item \emph{specification}: what are the observable properties of the computation. 
  \end{compactitem}
  The foundation (a triform theory in OMDoc/MMT) will be published as
  \localdelivref{psfoundation}.
\end{task}

\begin{task}[title=Knowledge-based code infrastructure, id=research-categories,lead=PS,partners={ZH,JU,FAU},PM=33,wphases=12-48,issue=127]
  Over the last decades, computational components, and in particular
  Axiom, MuPAD, \GAP, or \Sage, have embedded more and more
  mathematical knowledge directly inside the code, as a way to better
  structure it for expressiveness, flexibility, composability,
  documentation, and robustness. In this task we will review the
  various approaches taken in these software (e.g. categories and
  dynamic class hierarchies) and in proof assistants like Coq
  (e.g. static type systems), and compare their respective strength
  and weaknesses on concrete case studies. We will also explore
  whether paradigms offered by recent programming languages like Julia
  or Scala could enable a better implementation. Based on this we will
  suggest and experiment with design improvements, and explore
  challenges such as the compilation, verification, or
  interoperability of such code.
\end{task}

\begin{task}[title=OEIS Case Study (Coverage and automated Import),id=data-OEIS,lead=FAU,partners={JU},
  PM=6,wphases=12-18,issue=128]
In this case study we test the practical coverage of the trifunctional modules, by
transforming an existing, high-profile database (the Online Sequence of Integer
Sequences\footnote{\url{http://www.oeis.org}}) into OMDoc/MMT. The OEIS has about 250
thousand sequences, with formulae, descriptions, definitions, references, software,
etc. in a structured text file (but no standardised format for formulae and references),
so we expect to get 250 k theories. Having the OEIS in OMDoc/MMT form allows to do
Knowledge Management services (presentation, definition lookup, formula search, ...) in
\MathHub (see \WPref{UI}). The OEIS is a good case study, since the data is licensed under
a Creative Commons license which allows derived works. The large size will allow
statistically significant semantic cross-validation of the heuristic transformation
process and thus achieve a significant community resource.

The results of the import are reported in \localdelivref{conv}.
\end{task}

% Michael, I think triformal theories would be easier to start with findstat.org
% There are many reasons: more consistent structure in the mathematical data, more established research patterns, more consistent database storage, tighter integration of the code with sage code (in fact copy paste), etc

\begin{task}[title=FindStat Case Study (Triformal Theories),id=data-findstat,
  lead=FAU,partners={ZH},PM=9,wphases=18-30!.5,issue=129]
  In this task we would develop triformal theories for the \FindStat project \footnote{\url{http://www.findstat.org}} to test the
  design from \localtaskref{data-foundationCAS}.  Similarly to the previous task, in this
  case study, we first develop a thorough OMDoc/MMT model, which should only involve a
  handful of MMT theories (combinatorial collections, maps, statistics,...), each with a
  few hundred realisations. Together with   \WPref{UI}, this will again allow for
  easier knowledge management services, and in particular improved search services.

  This Task will be co-developed with \localtaskref{data-foundationCAS}, it will validate
  the design of triformal theories and be iterated to test the design changes. Results
  will be reported in \localdelivref{conv}.
\end{task}

\begin{task}[title=\LMFDB Case Study (Triformal Theories),id=data-LMFDB,
  lead=JU,partners={ZH,UW},PM=24,wphases={12-24!.25,24-48!.7},issue=130]
  In this task we would develop triformal theories for an exemplary part of the \LMFDB
  project to test the design from \localtaskref{data-foundationCAS}.  We will identify a
  fragment of the \LMFDB that we want to model and design the model (see
  \localdelivref{conv}). 

  Then we will perform cross-validation of the three model parts against each other
  (essentially model-based testing of software and inference; see
  \localdelivref{lfmverif}). Once this has been successful for the chosen fragment, we
  will try to semi-automatically extend the import and model to the whole \LMFDB to gain
  coverage and integrate it fully into the \DKS base. We expect that this will entail
  quite a lot of work in refactoring the \LMFDB proper, which will benefit the \LMFDB
  community independently of its use in \TheProject.

  Finally, we will pick an algorithm from the \LMFDB and verify it against its
  specification and the computational foundation developed in
  \localtaskref{data-foundationCAS}; this is the final validation of the case study. The
  results are reported in \localdelivref{lfmint}.
  \end{task}

\begin{task}[title=Memoisation and production of new data,id=data-memo,
  lead=SA,partners={US,PS,UW},PM=12,wphases=24-42!.6,issue=131]
  Many CAS users run large and intensive computations, for which they want to collect the
  results while simultaneously working on software improvements. \GAP retains computed
  attribute values of objects within a session; \Sage currently has a limited
  \texttt{cached\_method}. Neither offers storage that is persistent across sessions or
  supports publication of the result or sharing within a collaboration. We will use,
  extend and contribute back to, an appropriate established persistent memoisation
  infrastructure, such as \texttt{python-joblib}, \texttt{redis-simple-cache} or
  \texttt{dogpile.cache}, adding features needed for storage and use of results in
  mathematical research. We will design something that is simple to deploy and configure,
  and makes it easy to share results in a controlled manner, but provides enough assurance
  to enable the user to rely on the data, give proper credit to the original computation
  and rerun the computation if they want to. Results are reported in \localdelivref{persistent-memoization}.

%Mock code:
%    \begin{lstlisting}
%       mycloud = storage("ssh:xxx@yyy/zzz")
%       memoize(sage.combinat...., storage=mycloud, input=ZZ, output=Posets(), key="catalan")
%    \end{lstlisting}
\end{task}

\begin{task}[id=mws,title=Math Search Engine,lead=JU,PM=10,wphases={3-9!.3,21-42!.5},issue=132]
  The advantage of having a unified \DKS base for a math VRE is that we can navigate the
  combined information space of all the underlying tools, systems and resources integrated
  into the VRE. The negative effect is that this aggravates the already serious problem of
  finding anything. Therefore we will adapt the existing MathWebSearch Engine
  (MWS~\cite{ProKoh:mwssofse12,MWSProj:on}) to the \DKS base system. MWS consists of a web
  service that indexes mathematical documents (formula/text) and a web front-end that
  allows users to query the index. Formula queries are highly efficient (25$\mu s$/query)
  and can be combined with keyword/full text search queries. An initial search engine for
  papers and system documentation will be established early in the project (see
  \localdelivref{mws}). For an integration into the \DKS base we only need to build new
  harvesters -- i.e. programs that generate keywords and formula URL/pairs from the
  contents (see \localdelivref{notebooksearch}).

  For the data/software components in \DKS this is true in principle, but the formulae in
  code can take many more forms and the notion of a hit URL is not as clear. But the
  theory graph structure and foundation change morphisms can be integrated into search so
  that even systems that are incompatible at first glance can be searched under one
  interface~\cite{KohIan:ssmk12}.

  But this puts high demands on the search interface (user inputs are usually only
  meaningful with respect to a given context). We will explore this together with the
  notebook development -- semantically annotated notebooks and active documents serve as
  an explicit context here together with \WPref{UI}; results of this integration will be
  reported in \localdelivref{nbad-search}.
\end{task}

\end{tasklist}

\begin{wpdelivs}
\begin{wpdeliv}[id=mws,miles=startup,due=9,nature=OTHER,dissem=PU,lead=JU,issue=133]
    {Full-text Search (Formulae + Keywords) over LaTeX-based Documents
      (e.g. the arXiv subset)}
  \end{wpdeliv}
  \begin{wpdeliv}[due=12,miles=startup,id=design,dissem=PU,nature=R,lead=JU,issue=136]
    {Initial \DKS base Design (including base survey and Requirements Workshop Report)}
  \end{wpdeliv}
  \begin{wpdeliv}[due=15,miles=proto1,id=dkstheories,dissem=PU,nature=R,lead=JU,issue=137]
    {Design of Triform (DKS) Theories (Specification/RNC Schema/Examples) and 
      Implementation of Triform Theories in the MMT API}
  \end{wpdeliv}
  \begin{wpdeliv}[due=24,miles=proto1,id=conv,dissem=PU,nature=DEC,lead=ZH,issue=138]
    {Conversion of existing and new Databases (\LMFDB, OEIS, \FindStat) to unified interoperable
      System: }
  \end{wpdeliv}
  \begin{wpdeliv}[due=24,miles=proto1,id=psfoundation,dissem=PU,nature=OTHER,lead=FAU,issue=139]
    {\Python/\Sage Computational Foundation Module in OMDoc/MMT}
  \end{wpdeliv}
  \begin{wpdeliv}[id=notebooksearch,due=30,nature=OTHER,dissem=PU,lead=FAU,issue=140]
    {Full-text search (Formulae + Keywords) over CAS Modules and Notebooks} (see
      \taskref{UI}{structdocs})
  \end{wpdeliv}
  \begin{wpdeliv}[due=36,miles=community,id=pssem,dissem=PU,nature=OTHER,lead=FAU,issue=141]
    {\Python/\Sage Declarative Semantics in OMDoc/MMT}
  \end{wpdeliv}
  \begin{wpdeliv}[due=36,miles=community,id=lfmverif,dissem=PU,nature=OTHER,lead=ZH,issue=142]
    {\LMFDB Algorithm Verification with respect to a Triformal Theory}
  \end{wpdeliv}
  \begin{wpdeliv}[due=42,miles=eval,id=persistent-memoization,dissem=PU,nature=OTHER,lead=SA,issue=143]
    {Shared persistent Memoisation Library for \Python/\Sage} 
  \end{wpdeliv}
  \begin{wpdeliv}[id=nbad-search,miles=eval,due=42,nature=OTHER,dissem=PU,lead=FAU,issue=134]
    {Search from Notebooks/Active Documents Interface} Often it is important to have some
    local context to inform search, therefore search from the notebook/active documents
    interface is an interesting and useful development target.
  \end{wpdeliv}
  \begin{wpdeliv}[due=48,miles=eval,id=lfmint,dissem=PU,nature=R,lead=FAU,issue=135]
    {\LMFDB Integration of Algorithms, Data and Presentation}
  \end{wpdeliv}
\end{wpdelivs}
  
\begin{comment}
Another connection: on several occasions, we found that software was the best way to
represent certain databases of mathematical knowledge. E.g. in Algebraic Combinatorics we
have a whole zoo of Hopf algebras. Many of them are implemented in MuPAD/Sage by
specifying the objects that index the basis together with computation rules for the
product and coproduct. When we want to retrieve information about such algebras, it's
usually much more convenient to look at the code than to search through the
literature. Especially since the code is usually more correct than the literature because
it's *tested*.

We may also think of providing an interface to \LMFDB via SCSCP
protocol (http://www.symbolic-computing.org/scscp) so it may
be accessed by a variety of other systems (see their current
list at http://www.symbolic-computing.org/scscp). But it's probably as
good to access it via \Sage.

\end{comment}
\end{workpackage}
%%% Local Variables:
%%% mode: latex
%%% TeX-master: "../proposal"
%%% End:

%  LocalWords:  workpackage dksbases wphases wpobjectives standardise visualisation emph
%  LocalWords:  wpdescription Swinnerton-Dyer Millenium Borcherds optimised tasklist conv
%  LocalWords:  parallelizable maximise organise biform specialised trifunctional TOWRITE
%  LocalWords:  triformal findstat.org data-findstat localdelivref realisations texttt wrt
%  LocalWords:  Memoization python-joblib texttt redis-simple-cache texttt dogpile.cache
%  LocalWords:  lstlisting mycloud memoize sage.combinat wpdelivs wpdeliv dissem Polymake
%  LocalWords:  Recomputation wsrep dkstheories dksimp pssyntax psfoundation pssem lfmmod
%  LocalWords:  modelling lfmval lfmverif lfmint oeisparser oeisvalidation Hopf coproduct
%  LocalWords:  compactitem decentralised Logilab ensuremath xspace ldots compactenum mws
%  LocalWords:  textbf athematical uments RabKoh btc07 KohManRab aumftg13 localdelivref
%  LocalWords:  lmfmod lmfval findstat ProKoh mwssofse12 MWSProj KohIan ssmk12 taskref
%  LocalWords:  nbad-search devel-workshops Jupyter-based structdocs
\newpage
%\input{WorkPackages/DevelopmentModelsForAnAcademicFreeSoftwareEcosystem}
%\input{WorkPackages/SupportingTheMathematicalProcess}
\TOWRITE{UM}{Proofread WP 7 Social aspects pass 2}
\begin{draft}
\begin{verbatim}
- [X] have all the tasks in this Work Package a lead institution?
- [X] have all deliverables in the WP a lead institution?
- [X] do all tasks list all sites involved in them? 
- [X] does the table of sites and their PM efforts match lists of sites for each task?
      (each site from the table is listed in all relevant tasks, and no site is listed
      only in the table or only at some task)
\end{verbatim}
\end{draft}


\begin{workpackage}[id=social-aspects,wphases=0-48,
  title=Social Aspects,
  lead=UO,
  UORM=23,USHRM=24]

\begin{wpobjectives}

The processes by which mathematical knowledge and mathematical
software are developed, validated and applied are quite
distinctive. In other sciences, the universe provides ``ground truth''
and the scientific texts or theories can be validated against that by
experiment. In mathematics the text itself is the ground truth. The
traditional model of mathematical research is a mathematician, or a
small group of mathematicians, standing around a blackboard, producing
a proof they would ``clean up'': remove all traces of the process
that led to its discovery and then submit the ``clean'' text to
their peers for review.

Mathematicians have adopted new technology in a variety of ways: email
and shared documents are used to collaborate on problem-solving and
writing; larger ``crowdsourcing'' \cite{polymath_SIAM, PolymathBlog},
arrangements pull together diverse experts; symbolic computation
tackles huge routine calculations; and computers check proofs that are
too long and complicated for a human to comprehend. These
technologies reveal (since email messages, version control
systems and bulletin boards can be analysed) and alter the ways in
which mathematicians collaborate.

In an EPSRC funded project ``The Social Machine of Mathematics''
Martin and others are bringing together rigorous methods from the social 
sciences to study these collaborative processes. Combining this
research with the algorithmic game theory expertise of Elkind and Pasechnik,
in this work package we intend to pursue the following objectives: 

\begin{compactitem}
\item incorporate the insights from this and similar projects into the
  design of \TheProject VRE, ensuring that it supports the ways in
  which mathematicians really work, rather than the way software
  developers---or indeed mathematicians---think they do;
\item extend this work to study the collaborative processes of free
  open source (mathematical) software development so as to produce
  guidelines for best practice as well as to develop ideas for extending existing processes
  to a ``system of systems''.
\end{compactitem} 
\end{wpobjectives}

\begin{wpdescription}

``Crowdsourcing''---fine-grained collaborative development of ideas,
  proofs or software---is a common theme to both objectives. The purpose
  of a VRE is to allow effective crowdsourcing of computationally
  supported mathematical results (theorems, proofs, etc.), 
  while free software development is inherently a collaborative process, 
  and we wish to study the best ways of allowing it to scale.

  In a sense, mathematics has been a crowdsourced endeavour, dating as
  far back as the foundation of the Royal Society (UK) in the seventeenth
  century.  The first scientific journals were published collections
  of letters received, posing questions and observations and offering
  solutions.  Although limited by the speed of physical post, this
  model had much in common with the public email lists that
  underpinned collaborative software development in the 1990s.

  In recent years, the internet and critical tools such as distributed
  version control have supported much more widespread and finer-grained
  forms of crowdsourcing, first in software development, and, more recently, 
  in mathematics: examples are provided by online mathematics communities, such as Math-overflow
  \cite{mathoverflow} and Polymath Projects \cite{polymath_SIAM,
  PolymathBlog}.  Supporting and encouraging ``Mutual
  crowdsourcing'' is the main driving force for developing and
  maintaining any large-scale open-source virtual research
  environment.

  In this work package we will build on the work of Prof. Martin and her collaborators
  on the EPSRC project, and, in particular, their study of crowdsourcing, and 
  integrate their findings with tools provided by the burgeoning field 
  of algorithmic mechanism design in order to to 
  optimise crowdsourcing workflows in open-source VREs.

\end{wpdescription}

\begin{tasklist}
\begin{task}[title=Social Science Input to
    Design,id=social-input,lead=UO,PM=18, partners={UO,PS},wphases={0-3!.5,12-42!.5},issue=144]
The purpose of this task is to ensure that the design of \TheProject
VRE reflects the lessons learned by social scientists studying the
ways in which mathematicians actually collaborate and work. Since UO and
Martin in particular are already central in the community
working in this area, we are well placed to ensure that this happens. 

As soon as the project begins, team members at UO will combine their
own work with a review of the published literature, and identify and
meet with key research groups in this area, in order to distill relevant
current knowledge for use in the design phases of other parts of the
project. They will  present the lessons
learned at project meetings and workshops and deliver it as a report
\localdelivref{social-report-three} (part I) in month 3.

After that, they will monitor the further development of this area and
ensure that any new insights are communicated promptly to the rest of
the project. This will be synthesised for archival purposes into two
further reports 
\localdelivref{social-report-three} in months 24 (part II) and 42 (part III).

We will survey the data needed to assess development models of
large-scale academic open-source projects, such as the probable
correlation between the size of the atomic contribution vs. the speed
of the contribution making it into the code, and collect appropriate
statistical data, to be published as a report (and possibly a conference
publication) \localdelivref{social-datareport}. 
The latter will require non-trivial amount of
programming work, even only for the test system, \Sage.
\end{task}

\begin{task}[title=Implications of VREs for Publication,id=social-output,lead=USH,PM=12,wphases=12-42,partners={UO},issue=145]
  A key aspect of the \TheProject VRE is support for the full
  life-cycle of mathematical research, up to, and after publication of
  results. While it is necessary to support established models of
  publication, which are central to mathematical practice and academic
  life, it is also appropriate to explore whether VRE technologies may enable 
  novel models for the distribution of scientific output 
  that are more effective for new forms of mathematical results.

  The current model for dissemination of scientific output stems from
  an era when the printing press was dominant. The process has become
  formalised through peer review and publication of journals. The PDF
  format widely used for distribution of documents reflects the
  \textit{status quo} that a scientific paper is written as if for
  printing and remains an unchanging document. In scientific blogging
  we are seeing that more rapid propagation of ideas can occur when
  the constraints of the printed format are relaxed; however, these
  dissemination routes lack the formalisation that ensures (usually) fair
  attribution of ideas and commentary.

  We will prototype and evaluate tools and ideas for dissemination of
  scientific knowledge that do not rely on a static format and allow
  for the full spectrum of scientific debate.  The tools will
  enable proper credit allocation by encouraging shared
  attribution of ideas, software and data. This will interact with
  work in \WPref{dksbases} concerning attribution and citeability for
  mathematical databases.

  Tools to be prototyped include live posters for distribution of
  knowledge, designed for integration with either large touch screens
  or smaller tablets \localdelivref{social-poster} as well as extensions
  of the Jupyter project that would provide facilities for commenting on
  notebooks, which we expect to encourage debate on mathematical and computational
  ideas \localdelivref{jupyter-comment}.
\end{task}


\begin{task}[title=Mechanism Design for Free Software Development,PM=15,lead=UO,
  wphases=6-48!.5,id=isocial-decisionmaking,issue=146]
While crowdsourced open-source software development has become an
incredibly powerful force in recent years, it still has limitations. 
Open source projects tend to be fragile, in the community sense, and
suffer from disagreements that ultimately result in ``forks'' and the
resulting duplication of effort. We will analyse this phenomenon in the framework 
of algorithmic game theory, and try to design finely tuned systems of
incentives and rewards for contribution so as to increase the stability of
the community and its useful output.

We will focus on three areas: 
(1) prioritisation of bug fixes and feature requests to achieve reliable and useful systems; 
(2) effective cooperation among multiple collaborative projects; and 
(3) making decisions about the strategic direction of the system.  

We will use prioritisation as a testbed for designing incentives that encourage all
participants to contribute towards sustained development
of the most important parts of the system.
To this end, we will use
ideas from the burgeoning field of mechanism design \cite{AGTbook} and
in particular recent research on crowdsourcing in algorithmic
mechanism design \cite{crowds}.  While doing so, we will apply
outcomes to a case study system---\Sage.  

The reason why prioritisation poses a challenge in the 
development of open-source academic software is that this process is task-driven:
typically, tasks (also known as tickets) are posted on a website, and their
priorities are set in an ad hoc manner.  This model is usually
good enough for simple bug fixing, but for more elaborate tasks it often
leads to unacceptable delays. We will apply preference
and opinion aggregation techniques \cite{pref-aggr} to develop a
community prioritisation scheme for bugs and feature requests (which may rely on a reputation scores
technique, such as one used on MathOverflow),
and implement this scheme as a TRAC \cite{Trac} add-on 
\localdelivref{social-tracaddon}.
As \Sage is using the TRAC server \cite{trac-sagemath}, 
this will be easy to test on our testbed system.

Trusting results of computer calculations is crucial for
usability; channels for communicating bug reports and fixes need to be
carefully analysed from social point of view.  Commercial 
closed-source computer algebra and other computational systems often fail to
react to bug reports in a timely manner, and sometimes fall
into the short-sighted trap of hiding bugs from potential and current
users \cite{misfort}. Open source systems are only marginally better
in this respect, as indicated by recent computer security scares, such as the one
around Bash \cite{shellshock}.  A game-theoretic analysis of
this situation will be attempted.

A key strength of free and open-source software models is the ability
to build upon pre-existing software. \GAP, \PariGP, \Singular and
especially \Sage have made heavy use of this ability. Problems arise over
time, however, as priorities of the system developers 
diverge. For instance, bugs reported by so-called ``downstream'' systems may not be
given the same priority as bugs reported by direct
users of the ``upstream'' system, or ignored altogether; 
similarly, incompatible changes can be made as long as they are acceptable 
to the direct user community, even if they cause problems
for a dependent system. We will explore how sociological and game-theoretic 
insights can be used to reduce these problems.

The results of this task will be summarised in \localdelivref{social-gametheoretic}
and reported at relevant AI workshops and conferences.
\end{task}

\begin{task}[title=Evaluation of Micromagnetic VRE,lead=XFEL,PM=6,
id=oommf-nb-evaluation,partners={UO,PS},wphases=32-44!0.5,issue=147]
  % 4 person months, 1 person month investigator time
  We will use the micromagnetic VRE demonstrator
  (\taskref{UI}{oommf-tutorial-and-documentation}), its dissemination
  workshops \linebreak(\taskref{dissem}{dissemination-of-oommf-nb-workshops})
  and interactions with its users and contributors in the
  micromagnetic community to evaluate, reflect and report on the project,
  taking into account technical and social aspects.

  A survey will be developed and used to gather user input and
  feedback on usefulness of the provided capabilities, with particular
  focus on the capabilities of the micromagnetic VRE to (i) enable new
  and better science, to (ii) allow to make progress effectively, to
  (iii) carry out computational science reproducibly, to (iv)
  collaboratively enable trust and to (v) become a self-sustained
  project from community contributions. Amongst other channels, we
  will target attendees of the micromagnetic VRE dissemination
  workshops (\taskref{dissem}{dissemination-of-oommf-nb-workshops}) to
  gather data.

  All results and insights will be summarised in a public document
  (\localdelivref{oommf-nb-evaluation}) and reported at appropriate
  workshops and conferences to share the lessons learned from this
  \Jupyter-based VRE for micromagnetics. We will create a manuscript
  for journal publication, summarising the demonstrator project and
  this evaluation. An important point of this publication is to
  provide a reference that can be cited by publications making use of
  the new micromagnetic VRE, to allow tracking of uptake and
  development of this VRE beyond the life time of this H2020 project.
\end{task}



\end{tasklist}

% Things to investigate?
% - User surveys. Cf. https://groups.google.com/d/msg/sage-devel/v8Kfky4p6D4/_xRM0bggCo8J
% - The discussion about Code of Conducts and the like

\begin{wpdelivs}
\begin{wpdeliv}[due=18,miles=proto1,id=social-datareport,dissem=PU,nature=R,lead=UO,issue=148, status=submitted]
{The flow of code and patches in open source projects}
\end{wpdeliv}

\begin{wpdeliv}[due=24,miles=proto1,id=social-tracaddon,dissem=PU,nature=OTHER,lead=UO,issue=149]
{TRAC add-on to manage ticket prioritisation}
\end{wpdeliv}

\begin{wpdeliv}[due=24,miles=proto1,id=jupyter-comment,dissem=PU,nature=DEM,lead=USH,issue=150]
   {Demo: Mechanism for comments on posted Jupyter notebooks} 
\end{wpdeliv}

 \begin{wpdeliv}[due=36,miles=community,id=social-poster,dissem=PU,nature=DEM,lead=USH,issue=151]
   {Demo: Jupyter Notebook Live Poster} 
\end{wpdeliv}

\begin{wpdeliv}[due=42,miles=eval,id=social-report-three,dissem=PU,nature=R,lead=UO,issue=152]
 {Report on relevant research in sociology of mathematics and lessons
   for design of \TheProject VRE, parts I-III, with part I (resp. II) due at month 3 (resp. 24)}
\end{wpdeliv}
\begin{wpdeliv}[due=42,miles=eval,id=social-publishing-report,dissem=PU,nature=R,lead=USH,issue=153]
{Review of new publication mechanisms, including evaluation of
  demonstrator projects}
\end{wpdeliv}

\begin{wpdeliv}[due=42,miles=eval,id=social-gametheoretic,dissem=PU,nature=R,lead=UO,issue=154]
{Game-theoretic analysis of development practices in open-source VREs}
\end{wpdeliv}

 \begin{wpdeliv}[due=48,miles=eval,id=oommf-nb-evaluation,dissem=PU,nature=R,lead=XFEL,issue=155]
      {Micromagnetic VRE environment evaluation report}
\end{wpdeliv}
\end{wpdelivs}
\end{workpackage}
%%% Local Variables:
%%% mode: latex
%%% TeX-master: "../proposal"
%%% End:

%  LocalWords:  workpackage wphases TOWRITE wpobjectives analyse wpdescription AGTbook
%  LocalWords:  mathoverflow Sagemath pref-aggr prioritisation trac-sagemath Trac misfort
%  LocalWords:  analysed shellshock tasklist datacollection decisionmaking incentivised
%  LocalWords:  OOMMFNB taskref oommf-python-interface oommf-tutorial-and-documentation
%  LocalWords:  micromagnetic dissem dissemination-of-oommf-nb-virtual-environment texttt
%  LocalWords:  dissemination-of-oommf-nb-workshops summarised delivref wpdelivs wpdeliv
%  LocalWords:  oommf-nb-evaluation compactitem recomputation-style phenomenom endeavour
%  LocalWords:  localdelivref synthesised social-datareport textit WPref dksbases Elkind
%  LocalWords:  citeability isocial-decisionmaking social-tracaddon social-gametheoretic
%  LocalWords:  linebreak micromagnetics summarising Pasechnik
\newpage
\end{workplan}

\ifgrantagreement
\endgroup
\setcounter{page}{\value{savepage}}
\fi

%%% Local Variables:
%%% mode: latex
%%% TeX-master: "../proposal"
%%% End:

%  LocalWords:  newpage workpackages workplan



%%% Local Variables:
%%% mode: latex
%%% TeX-master: "proposal"
%%% End:
