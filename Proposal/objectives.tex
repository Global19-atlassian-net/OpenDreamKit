\eucommentary{1-2 pages}
\eucommentary{\emph{Describe the specific objectives for the project,
which should be clear, measurable, realistic and achievable within the
duration of the project. Objectives should be consistent with the expected
exploitation and impact of the project (see section 2).}}

% NL: reworded this paragraph, it was a little awkward, hope I
% maintained the right sense. Moved it up here to give some context to the aims (it formerly appeared between aims and objectives).
Our research has many and varied aspects. To
construct the \TheProject virtual research environment toolkit we must
consider component architecture, user interfaces, deployment
frameworks and standardisation of software systems.  We also study
the social and technical questions needed to ensure the impact and sustainability of the
virtual environment framework: its relationship to academic publication; open source
tools and model development; data archiving and sharing and the reproducibility of
mathematical experiments.

The specific aims of \TheProject are:
\begin{compactenum}[\textbf{Aim} 1:]
\item \label{aim:collaboration} Improve the productivity of
  researchers in pure mathematics and applications by promoting
  collaborations based on mathematical \textbf{software},
  \textbf{data}, and \textbf{knowledge}.
\item \label{aim:vre} Make it easy for teams of researchers of any
  size to set up custom, collaborative \emph{Virtual Research
    Environments} tailored to their specific needs, resources and
  workflows. The \VREs should support the entire life-cycle of
  computational work in mathematical research, from initial
  exploration to publication, teaching and outreach.
  % and bridge the gaps between
  % code, published results, and educational material.
\item \label{aim:sharing} Identify and promote best practices in
  computational mathematical research including: making results easily
  reproducible; producing reusable and easily accessible
  software; sharing data in a semantically sound way; exploiting and
  supporting the growing ecosystem of computational tools.
\item \label{aim:impact} Maximise sustainability and impact in
  mathematics, neighbouring fields, and scientific computing.
\end{compactenum}

We will achieve our aims through nine objectives, as listed below.

\begin{compactenum}[\textbf{Objective} 1:]
\item\label{objective:framework} To develop and standardise an
  architecture allowing combination of mathematical, data and software
  components with off-the-shelf computing infrastructure to produce
  specialised \VREs for different communities.
  % NL: this was listed as too long, and I think the following detail
  % belongs in tasks.  The architecture will take the form of
  % standards documents and APIs equipped, where appropriate, with
  % formal or informal mathematical semantics to ensure interactions
  % are mathematically sound.
  This
  primarily addresses Aim \ref{aim:vre}, thereby contributing to Aims
  \ref{aim:collaboration} and~\ref{aim:sharing}. %%\TODO{This is a bit long}

\item\label{objectives:core} To develop open source core components
  for \VREs where existing software is not suitable. These components
  will support a variety of platforms, including standard cloud
  computing and clusters. This primarily addresses Aim~\ref{aim:vre},
  thereby contributing to Aim \ref{aim:collaboration}
  and~\ref{aim:sharing}.

\item \label{objective:community} To bring together research
  communities (e.g. users of \Jupyter, \Sage, \Singular, and \GAP) to
  symbiotically exploit overlaps in tool creation building efforts,
  avoid duplication of effort in different disciplines, and share best
  practice. This supports Aims~\ref{aim:collaboration},
  \ref{aim:sharing} and~\ref{aim:impact}.

\item \label{objective:updates} Update a range of existing open source
  mathematical software systems for seamless deployment and efficient
  execution within the VRE architecture of objective~\ref{objective:framework}.
  This fulfills part of Aim~\ref{aim:vre}.


  % Our tools will span the entire life-cycle of a research idea, . They will. This
  % project is based on existing, proven open source technologies
  % developed by our team over the last decade that have been widely
  % adopted in academia and industry.

% NL: there was a bit of an overlap with objective:community here,
% I've tried to tease it out a bit, hope that it still hits the
% target.
\item \label{objective:sustainable} Ensure that our ecosystem of
  interoperable open source components is \emph{sustainable} by
  promoting collaborative software development and outsourcing
  development to larger communities whenever suitable. This fulfills
  part of Aims~\ref{aim:sharing} and~\ref{aim:impact}.

% NL: edited to read more like an objective than a statement of intent.
\item \label{objective:social} Promote collaborative mathematics and
  science by exploring the social phenomena that underpin these
  endeavours: how do researchers collaborate in Mathematics and
  Computational Sciences?  What can be the role of \VREs?  How can
  collaborators within a VRE be credited and incentivised? This
  addresses parts of Aims~\ref{aim:sharing}, \ref{aim:collaboration},
  and~\ref{aim:vre}.

\item \label{objective:data} Identify and extend ontologies and
  standards to facilitate safe and efficient storage, reuse,
  interoperation and sharing of rich mathematical data whilst taking
  account of provenance and citability. This fulfills parts of
  Aims~\ref{aim:vre} and~\ref{aim:sharing}.

\item \label{objective:demo} Demonstrate the effectiveness of Virtual
  Research Environments built on top of \TheProject components for a
  number of real-world use cases that traverse domains. This addresses
  part of Aim~\ref{aim:vre} and through documenting best practice in
  reproducible demonstrator documents Aim~\ref{aim:sharing}.

%Long term sustainability
\item \label{objective:disseminate} Promote and disseminate
  \TheProject to the scientific community by active communication,
  workshop organisation, and training in the spirit of open-source
  software. This addresses Aim~\ref{aim:impact}.


\end{compactenum}

\subsection*{Detailed Descriptions of Objectives} % delete if a
                                % heading isn't needed here

\paragraph{Objective~\ref{objective:framework}: Virtual Research
  Environment Kit}\

Computational techniques have become a core asset for research in pure
mathematics and its applications in the last decades. Mathematics
communities have come together to develop powerful computational
tools (e.g. GAP, \PariGP, \Sage or Singular) and valuable on-line
services (e.g. the Encyclopedia of Integer Sequences\footnote{\url{http://oeis.org}} and the ATLAS of Finite Group Representations\footnote{
\url{http://brauer.maths.qmul.ac.uk/Atlas/v3/}}). In building these
systems, mathematicians have gained strong experience in collaborative
software development, with pioneering work and continuing leadership
in Europe.

A number of approaches to linking these resources have been developed,
such as the SCSCP protocol from the Framework 6 SCIEnce
project\footnote{\url{http://www.symbolic-computing.org}}, and the
incorporation of a variety of free software tools in the \Sage system,
but the overall model is still that of a single mathematician running
programs or interacting with a ``notebook'' page. The software
provides little or no support for other aspects of mathematical
research: collaboration, archival, reproducibility or linkage between
programs, data and publication. Databases are updated mainly by
mathematicians directly, retaining no record of the source of new
entries, and providing no way of referring to the actual version of
the data used in a particular computation.

In Objective \ref{objective:framework} we will \emph{design an
  architecture} which will allow existing mathematical software
systems, off-the shelf non-mathematical tools and a small number of new
components to be flexibly combined to produce versatile \VREs that will support
collaborative mathematical research throughout its entire
life-cycle. This will include software APIs, standards, and
frameworks for assuring the semantic consistency of similar
mathematical objects in different systems. It will be informed by the
outputs of Objective \ref{objective:social}, ensuring that the \VREs
fit the ways that mathematicians actually work.

Our research covers all aspects of the ecosystem, both technical
(software development models; user
    interfaces; virtual environments; deployment frameworks; novel
    collaborative tools; component architecture; design;
    standardisation of software components and databases)
and social/collaborative
(publication; data archive; reproducibility of experiments;
development models; development tools; social aspects).

It will build on the success of the open source ecosystem and
consolidate Europe's leading position in computational mathematics.
Following the call specifications, all software, data, and
publications resulting from this proposal will be open.

\paragraph{Objective~\ref{objectives:core}: Core components}\

Most of the mathematical capabilities of our software will come
from existing or updated open source mathematical systems (e.g. the
\GAP library for computational group theory and \PariGP for number
theory). Generic services such as storage, version control
(e.g. github), authentication and resource accounting will come from
off-the-shelf components building on standard infrastructures.

However, core \emph{new tools} will need to be developed or
adapted. One example is a general infrastructure for mathematical
databases, covering some of the types of data values and search
criteria common in mathematics, but rare outside, and issues such as
provenance and citation that are common to most mathematical
databases. Other examples include adapting user interface and
collaboration tools to support mathematical notation.

\paragraph{Objective~\ref{objective:community}: Community Building across Disciplines}\

Open source development is most efficient when the load is shared as
widely as possible. However, across different communities a lack of
communication can mean that good ideas are re-invented or
re-implemented, when a shared resource would be more efficient. By
fostering a more \emph{cross-disciplinary} community, sharing tools
where possible and by creating generic tools for wide distribution we
will reduce duplication of effort. This will lead to high
\emph{quality} software that is more \emph{sustainable}. The
maintenance and development effort can be focused on one tool rather
than a disparate spread of codebases. This will ensure innovative
ideas and best-practice are shared more effectively, increasing
research productivity.

While each of the communities such as the developers of \Sage,
\Singular, and \GAP need somewhat special features for their research,
they are united through being (i)~focussed on mathematical challenges,
and (ii)~needing a computational workflow. \IPython and the \Jupyter
Notebook are used widely in science and engineering. These communities
are based on (iii)~applications of mathematics that also require
computational workflows for collaborative research and
dissemination. These three common attributes distill the requirements
for core features of the \VREs described in this proposal. Community
building will also help to sustain ongoing and community driven
maintenance of a such a tool.


\paragraph{Objective~\ref{objective:updates}: Updates to Mathematical
  Software Components}\

Our vision leverages the community's decades-long investment in a
range of open source mathematical software systems. These systems are
complex, widely used and powerful, but generally designed for
operation as stand-alone programs, not as part of an integrated
VRE. Many are also not well-suited for modern platforms, needing work
to better support parallel programming, virtualisation and HPC
platforms. We will update these systems to interoperate seamlessly and
comply with best practice for portability and platform integration.


\paragraph{Objective~\ref{objective:sustainable}: A Sustainable
  Ecosystem of Software Components}\

The success of large specialised software like \PariGP, \Singular or
\GAP in the last decades has shown the viability of the academic open
source development model. The rapid takeoff of \Sage in the last
decade has proven the viability of the ``developed by users for
users'' model for general purpose systems in pure mathematics.  \Sage
development is driven by an international community of about 150
active developers, many based in universities.  Most activities are
funded indirectly by research grants targeted at specific development
in mathematics, where the software component is often an indirect
outcome.

This somewhat piecemeal approach is enabled by (i) reusing existing
components wherever possible (including hundreds of specialised open
source math libraries and the \Python programming language, with its
developers' tools and huge library) (ii) spinning off software
development (e.g. the \Cython compiler) to larger communities whenever
possible and (iii) carefully designing the development workflow.

However, critical long-term non-mathematical features:
e.g. portability; modularisation; packaging; user interfaces; large
data; parallelism; outreach toward related software, have lagged
behind. Principally this is because these components are not credible
indirect developments of stand alone projects. They need to be
assigned to a small group of full time developers. Regular funding is
also needed to improve dissemination of the toolkit to ensure the
benefits of more productive pipeline of research are felt by the wider
research community that is critically dependent on mathematical
developments. This grant will pump prime that process enabling longer
term planning and a more structured approach to component development
and assimilation.

We envisage that with the growth of the user base a core group of
institutions or companies will hire full-time developers to support
the critical needs of their in-house research or development.
Opportunities for such hiring are, for example, actively investigated
at the Laboratoire de Recherche en Informatique. At the scale of a
large university or company, the cost of software licenses for
commercial equivalents to \Sage can easily outstrip the cost of a
small team of developers. Our proposal for VRE goes beyond any
commercial software provision and bridges the gap between end users
and developers that typically delays the advance of commercial
systems.

%\TODO{Examples: LRI? Full time devs supported by research grant, like
%  for Linbox? Others?}

To reduce the number of required full time developers \TheProject will
invest toward factoring out joint needs, and outsourcing or spinning
off more components to larger communities.  \TheProject will save the
mathematics community from duplication of effort, by first outsourcing
the development of the user interface of each computational component
to \Jupyter, ensuring that \Jupyter stands up to the stringent needs
of the community. \Jupyter's large industrial and academic user base
will benefit from these contributions, but is not reliant on the
mathematics community (either in development effort or for funding) to
remain sustainable.

\TheProject will also foster productivity within the ecosystem by
investigating better collaboration processes between components and
identifying, sharing, and promoting software development best
practices.

% open source development models for
\paragraph{Objective~\ref{objective:social}: Engineering Social
  Interactions in Open Source \VREs}\

Scientists interact in the process of scientific discovery in a
variety of ways.  In particular, researchers in mathematics and
adjacent theoretic disciplines often refer to minds of collaborators
as some sort of laboratories. With the advent of internet the volume
of scientific communication increased by orders of magnitude. Recent
successful massively collaborative online projects to attack
mathematical problems, known as \emph{Polymath}, initiated by Gowers
and Tao, were not feasible 20 years ago. However, not all initiatives
are successful. There is a social aspect to interactions of this type
that is critical to a productive collaboration.

In many ways the process of designing, development and maintenance of
an open-source VRE, in which mathematics and algorithms play a key
role, closely resembles \emph{Polymath} efforts.  Thus \TheProject\
\VREs are perfect objects to study, as there is plenty of data to
analyse, and opportunity to tweak the development workflow to obtain
more relevant data, if needed.

Social aspects of interactions have become a focus of attention of a
burgeoning field of \emph{algorithmic game theory}, which provides
tools to engineer environments where all participants are incentivised
to contribute to the common good.  Finding an optimal way to allocate
reputation scores to participants of an online trading platform, such
as Ebay, is just one example where these tools are used. We will
investigate optimal ways to allocate reputation scores to developers
and users of on open source VRE for their contributions, to facilitate
the ``mutual crowdsourcing'' that is taking place as the VRE toolkit
evolves, using the \TheProject\ \VREs as a testing ground.

Another set of tools deals with questions of stability of coalitions
and collective decision making.  These are applicable to the questions
of stability of the community of developers of open source \VREs. Not
all open source projects achieve a stable, sustainable, status: forks
are created, developers leave, community interest dwindles. This
results in a waste of resources.  We develop tools for improving the
stability of open source projects using cooperative game theory. These
tools will be applied to our own development efforts in \TheProject
\VREs.


\paragraph{Objective~\ref{objective:data}: Next Generation Mathematical Databases}\

Mathematics has a rich notion of data: it can be either
numeric or symbolic data; knowledge about mathematical objects given as
statements (definitions, theorems or proofs); or software that computes
with these mathematical objects.
%
All this data is really a common resource, and should be maintained as
such. Much of this proposal, and the prior work of many of the experts
involved, is concerned with open source mathematical software, through
permissive licensing of their work.

The objective described here is to \emph{build infrastructure},
enabling mathematicians to collaboratively build this common resource,
while fostering a virtuous circle of interoperability between these
different types of data: a mathematician might implement an
algorithm, to be run later on numerical data collected by a
scientist.

\paragraph{Objective~\ref{objective:demo}: Collaborative Research Environments that Transcend Domains}\

Wide dissemination of our \VREs is critical to ensure sustainability
and reduce duplication of effort between communities. This
dissemination is not restricted to the traditional arena of
conferences, journal papers and workshops, but should exploit the high
bandwidth communication provided by the internet. To ensure
applicability of our framework, we will create a number of
\emph{demonstrators} to highlight the power of \TheProject{}
(\taskref{UI}{structdocs}, \taskref{dissem}{ibook}) across mathematics,
engineering and science. They will act so as to provide recipes for
state-of-the-art computational infrastructure tools, and provide
avenues for ensuring the repeatability of mathematical analysis.

In particular, we will create a \emph{repository of interactive
  notebooks} \taskref{dissem}{project-intro} and books across a range
of application domains (e.g. engineering mechanics, biology and
physics). The notebooks will demonstrate a variety of numerical and
symbolic techniques in self-contained executable documents. We expect
these exemplars to feed in to education at high schools and
universities (both undergraduate and postgraduate level). They will
also provide a resource for outreach and self-study.

Our demonstrator notebooks will also act as demonstrators of the
features developed in \TheProject. Having been incorporated and
developed by this project, they can be re-executed to serve both as a
regression test and to form part of the documentation of \TheProject.

%% We
%% will further develop notebook tools for a magnetic materials
%% simulation package to demonstrate the value of \TheProject{} for
%% leading edge computational science, and develop the corresponding
%% executable tutorials and documentation.


\paragraph{Objective~\ref{objective:disseminate}: Training and Dissemination}\

The success of any research software or service is strongly linked to
its ability to attract and retain a large number of users. The
different communities (Sage, Gap, \PariGP, Singular, \Jupyter, ...)
have each developed sustainable networks. For example, Sage has
accumulated thousands of users in under 10 years. This has been
achieved thanks to a very strong community building philosophy,
especially through the organisation of ``Sage-Days'' all over the
world. The first Sage-Days was held in 2006; to date there has been
at least 63 of them, including 10 during 2014, as well as Sage
Education days, Sage Bug days, Sage Doc days, Sage Days aimed
specifically at women, and more. Many of the \TheProject{} project
members have been involved in these events either as organisers or
participants, and are convinced that they are a most efficient way to
promote our software. More precisely, our objective is to create a
constant dialogue between the different communities, through frequent
workshops, conferences, user groups, and mailing lists. By building on
existing tools, we intend to involve the communities in the
development process itself in the spirit of open-source software.

We also intend to reach a larger crowd of researchers by minimising
technical (non-research) obstacles to access existing tools: building
better documentation and tutorials, developing easy-to-install
distributions, enabling easy web and cloud access, better user
interfaces, better interactions between different software.  We will
run a series of workshops to inject additional momentum into the
process. By doing this, our objective will be to \emph{help the
  communities to grow} themselves and interact together using our
work.



%%% Local Variables:
%%% mode: latex
%%% TeX-master: "proposal"
%%% End:
