\begin{participant}[PM=3,type=leadPI,gender=male]{Hans Fangohr}
  Hans Fangohr is Professor of Computational Modelling at the
  University of Southampton until August 2017 (3PM), then Senior Data
  Analysis Scientist at the European XFEL GmbH (3PM). He has studied Physics with
  specialisation in Computer Science and Applied Mathematics, gained
  his PhD in High Performance Computing (2002) in computer science and
  has since worked on the development of computational tools and
  application of those in interdisciplinary projects in science and engineering.

  He heads the University's interdisciplinary Computational Modelling
  Group (\url{http://cmg.soton.ac.uk}) at Southampton, and has more than 100
  publications on development of computational methods and applied
  computer simulation in magnetism, superconductivity, biochemistry,
  astrophysics and aircraft design.

  In 2013, he has attracted \EUR{5}m from the UK's Engineering and
  Physical Sciences Research Council (EPSRC) together with additional
  moneys from industry and his University of Southampton to fund the
  \EUR{12}m Centre for Doctoral Training in Next Generation
  Computational Modelling
  (\href{http://ngcm.soton.ac.uk}{ngcm.soton.ac.uk}) in the UK. This
  flagship activity will train about 75 PhD students (10 to 20
  starting every year, first cohort started in September 2014) in the
  state-of-the-art and best-practice in computational modelling, the
  programming of existing and emerging parallel hardware and to apply
  these skills and tools to carry out PhD research projects across a range of
  topics from Science and Engineering. The centre has chosen \IPython
  as a key tool to deliver this teaching, document and communicate
  computational exploration and drive reproducible computation to push
  for excellent computational science.

  Hans Fangohr has led the development of the Open Source \software{Nmag}
  software (\url{http://nmag.soton.ac.uk}), which provides a finite-element
  micromagnetic simulation suite to a community of material
  scientists, engineers and physicists who research magnetic
  nanostructures in academia and industry. He has designed the package
  in 2005 so that it has an \IPython-compatible \Python interface, to
  make the workflow of using the simulation package as accessible as
  possible to scientists without substantial computational
  background. He has extensive experience in micromagnetic simulation
  tool development and use, and due to this an outstanding understanding of the
  requirements for computational workflows in this micromagnetic
  research community.

  He has deep interest in excellence and innovation in learning and
  teaching. He has been awarded the prestigious Vice Chancellor's
  teaching award ($\pounds 1000$) three times (in 2006, 2010, 2013)
  for initiating and realising three separate innovations in the
  university's teaching delivery of computational engineering, and has
  been voted ``best lecturer'' and ``funniest lecturer'' of the year
  by the students. Other Universities in the UK and elsewhere have
  adopted his teaching methods and materials. He has attracted grants
  to further develop learning and teaching activities, and given
  invited talks at international meetings on efficient learning and
  teaching of computational methods.

  Hans Fangohr is chairing the UK's national Scientific Advisory
  Committee for High Performance Computing.
\end{participant}

%%% Local Variables:
%%% mode: latex
%%% TeX-master: "../proposal"
%%% End:
