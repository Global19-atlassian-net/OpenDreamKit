\begin{participant}[PM = 8, type=leadPI, gender=male]{Martin Sandve Aln\ae{}s}
Martin Sandve Aln\ae{}s is project leader for the Computational
Middleware project at the Centre for Biomedical Computing at Simula
Research Laboratory, a Norwegian Centre of Excellence doing
inter-disciplinary research in the intersection of mathematics,
physics, computer science, geoscience and medicine.

Aln\ae{}s received his PhD from the Department of Informatics,
University of Oslo, in 2009, and then worked as Senior Software
Developer at TANDBERG and Cisco before being hired as a Postdoctoral
Fellow in the Department for Biomedical Computing at Simula in 2011.
He is now a Senior Research Engineer at Simula since 2015.

Aln\ae{}s' research involves novel approaches to scientific software
design as well as application of computational fluid dynamics to blood
flow in aneurysms. Aln\ae{}s' is one of the main developers of the
FEniCS software suite for automation of finite element simulations,
where his main contributions have been on the use of domain specific
languages, symbolic computing, and code generation, to achieve high
productivity and high performance in a general framework.

\end{participant}
