\begin{participant}[type=PI,,gender=male]{Neil Lawrence}
  Neil Lawrence received his bachelor's degree in Mechanical
  Engineering from the University of Southampton in 1994. Following a
  period as an field engineer on oil rigs in the North Sea he returned
  to academia to complete his PhD in 2000 at the Computer Lab in
  Cambridge University. He spent a year at Microsoft Research in
  Cambridge before leaving to take up a Lectureship at the University
  of Sheffield, where he was subsequently appointed Senior Lecturer in
  2005. In January 2007 he took up a post as a Senior Research Fellow
  at the School of Computer Science in the University of Manchester
  where he worked in the Machine Learning and Optimisation research
  group. In August 2010 he returned to Sheffield to take up a
  collaborative Chair in Neuroscience and Computer Science.

  Neil's main research interest is machine learning through
  probabilistic models. He focuses on both the algorithmic side of
  these models and their application. He has a particular focus on
  applications in personalised health and computational biology, but
  happily dabbles in other areas such as speech, vision and graphics.

  Neil was Associate Editor in Chief for IEEE Transactions on Pattern
  Analysis and Machine Intelligence (from 2011-2013) and is an Action
  Editor for the Journal of Machine Learning Research. He was the
  founding editor of the JMLR Workshop and Conference Proceedings
  (2006) and is currently series editor. He was Programme Chair for
  AISTATS 2012 and has served on the programme committee of several
  international conferences. He was an area chair for the NIPS
  conference in 2005, 2006, 2012 and 2013, Workshops Chair in 2010 and
  Tutorials Chair in 2013. He was general chair of AISTATS in 2010 and
  AISTATS Programme Chair in 2012. He was Program Chair of NIPS in
  2014.

  Neil is a strong advocate of open source software in machine
  learning and has given many invited talks on the subject. Since 2004
  his research group has made all their implementations available,
  most recently using \Python and \IPython as the main medium for
  communicating their work. Their Gaussian process python software
  framework\footnote{\url{https://github.com/SheffieldML/GPy}} is
  becoming a standard platform for research in these methods and
  underpins a series of Summer Schools and four day road shows that
  Neil has led in the area.\footnote{\url{http://ml.dcs.shef.ac.uk/gpss/}}
\end{participant}
%%% Local Variables:
%%% mode: latex
%%% TeX-master: "../proposal"
%%% End:
