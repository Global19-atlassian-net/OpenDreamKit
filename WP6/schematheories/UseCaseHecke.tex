\documentclass{llncs}
%\pagestyle{plain}
\usepackage[utf8]{inputenc}

\usepackage{amsmath,amssymb}
\usepackage{hyperref}

\def\Z{{\mathbb Z}}
\def\Q{{\mathbb Q}}
\def\R{{\mathbb R}}
\def\C{{\mathbb C}}
\def\N{{\mathcal{N}}}
\def\O{{\mathcal{O}}}
\def\p{{\mathfrak{p}}}

\title{LMFDB use case: investigating Hecke fields of modular forms}
\author{
John Cremona\inst{1}
David Lowry-Duda\inst{1}
}
\institute{
University of Warwick
}

\begin{document}
\maketitle
\begin{abstract}
  We describe a use case for intercommunication between LMFDB data and
  either SageMath or PARI/GP.
\end{abstract}

\section{Introduction}
We describe a use case for intercommunication between LMFDB data and
either SageMath or PARI/GP.  The LMFDB contains many individual
instances of several kinds of Modular Forms (which are the ``MF'' in
``LMFDB''), including classical modular forms, Hilbert modular forms,
Bianchi modular forms and others.  Each of these objects has
associated to it an algebraic number field called its Hecke field.  A
mathematically interesting project is to study these Hecke fields for
the modular forms in each collection, collecting data about them such
as their degree or class number (see the next section for definitions)
and investigating how these are distributed.  Currently it is not even
known how many of these Hecke fields are themselves in the LMFDB's own
database of number fields; where they are, the associated invariants
can simply be looked up in the LMFDB itself, but in cases where the
Hecke field is not itself in the database, further computations using
SageMath would become possible.

Being able to access this LMFDB data directly from within a package
with the ability to carry out its own computations with number fields
would be a useful application.

\section{Modular forms and their Hecke fields}
We will not here give full definitions of the kinds of modular forms
to be considered since even for the simplest ones this is a
substantial mathematical theory dating back to the 19th century.  See
\url{http://www.lmfdb.org/knowledge/show/mf} and the links there for
brief definitions.  We limit ourselves to giving just those
characteristics which are necessary for the current application, in
the case of classical, Hilbert and Bianchi forms.

\subsection{Base field and level}
Each modular form has a \emph{base field} $F$ which is an algebraic
number field (that is, a finite extension of the rational field~$\Q$).
For classical modular forms we have $F=\Q$, for Hilbert Modular Forms
(HMFs) $F$ is a totally real number field (meaning that every
embedding of $f$ into $\C$ has image in $\R$, for example
$F=\Q(\sqrt{5})$), and for Bianchi Modular Forms (BMFs), $F$ is an
imaginary quadratic field (for example $F=\Q(\sqrt{-1})$).  We will
mostly restrict to the case of Hilbert Modular Forms (HMFs); strictly
speaking this includes classical modular forms (since $F=\Q$ is a
totally real field) but we follow convention in only applying the term
``Hilbert Modular Form'' when $F\not=\Q$.  Then the \emph{degree} of
$F$ is an integer $n\ge2$.

Our reason for excluding Bianchi Modular Forms from now on is simply
that at present the LMFDB has almost no BMFs whose Hecke field is
not~$\Q$.  The case of classical modular forms is mathematically very
similar, but the means of storing these forms in the LMFDB is
different, and also under current development, and would require a
different interface.

After specifying the base field, we must next give the \emph{level} of
the modular form.  This is an (integral) ideal $\N$ of the ring of
integers $\O_F$ of the base field $F$.

In general, modular forms may also have an associated
\emph{character}, but we only consider those with trivial character
here, both for simplicity and also because at present the LMFDB only
has HMFs with trivial character.  Similarly, in general modular forms
have a \emph{weight}, and everything we will say would apply (with
only minor changes) to forms of arbitrary weight; however, the HMFs
currently in the LMFDB all have the same weight (parallel weight~$2$),
so we will not mention the weight further.

\subsection{Hilbert newforms and their Hecke fields}
For a fixed base field~$F$, level~$\N$ (and fixed weight and
character), the HMFs with these parameters form a finite-dimensional
vector space.  In this space there is a finite set of distinguished
elements called \emph{newforms}; the objects actually stored in the
LMFDB are these newforms.  From now on when we refer to an HMF over
some base field $F$, will will usually mean a Hilbert newform.

Each newform $f$ is defined by and characterised by additional data in
addition to the base field, level, weight and character.  This
additional data serves to distinguish different newforms with the same
base field and level, but also carries with it information of great
significance in the study of these newforms.  Each newform has a
\emph{Fourier expansion} indexed by elements $\alpha\in\O_K$, with
Fourier coefficients $c_{\alpha}$ satisfying various multiplicative
relations, including $c_1=1$.  The \emph{Hecke field} of~$f$ is the
field~$K_f$ generated over $\Q$ by all these
coefficients~$c_{\alpha}$, which is an algebraic number field (a
finite extension of $\Q$).  The degree of the Hecke field~$K_f$ is
called the \emph{dimension} of the newform.  For example, to say that
the Hecke field $K_f=\Q$ is equivalent to the dimension being equal
to~$1$ and to the statement that all Fourier coefficients $c_{\alpha}$
lie in $\Q$.  Moreover all the Fourier coefficients are algebraic
integers, so in fact all $c_{\alpha}\in\O_{K_f}$, and when $K_f=\Q$
even $c_{\alpha}\in\Z$.

The Hecke field has an alternate definition which explains the name,
and is also used to establish some of the properties stated: on the
complex vector space of all HMFs of a given level (and fixed weight
and character) there is a commutative algebra of linear operators
acting, called the \emph{Hecke Algebra}.  The newforms are
``eigenforms'' for the action of this algebra, normalised by scaling
so that $c_1=1$, and the coefficients $c_{\alpha}$ also arise as the
eigenvalues of suitable operators in the Hecke algebra.  The
multiplicative relations mentioned above mean that each newform is
characterised by its eigenvalues, or coefficients, indexed by
\emph{prime} elements of~$\O_F$, and that the coefficient~$c_{\alpha}$
is unaffected by scaling $\alpha$ by units in $\O_F^*$.  Hence, and to
summarize this discussion succinctly:
\begin{enumerate}
\item Every Hilbert newform $f$ has a base field $F$ and a level $\N$
  which is an ideal in $\O_F$.
\item To $f$ is also associated a number field~$K_f$ called its Hecke
  field, containing all its Fourier coefficients (or Hecke
  eigenvalues).
\item The newform $f$ is uniquely determined by the collection
  $\{c_\p\}$ indexed by prime ideals of the base field $F$, where each
  $c_{\p}\in\O_{K_f}$.
\item For fixed $F$ and $\N$ there are only finitely many newforms,
  which may be distinguished by their coefficients $c_\p$ for a finite
  collection of prime ideals~$\p$ of~$F$.
\end{enumerate}

\section{Relevant LMFDB databases and collections}
At present the LMFDB data is stored in a Mongo database.  Under Mongo
terminology the entire dataset is subdivided first into ``databases''
and each database consists of a number of ``collections''.  The
records in each collection can in principle have widely varying
content, but in practice there is little variation and we may consider
each record to have the same structure with a fixed set of keys each
mapping to a value of a predefined type.

All the Hilbert modular form data in the LMFDB is stored in a database
called {\tt hmfs}.  Within this database there are three collections:
{\tt hmfs.fields}, {\tt hmfs.forms}, and {\tt hmfs.hecke}.

\subsection{The {\tt hmfs.fields} collection}
The collection {\tt hmfs.fields} contains an entry for each base field
(denoted $F$ above).  An inventory for the records in this collection
may be seen at \url{http://www.lmfdb.org/inventory/hmfs/fields/}.
Most relevant for the current discussion are the following keys:
\begin{itemize}
\item {\tt degree}: the degree of $F$ over $\Q$ (as an integer,
  for example $2$).
\item {\tt label}: a string holding the LMFDB label
  of the field, for example {\tt 2.2.5.1}.  These labels allow
  cross-referencing to the LMFDB fields database, where most data
  about the base field is stored.
\end{itemize}
All fields which occur as base fields for HMFs in the database are
themselves in the LMFDB's number field collection.  The other keys in
{\tt hmfs.fields} are technical and need not concern us.  The main
purpose of this collection is for displaying individual Hilbert
newform data on the LMFDB website \url{www.lmfdb.org}.

The collection {\tt hmfs.fields} contains $400$ entries, of base
fields of degrees between $2$ and~$6$.

\subsection{The {\tt hmfs.forms} collection}
The collection {\tt hmfs.forms} contains an entry for each Hilbert
newform (denoted $f$ above).  An inventory for the records in this
collection may be seen at
\url{http://www.lmfdb.org/inventory/hmfs/forms/}.  The relevant keys
for our purposes are as follows:
\begin{itemize}
\item Keys relating to the base field $F$:
\begin{itemize}
\item {\tt deg}:  the degree $[F:\Q]$;
  \item  {\tt field\_label}:  the LMFDB label of $F$.
\end{itemize}
\item Keys relating to the level  $\N$:
\begin{itemize}
\item {\tt level\_ideal}:  data from which the level may be
  constructed;
  \item {\tt level\_norm}: the norm of the level;
\item {\tt level\_label} the standardised label for the level.
\end{itemize}
\item Key to specify the newform among others with the same base field
  and level: {\tt label\_suffix}.
\item Key giving the degree of the Hecke field: {\tt dimension}.
\item Other keys give composite labels; {\tt label} is a concatenation
  of the base field label, the level label and the suffix and
  completely identifies this newform.
\end{itemize}

The collection {\tt hmfs.forms} contains $368356$ entries.  One
example is
\url{http://www.lmfdb.org/ModularForm/GL2/TotallyReal/4.4.16357.1/holomorphic/4.4.16357.1-55.1-c}.
Here the vase field has label~{\tt 4.4.16357.1}, the level has
label~{\tt 55.1} and the suffix specifying one particular newform at
this level is~{\tt c}; the full label is then {\tt
  4.4.16357.1-55.1-c}.  This example has dimension~$1$, meaning that
the Hecke field is $\Q$.  Another example, which has dimension~$286$
(the current maximum) is
\url{http://www.lmfdb.org/ModularForm/GL2/TotallyReal/2.2.296.1/holomorphic/2.2.296.1-29.1-c}.

\subsection{The {\tt hmfs.hecke} collection}
The Fourier coefficients (or Hecke eigenvalues) of the newforms are
stored in a separate collection {\tt hmfs.hecke} for technical
reasons.  In the previous section we saw that the only information
about the Hecke field stored in the main {\tt hmfs.forms} collection
is the dimension, which is the degree of the Hecke field.  Here we
specify the Hecke field itself and store the individual eigenvalues
for each newform.

There is one record {\tt hmfs.hecke} for each newforms in {\tt
  hmfs.forms}, containing the following data.
\begin{itemize}
\item {\tt label}: the full label of the newform, matching the
  associated record in {\tt hmfs.forms};
\item {\tt hecke\_polynomial}: a string representing a polynomial
  $g(x)$ with rational coefficients in the variable $x$.
\item {\tt hecke\_eigenvalues}: a (finite) list of the first several
  Hecke eigenvalues $c_{\p}$ of the newform, indexed by the primes
  $\p$ of $\O_F$. Each $c_{\p}$ is stored as a string representing the
  eigenvalue with respect to the power basis for the Hecke field in
  the generator whose minimal polynomial is $g(x)$.
\end{itemize}
Note that the stored defining polynomials $g(x)$ are not canonical
defining polynomials for the Hecke field, but essential arbitrary.
(Often $g(x)$ will be the characteristic polynomial of the first
nontrivial eigenvalue.)  Hence the same Hecke field may occur for
different newforms with different defining polynomials; this will need
to be taken into account when comparing Hecke fields: it is not
sufficient to just compare the defining polynomials.  For example, the
Hecke field $\Q(\sqrt{5})$ may occur with $g(x)=x^2-5$ and also with
$x^2-x-1$.

The collection {\tt hmfs.hecke} contains $368356$ entries, one for
each entry in {\tt hmfs.forms}.  The Hecke fields have degrees up to $286$.

\section{Possible investigations}
\subsection{Degrees of Hecke fields} Over all newforms, or just those
of each base field degree, describe the distribution of their
dimensions, i.e. the degrees of the Hecke fields.  This only requires
accessing the key {\tt hmfs.forms.dimension}, possibly subdivided by
the value of  {\tt hmfs.forms.degree}.

\subsection{The Hecke fields} Again this could be carried out over all
newforms or just those with a fixed base field degree.  For each Hecke
field degree, determine which distinct Hecke fields occur and with
what frequencies.  This will require a database query to obtain the
values of {\tt hmfs.hecke.hecke\_polynomial}, followed by construction
of a field with each distinct polynomial found and a field isomorphism
test, such as can be done by SageMath.  A variation of this would be
to use SageMath's ability to find a unique canonical defining
polynomial for every number field: this functionality is provided by
the {\tt C} library {\tt pari} but there is already an interface to
{\tt pari} from SageMath.

Note that both the field isomorphism test and the canonicalization
test become very expensive when the degree increases.  It may only be
possible to carry out this (and subsequent) steps in the investigation
for newforms of relatively low dimension, for this reason.  However,
no-one has yet systematically tried to look at these higher degree
fields at all, and would be interesting to have a list of them,
especially if it were possible to find simpler representations of
them.  Moreover there is an entry in the LMFDB wish-list to change the
way to represent Hecke eigenvalues to something much more efficient
and compact (in terms of storage space and of visual appearance on the
web page), and a necessary first step would be to find such optimised
representations.

\subsection{Hecke field invariants} Following from the previous
classification, a number of different invariants of each Hecke fields
could be computed in SageMath: for example the class number.

\subsection{Hecke eigenvalues and the ring they generate} Previous
steps have only concerned the Hecke fields as fields, ignoring the
lists of actual Hecke eigenvalues stored for each newform, in the {\tt
  hmfs.hecke} collection.  It would be interesting to study what
subring of the ring of integers of the Hecke field they generate, what
index it has in the ring of integers, and how many individual Hecke
eigenvalues are require to generate this ring.  All of these questions
could be studied systematically in SageMath.

%% \section{Interfaces needed}

\section{Conclusion}


\bibliographystyle{alpha}
\bibliography{}
\end{document}
