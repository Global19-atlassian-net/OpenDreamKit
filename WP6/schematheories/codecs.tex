For each type, we define codecs. These are pairs of bijective maps between the elements of that type and subsets of JSON.

To encode/decode a mathematical object into/from JSON, we need to provide its type and a codec for that type.
(Different codecs might encode different mathematical objects as the same JSON. Thus, the meaning of an arbitrary JSON is only determined if the codec is known.)

For codec $c$ for type $T$ and a mathematical object $t$ of $T$, we write $c(t)$ for the encoding of $t$.
For some types, we define a \emph{default codec} (whose name will start with $\cn{Standard}$.
If a default codec $d$ exists, we write $\ov{t}$ for $d(t)$.

A codec $c$ is a \emph{string-codec} if encodes every value as a JSON string.
For codecs $c_i$, we write $\cunion{c_1,\ldots,c_n}$ for the codec that encodes using $c_1$ and decodes according to the first $c_i$ that is applicable to the input

\subsection{JSON}

We use the following JSON values:
\begin{compactitem}
  \item $null$
  \item 32-bit integers
  \item IEEE double precision floating point numbers; technically, JSON does not distinguish between integers and floats, and we assume that, e.g., $1$ is an integer and $1.0$ is a float
  \item booleans
  \item strings
  \item lists $[j_1,\ldots,j_n]$ for JSON values $j_i$
  \item objects $\jobj{k_1:j_1,\ldots,k_nj_n}$ for strings $k_i$ and JSON values $j_i$
\end{compactitem}
We also say \emph{pair} for lists $[j,j']$ of length $2$.

\subsection{Base Types}

\paragraph{Natural and Integer Numbers}
All subtypes of $Z$ as well as integers modulo $m$ are encoded in the same way.

We have the following encodings for an integer $n$, we

$\cn{IntAsNumber}$: the JSON integer $n$ if $|n|$ is small enough and like $\cn{IntAsString}$ otherwise

$\cn{IntAsString}$: a string containing the decimal expansion of $n$

$\cn{IntAsList}$: a list $["base",b,l,d_1,\ldots,d_{|l|}]$ where $b,l,d_i$ are JSON integers such that $\sgn l=\sgn n$ and $(d_1,\ldots,d_|l|)$ is the list of digits of $n$ relative to base $2^b$.\footnote{The value $l$ may seem redundant in the encoding (except for carrying the sign). But it has the advantage that the canonical order on integers corresponds to the lexicographic order of JSON lists.}

$\cn{StandardInt}$: the codec $\cunion{\cn{IntAsNumber},\cn{IntAsString},\cn{IntAsList}}$

\paragraph{Rational Numbers}
$\cn{StandardRat}$ encodes the rational number $e/d$ for integers $e,d$ as
\begin{compactitem}
 \item $\ov{e}$ if $d=1$
 \item the pair $[e,d]$
\end{compactitem}
For encodings, the $e$ and $d$ are fully canceled and $d>0$. For decoding, all pairs with $d\neq 0$ are accepted.
For decoding the string $"e'/d'"$ where $e'$ and $d'$ are the string encodings of $e$ and $d$ are also accepted.

\paragraph{Real Numbers}
The codec $\cn{StandardReal}$ encodes a real number as follows:
\begin{compactitem}
 \item rational numbers like $\cn{StandardRat}$
 \item floats as JSON floats
 \item $n\sqrt{x}$ as the list $["root",\ov{n},\ov{x}]$ for $n\in N$ and $x\in Z$
 \item the strings "pi" or "e"
\end{compactitem}

\paragraph{Complex Numbers}
The codec $\cn{StandardComplex}$ encodes a complex number $z$ (in any form) as
\begin{compactitem}
 \item if $z$ is in Cartesian form:
  \begin{compactitem}
    \item $\ov{x}$ if $y=0$
    \item the object $\jobj{"re": \ov{x}, "im":\ov{y}}$ otherwise
  \end{compactitem}
 \item if $z$ is in polar form
  \begin{compactitem}
   \item the object $\jobj{"abs": \ov{r}, "unitarg": \ov{\alpha}}$ if $\phi=2\pi \alpha$ for $\alpha\in[0,1[$
   \item the object $\jobj{"abs": \ov{r}, "arg": \ov{\phi}}$ otherwise
  \end{compactitem}
 \item the object $["root-of-unity", \ov{n}]$ if $z$ is a root of unity
\end{compactitem}

\paragraph{p-Adic Numbers}
The codec $\cn{StandardQp}(p)$ for $Qp(p)$ encodes the $p$-adic number with unit $u$, valuation $v$, and precision $r$ as an object $\jobj{"unit":\ov{u},"valuation":\ov{v},"precision":\ov{r}}$.
\ednote{FR: treatment of $0$?}

\paragraph{Strings}
$\cn{StandardString}$ encodes strings as JSON strings.

\paragraph{Booleans}
$\cn{BooleanAsBoolean}$ encodes booleans as JSON boolean.

$\cn{BooleanAsInt}$ encodes booleans as $0$ or $1$.

$\cn{BooleanAsString}$ encodes booleans as the strings "true" or "false".

$\cn{StandardBoolean}$ is $\cunion{\cn{BooleanAsBoolean},\cn{BooleanAsInt},\cn{BooleanAsString}}$

\subsection{Aggregating Type Constructors}

Corresponding to type constructors $T$ that form types $T(T_1,\ldots,T_n)$ from existing types, we use codec constructors $C$ that form codecs $C(C_1,\ldots,C_n)$ for $T(T_1,\ldots,T_n)$ from codecs $c_i$ for $T_i$.

In the following, we assume codes $c_1,\ldots,c_n$ for the types $T_1,\ldots,T_n$.
We write $\vec{c}$ for $c_1,\ldots,c_n$.

\paragraph{Products}
The codec $\cn{StandardProduct}(\vec{c})$ for $T_1*\ldots*T_n$ encodes tuples $(t_1,\ldots,t_n)$ as JSON lists $[c_1(t_1),\ldots,c_n(t_n)]$.

\paragraph{Records}
The codec $\cn{StandardRecord}(\vec{c})$ for $\record{k_1:T_1,\ldots,k_n:T_n}$ encodes records $\record{k_1=t_1,\ldots,k_n=t_n}$ as JSON objects $\jobj{"k_1":c_1(t_1),\ldots,"k_n":c_n(t_n)}$.

\paragraph{Unions}
The codec $\cn{StandardUnion}(\vec{c})$ for $\record{T_1+\ldots T_n}$ encodes values $t$ of $T_i$ as the JSON pair $[i,c_i(t)]$.

\paragraph{Labeled Unions}
The codec $\cn{StandardLabeledUnion}(\vec{c})$ for $\union{k_1:T_1,\ldots,k_n:T_n}$ encodes values $k_i(t)$ for $t\in T_i$ as the JSON object $\jobj{"k_i":c_i(t_i)}$.

\subsection{Collecting Type Constructors}

We assume a codec $c$ for a type $T$.

\paragraph{Options}
The codec $\cn{StandardOption}(c)$ for $Opt(T)$ encodes values $t\in T$ as $c(t)$ and omitted values as the $null$ value.

\paragraph{Vectors}
We have the following encodings for a vector $Vec(n,T)$ of the form $(t_1, \ldots, t_n)$.

$\cn{VectorAsList}(c)$: the JSON list $[c(t_1),\ldots,c(t_n)]$,

$\cn{VectorAsString}(c,\mathtt{s})$ for some character $\mathtt{s}$: the string $"c(t_1)\mathtt{s}\ldots\mathtt{s}c(t_n)"$.
For decoding, any bracketed occurrences of $\mathtt{s}$, i.e., as in  $(\ldots\mathtt{s}\ldots)$, are not parsed as separators.
Moreover, white space characters around the separator $\mathtt{s}$ are ignored when decoding.

$\cn{StandardVector}(c)$ is $\cunion{\cn{VectorAsList}(c), {\cn{VectorAsString}(c,\mathtt{','})}}$.

\paragraph{Lists}
The codec $\cn{StandardList}(c)$ encodes lists in the same way as $\cn{StandardVector}$.

\paragraph{Sets}
The codec $\cn{StandardSet}(c)$ encodes sets in the same way as $\cn{StandardVector}$, where the elements are listed in any order but without repetitions.

\paragraph{Multisets}
The codec $\cn{StandardMultiset}(c)$ encodes multisets as lists $[[c(t_1),\ov{m_1}], \ldots, [c(t_n),\ov{m_n}]]$ of pairs $(t,m)$ where $t\in T$ is an element of the multiset and with multiplicity $m\in N$.
The order of pairs is not specified.
For encoding, the same $t_i$ will not occur twice, and $m_i\neq 0$. For decoding, these cases are accepted.

\paragraph{Hybrid Sets}
The codec $\cn{StandardHybridSet}(c)$ encodes hybrid sets in the same way as $\cn{StandardMultiset}$ except that the multiplicities may be negative.

\paragraph{Matrices}
The codec $\cn{RowMatrix}(m,n,c)$ for $Mat(m,n,T)$ encodes matrices $(t_{ij})$ as the list of lists $[[c(t_{11}),\ldots,c(t_{1n})], \ldots, [c(t_{m1}),\ldots,c(t_{mn})]$.

The codec $\cn{FlatRowMatrix}(m,n,c)$ for $Mat(m,n,T)$ encodes matrices $(t_{ij})$ as the flat list $[c(t_{11}),\ldots,c(t_{mn})]$.

The codecs $\cn{ColumnMatrix}(m,n,c)$ and $\cn{FlatColumnMatrix}$ for $Mat(m,n,T)$ are the corresponding column-wise encodings.

We put $\cn{StandardMatrix}=\cn{RowMatrix}$.

\subsection{Mathematical Structures}

\paragraph{Finite Maps}
Assume codecs $c,d$ for types $S,T$.

The codec $MapAsList(c,d)$ encodes the map $f\in FiniteMap(S,T)$ as the list of pairs $[[c(s_1),d(f(s_1))],\ldots,[c(s_n),d(f(s_n))]]$ where the $s_i\in S$ are the pairwise distinct arguments for which $f$ is defined (in any order).

If $c$ is a string-codec, the codec $MapAsObject$ encodes $f$ as the object $\jobj{c(s_1):d(f(s_1)),\ldots,c(s_n):d(f(s_n))}$.

We put $\cn{StandardMap}=\cn{MapAsList}$.

\paragraph{Polynomials}
Assume a codec $c$ for the underlying type $T$ of a ring $r$.\footnote{For example, we can use $T=C$ and $c=\cn{StandardComplex}$ for all polynomials whose coefficients are chosen from a subset of the complex numbers.}

The codec $PolynomialAsSparseList(r,[x_1,\ldots x_n],c)$ encodes $p$ as the list $[\ldots,[c(a_{\vec{i}}),[\ov{i_1},\ldots,\ov{i_n}]],\ldots,]$.
The entries of that list occur in any order, and no $\vec{i}$ occurs twice.
In the special case $n=1$, encoding uses $\ov{i_1}$ instead of $[\ov{i_1}]$; decoding accepts both forms.

For $n=1$, the codec $UnaryPolynomialAsDenseList(r,[x],c)$ encodes $p=\Sigma_{i=0}^d$ as the list $[c(a_0),\ldots,c(a_n)]$.

\paragraph{Fields}
The codec $\cn{StandardField}(c)$ for the type $Field$ encodes fields as follows:
\begin{compactitem}
 \item base fields as the string $"Q"$, $"R"$, and $"C"$
 \item named fields as strings
 \item field extensions $FieldExtension(F,p,a)$ with $p\in Polynomial(F,[x])$ as $["extension", StandardField(F), \ov{p}, a]$ \ednote{FR: the choice of encoding for $p$ is not easy; here I'm assuming we always have a default encoding}
 \item field extensions $Q(p,a)$ as $[\ov{p},a]$ \ednote{FR: Do we need a special case for this? What encoding should it be?}
 \item $Qsqrt(n)$ as the pair $["Qsqrt", \ov{n}]$
 \item $Qzeta(n)$ as the pair $["Qzeta", \ov{n}]$
\end{compactitem}

\paragraph{Structure Elements}
A codec of the type of elements of a structure $S$ is the same as a codec for the underlying type of $S$.

\subsection{Subtyping Property}

Our standard codecs mirror the subtype relationship between the types.
If $S<: T$, then the elements of $S$ are encoded in the same way by the standard codecs for $S$ and $T$.

\ednote{FR: this remark could be promoted to a theorem}
