For each type, we define codecs. These are pairs of bijective maps between the elements of that type and subsets of JSON.

To encode/decode a mathematical object into/from JSON, we need to provide its type and a codec for that type.
(Different codecs might encode different mathematical objects as the same JSON. Thus, the meaning of an arbitrary JSON is only determined if the codec is known.)

For codec $C$ for type $T$ and a mathematical object $t$ of $T$, we write $C(t)$ for the encoding of $t$.
For some types, we define a \emph{default codec} (whose name will start with $\cn{Standard}$.
If a default codec $D$ exists, we write $\ov{t}$ for $D(t)$.

A codec $C$ is a \emph{string-codec} if encodes every value as a JSON string.
For codecs $C_i$, we write $\cunion{C_1,\ldots,C_n}$ for the codec that encodes using $C_1$ and decodes according to the first $C_i$ that is applicable to the input

\subsection{JSON}

We use the following JSON values:
\begin{compactitem}
  \item $null$
  \item 32-bit integers
  \item booleans
  \item strings
  \item lists $[j_1,\ldots,j_n]$ for JSON values $j_i$
  \item objects $\jobj{k_1:j_1,\ldots,k_nj_n}$ for strings $k_i$ and JSON values $j_i$
\end{compactitem}
We also say \emph{pair} for lists $[j,j']$ of length $2$.

\subsection{Base Types}

Our standard codecs for number types mirror the subtype relationship between the types.
For example, a natural number is encoded in the same way by the standard codecs for any number type.

\paragraph{Natural and Integer Numbers}
Natural numbers and integers are encoded in the same way.

We have the following encodings for an integer $n$, we

$\cn{IntAsNumber}$: the JSON integer $n$ if $|n|$ is small enough and like $\cn{IntAsString}$ otherwise

$\cn{IntAsString}$: a string containing the decimal expansion of $n$

$\cn{IntAsList}$: a list $["base",b,l,d_1,\ldots,d_{|l|}]$ where $b,l,d_i$ are JSON integers such that $\sgn l=\sgn n$ and $(d_1,\ldots,d_|l|)$ is the list of digits of $n$ relative to base $2^b$.\footnote{The value $l$ may seem redundant in the encoding (except for carrying the sign). But it has the advantage that the canonical order on integers corresponds to the lexicographic order of JSON lists.}

$\cn{StandardInt}$: the codec $\cunion{\cn{IntAsNumber},\cn{IntAsString},\cn{IntAsList}}$

\paragraph{Rational Numbers}
$\cn{StandardRat}$ encodes the rational number $e/d$ for integers $e,d$ as
\begin{compactitem}
 \item $\ov{e}$ if $d=1$
 \item the pair $[e,d]$
\end{compactitem}
For encodings, the $e$ and $d$ are fully canceled and $d>0$. For decoding, all pairs with $d\neq 0$ are accepted.
For decoding the string $"e'/d'"$ where $e'$ and $d'$ are the string encodings of $e$ and $d$ are also accepted.\ednote{FR: is this needed}

\paragraph{Real Numbers}
A real number can be given in one of the following forms:
\begin{compactitem}
 \item a rational number
 \item a root $\sqrt[n]{x}$ for a natural number $n$ and an integer $x$
 \item the strings "pi" and "e"
\end{compactitem}

The codec $\cn{StandardReal}$ encodes a real number as follows:
\begin{compactitem}
 \item rational numbers like $\cn{StandardRat}$
 \item $n\sqrt{x}$ as the list $["root",\ov{n},\ov{x}]$ for a natural number $n$ and an integer $x$
 \item the strings "pi" or "e"
\end{compactitem}

\paragraph{Complex Numbers}
A complex number can be given in one of the following forms:
\begin{compactitem}
 \item Cartesian form $x+yi$
 \item polar form $r e^{i\phi}$
 \item root of unity $\zeta_n$
\end{compactitem}

The codec $\cn{StandardComplex}$ encodes a complex number $z$ (in any form) as
\begin{compactitem}
 \item if $z$ is in Cartesian form:
  \begin{compactitem}
    \item $\ov{x}$ if $y=0$
    \item the object $\jobj{"re": \ov{x}, "im":\ov{y}}$ otherwise
  \end{compactitem}
 \item if $z$ is in polar form
  \begin{compactitem}
   \item the object $\jobj{"abs": \ov{r}, "unitarg": \ov{\alpha}}$ if $\phi=2\pi \alpha$ for $\alpha\in[0,1[$
   \item the object $\jobj{"abs": \ov{r}, "arg": \ov{\phi}}$ otherwise
  \end{compactitem}
 \item the object $["root-of-unity", \ov{n}]$ if $z$ is a root of unity
\end{compactitem}

\paragraph{Strings}
$\cn{StandardString}$ encodes strings as JSON strings.

\paragraph{Booleans}
$\cn{BooleanAsBoolean}$ encodes booleans as JSON boolean.

$\cn{BooleanAsInt}$ encodes booleans as $0$ or $1$.

$\cn{BooleanAsString}$ encodes booleans as the strings "true" or "false".

$\cn{StandardBoolean}$ is $\cunion{\cn{BooleanAsBoolean},\cn{BooleanAsInt},\cn{BooleanAsString}}$


\subsection{Aggregating Type Constructors}

Corresponding to type constructors $T$ that form types $T(T_1,\ldots,T_n)$ from existing types, we use codec constructors $C$ that form codecs $C(C_1,\ldots,C_n)$ for $T(T_1,\ldots,T_n)$ from codecs $C_i$ for $T_i$.

In the following, we assume codes $C_1,\ldots,C_n$ for the types $T_1,\ldots,T_n$.
We write $\vec{C}$ for $C_1,\ldots,C_n$.

\paragraph{Products}
The codec $\cn{StandardProduct}(\vec{C})$ for $T_1*\ldots*T_n$ encodes tuples $(t_1,\ldots,t_n)$ as JSON lists $[C_1(t_1),\ldots,C_n(t_n)]$.

\paragraph{Records}
The codec $\cn{StandardRecord}(\vec{C})$ for $\record{k_1:T_1,\ldots,k_n:T_n}$ encodes records $\record{k_1=t_1,\ldots,k_n=t_n}$ as JSON objects $\jobj{"k_1":C_1(t_1),\ldots,"k_n":C_n(t_n)}$.

\paragraph{Unions}
The codec $\cn{StandardUnion}(\vec{C})$ for $\record{T_1+\ldots T_n}$ encodes values $t$ of $T_i$ as the JSON pair $[i,C_i(t)]$.

\paragraph{Labeled Unions}
The codec $\cn{StandardLabeledUnion}(\vec{C})$ for $\union{k_1:T_1,\ldots,k_n:T_n}$ encodes values $k_i(t)$ for $t\in T_i$ as the JSON object $\jobj{"k_i":C_i(t_i)}$.

\subsection{Collecting Type Constructors}

We assume a codec $C$ for a type $T$.

\paragraph{Options}
The codec $\cn{StandardOption}(C)$ for $Opt(T)$ encodes values $t\in T$ as $C(t)$ and omitted values as the $null$ value.

\paragraph{Vectors}
The codec $\cn{StandardVector}(n,C)$ for $Vec(n,T)$ encodes vectors $(t_1,\ldots,t_n)$ as the JSON list $[C(t_1),\ldots,C(t_n)]$.

\paragraph{Lists}
The codec $\cn{StandardList}(C)$ encodes lists in the same way as $\cn{StandardVector}$.

If $C$ encodes 
The codec $\cn{StandardList}(C)$ encodes lists in the same way as $\cn{StandardVector}$.


\paragraph{Sets}
The codec $\cn{StandardSet}(C)$ encodes sets in the same way as $\cn{StandardVector}$, where the elements are listed in any order but without repetitions.

\paragraph{Multisets}
The codec $\cn{StandardMultiset}(C)$ encodes multisets as lists $[[C(t_1),\ov{m_1}], \ldots, [C(t_n),\ov{m_n}]]$ of pairs $(t,m)$ where $t\in T$ is an element of the multiset and with multiplicity $m\in Nat$.
The order of pairs is not specified.
For encoding, the same $t_i$ will not occur twice, and $m_i\neq 0$. For decoding, these cases are accepted.

\paragraph{Hybrid Sets}
The codec $\cn{StandardHypbridSet}(C)$ encodes hybrid sets in the same way as $\cn{StandardMultiset}$ except that the multiplicities may be negative.

\paragraph{Matrices}
The codec $\cn{RowMatrix}(m,n,C)$ for $Mat(m,n,T)$ encodes matrices $(t_{ij})$ as the list of lists $[[C(t_{11}),\ldots,C(t_{1n})], \ldots, [C(t_{m1}),\ldots,C(t_{mn})]$.

The codec $\cn{ColumnMatrix}(m,n,C)$ for $Mat(m,n,T)$ is the corresponding column-wise encoding.

We put $\cn{StandardMatrix}=\cn{RowMatrix}$.

\paragraph{Two-dimensional List}
The codecs $\cn{RowList2}$, $\cn{ColumnList2}$, and $\cn{StandardList2}$ encode values of type $List2(T)$ in the same way as $\cn{RowMatrix}$, $\cn{ColumnMatrix}$, and $\cn{StandardMatrix}$.

\subsection{Mathematical Structures}

\paragraph{Finite Maps}
Assume codecs $C,D$ for types $S,T$.

The codec $MapAsList(C,D)$ encodes the map $f\in FiniteMap(S,T)$ as the list of pairs $[[C(s_1),D(f(s_1))],\ldots,[C(s_n),D(f(s_n))]]$ where the $s_i\in S$ are the pairwise distinct arguments for which $f$ is defined (in any order).

If $C$ is a string-codec, the codec $MapAsObject$ encodes $f$ as the object $\jobj{C(s_1):D(f(s_1)),\ldots,C(s_n):D(f(s_n))}$.

We put $\cn{StandardMap}=\cn{MapAsList}$.

\paragraph{Polynomials}
Assume a codec $C$ for the underlying type $T$ of a ring $R$.\footnote{For example, we can use $T=Complex$ and $C=\cn{StandardComplex}$ for all polynomials whose coefficients are chosen from a subset of the complex numbers.}

Consider polynomials of the form $p=\Sigma_{\vec{i}\in Nat^n} a_{\vec{i}} \vec{x}^{\vec{i}}\in Polynomial(R,[x_1,\ldots,x_n])$ where $(x_1,\ldots,x^n)^{(i_1,\ldots,i_n)}$ abbreviates $x_1^{i_1}\cdot \ldots\cdot x_n^{i_n}$.
The codec $PolynomialAsSparseList(R,[x_1,\ldots x_n],C)$ encodes $p$ as the list $[\ldots,[C(a_{\vec{i}}),[\ov{i_1},\ldots,\ov{i_n}]],\ldots,]$.
The entries of that list occur in any order, and no $\vec{i}$ occurs twice.
In the special case $n=1$, encoding uses $\ov{i_1}$ instead of $[\ov{i_1}]$; decoding accepts both forms.

For $n=1$, the codec $UnaryPolynomialAsDenseList(R,[x],C)$ encodes $p=\Sigma_{i=0}^d$ as the list $[C(a_0),\ldots,C(a_n)]$.

\paragraph{Fields}
The following forms of fields are supported:\ednote{@John: Fields are the most difficult type so far. Please check if this in particular.}
\begin{compactitem}
 \item base fields $Rat$, $Real$, and $Complex$
 \item named fields identified by a string
 \item polynomial field extensions $Rat(p)$ of $Rat$ for a polynomial $p\in Polynomial(Rat,[x])$ (for any variable name $x$)
 \item extensions of $Rat$ with $\sqrt n$ written as $Qsqrt(n)$
 \item extensions of $Rat$ with a root $\zeta_n$ of unity written as $Qzeta(n)$
\end{compactitem}

Given a codec $C$ for $Polynomial(Rat,[x])$, the codec $\cn{StandardField}(C)$ for the type $Field$ encodes fields as follows:\footnote{Note that is different from codecs for encoding the elements of an individual fields.}
\begin{compactitem}
 \item base fields as the string $"Rat"$, $"Real"$, and $"Complex"$
 \item named fields as strings
 \item field extensions $Rat(p)$ as $C(p)$
 \item $Qsqrt(n)$ as the pair $["Qsqrt", \ov{n}]$
 \item $Qzeta(n)$ as the pair $["Qzeta", \ov{n}]$
\end{compactitem}
