% !TEX root = ../thesis.tex

The mathematical software systems to be integrated via the MitM approach have so far been computation-oriented, e.g., computer algebra systems.
Their API theories typically declare types and functions on these types (the latter including constants seen as nullary functions).
Even though database systems differ drastically from these in many respects, they are very similar at the MitM level: a database like \lmfdb defines
\begin{compactitem}
 \item some types: each table's schema is essentially one type definition,
 \item many constants: each table entry is one constant of the corresponding type.
\end{compactitem}
Thus, we can apply essentially the same approach.
In particular, the API theories must contain definitions of the database schemas. 

From a system perspective, virtual theories behave just like concrete theories, but without the assumption of being able to load all declarations from some file on disk at once.
Instead, virtual theories load declarations in a lazy fashion when they are needed. 
\mmt stores concrete theories as XML files.
Because most external knowledge bases use databases with low-level APIs, we must allow virtual theories to be stored in external database.
Apart from standard software engineering tasks, this leaves three conceptual problems we had to solve:
\begin{compactenum}[\bf P1]
\item Turn the database schemas and tables into \ommt theories and declarations. 
\item Lift data in \emph{physical} representation (as records of the
  underlying database) to \ommt object in \emph{semantic} representation.
\item Translate semantic queries to queries about physical representations so
  that they can be executed directly on the database without loading the entire theory into
  \mmt.
\end{compactenum}
We deal with \textbf{P1} here, with \textbf{P2} in Section~\ref{sec:access}, and with \textbf{P3} in Section~\ref{sec:qmt}. 

\begin{figure}[ht]\centering
    \begingroup
    \pgfdeclarelayer{background}
    \pgfdeclarelayer{foreground}
    \pgfsetlayers{background,foreground}
    
    \resizebox{\textwidth}{0.75\textwidth}{
      \begin{tikzpicture}[xscale=4,yscale=2.2]\footnotesize
        \begin{pgfonlayer}{foreground}
          \tikzstyle{human}    = [red,dashed,thick]
          \tikzstyle{withshadow}  = [draw,drop shadow={opacity=.5},fill=white]
          \tikzstyle{interface}   = [fill=blue!30]
          \tikzstyle{database}    = [cylinder,cylinder uses custom fill,
            cylinder body fill=yellow!50,cylinder end fill=yellow!50,
            shape border rotate=90,
            aspect=0.25,draw]
          
          % Ontology layer
          \node[thy] (numbers) at (0,1) {
            \begin{tabular}{lll}
              \multicolumn{3}{l}{\textsf{Numbers}}\\\hline\hline
              $\mathbb{Z}^{+}$        & : & \typett\\
              $\mathbb{Z}$            & : & \typett\\\hline
              \multicolumn{3}{l}{$\mathbb{Z}^{+} \subset \mathbb{Z}$}
            \end{tabular}
          };

          \node[thy] (matrices) at (1.5,1) {
            \begin{tabular}{lll}
              \multicolumn{3}{l}{\textsf{Matrices}}\\\hline\hline
              \plaintt{matrix} & : & $\typett \rightarrow \mathbb{Z}^{+}\rightarrow \mathbb{Z}^{+} \rightarrow \typett$
            \end{tabular}
          };

          \node[thy] (codecs) at (0.75,0) {
            \begin{tabular}{lll}
              \multicolumn{3}{l}{\textsf{Codecs}}\\\hline\hline
              \codectt                  & : & $\typett \rightarrow \typett$\\\hline
              \plaintt{standardInt}     & : & $\codectt\; \mathbb{Z}$\\
              \plaintt{standardMatrix}  & : & $\left\{T, n, m\right\} \codectt\; T \rightarrow \codectt\; \plaintt{matrix}(n, m, T)$\\
            \end{tabular}
          };

          \draw[include] (numbers) -- (matrices);
          \draw[include] (matrices) -- (codecs);
          
          \begin{pgfonlayer}{background}
            \node[draw=none,fill=green!30,rounded corners=1cm,fit=(numbers) (matrices) (codecs),inner sep=10pt] {};
          \end{pgfonlayer}
        
          % Model Layer
          \node[thy,fill=purple!30] (ec) at (2.25,-1.20) {
            \begin{tabular}{lll}
              \multicolumn{3}{l}{\textsf{Elliptic Curve}}\\\hline\hline
              \plaintt{ec}            & : & \typett\\\hline
              \plaintt{from\_record}  & : & $\plaintt{record} \rightarrow \plaintt{ec}$ \\\hline
              \plaintt{curveDegree}   & : & $\plaintt{ec} \rightarrow \mathbb{Z}$ \\
              \plaintt{isogenyMatrix} & : & $\plaintt{ec} \rightarrow \plaintt{matrix}(3, 3, \mathbb{Z})$ 
            \end{tabular}
          };

          \node[thy,interface] (ecschema) at (2.0,-2.5) {
            \begin{tabular}{lll}
              \multicolumn{3}{l}{\textsf{Elliptic Curve Schema Theory}}\\\hline\hline
              $\plaintt{degree}$            & \uri{?implements}  & \plaintt{curveDegree} \\
                                            & \uri{?codec}       & \plaintt{StandardInt} \\\hline
              $\plaintt{isogeny\_matrix}$   & \uri{?implements}  & \plaintt{isogenyMatrix} \\
                                            & \uri{?codec}       & $\plaintt{StandardMatrix}(3, 3, \plaintt{StandardInt})$ 
            \end{tabular}
          };

          % Database Layer
          \node[database] (mongodb) at (-.5,-2.5) {
            \textsf{\lmfdb Elliptic Curves}
          };

          \node[thy,interface] (dbtheory) at (0,-1.20) {
            \begin{tabular}{lllll}
              \multicolumn{5}{l}{\textsf{Elliptic Curve Database Theory}}\\\hline\hline
              \plaintt{11a1} & : & $\plaintt{ec}$ & $=$ & \dots\\
              \plaintt{11a2} & : & $\plaintt{ec}$ & $=$ & \dots\\
              \dots
            \end{tabular}
          };
          \draw[include] (matrices) to[bend left=20] (ec);
          \draw[include] (ec) -- (dbtheory);
          
          \draw[human,->] (dbtheory) -- node[right]{\scriptsize {lazily loads from}} (mongodb);
          \draw[human,->] (ecschema) -- node[right]{\scriptsize {implements}} (ec);
          \draw[human,->] (ecschema) -- node[above]{\scriptsize {describes}} (mongodb);
        \end{pgfonlayer}
      \end{tikzpicture}
    }
    \endgroup
  \caption[Virtual Theory Architecture]{
    Virtual theory for \lmfdb elliptic curves (some declarations omitted) 
  }
  \label{fig:vtarch}
\end{figure}
A sketch of our overall solution is given in Figure~\ref{fig:vtarch}.
The math in the middle comprises preexisting formalizations of general mathematics, here numbers and matrices (in green), and novel \lmfdb-specific ones, here elliptic curves (in red).
Moreover, we introduce a specification of various codecs to translate between physical and semantic representations.
The remaining theories (in blue) form the \lmfdb API theories: the schema theory and the database theory, which we describe below.

The set of constants in a database table -- while finite -- can be arbitrarily large.
In particular, all \lmfdb tables\footnote{Technically, until July 2018,
  \lmfdb was implemented using MongoDB and comprises a set of sets
  (each one called a database) of JSON objects.  MongoDB allows each
  JSON object in a collection to be different (with a different
  schema), though in practice almost all objects in each collection
  had the same schema apart from some missing data components.
  The schema for each collection had to be documented elsewhere, in an
  inventory, which since 2017 has been itself stored as a database
  within the LMFDB.  During 2018, however, work has been ongoing to
  migrate the LMFDB to use PostgreSQL (with a fixed schema for each
  table) as the underlying database, without any change to the extrnal
  API.  In both cases, due to the conventions used, we can understand
  the LMFDB conceptually as a set of tables of a relational database,
  keeping in mind that every row is a tuple of arbitrary JSON
  objects.}
are just finite subsets of infinite sets, whose size is not limited by mathematical specifications but by computational power: the database holds all objects that users have computed so far and grows constantly as more objects are computed.
\lmfdb tables usually include a naming system that defines unique identifiers (which are used as the database keys) for these objects, and these identifiers are predetermined even for those objects that have not been computed yet.
Thus, it is not practical to fix a set of concrete API theories.
Instead, the API theories must be split into two parts: for each database table, we need
\begin{compactitem}
  \item a concrete theory called the \textbf{schema theory} that defines the schema  and other relevant information about the type of objects in the table and
  \item a virtual theory called the \textbf{database theory} that contains one definition for each value of that type (using the \lmfdb identifier as the name of the defined constant). 
\end{compactitem}

%% \begin{oldpart}{MK@JC: you may want to reformulate this
%%     paragraph. Here would also be the right place to describe the
%%     development of the inventory and how this was triggered by ODK and
%%     helpful in the migration to Postgres. ... }
 \lmfdb's original
  technical realization, using MongoDB, did not require formalizing
  the schema of each table.  Instead, the tables were generated
  systematically and therefore followed an implicit schema that could
  -- in principle -- be obtained from the documentation or
  reverse-engineered from the tables.  Until 2017 the documentation of
  these implicit schemas was created and maintained manually by LMFDB
  developers, and as a result was incomplete and frequently out of
  date.  During 2017-2018 a new LMFDB Inventory was created, taking as
  its starting point both the manually prepared inventory (which
  contained human definitions and explanations of the content of each
  data field) and a new dynamically created schema obtained by
  analysis of the data in each collection.  This process, which was
  both necessitated by the requirements of the Math-in-the-Middle
  approach and made possible in practice through the provision of
  research software engineers funded by ODK, revealed numerous
  inconsistencies in the LMFDB data which developers have since been
  able to address.  Moreover, having this detailed schema for each
  collection in the MongoDB databases also fed in to the migration
  process, expected to be complete by August 2018, in which the
  MongoDB free-format collections are being replaced by PostgreSQL
  tables each of which has a completely specified and formalised
  schema.  Note that (and here \lmfdb critically differs from,
  e.g. the OEIS), the mathematical definitions and concepts involved
  in the LMFDB data and tables is extremely deep, so that
  reverse-engineering the associated schemas from the data itself is
  only possible in practice with the aid of experts.  As the first
  such table for which a formal schema was to be created, before the
  development of the new comprehensive Inventory, we chose one for which
  the existing documentation was most complete, and which originated
  with one of the current authors (John Cremona) who sat down with the
  Math-in-the-Middle team at an ODK workshop to formalize the
  corresponding schema in \ommt.
%% \end{oldpart}

In the following, we will use this table as a running example.
Our methods extend immediately to any other table once its schema has been formalized.

Our formalization models elliptic curves in a very simple fashion by using an abstract type \identifier{ec}. 
The constructor \identifier{from\_record} takes an \mmt record and returns an elliptic curve. 
Properties of elliptic curves are formalized as functions out of this type.
We list only two here as examples: the \textsf{degree}, an integer, and the \textsf{isogeny matrix}, a $3 \times 3$ matrix of integers.
We omit the relevant axioms, which are not essential for our purposes here.
Recall that the Math-in-the-Middle approach models mathematical knowledge ``in the middle'' independent of any particular system.
This is exactly the case here -- the model of elliptic curves does not rely on \lmfdb, nor any other system, so that we can integrate other knowledge sources about elliptic curves or to future versions of the \lmfdb with changed structure. 


%%% Local Variables:
%%% mode: latex 
%%% mode: visual-line
%%% fill-column: 5000
%%% TeX-master: "report"
%%% End:

%  LocalWords:  compactitem oldpart realization formalizing lmfdb sec:vt ommt textbf ec emph 4,yscale 50,cylinder 0.25,draw 30,rounded 1cm,fit formalizations
%  LocalWords:  centering begingroup pgfdeclarelayer pgfsetlayers background,foreground
%  LocalWords:  resizebox textwidth textwidth tikzpicture xscale 4,yscale pgfonlayer
%  LocalWords:  tikzstyle red,dashed,thick withshadow draw,drop cylinder,cylinder hline
%  LocalWords:  50,cylinder 0.25,draw hline mathbb plaintt rightarrow rightarrow codectt
%  LocalWords:  rightarrow none,fill 30,rounded 1cm,fit thy,fill isogenyMatrix ecschema
%  LocalWords:  thy,interface isogeny mongodb dbtheory endgroup fig:vtarch sec:access
%  LocalWords:  sec:qmt colored colored colored colored formalized textit
