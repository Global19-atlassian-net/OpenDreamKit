There is a large and vibrant ecosystem of open-source software systems for mathematics. 
These range from calculators, which perform simple computations, via mathematical databases, which curate collections of mathematical objects, to powerful modeling tools and computer algebra systems (CAS).

These systems can be very specific, often focusing on a narrow area of mathematics.
For example, among databases, the ``Online Encyclopedia of Integer Sequences'' (OEIS) focuses on sequences over $\mathbb{Z}$ and their properties, and the ``L-Functions and Modular Forms Database'' (\LMFDB)~\cite{Cremona:LMFDB16,lmfdb:on} on objects in number theory pertaining to Langland's program.
Among CAS, \GAP~\cite{GAP:on} excels at discrete algebra with a focus on group theory, \Singular~\cite{singular:on} focuses on polynomial computations with special emphasis on commutative and non-commutative algebra, algebraic geometry, and singularity theory.
Finally, \Sage~\cite{SageMath:on} aims to be a general purpose software for computational pure mathematics by integrating many systems including the aforementioned ones, together with a large body of code for the {\Sage} library itself written in Python.

For a mathematician, however, (a user, which we call Jane) the systems themselves are not relevant.
Instead, she only cares about being able to solve problems. 
Because it is typically not possible to solve a mathematical problem within a single system, Jane has to work with multiple systems and combine the results to reach a solution. 
Currently there is very little tool support for this in practice; so Jane has to isolate sub-problems that the respective systems are amenable to, formulate them in the respective input language, collect intermediate results, and reformulate them for the next system --- a tedious and error-prone process at best, a significant impediment to scientific progress at worst.
Solutions for some situations certainly exist, which can help get Jane unstuck, but these are ad-hoc and only for specific often-used system combinations. 
Moreover, each of these ad hoc solutions requires a lot of maintenance and scales badly to
multi-system integration.
To add insult to injury, the knowledge bases Jane would like to use --- ranging from Wikipedia to theorem prover libraries --- are usually only accessible via the restricted API of a dedicated web information system or the low-level API to the raw database content.  
What we would want is a ``programmatic, mathematical API'' which would give access to the knowledge-bases programmatically via their mathematical constructions and properties.

This type of system integration has been tried before in middleware-based approaches like CORBA~\cite{Omg:Corba:Web}, which uses object-oriented modeling and stub generation for the exchange of business objects, REST-based Web Services~\cite{WSDL:on,Mitra:soapPrimer03} that exchange XML or JSON-based objects, or more recently Protocol Buffers~\cite{protobuffers:on}, which model structured data as JSON-like text files to name just three of many.
These middleware approaches allow to integrate disparate systems across machine, programming language, and even operating system borders, and work well, if the respective object models are compatible, which the respective communities try to achieve by standardizing domain ontologies and schemata. This schema standardization and mediation between differing object models turns out to be the main problem involved in interoperability.

In the domain of mathematics, domain model standardization is inherently difficult: The set of potential object types is essentially infinite and different but equivalent constructions are a feature systematically used to understand the objects.
There are essentially two approaches to tackling the problem of system interoperability and integration.
The first -- we call it the \textbf{ad-hoc-integration} approach -- uses a programming language for translating object data structures between systems, possibly using shared memory between systems. The most prominent proponent is the \Sage system, which uses Python as the glue language and various Python-to-X bridges for Master/Slave-type integration.
Here, the ``preservation of semantics'' property that is so important for a successful (i.e. sound and efficient) system integration is embedded in the glue code and has to be painstakingly maintained by the community over system releases. 

The second approach uses a standard language for representing mathematical objects in terms of mathematical formulae. It is used in OpenMath~\cite{BusCapCar:2oms04} and  (content) MathML~\cite{CarlisleEd:MathML3:base}.
Their underlying object models are isomorphic ~\cite{KohRab:som12}: mathematical objects are represented as application and binding expressions from symbols and variables\footnote{and a couple of basic programming-inspired data types like numbers, strings, and byte arrays.}.
Here the simplicity in the object schema is achieved by an open-ended set of symbols, which are characterized by a name and a reference to an OpenMath ``Content Dictionary'' (CD), i.e. a structured document that defines the meaning of the symbol.
Again, interoperability hinges on the availability and standardization of the CDs, which ensure that protocol-based distribution of expressions preserves their meaning.
In the past, the main problem of adoption of OpenMath/cMathML has been the lack of high-impact CDs, which had to be manually encoded by the respective community.


\paragraph{Towards a Virtual Research Environment for Mathematics}
One goal of the OpenDreamKit project is tackling these problems systematically by building virtual research environments (VRE) on top of the existing systems.
To build a VRE from individual systems, we need a joint user interface --- the OpenDreamKit project adopts Jupyter~\cite{jupyter-project:on} and active documents~\cite{KohDavGin:psewads11} --- and an interoperability layer that allows passing problems and results between the disparate systems.
For the latter, it proposes the Math-in-the-Middle paradigm (MitM~\cite{DehKohKon:iop16}): an interoperability framework based on a central, system-independent ontology of mathematical knowledge and system API theories that specify the interfaces of the various systems in the same, modular knowledge representation format.
We use \ommt as this format together with alignments the describe the relations between the functions declared in the ontology and those realized in the various system APIs.

\paragraph{Contribution}
In this \papertype we instantiate the MitM paradigm in two concrete case studies.
In the first one, we show distributed computation involving the \GAP, \Sage, and \Singular systems.
In the second one, we show the integration of the mathematical knowledge base \LMFDB into MitM-based computation.\ednote{maybe preview contributions of the MitM paradigm here}

%\begin{oldpart}{FR: moved here from the LMFDb paper; if we want to use this text, it should be integrated here somewhere}
%This \papertype takes a step into this direction by interpreting large knowledge bases as \ommt theories -- modular representations of mathematical objects and their properties. 
%For this, we generalize \ommt theories to ``virtual theories'' -- theories so big that they do not fit into main memory -- and update its knowledge management algorithms so that they can work directly with objects stored in external knowledge bases.
%An additional technical contribution is the introduction of a codec system that bridges between low-level encodings in databases and the abstract construction of mathematical objects.
%
%Our diagnosis is that {\lmfdb} -- and most other mathematical knowledge databases -- suffer from two problems:
%\begin{compactitem}
%\item \emph{human/computer mismatch}: humans have problems interacting with \lmfdb programmatically, because they must speak the system language instead of mathematical language
%\item \emph{computer/computer mismatch}: mathematical computer systems cannot interoperate with \lmfdb without extending their code, because their system languages differ.
%\end{compactitem}
%Using the MitM approach, we can solve both problems at the same time by lifting the communication to the level of \ommt-encoded MitM objects, which both MitM-compatible software systems and humans can understand.
%\end{oldpart}

For the CAS case study, we will use the following running example from computational group theory: Jane wants to experiment with invariant theory of finite groups.
She works in the polynomial ring $R=\mathbb{Z}[X_1,\ldots,X_n]$, and wants to construct an ideal $I$ in this ring that is fixed by a group $G \leq S_n$ acting on the variables, linking properties of the group to properties of $I$ and the quotient of $R$ by $I$.

To construct an ideal that is invariant
under the group action, it is natural to pick some polynomial $p$ from $R$
and consider the ideal $I$ of $R$ that is generated by all elements of the orbit
$O=Orbit(G,R,p)\subseteq R$.
For effective further computation with $I$, she needs a G\"obner base of $I$.

Jane is a \Sage user and wants to receive the result in \Sage, but she wants to
use \GAP's orbit algorithm and \Singular's Gr\"obner base algorithm, which she
knows to be very efficient. For the sake of example, we will work with $n=4$,
$G=D_4$ (the dihedral group\footnote{Incidentally, this group is called $D_4$ in
\Sage but $D_8$ in \GAP due to differing conventions in different mathematical
communities -- a small example of the obstacles to system interoperability that
MitM tackles.}), and $p=3\cdot X_1+2\cdot X_2$, but our results apply to
arbitrary values.

\def\Q{{\mathbb Q}}
\def\N{{\mathcal{N}}}
For the \LMFDB case study, Jane wants to investigate the number fields which are generated by the coefficients of Hilbert modular forms (HMFs).
For many totally real number fields~$F$ of low degree, and for many levels~$\N$\ for each field, the LMFDB contains information about all HMFs of level~$\N$ (of parallel weight~$2$ and trivial character).
Each of these HMFs is an eigenform for the Hecke algebra with eigenvalues generating a number field; the same number field contains the coefficients of the standard Fourier expansion of the HMF, which are expressible in terms of the eigenvalues.
In the LMFDB, each HMF's Hecke field $K$ is stored by means of a defining polynomial which has been obtained as a by-product of the computation of the HMF itself, and is in no way canonical or minimal, making study of these fields difficult and --- even in simple cases --- obscure.
For example, the Hecke field $K=\Q(\sqrt{2})$ may occur for more than one HMF, defined by the polynomial $x^2-2$ for some and by the polynomial $x^2-2x-1$ for others.
Hence Jane would like to be able to extract these defining polynomials from the LMFDB, use them to define number fields in \Sage, find simpler polynomials defining the same fields, and study their arithmetic properties (for example, their class numbers).
To this end, some of the Hecke fields may themselves be in the LMFDB's collection of number fields, in which case the information about them which Jane needs is already computed and stored, but this will in most cases be hidden since the defining polynomials used in the HMF database will often not be the one stored in the number fields database.

\paragraph{Overview}
In Section~\ref{sec:mitm}, we recap the MitM paradigm.
%MitM solutions consist of three parts: a central ontology, specifications of the abstract languages of the involved systems, and the distributed computation infrastructure that connects the systems via the ontology as an intermediate representation.
%The rest of the \papertype develops these three parts for our case study:
In Section~\ref{sec:cgt}, we describe the MitM ontology.
While the ontology describes the \emph{abstract} syntax of mathematical objects, Section~\ref{sec:codecs} introduces a codec framework for describing their \emph{concrete} syntax in different systems.
In Section~\ref{sec:apit} and~\ref{sec:databases}, we describe the integration of the computation systems \GAP, \Sage, and \Singular and the \lmfdb databases with the MitM architecture.
In Section~\ref{sec:case}, we present the resulting virtual research environment built on
these systems in action.
Section~\ref{sec:concl} concludes the paper.

%%% Local Variables:
%%% mode: latex
%%% TeX-master: "report"
%%% End:

%  LocalWords:  sec:intro Sloane:oeis12,oeis mathbb Cremona:LMFDB16,lmfdb:on GAP:on \Sage:on singular:on DehKohKon:iop16 emph sec:mitm sec:cgt sec:apit sec:mitm_poc sec:concl MitM-based sec:case Jupyter jupyter-project:on KohDavGin:psewads11 ldots,X_n leq obne cdot cdot WSDL:on,Mitra:soapPrimer03 protobuffers:on standardizing standardization KohRab:som12 characterized textbf
%  LocalWords:  SageMath:on ldots,X_4 subseteq obner Groebner formalizations ednote ommt
%  LocalWords:  realized papertype oldpart lmfdb compactitem mathcal eigenform
