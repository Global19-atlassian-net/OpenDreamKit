\section{Introduction}\label{sec:intro}

There is a large and vibrant ecosystem of open-source software systems for mathematics. 
These range from calculators, which perform simple computations, via mathematical databases, which curate collections of mathematical objects, to powerful modeling tools and computer algebra systems (CAS).

These systems can be very specific, focusing on a narrow area of mathematics.  For example, among databases, the ``Online Encyclopedia of Integer Sequences'' (OEIS) focuses on sequences over $\mathbb{Z}$ and their properties, and the ``L-Functions and Modular Forms Database'' (\LMFDB)~\cite{Cremona:LMFDB16,lmfdb:on} on objects in number theory pertaining to Langland's program.  Among CAS, \GAP~\cite{GAP:on} excels at discrete algebra with a focus on group theory, \Singular~\cite{singular:on} focuses on polynomial computations with special emphasis on commutative and non-commutative algebra, algebraic geometry, and singularity theory. Finally \Sage~\cite{SageMath:on} aims to be a general purpose software for computational pure mathematics by integrating many systems including the aforementioned ones, together with a large body of code for the \Sage\ library itself written in Python.

For a mathematician, however, (a user, which we call Jane) the systems themselves are not relevant.
Instead, she only cares about being able to solve problems. 
Because it is typically not possible to solve a mathematical problem within a single system, Jane has to work with multiple systems and combine the results to reach a solution. 
Currently there is little tool support for this in practice; so Jane has to isolate sub-problems that the respective systems are amenable to, formulate them in the respective input language, collect intermediate results, and reformulate them for the next system -- a tedious and error-prone process at best, a significant impediment to scientific progress at worst.
Solutions for some situations certainly exist, which can help get Jane unstuck, but these are ad-hoc and only for specific often-used system combinations. 
Moreover, each of these ad hoc solutions requires a lot of maintenance and scales badly to
multi-system integration.
\begin{newpart}{MK: copied here}
  To add insult to injury, the knowledge-bases Jane would like to use-- ranging from Wikipedia to theorem prover libraries -- are usually only accessible via a dedicated web information system or a low-level API at the level of the raw database content.  
What we would want is a ``programmatic, mathematical API'' which would give access to the knowledge-bases programmatically via their mathematical constructions and properties.
\end{newpart}

\ednote{NT: one should say something about previous attempts, e.g. OpenMath, why it does not do the job, and how we mean to fix that.}

\paragraph{Towards a Virtual Research Environment for Mathematics} One goal of the OpenDreamKit project is tackling these problems systematically by building virtual research environments (VRE) on top of the existing systems.  To build a VRE from individual systems, we need a joint user interface -- the OpenDreamKit project adopts Jupyter~\cite{jupyter-project:on} and active documents~\cite{KohDavGin:psewads11} -- and an interoperability layer that allows passing problems and results between the disparate systems.  For the latter, it proposes the Math-in-the-Middle (MitM~\cite{DehKohKon:iop16}) paradigm, an interoperability framework based on a central, system-independent ontology of mathematical knowledge and system API theories that specify the interfaces of the various systems in the same, modular knowledge representation format: OMDoc/MMT aligned with the MitM ontology by special representations of mathematical equivalence called \textbf{alignments}.\ednote{MK: rework this sentence}.

\paragraph{Contribution}
In this \papertype we instantiate the MitM paradigm in two concrete case studies. In the first we show distributed computation involving the \GAP, \Sage, and \Singular systems; in the second we show the integration of the mathematical knowledge base \LMFDB into MitM-based computation.\ednote{MK: preview the contribution of the MitM paradigm here.}

\begin{newpart}{MK: for the second part}
    This \papertype takes a step into this direction by interpreting large knowledge bases as \ommt theories -- modular representations of mathematical objects and their properties. 
  For this, we generalize \ommt theories to ``virtual theories'' -- theories so big that they do not fit into main memory -- and update its knowledge management algorithms so that they can work directly with objects stored in external knowledge bases.
  An additional technical contribution is the introduction of a codec system that bridges between low-level encodings in databases and the abstract construction of mathematical objects.
\end{newpart}

\paragraph{Case Studies and Running Examples}
For the CAS case study, we will use the following running example from computational group theory: Jane wants to experiment with invariant theory of finite groups.
She works in the polynomial ring $R=\mathbb{Z}[X_1,\ldots,X_n]$, and wants to construct an ideal $I$ in this ring that is fixed by a group $G \leq S_n$ acting on the variables, linking properties of the group to properties of $I$ and the quotient of $R$ by $I$.

To construct an ideal that is invariant
under the group action, it is natural to pick some polynomial $p$ from $R$
and consider the ideal $I$ of $R$ that is generated by all elements of the orbit
$O=Orbit(G,R,p)\subseteq R$.
For effective further computation with $I$, she needs a G\"obner base of $I$.

Jane is a \Sage user and wants to receive the result in \Sage, but she wants to
use \GAP's orbit algorithm and \Singular's Gr\"obner base algorithm, which she
knows to be very efficient. For the sake of example, we will work with $n=4$,
$G=D_4$ (the dihedral group\footnote{Incidentally, this group is called $D_4$ in
\Sage but $D_8$ in \GAP due to differing conventions in different mathematical
communities -- a small example of the obstacles to system interoperability that
MitM tackles.}), and $p=3\cdot X_1+2\cdot X_2$, but our results apply to
arbitrary values.

\def\Q{{\mathbb Q}} \def\N{{\mathcal{N}}} For the \LMFDB case study, Jane wants to investigate the number fields which are generated by the coefficients of Hilbert modular forms (HMFs).  For many totally real number fields~$F$ of low degree, and for many levels~$\N$\ for each field, the LMFDB contains information about all HMFs of level~$\N$ (of parallel weight~$2$ and trivial character).  Each of these HMFs is an eigenform for the Hecke algebra with eigenvalues generating a number field; the same number field contains the coefficients of the standard Fourier expansion of the HMF, which are expressable in terms of the eigenvalues.  In the LMFDB, each HMF's Hecke field $K$ is stored by means of a defining polynomial which has been obtained as a by-product of the computation of the HMF itself, and is in no way canonical or minimal, making study of these fields difficult and -- even in simple cases -- obscure.  For example, the Hecke field $K=\Q(\sqrt{2})$ may occur for more than one HMF, defined by the polynomial $x^2-2$ for some and by the polynomial $x^2-2x-1$ for others.  Hence Jane would like to be able to extract these defining polynomials from the LMFDB, use them to define number fields in \Sage, find simpler polynomials defining the same fields, and study their arithmetic properties (for example, their class numbers).  To this end, some of the Hecke fields may themselves be in the LMFDB's collection of number fields, in which case the information about them which Jane needs is already computed and stored, but this will in most cases be hidden since the defining polynomials used in the HMF database will often not be the one stored in the number fields database.

\begin{newpart}{MK: rework}
\paragraph{Overview}
In Section~\ref{sec:mitm}, we recap the MitM paradigm.  MitM solutions consist of three
parts: a central ontology, specifications of the abstract languages of the involved
systems (which we call \emph{system dialects}), and the distributed computation
infrastructure that connects the systems via the ontology as an intermediate
representation.  The rest of the \papertype develops these three parts for our case study:
In Section~\ref{sec:cgt}, we contribute a fragment to the MitM ontology that formalizes
computational group theory.  In Section~\ref{sec:apit}, we specify the abstract languages
of \GAP, \Sage, and \Singular and their relation to the ontology.  Finally in
Section~\ref{sec:case}, we present the resulting virtual research environment built on
these systems in action. Section~\ref{sec:sota} discusses \lmfdb and its interface as an
example of a very large state-of-the-art mathematical knowledge base, and
Section~\ref{sec:vt} shows how it can be represented as a set of virtual theories.
Section~\ref{sec:access} introduces the codec architecture and describes how to access
virtual theories at the semantic/mathematical level, and Section~\ref{sec:qmt} makes QMT
queries aware of virtual theories.  Section~\ref{sec:concl} concludes the paper.


Section~\ref{sec:concl} concludes the \papertype and compares MitM-based interoperability with other approaches.
\end{newpart}

%%% Local Variables:
%%% mode: latex
%%% mode: visual-line
%%% fill-column: 5000
%%% TeX-master: "report"
%%% End:

%  LocalWords:  sec:intro Sloane:oeis12,oeis mathbb Cremona:LMFDB16,lmfdb:on GAP:on \Sage:on singular:on DehKohKon:iop16 emph sec:mitm sec:cgt sec:apit sec:mitm_poc sec:concl MitM-based sec:case Jupyter jupyter-project:on KohDavGin:psewads11 ldots,X_n leq obne cdot cdot
%  LocalWords:  SageMath:on ldots,X_4 subseteq obner Groebner formalizations
