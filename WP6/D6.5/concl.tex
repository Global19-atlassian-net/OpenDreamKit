We have implemented the MitM approach to integrating mathematical software system based on formalizations of the underlying mathematical knowledge.
The main investment here was the curation of an MitM Ontology, the generation of formal specifications of system APIs for \Sage, \GAP, and \Singular, identifying the alignments of these APIs with the ontology, implementing an MitM server that can use alignments to translate between systems, and implementing the \SCSCP protocol for all involved systems. 

Our case study showed that MitM-based integration is an achievable goal.
Delegation-based workflows can either be programmed directly or embedded into the interaction language of the mathematical software systems.

The main advantages and challenges claimed by the MitM framework come from its loosely coupled and knowledge-based nature.
Compared to ad-hoc translations, MitM-based interoperability is relatively expensive as objects have to be serialized into (possibly large) \OMMT objects, transferred via \SCSCP to \MMT, parsed, translated into another system dialect, serialized and transferred, and parsed again.
On the other hand, instead of implementing and maintaining $n^2$ translations, we only have to establish and maintain $n$ collections of system APIs and their alignments to the
MitM ontology.
This makes the management of interoperability much more tractable:
\begin{compactenum}
\item The MitM ontology is developed and maintained as a shared resource by the community.
We expect it to be well-maintained, since it can directly be used as a documentation of the functionality of the respective systems.
\item All the workflows are star-shaped: instead of requiring expert knowledge in two systems -- a rare commodity even in open-source projects, and even for the system experts involved in this paper -- and keeping up with their changes, the MitM approach only needs expertise and change management for single systems.
\end{compactenum}
All in all, these translate into a ``business model'' for MitM-based cooperation in terms of the necessary investment and achievable results, which is based on the well-known \emph{network effects}: the joining costs are in the size of the respective system, whereas the rewards -- i.e. the functionality available by delegation -- is in the size of the network.

This network effect can be enhanced by technical refinements we are currently studying:
For instance, if we annotate alignments with a ``priority'' value that specifies how canonically/efficiently/powerfully a given system implements a given MitM operation, then we can let
the \MMT mediator automatically choose a suitable target system for a requested computation (as opposed to our current setup where Jane specifies which systems she wants to use). On the other hand, for workflows where we do not need or want service-discovery, alignments can be ``compiled'' into $n^2$ transport-efficient direct translations that may even eliminate the need for serialization and parsing.

\subsubsection*{Acknowledgements}
The authors gratefully acknowledge the fruitful discussions with other participants of
work package WP6, in particular Alexander Konovalov on \SCSCP, Paul Dehaye on the \Sage
export and the organization of the MitM ontology, and Luca de Feo on OpenMath phrasebooks
and the \SCSCP library in python.  We acknowledge financial support from the OpenDreamKit
Horizon 2020 European Research Infrastructures project (\#676541) and DFG project RA-18723-1 OAF.

%%% Local Variables:
%%% mode: latex
%%% TeX-master: "paper"
%%% End:

%  LocalWords:  sec:concl MitM-based itemize subsubsection Dehaye organization serialized
%  LocalWords:  math-savy emph serialization formalizations
