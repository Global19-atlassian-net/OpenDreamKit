\section{Conclusions and Future Work}\label{sec:Future}

We improved the digitalization of the \oeis by parsing the formulae and suggested a way to
improve the formula language in \oeis. Even though our parser can definitely be improved,
it already supports two important added-value services. First, the \mws instance on \oeis
which allows the users to search the \oeis by text formula queries. Second, a way of
generating knowledge from \oeis, specifically, relations between sequences. The relation
finding experiment presented above only uses very simple mechanisms for finding relations
between generating functions. We make the parsed and induced formulae available at
\url{https://eluzhnica.github.com/ISFA} to allow other parties to extend our methods and
find even more relations.  The improved language and its parser are still at a rudimentary
phase. We plan to make a UI that parse-checks the formulae in real time during
submissions. This makes it easier for the contributors to go along with the new
system. Also, since SageMath already has a module for \oeis we plan to import the parsed
formulas in the module.



%%% Local Variables:
%%% mode: latex
%%% TeX-master: "report"
%%% End:
