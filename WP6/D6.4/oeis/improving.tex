\section{Improving the OEIS Formula Language} \label{sec:Improve}

Section \ref{sec:form-parser} introduced the problems when parsing the \oeis. In this section we are going to show an
 improved formula language and the parser for it.

OEIS has a strong policy advocating human readable math syntax. Users have a pretty strong autonomy on what notation
they use.
In the FAQ of OEIS we can see these:

\begin{lstlisting}
Q: Is it OK to use LaTex or Maple notation in equations?
A: No, please don't!

Use notation that can be understood by humans.
Say     (1+x)/(1-x)     not     $\frac{(1+x)}{(1-x)}$.
Say     n^2/2     not     1/2n^2.
Say     A/(B*C*D)     not     A/B/C/D
Say     sigma     not     $\sigma$.
\end{lstlisting}

The above explicitly suggests a human readable format and implicitly not using any sort of delineation. Furthermore,
the FAQ also contains:

\begin{lstlisting}
Q: How should I indicate summations?
A: There are many acceptable ways to do this.

All of the following examples are OK.

sum_{i=1..n} i^2 + i

sum_i=1..n (i^2+i)

sum_{d|n} d^3

sum_{ 2 <= p <= n, p prime} p^2

Other styles also acceptable, as long as they are clear!
\end{lstlisting}
This offers too much autonomy in the choice of constructs and even the proposed notations lead to ambiguous usages.
For instance, in the expression below, it is unclear if the first expression equivalent to any of the ones below it.

\begin{lstlisting}
x^2 + sum_{i=1..n} i^2 + 2
x^2 + sum_{i=1..n} (i^2) + 2
x^2 + sum_{i=1..n} (i^2 + 2)
\end{lstlisting}


While human readability is certainly important, there better ways to achieve it while making it easier for the
parsers to mine the invaluable resources in the \oeis. In fact, once the parsers successfully parse the formulas, we
can then have MathML presentations of the formulas. In this section we show the an improved language for which we
have implemented a parser. The approach should be read having in mind the practicality of adopting the implementation
 from the \oeis which explains the choice of a less rigorous language. We only address a part of the language,
 specifically the part that is attempted by the current parser.

\subsection{Language}
We categorize two levels of ambiguity:

\begin{enumerate}
    \item Having no clearly defined semantic interpretation. For instance, in $phi(x+2)$, the  function that $phi$
    refers to is not defined.

    \item Having two or more clearly defined and distinct semantic interpretations of an expression. An example is
    $pi(x+2)$, where $pi$ could be function or a variable.
\end{enumerate}

The level of severity in terms of ambiguity is debatable. If the goal is computing with these expressions, then the
second ambiguity might be more severe than the first. In the example above, if we know the function to which $pi$
refers to, then it is possible to compute all possible semantic interpretations of the expression.

In the \oeis context, the first level is a bigger issue, mainly due to the fact that there are special sections in
\oeis for computational purposes. These sections contain executable code for Mathematica, Maple etc.

\oeis has to two different formats, the \emph{internal} format and the \emph{text} format which is shown to the users
. Our modifications are applicable to the internal format, while the text format does not necessarily change.

\paragraph{Delineating formulas} Currently, the majority of failures of the existing parser are attributed to lack of
 delineation.
So we add deliniators to the internal format. Once the formulae are parsed, the text format can optionally change to
show them in MathML.

\paragraph{Ambiguity and Open Set of Primitives} The new language disallows implicit multiplication. However, the
explicit multiplication will only show up in the internal format, while the text format still includes the implicit
multiplication.

\begin{wrapfigure}{r}{8.0cm}
\vspace{-0.5cm}
\begin{center}
\begin{tabular}{|lll|}
\hline
$\mathit{S}$ & \bbc & $\mathit{Func}\; | \mathit{Rel}\; | \mathit{Const}\; | \mathit{Esc}$\\
$\mathit{Func}$ & \bbc & $\textit{:func}\; \mathit{Name}\; \mathit{Arg} \; \mathit{Form} \; \mathit{Prec} \;
\mathit{Desc} $ \\
$\mathit{Rel}$ & \bbc & \textit{:rel}$\; \mathit{Name}\; \mathit{Form} \; \mathit{Prec} \; \mathit{Desc} $ \\
$\mathit{Const}$ & \bbc & \textit{:const}$\; \mathit{Name}\; \mathit{Desc} $ \\
$\mathit{Esc}$ & \bbc & $\textit{:esc}\; \mathit{Name}\; \mathit{Desc} $ \\
$\mathit{Name}$ & \bbc & Name of the function \\
$\mathit{Arg}$ & \bbc & $\mathtt{(\mathit{W}, ..., \mathit{W})|(\mathit{W}, ..., \mathit{W}) \mathit{Arg}} $ \\
$\mathit{Form}$ & \bbc & \textit{prefix | suffix | infix} \\
$\mathit{Prec}$ & \bbc & [$1$-$2^{31}$] \\
$\mathit{Desc}$ & \bbc & Description \\
$\mathit{W}$     & \bbc & [a-zA-Z0-9\_] \\
\hline

\end{tabular}
\caption{Grammar of \oeis language theory}\label{fig:theory-form}
\end{center}
\end{wrapfigure}

The other problems are solved by having an explicit source of knowledge. This means having a document which lists the
 in fix, postfix, relational operators and the constants. This is what we call the \emph{language theory} of an \oeis
 document. These theories do not have complex inheritance system. They can be either global in scope to the whole
 \oeis, or local, only in the scope of one \oeis document. This is reduces the reusability, however with this we
 intend to increase the readability and user friendliness since there are usually a very limited number of operators
 defined per document. These definitions follow a simple format, declaring the operator, the argument pattern, the
 type, the precedence and the meaning in a few words.
Assuming the description is good the definitions can be manually converted into more formal notations to be used for
computation purposes.
Other forms of ambiguity are generally solved in the same way: we get closer to the rigorous math notation. For
instance, the power operators need to be properly bracketed - \verb|T^y(x^2+2)| should be written as \verb|(T(x^2+2))^y|.
 Nonetheless, unbracketed function application is allowed since it is already disambiguated by disallowing
implicit multiplication.

Due to the natural language usage we have unusual expressions. For instance we can spot expressions of the form:
\begin{verbatim}
    x^y + 2*triangular number
    2*x + perfect square
    sum_{i in fibbonaci numbers}
\end{verbatim}
We have the constructs that we call \emph{escapers} to allow creating these expressions. With escapers the syntax
would then become:

\begin{verbatim}
    x^y + 2*{{triangular number}}
    2*x + {{perfect square}}}
    sum_{i in {{fibbonaci numbers}}}
\end{verbatim}

These constructs can contain natural language, however they still have to be defined in the theory of the document.

Below is an example definition of the theory for the sequence \verb|A000045| which is partially shown in Section ??.

\renewcommand{\lstlistingname}{Theory}

\begin{lstlisting}[label=lst:theoryA000045,caption=Example theory for A000045]
:func     sqrt    (x1)        prefix  1   The square root of x1
:func     sum     (x1, x2)    prefix  1   Sum over x1 for values defined in x2.
:const    pi                              The number 3.14159
:rel      >=                  infix   2   Greater than or equal
:func     +                   infix   1   Sum
:rel      to                  infix   2   Range
:func     F                   prefix  1   The Fibonacci function
\end{lstlisting}

We have shown all of the definitions in one theory for simplicity. In a normal setting, some of them would be defined
 in the global theory. We have also omitted some operators for brevity.
Using the rules above the formula language for the sequence document would then be transformed as shown below.

\begin{lstlisting}[caption=Adjusted A000045 based on Theory \ref{lst:theoryA000045}]
%F A000045 $F(n) = ((1+sqrt(5))^n-(1-sqrt(5))^n)/(2^n*sqrt(5))$
%F A000045 G.f.: $Sum_{n>=0}{x^n * Product_{k,1 to n} {(k + x)/(1 + k*x)}}$. - _Paul D. Hanna_, Oct 26 2013
%F A000045 This is a divisibility sequence; that is, if n divides m, then a(n) divides a(m)
\end{lstlisting}

Notice that the last line could very easily be delineated as math, but we did not do it
for illustration purposes.

A complete definition for the theory definition language has been given in Figure
\ref{fig:theory-form}. The defined operators are considered to be left associative.

\paragraph{Implementation} The parser of the new language and the parser of the theory
definition language have been implemented using the Packrat Parser \cite{packrat} in
Scala. The theory definitions are first parsed and then fed to the parser for the formula
language.  The current limitation of the formula language parser is that it doesn't
support the operator precedences. It parses the prefix, infix and postfix operators in
order.


%%% Local Variables:
%%% mode: latex
%%% TeX-master: "report"
%%% End:
