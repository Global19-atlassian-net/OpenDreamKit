\section{Related Work and Comparisons} \label{sec:Related}

The content representation of mathematical expressions has been widely researched from
different perspectives. The main problem of processing natural language along a complex
mathematical symbolism has been mainly approached in two ways \cite{Ginev-11}. The first
approach tries to make a controlled natural language and make the language incrementally
more powerful while disallowing ambiguities. The ambiguity resolution here is usually
based on an explicit context and/or type inference. The top-down approach tries to model
the existing discourse of mathematics.  The parsing of mathematical expressions in done
mainly using context-free grammars aided with a type system or special purpose grammars
that are based on type systems. The work done using these approaches generally aims to
parse a large set of topics in the mathematical discourse. In our case we have a more
specific mathematical topic to parse and the discourse is only limited to mathematical
expressions. Our current approach is a top-down approach as we will see in Section
\ref{sec:form-parser}, although it is later amalgamated with techniques from the
controlled language approach by introducing a new formula language for \oeis.

On the other hand, the work in finding relations between sequences has been centered
around numerical based approaches, mainly due to the representation of the
\oeis. Generally, the high level approach is to apply different numerical transformations
on the available starting values of the integer sequences and then check for matches. This
match checking is usually done with exact matches on all starting values of a sequence or
only a subset of it, or even analyzing the sequence that comes out of some predefined
error function. Transformations are tuned for relations that the author is interested in,
keeping in mind the search costs. Two approaches of this nature have been compiled in
Appendix \ref{App:AppendixA}.

A disadvantage is that the found relations are only conjectures since the matching is done on a finite subsequence,
which usually is less than 30 terms. An extreme example of two sequences that match for a long time but are not
equal, is the following:

$$\floor*{\frac{2n}{log(2)}}  \qquad\text{and}\qquad \ceil*{\frac{2}{2^{1/n}-1}}$$

They agree for the first 777451915729367 terms \cite{oeis-paper2}!

In fact, a similar approach from Ralf Stephan \cite{Ralf} yielded 117 conjectures from
which 17 of them are still conjectures \cite{RalfUpdates}, which shows that these
conjectures are not always easy to prove or disprove.


%%% Local Variables:
%%% mode: latex
%%% TeX-master: "../report"
%%% End:
