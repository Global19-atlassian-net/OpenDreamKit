\begin{appendix}
\section{Related Work} \label{App:AppendixA}
OEIS has the mail server called Superseeker which tries over 120 transformation on the \oeis sequences to match
a given sequence. This has also been used to generate relations between the current sequences by running it over the
whole \oeis corpus. The steps involved in the Superseeker for a given sequence $S$ are as follows \cite{Peter}:

\begin{enumerate}
\item Do an exact match on the current \oeis sequences.
\item Test if $S[n]$ is a polynomial in $n$.
\item Test if the difference of some order ${S_d[n]}$ are periodic.
\item Test if any row of the difference table of some depth is essentially constant.
\item Form some generating functions for the sequence for some predefined types (ie. ordinary gf, exponential gf etc.)
\item Look for a linear recurrence with polynomial coefficients for the coefficients of the above 6 types of
generating functions.
\item Apply transformations to the sequence and lookup the result in the table.
\item If the original sequence is not in the table, find the 3 closest sequences in the table using the $L1$ metric.
Only those with $L1 \leq 3$ are reported.

\end{enumerate}

An approach from Peter Liu \cite{Peter} extends the search space of finding relations. While Superseeker uses the
given sequence or its transformation sequences to query the database and the queries are exact match searches, his
approach is to use the given sequence itself to query the database and the queries are not just exact matches. They
could be the linear combination of two sequences in the table, or the affine transformation of a sequence or the
polynomial of a sequence bounded by a constant. More specifically, if $S$ is a given sequence and $D$ the sequence in
 database, the searched relations are as follows \cite{Peter}:

\begin{enumerate}
\item Is there an exact match between $S$ and $T_i \in D$?
\item Is $S$ simply some sequence in the $D$ shifted?
\item Are all member of $S$ contained in some sequence $T_i \in D$?
\item Is $S$ an affine transformation of some sequence $T_i \in D$?
\item Can $S$ be written as a polynomial of degree bounded by a constant?
\item Can $S$ be written as the linear combination of two $T_i$ and $T_j \in D$?
\item Is the $S$ close to a $T_i$?
\item Does $S$ leave a constant remainder when divided by a $T_i$?
\item Does affine transformation of $S$ give a subsequence of some $T_i \in D$?
\end{enumerate}

Of course, these queries have to be efficient enough to run over the corpus and the author realizes that through
pre-processing the database, use of hashing techniques and number theory. His approach yielded some new relations
that the Superseeker could not find, speaking of the time that Peter Liu's work has been published. From query 7, the
 author found relations of the form $T_{1923} = 2*T_{825} + T_{1428}$ or $T_{1746} = 2*T_{1205}$.

While from query 4, there were more relations found and one of those is $T_{1567} = 2*T_{1049} -1 = 2*T_{616} +1 =
4*T_{391} - 3$.
\end{appendix}

%%% Local Variables:
%%% mode: latex
%%% TeX-master: "../report"
%%% End:
