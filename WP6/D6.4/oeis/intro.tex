 \section{Introduction}

 Integer sequences are important mathematical objects that appear in many areas of
 mathematics and science and are studied in their own right.  The On-line Encyclopedia of
 Integer Sequences (\oeis) \cite{oeis} is a publicly accessible, searchable database
 documenting such sequences and collecting knowledge about them. The effort was started in
 $1964$ by N. J. A. Sloane and led to a book \cite{handbook-is} describing $2372$
 sequences which was later extended to over $5000$ in \cite{encyc-is}. The online version
 \cite{oeis-paper} started in $1994$ and currently contains over $250 000$ documents from
 thousands of contributors with $15 000$ new entries being added each year
 \cite{oeis-paper2}. Documents contain varied information about each sequence such as the
 beginning of the sequence, its name or description, formulas describing it, or computer
 programs in various languages for generating it.

 The \oeis library is an important resource for mathematicians. It helps to identify and
 reference sequences encountered in their work and there are currently over $4000$ books
 and articles that reference it.  Sequences can be looked up using a text-based search
 functionality that \oeis provides, most notably by giving the name (e.g. ``Fibonacci'')
 or starting values (e.g. ``$1,2,3,5,8,13,21$''). However, given that the source documents
 describing the sequences are mostly informal text, more semantic methods of knowledge
 management and information retrieval are limited.

 In this report we tackle this problem by building a (partial) parser for the source
 documents and importing the \oeis library into the \omdmmt format which is designed for
 better machine support and interoperability. This opens up the \oeis library to
 \omdoc-based knowledge management applications, which we exemplify by a semantic search
 application based on the \mws \cite{KohPro:man13} system that permits searching for text
 and formulas and by a relation finder.  In addition, we design another language for \oeis
 and build a parser for that which we are going to show here.

%In Section \ref{sec:prel} we briefly introduce the \omdmmt language and the \mmt system.
 This report is organized as follows: in Section \ref{sec:Preliminaries} we briefly
 introduce the \omdmmt language, \oeis and the theory of generating functions. In Section
 \ref{sec:Related} we discuss related work and in Section \ref{sec:form-parser} we
 describe our import of the \oeis library into \omdoc. In Section \ref{sec:Search} we show
 an initial application of our import by providing formula search for the \oeis
 library. Then, in Section \ref{sec:Enrich} we present the relation finder and in Section
 \ref{sec:Improve} we present the improved formula language. Finally, in Section
 \ref{sec:Future} we discuss future work and conclude.


%%% Local Variables:
%%% mode: latex
%%% TeX-master: "../report"
%%% End:
