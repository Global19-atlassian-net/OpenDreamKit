\documentclass{llncs}
\pagestyle{plain}
\usepackage[show]{ed}
\usepackage[utf8]{inputenc}
\usepackage{xspace}
\usepackage[style=alphabetic,backend=bibtex,isbn=false]{biblatex}
\addbibresource{../../lib/kbibs/kwarcpubs.bib}
\addbibresource{../../lib/kbibs/extpubs.bib}
\addbibresource{../../lib/kbibs/kwarccrossrefs.bib}
\addbibresource{../../lib/kbibs/extcrossrefs.bib}
\addbibresource{rest.bib}% add bibs here!
\renewbibmacro*{event+venue+date}{}
\renewbibmacro*{doi+eprint+url}{%
  \iftoggle{bbx:doi}
    {\printfield{doi}\iffieldundef{doi}{}{\clearfield{url}}}
    {}%
  \newunit\newblock
  \iftoggle{bbx:eprint}
    {\usebibmacro{eprint}}
    {}%
  \newunit\newblock
  \iftoggle{bbx:url}
    {\usebibmacro{url+urldate}}
    {}}

\usepackage{hyperref}
\title{Math-in-the-Middle Ontology for CAS Interoperability}
\author{
Michael Kohlhase\inst{1} 
%Alexander Konovalov\inst{3} 
%Samuel Lelièvre\inst{4} 
Dennis M\"uller\inst{1} 
Markus Pfeiffer\inst{2} 
%Florian Rabe\inst{2} 
%Nicolas~M.~Thiéry\inst{4} 
%Tom Wiesing\inst{2}
}

\institute{
   FAU Erlangen-N\"urnberg
   \and University of St~Andrews 
%   \and Universit\'e Paris-Sud
}
\begin{document}
\maketitle
\begin{abstract}
  \ednote{twb}
\end{abstract}

\section{Introduction}\label{sec:intro}
\ednote{We build on~\cite{DehKohKon:iop16} (MitM paradigm) and \cite{MueGauKal:cacfms17}
(alignments).}
\begin{todolist}{follow this}
\item generally we want to show that the promises in the CICM paper become reality
\item show the generated system ontologies taking GAP as an example 
\item contrast the SageMath export with that 
\item with the example of GAP group theory make the MitM ontology (part)
\item star-formed alignments and how we come by them, 
\item talk about SCSCP and GAP/Sage/LMFDB dialects and intra-MMT translation
\item need a more elaborate example for Sage/GAP interop, e.g. SAGE polynomials and GAP
  permutation groups. 
\end{todolist}
\section{Conclusion}\label{sec:concl}
\ednote{how can YOU help extend the MitM?}
\printbibliography
\end{document}
%%% Local Variables:
%%% mode: latex
%%% TeX-master: t
%%% End:
