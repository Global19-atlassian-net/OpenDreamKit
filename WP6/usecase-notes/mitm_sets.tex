\documentclass{article}
\title{Compiling Math-in-the-Middle Interoperability into P2P Integration}
\author{Michael Kohlhase\\FAU Erlangen-N\"urnberg}
\usepackage[style=alphabetic,backend=bibtex]{biblatex}
\addbibresource{../../lib/kbibs/kwarcpubs.bib}
\addbibresource{../../lib/kbibs/extpubs.bib}
\addbibresource{../../lib/kbibs/kwarccrossrefs.bib}
\addbibresource{../../lib/kbibs/extcrossrefs.bib}
\def\cn#1{\ensuremath{\mathsf{#1}}}
\def\defemph{\textbf}
\usepackage{tikz}
\usetikzlibrary{mmt}
\usepackage{wrapfig}
\begin{document}
\maketitle

\section{Introduction}

In \cite{DehKohKon:iop16} we have introduced the Math-in-the-Middle (MitM) paradigm for
interoperability for mathematical software systems. In a nutshell the MitM ontology
modularly formalizes textbook mathematics as an MMT theory, we generate system API
theories the formalize the types, constructors, and interface functions of the systems,
and we curate alignmnets between the API theories and the MitM ontologies that can be used
to translate objects from the systems to MitM and thus use the MitM ontology as a pivot
for translating between systems; see \cite{KohMuePfe:kbimss17} for details and use cases.

We claimed in that we can also compile MitM interoperability into more efficient
peer-to-peer translations. In this note we look at this in detail.

\section{Preliminaries: Abstract and Concrete Representations}

We call a representation of a mathematical structure \defemph{abstract}, if it corresponds
to (one of) the usual textbook definitions and \defemph{concrete}, if it realizes an
abstract representation in terms of an representation that is already established or
computationally more efficient. 

\begin{wrapfigure}r{1.5cm}\vspace*{-1em}
  \begin{tikzpicture}[yscale=1.5]
    \node (a) at (0,1) {\cn{A}};
    \node (c) at (0,0) {\cn{C}};
    \draw[view] (a) to[bend left=25] node[right] {$i$} (c);
    \draw[view,dotted] (c) to[bend left=25] node[left] {$e$} (a);
  \end{tikzpicture}
\end{wrapfigure}
In MMT, we can formalize abstract and concrete representations with theories and views; We
refer the reader to~\cite{KohRabSac:fvip11} for an in-depth discussion; abstract theories
are called ``specification'' and concrete ones ``implementation'' there.  The general
situation is as in the diagram on the right: there is a view $i$ from the the abstract
theory $\cn{A}$ to the concrete theory $\cn{C}$ and a -- possibly partial -- embedding
view back.

Initially, the MitM ontology was mainly intended for abstract theories, but
following~\cite{KohRabSac:fvip11}, we propose to extend it with concrete theories for the
``platonic implementations'' in computational systems. An example for this is the theory
of polynomials, which are implemented 


\section{Sharing Finite Sets between Sage and GAP}

We look at the case of representing finite sets in the MitM ontology and how that can use
bea for compiling glue code. Background: GAP represents sets as ordered duplicate-free
lists and Sage uses the underlying python set type, which is realized as unordered
duplicate-free lists\footnote{We disregard for the moment that python allows mixed-type
  lists.}.

\begin{figure}[ht]\centering
\begin{tikzpicture}[xscale=2.5,yscale=1.3]
  \node[thy] (fs) at (0,3) {$\mmtthy{FinSet}{set}{}$}; 
  \node[thy] (l) at (-1,3) {$\mmtthy{list}{cons,nil}{}$}; 
  \node[thy] (ps) at (1,2.5) {$\mmtthy{FinPOSet}{<}{}$};
  \node[thy] (lnd) at (-1,1) {$\mmtthy{ListNoDup}{}{}$};
  \node[thy] (o) at (1,1) {$\mmtthy{OListNoDup}{sort}{}$};
  \node[thy] (s) at (-1,0) {$\mmtthy{Sage.set}{set}{}$};
  \node[thy] (g) at (1,0) {$\mmtthy{Gap.set}{set}{}$};
  \draw[include] (fs) -- (ps);
  \draw[include] (l) -- (lnd);
  \draw[biview] (fs) -- (lnd);
  \draw[biview] (ps) -- (o);
  \draw[view] (lnd) to[bend left=15] node[above]{\scriptsize\cn{sort}} (o); 
  \draw[view] (o) to[bend left=15] node[below]{\scriptsize\cn{id}} (lnd); 
\end{tikzpicture}
\caption{Abstract and Concrete Finite Sets and their Implementations}\label{fig:sets}
\end{figure}


\printbibliography
\end{document}
%%% Local Variables:
%%% mode: latex
%%% TeX-master: t
%%% End:
