\documentclass{article}
\title{Compiling Math-in-the-Middle Interoperability into P2P Integration}
\author{Michael Kohlhase\\FAU Erlangen-N\"urnberg}
\usepackage[style=alphabetic,backend=bibtex]{biblatex}
\addbibresource{../../lib/kbibs/kwarcpubs.bib}
\addbibresource{../../lib/kbibs/extpubs.bib}
\addbibresource{../../lib/kbibs/kwarccrossrefs.bib}
\addbibresource{../../lib/kbibs/extcrossrefs.bib}
\def\cn#1{\ensuremath{\mathsf{#1}}}
\usepackage{tikz}
\usetikzlibrary{mmt}
\begin{document}
\maketitle

We look at the case of representing finite sets in the Math-in-the-Middle ontology and how
that can be  
\begin{figure}[ht]\centering
\begin{tikzpicture}[xscale=1.8,yscale=1.3]
  \node[thy] (fs) at (0,3) {\cn{FinSet}};
  \node[thy] (ps) at (0,2) {\cn{FinPOSet}};
  \node[thy] (l) at (-1,1) {\cn{ListNoDup}};
  \node[thy] (o) at (1,1) {\cn{OListNoDup}};
  \node[thy] (s) at (-1,0) {\cn{Sage.set}};
  \node[thy] (g) at (1,0) {\cn{Gap.set}};
  \draw[include] (fs) -- (ps);
  \draw[biview] (fs) -- (l);
  \draw[biview] (ps) -- (o);
  \draw[view] (l) to[bend left=15] node[above]{\scriptsize\cn{sort}} (o); 
  \draw[view] (o) to[bend left=15] node[below]{\scriptsize\cn{sort}} (l); 
\end{tikzpicture}
\end{figure}
\printbibliography
\end{document}
%%% Local Variables:
%%% mode: latex
%%% TeX-master: t
%%% End:
