\section{The MitM Ontology for Computational Group Theory}\label{sec:cgt}
\begin{todolist}{MK@MP+DM: describe your work here}
\item talk about the levels (abstract, concrete subgroup theory, computational)
\item talk about alignments from the IFT to the CGT, how they work, building
  on~\cite{MueRoYuRa:abtafs17,MueGauKal:cacfms17} 
\end{todolist}

To create a working example, we turned towards one of the topics best
understood by GAP: Computation with (permutation) groups, and formalise them in
MMT.

Starting from the theory of groups, we formalised the notions of
subgroups, homomorphisms, group actions, stabilisers, orbits, and some more
advanced concepts such as normalisers, and p-Sylow-subgroups.\footnote{we could
  also mention that we created views that ``prove''(?) equivalence of two
  different formal definitions of group actions, one via an action map, the
  other via an action homomorphism; both concepts exist in GAP, too }

Since GAP\footnote{any computational system needs to} knows about different
concrete representations of groups, for example universal representations such
as permutation groups, matrix groups (over finite fields), or finitely presented
groups, or specialised representations such as polyclically presented groups
there needs to be a layer of formalisation that
understands these representations.

We formalised the theory of symmetric groups of a set; in GAP permutation groups
are represented as subgroups (with finite support) of the symmetric group of
$\mathbb{N}$, and often one concretely has an isomorphism between the group one
is interested in and a subgroup of $S_{\mathbb{N}}$, for example
via a group action.

Most of this information is encoded in GAP as well, but some of it is much less
explicit.

\footnote{Do we mention that this attempt at formalising group theory lead to
  improvements in MMT?}
\footnote{We might want to be a bit careful/mention implementations of group
  theory for example in COQ where they did the Odd-Order-Proof?}
\footnote{We need to describe how we'd do alignments now; by hand?
  semi-automated? automated? what do we actually align!}
%%% Local Variables:
%%% mode: latex
%%% TeX-master: "paper"
%%% End:
