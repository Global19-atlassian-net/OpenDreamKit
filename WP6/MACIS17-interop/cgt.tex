\section{The MitM Ontology for Computational Group Theory}\label{sec:cgt}
\begin{todolist}{MK@MP+DM: describe your work here}
\item talk about the levels (abstract, concrete subgroup theory, computational)
\item talk about alignments from the IFT to the CGT, how they work, building
  on~\cite{MueRoYuRa:abtafs17,MueGauKal:cacfms17} 
\end{todolist}

To create a working example, we turned towards one of the topics best
understood by GAP: Computation with (permutation) groups, and formalise it in
MMT.
This a first part of the MitM Ontology.


\subsection{Layers of Abstraction}\ednote{MP@DM this could do with a picture a
  bit like the one in your Alignments paper; if you have the source for it, I
  coudl adapt it}
The layers: abstract, representation, canonical(?), system dialect

We discovered that formalisation of CGT requires different levels of
abstraction\ednote{MP: This will probably be true for any type of object
  in this game}. At the highest level there is the theory of \emph{Groups}: the
group axioms, generating sets, homomorphisms, group actions, stabilisers,
and orbits. This also easily leads into definitions of
centralisers\footnote{stabilisers of elements under conjugation} and
normalisers\footnote{stabilisers of subgroups under
conjugation}, stabiliser chains,  Sylow-$p$ subgroups, Hall subroups, and many
other concepts. 

MMT also allows expressing that there are different equivalent definitions of a
concept: We defined group actions in two ways and used \emph{views} to show
their equivalence.

\ednote{MP: We need to be able to talk about elements/subsets of groups,
  elements/subsets of $S_n$ that generate groups}

\medskip

Abstract groups can be represented in many ways as more concrete mathematical
objects: as groups of permutations, groups of matrices, finitely presented
groups, or as a polycyclic presentation.\ednote{MP: Not sure this is relevant but the first
three of these are universal: every group has a representation as such an
object, whereas the last is a specialised representation for polycyclic groups}

Additionally, mathematicians often compute with canonical representatives of an
isomorphism class of groups: When a group theorist talks about the ``Dihedral
group of order 8'', they often have a particular canonical representation in
mind, for example as a permutation group that acts on the square by rotations
and reflections, but in GAP this group would be represented as a group of
permutations of (usually) the corners of the square, or a polycyclic
presentation.\ednote{MP: I think this needs better explanation}

These representations also arise naturally from \emph{group actions}: If we are
considering symmetry in a setting where we want to apply group theory, we start
with a group action.\ednote{MP: More concrete? More ``gripping''? I already
talked about the canonical example with the dihedral group}

The universal tool to bridge the gap between groups, representations and
canonical representatives are group homomorphisms.

\medskip

At the lowest level there are implementation details: Permutation groups in GAP
are considered as finite subgroups of the group $S_{\mathbb{N}+}$, and defined by
providing a set of generating permutations. GAP then computes a stabiliser chain
for a group that was defined this way, and naturally considers the group to be a
subgroup of $S_{[1..n]}$, where $n$ is the largest point moved.

\ednote{MP: There might have to be more layers, but these are the main ones I
  can think about right now}

\subsection{Alignments}

Alignments are currently produced by hand: \ednote{MP: potential for
  automated or semi-automated production of alignments from our exports}
For example the filter \texttt{IsGroup} is aligned with \texttt{Group}, and the
filter \texttt{IsPermGroup} is aligned with \texttt{Subgroup SymmetricGroup
  [1..n]}.
\ednote{MP: Need to be more concrete here, in particular we should maybe
  describe how GAP's notion of an action homomorphism tranlates through this?
  Also is this even correct?}

We formalised the theory of symmetric groups of a set; in GAP permutation groups
are represented as subgroups (with finite support) of the symmetric group of
$\mathbb{N}$, and often one concretely has an isomorphism between the group one
is interested in and a subgroup of $S_{\mathbb{N}}$, for example
via a group action.

\texttt{SylowSubgroup} are more difficult: They are special groups in their
own right, namely groups whose size is a prime-power, but we also want them
to be identified with a certain subgroup of the group we are working
with.\ednote{MP: While I believe this to be an excellent additional example
  for MMT formalisation, this could be going too far for this paper}


\ednote{MP+FR+DM: Mention that this attempt at formalising group theory
  lead to improvements in MMT?}
\ednote{MP@ALL: We might want to be a bit careful/mention implementations of group
  theory for example in COQ where they did the Odd-Order-Proof?}
%%% Local Variables:
%%% mode: latex
%%% TeX-master: "paper"
%%% End:
