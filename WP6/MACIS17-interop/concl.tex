\section{Conclusion}\label{sec:concl}
We have implemented the MitM approach to integrating mathematical software system based on
flexiformalizations of the underlying mathematical knowledge. The main investment here was
the curation of a MitM Ontology, the generation of system API CDs, and a corresponding set
of alignments, as well as implementing the OpenMath \SCSCP protocol for \Python and Scala
(for the \MMT system). Our two integration case studies show that the MitM-based
integration is an achievable goal; delegation-based workflows an either be programmed
directly or embedded into the interaction language primitives of the mathematical software
system that acts as a master. 

The main advantages and challenges claimed by the MitM framework come from it's loosely
coupled, knowledge based nature. Compared to ad-hoc translations, MitM-based
interoperability is relatively clumsy, since objects have to be serialized into (in
practice rather large, XML-based) OpenMath objects, transferred via \SCSCP to \MMT, parsed,
translated into another system dialect, serialized and transferred, and parsed again. On
the other hand, instead of implementing and maintaining $n^2$ translations, we only have
to establish and maintain $n$ collections of system API CDs and their alignments to the
MitM ontology. This makes management of interoperability much more tractable, since
\begin{compactenum}
\item the MitM ontology is developed and maintained as a shared
  resource by the community. We expect it to be well-maintained, since it can directly be
  used as a documentation of the functionality of the respective systems.
\item all the workflows are star-shaped: instead of requiring expert knowledge in two
  systems -- a rare commodity even in open-source projects -- and keeping up with changes
  for each of the $n^2$ systems. The MitM approach only needs expertise and change
  management for single systems -- any math-savy developer can do that.
\end{compactenum}
All in all, these translate into a ``business model'' for the general MitM-based
cooperation mode in terms of the necessary investment and achievable results, which is
based on the well-known \emph{network effects}: the joining costs are in the size of the
respective system, whereas the rewards -- i.e. the functionality available by delegation
-- is in the size of the network.

This network effect can be enhanced by technical refinements we are currently studying:
For instance, if we annotate alignments with a ``priority'' value that specifies how
canonical/efficient/powerful a given system is for a given MitM concept, then we can let
the \MMT mediator\ednote{introduce the concept of \MMT acting as a knowledge-based
  mediator above} flexibly choose a suitable target system for a requested computation at
run time (from the systems available on the \SCSCP network at a given time). On the other
hand, for workflows where we do not need or want service-discovery, alignments can be
``compiled'' into $n^2$ transport-efficient ad-hoc translations that may even eliminate
the need for OpenMath serialization and dialect mediation.

\subsubsection*{Acknowledgements}
The authors gratefully acknowledge the fruitful discussions with other participants of
work package WP6, in particular Alexander Konovalov on \SCSCP, Paul Dehaye on the \Sage
export and the organization of the MitM ontology, and Luca de Feo on OpenMath Phrasebooks
and a \SCSCP library in python.  We acknowledge financial support from the OpenDreamKit
Horizon 2020 European Research Infrastructures project (\#676541).

%%% Local Variables:
%%% mode: latex
%%% TeX-master: "paper"
%%% End:

%  LocalWords:  sec:concl MitM-based itemize subsubsection Dehaye organization serialized
%  LocalWords:  math-savy emph serialization
