\section{Conclusion}\label{sec:concl}
\begin{todolist}{MK@MK: summary}
\item what is the ``business model'' for the general MitM-based cooperation model?
  \begin{itemize}
  \item talk about ``singular and GAP work in SAGE today'', but ad-hoc and non-scalable 
  \item MitM makes alignments that are at the basis of interoperability star-shaped and
    thus scalable.
  \item investment: small network joining costs in the size of the system, rewards:
    network effect in the size of the network. I.e. we do not need quadratically many
    programmers
  \item talk about compiling $n^2$ s2s translations from the MitM alignments. 
  \end{itemize}
\item how can YOU help extend the MitM?
\end{todolist}

\subsubsection*{Acknowledgements}
The authors gratefully acknowledge the fruitful discussions with other participants of
work package WP6, in particular Alexander Konovalov on SCSCP, Paul Dehaye on the SageMath
export and the organization of the MitM ontology, and Luca de Feo on OpenMath Phrasebooks
and a SCSCP library in python.  We acknowledge financial support from the OpenDreamKit
Horizon 2020 European Research Infrastructures project (\#676541).

%%% Local Variables:
%%% mode: latex
%%% TeX-master: "paper"
%%% End:

%  LocalWords:  sec:concl MitM-based itemize
