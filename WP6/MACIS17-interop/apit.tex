\section{API Content Dictionaries for GAP and SageMath}\label{sec:apit}
\begin{todolist}{MK: some of this has already been discussed in the CICM16 paper, }
\item MK@NT+DM: describe the SageMath API theory and how it is generated; this is
  new. 
\item MK@MK+NT+MP: compare and discuss the different generation approaches\ednote{MK: do
    we have an APIT for Singular yet? MP: We do not, we should whip one up, at
    least for the most basic bits of Singular}
\item MK@MK: conclude the section by a discussion about OpenMath and Dialects.
\end{todolist}


\subsection{GAP API Content Dictionaries}
\ednote{MK@MP+DM: describe the GAP API theory and how they are generated; give the numbers, but only give the diffs to the CICM16 paper.}

In \cite{DehKohKon:iop16} we describe our approach to export knowledge in the form of type information from the GAP system. 
We export all created types together with relationships between them to a json file, which is then imported into MMT.

We also defined the primitives of GAP's type system in OMDoc/MMT\ednote{@MP it would be
  really great if you finally wrote that paper about GAP's Type system formalisation
  MMT...}.\ednote{@DM, could you say anything about the MMT side of this?}

GAP's type system is based on sub-typing: \emph{Filters} express finer and finer subtypes of the universal \emph{IsObject}.
An object in GAP can learn about its properties, meaning its type is refined: A group can learn that it is abelian or nilpotent and hence its type changes.

For a successful implementation of MitM system it is also necessary to know how any given object in a running session has been constructed (Constructor annotation) and what it has learned about itself, i.e we need to describe filters\ednote{MP: It occurs to me that here be some dragons to be slain}

For this GAP was instrumented with code to store this information, so that the MitM system can now query GAP for an OM representation of any object that exists in GAP.  \ednote{MK@MP: we also need to talk about the OM Phrasebook (see Section~\ref{sec:mitm}), and how that works. Actually I think that this is really the new part we want to discuss here.}

\subsection{SageMath API Content Dictionaries}
\ednote{MK@NT: please write something here, this can be a bit more elaborate than the GAP part above, since GAP has already been described in the CICM16 paper. What is the state of the OM Phrasebook for SageMath?}

\subsection{Creation Of Singular API CDs}
\ednote{MK: it would be great, if we could do this and als have an estimation of the time used for this.}
\subsection{Comparison of Approaches}

%%% Local Variables:
%%% mode: latex
%%% TeX-master: "paper"
%%% End:

%  LocalWords:  sec:apit DehKohKon:iop16 json emph emph sec:mitm
