\section{API Content Dictionaries for \GAP and \Sage}\label{sec:apit}
In our case study, we want to show integration of the \GAP, \Singular, and \Sage systems
in the MitM paradigm, so we need API CDs for all of them. The three systems are
sufficiently different that we can consider the developmen presented in this section a
meaningful case study in the methodology and investment of exposing the APIs of real-world
systems in the form of OMDoc/MMT content dictionaries.

\begin{todolist}{MK: some of this has already been discussed in the CICM16 paper, }
\item MK@NT+DM: describe the \Sage API theory and how it is generated; this is
  new. 
\item MK@MK+NT+MP: compare and discuss the different generation approaches\ednote{MK: do
    we have an APIT for \Singular yet? MP: We do not, we should whip one up, at
    least for the most basic bits of \Singular}
\item MK@MK: conclude the section by a discussion about OpenMath and Dialects.
\end{todolist}


\subsection{\GAP API Content Dictionaries}
\ednote{MK@MP+DM: describe the \GAP API theory and how they are generated; give the numbers, but only give the diffs to the CICM16 paper.

DM: the diff should actually be zero, since as far as I know we haven't expanded the export by additional symbols in the meantime. @MP correct me if I'm wrong}

In \cite{DehKohKon:iop16} we describe our approach to export knowledge in the
form of type information from the \GAP system and produce a full \GAP API CD.

We defined the primitives of \GAP's type system in OMDoc/MMT, and it is
subject to future investigation whether MMT can serve as an external type-checker
for \GAP.

\GAP's type system is based on sub-typing: \emph{Filters} express finer and finer
subtypes of the universal \emph{IsObject}. An object in \GAP can learn about its
properties, meaning its type is refined: A group can learn that it is abelian or
nilpotent and hence its type changes.

For a successful implementation of an MitM system it is necessary to know how
a given object in an ongoing session has been created. Formally objects are
created by applying a \emph{constructor} to a list of arguments which are
themselves objects.
Since \GAP has only a weak notion of object construction, we achieve this by
identifying functions that create objects, and then instrumenting them to store
the relevant arguments they were called with.
The instrumentation itself is minimal (57 lines of \GAP code, plus 100 lines for
printing objects as OpenMath). The main challenge was
to identify functions that construct objects, and which arguments of those
functions are then stored in the object.
In \GAP objects are created by calling the function \texttt{Objectify} with a
type and some data, hence all that was needed was an analysis of call-sites to
this function and some light inference of the enclosing function.
Developing the instrumentation lead to many improvements of type-handling
infrastructure in \GAP itself, and the instrumentation will be integrated into a
future version of \GAP, making \GAP fully MitM capable.
\ednote{@MP think about
  ``why not just'' using the arguments to Objectify, this would of course also
  be stored, but looks even more minimal}

In the \GAP library there are 665 calls to Objectify, and in the standard package
distribution there are an additional 1664.

With the constructor annotation in place it is possible to have \GAP represent
any object in a running session as either a primitive type (integers,
permutations, transformations, lists, floats, strings), for which standard
OpenMath CDs are used, or a constructor applied to a list of arguments. This
lends itself perfectly to a representation in the \GAP API CD.

\ednote{MP: Put an example of OM\_Print here, maybe for a group, or for Cosets
  (as they are something that the standard OpenMath CDs in \GAP cannot do)}

\ednote{MK@MP: we also need to talk about the OM Phrasebook (see
  Section~\ref{sec:mitm}), and how that works. Actually I think that this is
  really the new part we want to discuss here.}

\subsection{\Sage API Content Dictionaries}
\ednote{MK@NT: please write something here, this can be a bit more elaborate
than the \GAP part above, since \GAP has already been described in the CICM16
paper. What is the state of the OM Phrasebook for \Sage?}

As in \GAP, a large part of the mathematical knowledge embedded in the
\Sage library is encoded using its type system. This library is
written in the \Python programming language which comes with a
traditional object oriented dynamic type system.
For example The MiTM ontology of Figure~\ref{fig:cgtontology}
translates into a hierarchy of four abstract classes (\texttt{Group},
\texttt{PermutationGroup}, \texttt{MatrixGroup},
\texttt{FinitelyPresentedGroup}) and concrete classes
(\texttt{SymmetricGroup}, \texttt{MathieuGroup},
\texttt{LinearMatrixGroup}, ...).

Altogether, the hierarchy of classes of \Sage contains thousands of
abstract and concrete classes, with heavy use of multiple inheritance.
To tame code bloat and make such a deep and large hierarchy
maintainable, \Python's type system is enriched with a category system
that collects closely related abstract classes (e.g. \texttt{Group},
\texttt{GroupElement}, \texttt{GroupMorphism}, \texttt{GroupHomset}),
together with explicitly represented mathematical knowledge, in a
so-called \emph{category} (e.g that of \texttt{Groups}).
See~\ref{Sage,Sage.Categories} for details.

In \cite{DehKohKon:iop16} we describe the use of annotations in the
code to enrich the mathematical knowledge in \Sage's categories with
alignments with other systems, notably MMT. This knowledge is then
exported to generate interfaces theories. We also describe how this
can be used to automatically generate \emph{handle interfaces} with
other systems like e.g. \GAP.

TODO: explain what a handle interface is; advantages and limitations.

TODO: next step: also export constructors to enable non-handle
interfaces where objects are actually exchanged. Besides, by nature
certain areas of \Sage (e.g. graph theory, elliptic curves, ...) have
shallow hierarchy of classes; there categories become irrelevant and
are not used. \ednote{statistics would be useful here}

TODO: using introspection to export the information; instrument
TestSuite to export all objects; parents and unique representation
objects have a constructor method. pickling by construction, ...

\subsection{\Singular API Content Dictionaries}
\ednote{MK: it would be great, if we could do this and als have an estimation of
the time used for this.}

For \Singular we used standard OpenMath CDs for polynomials and wrapped
\Singular in a thin layer of python code that provides \SCSCP
communication. This work was undertaken by the sixth author as part of a summer
internship in about a week without prior expert
knowledge of the system.\ednote{MP@MK What is the correct way of phrasing this?}

\subsection{Comparison of Approaches}

%%% Local Variables:
%%% mode: latex
%%% TeX-master: "paper"
%%% End:

%  LocalWords:  sec:apit DehKohKon:iop16 json emph emph sec:mitm fig:cgtontology
%  LocalWords:  GroupHomset
