\begin{wrapfigure}r{4.3cm}\vspace*{-2em}
  \tikzinput{mistargraph}\vspace*{-1em}
  \caption{MitM Paradigm}\label{fig:mitm}\vspace*{-1.5em}
\end{wrapfigure}
Figure~\ref{fig:mitm} shows the basic MitM design.
We want to make the systems $A$ to $H$ interoperable.
A P2P translation regime ($n^2$ translations between $n$ systems) is already intractable for the systems in the OpenDreamKit project (more than a dozen).
Alternatively, an ``industry standard'' regime, where one system's language is declared as the standard is infeasible, since no system subsumes the others in terms of coverage -- not to mention the political problems such a standardization would induce.
Instead, MitM uses a central mathematical ontology that provides an independent mediating vocabulary, via which all participating systems are aligned.
All mathematical knowledge shared between the systems and exposed to the high-level VRE user is expressed using the vocabulary of this ontology. 
Morally, the MitM ontology already exists in the published mathematical literature, we only have to flexiformalize it sufficiently to act as a system interoperability resource. 

The following sections describe the three components of the MitM paradigm in more detail.

\subsection{The MitM Ontology}\label{sec:mitm:recap}

In the center, we have the \textbf{MitM Ontology}, which is a formalization of
the mathematical knowledge behind the systems $A$ to $H$ as a theory graph in
the \OMMT format~\cite{Kohlhase:OMDoc1.2,RabKoh:WSMSML13,uniformal:on}. We do
not go into the details of \OMMT here. For our purposes, it
suffices to assume that a theory graph formalizes a language for mathematical objects as a set of typed symbols with a (formal or informal) specification of their semantics.
For example, the symbol $\mathtt{PolynomialRing}$ takes a ring of coefficients $r$ and a number $n$ of variables and returns the ring $r[X_1,\ldots,X_n]$ of polynomials.

Note that the purpose of the MitM ontology is not the formal verification of
mathematical theorems (as for most formalizations
%  such as
% \cite{Gonthier+:mcpoot13}
 for group theory), but to act as a pivot point for
integrating systems. This means that it can be much nearer to the informal but
rigorous presentation of mathematical knowledge in the literature. While each
system makes compromises and optimizations needed for a particular application
domain, the MitM ontology follows existing and already informally standardized
mathematical knowledge and can thus serve as a standard interface layer between
systems

Importantly, the MitM ontology does not have to include any definitions\footnote{Though admittedly, definitions help nail down the MitM concepts in sufficient detail. } or proofs --- it only has to declare the types of all relevant symbols and state (but not prove) the relevant theorems.
This makes it possible for users like Jane to write and extend MitM ontologies quickly, whereas other formalizations usually require extensive efforts by specialists.

\subsection{Specifying Individual System Dialects}\label{sec:mitm:dialect}

\paragraph{System Dialects}
It is unavoidable that each system induces its own language for mathematical objects.
This is the cause of much incompatibility because even subtle differences make naive integration impossible.
Moreover, due to the difficulty of the involved mathematics and the effort of maintaining the implementations, such differences are aplenty.

Fortunately, we can at least easily abstract from the user-facing surface syntax of these languages:
scalable interoperability can anyway only be achieved by acting on the internal data structures of the systems.
Thus, only the much simpler internal abstract syntax needs to be considered.

The symbols that build the abstract syntax trees can be split into two kinds: \emph{constructors} build primitive objects without involving computation, and \emph{operations}
compute objects from other objects (including predicates, which we see as operations that return booleans).
For purposes of interoperability it is possible to abstract from this distinction and consider both as typed symbols.
This abstraction is important because systems often disagree on the choice of primitive objects.
Thus, we can represent the interfaces of the systems $A$ to $H$ as \OMMT theory graphs $a$ to $h$ that declare the constructors and operations (but omit all implementations of the operations) of the respective system.

Given the theory graph $a$, we can express all objects in the language of system $A$ as \OMMT objects using the symbols of $a$.
We call this language the \textbf{\OMMT system dialect} for $A$ and refer to these objects as $A$-objects.
It is relatively straightforward to write (or even automatically generate) the theory graph $a$ and to implement a serializer and parser for $A$-objects as a part of $A$.
This is because no consideration of interoperability and thus no communication with the developers of other systems is needed.

%For example, the theory graph for \GAP contains a symbol $\mathtt{}$.

\paragraph{Alignments with the Ontology}
The above reduces the interoperability problem to relating each system dialect to the MitM ontology.
Each system dialect overlaps with the language of the ontology, but no system implements all ontology symbols and every systems implements many idiosyncratic operations that are not part of the ontology.
Some system symbols are related to corresponding symbols in the MitM ontology. Then we can use the MitM ontology as an intermediate representation to bridge between any two systems, e.g., by translating $A$-objects to the corresponding ontology objects and then those to the corresponding $B$-objects.

However, even when $A$ and $B$ deal with the ``same mathematical objects'', these may be constructed and represented differently, e.g., symbols can differ in name,
argument order/number, types, etc.
A major difficulty for system interoperability is bridging these differences.
To formalize the details of this relation, \cite{MueGauKal:cacfms17} introduced \textbf{\OMMT alignments}.
Technically, these are pairs of \OMMT symbol identifiers decorated by a set of key-value pairs.
Whenever two systems' APIs declare symbols for the same mathematics operation, the alignments with the MitM ontology determine which ontology objects correspond to system objects.

The alignment of $a$-symbols to ontology symbols must be spelled out manually.
But this is usually straightforward and easy even for inexperienced users.
\ednote{give an example of an alignment here, e.g., for PolynomialRing}

Thus we can reduce the problem of interfacing $n$ systems to
\begin{inparaenum}[\em i\rm)]
\item curating a MitM ontology for the joint mathematical domain,
\item generating $n$ theory graphs for system dialects,
\item maintaining $n$ collections of alignments into the MitM ontology.
\end{inparaenum}

Alignments form an independent part of the MitM interoperability infrastructure.
Incidentally they obey a distinct development schedule: the MitM ontology will be under continuous development --- it formalizes the common knowledge of the community about a mathematical domain.
The system dialects are released together with the systems according to their respective development cycle.
The alignments bridge between them and have to mediate these cycles.

\subsection{MitM-based Distributed Computation}\label{sec:mitm:comms}

The final missing piece for a system interoperability layer for a VRE toolkit is a practical way of transporting objects between systems.
This requires two steps.

Firstly, if the system theory graphs and alignments are known, we can automatically translate $A$-objects to $B$-objects in two steps: $A$ to ontology and ontology to $B$.
This two-step translation has been implemented in \cite{MueRoYuRa:abtafs17} based on the \MMT system~\cite{Rabe:MAGMS13,uniformal:on} -- an implementation of the \OMMT format and a set of mathematical knowledge management algorithms.

Secondly, each system $A$ has to be able to serialize/parse $A$-objects and to send them to/receive them from \MMT.
In the OpenDreamKit project we use the OpenMath \SCSCP (Symbolic Computation Software Composability) protocol~\cite{SCSCP-1.3} for that. 
%\SCSCP is essentially a distributed remote-procedure-call system based on OpenMath, which is itself the fragment of \OMMT used for representing objects.
It is straightforward to extend a parser/serializer for $A$-objects to an \SCSCP clients/server by implementing the \SCSCP protocol on top of, e.g., sockets or using an existing \SCSCP library. 

% \begin{oldpart}{FR: all of this are new results and should be moved to the next section. In any case, this example does not fit well with the case study and maybe should be removed altogether. Instead, I think we should specify the problem solved by the case study in the following sections.}
% For example, say Jane wishes to check the hypothesis that all Abelian transitive groups are cyclic.\footnote{This example is mathematically trivial.
% We have chosen it because it shows complex system interaction rather than mathematical plausibility.
% A realistic, mathematically motivated example will be discussed below.}
% using a MitM-based mathematical VRE.
% Jane realizes that the LMFDB database contains a set of known Abelian transitive groups.
% Furthermore, she realizes that the \GAP system can check if a given group is cyclic. 
% To check his hypothesis, she could retrieve all relevant groups from LMFDB, and then use \GAP to check if each of these is cyclic. 

% As Jane is most familiar with the \Sage CAS, she expresses this operation as
% \begin{lstlisting}
% gap   = server(system="gap", transport="scscp", mediator="mmt")
% lmfdb = server(system="lmfdb", transport="scscp", mediator="mmt")
% all(gap.is_cyclic(g) for g in lmfdb.query("transitive_groups", abelian=True))
% \end{lstlisting}

% \begin{wrapfigure}r{6.6cm}\vspace*{-2em}
%   \tikzinput{mitmcomm}\vspace*{-1em}
%   \caption{MitM-based Interoperability}\label{fig:mitmcomm}\vspace*{-1em}
% \end{wrapfigure}

% At the \Sage level, the \lstinline|all| predicate \ednote{MK@NT: is
%   ``all'' something that already exists in \Sage? Does it behave like
%   this? Please make sure that the text in this paragraph makes sense?
%   NT: Yes, ``all'' is a Python builtin with exactly that syntax and behavior.}

%  maps the first argument, interpreted as a predicate over the list in the second argument.
% The arguments are the results of remote procedure calls via \SCSCP. 
% The second one sends the query for transitive groups to the LMFDB \SCSCP server, via \SCSCP protocol and the second sends the query whether \lstinline|g| (from the list returned by LMFDB) is cyclic.
% At the system level, the interaction is mediated by the \MMT system and we have the following atomic communication acts: 
% \begin{compactenum}
% \item \Sage sends a \SCSCP remote procedure call for 
%   \begin{lstlisting}[mathescape]
%     map(scscp(gap,is_cyclic,scscp(lmfdb,query("transitivegroup")))$\text{\footnotemark}$
% \end{lstlisting}
%   \footnotetext{We are assuming that \textsf{all} in \Sage is extended to produce the
%     map term when it has \textsf{scscp} arguments.}  to \MMT as an OM object $Q$ in \Sage
%   dialect\ednote{MK@NT; could you please Sageify this as well? Or do we not need that?}
%   \ednote{NT@all: I don't think that the behavior of ``all'' can be
%     changed; Sage will be evaluating the ``all'' call with one query
%     to scscp/lmfdb and on scscp/gap call per group `g`. Do we have a
%     specific reason to wish for Sage to issue a single ``map'' call to
%     scscp? If yes, we need a different syntax. If not we just have to update
%     this itemized description.}
% \item \MMT interprets the inner term \lstinline|scscp(lmfdb,query("transitivegroup"))|, translates it into a LMFDB query $Q'$ in LMFDB dialect and sends it to the LMFDB \SCSCP server via \SCSCP. 
% \item LMFDB answers the query with a list $34G$ of 34 transitive groups as OM objects in LMFDB dialect.
% \item \MMT translates $34G$ into the \GAP dialect and synthesizes a the OM object $M$ that calls the \GAP map operation of the (\GAP equivalent of )\lstinline|is_cyclic| on $34G$. This is sent to \GAP. 
% \item \GAP interprets $G$, and returns a list of 34 (\GAP) truth values to \MMT. 
% \item \MMT translates them into a list of 34 \Sage truth values, which it sends back to \Sage. 
% \item \Sage interprets this list as a \Sage list and the \lstinline|every| predicate checks that all elements are ``true'', in which case it succeeds. 
% \end{compactenum}
% \end{oldpart}

%%% Local Variables:
%%% mode: latex
%%% TeX-master: "paper"
%%% End:

%  LocalWords:  sec:mitm Jupyter jupyter-project:on KohDavGin:psewads11 BusCapCar:2oms04 standardization DehKohKon:iop16 textbf RabKoh:WSMSML13,uniformal:on MueGauKal:cacfms17 sec:apit serialize normalizer mathcal compute-the-normalizer MitM-based centering mitmcomm MueRoYuRa:abtafs17 Rabe:MAGMS13,uniformal:on inparaenum Sagey lstlisting scscp mmt,is_cyclic scscp lmfdb mmt,query transitivegroup lstinline gap,is_cyclic,scscp lmfdb,query synthesizes mathescape Sageify mathtt ldots,X_n Gonthier
%  LocalWords:  wrapfigure vspace tikzinput mistargraph fig:mitm mitm_poc emph mcpoot13
%  LocalWords:  formalizations optimizations standardized serializer oldpart behavior
%  LocalWords:  itemized
