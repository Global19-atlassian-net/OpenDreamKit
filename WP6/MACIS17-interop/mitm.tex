\section{Math-in-the-Middle Interoperability}\label{sec:mitm}

To build a VRE toolkit from open-source (or open-API\footnote{It turns out that it is
  sufficient to have a well-documented API that gives access to the internal
  (mathematical) data structures and functionality, so closed-source systems can also be
  part of the MitM paradigm if they do.}) systems, we need a joint user interface -- the
OpenDreamKit project adopts Jupyter~\cite{jupyter-project:on} and active
documents~\cite{KohDavGin:psewads11} -- and an interoperability layer that allows to pass
problems and results between the disparate systems. 

For the latter -- and that is the topic of this paper -- we need a way to mediate between
the interaction layers of the respective systems.  \ednote{NT: Some grammar issues in the
  following sentence which make it a bit confusing; not sure how to fix it myself.}
Generally, mathematical software systems apply operations on mathematical objects, which
can be expressed as applications of domain-specific constructors to primitive objects like
numbers and other objects. The results of these operations are again such objects. If we
abstract from issues of surface syntax -- which we can by assuming that the systems use a
standard like OpenMath~\cite{BusCapCar:2oms04}, the interface languages differ in their
vocabularies -- the operations and constructors. The difficulty in the translation of
objects and functionalities between systems is to \emph{align} the respective
vocabularies:

Even when $A$ and $B$ deal with the mathematically ``same objects'',
these may be constructed differently; e.g. using constructors in
system $A$ and operations in $B$ or vice versa. Even if $A$ and $B$
share the same constructors and operations, these can differ in
argument order/number, types, etc.

\subsection{The MitM Paradigm}\label{sec:mitm:recap}

Obviously, a P2P translation regime ($n^2$ translations between all systems) is already
intractable for the systems in the OpenDreamKit project (more than a dozen), and an
``industry standard'' regime where one interaction language is declared as a ``standard''
is infeasible, since no system subsumes the others in terms of
coverage -- not mentioning the
political problems such a standardization would induce. Instead we have proposed the
``Math-in-the-Middle'' (MitM) approach, where we use ``mathematical knowledge'' as an
independent mediating vocabulary, and all system vocabularies are aligned to that. After
all, the mathematics behind the systems is published using this vocabulary.

\begin{wrapfigure}r{4.3cm}\vspace*{-2em}
  \tikzinput{mistargraph}\vspace*{-1em}
  \caption{MitM Paradigm}\label{fig:mitm}\vspace*{-1.5em}
\end{wrapfigure}
Figure~\ref{fig:mitm} shows the setup of the MitM paradigm -- see~\cite{DehKohKon:iop16}
for details. In the center, we have the \textbf{MitM Ontology}, which is a modular
flexiformalization of the mathematical knowledge behind the systems $A$ to $H$ as a theory
graph in the OMDoc/MMT Format~\cite{Kohlhase:OMDoc1.2,RabKoh:WSMSML13,uniformal:on}. For
each of the systems $A$ to $H$, we generate a theory graph \textbf{API Content
  Dictionaries}\footnote{In~\cite{DehKohKon:iop16} we had called these ``interface
  theories'', but we would like to reserve this term for certain theories in the MitM
  ontology.}\ednote{MK: we may want to change this term before we submit; let's discuss!}
in OMDoc/MMT which is aligned with the corresponding symbols in the MitM ontology. As the
API CDs and the MitM Ontology are both in OMDoc/MMT -- inside the cloud in
Figure~\ref{fig:mitm} -- we can make use of \textbf{OMDoc/MMT
  alignments}~\cite{MueGauKal:cacfms17} which have been developed in the meantime for the
MitM alignment relation.\ednote{MK: talk about interface theories somewhere below}

\subsection{OpenMath System Dialects and Alignment-based Translation}\label{sec:mitm:dialect}

Given the collection $s$ of API CDs of a system $S$ (see section~\ref{sec:apit} for
examples), we can express all the API objects in the interface language of $S$ as OpenMath
objects that only use symbols from $s$.  We call this language the \textbf{OpenMath system
  dialect} for $S$ and refer to OM objects in this language as $S$-objects.  If system $S$
has an OpenMath \textbf{phrasebook} -- i.e. an OpenMath interface that allows to serialize
and parse all API objects of $S$ as OpenMath objects in the system dialect of $S$
(see~\cite[Section 1.5]{BusCapCar:2oms04}) -- then $S$ can use OpenMath (concretely
$S$-objects) for communication. As we have included the operations in the system API CDs
of $S$, we can express all the functionality of $S$ in terms of OpenMath objects as well :
if we want GAP to compute the normalizer of a group $G$, we can express this by the
OpenMath object $\mathcal{O}$ which is the application of the symbol $n$ to the OM object
representing $G$, where $n$ is the symbol from the GAP API CDs that represents the
``compute-the-normalizer'' operation.  Thus we can request GAP to compute the normalizer
of $G$ by sending $\mathcal{O}$ to GAP, which then decodes it in its phrasebook into a GAP
term, which it then computes, returning the result as an OM object $\mathcal{O}'$, again
via the GAP phrasebook.

Note that having OpenMath phrasebooks for all systems involved does not give us system interoperability yet, since all systems have different system dialects: 
To delegate a computation from $A$ to $B$, we need to translate $A$-objects into $B$-objects for the ``remote procedure call'' and vice versa for the results. This is where the OMDoc/MMT alignments come into play. 
Given a sufficiently complete set of MitM alignments we can translate between systems dialects using techniques from~\cite{MueRoYuRa:abtafs17}. This translation has been implemented in the MMT system~\cite{Rabe:MAGMS13,uniformal:on}   -- an implementation of the OMDoc/MMT format and a set of mathematical knowlege management algorithms. 
Thus we can reduce the problem of  interfacing $n^2$ systems to
\begin{inparaenum}[\em i\rm)]
\item curating a MitM ontology for the joint mathematical domain,
\item generating $n$ collections of API CDs, and 
\item maintaining $n$ collections of alignments into the MitM ontology.
\end{inparaenum}

\subsection{MitM-based Distributed Computation}\label{sec:mitm:comms}

The final missing piece for a system interoperability layer for a VRE toolkit is a way of transporting OM objects between systems. 
In the OpenDreamKit project we use the OpenMath SCSCP (Symbolic Computation Software Composability) protocol~\cite{SCSCP-1.3} for that. 
SCSCP is essentially a distributed remote-procedure-call system based on OpenMath. 
Procedure calls and results are expressed as OM objects as exemplified above; systems with an OpenMath phrasebooks can be extended to  SCSCP clients/servers by implementing the SCSCP protocol on top of e.g. sockets (or use a SCSCP library). 

The situation is best illustrated by an example. Say John wishes to check the hypothesis that all abelian transitive groups are cyclic\footnote{This example is mathematically trivial; we have chosen it because it shows complex system interaction rather than mathematical plausibility; a realistic, mathematically motivated example will be discussed below.} using a MitM-based mathematical VRE?  
John realizes that the LMFDB database contains a set of known abelian transitive groups.
Furthermore, she realizes that the GAP system can check if a given group is cyclic. 
To check his hypothesis, he could first retrieve all relevant groups from LMFDB, and then use GAP to check if each of these is cyclic. 

As John is most familiar with the SageMath CAS, we expresses the this operation as \ednote{MK@NT: this is made up, can you make it more Sagey?}
\begin{lstlisting}
every ([g] scscp(gap|mmt,is_cyclic(g)),scscp(lmfdb|mmt,query("transitivegroup"))) 
\end{lstlisting}

\begin{wrapfigure}r{6.6cm}\vspace*{-2em}
  \tikzinput{mitmcomm}\vspace*{-1em}
  \caption{MitM-based Interoperability}\label{fig:mitmcomm}\vspace*{-1em}
\end{wrapfigure}
At the SageMath level, the \lstinline|every| predicate maps the first argument, interpreted as a predicate over the list in the second argument. 
The arguments are the results of remote procedure calls via SCSCP. 
The second one sends the query for transitive groups to the LMFDB SCSCP server, via SCSCP protocol and the second sends the query whether \lstinline|g| (from the list returned by LMFDB) is cyclic.
At the system level, the interaction is mediated by the MMT system and we have the following atomic communication acts: 
\begin{compactenum}
\item SageMath sends a SCSCP remote procedure call for 
  \begin{lstlisting}[mathescape]
    map(scscp(gap,is_cyclic,scscp(lmfdb,query("transitivegroup")))$\text{\footnote {We are assuming that \textsf{every} in SageMath is extended to produce the map term when it has \textsf{scscp} arguments.}}$
\end{lstlisting}
  to MMT as an OM object $Q$ in SageMath dialect.  
\item MMT interprets the inner term \lstinline|scscp(lmfdb,query("transitivegroup"))|, translates it into a LMFDB query $Q'$ in LMFDB dialect and sends it to the LMFDB SCSCP server via SCSCP. 
\item LMFDB answers the query with a list $34G$ of 34 transitive groups as OM objects in LMFDB dialect.
\item MMT translates $34G$ into the GAP dialect and synthesizes a the OM object $M$ that calls the GAP map operation of the (GAP equivalent of )\lstinline|is_cyclic| on $34G$. This is sent to GAP. 
\item GAP interprets $G$, and returns a list of 34 (GAP) truth values to MMT. 
\item MMT translates them into a list of 34 SageMath truth values, which it sends back to SageMath. 
\item SageMath interprets this list as a Sage list and the \lstinline|every| predicate checks that all elements are ``true'', in which case it succeeds. 
\end{compactenum}

In the next sections, we will present a case study that instantiates the general paradigm
presented in this section with a concrete domain, systems, and a general implementation. 


%%% Local Variables:
%%% mode: latex
%%% TeX-master: "paper"
%%% End:

%  LocalWords:  sec:mitm Jupyter jupyter-project:on KohDavGin:psewads11 BusCapCar:2oms04 standardization DehKohKon:iop16 textbf RabKoh:WSMSML13,uniformal:on MueGauKal:cacfms17 sec:apit serialize normalizer mathcal compute-the-normalizer MitM-based centering mitmcomm MueRoYuRa:abtafs17 Rabe:MAGMS13,uniformal:on inparaenum Sagey lstlisting scscp mmt,is_cyclic scscp lmfdb mmt,query transitivegroup lstinline gap,is_cyclic,scscp lmfdb,query synthesizes mathescape
%  LocalWords:  wrapfigure vspace tikzinput mistargraph fig:mitm mitm_poc emph
