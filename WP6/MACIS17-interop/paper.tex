\documentclass{llncs}
\pagestyle{plain}
\usepackage[show]{ed}
\usepackage[utf8]{inputenc}
\usepackage{xspace}
\usepackage[style=alphabetic,backend=bibtex,isbn=false]{biblatex}
\addbibresource{../../lib/kbibs/kwarcpubs.bib}
\addbibresource{../../lib/kbibs/extpubs.bib}
\addbibresource{../../lib/kbibs/kwarccrossrefs.bib}
\addbibresource{../../lib/kbibs/extcrossrefs.bib}
\addbibresource{rest.bib}% add bibs here!
\renewbibmacro*{event+venue+date}{}
\renewbibmacro*{doi+eprint+url}{%
  \iftoggle{bbx:doi}
    {\printfield{doi}\iffieldundef{doi}{}{\clearfield{url}}}
    {}%
  \newunit\newblock
  \iftoggle{bbx:eprint}
    {\usebibmacro{eprint}}
    {}%
  \newunit\newblock
  \iftoggle{bbx:url}
    {\usebibmacro{url+urldate}}
    {}}

\usepackage{hyperref}
\title{REGULAR-T1: Knowledge-Based Interoperability for Mathematical Software Systems}
\author{
Victor Aether\inst{2} 
 Michael Kohlhase\inst{1} 
Dennis M\"uller\inst{1} 
Markus Pfeiffer\inst{2} 
Florian Rabe\inst{2} 
Nicolas~M.~Thiéry\inst{3} 
Tom Wiesing\inst{2}
}

\institute{
   FAU Erlangen-N\"urnberg
   \and University of St~Andrews 
   \and Universit\'e Paris-Sud
}
\begin{document}
\maketitle
\begin{abstract}
  \ednote{twb}
\end{abstract}

\section{Introduction}\label{sec:intro}
\ednote{We build on~\cite{DehKohKon:iop16} (MitM paradigm) and \cite{MueGauKal:cacfms17}
(alignments).}
\begin{todolist}{follow this}
\item generally we want to show that the promises in the CICM paper become reality
\item show the generated system ontologies taking GAP and SageMath as examples (and
  compare them)
\item with the example of GAP group theory make the MitM ontology part , in the levels
  (abstract, concrete subgroup theory, computational), show 
  \begin{itemize}
  \item star-formed alignments and how we come by them, 
  \item talk about SCSCP and GAP/Sage/ dialects and intra-MMT translation
  \item running example use case is to act on singular polynomials with GAP permutation
    groups
  \end{itemize}
\item what is the ``business model'' for the general MitM-based cooperation model? 
\end{todolist}
\section{Conclusion}\label{sec:concl}
\ednote{how can YOU help extend the MitM?}\ednote{investment: small network joining
  costs in the size of the system, rewards: network effect in the size of the network.}
\printbibliography
\end{document}
%%% Local Variables:
%%% mode: latex
%%% TeX-master: t
%%% End:

%  LocalWords:  maketitle twb sec:intro DehKohKon:iop16 MueGauKal:cacfms17
