Based on the \OMMT representation of the MitM ontology and the individual system and the alignments between them, we build a distributed group theory computation system from \GAP and \Singular systems.

\begin{figure}[ht]\centering
  \tikzinput{gap_singular_mitm_fig}
  \caption{MitM Interaction for \GAP-\Singular}\label{fig:mitmpoc}
\end{figure}

Figure~\ref{fig:mitmpoc} shows the overall architecture. The MathHub
system~\cite{IanJucKoh:sdm14,MathHub:on} acts as a versioned repository for the \OMMT
resources and includes an instance of the \MMT system~\cite{Rabe:MAGMS13} that provides
mathematical knowledge management services (type checking, HTML5 presentation,
alignment-based translation, etc.) for \OMMT resources.

We have extended \MMT with a \SCSCP client/server implementation~\cite{twiesing:msc17} in order to act as a central mediator.
For the \GAP server, we built on pre-existing \SCSCP support.
To obtain an \SCSCP server for \Singular, which does not have native \SCSCP support, we wrapped \Singular in a python script that includes the \lstinline|pyscscp| library~\cite{py-scscp:on}.
Finally, we add a \Python control script~\cite{MitM-PoC} as an example how to directly program the interaction and a \Sage client.
Both allow to specify the computation.

% \begin{oldpart}{MK: just copied here; Victor writes\\
%     ``\emph{A peer-to-peer connection must be made with the CAS servers, so that CAS
%       servers can, in turn, query MitM if during a computation they encounter a concept
%       that lies outside their field of knowledge. In application to this particular case,
%       it would be cleaner if, instead of asking MitM to produce permutations of a list,
%       the client simply queries MitM for the orbit of a polynomial by defining an action
%       of a member of the symmetric group on a polynomial. \GAP would then be able to
%       calculate the orbit by making the group act on the polynomial with the described
%       action and querying MitM for equality of polynomials, resulting in a linear-time
%       algorithm instead of quadratic-time behaviour displayed by the current client.}''  I
%     do not quite understand the maths here, maybe we can stillmake this happen?}
%   The control script follows the procedure:\ednote{MK: for this to make sense we would
%     have to describe what problem we want to solve.}
% \begin{enumerate}
%   \item Create an OpenMath polynomial.
%   \item Obtain a symmetric group of size that is equal to the number of variables 
%     in the polynomial from MitM.
%   \item Using the obtained group, query MitM for all permutations of the list 
%     of variables.
%   \item Create polynomials from the permutations of the list of variables.
%   \item Filter out the duplicate polynomials by querying MitM for equality of 
%     polynomials.
% \end{enumerate}
% While this is very much a brute-force algorithm to calculate an orbit of a
% polynomial, it showcases the ability of the client to query the MitM server that 
% is then forced to use multiple CAS without the client needing any knowledge of the
% underlying systems.
% \end{oldpart}
The \Sage client behaves exactly as described in
Section~\ref{sec:mitm:comms}\ednote{MK@NT/TW; it seems that we will have time to
  implement this after the extension. So we should make it happen.}

%%% Local Variables:
%%% mode: latex
%%% TeX-master: "paper"
%%% End:

%  LocalWords:  sec:case fig:mitmpoc IanJucKoh:sdm14,MathHub:on summarize sec:cgt pyscscp
%  LocalWords:  twiesing:msc17 centering tikzinput gap_singular_mitm_fig lstinline emph
%  LocalWords:  py-scscp:on oldpart
