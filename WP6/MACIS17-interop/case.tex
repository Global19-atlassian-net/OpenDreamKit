\section{Computational Group Theory Case Study}\label{sec:case}
Based on the MitM ontology fragment, the API CDs, and the alignments above, we build a
distributed group theory computation system from \GAP and \Singular
systems. Figure~\ref{fig:mitmpoc} shows the overall architecture. The MathHub
system~\cite{IanJucKoh:sdm14,MathHub:on} acts as a versioned repository for the \OMMT
resources and includes an instance of the \MMT system~\cite{Rabe:MAGMS13} that provides
mathematical knowledge management services (type checking, HTML5 presentation,
alignment-based transation, etc.) for \OMMT resources. For the purposes of the MitM
architecture, \MMT has been extended with a \SCSCP client/server
implementation~\cite{twiesing:msc17}. For the \GAP server, we could build on pre-existing
\SCSCP support, but \Singular does not have it, so we wrapped it in a python script that
includes the \lstinline|pyscscp| library~\cite{py-scscp:on}. Finally, we 

\begin{figure}[ht]\centering
  \tikzinput{gap_singular_mitm_fig}
  \caption{MitM Interaction for \GAP-\Singular}\label{fig:mitmpoc}
\end{figure}

%%% Local Variables:
%%% mode: latex
%%% TeX-master: "paper"
%%% End:

%  LocalWords:  sec:case fig:mitmpoc IanJucKoh:sdm14,MathHub:on summarize sec:cgt
%  LocalWords:  twiesing:msc17 centering tikzinput gap_singular_mitm_fig
