\section{Introduction}\label{sec:intro}

There is a large and vibrant ecosystem of open-source mathematical software systems. 
These systems can range from calculators, which are only capable of performing simple computations, via mathematical databases (curating collections of a mathematical objects), to powerful modeling tools and computer algebra systems (CAS).

Most of these systems are very specific -- they focus on one or very few aspects of mathematics. 
For example, the ``Online Encyclopedia of Integer Sequences'' (OEIS~\cite{Sloane:oeis12,oeis}) focuses on sequences over $\mathbb{Z}$ an their properties and the ``L-Functions and Modular Forms Database'' (LMFDB)~\cite{Cremona:LMFDB16,lmfdb:on} objects in number theory pertaining to Langland's program. 
\GAP~\cite{GAP:on} excels at discrete algebra, whereas \Sage~\cite{SageMath:on} focuses on Algebra and Geometry in general, and \Singular~\cite{singular:on} on polynomial computations, with special emphasis on commutative and non-commutative algebra, algebraic geometry, and singularity theory.

\ednote{Shall we mention any of the non open but well known one?
  Mathematica, Magma, ...}

\ednote{NT: \Sage's focus on Algebra and Geometry was in the very
  beginning; maybe it could be described last, as “aims to be a
  general purpose software for computational pure mathematics, by
  integrating many other systems, including the aforementioned”,
  currently using ad-hoc methods.}

For a mathematician however (a user; let us call her Jane) the systems themselves are not relevant; instead she only cares about being able to solve problems. 
Typically, it is not possible to solve a mathematical problem using only a single program. 
Thus Jane needs to work with multiple systems and combine the results to reach a solution. 
Currently there is very little help with this practice, so Jane has to isolate sub-problems the respective systems are amenable to, formulate them into the respective input language, collect results, and reformulate them for the next system --- a tedious and error-prone process at best, a significant impediment to scientific progress in its overall effect. 
Solutions for some situations certainly exist, which can help get Jane unstuck, but these are ad-hoc and for specific, often-used system combinations only. 
Each of these requires a lot of maintenance and does not scale to a larger set of specialist systems. 

The OpenDreamKit project, which aims at a mathematical VRE toolkit, proposes the Math-in-the-Middle (MitM~\cite{DehKohKon:iop16}) paradigm, an interoperability framework based on a flexiformal
representation of mathematical knowledge and aligns this with system-generated interface theories. 

In this paper we instantiate the MitM paradigm with a concrete domain
development and evaluate it on a distributed computation involving \GAP, \Sage, and \Singular.\ednote{ we generally we}
We will use the following example as a running example: Jane wants to act on singular polynomials with \GAP permutation groups\ednote{MK@(MP or VA): please expand and explain}
\ednote{MP: We might not be that far away from computing with \Singular \emph{polynomial ideals}, which is a really interesting use-case}.

In Section~\ref{sec:mitm} we will recap the MitM paradigm, clarify the role of the respective abstract system languages, and  detail distributed computation via the OpenMath \SCSCP protocol.
The rest of the paper instantiates this abstract paradigm in an an extended case study; we develop instances for all the components: Section~\ref{sec:cgt} presents a MitM ontology fragment for computational group theory, Section~\ref{sec:apit} discusses system API content dictionaries for \GAP, \Sage, and \Singular, the systems to be integrated in the case study, and Section~\ref{sec:mitm_poc} presents the ``VRE fragment'' built on \GAP and \Singular in action. 
Section~\ref{sec:concl} concludes the paper and compares MitM-based interoperability with ad-hoc implementations in terms of scalability and maintenance cycles. 

%%% Local Variables:
%%% mode: latex
%%% TeX-master: "paper"
%%% End:

%  LocalWords:  sec:intro Sloane:oeis12,oeis mathbb Cremona:LMFDB16,lmfdb:on GAP:on \Sage:on singular:on DehKohKon:iop16 emph sec:mitm sec:cgt sec:apit sec:mitm_poc sec:concl MitM-based
