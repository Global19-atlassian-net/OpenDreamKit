\subsection{Alignments}
\ednote{talk about alignments from the IFT to the CGT, how they work, building
  on~\cite{MueRoYuRa:abtafs17,MueGauKal:cacfms17}} 

The initial alignments are currently produced by hand, but from some of the
initial alignments and the \GAP API CDs we will be able to infer more alignments
automatically.

For example, the filter \texttt{IsGroup} is aligned with \texttt{Group}, and the
filter \texttt{IsPermGroup} is aligned with \texttt{Subgroup SymmetricGroup
  [1..n]}.
\ednote{MP: Need to be more concrete here, in particular we should maybe
  describe how \GAP's notion of an action homomorphism translates through this?
  Also is this even correct?}

We formalised the theory of symmetric groups of a set; in \GAP permutation groups
are represented as subgroups (with finite support) of the symmetric group of
$\mathbb{N}+$, and often one concretely has an isomorphism between the group one
is interested in and a subgroup of $S_{\mathbb{N}+}$, for example
via a group action.

\texttt{SylowSubgroup}s are more difficult: They are special groups in their
own right, namely groups whose size is a prime-power, but we also want them
to be identified with a certain subgroup of the group we are working
with.\ednote{MP: While I believe this to be an excellent additional example
  for MMT formalisation, this could be going too far for this paper}

\ednote{MP@ALL: We might want to be a bit careful/mention implementations of group
  theory for example in COQ where they did the Odd-Order-Proof?}

%%% Local Variables:
%%% mode: latex
%%% TeX-master: t
%%% End:
