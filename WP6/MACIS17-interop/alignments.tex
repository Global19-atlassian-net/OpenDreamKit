\section{Alignments}\label{sec:alignments}
The third component in the MitM paradigm consists of a set of alignments between the
system API CDs (cf. Section~\ref{sec:apit}) and the MitM ontology (Section~\ref{sec:cgt}). 

MitM alignments are essentially pairs of MMT URIs -- all types, constructors, and
operations represented in OMDoc/MMT have one -- decorated by an open set of keywords;
see~\cite{MueGauKal:cacfms17} for details. While the system API and the MitM Ontology form
the scaffolding for the MitM integration framework, the MitM alignments form the bridges
between the system.

A bridge between two systems $X$ and $Y$ consists of two alignments: an $X$-to-MitM an
MitM-to-$Y$ alignment composed to form an $X$-to-$Y$ alignment. With the alignment-based
translation in the MMT system (see~\ref{MueRoYuRa:abtafs17}) we can directly translate
between $X$-objects and $Y$-objects. 

Alignments form an independent part of the MitM interoperability
infrastructure. Incidentally they obey a distinct development schedule: if we look back at
Figure~\ref{fig:mitm}, then we see that the MitM ontology will be under continuous
development -- it formalizes the common knowledge of the community about a mathematical
domain. The system API CDs are released with the systems according to their respective
development cycle. The alignments bridge between them and have to mediate these cycles.

The initial alignments were produced by manually, but from some of the initial alignments
and the \GAP API CDs we will be able to infer more alignments automatically.  For example,
the filter \texttt{GAP:IsGroup}\ednote{MK: introduce the namespaces above} is aligned with
\texttt{mitm:Group}, and the filter \texttt{GAP:IsPermGroup} is aligned with
\texttt{mitm:Subgroup SymmetricGroup [1..n]}.  \ednote{MP: Need to be more concrete here,
  in particular we should maybe describe how \GAP's notion of an action homomorphism
  translates through this?  Also is this even correct?}

We formalised the theory of symmetric groups of a set; in \GAP permutation groups
are represented as subgroups (with finite support) of the symmetric group of
$\mathbb{N}+$, and often one concretely has an isomorphism between the group one
is interested in and a subgroup of $S_{\mathbb{N}+}$, for example
via a group action.

\texttt{SylowSubgroup}s are more difficult: They are special groups in their
own right, namely groups whose size is a prime-power, but we also want them
to be identified with a certain subgroup of the group we are working
with.\ednote{MP: While I believe this to be an excellent additional example
  for MMT formalisation, this could be going too far for this paper}

\begin{newpart}{MK@DM+NT: please check this }
  Another source of alignments are the existing ad-hoc \Sage-to-$X$ translations. These
  are (mainly) given as \Sage code annotations that relate \Sage operations and
  constructors with those of system $X$. We can harvest\ednote{MK@NT: I think you already
    do that, if so, please document above, if not, please do!} those as alignments between
  the \Sage API CDs and $X$ API CDs. These can be combined with the \Sage-to-MitM
  alignments into $X$-to-MitM alignments. While $X$-to-$Y$ alignments are the application,
  $X$/$Y$-to-MitM alignments scale better, since they can be arbitrarily combined and make
  the induced knowledge management workflows star-shaped.
\end{newpart}

%%% Local Variables:
%%% mode: latex
%%% TeX-master: "paper"
%%% End:

%  LocalWords:  sec:alignments sec:apit sec:cgt MueGauKal:cacfms17 MueRoYuRa:abtafs17
%  LocalWords:  fig:mitm mitm:Group mitm:Subgroup mathbb SylowSubgroup newpart
