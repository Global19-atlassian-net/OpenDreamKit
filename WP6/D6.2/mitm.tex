The \pn project intends to integrate multiple mathematical software systems into a VRE
toolkit. These systems are constituted by large collections of algorithms manipulating
highly optimized data structures representing mathematical objects with the intent of
solving specific computational problems. These systems overlap in the mathematical
objects they cover and the problems they can solve, but every system has aspects that are
not covered by any other system (as efficiently or generally). In particular, algorithms,
implementation languages, and data structures differ significantly between systems and are
optimized to their particular domain and intended performance profile. As the systems
represent decades worth of experience and development, a re-implementation is prohibitive
in cost and might lead to systems with greater coverage, but less efficiency.

Given this situation, the integration problem in \pn becomes a problem of establishing an
interoperability layer between systems. As we have seen in the previous section, the
mathematical knowledge --- for specifying the computational problems --- can be expressed
and made interoperable via views in the OMDoc/MMT format, specifying the exact data
structures and intended behavior of software systems --- and possibly verifying that the
implementation conforms to this is the realm of ``Formal Methods''. Again, the effort of
doing this for any of the systems in \pn is prohibitive and way beyond the scope of the
project.

\subsection{Specification of Interfaces}
If we analyze mathematical software from a specification-based viewpoint, then we see
three levels:
\begin{compactenum}[\rm\bf S1.]
\item \textbf{Math Specification} represents the underlying mathematical knowledge and
  the computational problems of the domain in a system-independent way.
\item \textbf{Interface Specification} represents the interfaces of mathematical software
  systems: the abstract data structures, and the input/output behavior (and possible
  side-effects) of the user-visible functions and procedures provided by the system.
\item \textbf{Implementation} gives concrete implementations of the interface
  specification in a specific programming language.
\end{compactenum}

Most modern programming languages support the organization of programs into software
libraries by separating the specification (\textbf{S2.}) and implementation levels
(\textbf{S3.}), allowing multiple implementations of a single interface specification. Examples include
abstract vs. concrete classes in object-oriented programming, signatures vs. structures in
SML, header files vs. C files in C, and operations vs. methods in \GAP.

In all of these languages, the interface specification level is utilized for intra-library interoperability, making use
of the more abstract description of the interface specification that can be instantiated
by its various implementations. The interface specifications usually tie the names of the
interface functions to argument and result types.

The specification of intended behavior is usually left to documentation facilities. This is the domain of the math specification
level (\textbf{S1.}).
This level is only marginally supported by programming languages, but a
central concern in the \pn project.
The math specification level differs from the other levels by using some kind of logical system that can express universal properties like
$\forall x,y. x=\operatorname{sqrt}(y) \Leftrightarrow x^2=y$.

Fortunately, we do not need to specify and verify all existing systems in order to integrate them into a VRE toolkit: The specification (of objects and intended behavior) at the mathematical and interface level is sufficient. In particular quality control
(establishment of correctness of the implementations) can be left to other
means\footnote{In particular, it is independent of interoperability of mathematical
  software systems.} and as a result we can resort to more lightweight methods for
establishing interoperability.

\subsection{The Math-in-the-Middle Paradigm}

In the \pn project we want to cover the software aspect of a math VRE toolkit via an
approach we call ``Math-In-The-Middle'' paradigm (MitM; see~\cite{DehKohKon:iop16} for
details and Figure~\ref{fig:mitm} for an overview diagram). In contrast to most
programming languages, the MitM paradigm concentrates on levels \textbf{S1.} and \textbf{S2.},
represents them in the OMDoc/MMT format and leaves the implementation (\textbf{S3.}) to
the respective systems.

Here the underlying mathematical knowledge (level \textbf{S1.}), the ``real math'', is
used as a reference ontology (in the ``middle'' -- hence the name) for the math VRE
toolkit. This ontology is represented as an OMDoc/MMT theory graph $M$ as described in the
previous section.

Additionally, for every systen in the \pn VRE toolkit we establish an interface
specification as an OMDoc/MMT theory graph $I$ and link it to the MitM ontology via
\textbf{interface views}. These fulfil two purposes: they align the namespaces of the
systems with the math specification and they specify the intended behavior of the systems
in terms of the MitM ontology: Recall that OMDoc/MMT views transport $I$-theorems to
$M$-theorems, so all properties expressed in these are conserved.

\begin{figure}[ht]\centering
  \def\myxscale{3}\def\myyscale{1.2}
  \documentclass{standalone}
\usepackage[mh]{mikoslides}
% this file defines root path local repository
\defpath{MathHub}{/Users/kohlhase/localmh/MathHub}
\mhcurrentrepos{MiKoMH/talks}
\libinput{WApersons}
% we also set the base URI for the LaTeXML transformation
\baseURI[\MathHub{}]{https://mathhub.info/MiKoMH/talks}

\usetikzlibrary{backgrounds,shapes,fit,shadows,mmt}
\begin{document}
\begin{tikzpicture}[xscale=2.4,yscale=.9]
  \tikzstyle{withshadow}=[draw,drop shadow={opacity=.5},fill=white]
   \tikzstyle{database} = [cylinder,cylinder uses custom fill,
      cylinder body fill=yellow!50,cylinder end fill=yellow!50,
      shape border rotate=90,
      aspect=0.25,draw]
   \tikzstyle{human} = [red,dashed,thick]
   \tikzstyle{machine} = [green,dashed,thick]

\node[thy]  (mf) at (.2,5.3) {MathF};
\node[thy,dashed]  (compf) at (0,6) {CompF};
\node[thy,dashed]  (pf) at (-.9,5.5) {PyF};
\node[thy,dashed]  (cf) at (1,5.5) {C\textsuperscript{++}F};
\node[thy,dashed]  (sf) at (-0.9,4.6) {Sage};
\node[thy,dashed]  (gf) at (1,4.6) {GAP};

\draw[include] (compf) -- (pf);
\draw[includeleft] (compf) -- (cf);
\draw[include] (pf) -- (sf);
\draw[includeleft] (cf) -- (gf);

\node[thy] (kec) at (0,3) {EC};
\node[thy,minimum height=.4cm] (kl) at (0,4) {\ldots};

\node[thy] (sec) at (-1,2) {SEC};
\node[thy,minimum height=.4cm] (sl) at (-1,3) {\ldots};

\node[thy] (gec) at (1,2) {GEC};
\node[thy,minimum height=.4cm] (gl) at (1,3) {\ldots};

\node[thy] (lec) at (-.3,1.2) {LEC};
\node[thy,minimum height=.4cm] (ll) at (.3,1.2) {\ldots};

\node (sc) at (-2,5) {SAGE};
\node[draw] (salg) at (-2,4) {Algorithms};
\node[database,dashed] (sdb) at (-2,2.8) {Database};
\node[draw] (skr) at (-2,1.9) {Knowledge};
\node[draw,machine] (sac) at (-2,1.1) {Abstract Classes};

\node (gc) at (2,5) {GAP};
\node[draw] (galg) at (2,4) {Algorithms};
\node[database,dashed] (gdb) at (2,2.8) {Database};
\node[draw] (gkr) at (2,1.9) {Knowledge};
\node[draw,machine] (gac) at (2,1.2) {AbstractClasses};

\node (lmfdb) at (0,-.1) {LMFDB};
\node[database] (ldb) at (1,-.5) {MongoDB};
\node[draw] (knowls) at (-1,-.5) {Knowls};
\node[draw,machine] (lac) at (0,-.5) {Abstract Classes};

  \begin{pgfonlayer}{background}
    \node[draw,cloud,fit=(sec) (sl),aspect=.4,inner sep=-3pt,withshadow,purple!30] (st) {};
    \node[draw,cloud,fit=(gec) (gl),aspect=.4,inner sep=-4pt,withshadow,purple!30] (gt) {};
    \node[draw,cloud,fit=(kec) (kl),aspect=.4,inner sep=0pt,withshadow,blue!30] (kt) {};
    \node[draw,cloud,fit=(lec) (ll),aspect=3.5,inner sep=-8pt,withshadow,purple!30] (lt) {};
  \end{pgfonlayer}

\begin{pgfonlayer}{background}
  \node[draw,withshadow,fit=(sc) (skr) (sac) (sdb),inner sep=1pt] {};
  \node[draw,withshadow,fit=(gc) (gkr) (gac) (gdb),inner sep=1pt] {};
  \node[draw,withshadow,fit=(lmfdb) (lac) (ldb) (knowls),inner sep=1pt] {};
\end{pgfonlayer}

\draw[view] (kec) -- (sec);
\draw[view] (kec) -- (gec);
\draw[view] (kec) -- (lec);
\draw[include] (kec) -- (kl);
\draw[include] (gec) -- (gl);
\draw[include] (sec) -- (sl);
\draw[include] (lec) -- (ll);
\draw[view] (kl) -- (sl);
\draw[view] (kl) -- (gl);
\draw[view] (kl) to[bend left=5] (ll);

\draw[meta] (mf)  to [bend right=10] (st);
\draw[meta] (sf) -- (st);
\draw[meta] (mf)  to [bend left=10] (gt);
\draw[meta] (gf) -- (gt);
\draw[meta] (mf) -- (kt);
\draw[meta] (compf) to[bend right=15] (kt);

\draw[human,->] (skr) -- node[above]{\scriptsize induce} (st);
\draw[human,->] (gkr) -- node[above]{\scriptsize induce} (gt);
\draw[human,->] (knowls) -- node[left,near end]{\scriptsize induce} (lt);

\draw[machine,->] (gt) to[bend right=30] node[below,near start]{\scriptsize generate} (gac);
\draw[machine,->] (st) to[bend left=30] node[below,near start]{\scriptsize generate} (sac);
\draw[human,->] (st) to[bend left=20] node[below]{\scriptsize refactor} (kt);
\draw[human,->] (gt) to[bend right=20] node[below]{\scriptsize refactor} (kt);
\draw[human,->] (lt) -- node[right]{\scriptsize refactor} (kt);
\end{tikzpicture}
\end{document}
%%% Local Variables: 
%%% mode: latex
%%% TeX-master: t
%%% End: 

  \caption{The MitM paradigm in detail. PyF, C${}^{++}$F and CompF are (basic)
    foundational theories for \python, C${}^{++}$ and a generic computational model. SEC,
    LEC and GEC are theories for \SageMath, \LMFDB and \GAP elliptic curves.}\label{fig:mitm}
\end{figure}

A sketch of the theory graph based on the example of elliptic curves can be
found in Figure~\ref{fig:odk_theories}, an overview of the paradigm can be found
in Figure~\ref{fig:mitm}.  We will not go into details here but show how this
architecture integrates the \emph{Software} and \emph{Knowledge Aspects}.
Clearly, the MitM ontology -- the purple cloud in the middle -- is a
specification of the underlying mathematical knowledge as an OMDoc/MMT theory
graph, while the system interface theories -- the blue clouds around it --
formally specify the names and types (i.e. the argument patterns) and intended
behaviour of the interface functions of the systems (often semi-formally to make
the MitM approach scalable). The OMDoc/MMT views -- the wavy arrows between the
theories -- are interpretation morphisms; in this particular case where they
connect the mathematical specification to the system theories, they express the
``implementation relation''. Thus the OMDoc/MMT framework already allows to
integrate the knowledge and software aspects for system interoperability.

\subsection{Design Decisions and Initial Evaluation}
The MitM paradigm we choose for the software (S) aspect of the \pn VRE toolkit essentially
takes two design decisions:
\begin{compactenum}[\bf D1.]
\item Formalizing the interface specification (\textbf{S2.}: names and
  types of the interface functions) of the systems is sufficient to ensure system
  interoperability.
\item Integrating the implementations at Level \textbf{S3.} --- e.g. \GAP or Python code --- into
  the OMDoc/MMT theory graphs is overkill:
  This code can anyway only be executed by the respective systems --- i.e. \GAP or \SageMath --- and is rarely exchanged between systems.
\end{compactenum}
Therefore we will base our foundation on OMDoc/MMT theory graphs directly rather than on an extension of OMDoc/MMT with
  ``biform theories''~\cite{KohManRab:aumftg13,Farmer:btc07} as envisioned in the
  proposal. Such biform theories could explicitly represent the code as well and thus enable the (partial) verification of mathematical software
  systems, but this is not on the critical path towards a mathematical VRE.

The MitM paradigm constitutes a lightweight alternative; identifying and refining it has
been one of the major achievements of the first year in \WPref{dksbases}.

To evaluate the paradigm and the design decisions we have implemented extensions to the
\GAP and \SageMath systems that export interface theory graphs in the OMDoc/MMT format
(see~Section~\ref{sec:cases} for details):
\begin{compactitem}
\item \GAP exports types, constructors, functions, data, and their documentation: 4097
  Objects exported (2996 unique) in ca. 210 theories.
\item \SageMath exports categories/types, annotates functions: 382 Categories using 25
  Axioms and (in total) 808 methods.
\end{compactitem}
These interface theories allow the representation of all mathematical objects in \GAP and
\SageMath as OpenMath2/MathML3 objects~\cite{BusCapCar:2oms03,CarlisleEd:MathML3} whose
symbols are grounded in the interface theories (interpreted as OpenMath content
dictionaries). \GAP already had an \textbf{OpenMath phrasebook} --- an import/export
facility for OpenMath objects --- and we have developed one for Python and
\SageMath~\cite{py-openmath:on}.

Even though the development of the MitM paradigm is still at an early stage, these case
studies show the potential of the approach. We hope that the nontrivial cost of curating
an ontology of mathematical knowledge and interface views to the interface theories will
be offset by its utility as a resource, and we are currently exploring this hope.
The unification of the knowledge representation components in the \MMT system
\begin{compactenum}
\item enables VRE-wide domain-centered (rather than system-centered) documentation: the
  namespace alignment given by the interface-views allows to re-use documentation for a
  concept, object, or model in the MitM ontology in all interface functions aligned with
  it.
\item can be leveraged for distributed computation via uniform protocols like the
  SCSCP~\cite{HorRoz:ossp09} and MONET-style service
  matching~\cite{CaprottiEtAl:MathServiceMatching04:tr} (the absence of content
  dictionaries --- now given as interface theories --- was the main hurdle that kept these
  from gaining more traction). Again, \GAP already had an SCSCP interface, and we are
  developing one for \SageMath at \cite{py-scscp:on}.
\item will lead to the wider adoption of best practices in mathematical knowledge
  management in the systems involved; in fact, this is already happening: the development
  of the \GAP interface theory exporter led to the discovery of hundreds of documentation
  errors and to a large-scale code refactoring that made type information more explicit
  and could lead to efficiency gains by extended static analysis in the future.
\end{compactenum}

%%% Local Variables:
%%% mode: latex
%%% TeX-master: "report"
%%% End:

%  LocalWords:  pn optimized behavior analyze compactenum textbf organization utilized
%  LocalWords:  characterized forall sqrt Leftrightarrow DehKohKon iop16 mitm centering
%  LocalWords:  myxscale myyscale emph emph formalizing KohManRab aumftg13 btc07 WPref
%  LocalWords:  dksbases compactitem domain-centered ossp09 py-scscp
