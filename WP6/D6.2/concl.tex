\section{Outlook and Conclusion}\label{sec:conclusion}

In this report we have outlined the initial design of a $\mathcal{DKS}$-base (the main
objective of \WPtref{dksbases}). The basis is the OMDoc/MMT format that allows a
foundation-independent representation of mathematical knowledge in theories that are
linked by imports for modular development and views for integrating knowledge from
different sources by interpretation. As anticipated in the \pn proposal, the OMDoc/MMT
format and its implementation in the \MMT system is sufficient to for the knowledge
($\mathcal{K}$) aspect of $\mathcal{DKS}$-bases.

The \pn workshops and the survey (see Section~\ref{sec:survey} and
Appendix~\ref{sec:raw-survey}) clarified that the system integration task is much more
critical in the development of a VRE toolkit than the classical ``Formal Methods'' tasks
of software verification or synthesis. As a consequence we base our integration of the
software aspect ($\mathcal{S}$) on the specifically-developed ``Math-in-the-Middle''
paradigm, which constitutes a much more lightweight approach by concentrating on abstract
(mathematics-level) specifications. We were able to show that this can represent this
OMDoc/MMT theory graphs, if we can generate interface theories for the \pn systems and
align them to a mathematical reference ontology (the MitM ontology) via OMDoc/MMT
views. We have established the feasibility of this by generating interface ontologies for
the \GAP and \SageMath systems and conjecture that carries over to other \pn systems.

For the data aspect ($\mathcal{D}$) of $\mathcal{DKS}$-bases we had to extend OMDoc/MMT to
allow \emph{virtual theories} that consist of arbitrarily many declarations and the \MMT
system to handle them efficiently, -- loading only small subsets into system memory for
processing. Thus we can connect to external databases which we model as a set of
well-typed records, that is list of (key, value) pairs. We introduced the concept of
record types inside \MMT, keys are symbols which are declared inside a schema theory and
values are \MMT literals translated from the physical database representations using
codecs. We used the LMFDB database of elliptic curves as a case study to test this
approach and implemented a multitude of codecs as a ``database connector'' that lifts the
database contents to virtual declarations in virtual theories that can be seamlessly
integrated into the \MMT system. 

In the future we want to expand on both the implementation and the concept as a whole. Right
now we can only translate database records into \MMT objects. While we want to use the form of
the objects used by \MMT as the primary representation we want to be able to translate these
objects to system specific objects, thereby building true \textit{DKS} theories. Each system
might have system-specific constructors and / or representations. In practice all systems will
have a constructor for these objects. These will take a set of arguments. These arguments will
either be primitive (in which case we can just encode them from \MMT using a codec) or be
complex objects themselves (in which case we can recurse into the entire procedure). Storing
these encodings inside \MMT we will be able to write thin interfaces to \MMT, which can then
easily retrieve objects from \MMT in their preferred representation.

Together with the opposite process -- the understanding of objects by using accessors from
arbitrary systems -- will also allow (almost) arbitrary systems and databases to exchange objects
via \MMT. We are already working on a \python Client implementation. This will not apply any
recoding to the objects -- it just retrieves records in an easily accessible form from \MMT. In
the future we are hoping to use this to integrate \MMT and \GAP and enable \GAP to use any kind
of object that \MMT has access to.

It is interesting to note that both the integration of the $\mathcal{S}$ and $\mathcal{D}$
aspects into OMDoc/MMT necessitate the generation of theories: pre-generated as interface
theories in the MitM paradigm, and on-the-fly in virtual theories. 

%%% Local Variables:
%%% mode: latex
%%% TeX-master: "report"
%%% End:

%  LocalWords:  mathcal WPtref dksbases pn emph textit
