\section{Outlook and Conclusion}\label{sec:conclusion}

In this report we have outlined our plans for building \textit{DKS} theories, splittig the
problem into \textit{Knowledge and System} integration via the Math-In-The-Middle approach and
\textit{Data and Knowledge} integration via data/knowledge theories.

\ednote{@Paul: Mention the survey again}

We introduced the basics of how we imagine \MMT data/knowledge theories. Here we connect to
external databases which we model as a set of well-typed records, that is list of (key, value)
pairs. We introduced the concept of record types inside \MMT, keys are symbols which are
declared inside a schema theory and values are \MMT literals translated from the physical
database representations using codecs.

We have showed our implementation so far. We used the LMFDB database of elliptic curves. We
implemented a multitude of codecs that should also prove useful in future expansions of this
implementation. We have demonstrated that it is possible to build \textit{Virutal Theories}
that integrate databases inside \MMT seamlessly so that it is not noticable that declarations
are actually retrieved from a database instead of being declared from within \MMT directly.

In the future we want to expand on both the implementation and the concept as a whole. Right
now we can only translate database records into \MMT objects. While we want to use the form of
the objects used by \MMT as the primary representation we want to be able to translate these
objects to system specifc objects, thereby building true \textit{DKS} theories. Each system
might have system-specific constructors and / or representations. In practice all systems will
have a constructor for these objects. These will take a set of arguments. These arguments will
either be primitive (in which case we can just encode them from \MMT using a codec) or be
complex objects themselves (in which case we can recurse into the entire procedure). Storing
these encodings inside \MMT we will be able to write thin interfaces to \MMT, which can then
easily retrieve objects from \MMT in their prefered representation.

Together with the opposite process -- the understanding of objects by using accessors from
arbitary systems -- will also allow (almost) arbitary systems and databases to exchange objects
via \MMT. We are already working on a \python Client implementation. This will not apply any
recoding to the objects -- it just retrieves records in an easily accessible form from \MMT. In
the future we are hoping to use this to integrate \MMT and \GAP and enable \GAP to use any kind
of object that \MMT has access to.

%%% Local Variables:
%%% mode: latex
%%% TeX-master: "report"
%%% End:
