So far we have only concerned ourselves with accessing virtual theories one declaration at a time.
Going beyond the scope of the present deliverable, it is desirable to also implement a \emph{querying} operation for virtual theories.
Conceptually, this means to ask the \MMT backend to load all those declarations of a virtual theory that satisfy a certain property.
This operation is much more difficult because the naive approach of loading all declarations and then filtering them does not scale at all.

In the future of the \pn project we want to investigate this question further.
At this point, a result of \pn is to sketch and discuss possible solutions to this problem.

\paragraph{Encoding Queries}
A straightforward solution is to translate the property into a query expression that the underlying database can evaluate efficiently.
This is similar to encoding objects as presented in Section~\ref{sec:codec} except that we have to encode complex expressions instead of only atomic values.
Our current implementation already anticipates this solution, but we have not implemented the specific connection to \LMFDB yet.

An example for a practically relevant query that is covered by this approach is to retrieve all elliptic curves whose conductor is $5$.
However, this approach is limited by the capabilities of the underlying database.
For example, MongoDB does not allow efficiently retrieving all elliptic curves whose conductor is divisible by $5$.\footnote{This particular query was posed to John Cremona, who curates the elliptic curves database, by a mathematician. He was unable to carry out the query directly in the current architecture.}


In other data systems, there may not even be a strong query language.
For example, \OEIS stores sequences in semi-normalized text files.
Therefore, it can answer a fixed set of query operators that combine text search with integer sequence--specific search in the precomputed prefixes of the sequences.
This allows querying for all OEIS sequences whose precomputed prefixes contain the integers $5$ and $10$ with any sequence of numbers in between.
But it cannot retrieve, e.g., all sequences containing $n$, $n+1$, $n+2$ for some $n$.

\paragraph{Indexing Sets of Values}
Going beyond the above approach, we can supplement existing database with custom mathematics-specific indices.
For example, we can easily create an index of all integers that occur anywhere in a database.
This would have the additional advantage of allowing queries across different databases, e.g., to find occurrences of the number $142857$ anywhere in \LMFDB, \OEIS, or any other data system.

The key design question here is what index to use.
Indexing all integers is the simplest possible attempt.

For every occurrence of an integer, the index can store additional information that can be used in the query, e.g.,
\begin{compactitem}
 \item which database does it occur in,
 \item which function does it serve (conductor, sequence element, etc.),
 \item what values occur nearby.
\end{compactitem}

An orthogonal improvement is to add for each integer in the index its factorization to the index.
This would allow fast queries based on divisibility.
Naturally, this yields a trade-off problem between the complexity of the index and the complexity of the queries that can be answered.
This problem is well-known in databases in general but has not been studied for mathematical data.

The most difficult design question is how to extend such an index to more complex values such as complex numbers, sequences, or polynomials.
Indeed, finding occurrences of polynomials, or simply storing an index of known factorizations of big polynomials is a concrete service currently needed by mathematicians.

Here the design space includes a gradual transition from indexing values to indexing mathematical expressions.
An integer-only index can be seen as the extreme case of the former.
The extreme case of the latter --- indexing arbitrary expressions --- has been explored already in the MathWebSearch system \delivref{dksbases}{mws}; it is optimized for substitution and unification queries, which are different from the integer-oriented queries discussed above.
We conjecture that after realizing an integer-only index, a key question will be how to combine the best of both kinds of indices.

\paragraph{Composite Queries}
The above has discussed only atomic queries.
Composite queries arise when we use operators such as in SPARQL, XQuery, or SQL to join, intersect, filter, or translate query results.

Here two cases must be distinguished.
Firstly, it may be possible to factor the query into atomic queries whose results can be aggregated into the overall result.
For example, we might ask for all elliptic curves whose conductor is divisible by $5$ and whose defining equation unifies with a certain pattern.
Such a query can be evaluated by taking the intersection of the two atomic queries.
This can be handled comparably simply.
For example, the query language developed for \MMT in \cite{Rabe:qlfml12} already combines ideas from XQuery and SPARQL to compose queries from atomic ones.


Second, it is much harder to answer queries that do not factor.
For example, we might ask for all elliptic curves whose conductor occurs in a certain sequence in the \OEIS.
Barring an unrealistically sophisticated index, such a query would require iterating over all elliptic curves or all entries of that sequence, and evaluate another query for each one.
Either one of these iterations might be prohibitively expensive.

%%% Local Variables:
%%% mode: latex
%%% TeX-master: "report"
%%% End:
