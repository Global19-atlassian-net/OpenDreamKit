OMDoc/MMT theories are limited when it comes to representing large amounts of data.
Conceptually, every database should be represented as one theory, but this can easily lead to very large theories.
For example, the theory for elliptic curves in the \LMFDB would contain $319,882$ symbol declarations: one definition for every curve.
Prior to \pn, \MMT could only load who{e theories into main memory, which made it insufficient as a basis for \DKS-bases.

Moreover, many data-driven theories are technically infinite collections of which only finite fragments have been explored so far.
For example, this applies to all databases in the \LMFDB (each of which enumerates a certain infinite class of mathematical objects) and \OEIS (each of which enumerates a certain integer sequence).
As the explored fragments grow, the set of symbol declarations in the corresponding \MMT theory must grow accordingly.

Therefore, we generalize \MMT theories to allow for a virtual, possibly infinite set of declarations, that is explored dynamically.
The combination of virtual theories with the Math-in-the-Middle approach yields our desired \DKS-bases.

\begin{mydef}[Virtual Theory]
  A \textbf{virtual theory} is like an \MMT theory but with a (possibly infinite) partially ordered set of declarations.
\end{mydef}

We give a trivial example of an infinite virtual theory for the natural numbers:
Besides the usual symbols for $0$ and $\mathtt{succ}$ as well as the Peano axioms, it contains the totally ordered set of one declaration for every natural number.
For example, we might have a declaration
 \[5:\mathtt{nat}=\mathtt{succ}(4)\]
to introduce a symbol for the number $5$.
In the presence of an addition operator, this theory might also contain one axiom for every pair of natural numbers, e.g., to state the truth of $2+2=4$.

This is a typical situation: We have an infinite (or very big) set of declarations that are generated systematically.
In some cases (as for the natural numbers above), every declaration can be easily generated on-demand.
Thus, one might think that virtual theories can be easily represented in a finitary way by storing the algorithm that produces the generations.

But this falls short in general.  For example, the generation of the declarations may be
so expensive that it is only practical if they are precomputed and stored in a database.
This is what happens in the \LMFDB (and was why the \LMFDB was introduced in the first
place).  It is also possible that there are multiple algorithms enumerating different
fragments of the virtual theory, or that no generating algorithm is known (e.g., for some
integer sequences in the \OEIS) and individual declarations must be collected manually.

\lstinputlisting[
  language=json,
  caption={JSON representation of a curve in LMFDB. },
  label=lst:odk-curve
]{examples/11a1.json}

For example, consider the database of elliptic curves in the \LMFDB.
A curve is typically defined by a JSON object such as the one in Listing~\ref{lst:odk-curve}.
This curve has the label \texttt{11a1} and it is non-trivial to translate a label into a full definition.

\paragraph{Implementation}
In practice we never need to access all of these declarations at once --- in most
scenarios we only need a very small subset of them, usually small enough to hold in memory.
This motivates the main idea behind how we have implemented virtual theories in the \MMT system.

\MMT already abstracts from physical storage backends (working copies, databases, etc.), from which theories are loaded.
We have extended this storage abstraction to allow loading not only theories but individual declarations on demand.
This is more difficult than it sounds because while theories have a self-contained semantics, declarations only make sense in the context of the containing theory.
Thus, we had to comb through the \MMT code base and generalize all processing to the case where a theory's declarations are only partially known.

We have also built an \LMFDB-specific implementation of this this generalized storage abstraction.
This instance dynamically queries \LMFDB for the appropriate entry, computes the corresponding declarations, and adds it to the in-memory representation of the virtual theory.
(Additionally, if that virtual theory is not in memory yet, it first creates it.)
We were able to retain an important feature of \MMT: The loading of declarations is transparent to the user.
The backend loads a declaration automatically when and if it is needed by some operation.
As each declaration has a unique URI\ednote{Need to check that this is mentioned in the MMT section /TW}, it is enough to know the set of URIs that need dynamic creation of declarations.
Instead of loading these from memory, the backend can make an appropriate query to lmfdb and create the declarations on demand.


%%% Local Variables:
%%% mode: latex
%%% TeX-master: "report"
%%% End:
