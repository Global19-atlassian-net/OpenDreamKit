\section{Data/Knowledge Theories}\label{sec:data}

To work with DK theories, the first thing we need to do is define them.
\begin{mydef}[DK theory]
  A Data-Knowledge (or DK for short) theory is a theory consisting of a potentially
  infinite set of declarations.
\end{mydef}

But what does this mean? To clarify this, we list a few examples here. The first thing to
note is that all K-theories represented inside \MMT are DK theories that contain a very
limited set of finitely many declarations. However this only part of what is covered by
this concept -- it becaomes interesting once we have many more declarations inside a
theory, in the sense that we have sufficiently many that we can not keep them all in
memory or want to materialise all of them on disk.

Consider the problem of formalising the natural numbers inside a theory. For this we could
use a theory with as little as eight declarations -- one for the type of natural numbers,
one for the number zero, one for the successor function and one for each of the five Peano
axiomes. This defines a DK theory however it is not very practical. We could use terms for
each of the natural numnbers, but if we want to write down a number like $365$, we would
have to apply the successor function $365$ times. We really want to have one declaration
for each natural number, i.e. have each number declared as the successor of the previous
one. This is in the spirit of DK theories.

The naive approach to implement this theory does not work -- we would need to hold
infitely many declarations in memory. Even if we had infinite memory, this would not be
enough -- we would need to materialise all of these definitions in the first place -- we
can not just write all of them down. This particular example can be solved by introducing
literals -- but that is not the point. It illustrates the basic problem that comes up in
many practical situations. We have a very big set of declarations that are generated
according to some schema. This can mean that there is an obvious pattern to generate all
of them -- like with natural numbers -- or that they are stored in some external database
or system. In the database system there is some pattern in how to generate the
declarations -- each declaration corresponds to a record inside the database.

\LMFDB is a mathematical database that contains a collection of elliptic curves. We can use
this to define a DK theory consisting of elliptic curves\footnote{And indeed we have done
  this. More on what \LMFDB is and what exactly we have achieved so far can be found below
  in Section~\ref{sec:lmfdb}}. We could furthermore extend this theory to not only include
elliptic curves, but all objects contained inside \LMFDB. Inside this theory, not all
declarations will look the same -- they will all be of different types. It should also be
noted that this theory is not something one would want to write down manually -- even
though there are finitely many declarations. Likely the number of declarations is vastly
bigger than what we can hold in main memory at any point.

In practice we will never need to access all of these declarations at once -- in most
scenarios we will only need a very small subset of them. A small subset that is big enough
to hold in memory. Hence we need some process to load the declarations inside main memory
once we need them -- or possibly even create them if they do not yet exist.

\begin{mydef}[Virtual theory]
  A Virtual Theory is an implementation of a DK theory that loads or creates declarations in main memory on demand and unloads them once memory space becomes limited.
\end{mydef}

This adresses the problems we mentioned above. On top of the requirements already defined,
we ideally want such an implementation to be invisible to the user -- they should not
notice the difference between a K theory and a DK theory that contains on-demand
declarations.

While there are many possible implementations for DK theories, here we only concern ourselves
with the case where the declaration come from some arbitary external database. On top of
retrieving declarations from DK theories, we also want to be able to create new declarations,
update existing ones and possibly even delete some. In the case of some underlying database
this corresponds perfectly to the CRUD -- Create, Read, Update, Delete -- operations. We can
just translate the operations and have the database take care of the implementation details. We
have already had a few more thoughts on how to enable efficient querying of theories -- we will
get to this later in Section~\ref{sec:querying}.

%%% Local Variables:
%%% mode: latex
%%% TeX-master: "report"
%%% End:
