The storages sketched in Section~\ref{sec:virtual} are not as simple as one might think.
A major complication is that scalable databases only provide relatively low-level data types.
For example, a typical relational database provides primitive types for, e.g., integers and strings, and tables contain records built from these.
JSON databases (as used in MongoDb~\cite{Chodorow:mdg10}, which is used in \LMFDB) or XML databases are slightly better by providing structured types like trees and lists.
But the sets of mathematical objects stored in mathematical data systems use much richer data types such as matrices, polynomials, permutations, and arbitrarily more complex types built from them.

Therefore, any data system must employ encodings that translate the actual mathematical objects into database objects.
This has been done ad hoc in the past and has proved both very difficult and --- due to differing or undocumented encodings --- an obstacle for system interoperability.
Therefore, we have developed a systematic method for encoding/decoding mathematical objects as database objects.
This allows formally specifying the schema of a database in such a way that \MMT storages can use it to encapsulate the encoding and provide users with a high-level view of a mathematical database.
A sketch of our method can be found in Figure~\ref{fig:codec_arch} --- we will give a
detailed explanation below.

\begin{figure}[ht]\centering
  \providecommand\myxscale{3.2}
  \providecommand\myyscale{1.5}
  \providecommand\myfontsize{\footnotesize}
  \documentclass{standalone}
\usepackage{tikz}
\usepackage{../json}
\usetikzlibrary{backgrounds,shapes,fit,shadows,arrows,shapes.geometric}
\makeatletter%%%%%%%%%%%%%%%%%%%%%%%%%%%%%%%%%%%%%%%%%%%%%%%%%%%%%%%%%%%%%%%%%
% A TIKZ library for MMT Theory Graphs
% copyright 2014 Michael Kohlhase; Released under the LPPL
%%%%%%%%%%%%%%%%%%%%%%%%%%%%%%%%%%%%%%%%%%%%%%%%%%%%%%%%%%%%%%%%%
%
% this library provides some standardized node and arrow styles for formatting MMT graph
% diagrams in tikz. The advantage is that we can classify the arrows and nodes
% symbolically and with the styles in this library achieve a uniform look that helps
% readability. 
%
%%%%%%%%%%%%%%%%%%%%%%%%%%%%%%%%%%%%%%%%%%%%%%%%%%%%%%%%%%%%%%%%%

%%% 1. Theories
% a generic theory
\def\outerthysep{.3mm}
\def\innerthysep{.5mm}
\tikzstyle{thy}=[draw,outer sep=\outerthysep,rounded corners,inner sep=\innerthysep]
% a primitive theory
\tikzstyle{primthy}=[thy,double]
% a theory graph 
\tikzstyle{thygraph}=[draw,outer sep=1mm,rounded corners,dashed]

%%% 2. Arrows 
%%% 2.1. Arrowtips (only internal)
\usetikzlibrary{arrows}
\newcommand{\@mmtarrowtip}{angle 45}
\newcommand{\@mmtreversearrowtip}{angle 45 reversed}
\newcommand{\@mmtarrowtipepi}{triangle 45}
\newcommand{\@mmtarrowtipmonoright}{right hook}
\newcommand{\@mmtarrowtipmonoleft}{left hook}
\newcommand{\@mmtarrowtippartial}{right to}
\newcommand{\@mmtarrowtippartialleft}{left to}
\newcommand{\@mmtreversearrowtippartial}{right to reversed}
\newcommand{\@mmtreversearrowtippartialleft}{left to reversed}

%%% 2.2 the arrow sstyles in graphs
\usetikzlibrary{decorations,decorations.pathmorphing,decorations.markings}
% any morphism 
\tikzstyle{morph}=[-\@mmtarrowtip,thick] 
\tikzstyle{mapsto}=[|-\@mmtarrowtip] %| any morphism 
% structures
\tikzstyle{struct}=[-\@mmtarrowtip,thick]
% inclusions: regular, partial, and left variants
\tikzstyle{include}=[\@mmtarrowtipmonoright-\@mmtarrowtip,thick]
\tikzstyle{revinclude}=[\@mmtarrowtip-\@mmtarrowtipmonoright,thick]
\tikzstyle{pinclude}=[\@mmtarrowtipmonoright-\@mmtarrowtippartial,thick]
\tikzstyle{includeleft}=[\@mmtarrowtipmonoleft-\@mmtarrowtip,thick]
\tikzstyle{pincludeleft}=[\@mmtarrowtipmonoleft-\@mmtarrowtippartialleft,thick]
% views: regular, mono, partial, and left variants
\tikzstyle{preview}=[decorate,
                                decoration={coil,aspect=0,amplitude=1pt,
                                                    segment length=6pt,
                                                    pre=lineto,pre length=3pt,
                                                    post=lineto,post length=5pt},
                                thick]
\tikzstyle{view}=[preview,-\@mmtarrowtip]
\tikzstyle{mview}=[preview,\@mmtarrowtipmonoright-\@mmtarrowtip]
\tikzstyle{pview}=[preview,-\@mmtarrowtippartial]
\tikzstyle{pmview}=[preview,\@mmtarrowtipmonoright-\@mmtarrowtippartial]
\tikzstyle{viewleft}=[preview,-\@mmtarrowtip]
\tikzstyle{mviewleft}=[preview,\@mmtarrowtipmonoleft-\@mmtarrowtip]
\tikzstyle{pviewleft}=[preview,-\@mmtarrowtippartialleft]
\tikzstyle{pmviewleft}=[preview,\@mmtarrowtipmonoleft-\@mmtarrowtippartialleft]
\tikzstyle{adoption}=[preview,thin,double,-\@mmtarrowtip]
% biviews: regular, partial, and left variants
\tikzstyle{biview}=[preview,\@mmtarrowtip-\@mmtarrowtip]
\tikzstyle{pbiview}=[preview,\@mmtreversearrowtippartial-\@mmtarrowtippartial]
\tikzstyle{pbiviewleft}=[preview,\@mmtreversearrowtippartialleft-\@mmtarrowtippartialleft]
% defining views (experimental)
\tikzstyle{defview}=[preview,densely dotted,-\@mmtarrowtip]
% meta-theory inclusion
\tikzstyle{meta}=[dotted,-\@mmtarrowtip,thick]
% conservative extensions (abbreviation)
\tikzstyle{conservative}=[hooks-\@mmtarrowtip,double] 
% conservative development
\tikzstyle{conservdev}=[hooks-\@mmtarrowtip,dashed,double] 

% antimorphisms as striktthroughs
\tikzset{anti/.style={
    decoration={markings, mark=between positions 0.2 and 0.8 step 4mm with {
        \draw [thick,-] ++ (-3pt,-3pt) -- (3pt,3pt);}},
    postaction={decorate}}}

% parallel markup
\tikzstyle{parallel}=[\@mmtarrowtip-\@mmtarrowtip,dashed]

%%%% 3. Realms 
\tikzstyle{prerealm}=[draw=blue!40,rectangle,rounded corners,inner sep=10pt,inner ysep=20pt]
\tikzstyle{realm}=[prerealm,fill=gray!4]
\tikzstyle{pillar}=[prerealm,fill=gray!10]

%%% 4. convenience macros
%%% 4.1 the \mmtthy macro takes three arguments, name, decl, axioms and makes a 
% table-like structure
\newcommand\mmtthy[3]{\def\@second{#2}\def\@third{#3}%
\begin{array}{l}\textsf{#1}%
\ifx\@second\@empty\else\\\hline #2\fi%
\ifx\@third\@empty\else\\\hline #3\fi%
\end{array}}
%%% 4.2 the \mmtar takes two arguments, some tikz options, and an arrow style. \nmmtar
% is a variant that also has a name on top.
\newcommand\mmtar[2][]{\raisebox{.5ex}{\tikz[#1]{\draw[#2] (0,0) -- (.6,0);}}}
\newcommand\nmmtar[3][]{\raisebox{.4ex}{\tikz[#1]{\draw[#2] (0,0) --
      node[above]{\ensuremath{\scriptstyle #3}} (.8,0);}}}
%%%% 3.3 Pushout symbols
%%  (after http://tex.stackexchange.com/questions/1144/pushouts-and-pullbacks)
%% \nepushout[<pos>]{<basenode>}{<targetnode>} positions a pushout symbol 
%% (available separately as \pushoutsymb) <pos> of the way between <basenode>
%%  and <targetnode>. 
\usetikzlibrary{calc}
\newcommand\@pushout[4][]{\def\@test{#1}%
\ifx\@test\@empty\def\@@num{.5}\else\def\@@num{#1}\fi%
\begin{scope}[shift=($(#2)!\@@num!(#3)$)]\@nameuse{@#4pushoutsymb}\end{scope}} 
\newcommand\@nepushoutsymb{\draw +(-.2,0) -- +(0,0)  -- +(0,-.2);\fill +(-.1,-.1) circle (.03);}
\newcommand\nepushoutsymb{\tikz{\@nepushoutsymb}}
\newcommand\nepushout[3][]{\@pushout[#1]{#2}{#3}{ne}}

\newcommand\@sepushoutsymb{\draw +(-.2,0) -- +(0,0)  -- +(0,.2);\fill +(-.1,.1) circle (.03);}
\newcommand\sepushoutsymb{\tikz{\@sepushoutsymb}}
\newcommand\sepushout[3][]{\@pushout[#1]{#2}{#3}{se}}

\newcommand\@nwpushoutsymb{\draw +(.2,0) -- +(0,0)  -- +(0,-.2);\fill +(.1,-.1) circle (.03);}
\newcommand\nwpushoutsymb{\tikz{\@nwpushoutsymb}}
\newcommand\nwpushout[3][]{\@pushout[#1]{#2}{#3}{nw}}

\newcommand\@swpushoutsymb{\draw +(.2,0) -- +(0,0)  -- +(0,.2);\fill +(.1,.1) circle (.03);}
\newcommand\swpushoutsymb{\tikz{\@swpushoutsymb}}
\newcommand\swpushout[3][]{\@pushout[#1]{#2}{#3}{sw}}

\newcommand\@npushoutsymb{\draw +(-.1,-.1) -- +(0,0)  -- +(.1,-.1);\fill +(0,-.1) circle (.03);}
\newcommand\npushoutsymb{\tikz{\@npushoutsymb}}
\newcommand\npushout[3][]{\@pushout[#1]{#2}{#3}{n}}

\newcommand\@spushoutsymb{\draw +(-.1,.1) -- +(0,0)  -- +(.1,.1);\fill +(0,.1) circle (.03);}
\newcommand\spushoutsymb{\tikz{\@spushoutsymb}}
\newcommand\spushout[3][]{\@pushout[#1]{#2}{#3}{s}}

\newcommand\@wpushoutsymb{\draw +(-.1,-.1) -- +(0,0)  -- +(-.1,.1);\fill +(-.1,0) circle (.03);}
\newcommand\wpushoutsymb{\tikz{\@wpushoutsymb}}
\newcommand\wpushout[3][]{\@pushout[#1]{#2}{#3}{w}}

\newcommand\@epushoutsymb{\draw +(.1,.1) -- +(0,0)  -- +(.1,-.1);\fill +(.1,0) circle (.03);}
\newcommand\epushoutsymb{\tikz{\@epushoutsymb}}
\newcommand\epushout[3][]{\@pushout[#1]{#2}{#3}{e}}

%%%%%% testing them: 
% \begin{tikzpicture}[scale=1.5]
%   \node (a) at (0,0) {a};
%   \node (b) at (1,0) {b};
%   \node (c) at (1,1) {c};
%   \node (d) at (0,1) {d};
%   \nepushout[.8]{a}c;
%   \nwpushout[.8]{b}d;
%   \sepushout[.8]{d}b;
%   \swpushout[.8]{c}a;
%   \draw (a) -- (b) -- (c) -- (d) -- (a);
%   \node (aa) at (4,0) {a};
%   \node (bb) at (3.5,.5) {b};
%   \node (cc) at (4,1) {c};
%   \node (dd) at (4.5,.5) {d};
%   \draw (aa) -- (bb) -- (cc) -- (dd) -- (aa);
%   \npushout[.8]{aa}{cc};
%   \spushout[.8]{cc}{aa};
%   \wpushout[.8]{bb}{dd};
%   \epushout[.8]{dd}{bb};
% \end{tikzpicture}

%%% Local Variables:
%%% mode: latex
%%% TeX-master: "test"
%%% End:
\makeatother
%MMT
\newcommand{\GS}{\textcolor{red}{\fbox{M}}}
\newcommand{\RS}{\textcolor{blue}{\fbox{D}}}
\newcommand{\US}{\textcolor{green}{\fbox{O}}}

\begin{document}
\def\thmo#1#2{\mathsf{#1}\colon\kern-.15em{#2}}
\providecommand\myxscale{3}
\providecommand\myyscale{1.5}
\providecommand\myfontsize{\footnotesize}

\begin{tikzpicture}[xscale=\myxscale,yscale=\myyscale]\myfontsize
  \tikzstyle{database}=[cylinder,shape border rotate=90,aspect=0.25,draw,cylinder uses custom fill,cylinder body fill=white!30,cylinder end fill=white!3]
  \node[database] (db) at (1,4) {
    \begin{tabular}{c}Database\\\\ LMFDB\\ (MongoDB)\end{tabular}
  };

  \node[draw] (record) at (2.5,4) {
    \begin{minipage}{3cm}
\begin{lstlisting}[language=json,mathescape,frame=none,aboveskip=2pt,belowskip=2pt]
label: "11a1",
conductor: 5,
$\dots$
\end{lstlisting}
\end{minipage}};

  \draw[->] (db) -- node[above]{contains} (record);

  \node[thy] (schema) at (2.5,2) {
    \begin{tabular}{l}
      \textsf{Record Schema Theory}\\\hline
      \texttt{conductor: int} $\US$\\
        \texttt{  ?codec standardInt} $\US$\\
        \texttt{  ?implements cond}\\
      $\RS$\\
      $\dots$
    \end{tabular}

  };

  \draw[->] (schema) -- node[above]{describes} (record);

  \node[thy] (mitm) at (0,2) {
    \begin{tabular}{l}
      \textsf{Math-In-The-Middle Theory}\\\hline
      \texttt{elliptic\_curve: type} $\RS$\\
      \texttt{cond: elliptic\_curve $\rightarrow$ int} $\RS$\\
      $\dots$
    \end{tabular}
  };

  \draw[->] (schema) -- node[above]{implements} (mitm);

  \node[thy] (virtual) at (1,0) {
    \begin{tabular}{l}
      \textsf{Virtual Theory}\\\hline
      \texttt{11a1: elliptic\_curve $\US$ = $\dots$} $\RS$\\
      $\dots$
    \end{tabular}
  };

  \draw[include] (virtual) -- node[above]{includes} (mitm);

\end{tikzpicture}
\end{document}

%%% Local Variables:
%%% mode: latex
%%% TeX-master: t
%%% End:

  \caption{Translation between Database and Mathematical Objects: The example shows the elliptic curves database of \LMFDB}
  \label{fig:codec_arch}
\end{figure}

\subsubsection{Codecs}

We fix an arbitrary data type of \textbf{codes}.  These are the primitive values of the
database.  Typical examples are strings or JSON objects.  We will use JSON for the purpose
of giving concrete examples.

Our goal is to define codecs that define the relation between \MMT objects and codes.

\begin{mydef}[Codec]
  For an \MMT object $t$, a \textbf{$t$-codec} is a pair $(\enc, \dec)$ where
  \begin{compactitem}
   \item $\enc$ is a partial function from \MMT objects of type $t$ to codes
   \item $\dec$ is a partial function from codes to \MMT objects of type $t$
  \end{compactitem}
  such that $\dec$ and $\enc$ are inverse to each other whenever defined.
\end{mydef}

Both encoding and decoding are partial functions.
This is to be expected for decoding: only certain codes are the result of encoding objects of type $t$.
It may be surprising for encoding.
Here we need partiality because \MMT objects may be arbitrarily complex expressions, not all of which denote mathematical values and can be encoded easily.
This can include non-normalized expressions or expressions with free variables or uninterpreted symbols.
We will see some examples below.

Recall the Math-in-the-Middle theory from Section~\ref{sec:MMT}.
A trivial example of an $\Int$-codec is essentially the identity function: $\enc$ maps integer literals to the corresponding JSON integer, and $\dec$ inverts it.
$\enc$ is partial because it only encodes literals, e.g., it does not encode expressions like $x+1$ or $\min \{x^4+x^3+x^2+x+1|x\in \mathbb{N}\}$.
(An object like $1+1$ can be encoded by first simplifying it to a literal $2$.)

One might think that it is sufficient to fix one $t$-codec for every type $t$.
However, that is not realistic.
In fact, already for the seemingly-trivial case of integers, we need multiple different codecs.

First of all, note that the $\Int$-codec given above cannot encode integer literals that do not fit into the $64$ bit integers of a typical JSON implementation.
In \LMFDB databases, we indeed encounter such integers very often.
Therefore, we have specified two additional codecs.
$\intAsString$ encodes integer literal as strings in decimal representation.
$\intAsString$ has the advantage of easily encoding all integers, but it is not
convenient for computations.  Therefore, \LMFDB users have occasionally used a smarter
encoding: $\intAsList$ encodes integer literals as lists of JSON integers such that the
list $[n,d_1,\ldots,d_n]$ corresponds to the integer whose base $2^{64}$ representation is
$(d_1\ldots d_n)$.  $\intAsList$ has the advantage that the lexicographic ordering can be
used for size comparisons without having to decode.

Along the same lines, we can define codecs for other simple data types.
For example, we can define a codec for matrices of integers that encodes pairs of pairs of integers as JSON lists of $4$ JSON integers.
This example is interesting because it comes up in the \LMFDB and has previously caused confusion: because the encoding was not fully documented, users mistakenly assumed that the mathematical type of the field is list of integers rather than pair of pair of integers.

\lstinputlisting[language=MMT,mathescape,
  caption={Fragment of the special Codec Theory in MMT},
  label=lst:mmt-codec
]{examples/odk-codec.mmt}

We collect and document all available codecs in a special \MMT theory, a fraction of which can be seen in Listing~\ref{lst:mmt-codec}.
Here, $\codec\,t$ is the type of codecs that can encode expressions of type $t$.

Note that our codec theory does not include any implementations (which usually require highly programming language--specific function calls.
Instead, it standardizes names for the codecs and documents their behavior so that each codec can be implemented faithfully in multiple programming languages.
This is in line with the Math-in-the-Middle approach of centrally storing the shared knowledge.

We have seeded this theory with a few important codecs.  And for every codec we describe,
we have already given a reference implementation in Scala~\cite{scala:webpage} so that the
\MMT system (which is written in Scala) can use all codecs.  In the future, \pn will
gradually build a library of relevant codecs and implement them in multiple programming
languages.

\paragraph{Codec Operators}
$\Int$ is an atomic type.
But mathematical types are usually very complex types.
Already, types like $\lst(\Int)$ provide substantial encoding problems because both integers and lists can be encoded in multiple ways.
To systematize these choices, we introduce codec operators.

\begin{mydef}[Codec Operator]
  For an \MMT symbol $t$ and an arity $n$, a $t$-codec operator takes an $o_1$-codec $C_1$, \ldots, an $o_n$-codec $C_n$ and returns a $t(o_1,\ldots,o_n)$ codec.
\end{mydef}

For example, Listing~\ref{lst:mmt-codec} declares a symbol $\lst$, which maps $t$-codecs to $\lst(\Int)$-codecs.
We specify it as the following codec operator $\standardList$ of arity $1$.
It takes a type $t$-codec $C$ and returns the following codec: $\enc$ maps the object $\cons(a_1,\cons(a_2,\ldots,\cons(a_n,\nil)\ldots))$ to the JSON list $[e_1,\ldots,e_n]$ where each $e_i$ is the result or encoding $a_i$ with $C$.
$\dec$ is defined accordingly.

Similar to the codecs, the special \MMT codec theory (Listing~\ref{lst:mmt-codec}) also includes standardised names for these.
Apart from $\standardList$, we also have straightforward codec operators for vectors (as lists of fixed length) and matrices (as lists of lists of fixed lengths).

\subsubsection{Schema Theories}

Codecs can transform between \MMT objects and codes, but we still have to specify which codecs to use for which types.
We use special theories for this purpose, which we call \emph{schema theories}.
These satisfy two functions.

Firstly, a schema theory describes the database schema: many databases (including the ones in \LMFDB) can be seen as sets of records conforming to a certain schema.
We represent these schemas as \MMT theories with one symbol declaration for each field.
The meta-theory of these theories is the math-in-the-middle theory so that the types of the symbols can be the intended mathematical types.
Secondly, we annotate meta-data to each declaration, providing the information which concept in the math-in-the-middel theory a field implements and which codecs to use for converting between the two.

Thus, the schema documents both the mathematical meaning of the fields and the physical encoding used when exchanging records conforming to the schema.
By giving the schema theory for each database, we can capture all knowledge necessary to automatically interface with it.

\lstinputlisting[language=MMT,mathescape,
  caption={Fragment of the Elliptic Curve Schema Theory in MMT},
  label=lst:schema
]{examples/schema.mmt}

As an example, the schema theory for the database of \emph{elliptic curves} in LMFDB is indicated in Figure~\ref{fig:codec_arch}.
A larger fragment is given in Listing~\ref{lst:schema}. 
It has meta-data linking it to the \emph{type} of elliptic curves in the corresponding math-in-the-middle theory, as well as meta-data telling \MMT which field in the database to use as \emph{names} for the resulting \MMT declarations --- in this case the field \texttt{label}, which corresponds to a unique LMFDB-internal naming scheme. The schema theory contains e.g. a declaration \texttt{degree} of type \texttt{pos} (for positive integers), corresponding to a field in the database by the same name, which is annotated with meta-data telling \MMT
\begin{enumerate}
\item to use the codec \texttt{standardPos} to convert from (and to) an LMFDB entry, and
\item that the field \texttt{degree} implements the function \texttt{modular\_degree} in the math-in-the-middle theory.
\end{enumerate}

We have implemented a new component of the \MMT system that takes an expression $c$ of type $\codec\, t$ and builds the appropriate $t$-codec by traversing $c$.
We use this in the storage instance for \LMFDB as follows:
\begin{compactenum}
 \item A declaration with name $n$ in theory $T$ is requested (e.g. the elliptic curve with label \texttt{11a1} in the virtual theory \texttt{lmfdb:db/elliptic\_curves?curves}).
 \item We load the schema theory $S$ for $T$ (e.g. \texttt{lmfdb:schema/elliptic\_curves?curves})
 \item We connect to \LMFDB and retrieve the corresponding record (e.g. the database entry for the curve with label \texttt{11a1}).
 \item We decode every field of the record according to the codec specified in $S$.
 \item We collect the decoded \MMT object in an \MMT record $r$, the mathematical representation of the requested object.
 \item We add the declaration $n\;:\;\cn{elliptic\_curve}\;=\;r$ to the corresponding virtual theory.
\end{compactenum}

\ednote{FR: add description of handling of constructors/assessors for a future paper; I think we can omit it for the deliverable as unnecessarily detailed}

%%% Local Variables:
%%% mode: latex
%%% TeX-master: "report"
%%% End:

%  LocalWords:  Chodorow mdg10 centering providecommand myxscale myyscale myfontsize tikz
%  LocalWords:  ednote enc subsubsection textbf mydef compactitem non-normalized mathbb
%  LocalWords:  ldots lstinputlisting mathescape lst mmt-codec standardizes behavior pn
%  LocalWords:  systematize arity emph texttt texttt
