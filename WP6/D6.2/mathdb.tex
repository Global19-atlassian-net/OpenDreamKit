\subsubsection{Codecs}

The first step towards implementing Virtual Theories is to create a theory that retrieves
declarations from inside a database. Databases are not commonly optimise for mathematical
objects. On the contrary -- databases usually enforce their own schema that does not
correspond in any way to the high-level mathematical objects that we want to
model. Furthermore the object representations are not in the form of something resembling
OpenMath / \MMT terms -- usually they are in primitive objects (integers, strings,
booleans, etc. ) or in data formats like JSON.

Hence we need to be able to translate between the low-level representations of the objects
stored in the database and the underlying mathematical objects (in the form of \MMT terms). In
most common databases, each record stored usually has multiple fields that each have values.
Each field commonly represents one property of the mathematical object. Together the fields and
their values are more than enough to uniquely define the mathematical object. The values
themselves are usually represented with some physical data type -- for example an integer can
be stored in the form of a 64-bit integer or in the form of a string if it becomes to big. In
some of the cases it is trivial to translate the encoded values to the proper values. In other
cases, it might not be as trivial as it seems. Take for example a mathematical object defined
by some matrix of integers. On one hand it could simply be represented as a list of list of
integers. On the other hand the database could use some kind of sparse representation, i.e.
store only the non-zero entries. To encompass this setup inside \MMT we introduce the concept
of codecs.

\begin{mydef}[Codec]
  A codec is a triple $(t, f, g)$ where $t$ is a type (represented as a term inside the
  Math-In-The-Middle Theory), $f$ a mapping from the encoded representations of the type
  to the intended values of the type and $g$ is a mapping from the actual values of the
  type to their physical encodings. We call the operation performed by $f$ a decoding and
  the operation performed by $g$ an encoding of the type $t$.
\end{mydef}

We can use codecs to translate between \MMT terms and the database objects. There are two
basic ways of creating codecs -- atomic codecs and codec operators. An atomic codec is a
codec of a very simple type, for example an encoding of integers as strings. In this case
$f$ would be a map from strings consisting of digits\footnote{Technically, there may also
  be a starting \texttt{"-"} or \texttt{"+"} sign, but the details are not important
  here. } to integers. This mapping does exactly what you would expect -- it maps the
string \texttt{"0"} to 0, the string \texttt{"1"} to 1, etc. $g$ would be an inverse of
sorts -- the map from integers to strings that maps 0 to \texttt{"0"}, 1 to \texttt{"1"},
etc. Note that here $g$ is the left inverse of $f$ (in the sense that $g \circ f$ is the
identity), but $f \circ g$ not neccessarily. In a general setting we want $g \circ f$ to
be the identity for all codecs, however in practive it rarely happens. A good example of
this is the problem of precision -- there is no chance of encoding all of the real numbers
in a physical (countable, probably even finite) datatype.  The second way of creating
codecs is by using codec operators. These take as input an atomic codec and give a codec
for a composite type. Take for example the list codec operator. It takes as input a codec
for an arbitary type $t$ and gives a codec for the type \texttt{List(t)}.

Inside \MMT we can represent the actual values as literals. For this we do not use the
representations from the database, but instead values that best represent the true
values. We also represent the codecs inside \MMT. Each codec can be represented as an \MMT
term, with each atomic codec and each codec operator having an associated symbol inside a
Codec theory We can use these symbols to represent arbitary codecs, for example a codec
for a matrix of integers would be represented by applying the symbol for a matrix codec
operator to the symbol for an integer codec. For each codec and codec operator we have an
associated Scala class. For atomic codecs, this class implements an \texttt{encode} and
\texttt{decode} method (implementing $g$ and $f$ respectively). In the case of Codec
operators, it implements a \texttt{build\_codec} method --
that takes as input an AtomicCodec and returns an codec for the composite type.

To be able to construct the codec that corresponds to a Term we have also implemented a
Coder class. It starts from a term representing a codec, finds the
appropriate atomic codec(s) and codec operator(s) and then uses them to construct a Codec
instance. This allows us given a literal value and a Term representing its type to encode
or decode terms in almost arbitary ways.

\subsubsection{Using Records and Schemas inside \MMT}

Codecs only solve half of the problem -- they can only translate values. We still need to
represent the records as a whole inside DK theories. For this we introduce a type of
records inside \MMT. This is just what you would expect -- a list of (key, value) pairs
with each key appearing at most once. In ou rimplementation the keys are \MMT symbols and
the values are arbitary \MMT values. We also introduce a projection operator which takes
an \MMT record and a key and returns the values of the key in the given record.

To translate the entire object, we assume that the records of one type inside the database
are homogeneous\footnote{That is each record has the same fields and the values have the
  same semantic types. Having the same semantic types means to be decodable with the same
  codec. }. For this purpose we introduce the concept of a schema theory. This schema
theory has a declaration for each field in the database. Furthermore we use meta-data to
annotate each declaration by stating which codec it uses. Since a codec can be represented
by an \MMT term this is easy to implement.

The schema theory tells us exactly how a record inside the database looks like and how to
turn it into an \MMT term. However we also want to be able to have a type to be able to
talk about the theory of all records in the database. This contains only two fixed
declaration: A type for the database objects we are talking about and a constructor
\texttt{from\_record}\ednote{check if this a good idea with Florian} that takes a generic
record and returns an member of this type. We dynamically introduce a set of virtual
declarations into this theory: One for each record in the database. When it is requested,
we check for all declared fields in the schema theory and use the appropriate codecs to
construct an \MMT record for the requested object. We then use the \texttt{from\_record}
constructor to define the \MMT term corresponding to the record.

We also want to be able to access specific fields from the retrieved objects. For this we
introduce a third theory that declares a set of accessors, that is a function from the
type of records in the database to the appropriate types. Each accessor also contains
meta-data that states which field from the schema theory it implements. With the help of
dynamically generated rules we can then simplify terms by extracting the appropriate
fields from the records that were used to construct them.

%%% Local Variables:
%%% mode: latex
%%% TeX-master: "report"
%%% End:
