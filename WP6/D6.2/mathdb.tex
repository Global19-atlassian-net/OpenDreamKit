The storages sketched in Section~\ref{sec:virtual} are not as simple as one might think.
A major complication is that scalable databases are only provide relatively low-level data types.
For example, a typical relational database provides primitive types for, e.g., integers and strings, and tables contain records built from these.
JSON databases (as used in MongoDb\cite{}, which is used in \LMFDB) or XML databases are slightly better by providing structured types like trees and lists.
But the sets of mathematical objects stored in mathematical data systems use much richer data types such as matrices and polynomials and arbitrarily more complex types.

Therefore, any data system must employ encodings that translate the actual mathematical objects into database objects.
This has been done ad hoc in the past and has proved both very difficult and --- due to differing or undocumented encodings --- an obstacle for system interoperability.
Therefore, we have developed systematic method for encoding/decoding mathematical objects as database objects.
This allows formally specifying the schema of a database in such a way that \MMT storages can use it to encapsulate the encoding and provide users with a high-level view of a mathematical database.

\ednote{macros are working, possibly re-check them /TW}

\subsubsection{Codecs}

We fix an arbitrary data type of \textbf{codes}.
These are the primitive values of the database.
Typical examples are strings or JSON objects.
We will JSON for the purpose of giving concrete examples.

Our goal is to define codecs that define the relation between \MMT objects and codes.

\begin{mydef}[Codec]
  For an \MMT object $t$, a $t$-codec is a pair $(\enc, \dec)$ where
  \begin{compactitem}
   \item $\enc$ is a partial function from \MMT objects of type $t$ to codes
   \item $\dec$ is a partial function from codes to $\MMT$ objects of type $t$
  \end{compactitem}
  such that $\dec$ and $\enc$ are inverse to each other whenever defined.
\end{mydef}

Both encoding and decoding are partial functions.
This is to be expected for decoding: only certain codes are the result of encoding objects of type $t$.
It may be surprising for encoding.
Here we need partiality because \MMT objects may be arbitrarily complex expressions.
This can include non-normalized expressions or expressions with free variables or uninterpreted symbols.
This will become clear in the examples below.

Recall the Math-in-the-Middle theory from Section~\ref{sec:MMT}.
\ednote{@TW: make sure all symbols mentioned here are part of the earlier example}
A trivial example of an $\Int$-codec is essentially the identity function: $\enc$ maps integer literals to the corresponding JSON integer, and $\dec$ inverts it.
$\enc$ is partial because it only encodes literals, e.g., it does not encode expressions like $x+1$ or $\min \{x^4+x^3+x^2+x+1|x\in \mathbb{N}\}$.
(An object like $1+1$ can be encoded by first simplifying it to a literal $2$.)
More subtly, it cannot encode integer literals that do not fit into the $64$ bit integers of a typical JSON implementation.

But this is not the only reasonable $\Int$-codec.
In \LMFDB databases, we often encounter integers that do not fit into $64$ bits, and we have specify two additional codecs.
$\intAsString$ encodes integer literal as strings in decimal representation.
$\intAsString$ has the advantage of easily encoding all integers.
But it is not convenient for computations.
Therefore, \LMFDB users have occasionally used a smarter encoding: $\intAsList$ encodes integer literals as lists of JSON integers such that the list $[n,d_1,\ldots,d_n]$ corresponds to the integer whose base $2^{64}$ representation is $(d_1\ldots d_n)$.
$\intAsList$ has the advantage that the lexicographic ordering can be used for size comparisons without having to decode.

Along the same lines, we can define codecs for other simple data types.
For example, we can define a codec for matrices of integers that encodes pairs of pairs of integers as a JSON list of $4$ JSON integers.
This example is interesting because it comes up in the \LMFDB and has confused some programmers: because the encoding was not fully documented, it was mistakenly assumed that the respective mathematical type of the value is that of lists of integers.

We collect and document all available codecs in a special \MMT theory:
\ednote{@TW: give and explain relevant fragment of codec theory}
Note that our codec theory does not include any implementations (which usually require highly programming language--specific function calls.
Instead, it standardizes names for the codecs and documents their behavior so that each codec can be implemented faithfully in multiple programming languages.
This is in line with the Math-in-the-Middle approach of centrally storing the shared knowledge.

We have seeded this theory with a few important codecs.
And for every codec we describe, we have already given a reference implementation in Scala so that the \MMT system (which is written in Scala) can use all codecs.
In the future, \pn will gradually build a library of relevant codecs and implement them in multiple programming languages.

\paragraph{Codec Operators}
$\Int$ is an atomic type.
But mathematical types are usually very complex types.
Already, types like $\lst(\Int)$ provide substantial encoding problems because both integers and lists can be encoded differently.
To systematize these choices, we introduce codec operators.

\begin{mydef}[Codec Operator]
  For an \MMT symbol $t$ and an arity $n$, a $t$-codec operator takes an $o_1$-codec $C_1$, \ldots, an $o_n$-codec $C_n$ and returns a $t(o_1,\ldots,o_n)$ codec.
\end{mydef}

For example, let us define a $\lst$-codec operator $\lstAsList$ of arity $1$.
It takes a type $t$-codec $C$ and returns the following codec: $\enc$ maps the object $\cons(a_1,\cons(a_2,\ldots,\cons(a_n,\nil)\ldots))$ to the JSON list $[e_1,\ldots,e_n]$ where each $e_i$ is the result or encoding $a_i$ with $C$.
$\dec$ is defined accordingly.
\ednote{@TW: give declaration of $\lstAsList$ in codec theory, mention a few other codec operators that we have (if any)}

\subsubsection{Schema Theories}

\ednote{@DM: Can you check/prettify/detailify the descriptions in this section. Maybe add the URIs that we use LMFDb or add details.}

Codecs can transform between \MMT objects and codes, but we still have to specify which codecs to use for which types.
We use special theories for this purpose, which we call \emph{schema theories}.
These satisfy two functions.

Firstly, a schema theory describes the database schema: many databases (including the ones in \LMFDB) can be seen as sets of records conforming to a certain schema.
We represent these schemas as \MMT theories with one symbol declaration for each field.
The meta-theory of these theories is the math-in-the-middle theory \ednote{@TW replace with actual name of the theory as introduced in Sect. 3} so that the types of the symbols can be the intended mathematical types.
Secondly, we annotate meta-data to each declaration, providing the information which concept in the math-in-the-middel theory a field implements and which codecs to use for converting between the two.

Thus, the schema documents both the mathematical meaning of the fields and the physical encoding used when exchanging records conforming to the schema.
By giving the schema theory for each database, we can capture all knowledge necessary to automatically interface with it.

\ednote{@TW: give simple example of elliptic curve schema theory\\DM: Done.}
As an example, the schema theory for the database of \emph{elliptic curves} in LMFDB
has meta-data linking it to the \emph{type} of elliptic curves in the corresponding math-in-the-middle theory, as well as meta-data telling \MMT which field in the database to use as \emph{names} for the resulting \MMT declarations - in this case the field \texttt{label}, which corresponds to a unique LMFDB-internal naming scheme. The schema theory contains e.g. a declaration \texttt{degree} of type \texttt{pos} (for positive  integers), corresponding to a field in the database by the same name, which is annotated with meta-data telling \MMT
\begin{enumerate}
\item to use the codec \texttt{standardPos} to convert from (and to) an LMFDB entry, and
\item that the field \texttt{degree} implements the function \texttt{modular\_degree} in the math-in-the-middle theory.
\end{enumerate}

We have implemented a new component of the \MMT system that takes an expression $c$ of type $\codec t$ and builds the appropriate $\tm t$-codec by traversing $c$.
We use this in the storage instance for \LMFDB as follows:\ednote{@DM add details here, e.g., explain theory names in general and/or for elliptic curve example}
\begin{compactenum}
 \item A declaration with name $n$ in theory $T$ is requested (e.g. the elliptic curve with label \texttt{11a1} in the theory \texttt{lmfdb:db/elliptic\_curves?curves}).
 \item We load the schema theory $S$ for $T$ (e.g. \texttt{lmfdb:schema/elliptic\_curves?curves})
 \item We connect to \LMFDB and retrieve the corresponding record (e.g. the database entry for the curve with label \texttt{11a1}).
 \item We decode every field of the record according to the codec specified in $S$.
 \item We collect the decoded \MMT object in an \MMT record $r$, the mathematical representation of the requested object.
 \item We add the declaration $n=r$ to the corresponding virtual theory.
\end{compactenum}

\ednote{FR: add description of handling of constructors/assessors for a future paper; I think we can omit it for the deliverable as unnecessarily detailed}

%%% Local Variables:
%%% mode: latex
%%% TeX-master: "report"
%%% End:
