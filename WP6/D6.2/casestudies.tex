\section{Case Studies}\label{sec:cases}

While our theoretical model of DK theories and our architectural design of virtual
theories are applicable to a wide variety of databases, in the scope of the OpenDreamKit
project we want to conduct a few case studies and connect to some databases in
particular. Our efforts so far are based on two of these case studies of which we want to
give a short overview below.

\subsection{GAP}\label{sec:gap}
GAP \cite{gap} is a computer algebra system with a particular emphasis on group theory and discrete mathematics in general. Its fundamental
ontology consists of \emph{objects} (e.g. a monoid) satisfying various (composite or elementary) \emph{filters} (e.g. \emph{isAbelian}) which
can be thought of as the \emph{types} of objects.
On top of these filters, operations are defined (e.g. computing the \emph{degree} of a group) which are implemented by arbitrarily many concrete implementations
called \emph{methods}. The user only ever applies operations - the GAP system then uses a sophisticated method selection algorithm based on
the specific (additional) filters satisfied by a given object.

We have a working specification of GAP's ontology in \MMT, which server as a meta-theory for a working GAP import. The result is currently an import
of 4097 filters and operations as \MMT symbols which are collected in approximately 200 theories.

So far, all the imported operations have no information about their return types (i.e. the filters that apply to the returned object). Currently, work is done on the GAP system
to make that information available in general and for the export to \MMT specifically.

\subsection{Sage}\label{sec:sage}

\subsection{LMFDB}\label{sec:lmfdb}

LMFDB \cite{lmfdb} is a database of objects from number theory. It mostly consists of
L-functions, but also has a number of other sub-databases. It is built on top of MongoDB
and as such uses JSON to model all of its data.

We have already implemented the schema and codec architecture above to build a virtual
theory of elliptic curves in LMFDB. Even though this only is a very small part of LMFDB,
this can serve as a template for future implementations of the remaining parts of
LMFDB. We started out with having just a few fields of the curves available inside \MMT,
however adding the relevant codecs and making more fields available proofed to be a quick
and easy job. This has shown us that we are heading in the right direction with our ideas
so far.

In the future we want to extend the coverage of the existing set of theories. This will
include writing more schema theories and possibly introducing more codecs. This will
likely also lead to some refactoring inside LMFDB itself, as the community for the first
time will try to semantically describe its entire dataset.

\subsection{OEIS}

OEIS \cite{oeis} stands for On-Line Encyclopedia of Integer Sequences. It is a collection
of around 250 thousand integer sequences that are stored as pure text form. The OEIS is
licensed under Creative Commons and thus freely accessible.

We have already semantified the pure text format and, among other things, this has helped
us finding new relations between the existing sequences. A more detailed look at our
previous work can be found in \cite{LuzKoh:fsarfo16} and we will not go into details
here. So far these efforts have helped us to understand how the OEIS database is
structured.

Similar to lmfdb we plan on integrating this into our virtual theories architecture. We
are considering building one DK theory per sequence, where the declarations in each theory
contain the known elements of the sequence. We also plan to integrate this with our
infrastructure on knowledge management services, such as MathHub and MathWebSearch.


%%% Local Variables:
%%% mode: latex
%%% TeX-master: "report"
%%% End:

%  LocalWords:  lmfdb oeis LuzKoh fsarfo16
