\documentclass[a4paper]{article}

\usepackage[utf8]{inputenc}
\usepackage{url}
%\usepackage{wrapfig}
\usepackage{amsmath,amssymb,amsthm}

\usepackage{listings}

%\usepackage{array}
%\usepackage{xcolor}
\usepackage{paralist}

\newcommand{\cn}[1]{\ensuremath{\mathtt{#1}}}

\setlength{\hfuzz}{3pt} \hbadness=10001
\setcounter{tocdepth}{2} % for pdf bookmarks

\usepackage[bookmarks,linkcolor=red,citecolor=blue,urlcolor=gray,colorlinks,breaklinks,bookmarksopen,bookmarksnumbered]{hyperref}

%%%%%%%%%%%%%%%%%%%%%%%%%%%%%%%%%%%%%%%%%%%%%%%%%%%%%%%%%%%%%%%%%%%%%%%
% local macros and configurations

\usepackage{../../lib/florian_rabe/basics}
\usepackage{../../lib/florian_rabe/ed}
\usepackage{local}
\renewcommand{\bnf}[1]{{\color{red}#1}}

\pagestyle{plain} % remove for final version

\begin{document}

\lstset{basicstyle={\tt\footnotesize},breaklines}

\title{The Soft Type System of GAP}
\author{Florian Rabe}
\date{LRI Paris, University Erlangen-Nuremberg}
\maketitle

\begin{abstract}
The question of how to design a good type system for mathematics remains open and challenging.
Despite many proposals, no best solution has emerged.
Of particular interest are soft type systems, which can be a compromise between sophisticated type theories and untyped systems.
We contribute to the discussion by giving an easily accessible high-level overview of GAP's soft type system.
\end{abstract}

\section{Introduction}
  \section{Introduction}\label{sec:intro}

\begin{newpart}{MK: adapted from Tom's Thesis}
There is a large and vibrant ecosystem of open-source mathematical software systems.
These systems can range from calculators, which are only capable of performing simple
computations, via mathematical databases (curating collections of a mathematical objects)
to powerful modeling tools and computer algebra systems (CAS). 

Most of these systems are very specific -- they focus on one or very few aspects of
mathematics.  For example, the ``Online Encyclopedia of Integer Sequences''
(OEIS~\cite{Sloane:oeis12,oeis}) focuses on sequences over $\mathbb{Z}$ an their
properties and the ``L-Functions and Modular Forms Database''
(LMFDB)~\cite{Cremona:LMFDB16,lmfdb:on} objects in number theory pertaining to Langland's
program.  GAP~\cite{GAP:on} excels at discrete algebra, whereas
SageMath~\cite{SageMath:on} focuses on Algebra and Geometry in general, and
Singular~\cite{singular:on} on polynomial computations, with special emphasis on
commutative and non-commutative algebra, algebraic geometry, and singularity theory.

For a mathematician however (a user; let us call her Jane) the systems themselves are not relevant, instead she only cares about being able to solve problems. 
Typically, it is not possible to solve a mathematical problem using only a single program. 
Thus Jane needs to work with multiple systems and combine the results to reach a solution. 
Currently there is very little help with this practice, so Jane has to isolate sub-problems the respective systems are amenable to, formulate them into the respective input language, collect results, and reformulate them for the next system a tedious and error-prone process at best, a significant impediment to scientific progress in its overall effect. 
Solutions for some situations certainly exist, which can help get Jane unstuck, but these are ad-hoc and for specific, often-used system combinations only. 
Each of these requires a lot of maintenance and does not scale to a larger set of specialist systems. 

The OpenDreamKit project, which aims at a mathematical VRE toolkit, proposes the Math-in-the-Middle (MitM~\cite{DehKohKon:iop16}) Paradigm, an interoperability framework based on a flexiformal
representation of mathematical knowledge and aligns this with system-generated interface
theories. 

In this paper we instantiate the MitM paradigm with a concrete domain development and
evaluate it on a distributed computing GAP, SageMath and Singular.\ednote{ we generally we
  want to show that the promises in the CICM paper become reality.}

We will use the following example as a running example: Jane wants to act on singular
polynomials with GAP permutation groups\ednote{MK@(MP|VA): }

 \ednote{MK: continue with the structure} 
\end{newpart}

%%% Local Variables:
%%% mode: latex
%%% TeX-master: "paper"
%%% End:


\section{Concepts}
  \subsection{Identifiers}

GAP uses a flat namespace: It identifies every operation by a name.

Different declarations whose types do not conflict, may use the same name.
These can be understood as giving two different types to the same operation.

Each operation can have multiple methods.
These do not have identifiers.
But operation references can be refine to a method reference by using the method's documentation string.

\subsection{Object Level}

GAP allows arbitrary run-time representations of mathematical objects (which is natural for efficiency).
These types of the run-time system are called families.
Each GAP object is typed by exactly one family.

An \textbf{object} is
\begin{compactitem}
  \item a literal,
  \item a list of objects,
  \item a function on object,
  \item any other object introduced by a user-declared family.
\end{compactitem}

Objects are typed.
A \textbf{type} is
\begin{compactitem}
  \item a base type (called a \textbf{family})
  \item a predicate subtype of some type by a unary predicate on objects (called a \textbf{filter})
\end{compactitem}

Users can declare new families.
But some families are built-in:
\begin{compactitem}
  \item one base type for each type of built-in literals:
    \begin{compactitem}
      \item cyclotomic numbers (elements of the algebraic closure of the rationals),
      \item booleans,
      \item strings,
    \end{compactitem}
  \item one base type each for several built-in operators that allow forming complex objects
    \begin{compactitem}
      \item homogeneous lists (called \textbf{collections}): lists of objects that have the same family,
      \item arbitrary lists of objects,
      \item functions on objects.
    \end{compactitem}
\end{compactitem}

A \textbf{filter} is a unary predicate on objects.
\begin{compactitem}
  \item the universal filter $\isobj$ (the filter of all objects)
  \item a category $C$,
  \item a property $P$,
  \item a conjunction $F\wedge G$ of filters.
\end{compactitem}
By convention, the names of atomic filters are of the form \lstinline|IsXXX|.

The \textbf{typing relation} is a binary relation between objects and filters.
We write it as $\has{O}{F}$.
It is defined by
\begin{compactitem}
  \item $\has{O}{\isobj}$ always
  \item $\has{O}{C}$ if $O$ was returned by a constructor of category $C$,
  \item $\has{O}{P}$ if evaluating $P$ on $O$ returns \lstinline|true|,
  \item $\has{O}{F\wedge G}$ if $\has{O}{F}$ and $\has{O}{G}$.
\end{compactitem}

For atomic filters $F$, the known state of the relation $\has{O}{F}$ is cached with $O$.
Thus, the type of every object can be inferred as the conjunction of atomic filters that are known to hold.
This type changes dynamically as more properties are evaluated.

\subsection{Declaration Level}

\paragraph{Categories}
A \textbf{category} declaration introduces a definition-less filter.
A category declaration consists of
\begin{compactitem}
  \item a name,
  \item a filter (called the superfilter).
\end{compactitem}
The concrete syntax is \lstinline|DeclareCategory(name: String, superfilter: Filter)|.

Categories can be used to represent the type of models of an abstract specification.
The details of the specification are formulated by declaring operations and properties on the category.

Because categories are definition-less, all categories are created empty.
The objects typed by the category are introduced by declaring constructors.
These are operations whose implementation explicitly marks the returned objects as having the category as a filter.

\paragraph{Operations}
An \textbf{operation} declaration introduces an $n$-ary\footnote{GAP has an implementation restriction of $n\leq 6$.} function on objects.
Operations are softly typed: Each $n$-ary operations provides a list of length $n$ providing the input filter of the respective argument.
Operations may also carry an optional return type, which defaults to $\isobj$ if omitted.\footnote{This is not implemented yet.}

The concrete syntax is
 \lstinline|DeclareOperation(name: String, inputfilters: Filter*, outputfilter: Filter?)|.

An \textbf{attribute} declaration can be seen as a special case of an operation that is unary.
The special treatment of attributes is important only for efficiency reasons: The values of attributes are cached with each object.
The concrete syntax is \lstinline|DeclareAttribute(name: String, inputfilter: Filter, outputfilter: Filter?)|.

A \textbf{property} declaration can be seen as a special case of an attribute that returns a boolean.
The special treatment of properties is important only because properties can be used as filters.
The concrete syntax is \lstinline|DeclareProperty(name: String, inputfilter: Filter)|.

An \textbf{constructor} declaration can be seen as a special case of an operation that returns an object of a given category.
The concrete syntax is \lstinline|DeclareConstructor(name: String, inputfilters: Filter*, outputfilter: Filter?)|.
\footnote{The return argument is not implemented yet.}
\footnote{The current implementation of constructors is somewhat awkward and may be subject to change. Currently, a constructor's first argument is special: It must be the expected return filter (rather than an object). This is used to allow method selection to choose a different method for different special cases.
A more elegant solution would be to allow every operation to declare that some of its arguments must be filters.
This would yield a very nice untyped version of bounded polymorphism, which is routine in typed programming languages.}

Conceptually, all operations are defined.
However, the definiens is declared separately through methods.

\paragraph{Methods}
Every operation can have multiple definitions, which are declared by methods.
A method declaration consists of
\begin{compactitem}
  \item the named of the operation,
  \item the input and output filters,
  \item the actual definition, as a function in the underlying programming language.
\end{compactitem}

The concrete syntax of a method declaration is
\lstinline|InstallMethod(operationname: String, inputfilters: Filter*, outputfilter: Filter?, definition: function)|.

The input and return filters of a method may be more restrictive than the filters used in the operation declaration.
More restrictive input filters can be used to represent overloading of operations or run-time polymorphism.
A more restrictive output filter can be used to indicate a sharper type than required by the operations.

When evaluating the application of an operation to arguments, one method is selected and its definition executed.
If more than one method is found, whose input filters are type the operation arguments, an internal ranking is used to disambiguate.

\subsection{Theory Level}

There is no explicit theory level.
Instead, theories are represented as categories, and theory morphisms as operations, and their relation is a special case of typing.

Therefore, we can treat each source file as a theory.

\subsection{Document Level}

Source files are grouped into folders and \textbf{packages}.
The package bundled with GAP is called the \textbf{library}.
  
%\section{Relationships to Other Type Theories}
%  \ednote{This section does not work yet. I forgot to quiz Markus on how the binary operations of algebraic structures are treated. I remember GAP hard-codes two binary operations.}

\paragraph{A Logic for GAP Theories}
We can identify a logic and a group of theories that can be naturally embedded into GAP's type system.

Any GAP filter can be used as a type.

Every theory implicitly declares a fixed base type $u$ for the universe.

Then it may have two kinds of declarations:
 \begin{compactitem}
   \item includes of another theory,
   \item function symbols $f:a_1\times\ldots\times a_n\to a_0$ where each $a_i$ is a type (either $u$ or some GAP type),
   \item potential axioms: code in GAP's underlying programming language that evaluates to a boolean
 \end{compactitem}

\paragraph{Representing Theories in GAP}
A theory with name $T$, includes $T_1,\ldots,T_k$, function symbols $f_1,\ldots,f_l$, and potential axioms $p_1,\ldots,p_m$ is translated to GAP as follows:
\begin{compactitem}
 \item We declare a category with name $T$ and superfilter is $T_1\wedge \ldots \wedge T_n$.
 \item We declare an operation for each $f_i$.
 \item We declare a property for each $p_i$.
\end{compactitem}

Now the filter $T\wedge p_1$ represents the type of models of $T$ that satisfy the axiom $p_1$, and accordingly for every subset of the potential axioms.

\paragraph{Extracting Explicit Theories from GAP's}
Because GAP does not enforce an abstraction boundary between theories and types, it is not generally feasible to extract explicit theories from GAP.

A heuristic extraction might be possible by trying to identity groups of GAP declarations for which the above operation can be inverted to yield a theory.
  
\section{Conclusion}
  The main achievements made in Work Package 6 over the last year of the OpenDreamKit project were:
\begin{compactenum}
\item the re-conceptualization of integrating the different aspects of doing mathematics, which led to a better understanding of the nature and intended semantics of VRE components (see Section~\ref{sec:tetrapod}),
\item the integration of a major formal knowledge base into the MitM Ontology, which provides the pivotal point for system integration and service discovery (see Section~\ref{sec:knowledge}),
\item the development of a semantic model for mathematical datasets (see Section~\ref{sec:data}), which has been used in \WPref{dksbases} in two ways: 
  \begin{compactitem}
  \item The Warwick group inventoried all the LFMDB datasets, and to (manually) recover their specifications (schema information) at the mathematical and data base level.
    In essence this retrofits the existing LFMDB project with the a more semantic level and has led to a vastly improved and more semantic API for LMFDB (see \url{http://www.lmfdb.org/api2/}) that has recently come online.
    Moreover, a \Sage interface based on this new API is currently under development.
  \item The Erlangen group built a from-scratch implementation of a hub for mathematical data (see Section~\ref{sec:hub}).
  \end{compactitem}
\item special and adapted search facilities for all kinds of mathematical data and VRE components (see Section~\ref{sec:software}),
\item a standalone implementation of persistent memoization in Python and GAP (see \delivref{dksbases}{persistent-memoization}).
\end{compactenum}
Several of these were described in detail in this report.
In order to describe the general picture, we briefly go over the various parts in the rest of this section.

\paragraph{Knowledge}
We have introduced an upper ontology for formal mathematical libraries (ULO), which we propose as a community standard, and we exemplified its usefulness at a large scale.
We posit ULO as an interface layer that enables a separation of concerns between library maintainers and users/application developers.
Regarding the former, we have shown how ULO data can be extracted from formal knowledge libraries such as Isabelle.
We encourage other library maintainers to build similar extractors.
Regarding the latter, we have shown how powerful, scalable applications like querying can be built with relative ease on top of ULO datasets.
We encourage other users and library-near developers to build similar ULO applications, or using future datasets provided for other libraries.

Finally, we expect our own and other researchers' applications to generate feedback on the specific design of ULO, most likely identifying various omissions and ambiguities.
We will collect these and make them available for a future release of ULO 1.0, which should culminate in a standardization process.

\paragraph{Data}
We have analyzed the state of research data in mathematics with a focus on the instantiation of the general FAIR principles to mathematical data.
Realizing FAIR mathematical data is much more difficult than for other disciplines because mathematical data is inherently complex, so much so that datasets can only be understood (both by humans or machines) if their semantics is not only evident but itself suitable for automated processing.
Thus, the accessibility of the mathematical meaning of the data in all its depth becomes a prerequisite to any strong infrastructure for FAIR mathematical data.

Based on these observations, we developed the concept of Deep FAIR research data in mathematics.
As a first step towards developing a Deep FAIR--enabling standard for mathematical datasets, we focused on relational datasets.
We presented the prototypical \dmh system, which lets mathematicians integrate a dataset by specifying its semantics using a central knowledge and codec collection.
We expect that \dmh also helps alleviate the problem of \emph{disappearing datasets}:
Many datasets are created in the scope of small, underfunded or unfunded research projects, often by junior researchers or PhD students, and are often abandoned when developer change research areas or pursue a non-academic career.

\paragraph{Software: computational mathematical documents}
For the S aspect of what was called D/K/S-structures in the \pn proposal or the \textbf{narration} and \textbf{computation} aspects of the finer tetrapod model from Figure~\ref{fig:tetrapod}, we have developed a formula harvester for Jupyter notebooks and a formula search engine that builds on them.

To make this possible, we had to invest a heavy dose of software engineering into the MathWebSearch system: Even though the system has successfully been used as a formula search engine in the zbMATH publication information system (see \url{https://zbmath.org/formulae/}), the deployment of the system required a lot of domain-specific development and workflow integration.
To this end we have developed Go bindings for the MathWebSearch daemon, documented the interfaces, and provided a web application wrapper.
With this, specific applications only need a domain-specific harvester and minimal customization of our generic front-end. 
We have exercised that for the Jupyter Search engine (as envisioned in task \taskref{dksbases}{mws}) and analogously for a formula search engine for the $n$-category Cafe (nLab, see \url{https://nlabsearch.mathweb.org/}).


\paragraph{Persistent Memoization}
As an integration layer between computation and data, we have developed a persistent memoization infrastructure.
Even though it is called ``persistent'' memoization, the temporal scope of the memoized data is potentially less than the eternity-scope of datasets in LMFDB and \dmh.
Indeed, the characteristic innovation in \delivref{dksbases}{persistent-memoization} is that mathematical objects and data can be shared across multiple computations, in multiple systems of in different runs of the same system.
It allows omitting the semantic level, in which case systems are required to ensure data is read in with the same meaning that it had when it was written out.
But it is flexible enough to use \dmh, or a variant of it, as the physical storage of the data.
Thus, the borders between persistent memoization and mathematical datasets become fluent: indeed, datasets often start as private computation caches and gradually become complete, curated, published datasets.
We will study the spectrum and conversions from memoization caches to \dmh datasets in the future. 

%%% Local Variables:
%%% mode: latex
%%% mode: visual-line
%%% fill-column: 5000
%%% TeX-master: "report"
%%% End:

%  LocalWords:  standardization analyzed Realizing emph ednote summarizes re-conceptualization WPref dksbases compactitem dmh delivref textbf textbf Jupyter zbMATH taskref mws customization



\bibliographystyle{alpha}
\bibliography{../../lib/florian_rabe/bib/rabe,../../lib/florian_rabe/bib/systems,../../lib/florian_rabe/bib/pub_rabe}

\end{document}

%%% Local Variables:
%%% mode: latex
%%% TeX-master: t
%%% End:

%  LocalWords:  Cezary Kaliszyk maketitle conc
