\documentclass{llncs}
\pagestyle{plain}
\usepackage[show]{ed}
\usepackage[utf8]{inputenc}
\usepackage{xspace}
\usepackage{amssymb}
\usepackage[style=alphabetic,backend=bibtex,isbn=false]{biblatex}
\addbibresource{../../lib/kbibs/kwarcpubs.bib}
\addbibresource{../../lib/kbibs/extpubs.bib}
\addbibresource{../../lib/kbibs/kwarccrossrefs.bib}
\addbibresource{../../lib/kbibs/extcrossrefs.bib}
\addbibresource{rest.bib}% add bibs here!
\renewbibmacro*{event+venue+date}{}
\renewbibmacro*{doi+eprint+url}{%
  \iftoggle{bbx:doi}
    {\printfield{doi}\iffieldundef{doi}{}{\clearfield{url}}}
    {}%
  \newunit\newblock
  \iftoggle{bbx:eprint}
    {\usebibmacro{eprint}}
    {}%
  \newunit\newblock
  \iftoggle{bbx:url}
    {\usebibmacro{url+urldate}}
    {}}

\usepackage{hyperref}
\title{Virtual Theories -- A Uniform Interface to Mathematical Knowledge Bases}
\author{
Tom Wiesing\inst{1}
Michael Kohlhase\inst{1} 
Florian Rabe\inst{2} 
}

\institute{
   FAU Erlangen-N\"urnberg
   \and Jacobs University Bremen
%   \and Universit\'e Paris-Sud
}
\begin{document}
\maketitle
\begin{abstract}
  There are various mathematical knowledge collections and information systems
  available. They range from completely informal ones like Wikipedia or the Cornell arXiv,
  zbMath, and MathSciNet via mathematical object databases like the GAP group libraries,
  the Online Encyclopedia of Integer sequences (OEIS), and the L-functions and Modular
  Forms Database (LMFDB) to theorem prover libraires like those of Mizar, Coq, PVS, and
  the HOL systems.\ednote{continue}
\end{abstract}

\section{Introduction}\label{sec:intro}
\begin{todolist}{follow this}
\item mathematical knowledge bases are non-interoperable, and usually only accessible to
  humans (not programmatically; state of the art is to export data, manually massage it
  into SageMath form and make a SageMath module (PIP) to be included into SageMath). 
\item If they are, then they only expose the internal database records, not the
  mathematical objects. e.g. elliptic curves over the rationals as number quadruples where
  one of the numbers is represented as a string.
\item we want to compute with them, e.g. to find new relations between sequences in the
  OEIS (see~\cite{LuzKoh:fsarfo16}), or to find all elliptic curves in the LMFDB whose
  conductor is divisible by 5. 
\item we want to link them, so the contents must be interoperable $\leadsto$ Idea: use
  the MitM approach introduced in \cite{DehKohKon:iop16}, and view math knowledge bases as
  OMDoc/MMT theory graphs (the ABox component of the MitM ontology).
\item 
\end{todolist}
\section{Conclusion}\label{sec:concl}
\printbibliography
\end{document}
%%% Local Variables:
%%% mode: latex
%%% TeX-master: t
%%% End:

%  LocalWords:  maketitle twb sec:intro DehKohKon:iop16 MueGauKal:cacfms17
