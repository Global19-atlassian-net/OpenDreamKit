% !TEX root = ../thesis.tex
\section{Accessing Virtual Theories}\label{sec:access}

We now address \textbf{P2}: lifting physical to semantic representations.
Intuitively, it is straightforward how to implement a virtual theory $V$: we use an initially empty concrete theory $C$, and whenever an identifier \textsf{id} of $V$ is requested, \mmt dynamically adds the corresponding declaration of \textsf{id} to $C$.
\mmt already abstracts from the physical realizations of persistent storage using the \emph{backend} interface: essentially a backend is any component that allows loading declarations.
Thus, we only have to implement a new backend that connects to \lmfdb, retrieves the JSON object with identifier \textsf{id}, and turns it into an \ommt declaration.

However, this glosses over a major problem: the databases used for the scalable physical storage of large datasets usually offer only very simple data structures.
For example, a JSON database (as underlies \lmfdb) offers only limited-precision integers, boolean, strings, lists, and records as primitive objects and does not provide a type system.
Consequently, the objects stored in the database are very different from the sophisticated mathematical objects expected by the schema theory.
Therefore, databases like \lmfdb must encode this complex mathematical objects as simple database objects.

\subsection{Concrete Encodings of Mathematical Objects}\label{sec:vt:translation}

Consider, for example, the field \identifier{degree} from Figure~\ref{fig:lmfdbexample} above.
Its \emph{semantic} type in the MitM-formalization is $\mathbb{Z}$.
However, its \emph{physical} type in \lmfdb is \identifier{IEEE 754} a mixture of $64$-bit floating point numbers and strings:
integers that exceeds $2^{53}-1$ are stored as JSON strings containing the corresponding decimal representation.
We speak of \emph{encoding} when translating semantic objects to their physical representations and of \emph{decoding} in the dual case, and we speak of \emph{codecs} when referring to a pair of an encoding and a decoding function.

To formally specify codecs, we introduce a new \ommt theory \texttt{Codecs} as a part of the MitM ontology.
Our codecs are indexed by semantic types: the type constructor \codectt maps a semantic type to a new type of codecs for it.
For instance, the object \identifier{StandardInt} of type $\codectt\;\mathbb{Z}$ is a codec that translates between \lmfdb's idiosyncratic float/string-representation and MitM's integers.
Note that there can be multiple different codecs for the same semantic type.
For example, \identifier{IntAsArray} encodes integers $x$ as lists of $64$-bit integers consisting of the digits of $x$ with respect to base $2^{64}$.

\begin{figure}[ht]\centering
  \begin{tikzpicture}\footnotesize
    \node[thy] (codecs) at (0,0) {
      \begin{tabularx}{.84\textwidth}{lll|X}
        \multicolumn{4}{l}{\textsf{Codecs}} \\\hline\hline   
        \identifier{codec}    & : & \multicolumn{1}{l}{$\typett\rightarrow\typett$} & \\\hline
        \identifier{StandardPos}    & : & $\codectt\; \mathbb{Z}^{+}$   & \multirow{3}{*}{\begin{minipage}{3.8in}
                                                                                      JSON number if small enough, \\
                                                                                      else JSON string of decimal expansion
                                                                                      \end{minipage}}\\
        \identifier{StandardNat}    & : & $\codectt\; \mathbb{N}$       & \\
        \identifier{StandardInt}    & : & $\codectt\; \mathbb{Z}$       & \\\hline
        \identifier{IntAsArray}     & : & $\codectt\; \mathbb{Z}$       & JSON List of Numbers\\
        \identifier{IntAsString}    & : & $\codectt\; \mathbb{Z}$       & JSON String of decimal expansion\\\hline
        \identifier{StandardBool}   & : & $\codectt\; \mathbb{B}$       & JSON Booleans \\
        \identifier{BoolAsInt}      & : & $\codectt\; \mathbb{B}$       & JSON Numbers $0$ or $1$ \\\hline
        \identifier{StandardString} & : & $\codectt\; \mathbb{S}$       & JSON Strings \\
      \end{tabularx}
    };
  \end{tikzpicture}
  \caption[List of Codecs]{
    Codecs specified in \mmt ($\mathbb{N}$, $\mathbb{Z}$, $\mathbb{Z}^{+}$ are as usual, $\mathbb{B}$ are booleans, and
    $\mathbb{S}$ are Unicode strings)
  }
  \label{fig:codecs}
\end{figure}

We do not (and do not have to) define the actual encoding/decoding functions in \ommt.
It is more important to identify the codecs needed in practice, introduce names for them, and spell out their semantics.
Then it is straightforward to implement them in any other programming language used interfacing with \lmfdb.

In particular, we have implemented them in Scala, the language underlying the \mmt system.
Additionally, the \textsf{Codecs} theory annotates each codec declaration with a reference to the Scala class implementing the codec.
That way, \mmt can run the encoding/decoding functions of the codec.

\begin{wrapfigure}r{2.7cm}
$M = \left(
    \begin{array}{ccc}
      1 & 5 & 25 \\
      5 & 1 & 5 \\
      25 & 5 & 1 \end{array} 
  \right)$\vspace*{-1em}
\end{wrapfigure}

The above is only sufficient for atomic semantic types, which typically correspond to one (or more) atomic codecs.
Consider now the field \identifier{isogeny\_matrix} of elliptic curves.
The semantic representation of one possible value (namely for the curve \identifier{11a1}) of this field is the matrix on the right.

The semantic type operator $\mathrm{Matrix}$ takes $1$ type argument (the element type, integers in this case) and two value arguments (the dimensions, $3$ and $3$ in this case) and constructs the respective matrix type. 
In principle, one could give a codec for each matrix type that comes up in a database schema. 
But a much more elegant solution is to specify \textbf{codec operator}s in analogy to type operators.
A codec operator for a type operator with $k$ type and $l$ value arguments, takes $k$ codec and $l$ value arguments.
For example, a codec operator for matrices takes a codec $C:\codectt\, E$ for the element type $E$ and the dimensions $m$ and $n$ and returns a codec of type $\codectt\,(\mathrm{Matrix}\,E\,m\,n)$.

\begin{figure}[ht]\centering
  \begin{tikzpicture}\footnotesize
    \node[thy] (codecs) at (0,0) {
      \begin{tabularx}{\textwidth}{lll|X}
        \multicolumn{4}{l}{\textsf{Codecs (continued)}} \\\hline\hline   
        \identifier{StandardList}    & : & 
                 $\left\{T\right\} \codectt\; T \rightarrow \codectt\; \mathrm{List}(T)$ & 
                  JSON list, recursively coding each element of the list\\\hline
        \identifier{StandardVector}    & : & 
                  $\left\{T, n\right\} \codectt\; T \rightarrow \codectt\; \mathrm{Vector}(n, T)$ & 
                   JSON list of fixed length $n$\\\hline
        \identifier{StandardMatrix}    & : & 
                   $\left\{T, n, m\right\} \codectt\; T \rightarrow \codectt\; \mathrm{Matrix}(n, m, T)$ & 
                   JSON list of $n$ lists of length $m$\\
      \end{tabularx}
    };
  \end{tikzpicture}
  \caption[List of Codec Operators]{
    Second annotated subset of the codecs theory containing a selection of codec operators found in \mmt. 
    Compare with Figure~\ref{fig:codecs}. 
  }
  \label{fig:codecops}
\end{figure}
Like codecs, codec operators are represented in \mmt in two ways: as declarations inside the theory \identifier{Codecs} (see Figure~\ref{fig:codecops} for a list, compare again with Figure~\ref{fig:vtarch}) and as a corresponding Scala function that maps codecs to codecs. 
When reading the declarations, note that we make use of the dependent function types of the MitM foundation: curly brackets denote dependent function arguments, i.e., argumetns that may occur in later argument types and the result type.

With these declarations, we recover the \lmfdb encoding of isogeny matrices by applying the codec operator \identifier{StandardMatrix}, which encodes matrices as lists of lists, to the codec \identifier{StandardInt} and the dimension $3$ and $3$.
The resulting codec $\plaintt{StandardMatrix}(\mathbb{Z}, 3, 3, \plaintt{StandardInt})$ encodes the above matrix as\ \lstinline|[[1.0,5.0,25.0],[5.0,1.0,5.0],[25.0,5.0,1.0]]|.

\subsection{Choosing Encodings in Schema Theories}\label{sec:vt:schema}

If we ignore encoding issues, schema theories are straightforward: they contain one declaration of the same name for each field within an \lmfdb record. 
This specifies only the semantic type of each field and does not relate it to the MitM formalization.
To handle the encoding as a physical type, we annotate each declaration with the codec that the databases for the values of that field.
Moreover, to connect the schema theories to the MitM formalization, we additionally annotate each field with the corresponding property of elliptic curves from the MitM theory.
We can now understand the last unexplained parts of Fig.~\ref{fig:vtarch}.
\identifier{?implements} is the symbol used to annotate the metadatum, which MitM property a schema field corresponds to.
And \identifier{?codec} similarly annotates the codec to each field.

For example, the \identifier{degree} field implements the \identifier{curveDegree} property in the elliptic curve theory and uses the \identifier{StandardInt} codec.
Thus, the schema theories determine the entire relation between semantic and physical objects.

The database theory is a virtual theory and contains one declaration per \lmfdb record.
Given the URI of an object in the respective database, our \mmt backend for \lmfdb first retrieves the appropriate record from {\lmfdb} -- in the case of \identifier{11a1} this corresponds to retrieving the JSON found in Figure~\ref{fig:lmfdbexample}. 
Then, for each field, it uses the annotated codec (which is an \ommt expression) to build an actual codec (as a runnable Scala function) and runs its decoding function.
Next, it passes the resulting record to the \identifier{from\_record} constructor, which yields an elliptic curve in the MitM theories.
Finally, this elliptic curve is added as a new declaration in the database theory.


%%% Local Variables:
%%% mode: latex
%%% TeX-master: "paper"
%%% End:

%  LocalWords:  sec:vt lmfdb ommt textit textit realized tikzpicture tabularx textwidth sec:access realizations emph wrapfigure vspace lstlisting
%  LocalWords:  hline hline rightarrow codectt mathbb multirow fig:codecs isogeny mathrm
%  LocalWords:  characterized noindent formalized fig:codecops plaintt plaintt begingroup
%  LocalWords:  pgfdeclarelayer pgfsetlayers background,foreground resizebox xscale ec
%  LocalWords:  4,yscale pgfonlayer tikzstyle red,dashed,thick withshadow draw,drop oding
%  LocalWords:  cylinder,cylinder 50,cylinder 0.25,draw none,fill 30,rounded 1cm,fit
%  LocalWords:  thy,interface ecschema draw,cloud,fit 4,withshadow,inner 4pt,purple
%  LocalWords:  mongodb dbtheory endgroup fig:vtarch colored colored colored colored
%  LocalWords:  fig:lmfdbexample textbf centering RabKoh:WSMSML13 sec:mmtmitm
%  LocalWords:  internalizes
