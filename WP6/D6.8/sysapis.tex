\subsection{Generated System API Theories}\label{sec:sysapis:gentheories}

We have generated and cross-linked system APIs for the computer algebra systems GAP,
SageMath, and SINGUALAR, they can be found at
\url{https://gl.mathhub.info/ODK/{GAP,SAGE,SINGULAR,LMFDB,knowls}}.\ednote{MK: write some more}
These contain:
\begin{itemize}
	\item[\texttt{GAP}] (210 Theories, 8470 Symbols, 64 Commits)
		\footnote{Since the generated API theories do not come from \texttt{.mmt} source files, 
		numbers of files and LoF are not informative metrics. Approximately, one \texttt{.mmt} 
		file contains on average 2--3 theories, and 1--2 LoF correspond to one symbol.} 
		Generated from a JSON export from the GAP system. A theory corresponds to a
		source file of a GAP package, a symbol represents a GAP method or operation.
	\item[\texttt{Sage}] (5431 Theories, 7279 Symbols, 73 Commits) Analogously generated 
		from a JSON export. Theories correspond to Sage categories, symbols to methods.
	\item[\texttt{Singular}] (179 Theories, 4519 Constants, 6 Commits) \ednote{find out what theories represent?}
	\item[\texttt{LMFDB}] (161 Commits) Consists of schema theories for LMFDB databases 
		(currently 5 databases covered, 182LoF) and interface theories for non-mathematical or 
		LMFDB-specific concepts (labels, descriptors) represented therein (3 Files, 123 LoF)
\end{itemize}

\subsection{The MitM Alignments}

The MitM alignments are available from \url{https://gl.mathhub.info/alignments/Public} (6
files, 1040 LoF, 240 Commits). They are represented as text files containing pairs of MMT
URIs annotated with various semantic classification schemata;
see~\cite{MueGauKal:cacfms17} for details.\ednote{MK: what else can we write about them?}
\ednote{this figure does not fit in the text width}
\begin{figure}[ht]\centering
  \tikzinput[width=.98\textwidth]{../D6.5/alignmentimg}
  \caption{Alignments between the MitM Ontology and the \GAP API}\label{fig:cgtontology}
\end{figure}

As an example, Figure~\ref{fig:cgtontology} sketches the alignments between the
computational group theory ontology from above and the constructors and operations of
\GAP.

%%% Local Variables:
%%% mode: visual-line
%%% fill-column: 5000
%%% mode: latex
%%% TeX-master: "report"
%%% End:
