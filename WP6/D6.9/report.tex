\documentclass{deliverablereport}

\deliverable{dksbases}{persistent-memoization}
\deliverydate{XX/YY/201Z}
\duedate{28/02/2019 (M42)}
\author{Michael Torpey}

\usepackage{multicol}
\usepackage{multirow}

\newcommand{\pypersist}{\texttt{pypersist}}

\begin{document}
\maketitle
% This will be the abstract, fetched from the github description
\githubissuedescription

% write the report here

\section{Introduction}
\label{sec:intro}

% What is memoisation?
\textbf{Persistent memoisation} refers to the computational practice of storing
program results whenever they are computed, and then looking up and retrieving
them later, instead of re-running programs that are guaranteed to return the
same answer.  This approach allows users to avoid unnecessary computation, and
makes it possible to create an archive of results that can be used for other
purposes later.

% Why is it good for maths?
This approach may be useful in a wide range of fields, but it is particularly
relevant to mathematics because of the type of problems mathematicians wish to
solve.  Mathematics is a rare example of a field with algorithms that require a
lot of processing, but have a small input and small output.

Many algorithms in the real world have a large input and a small output -- for
example, the problem of determining which streamed TV show to recommend to a
viewer takes as its input a huge amount of data about past choices consumers
have made, and returns a single title as its output.  The output of this
function can be stored easily, but so much time would be spent processing the
large input, and the same input would be repeated so infrequently, that this
type of memoisation would have little advantage.

Conversely, many algorithms have a small input and a large output -- for
example, rendering a plain text file on a screen takes a small amount of data as
an input, but creates an image which is likely to be large if stored.  In this
case, the input can be dealt with quickly, but memoising the output using a disk
is likely to be slower than recomputing it from scratch each time.

In mathematics, however, we frequently encounter computational problems with
both small input and small output.  For instance, we may wish to know whether a
given permutation group $G$ is simple; the input may be two or three generators
for $G$, which between them use only a few bytes on a computer, and the output
is a single boolean.  But despite the small sizes of these data sets, the
algorithm may take a long time to complete, and may use a large amount of memory
while in progress.  Thus, memoising results of this type will use very little
disk space, require very little processing, and may avoid a huge amount of work
when recalling previously computed results.

Furthermore, there is no reason that a memoisation cache should be limited to a
single user.  A memoisation framework could save results to a shared directory
for immediate use by other researchers connected by a network, or could even
use an online database for the cache, so long as sufficient precautions were
taken to avoid reading and writing clashes.

Besides the obvious advantages of recalling cached results instead of repeating
work, a memoisation framework that allows a human-readable cache could be used
by researchers to create a reproducible record of computed results that can be
included in papers, or shared with collaborators that have no knowledge of the
memoisation system.

The aim of ODK Task 6.9 is to establish a persistent memoisation framework to
cache results in Python and GAP across sessions, in a way that is easy to deploy
and configure, and allows for results to be shared reliably between different
researchers.  Deliverable 6.9 requires a ``shared persistent memoisation library
for Python/Sage'', which is fulfilled by \pypersist{}, a new Python library
written by the current author.  The corresponding software for GAP is in
progress, in the form of GAP's \texttt{Memoisation} package, which aims to have
the same features as \pypersist{}, and to be as compatible as possible with it.

In this report, we will present a review of some existing tools for memoisation
(Section \ref{sec:existing}), describe the new Python library \pypersist{}
(Section \ref{sec:pypersist}), and give an overview of the future direction of
this project (Section \ref{sec:future}).

\section{Existing tools}
\label{sec:existing}

In this section we will consider some tools that already exist for memoisation,
with a focus on Python and Sage tools.  Some of the tools we discuss are only
intended for a subset of the complete memoisation process, while others are more
full-featured but lack some important desirable features.  A comparison of these
tools is shown in Table \ref{tab:comparison-of-tools}.

\begin{table}[h]
  \renewcommand{\arraystretch}{1.2}
  \begin{tabular}{|l|c|c|c|c|c|c|c|c|c|c|c|c|}\cline{2-13}
    \multicolumn{1}{c|}{ }
    & \rotatebox{270}{Updated in last year}
    & \rotatebox{270}{Python versions}
    & \rotatebox{270}{Function decorator}
    & \rotatebox{270}{Memory caching}
    & \rotatebox{270}{Disk caching}
    & \rotatebox{270}{Database caching}
    & \rotatebox{270}{Custom keys}
    & \rotatebox{270}{Custom pickling}
    & \rotatebox{270}{Metadata}
    & \rotatebox{270}{Sage support}
    & \rotatebox{270}{Cython functions}
    & \rotatebox{270}{Implemented in Cython~~}
    \\ \hline
    GAP operation/attribute & \checkmark & --- &  & \checkmark &  &  &  &  &  &  &  &  \\ \hline
    Sage \texttt{func\_persist} & \checkmark & Sage & \checkmark & \checkmark & \checkmark &  &  &  &  & \checkmark &  &  \\ \hline
    Sage \texttt{cached\_function} & \checkmark & Sage & \checkmark & \checkmark &  &  & \checkmark &  &  & \checkmark & \checkmark & \checkmark \\ \hline
    Sage \texttt{cached\_method} & \checkmark & Sage & \checkmark & \checkmark &  &  & \checkmark &  &  & \checkmark & \checkmark & \checkmark \\ \hline
    \texttt{persist} &  & $2/3$ &  & \checkmark & \checkmark &  &  &  &  &  &  &  \\ \hline
    \texttt{PyMemoize} & \checkmark & $2/3$ & \checkmark & \checkmark & \checkmark &  &  &  &  &  &  &  \\ \hline
    \texttt{redis-simple-cache} &  & $2/3$ & \checkmark & \checkmark & \checkmark & \checkmark &  &  &  &  &  &  \\ \hline
    \texttt{dogpile.cache} & \checkmark & $2/3$ & \checkmark & \checkmark & \checkmark & \checkmark & \checkmark &  &  &  &  &  \\ \hline
    \texttt{pypersist} & \checkmark & $2/3$ & \checkmark & \checkmark & \checkmark & \checkmark & \checkmark & \checkmark & \checkmark & \checkmark &  &  \\
    \hline
  \end{tabular}
  \vspace{7pt}
  \caption{Comparison of memoisation tools}
  \label{tab:comparison-of-tools}
\end{table}

\subsection{GAP}
First, we should consider the existing persistence tools built into GAP and
Sage.  GAP functions can be installed as methods for an object's
\emph{operations} and \emph{attributes}, and if they are, their results are
stored for the duration of the current GAP session.  However, these results are
never saved to disk unless the session itself is stored, and the option is only
available for methods of an object.

\subsection{Sage}
There exists in Sage a simple function decorator, \texttt{func\_persist}, which
can be applied to a function to save its output to disk.  This is a working
memoisation tool, and even this functionality has great practical use, but the
tool has very little configurability: the only option that can be given is to
specify the directory in which results are stored.  Sage also contains two more
configurable tools for memoising functions and methods, respectively named
\texttt{cached\_function} and \texttt{cached\_method}.  These allow a function
or method to be memoised with a custom key-generating function, meaning that
arguments can be pre-processed in an intelligent way, perhaps to sort arguments,
or discard arguments that do not affect the return value of a function.
They are also implemented in Cython, meaning that they are likely to perform
faster than \texttt{func\_persist}, and have support for functions written in
Cython.
However, they only support memory caching -- that is, they only store results in
memory while the current program is running, and do not save results to disk for
use in a later session.

These three decorators are useful, but limited.  Not only do they lack certain
options such as database caching and custom output pickling, but they rely on
other parts of the Sage system, and therefore cannot be used more broadly in
generic Python programs.  Python memoisation tools exist outside Sage that can
in principle be applied to Sage.

\subsection{Python}
Many memoisation tools are available on PyPI, each with a slightly different
philosophy and set of features.  Table \ref{tab:comparison-of-tools} summarises
four examples of these, indicating each one's features and status:
\texttt{persist} and \texttt{redis-simple-cache} have not been updated in
several years, while \texttt{PyMemoize} lacks even the custom key function
provided by the Sage tools.

Perhaps the most promising of the existing tools is \texttt{dogpile.cache}, an
API for caching function outputs to a variety of backends; this is a
well-established, full-featured system, but it lacks certain features such as
custom result pickling -- particularly useful to mathematicians who may want to
store output using OpenMath, or in some human-readable form for sharing.

There would be an argument for modifying and improving \texttt{dogpile.cache} to
add these features, rather than creating a new tool.  However, given the wide
scope of the \texttt{dogpile} project, which is largely focused on thread
management, it was decided to proceed with a new tool that could be adapted more
easily to suit the specific requirements of the project, such as bespoke support
for Sage-specific objects, and compatibility with the GAP memoisation package
which would be developed in parallel.

To fulfill these requirements, the \pypersist{} package was created.  It runs in
Python version 2 and 3, and it contains all the features mentioned in Table
\ref{tab:comparison-of-tools}, except those features related to Cython.  More
information about \pypersist{} is given in Section \ref{sec:pypersist}.

\section{pypersist}
\label{sec:pypersist}

\pypersist{} is a package for Python 2 and 3, created specifically to address
the requirements of this deliverable.  It has all the features considered in
Table \ref{tab:comparison-of-tools}, having been designed with these features in
mind.

\subsection{Features}
\label{sec:features}

We now summarise some of the features of \pypersist{}.  The detailed workings of
the package are outside the scope of this report, but full documentation is
available at \url{https://pypersist.readthedocs.io}, and an extensive annotated
demonstration of the package's features is given in an interactive notebook
hosted on Binder at
\url{https://mybinder.org/v2/gh/mtorpey/pypersist/master?filepath=binder/demo.ipynb}.

\paragraph{Decorator}
In order to use all the features of the package, the \texttt{persist} decorator
is the only thing that needs to be imported.  After this, a function can be
memoised by simply adding \texttt{@persist} above it.  This minimal setup
produces memoisation with safe defaults, and allows users to exploit the package
with no additional knowledge of its features.  However, using
\texttt{@persist(...)} allows further customisation via arguments.

\paragraph{Configurable cache location}
By default, results are saved to the \texttt{persist/<funcname>} subdirectory of
the current working directory.  However, by providing a string as the
\texttt{cache} argument of \texttt{@persist}, this location can be changed.

\paragraph{Configurable function name}
In order to avoid confusion between two different functions with the same name,
a \texttt{funcname} argument can be supplied to \texttt{@persist}, which should
be a string that uniquely identifies the function.  This string could include a
version number, author name, or anything else required to ensure uniqueness.

\paragraph{MongoDB cache}
Support is provided for caching results in a MongoDB database, via the REST
interface provided by Python's \texttt{Eve} package.  If a \texttt{cache}
argument is provided which starts with \texttt{mongodb://}, then it is
interpreted as the address of a MongoDB server which can send and receive
results for the function, and that server will be used instead of saving results
locally.  Such a server can be set up using the \texttt{mongodb\_server/run.py}
script distributed with \pypersist{}.

\paragraph{Custom key function}
Each entry in a function's cache is stored using a unique key based on the
arguments supplied to the function; two sets of arguments should have the same
key only if they produce the same output.  By default, we use a tuple of the
arguments, with their names, sorted alphabetically, and with any default
arguments removed.  This allows for equivalent calls such as \texttt{foo(x=3,
  y=5)} and \texttt{foo(y=5, x=3)} to be treated equally.  However, the
decorator takes an optional \texttt{key} argument, which should be a function
that takes a list of arguments and produces a key based on them.  A developer
could thus choose to ignore arguments that control, for example, verbosity, or
other options that do not affect the function's return value.

\paragraph{Custom hashing}
When a result is stored in the cache, it is stored using a hash of its key.
This allows a record to be looked up quickly, and perhaps removes the necessity
of storing the keys at all.  By default, a key is hashed using its SHA-256
encoding, which makes it extremely unlikely that a hash collision will ever be
encountered.  However, a custom hash function can be specified using a
\texttt{hash} argument to the decorator, and in this case it will be used to
produce a hash instead of the default method.  If an injective hash function is
chosen, then its inverse can be specified using the \texttt{unhash} argument;
this allows hashes to be double-checked, and a list of previously-encountered
keys to be produced.

\paragraph{Storing keys}
By default, a key is not stored after it has been hashed, and correctness is
assumed due to the vanishingly small probability of an SHA-256 hash collision.
However, for complete correctness, the decorator takes an argument
\texttt{storekey} which, if set to true, stores keys along with results, and
checks them when looking up results, raising an error if a hash collision is
detected.  This also allows a user to iterate over the stored keys of a cache,
as mentioned below.

\paragraph{Custom pickling}
In order to store an object in a cache, it needs to be preserved on disk or in a
database.  For this purpose, we require sometimes complex objects to be
\textit{pickled} and \textit{unpickled} -- that is, converted to a string for
storing, and converted back when retrieved later.  The default methods use
Python's \texttt{pickle} module (or, if appropriate, Sage's own pickling
functions) and a base 64 encoding to produce strings that
can be written without invisible or difficult-to-display characters.  However,
like much of \texttt{pypersist}'s functionality, custom \texttt{pickle} and
\texttt{unpickle} functions can be specified that do this a different way.  This
can allow results to be stored in a human-readable way which can then be used
outside Python, or even to be stored using a format such as OpenMath for use in
a completely different computer system.

\paragraph{Manual cache interaction}
It may be necessary at some time to modify a memoisation cache manually --
perhaps to remove an inaccurate result, or to add a result that was found before
the cache was created.  This can be done directly in \pypersist{} -- a memoised
function \texttt{foo} has an attribute \texttt{foo.cache} which can be used much
like a dictionary, reading, setting and deleting entries as desired.
Furthermore, if \texttt{storekey} is set to true, or if an \texttt{unhash}
method is provided, \texttt{foo.cache} is an iterable object with attributes
\texttt{keys}, \texttt{values} and \texttt{items} that can be used in loops.

\subsection{Software engineering}
\texttt{pypersist} was written using modern open-source software technologies
and practices, with an emphasis on clarity of understanding and ease of use.
The report on OpenDreamKit D1.5 contains, in Section 4.3.2.1, a checklist of
software engineering best practices.  The package fulfills all the requirements
in that list, as summarised in Table \ref{tab:pypersist-se-check}.

\begin{table}[h]
  \renewcommand{\arraystretch}{1.2}
  \begin{tabular}{|p{5.1cm}|c|p{9.5cm}|}\hline
    Version control & \checkmark & Git \\ \hline
    Tests & \checkmark & \multirow{2}{*}{\emph{pytest} suite with 98\% code coverage} \\ \cline{1-2}
    Automated tests & \checkmark & \\ \hline
    Continuous integration & \checkmark & Travis runs test suite for every commit \\ \hline
    Automatic building of releases & \checkmark & PyPI release script \\ \hline
  \end{tabular}
  \vspace{0pt}
  \caption{Software engineering good practice checklist for \pypersist{}}
  \label{tab:pypersist-se-check}
\end{table}

The project was developed from the beginning using Git, and the full history is
available in the repository.  A test suite using \textit{pytest} is included in
the repository, and is run by Travis every time a commit or pull request is
added on Github.  The comprehensiveness of this suite is measured by
\textit{codecov}, and covers 98\% of the code in the package, with the only
untested lines being those that handle difficult-to-produce HTTP errors.  The
package is built for PyPI using Python's well-established \texttt{setuptools}
module, and several such releases have already been made.

\subsection{Dissemination}
The report on OpenDreamKit D1.5 also contains, in Section 4.3.2.2, a checklist
of best practices for dissemination.  The package also fulfills all of these
requirements, as summarised in Table \ref{tab:pypersist-diss-check}.

\begin{table}[h]
  \renewcommand{\arraystretch}{1.2}
  \begin{tabular}{|p{5.1cm}|c|p{9.5cm}|}\hline
    Host code publicly & \checkmark & \url{https://github.com/mtorpey/pypersist} \\ \hline
    Reference Manual (APIs) & \checkmark & \url{https://pypersist.readthedocs.io} \\ \hline
    Tutorial (for beginning users) & \checkmark & \multirow{3}{9.5cm}{Examples section in readme, and Binder demo with extensive annotations: \texttt{binder/demo.ipynb} \\ (linked from readme and manual)} \\ \cline{1-2}
    Examples & \checkmark & \\ \cline{1-2}
    Live interactive online demos & \checkmark & \\ \hline
    Support mechanisms & \checkmark & Github issues \\ \hline
    How to cite the output? & \checkmark & Explained in readme and manual \\ \hline
    Installation mechanism & \checkmark & Installation via \emph{pip}, explained in readme and manual \\ \hline
    High level description accessible to non-experts & \checkmark & In readme, manual and Binder \\ \hline
    URLs/Blog/etc to and from OpenDreamKit project & \checkmark & Links to OpenDreamKit in readme, in manual, and on PyPI distribution page \\ \hline
    Grant acknowledgements & \checkmark & Acknowledged in readme and manual \\ \hline
    Open Source license & \checkmark & GPL v2 or later \\ \hline
    Workshop & \checkmark & \multirow{2}{9.6cm}{Demo and discussion at \emph{Free Computational Mathematics} conference, CIRM, Luminy, France, Feb 2019} \\ \cline{1-2}
    Engaging users & \checkmark & \\ \hline
  \end{tabular}
  \vspace{0pt}
  \caption{Dissemination good practice checklist for \pypersist{}}
  \label{tab:pypersist-diss-check}
\end{table}

Development of \pypersist{} was open from the start, available publicly on
Github, including its full commit history.  The issue tracker on its Github page
has been used to track upcoming new features, as well as bugs and other
problems.  All documentation, tools, and setup are included directly in the Git
repository, so that everything is kept in one place and can be tracked as a
single unit.

The package's documentation is mostly generated directly from the docstrings in
its source files.  This means that code is documented where it is written,
making the code easier to understand and avoiding the need to duplicate
documentation with extra comments.  Numpy-style docstrings are used throughout
-- this widely used standard was chosen for its minimal syntax, which allows the
strings to be easily read and understood in plaintext without any processing or
compilation.  A readme is included, written in Markdown, with a simple
two-sentence introduction, a guide to installation, and a small number of examples
showing how to get started.  Autodoc and Sphinx are used to convert the
docstrings and the readme, with very little additional documentation required,
into a legible manual, which is hosted on \textit{ReadTheDocs} and linked from
the readme itself.  This manual is updated automatically on each push to the
repository.  In this way, the documentation is accessible, clear, and
maintainable.

The minimal examples in the readme are supplemented by a Jupyter notebook which
is also included with the project.  This notebook imports \pypersist{} and
demonstrates every feature available, with annotations explaining each example.
The notebook can be loaded locally by anyone with a Jupyter installation, but
for better accessibility it is also hosted on Binder, with its address linked at
the bottom of the readme, as well as in a badge at the top.  This demo loads the
most recent version of \pypersist{}, and therefore also acts as an additional
test suite, since it requires the package to be in a working state in order to
demonstrate its features.  The outputs of the executable cells are not saved
into the notebook -- instead it is left to readers to execute the cells
themselves, producing an interactive experience which may encourage the reader to
experiment, and may lead to a better understanding of how the package works.
This notebook was demonstrated at the \emph{Free Computational Mathematics}
conference held in CIRM, Luminy, France, in February 2019, prompting a
discussion and feedback from users.

In order for \pypersist{} to be used, it must be easy to install.  The package
is hosted on PyPI at \url{https://pypi.org/project/pypersist}, making
installation as simple as typing \texttt{pip install pypersist} for any user
connected to the internet.  This installation method is explained in the
project's readme, and therefore also on the first page of its manual.

% 3rd checklist: perhaps doesn't add anything to include?
% \subsection{Pathways to impact}
% \begin{itemize}
% \item[{$\square$}] Does the software address the needs of the users?
% \item[{$\square$}] Workshops to gather feedback
% \end{itemize}

\section{Future direction}
\label{sec:future}

Although \pypersist{} now has many features and is well-tested, it is still in
development.  It is hoped that as more developers begin to use it, they will
provide feedback that will inform the development further, whether by revealing
bugs, or by requesting new features.  There are already several features that
are planned, and work will continue on this project as time goes on.  In this
section we describe some aims for the future direction of Task 6.9, in
\pypersist{} and beyond.

\subsection{Features}

\paragraph{CouchDB}
Support currently exists in \pypersist{} for a MongoDB database backend.  This
works well, but there are certain features offered by the CouchDB system that
might also be useful.  CouchDB emphasises ease of setup, and allows information
to be added and retrieved from the database via HTTP, with a human-readable
web-based front-end that would allow human-readable results to be shared
instantly on computation via a website.  It is likely that implementing this
would require little deviation from the existing code for MongoDB, and could
result in big improvements in ease of use, ease of setup, and interoperability.

\paragraph{Better metadata}
\pypersist{} has the capacity to store metadata, which may be useful for storing
information about a given computation.  However, no
metadata is stored by default, and metadata is not validated in any way when a
result is loaded from the cache.  Some decision should be made on what metadata
should always be stored, and functionality should be added that allows this
metadata to be used where relevant, for example by allowing a cache to be
filtered by date or user.
Some useful fields could be where and when a result was originally computed,
comments from the user or the author of the function, and information about the
system that produced the result.

\paragraph{Provenance tracking}
One use of metadata would be to record where a given result came from, allowing
researchers to evaluate the trust they are willing to place in it.  A limited
form of this can be achieved by sharing results using a version control system
such as Git, in which the timestamp and author of a commit give some useful
information.  But a more full-featured provenance-tracking system could be added
in the future, recording the exact time and place that a computation was
performed, as well as the user that called it, and version numbers for both
\pypersist{} and the function being memoised.

\paragraph{Verbosity}
Results in \pypersist{} are stored and retrieved silently, and users are not
made aware of its existence in any way except when errors are raised (for
example due to a hash collision or HTTP error).  It may be useful at times to
display messages to the screen showing the progress of memoisation, so that
interested users have a record of exactly what is being written and read at what
time and in what location.  Different levels of verbosity showing different
levels of information may be useful.

\paragraph{GAP version}
Though Deliverable 6.9 only requires a ``shared persistent memoisation library
for Python/Sage'', Task 6.9 also discusses a GAP version of this library.  GAP
would benefit from all the features included in \pypersist{}, and a GAP version
of this library would also be able to interface with the same cache as the
Python version in certain circumstances, allowing results to be shared across
different computer algebra systems as well as across different sessions.  A
prototype of this already exists in the form of GAP's \texttt{Memoisation}
package, which was developed in parallel with \pypersist{}.  However, this
prototype is not yet stable or full-featured, and should be developed into a
tool comparable to the Python version.

\paragraph{Sage integration}
As shown in Table \ref{tab:comparison-of-tools}, there exist three different
tools already in Sage for memoising data.  Now that \pypersist{} has overtaken
them in terms of features, it is possible that it could replace those tools
entirely.  With careful integration and testing to ensure that there were no
regressions, \pypersist{} could be imported and used instead of those tools,
possibly with the addition of some Sage-specific options that would allow it to
be used as effectively as possible.  However, \pypersist{} should remain a
generic Python package that can be used outside as well as inside Sage.

\subsection{Reporting}
It is planned that \pypersist{} version 1.0 will be released by the end of the
OpenDreamKit project.  The work done in preparation for that release, including
the features mentioned above, may be summarised in a later report that would act
as an addendum to this document.

\end{document}

%%% Local Variables:
%%% mode: latex
%%% TeX-master: t
%%% End:

