\documentclass{deliverablereport}

\deliverable{dksbases}{persistent-memoization}
\deliverydate{02/04/2019}
\duedate{28/02/2019 (M42)}
\author{Michael Torpey}

\usepackage{multicol}
\usepackage{multirow}

\newcommand{\pypersist}{\textsc{pypersist}}
\newcommand{\Memoisation}{\textsc{memoisation}}

\begin{document}
\maketitle
% This will be the abstract, fetched from the github description
% TODO: update abstract on Github
\githubissuedescription
\clearpage


% write the report here

\section{Introduction}
\label{sec:intro}

% What is memoisation?
\textbf{Persistent memoisation} refers to the computational practice of storing
program results whenever they are computed, and then looking up and retrieving
them later, instead of re-running programs that are guaranteed to return the
same answer.  This approach allows users to avoid unnecessary computation, and
makes it possible to create an archive of results that can be used for other
purposes later.

A memoised function can be seen as a middle ground between a database and a
computer program: a function can have a large (perhaps infinite) domain which
makes storing all its outputs in a database impractical; but it may also have a
long enough run-time that re-computing an output each time it is called is
undesirable.  The practice of memoisation -- that is, recalling results that
happen to have been pre-computed, and computing results that are unknown --
bridges the gap between these two extremes.

% Why is it good for maths?
This approach may be useful in a wide range of fields, but it is particularly
relevant to mathematics because of the type of problems mathematicians wish to
solve.  Mathematics is a rare example of a field with algorithms that typically
require a lot of processing, but have a small input and small output.

Many algorithms in the real world have a large input and a small output -- for
example, the problem of determining which streamed TV show to recommend to a
viewer takes as its input a huge amount of data about past choices consumers
have made, and returns a single title as its output.  The output of this
function can be stored easily, but so much time would be spent processing the
large input, and the same input would be repeated so infrequently, that this
type of memoisation would have little advantage.

Conversely, many algorithms have a small input and a large output -- for
example, rendering a plain text file on a screen takes a small amount of data as
an input, but creates an image which is likely to be large if stored.  In this
case, the input can be dealt with quickly, but memoising the output using a disk
is likely to be slower than recomputing it from scratch each time.

In mathematics, however, we frequently encounter computational problems with
both small input and small output.  For instance, we may wish to know whether a
given permutation group $G$ is simple; the input may be two or three generators
for $G$, which between them use only a few bytes on a computer, and the output
is a single boolean.  But despite the small sizes of these data sets, the
algorithm may take a long time to complete, and may use a large amount of memory
while in progress.  Thus, memoising results of this type will use very little
disk space, require very little processing, and may avoid a huge amount of work
when recalling previously computed results.

It is also worth noting that pure mathematical problems are likely to have
discrete inputs and outputs, and are therefore unlikely to suffer from numerical
noise.  This makes them even more suitable for memoisation, since we do not have
a high density of almost-equal inputs that nonetheless would have to be memoised
separately.

Furthermore, there is no reason that a memoisation cache should be limited to a
single user.  A memoisation framework could save results to a shared directory
for immediate use by other researchers connected by a network, or could even
use an online database for the cache, so long as sufficient precautions were
taken to avoid reading and writing clashes.

Besides the obvious advantages of recalling cached results instead of repeating
work, a memoisation framework that allows a human-readable cache could be used
by researchers to create a reproducible record of computed results that can be
included in papers, or shared with collaborators that have no knowledge of the
memoisation system.

The aim of \ODK Task 6.9 is to establish a persistent memoisation framework to
cache results in Python and \GAP across sessions, in a way that is easy to deploy
and configure, and allows for results to be shared reliably between different
researchers.  Deliverable 6.9 requires a ``shared persistent memoisation library
for Python/\Sage'', which is fulfilled by \pypersist{}, a new Python library
written by the current author.  The corresponding software for \GAP has also been
created, in the form of \GAP's \Memoisation{} package, which has
approximately the same features as \pypersist{}, and aims to be as compatible as
possible with it.

In this report, we will present a review of some existing tools for memoisation
(Section \ref{sec:existing}), describe the new Python library \pypersist{} and
the new \GAP package \Memoisation{} (Section \ref{sec:pypersist}), show how they
can be used together (Section \ref{sec:cross-system}), and give an overview of
the future direction of this project (Section \ref{sec:future}).

\section{Existing tools}
\label{sec:existing}

In this section we will consider some tools that already exist for memoisation,
in Python, \Sage and \GAP.  Some of the tools we discuss are only
intended for a subset of the complete memoisation process, while others are more
full-featured but lack some important desirable features.  A comparison of these
tools is shown in Table \ref{tab:comparison-of-tools}.

\begin{table}[h]
  \renewcommand{\arraystretch}{1.2}
  \begin{tabular}{|l|c|c|c|c|c|c|c|c|c|c|c|c|c|}\cline{2-14}
    \multicolumn{1}{c|}{ }
    & \rotatebox{270}{Updated in last year}
    & \rotatebox{270}{Python versions}
    & \rotatebox{270}{Function decorator}
    & \rotatebox{270}{Memory caching}
    & \rotatebox{270}{Disk caching}
    & \rotatebox{270}{Database caching}
    & \rotatebox{270}{Method support}
    & \rotatebox{270}{Compiled function support}
    & \rotatebox{270}{Custom keys}
    & \rotatebox{270}{Custom pickling}
    & \rotatebox{270}{Metadata}
    & \rotatebox{270}{\Sage support}
    & \rotatebox{270}{Compiled~~}
    \\ \hline
    \Sage \texttt{func\_persist} & \checkmark & \Sage & \checkmark & \checkmark & \checkmark &  &  &  &  &  & & \checkmark  &  \\ \hline
    \Sage \texttt{cached\_function} & \checkmark & \Sage & \checkmark & \checkmark &  &  &  & \checkmark & \checkmark &  &  & \checkmark & \checkmark \\ \hline
    \Sage \texttt{cached\_method} & \checkmark & \Sage & \checkmark & \checkmark &  &  & \checkmark & \checkmark & \checkmark &  &  & \checkmark & \checkmark \\ \hline
    \texttt{persist} &  & $2/3$ &  & \checkmark & \checkmark &  &  &  &  &  &  &  &  \\ \hline
    \texttt{PyMemoize} & \checkmark & $2/3$ & \checkmark & \checkmark & \checkmark &  & \checkmark &  &  &  &  &  &  \\ \hline
    \texttt{redis-simple-cache} &  & $2/3$ & \checkmark & \checkmark & \checkmark & \checkmark & \checkmark &  &  &  &  &  &  \\ \hline
    \texttt{dogpile.cache} & \checkmark & $2/3$ & \checkmark & \checkmark & \checkmark & \checkmark & \checkmark & & \checkmark  &  &  &  &  \\ \hline
    \pypersist{} & \checkmark & $2/3/\Sage$ & \checkmark & \checkmark & \checkmark & \checkmark & \checkmark & & \checkmark & \checkmark & \checkmark & \checkmark &  \\ \hline
    \hline
    \GAP operation/attribute & \checkmark & --- &  & \checkmark &  &  & \checkmark & \checkmark &  &  &  &  &  \\ \hline
    \Memoisation{} & \checkmark & --- &  & \checkmark & \checkmark & \checkmark & \checkmark & \checkmark & \checkmark & \checkmark & \checkmark &  &  \\ \hline
  \end{tabular}
  \vspace{7pt}
  \caption{Comparison of memoisation tools in Python/\Sage and in \GAP}
  \label{tab:comparison-of-tools}
\end{table}

\subsection{\GAP}
\GAP functions can be installed as methods for an object's
\emph{operations} and \emph{attributes}, and if they are, their results are
stored for the duration of the current \GAP session.  However, these results are
never saved to disk unless the session itself is stored, and the option is only
available for methods of an object.  Since there is a lack of any other
well-established tools for memoising functions, the \Memoisation{} package was
created.  It contains all the features described in Table
\ref{tab:comparison-of-tools}, except for those specific to Python, and for the
fact that it is written in pure \GAP and is therefore not compiled.  The
features are shown in more detail in the next chapter.

\subsection{\Sage}
There exists in \Sage a simple function decorator, \texttt{func\_persist}, which
can be applied to a function to save its output to disk.  This is a working
memoisation tool, and even this functionality has great practical use, but the
tool has very little configurability: the only option that can be given is to
specify the directory in which results are stored.  \Sage also contains two more
configurable tools for memoising functions and methods, respectively named
\texttt{cached\_function} and \texttt{cached\_method}.  These allow a function
or method to be memoised with a custom key-generating function, meaning that
arguments can be pre-processed in an intelligent way, perhaps to sort arguments,
or discard arguments that do not affect the return value of a function.
They are also implemented in Cython, meaning that they are compiled
rather than interpreted, and perform
faster than \texttt{func\_persist}. They also support compiled functions written in
Cython.
However, they only support memory caching -- that is, they only store results in
memory while the current program is running, and do not save results to disk for
use in a later session.

These three decorators are useful, but limited.  Not only do they lack certain
options such as database caching and custom output pickling, but they rely on
other parts of the \Sage system, and therefore cannot be used more broadly in
generic Python programs.  Python memoisation tools exist outside \Sage that can
in principle be applied to \Sage.

\subsection{Python}
Many memoisation tools are available on PyPI, each with a slightly different
philosophy and set of features.  Table \ref{tab:comparison-of-tools} summarises
four examples of these, indicating each one's features and status:
\texttt{persist} and \texttt{redis-simple-cache} have not been updated in
several years, while \texttt{PyMemoize} lacks even the custom key function
provided by the \Sage tools.

Perhaps the most promising of the existing tools is \texttt{dogpile.cache}, an
API for caching function and method outputs to a variety of backends; this is a
well-established, full-featured system, but it lacks certain features such as
custom result pickling -- particularly useful to mathematicians who may want to
store output using OpenMath, or in some human-readable form for sharing.

There would be an argument for modifying and improving \texttt{dogpile.cache} to
add these features, rather than creating a new tool.  However, given the wide
scope of the \texttt{dogpile} project, which is largely focused on thread
management, it was decided to proceed with a new tool that could be adapted more
easily to suit the specific requirements of the project, such as bespoke support
for \Sage-specific objects, and compatibility with the \GAP \Memoisation{} package
which was developed in parallel.

To fulfill these requirements, the \pypersist{} package was created.  It runs in
standard Python versions 2 and 3, as well as in \Sage, and it contains all the features mentioned in Table
\ref{tab:comparison-of-tools} except those features related to Cython.  More
information about \pypersist{} is given in Section \ref{sec:pypersist}.

\section{\pypersist{} and \Memoisation{}}
\label{sec:pypersist}

To address the requirements of this deliverable, two closely related pieces of
software were created: the Python package \pypersist{} and the \GAP package
\Memoisation{}.  They have approximately the same features, as shown in Table
\ref{tab:comparison-of-tools}, and they were designed to be compatible, as
explained in Section \ref{sec:cross-system}.

\subsection{Features}
\label{sec:features}

We now summarise some of the features of \pypersist{} and \Memoisation{}.  The
detailed workings of the packages are outside the scope of this report, but full
documentation is available at \url{https://pypersist.readthedocs.io} and
\url{https://gap-packages.github.io/Memoisation/}.  The features of the packages
are also demonstrated in interactive notebooks hosted on Binder and linked in
the respective readme files.
% \url{https://mybinder.org/v2/gh/mtorpey/pypersist/master?filepath=binder/demo.ipynb}.

\paragraph{Usage}
\pypersist{} provides a function decorator, \texttt{persist}, which is used to
memoise a function.  After this decorator is imported, a Python function can be
memoised by simply typing \texttt{@persist} above it.  In \GAP, which does not
have function decorators, the \Memoisation{} package instead provides
\texttt{MemoisedFunction}, which simply takes a function and returns a memoised
version of it.  This minimal setup produces memoisation with safe defaults, and
allows users to exploit these packages with no additional knowledge of their
features.  However, further customisation is possible by adding arguments to the
decorator in the form \texttt{@persist(...)} or by providing an options record
as the second argument to \texttt{MemoisedFunction}.  Examples are shown in
Figure \ref{fig:how-to-call}.

\begin{figure}[h]
  \centering
  \begin{minipage}[t]{.45\textwidth}
    {\tiny
    \begin{verbatim}
@persist(funcname="power_number",
         cache="file://my_results",
         pickle=str,
         unpickle=int)
def power(x, y):
    return x ** y
    \end{verbatim}
    }
  \end{minipage}
  \quad
  \begin{minipage}[t]{.45\textwidth}
    {\tiny
    \begin{verbatim}
power := MemoisedFunction({x, y} -> x ^ y,
      rec(funcname := "power_number",
          cache := "file://my_results",
          pickle := String,
          unpickle := Int));
    \end{verbatim}
    }
  \end{minipage}
  \caption{Memoising a function in \pypersist{} and \Memoisation{}}
  \label{fig:how-to-call}
\end{figure}

\paragraph{Configurable cache location}
By default, results in \pypersist{} and \Memoisation{} are saved respectively to
the \texttt{persist/<funcname>} and \texttt{memo/<funcname>} subdirectories of
the current working directory.  However, by providing a string as the
\texttt{cache} argument of \texttt{@persist} or \texttt{MemoisedFunction}, this
location can be changed (see Figure \ref{fig:how-to-call}).

\paragraph{Configurable function name}
In order to avoid confusion between two different functions with the same name,
a \texttt{funcname} argument can be supplied, which should
be a string that uniquely identifies the function.  This string could include a
version number, author name, or anything else required to ensure uniqueness.

\paragraph{MongoDB cache}
Support is provided for caching results in a MongoDB database, via the REST
interface provided by Python's \texttt{eve} package.  If a \texttt{cache}
argument is provided which starts with \texttt{mongodb://}, then it is
interpreted as the address of a MongoDB server which can send and receive
results for the function, and that server will be used instead of saving results
locally.  Such a server can be set up using the \texttt{mongodb\_server/run.py}
script distributed identically with \pypersist{} and \Memoisation{}.

\paragraph{Custom key function}
Each entry in a function's cache is stored using a unique key based on the
arguments supplied to the function; two sets of arguments should have the same
key only if they produce the same output.  By default, \pypersist{} uses a tuple of the
arguments, with their names, sorted alphabetically, and with any default arguments removed;
this allows for equivalent calls such as \texttt{foo(x=3,
  y=5)} and \texttt{foo(y=5, x=3)} to be treated equally.
Since \GAP does not support keyword arguments,
\Memoisation{} simply uses a list of the arguments.  However, in both systems, users can
specify a \texttt{key} argument, which should be a function
that takes the same arguments as the memoised function and produces a key based on them.  A developer
could thus choose to ignore arguments that control, for example, verbosity, or
other options that do not affect the function's return value.

\paragraph{Custom hashing}
When a result is stored in the cache, it is stored using a hash of its key.
This allows a record to be looked up quickly, and perhaps removes the necessity
of storing the keys at all.  By default, a key is hashed using its SHA-256
encoding, which makes it extremely unlikely that a hash collision will ever be
encountered.  However, a custom hash function can be specified using a
\texttt{hash} argument to the decorator, and in this case it will be used to
produce a hash instead of the default method.  If an injective hash function is
chosen, then its inverse can be specified using the \texttt{unhash} argument;
this allows hashes to be double-checked, and a list of previously-encountered
keys to be produced.

\paragraph{Storing keys}
By default, a key is not stored after it has been hashed, and correctness is
assumed due to the vanishingly small probability of an SHA-256 hash collision.
However, for complete correctness, the packages support an argument
\texttt{storekey} which, if set to true, stores keys along with results, and
checks them when looking up results, raising an error if a hash collision is
detected.  This also allows a user to iterate over the stored keys of a cache,
as mentioned below.

\paragraph{Custom pickling}
In order to store an object in a cache, it needs to be preserved on disk or in a
database.  For this purpose, we require sometimes complex objects to be
\textit{pickled} and \textit{unpickled} -- that is, converted to a string for
storing, and converted back when retrieved later.  The default methods use
Python's \texttt{pickle} module (or, if appropriate, \Sage's own pickling
functions) or \GAP's \texttt{IO\_Pickle}, and a base 64 encoding to produce strings that
can be written without invisible or difficult-to-display characters.  However,
like much of these packages' functionality, custom \texttt{pickle} and
\texttt{unpickle} functions can be specified that do this a different way.  This
can allow results to be stored in a human-readable way which can then be used
outside Python and \GAP, or even to be stored using a format such as OpenMath for use in
a completely different computer system.

\paragraph{Manual cache interaction}
It may be necessary at some time to modify a memoisation cache manually --
perhaps to remove an inaccurate result, or to add a result that was found before
the cache was created.  This can be done directly in both packages.
In \pypersist{}, a memoised
function \texttt{foo} has an attribute \texttt{foo.cache} which can be used much
like a dictionary, reading, setting and deleting entries as desired.
Furthermore, if \texttt{storekey} is set to true, or if an \texttt{unhash}
method is provided, \texttt{foo.cache} is an iterable object with attributes
\texttt{keys}, \texttt{values} and \texttt{items} that can be used in loops.
Similarly in \Memoisation{}, a memoised function \texttt{foo} has a component
\texttt{foo!.cache} which satisfies \texttt{IsLookupDictionary}; it can
therefore be used with the \GAP functions \texttt{AddDictionary},
\texttt{KnowsDictionary} and \texttt{LookupDictionary}, if a user wishes to
manipulate it directly.

\subsection{Software engineering}
Both \pypersist{} and \Memoisation{} were written using modern open-source software technologies
and practices, with an emphasis on clarity of understanding and ease of use.
The report on \ODK D1.5 contains, in Section 4.3.2.1, a checklist of
software engineering best practices.  The packages fulfill all the requirements
in that list, as summarised in Tables \ref{tab:pypersist-se-check} and \ref{tab:memo-se-check}.

\begin{table}[h]
  \renewcommand{\arraystretch}{1.2}
  \begin{tabular}{|p{5.1cm}|c|p{9.5cm}|}\hline
    Version control & \checkmark & Git \\ \hline
    Tests & \checkmark & \multirow{2}{*}{\emph{pytest} suite with 98\% code coverage} \\ \cline{1-2}
    Automated tests & \checkmark & \\ \hline
    Continuous integration & \checkmark & Travis runs test suite for every commit \\ \hline
    Automatic building of releases & \checkmark & PyPI release script \\ \hline
  \end{tabular}
  \vspace{0pt}
  \caption{Software engineering good practice checklist for \pypersist{}}
  \label{tab:pypersist-se-check}
\end{table}

\begin{table}[h]
  \renewcommand{\arraystretch}{1.2}
  \begin{tabular}{|p{5.1cm}|c|p{9.5cm}|}\hline
    Version control & \checkmark & Git \\ \hline
    Tests & \checkmark & \multirow{2}{*}{\GAP test suite with 98\% code coverage} \\ \cline{1-2}
    Automated tests & \checkmark & \\ \hline
    Continuous integration & \checkmark & Travis runs test suite for every commit \\ \hline
    Automatic building of releases & \checkmark & \texttt{ReleaseTools} for \GAP (see D5.15) \\ \hline
  \end{tabular}
  \vspace{0pt}
  \caption{Software engineering good practice checklist for \Memoisation{}}
  \label{tab:memo-se-check}
\end{table}

\subsection{Dissemination}
The report on \ODK D1.5 also contains, in Section 4.3.2.2, a checklist
of best practices for dissemination.  The packages also fulfill all of these
requirements, as summarised in Tables \ref{tab:pypersist-diss-check} and \ref{tab:memo-diss-check}.

\begin{table}[h]
  \renewcommand{\arraystretch}{1.2}
  \begin{tabular}{|p{5.1cm}|c|p{9.5cm}|}\hline
    Host code publicly & \checkmark & \url{https://github.com/mtorpey/pypersist} \\ \hline
    Reference Manual (APIs) & \checkmark & \url{https://pypersist.readthedocs.io} \\ \hline
    Tutorial (for beginning users) & \checkmark & \multirow{3}{9.5cm}{Examples section in readme, and Binder demo with extensive annotations: \texttt{binder/demo.ipynb} \\ (linked from readme and manual)} \\ \cline{1-2}
    Examples & \checkmark & \\ \cline{1-2}
    Live interactive online demos & \checkmark & \\ \hline
    Support mechanisms & \checkmark & Github issues \\ \hline
    How to cite the output? & \checkmark & Explained in readme and manual \\ \hline
    Installation mechanism & \checkmark & Installation via \emph{pip}, explained in readme and manual \\ \hline
    High level description accessible to non-experts & \checkmark & In readme, manual and Binder \\ \hline
    URLs/Blog/etc to and from \ODK project & \checkmark & Links to \ODK in readme, in manual, and on PyPI distribution page \\ \hline
    Grant acknowledgements & \checkmark & Acknowledged in readme and manual \\ \hline
    Open Source license & \checkmark & GPL v2 or later \\ \hline
    Workshop & \checkmark & \multirow{2}{9.6cm}{Demo and discussion at \emph{Free Computational Mathematics} (CIRM, Luminy, Month 42)} \\ \cline{1-2}
    Engaging users & \checkmark & \\ \hline
  \end{tabular}
  \vspace{0pt}
  \caption{Dissemination good practice checklist for \pypersist{}}
  \label{tab:pypersist-diss-check}
\end{table}

\begin{table}[h]
  \renewcommand{\arraystretch}{1.2}
  \begin{tabular}{|p{5.1cm}|c|p{9.5cm}|}\hline
    Host code publicly & \checkmark & \href{https://github.com/gap-packages/Memoisation}{\texttt{github.com/gap-packages/Memoisation}} \\ \hline
    Reference Manual (APIs) & \checkmark & \href{https://gap-packages.github.io/Memoisation/doc/chap0.html}{\texttt{gap-packages.github.io/Memoisation}} \\ \hline
    Tutorial (for beginning users) & \checkmark & \multirow{3}{9.5cm}{Examples section in readme, and Binder demo at \texttt{binder/demo.ipynb} (linked from readme)} \\ \cline{1-2}
    Examples & \checkmark & \\ \cline{1-2}
    Live interactive online demos & \checkmark & \\ \hline
    Support mechanisms & \checkmark & Github issues \\ \hline
    How to cite the output? & \checkmark & Explained in readme and on website \\ \hline
    Installation mechanism & \checkmark & Installation via \texttt{PackageManager}, explained in readme and manual \\ \hline
    High level description accessible to non-experts & \checkmark & In readme and manual \\ \hline
    URLs/Blog/etc to and from \ODK project & \checkmark & \ODK link and logo in readme \\ \hline
    Grant acknowledgements & \checkmark & Acknowledged in readme \\ \hline
    Open Source license & \checkmark & BSD 3-clause licence \\ \hline
    Workshop & \checkmark & \multirow{2}{9.6cm}{Demo and discussion at \textit{Workshop on Mathematical Data} (Cernay, Month 48)} \\ \cline{1-2}
    Engaging users & \checkmark & \\ \hline
  \end{tabular}
  \vspace{0pt}
  \caption{Dissemination good practice checklist for \Memoisation{}}
  \label{tab:memo-diss-check}
\end{table}

Development of both packages was open from the start, available publicly on
Github, including their full commit histories.  The issue trackers on their Github pages
have been used to track upcoming new features, as well as bugs and other
problems.  All documentation, tools, and setup are included directly in the Git
repository, so that everything is kept in one place and can be tracked as a
single unit.

The packages' documentation is mostly generated directly from the source files,
using Sphinx and GAPDoc/Autodoc respectively.
This means that code is documented where it is written,
making the code easier to understand and avoiding the need to duplicate
documentation with extra comments.  Numpy-style docstrings are used throughout \pypersist{}
-- this widely used standard was chosen for its minimal syntax, which allows the
strings to be easily read and understood in plaintext without any processing or
compilation.  A readme is included in each project, written in Markdown, with a simple
two-sentence introduction, a guide to installation, and a small number of examples
showing how to get started.  The manual for each package is hosted online,
\pypersist{} on \textit{ReadTheDocs} and \Memoisation{} on its own website, both
of which are linked from the appropriate Github page.  In this way, the
documentation is accessible, clear, and
maintainable.

The minimal examples in the readme are supplemented by the Jupyter notebooks
included with the projects.
These notebook can be loaded locally by anyone with a Jupyter installation, but
for better accessibility they are also hosted on Binder, with the address linked at
the bottom of each readme, as well as in a badge at the top.  The \pypersist{} demo loads the
most recent version of \pypersist{}, and therefore also acts as an additional
test suite, since it requires the package to be in a working state in order to
demonstrate its features.  The outputs of the executable cells are not saved
into the notebooks -- instead it is left to readers to execute the cells
themselves, producing an interactive experience which may encourage the reader to
experiment, and may lead to a better understanding of how the package works.
The \pypersist{} notebook was demonstrated at the \emph{Free Computational Mathematics}
conference held in CIRM, Luminy, France, in Month 42, prompting a
discussion and feedback from users.  A demo of both packages working together
was shown at the \textit{Workshop on Mathematical Data} held in Cernay
in Month 48 (it approximately followed Section \ref{sec:cross-system}) also
prompting discussion.

In order for the packages to be used, they must be easy to install.  The
\pypersist{} package is hosted on PyPI at
\url{https://pypi.org/project/pypersist}, making installation as simple as
typing \texttt{pip install pypersist} for any user connected to the internet.
The \Memoisation{} package can now be installed using \GAP's own package manager,
another new piece of \GAP infrastructure from \ODK.  Each installation method is
explained in the appropriate project's readme, and the procedure for
\pypersist{} is also on the first page of its manual.

% 3rd checklist: perhaps doesn't add anything to include?
% \subsection{Pathways to impact}
% \begin{itemize}
% \item[{$\square$}] Does the software address the needs of the users?
% \item[{$\square$}] Workshops to gather feedback
% \end{itemize}

\section{Interoperability}
\label{sec:cross-system}

By default, \pypersist{} caches are not compatible with \Memoisation{} caches.
\GAP and Python have different data types, and the default \texttt{pickle} and
\texttt{hash} functions are different between the two packages.  For this
reason, the default cache locations are separate when caching to disk, and
different namespaces are used when writing to a database.

However, it is possible to customise a cache in a variety of ways, as described
in Section \ref{sec:features}, and it is certainly possible to specify options
that allow a \pypersist{} cache and a \Memoisation{} cache to share results with
each other.  The following example was presented as a demo at the
\textit{Workshop on Mathematical Data} mentioned above.

\subsection{Example}
\label{sec:cross-system-example}

Consider the following fragment of Python code.  It defines a function
\texttt{prime\_factors} that performs deliberately badly, including a 2-second
sleep, but should correctly calculate the prime factors of an integer.

{\tiny
\begin{verbatim}
from time import sleep
from pypersist import persist

@persist(
    cache="file://shared",
    key=lambda n: n,
    hash=str,
    pickle=lambda L: "\n".join(map(str, L)),
    unpickle=lambda s: [int(n) for n in s.split("\n") if n!=""],
)
def prime_factors(n):
    factors = []
    d = 2
    while n > 1:
        while n % d != 0:
            d += 1
        factors.append(d)
        n /= d
    sleep(2)
    return factors
\end{verbatim}
}

The function itself has been memoised by being decorated with \texttt{@persist},
and several custom options have been specified:
\begin{itemize}
\item \texttt{cache} defines the location of the cache as being inside
  \texttt{shared/} in the current directory;
\item \texttt{key} defines the single input \texttt{n} as a unique key for each
  memoised result;
\item \texttt{hash} is the function that simply writes an integer out as a
  string in base 10 (this, with ``.out'', will be the filename);
\item \texttt{pickle} takes the list of prime factors, and prints each one to a
  separate line in a text file, also in base 10;
\item \texttt{unpickle} turns this file back into the list of integers.
\end{itemize}
This will result in quite a human-readable cache: results will be stored inside
the present directory, in \texttt{shared/prime\_factors/}, and each result will
be saved in a single file with \texttt{<n>.out} as the filename, and one prime
factor printed on each line.  This cache could be used as an educational tool,
and bears no traces of having been created by \pypersist{} at all.

\GAP has a much better method for computing prime factors: the library function
\texttt{Factors}.  We memoise this as follows.

{\tiny
\begin{verbatim}
LoadPackage("Memoisation");

prime_factors := MemoisedFunction(Factors, rec(
    funcname := "prime_factors",
    cache := "file://shared",
    key := IdFunc,
    hash := String,
    pickle := L -> JoinStringsWithSeparator(List(L, String), "\n"),
    unpickle := str -> List(SplitString(str, "\n"), Int)
));
\end{verbatim}
}

Note that all the same memoisation options have been specified: \texttt{cache},
\texttt{key}, \texttt{hash}, \texttt{pickle} and \texttt{unpickle} have been
translated into \GAP code and included just as in the \pypersist{} example.
However, note that an additional argument \texttt{funcname} has been specified;
this is so that we use the directory \texttt{shared/prime\_factors/} instead of
the default \texttt{shared/Factors/}.

These two functions will now load and store each other's results.  If we compute
a new example in Python first, it will take a whole 2 seconds to complete, but
if we calculate it first in \GAP and then in Python, it will appear to complete
instantly, as the overhead of reading a list of integers from a file is
negligible.  This is not only an example of how flexible a cache can be in these
two packages, but also an example of how different mathematical systems can
communicate with each other and share results.

\section{Future direction}
\label{sec:future}

Although \pypersist{} and \Memoisation{} now have many features and are
well-tested, they are still in
development.  It is hoped that as more developers begin to use them, they will
provide feedback that will inform development further, whether by revealing
bugs, or by requesting new features.  There are already several features that
are planned, and work will continue on this project as time goes on.  In this
section we describe some aims for the future direction of persistent memoisation
in \GAP, \Sage and Python.

\subsection{Features}

\paragraph{CouchDB}
Support currently exists in \pypersist{} and \Memoisation{} for a MongoDB database backend.  This
works well, but there are certain features offered by the CouchDB system that
might also be useful.  CouchDB emphasises ease of setup, and allows information
to be added and retrieved from the database via HTTP, with a human-readable
web-based front-end that would allow human-readable results to be shared
instantly on computation via a website.  It is likely that implementing this
would require little deviation from the existing code for MongoDB, and could
result in big improvements in ease of use, ease of setup, and interoperability.

\paragraph{Better metadata}
\pypersist{} and \Memoisation{} have the capacity to store metadata, which may be useful for storing
information about a given computation.  However, no
metadata is stored by default, and metadata is not validated in any way when a
result is loaded from the cache.  Some decision should be made on what metadata
should always be stored, and functionality should be added that allows this
metadata to be used where relevant, for example by allowing a cache to be
filtered by date or user.
Some useful fields could be where and when a result was originally computed,
comments from the user or the author of the function, and information about the
system that produced the result.

\paragraph{Provenance tracking}
One use of metadata would be to record where a given result came from, allowing
researchers to evaluate the trust they are willing to place in it.  A limited
form of this can be achieved by sharing results using a version control system
such as Git, in which the timestamp and author of a commit give some useful
information.  But a more full-featured provenance-tracking system could be added
in the future, recording the exact time and place that a computation was
performed, as well as the user that called it, and version numbers for both
\pypersist{}/\Memoisation{} and the function being memoised.

\paragraph{Permissions}
In a multi-user memoisation cache, it would be useful to have explicit support
for permissions handling.  Most of the functionality for this could be delegated
to the backend -- for example, any database system has support for user
permissions, and Git repositories can be set up to accept certain requests from
only certain users -- but it would be useful if \pypersist{} and \Memoisation{} handled this
properly instead of simply raising an error when an interaction is unsuccessful.

\paragraph{Verbosity}
Results in \pypersist{} are stored and retrieved silently, and users are not
made aware of its existence in any way except when errors are raised (for
example due to a hash collision or HTTP error).  The \Memoisation{} package, on the other hand,
displays messages to the screen showing its progress, so that
interested users have a record of exactly what is being written and read at what
time and in what location.  Different levels of verbosity are also available, showing different
levels of information.  This verbosity should be included in \pypersist{} as well.

\paragraph{\Sage integration}
As shown in Table \ref{tab:comparison-of-tools}, there exist three different
tools already in \Sage for memoising data.  Now that \pypersist{} has overtaken
\texttt{func\_persist} in terms of features, it is possible that it could replace this tool
entirely; with some additional work (support for compiled functions,
optimization), it could replace \texttt{cached\_function} and
\texttt{cached\_method} as well.
With careful integration and testing to ensure that there were no
regressions, \pypersist{} could be imported and used instead of those tools,
possibly with the addition of some \Sage-specific options that would allow it to
be used as effectively as possible.  However, \pypersist{} should remain a
generic Python package that can be used outside as well as inside \Sage.

\paragraph{Search by property}
The core behaviour of a memoisation system is to look up a single input and
return a single output if one exists.  However, once a cache with a number of
results has been built up, a user may wish to search through it to find all
input--output pairs that satisfy a particular property.  For instance, if a
boolean function called \texttt{is\_prime} is memoised, and has been used to
test the primality of several integers, a researcher who desires prime numbers
might search through the cache for all inputs that returned \texttt{true}.  This
can be achieved using the current system, but some kind of indexing might be
helpful, to allow answers to be returned more quickly.

\end{document}

%%% Local Variables:
%%% mode: latex
%%% TeX-master: t
%%% End:
