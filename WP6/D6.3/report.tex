\documentclass{deliverablereport}

\usepackage[style=alphabetic,backend=biber]{biblatex}
\addbibresource{../../lib/kbibs/kwarcpubs.bib}
\addbibresource{../../lib/kbibs/extpubs.bib}
\addbibresource{../../lib/kbibs/kwarccrossrefs.bib}
\addbibresource{../../lib/kbibs/extcrossrefs.bib}
\addbibresource{../../lib/deliverables.bib}
% temporary fix due to http://tex.stackexchange.com/questions/311426/bibliography-error-use-of-blxbblverbaddi-doesnt-match-its-definition-ve
\makeatletter\def\blx@maxline{77}\makeatother

% \usepackage{local}

\deliverable{dksbases}{dkstheories}
\deliverydate{10/09/2016}
\duedate{30/11/2016 (Month 15)}

\author{Michael Kohlhase, Florian Rabe, Tom Wiesing, Paul-Olivier Dehaye, Dennis M\"uller}

\begin{document}
\begin{abstract}
  This deliverable is subsumed by deliverable \delivref{dksbases}{design}, which already
  covers all the material to be reported on three months early.
\end{abstract}

\maketitle
\githubissuedescription\par\bigskip\hrule\par\bigskip

The \pn proposal had envisioned \WPtref{dksbases} as a foundational enterprise that would
develop a knowledge-based architecture over the course of the project and would allow to
re-engineer ``ad-hoc'' interfaces between systems (e.g. from
\taskref{component-architecture}{interface-systems}) on a more ``semantic'' basis -- the
knowledge aspect (K). Consequently, the proposal envisioned concentrating the data (D)
aspect on the mathematical knowledge bases (specifically LMFDB, OEIS, and FindStat) and
proposed a host of foundational investigations of mathematical for the software (S) aspect
with applications e.g. in the verification of algorithms.

Already the kickoff meeting in Paris in September 2015 revealed that the D/K/S aspects are
much more tightly coupled in systems than anticipated. This was confirmed by the DKS
survey conducted subsequently (see Section 2 of \delivref{dksbases}{design}). In
particular, the participants of \WPref{dksbases} identified the interoperability of \pn
systems to be one of the most critical steps in creating a VRE toolkit. Thus we
prioritized tasks \taskref{dksbases}{data-assessment}, \taskref{dksbases}{data-triform},
\taskref{dksbases}{data-design} and organized a series of workshops and code-maratons to
develop a semantic foundation for system interoperability and simultaneously test it in
implementations.

As a consequence, we have completed -- in parallel the initial design of D/K/S-bases (for
deliverable \delivref{dksbases}{design}) -- the initial implementation of a \DKS base
format based on OMDoc/MMT together and the implementation of a \DKS base system itself
based on the MMT system (both for \delivref{dksbases}{dkstheories}), all activities
fuelling each other.  \delivref{dksbases}{dkstheories} was thus completed three months
ahead of schedule.  Note that the RNC schema envisioned in the title proved un-necessary
since with the refined Math-in-the-Middle (MitM) design the normal OMDoc/MMT schema is
sufficient.

Due to the resulting tight coupling between \delivref{dksbases}{design} and
\delivref{dksbases}{dkstheories}, and for the convenience of the reader, we have decided
to report on both deliverables together; see the report for deliverable
\delivref{dksbases}{design}. When the design has further matured through work in the \pn
project, we plan to describe the MitM paradigm of integration of mathematical software
systems into a VRE toolkit in a journal paper. We envision submission around month 27. 
\newpage
\end{document}

%%% Local Variables:
%%% ispell-local-dictionary: "en"
%%% mode: latex
%%% TeX-master: t
%%% End:


%  LocalWords:  delivref dksbases maketitle githubissuedescription bigskip hrule bigskip
%  LocalWords:  pn WPtref taskref WPref prioritized organized dkstheories newpage
