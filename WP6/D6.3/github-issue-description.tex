\section*{\texorpdfstring{Deliverable description, as taken from Github
issue
\href{https://github.com/OpenDreamKit/OpenDreamKit/issues/137}{\#137} on
2016-09-19}{Deliverable description, as taken from Github issue \#137 on 2016-09-19}}\label{deliverable-description-as-taken-from-github-issue-137-on-2016-09-19}

\begin{itemize}
\tightlist
\item
  \textbf{WP6:}
  \href{https://github.com/OpenDreamKit/OpenDreamKit/tree/master/WP6}{Data/Knowledge/Software-Bases}
\item
  \textbf{Lead Institution:} Jacobs University Bremen
\item
  \textbf{Due:} 2016-11-30 (month 15)
\item
  \textbf{Delivered:} 2016-09-??, together with D6.2
  (\href{https://github.com/OpenDreamKit/OpenDreamKit/issues/136}{\#136})
\item
  \textbf{Nature:} Report
\item
  \textbf{Task:} T6.2
  (\href{https://github.com/OpenDreamKit/OpenDreamKit/issues/124}{\#124})
\item
  \textbf{Proposal:}
  \href{https://github.com/OpenDreamKit/OpenDreamKit/raw/master/Proposal/proposal-www.pdf}{55}
\item
  \textbf{Final report:} bundled with the
  \href{https://github.com/OpenDreamKit/OpenDreamKit/raw/master/WP6/D6.2/report-final.pdf}{report}
  for D6.2
  (\href{https://github.com/OpenDreamKit/OpenDreamKit/issues/136}{\#136})
\end{itemize}

The OpenDreamKit proposal had envisioned WP6:
\emph{Data/Knowledge/Software bases} as a foundational enterprise that
would develop a knowledge-based architecture over the course of the
project and would allow to re-engineer \emph{ad-hoc} interfaces between
systems (e.g.~from T3.2
(\href{https://github.com/OpenDreamKit/OpenDreamKit/issues/51}{\#51}))
on a more \emph{semantic} basis -- the knowledge aspect (K).
Consequently, the proposal envisioned concentrating the data (D) aspect
on the mathematical knowledge bases (specifically LMFDB, OEIS, and
FindStat) and proposed a host of foundational investigations of
mathematical for the software (S) aspect with applications e.g.~in the
verification of algorithms.

Already the kickoff meeting in Paris in September 2015 revealed that the
D/K/S aspects are much more tightly coupled in systems than anticipated.
This was confirmed by the DKS survey conducted subsequently (see Section
2 of D6.2
(\href{https://github.com/OpenDreamKit/OpenDreamKit/issues/136}{\#136})).
In particular, the participants of WP6 identified the interoperability
of OpenDreamKit systems to be one of the most critical steps in creating
a VRE toolkit. Thus we prioritized tasks T6.1
(\href{https://github.com/OpenDreamKit/OpenDreamKit/issues/123}{\#123}),
T6.2
(\href{https://github.com/OpenDreamKit/OpenDreamKit/issues/124}{\#124}),
T6.3
(\href{https://github.com/OpenDreamKit/OpenDreamKit/issues/125}{\#125})
and organized a series of workshops and code-maratons to develop a
semantic foundation for system interoperability and simultaneously test
it in implementations.

As a consequence, we have completed -- in parallel the initial design of
D/K/S-bases (for deliverable D6.2
(\href{https://github.com/OpenDreamKit/OpenDreamKit/issues/136}{\#136}))
-- the initial implementation of a DKS base format based on OMDoc/MMT
together and the implementation of a DKS base system itself based on the
MMT system (both for D6.3
(\href{https://github.com/OpenDreamKit/OpenDreamKit/issues/137}{\#137})),
all activities fuelling each other. D6.3
(\href{https://github.com/OpenDreamKit/OpenDreamKit/issues/137}{\#137})
was thus completed about three months ahead of schedule. Note that the
RNC schema envisioned in the title proved un-necessary since with the
refined Math-in-the-Middle (MitM) design the normal OMDoc/MMT schema is
sufficient.

Due to the resulting tight coupling between D6.2
(\href{https://github.com/OpenDreamKit/OpenDreamKit/issues/136}{\#136})
and D6.3
(\href{https://github.com/OpenDreamKit/OpenDreamKit/issues/137}{\#137}),
and for the convenience of the reader, we have decided to report on both
deliverables together; see the report for deliverable D6.2
(\href{https://github.com/OpenDreamKit/OpenDreamKit/issues/136}{\#136}).
When the design has further matured through work in the OpenDreamKit
project, we plan to describe the MitM paradigm of integration of
mathematical software systems into a VRE toolkit in a journal paper. We
envision submission around month 27.
