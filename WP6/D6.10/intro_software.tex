\subsection{Mixed Computational and Narrative Documents}\label{subsec:software}
Finally, for computational mathematical documents (the S aspect of D/K/S structures from the \pn proposal or the \textbf{narration} and \textbf{computation} aspects of the finer tetrapod model from Figure~\ref{fig:tetrapod}).
Here we have developed a formula harvester for Jupyter notebooks and a formula search engine that builds on them (as envisioned in task \taskref{dksbases}{mws}).

To make this possible, we had to invest a heavy dose of software engineering into the MathWebSearch system: Even though the system has successfully been used as a formula search engine in zbMATH publication information system (see \url{https://zbmath.org/formulae/}), the deployment of the system required a lot of domain-specific development and workflow integration.
To this end we have developed Go bindings for the MathWebSearch daemon, documented the interfaces, and provide a web application wrapper.
With this, specific applications only need a domain-specific harvester and minimal customization of a generic front-end. 


%%% Local Variables:
%%% mode: latex
%%% mode: visual-line
%%% fill-column: 5000
%%% TeX-master: "report"
%%% End:

%  LocalWords:  textbf textbf Jupyter taskref dksbases mws zbMATH
