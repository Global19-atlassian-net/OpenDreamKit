\subsection{Formal Knowledge Bases}\label{sec:isabelle}
For the Knowledge (K) aspect (see \textbf{Inference} in Figure~\ref{fig:tetrapod}) we have developed an exporter from library the Isabelle Theorem prover (Archive of Formal Proof) to an extensive RDF triple store, which can be queried by standard SPARQL queries for semantic search.
  
For many decades, the development of a universal database of all mathematical knowledge, as envisioned, e.g., in the QED manifesto \cite{qed}, has been a major driving force of computer mathematics.
Today a variety of such libraries are available.
These are most prominently developed in proof assistants such as Coq \cite{coq} or Isabelle \cite{isabelle} and are treasure troves of detailed mathematical knowledge.
However, despite the enormous potential for many applications of and in virtual research environments, this treasure is usually locked into system- and logic-specific representations that can only be understood by the respective theorem prover system.
For example, this precludes applications such as finding related object in knowledge bases from within computation-oriented systems as used in OpenDreamKit.

Therefore, we have developed interface standards that allow maintainers of formal libraries to make their content available to outside systems.
In this deliverable, we report on complementing our existing OMDoc/MMT standard for representing entire knowledge bases with a new Upper Library Ontology (ULO).
ULO is a standard ontology for exchanging high-level information about mathematical libraries that systematically abstracts all symbolic knowledge away and only retain what can be easily represented relationally.
That allows for semantic web-style representations, for which simple and standardized formalisms such as OWL2 \cite{w3c:owl2-xml}, RDF \cite{rdf}, and SPARQL~\cite{w3c:SPARQL-Rec:13} as well as highly scalable tools are readily available.
While it is well-known that ontology language--based relational formalisms are inappropriate for symbolic knowledge like formulas, algorithms, and proofs, it is this high-level information that is often critical important for integration into virtual research environments, e.g., to realize benefits like search.

We report on this in Section~\ref{sec:knowledge}.
The bulk of this section is taken by a report on the new task (\tasktref{dksbases}{isabelle} that we have added to \pn in the last amendment of the grant agreement.
In this task, we export the large Isabelle knowledge bases as both OMDoc/MMT and ULO format, resulting in datasets in the double-digit GB range.
We show the utility of the generated ULO data by setting up a relational query engine that provides easy access to certain library information that was previously hard or impossible to determine.

\paragraph{Relational Datasets}
Many mathematical datasets take the form of large SQL/CSV--style datasets that enumerate all or a selection of interesting objects satisfying certain properties, e.g., all finite groups up to a certain size.
These can be produced in bulk or grow from individual user submissions.

They are generally produced, published, and maintained with virtually no systematic attention to the FAIR principles~\cite{FAIR,WilDumAal:FAIR16} for making data findable, accessible, interoperable, and reusable.
In fact, often the sharing of data is an afterthought --- see~\cite{Bercic:cmo:wiki} for an overview of mathematical datasets and their ``FAIR-readiness''.

Moreover, the inherent complexity of mathematical data makes it very difficult to share in practice: even freely accessible datasets are often very hard or impossible to reuse, let alone make machine-interoperable because there is no systematic way of specifying the relation between the raw data and its mathematical meaning. 
Therefore, unfortunately FAIR mathematical datasets essentially do not exist today.


%%% Local Variables:
%%% mode: latex
%%% mode: visual-line
%%% fill-column: 5000
%%% TeX-master: "report"
%%% End:

%  LocalWords:  textbf standardized rdf tasktref dksbases
