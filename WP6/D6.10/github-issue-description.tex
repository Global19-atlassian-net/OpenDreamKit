\hypertarget{deliverable-description-as-taken-from-github-issue-134-on-2019-08-31}{%
\section*{\texorpdfstring{Deliverable description, as taken from Github
issue
\href{https://github.com/OpenDreamKit/OpenDreamKit/issues/134}{\#134} on
2019-08-31}{Deliverable description, as taken from Github issue \#134 on 2019-08-31}}\label{deliverable-description-as-taken-from-github-issue-134-on-2019-08-31}}

\begin{itemize}
\tightlist
\item
  \textbf{WP6:}
  \href{https://github.com/OpenDreamKit/OpenDreamKit/tree/master/WP6}{Data/Knowledge/Software-Bases}
\item
  \textbf{Lead Institution:} FAU
\item
  \textbf{Due:} 2019-08-31 (month 48; postponed from M42 in the last
  amendment)
\item
  \textbf{Nature:} Other
\item
  \textbf{Task:} T6.10
  (\href{https://github.com/OpenDreamKit/OpenDreamKit/issues/132}{\#132}),
  T6.11
  (\href{https://github.com/OpenDreamKit/OpenDreamKit/issues/286}{\#286})
\item
  \textbf{Proposal:}
  \href{https://github.com/OpenDreamKit/OpenDreamKit/raw/master/Proposal/proposal-www.pdf}{p.54}
\item
  \textbf{\href{https://github.com/OpenDreamKit/OpenDreamKit/raw/master/WP6/D6.10/report-final.pdf}{Final
  report}}
  (\href{https://github.com/OpenDreamKit/OpenDreamKit/raw/master/WP6/D6.10/}{sources})
\end{itemize}

This report summarizes the achievements in Work Package 6 over the last
year of the OpenDreamKit project. Namely it covers results of
\textbf{T6.10}: Math Search Engine and \textbf{T6.11}: Isabelle Case
Study and tasks \textbf{T6.6}-\textbf{6.8} (case studies in mathematical
data sets).

In the last year, significant progress has been made in four areas:

\begin{itemize}
\tightlist
\item
  the understanding of D/K/S-bases: we have codified the experiences in
  the OpenDreamKit project into a tetrapodal view on "doing
  mathematics", which extends the Data/Knowledge/Software model of the
  proposal by "narration" and "organization" aspects.
\item
  the Knowledge (K) aspect, where we have developed an exporter from
  library the Isabelle Theorem prover (Archive of Formal Proof) to an
  extensive RDF triple store, which can be queried by standard SPARQL
  queries for semantic search.
\item
  the data (D) aspect, where we have

  \begin{itemize}
  \tightlist
  \item
    developed both an innovative model of (deep) FAIR in mathematics
    (and have integrated it with the MitM paradigm developed in
    OpenDreamKit),
  \item
    have implemented in a prototypical system (data.mathhub), and have
  \item
    evaluated it on the mathematical community outside the core
    OpenDreamKit community.
  \end{itemize}
\item
  computational mathematical documents (the S aspect of D/K/S or the
  "narration" and "computation" aspects of the finer tetrapod model).
  Here we have developed a formula harvester for jupyter notebooks and a
  formula search engine that builds on them (as envisioned in task T10
  of WP6).
\end{itemize}

Note: The title of this deliverable was originally entitled
\emph{Full-text search (Formulae + Keywords) in OpenDreamKit}. However,
in the last grant agreement amendment, the scope was broadened to a
report on the remaining \textbf{WP6} activities and achievements -\/-
also to account for the new task \textbf{T6.11}. The title was changed
to better reflect the actual content which, beyond Full-text search,
targets the integration of mathematical data in Virtual Research
Environments.
