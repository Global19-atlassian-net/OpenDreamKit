\paragraph{Mathematical Datasets}
% Innovations based on mathematical knowledge and algorithms yield many improvements in science, engineering, economy, ecology, health care, security, and society overall. 
% For example, our global positioning system (GPS) needs the mathematics of relativistic physics, our mobile phones use frequencies allocated through combinatorial optimization, the combinatorics of our genome yields clues to curing rare diseases, the privacy of our communications depends on cryptographic protocols steeped in number theory, and our national security is relying on the mathematical analysis of increasingly complex networks. 
% Fundamental mathematical research and its direct application in practical situations enable many engineering, science, and business innovations that enrich society and mankind.\ednote{FR: throw out this paragraph? Or replace with one sentence?}

Modern mathematical research increasingly depends on collaborative tools, computational environments, and online databases, and these are changing the way mathematical research is conducted and how it is turned into applications.
For example, engineers now use mathematical tools to build and simulate physical models based on systems of differential equations with millions of variables, combining building blocks and algorithms taken from libraries shared all over the internet.

Traditionally, mathematics has not paid particular attention to the creation and sharing of data --- the careful computation and publication of logarithm tables is a typical example of the extent and method.
This has changed with the advent of computer-supported mathematics, and the practice of modern mathematics is increasingly data-driven.
Today it is routine to use mathematical datasets in the Gigabyte range, including both human-curated and machine-produced data.
Examples include the L-Functions and Modular Forms Database (LMFDB; $\sim 1$ TB data in number theory)~\cite{Cremona:LMFDB16,lmfdb:on} and the GAP Small Groups Library~\cite{GapSmallGroups:on} with $\sim 450$ million finite groups.  
In a few, but increasingly many areas, mathematics has even acquired traits of experimental sciences in that mathematical reality is ``measured'' at large scale by running computations.

There is wide agreement in mathematics that these datasets should be a common resource and be open and freely available.
Moreover, the software used to produce them is usually open source and free as well.
%While FAIR does not necessarily mean Open~\cite[2.3]{FAIR}, the appropriate Openness for almost all mathematical data is indeed open source and free.
Such an ecosystem is embraced by the mathematics community as a general vision for their future research infrastructure~\cite{NAS14}, adopted by the International Mathematical Union as the Global Digital Mathematics Library initiative \cite{GDML:on}.

To better understand the scale of the problem, Figure~\ref{fig:datasets} gives an overview over some state-of-the-art datasets.
Here we already use the division into four kinds of mathematical data that we will develop in Section~\ref{sec:fair}.

\begin{figure*}[htp]\centering\small\def\cite#1{}
  \begin{tabular}{| p{.35\textwidth} | p{0.55\textwidth}|}\hline
  Dataset & Description \\\hline\hline
  \multicolumn{2}{|l|}{\textbf{Symbolic Data}} \\\hline
  Theorem prover libraries \cite{OAFproject:on}  & $\approx 5$ proof libraries, $\approx 10^5$ theorems each, $\approx 200$ GB \\\hline
  Computer algebra systems \cite{sagemath} & e.g., SageMath distribution bundles $\approx 4$ GB of various tools and libraries\\\hline
  Modelica libraries \cite{Modelica:on} &$> 10$ official, $> 100$ open-source, $\approx 50$ commercial,
      $> 5.000$ classes in the Standard Library, industrial models can reach $.5$M equations \\\hline
  \multicolumn{2}{|l|}{\textbf{Relational Data}} \\\hline
 Integer Sequences \cite{OEIS:on} & $\approx 330$K sequences, $\approx 1$ TB  \\\hline
 Sequence Identities \cite{kwarc:datahost:on} & $\approx .3$M sequence identities, $\approx 2.5$ TB \\\hline
 Highly symmetric graphs, maps, polytopes \cite{ConderCensuses:on, HartleyPolytopes:on, LeemansPolytopes:on, PotocnikCensuses:on, RoyleVT:on, WilsonET:on} & $\approx 30$ datasets, $\approx 2\cdot10^6$ objects, $\approx 1$ TB \\\hline
  Finite lattices \cite{KohLat:on, LeeLat:on, MalLat:on} & $7$ datasets, $\approx 17 \cdot 10^9$ objects, $\approx 1.5$ TB \\\hline
  Combinatorial statistics and maps \cite{findstat} & $\approx1.500$ objects \\\hline
  SageMath databases \cite{SageDB:on} & $12$ datasets \\\hline
  $L$-functions and modular forms \cite{lmfdb:on} & $\approx 80$ datasets, $\approx 10^9$ objects, $\approx 1$ TB \\\hline
  \multicolumn{2}{|l|}{\textbf{Linked Data}} \\\hline
   zbMATH \cite{zbMATH:on} & $\approx 4$M publication records with semantic data, $\approx 30$M reference data, $>1$M disambig. authors, $\approx 2,7$M full text links: $\approx 1$M OA \\\hline
  swMATH \cite{swMATH:on} & $\approx 25$K software records with $> 300$K links to $> 180$K publications \\\hline
  EuDML  \cite{EuDML:on} & $\approx 260$K open full-text publications \\\hline
  Wikidata  \cite{wikidata:on} & $34$ GB linked data, thereof about $4$K formula entities, interlinked, e.g., with named theorems, persons, and/or publications \\\hline
  \multicolumn{2}{|l|}{\textbf{Narrative Data}} \\\hline
  arXiv.org & $\approx 300$K math preprints (of $\approx1.6$M) most with {\LaTeX} sources\\\hline
  EuDML  \cite{EuDML:on} & $\approx 260$K open full-text publications, digitized journal back issues \\\hline
  MathOverFlow & $\approx 1,1$M questions/answers, $\geq11$K answer authors \\\hline
  Stacks project & $\geq 6000$ pages, semantically annotated, curated, searchable textbook \\\hline
  nLab & $\geq 13$K pages on category theory and applications\\\hline
\end{tabular}
  \caption{Summary of mathematical datasets}\label{fig:datasets}
\end{figure*}

\paragraph{State of FAIRness for Mathematical Datasets}
Mathematical datasets are generally produced, published, and maintained with virtually no systematic attention to the FAIR principles~\cite{FAIR,WilDumAal:FAIR16} for making data findable, accessible, interoperable, and reusable.
In fact, often the sharing of data is an afterthought --- see~\cite{Bercic:cmo:wiki} for an overview of mathematical datasets and their ``FAIR-readiness''.

Moreover, the inherent complexity of mathematical data makes it very difficult to share in practice: even freely accessible datasets are often very hard or impossible to reuse, let alone make machine-interoperable because there is no systematic way of specifying the relation between the raw data and its mathematical meaning.  Therefore, unfortunately FAIR mathematics essentially does not exist today.

\paragraph{Motivation}
Our ultimate goal is to standardize a framework for representing mathematical datasets.
As a first step, we present MathDataHub, an infrastructure for systematically sharing relational datasets.

Such a standard for FAIR data representations in mathematics would lead to several incidental benefits:
\begin{compactitem}
\item increased productivity for mathematicians by allowing them to focus on the mathematical datasets themselves while leaving issues of encoding, management, and search to dedicated systems,
\item improved reliability of published results as the research community can more easily scrutinize the underlying data,
\item collaborations via shared datasets that are currently prohibitively expensive due to the difficulty of understanding other researchers' data, including collaborations across disciplines and with industry practitioners, who are currently excluded due to the difficulty of understanding the datasets,
\item reward mathematicians for sharing datasets (which is currently often not the case), e.g., by making datasets citable and their reuse known,
%\item It allows making the provenance of datasets more explicit.
\item more sustainable research by guaranteeing that datasets can be archived and their meaning understood in perpetuity (which is essential especially in mathematics).
\end{compactitem}

\paragraph{Contribution}
In this article we survey and systematize how mathematical data is represented and shared and analyze how it enables or prevents FAIR mathematics.
We pay particular attention to the mathematics-specific aspects of FAIR sharing, which, as we will observe, go significantly beyond the original formulation of FAIR.

As a first step towards a universal framework, and as a concrete example of FAIR-enabling mathematics-specific infrastructure, we introduce MathDataHub.
This is a platform for sharing relational mathematical datasets in a way that systematically enables FAIRness.

\paragraph{Overview}
In the next section, we survey the particular challenges to FAIR sharing in mathematics.
In Section~\ref{sec:deep}, we develop the concept of ``deep FAIR'' to accommodate for the semantics issues, and in Section~\ref{sec:hub} we present a prototypical system that can help achieve them for the case of relational data.
Section~\ref{sec:concl} concludes the article
%%% Local Variables:
%%% mode: latex
%%% mode: visual-line
%%% fill-column: 5000
%%% TeX-master: "report"
%%% End:

%  LocalWords:  Cremona:LMFDB16,lmfdb:on inparahighlight emph systematize standardizing compactitem scrutinize ednote standardized textbf claim:oligopolization oldpart noindent optimization revolutionized smallskip inparaenum compactenum Modelica zbMATH zbMATH:on swMATH formalized lmfdb:on transdisciplinarily revolutionize commercialization centering textwidth hline OAFproject:on ConderCensuses:on LeemansPolytopes:on PotocnikCensuses:on cdot10 cdot disambig BilTen:fingerprint13 sec:concl sagemath digitized geq11 geq analyze
