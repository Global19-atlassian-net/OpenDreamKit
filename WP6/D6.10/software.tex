\ednote{Write a general notebook introduction, cite D4.2, D4.3}

\subsection{Formula Search Engine}

\ednote{
    Recycle some existing text from somewhere
    Spin: MathWebSearch was an experimental tool that required a lot of work to setup per application
}

MathWebSearch \ednote{cite} is a search engine enabling semantic formula search. 
For example, given a query like $a^2 + b^2 = c^2$\ednote{Make formulae red} it will find documents containing $3^2 + 4^2 = 5^2$ or $x^2 + y^2 = z^2$.
It was originally developed by \ednote{TODO}. 

The MathWebSearch daemon takes in a set of Content MathML\ednote{cite original}-encoded formulae.
It then builds an index using a technique called Substitution Tree Indexing
Having generated and loaded this index into memory, it is then capable of answering queries.
The queries are also ContentMathML with a custom extension for marking query variables.
The answer to a query consists of the set of matching formulae and the appropriate variable subsiutions. 
Additionally meta-information, such as the URLs of the documents that the formulae come from, is returned. 
For details, we refer the interested reader to \ednote{cite}.  

However, the daemon itself is not sufficient to fully use MathWebSearch. 
Conceptually, an instance of MathWebSearch consists of four components:
\ednote{Do we want to have a picture here?}
\begin{itemize}
    \item A program called a \textbf{Harvester} which is given a provided with a corpus of documents and extracts the set of formulae contained in it;
    \item the \textbf{MathWebSearch daemon} itself, which based on the harvested formulae, generates and maintains an index as described above;
    \item an \textbf{query input parser}\ednote{Name?} which converts user input into a query the daemon can understand
    \item a \textbf{frontend} which sends user input to the query parser and daemon, and presents the query result to the user.  
\end{itemize}

\ednote{
    - ArXiV search
    - Zentralblatt search
}

\subsection{Enabling Formula Search Deployments}

To use MathWebSearch inside OpenDreamKit, we need to be able to flexibly use it as a new component. 
To achieve this, we implemented deployment infrastructure on top of the core MathWebSearch daemon. 
In particular, to enable easy usage we made all components available as Docker Containers \ednote{Cite docker and refer to the appropriate WPs}. 
These components are

\ednote{Link all of these to GitHub repositories}
\begin{enumerate}
    \item the core \textbf{MathWebSearch Daemon} itself \ednote{github.com/MathWebSearch/mws}
    \item \textbf{mwsapi}, a thin API layer making it much easier for the frontend
\end{enumerate}

The inter

- we created a stack of tools on top of MWS to enable easier deployment
- for this process we made

- deployment of MWS previously required a lot of effort
- for this task: built infrastructure for managing large harvests and building interfaces
    - auto-updating of index
    - deployment setup => being able to spin up new instances more easily
    - api wrapper (more flexible frontends)
    - flexible way of dropping in harvesters

\subsection{Building a Notebook Search}

\ednote{
- describe sage notebooks
- describe how we implemented the hharvester
- describe the dedidcated frontend
- describe the simplicity of this implementation wrt the previous parts
}

\subsection{Future Plans for MathWebSearch}

- describe the custom component / frontends for NLAB
- temasearch
- python ast search + input syntax?