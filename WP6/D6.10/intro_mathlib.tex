\subsection{Mathematical Libraries}\label{sec:mathlib}
% Innovations based on mathematical knowledge and algorithms yield many improvements in science, engineering, economy, ecology, health care, security, and society overall. 
% For example, our global positioning system (GPS) needs the mathematics of relativistic physics, our mobile phones use frequencies allocated through combinatorial optimization, the combinatorics of our genome yields clues to curing rare diseases, the privacy of our communications depends on cryptographic protocols steeped in number theory, and our national security is relying on the mathematical analysis of increasingly complex networks. 
% Fundamental mathematical research and its direct application in practical situations enable many engineering, science, and business innovations that enrich society and mankind.\ednote{FR: throw out this paragraph? Or replace with one sentence?}

% Modern mathematical research increasingly depends on collaborative tools, computational environments, and online databases, and these are changing the way mathematical research is conducted and how it is turned into applications.
% For example, engineers now use mathematical tools to build and simulate physical models based on systems of differential equations with millions of variables, combining building blocks and algorithms taken from libraries shared all over the internet.


For the the data (D) aspect of ``doing mathematics'' (see the \textbf{Tabulation} corner in Figure~\ref{fig:tetrapod}) we have
\begin{compactenum}
\item developed an both an innovative model of (deep) FAIR in mathematics (and have integrated it with the MitM paradigm developed in OpenDreamKit),
\item have implemented in a prototypical system (data.mathhub), and have
\item evaluated it on the mathematical community outside the core OpenDreamKit community.
\end{compactenum}

Traditionally, mathematics has not paid particular attention to the creation and sharing of data --- the careful computation and publication of logarithm tables is a typical example of the extent and method.
This has changed with the advent of computer-supported mathematics, and the practice of modern mathematics is increasingly data-driven.
Today it is routine to use mathematical datasets in the Gigabyte range, including both human-curated and machine-produced data.
Examples include the L-Functions and Modular Forms Database (LMFDB; $\sim 1$ TB data in number theory)~\cite{Cremona:LMFDB16,lmfdb:on} and the GAP Small Groups Library~\cite{GapSmallGroups:on} with $\sim 450$ million finite groups.  
In a few, but increasingly many areas, mathematics has even acquired traits of experimental sciences in that mathematical reality is ``measured'' at large scale by running computations.

There is wide agreement in mathematics that these datasets should be a common resource and be open and freely available.
Moreover, the software used to produce them is usually open source and free as well.
%While FAIR does not necessarily mean Open~\cite[2.3]{FAIR}, the appropriate Openness for almost all mathematical data is indeed open source and free.
Such an ecosystem is embraced by the mathematics community as a general vision for their future research infrastructure~\cite{NAS14}, adopted by the International Mathematical Union as the Global Digital Mathematics Library initiative \cite{GDML:on}.

To better understand the scale of the problem, Figure~\ref{fig:datasets} gives an overview over some state-of-the-art libraries.
Here we already use the division into four kinds of mathematical data that we will develop in Section~\ref{sec:fair}.

\begin{figure*}[htp]\centering\small\def\cite#1{}
  \begin{tabular}{| p{.35\textwidth} | p{0.55\textwidth}|}\hline
  Dataset & Description \\\hline\hline
  \multicolumn{2}{|l|}{\textbf{Symbolic Knowledge}} \\\hline
  Theorem prover libraries \cite{OAFproject:on}  & $\approx 5$ proof libraries, $\approx 10^5$ theorems each, $\approx 200$ GB \\\hline
  Computer algebra systems \cite{sagemath} & e.g., SageMath distribution bundles $\approx 4$ GB of various tools and libraries\\\hline
  Modelica libraries \cite{Modelica:on} &$> 10$ official, $> 100$ open-source, $\approx 50$ commercial,
      $> 5.000$ classes in the Standard Library, industrial models can reach $.5$M equations \\\hline
  \multicolumn{2}{|l|}{\textbf{Relational Data}} \\\hline
 Integer Sequences \cite{OEIS:on} & $\approx 330$K sequences, $\approx 1$ TB  \\\hline
 Sequence Identities \cite{kwarc:datahost:on} & $\approx .3$M sequence identities, $\approx 2.5$ TB \\\hline
 Highly symmetric graphs, maps, polytopes \cite{ConderCensuses:on, HartleyPolytopes:on, LeemansPolytopes:on, PotocnikCensuses:on, RoyleVT:on, WilsonET:on} & $\approx 30$ datasets, $\approx 2\cdot10^6$ objects, $\approx 1$ TB \\\hline
  Finite lattices \cite{KohLat:on, LeeLat:on, MalLat:on} & $7$ datasets, $\approx 17 \cdot 10^9$ objects, $\approx 1.5$ TB \\\hline
  Combinatorial statistics and maps \cite{findstat} & $\approx1.500$ objects \\\hline
  SageMath databases \cite{SageDB:on} & $12$ datasets \\\hline
  $L$-functions and modular forms \cite{lmfdb:on} & $\approx 80$ datasets, $\approx 10^9$ objects, $\approx 1$ TB \\\hline
  \multicolumn{2}{|l|}{\textbf{Linked Data}} \\\hline
   zbMATH \cite{zbMATH:on} & $\approx 4$M publication records with semantic data, $\approx 30$M reference data, $>1$M disambig. authors, $\approx 2,7$M full text links: $\approx 1$M OA \\\hline
  swMATH \cite{swMATH:on} & $\approx 25$K software records with $> 300$K links to $> 180$K publications \\\hline
  EuDML  \cite{EuDML:on} & $\approx 260$K open full-text publications \\\hline
  Wikidata  \cite{wikidata:on} & $34$ GB linked data, thereof about $4$K formula entities, interlinked, e.g., with named theorems, persons, and/or publications \\\hline
  \multicolumn{2}{|l|}{\textbf{Narrative Data}} \\\hline
  arXiv.org & $\approx 300$K math preprints (of $\approx1.6$M) most with {\LaTeX} sources\\\hline
  EuDML  \cite{EuDML:on} & $\approx 260$K open full-text publications, digitized journal back issues \\\hline
  MathOverFlow & $\approx 1,1$M questions/answers, $\geq11$K answer authors \\\hline
  Stacks project & $\geq 6000$ pages, semantically annotated, curated, searchable textbook \\\hline
  nLab & $\geq 13$K pages on category theory and applications\\\hline
\end{tabular}
  \caption{Summary of mathematical libraries}\label{fig:datasets}
\end{figure*}

\ednote{add the collection used for the jupyter notebooks to the table}

We present a uniform infrastructure for sharing, finding, and searching mathematical libraries of any kind.
Moreover, we seed this infrastructure with a collection of representative libraries (all of them state of the art and large) as described below.


%%% Local Variables:
%%% mode: latex
%%% mode: visual-line
%%% fill-column: 5000
%%% TeX-master: "report"
%%% End:
