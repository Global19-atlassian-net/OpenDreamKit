\section*{\texorpdfstring{Deliverable description, as taken from Github
issue
\href{https://github.com/OpenDreamKit/OpenDreamKit/issues/148}{\#148} on
2017-02-27}
{Deliverable description, as taken from Github issue \#148 on 2017-02-27 \{.notoc\}\textbackslash{}n}}\label{deliverable-description-as-taken-from-github-issue-148-on-2017-02-27-.notocn}

\begin{itemize}
\tightlist
\item
  \textbf{WP7:}
  \href{https://github.com/OpenDreamKit/OpenDreamKit/tree/master/WP7}{Social
  Aspects}
\item
  \textbf{Lead Institution:} University of Oxford
\item
  \textbf{Due:} 2017-02-28 (month 18)
\item
  \textbf{Nature:} Report
\item
  \textbf{Task:} T7.1
  (\href{https://github.com/OpenDreamKit/OpenDreamKit/issues/144}{\#144}):
  Social input to design
\item
  \textbf{Proposal:}
  \href{https://github.com/OpenDreamKit/OpenDreamKit/raw/master/Proposal/proposal-www.pdf}{p.~58}
\item
  \textbf{\href{https://github.com/OpenDreamKit/OpenDreamKit/raw/master/WP7/D7.1/report-final.pdf}{Final
  report}}
\end{itemize}

OpenDreamKit builds on top of a large ecosystem of (mostly) academic
open-source systems, many of which are large-scale themselves: for
example our chosen test system \href{http://sagemath.org}{SageMath} is
the outcome of a decade of work by hundred of contributors; many others
are decades old. The social engineering aspects involved in such a large
ecosystems are therefore both intricate and central for its long run
sustainability. This motivates OpenDreamKit's objective in WP7 of
studying the collaborative processes of free open source (mathematical)
software development so as to produce guidelines for best practice as
well as to develop ideas for extending existing processes to an
``ecosystem of systems''.

In this deliverable we survey the methodology, data, and tools needed to
assess development models of large-scale academic open-source systems,
such as the probable correlation between the size of the atomic
contribution vs.~the speed of the contribution making it into the code,
and collect appropriate statistical data, to be published as a report
(and possibly a conference publication). While in the proposal it was
assumed that the latter might require non-trivial amount of programming
work, even only for our test system, great open-source tools to address
precisely these kinds of questions were released last year, and we used
one of them instead.

Accomplishments:

\begin{itemize}
\tightlist
\item
  \(\checkmark\) a large number of publications and online sources was
  reviewed for applicability
\item
  \(\checkmark\) various analytic tools were tried on a sample of
  SageMath components
\item
  \(\checkmark\) results were summarised in a report, with conclusions
  and pointers to further possible developments
\end{itemize}
