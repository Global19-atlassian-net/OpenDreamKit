\documentclass{deliverablereport}

\usepackage[style=alphabetic,backend=bibtex]{biblatex}
\addbibresource{../../lib/kbibs/kwarcpubs.bib}
\addbibresource{../../lib/kbibs/extpubs.bib}
\addbibresource{../../lib/kbibs/kwarccrossrefs.bib}
\addbibresource{../../lib/kbibs/extcrossrefs.bib}
\addbibresource{../../lib/deliverables.bib}
%\addbibresource{../../lib/publications.bib}
\addbibresource{rest.bib}
% temporary fix due to http://tex.stackexchange.com/questions/311426/bibliography-error-use-of-blxbblverbaddi-doesnt-match-its-definition-ve
\makeatletter\def\blx@maxline{77}\makeatother

\deliverable{social-aspects}{social-datareport}
\deliverydate{02/27/2017}
\duedate{02/27/2017 (Month 18)}
\def\pn{OpenDreamKit}
\author{ }

\begin{document}
\maketitle
%  Work Package WP6 develops a novel, foundational, knowledge-based framework for
  interfacing existing open source mathematical software systems and knowledge bases into
  a mathematical VRE, where systems can delegate functionalities among each other
  seamlessly without losing semantics.

  The overall Math-in-the-Middle (MitM) Framework developed in WP6 over the last three
  years is described in D6.5; this Report complements it by describing the curated
  contents Math-in-the-Middle (MitM) Ontology which serves as a reference and pivotal
  point for translations between the various input languages of mathematical software
  systems and knowledge bases.

  In a nutshell, the MitM Ontology describes the mathematical objects, concepts, and their
  relations in a general, system-agnostic way in an OMDoc/MMT theory graph while the
  mathematical systems export API theories that describe the system interface language in
  terms of types, classes, constructors, and functions -- again in OMDoc/MMT. These two
  levels of descriptions are linked by OMDoc/MMT alignments that allow the translation of
  expressions between systems.

%%% Local Variables:
%%% mode: visual-line
%%% fill-column: 5000
%%% mode: latex 
%%% TeX-master: "report"
%%% End:

\strut\githubissuedescription
\newpage\tableofcontents\newpage

Large-scale and to a lesser extent medium-scale open-source software 
is as a rule a product of a collaborative effort spanning many years of
development, improvements, bug fixes, ports to new platforms,
and partial or even full rewrites. A number of interesting related questions
arise in this context.
\begin{enumerate}
\item What are available data, measurement parameters and tools
allowing for analysis?
\item Can one assess the usefulness of the project by estimating
how ``alive'' it is, i.e. how much it is changing over time?
\item Can one reliably range the contributors
by the effort put into the project?
\item Can one produce recommendations on the team size and composition
to ensure project's well-being?
\item Does the ``openness'' of the project matter?
\item Reliability of the system, stability of APIs etc., 
reproducibility of computation results, etc.
\end{enumerate}

\section{Data, parameters and tools}

The main sources of information about the history of a project are versions of
its source code and logs of various relevant communications, discussions, and
test results.  Before the wide acceptance of {\em revision control systems} 
(RCS) \cite{OSullivan:MakingSenseOfRCS} such as  CVS \cite{CVSWeb} and
Git \cite{ChaStr:pg14} the only readily available source code data came from
regular (often infrequent) public releases. Then it has become
more and more widespread, although not universal (cf. e.g. GAP \cite{gap},
which only in the past few years made its RCS 
public---and has not released earlier RCS data)
to keep the RCS {\em trees} holding {\em commits}---code changes
accompanied by comments---available online with read access for the public.

Communications on the project take basically three (not totally disjoint)
forms: mailing lists/bulletin
boards and tracker/code reviewing systems,
such as Trac \cite{wp7:trac}, Redmine, Github \cite{wp7:github},
Bitbucket, Gitlab, etc, and documentation
systems/wikis. The latter is open by nature, whereas for the first two
the prevailing trend, at least in the domain related to the
ODK themes, for these is the ever increased
openness of the development process.

The current prevailing form of the analysis of the source code is
based on analysing authorship, frequency, and other parameters of commits
in the RCS. Most of the tools are in one or another way related to 
Github and its APIs to access RCS trees and collect statistics,
see e.g. \cite{wp7:afronshapeoss}.
Communications are analysed using various text mining and analysis tools,
such as FOSS Heartbeat \cite{wp7:fossheartbeat};
these are not dissimilar to tools used to analyse {\em social networks}
such as Facebook  \cite{wp7:russell2013mining}.
Interestingly, Github as a collaborative social network has been
analysed \cite{wp7:githubandsof,DBLP:journals/corr/LimaRM14,DBLP:journals/corr/VasilescuSWSB15,
wp7:Kalliamvakou:2014:PPM:2597073.2597074} and compared to Stack Overflow \cite{wp7:stackoverflow}.

\section{Alive or dead? (Cathedral or Bazaar?)}
It is not obvious whether extreme stability of the code base,
such as e.g. Knuth's \TeX \cite{Knuth:ttb84}, with releases
numbered by consecutive digits of $\pi$, and only occuring
once every 5 or 6 years, see
\url{http://tug.org/texlive/devsrc/Build/source/texk/web2c/tex.web}
is a bad sign.
Although actively developed OSS projects are certainly showing 
a different pattern, with dozens of commits per day, etc.
Different projects have vastly different cultures in regard of
commit frequency, commit size, etc., and this makes them 
hard to compare---something already observed in 
 \cite{raymond99:cathedral-bazaar}. 


\section{Ranging contributors}
This infomatiion may be extractable from a number of sources.

\subsection{Commits}
Are commits, as advocated by GitHub, a good way to range contributors by their
contributions? Indeed, for GitHub it is very easy to analyse the commits' (meta)data,
and not so as soon as real contributions are involved. To give one a simple example, how does
one measure the work done by quality assurance people, or by someone who does complicated debugging
and reports its results? Such an activity, if measured in commits, would fall under the radar almost
completely.

Types of commits...

\subsection{Trackers and other exchanges}

NB: ticket ranging? (see other deliverable in WP7)

\section{Team composition}

\section{Openness, licensing, etc}

\section{Reliability, stability of APIs, reproducibility}
Whenever a new external component is to be added to an OSS system,
a number of questions, potentially possible to be answered by
analysing the flow of code and patches in the component, arise.

Information on the reliability (i.e. how often critical level
bugs pop up, etc) of the system might be ...

Application programming interfaces (APIs) are extremely
important parts of large-scale software systems.
A very important question is of {\em stability} of APIs, that is,
whether the API was in place for some time already and did not change
for a period of time---otherwise it might happen that a change in APIs will
require a redesign of the OSS.
While some APIs are extremely stable, e.g. famous BLAS's and LAPACK's
\cite{2002:USB:567806.567807,Anderson:1990:LPL:110382.110385}
APIs for numerical linear algebra, it is much less certain in more
modern areas, such as GPU computing, web programming, etc.

In the case of systems making regular (and well-documented)
releases it is normally possible to check stability of APIs
directly from the documentation. However it is becoming more and
more usual that there are no relases as such, the most fresh
``master branch'' of the RCS tree of the system is all that
can be taken as the release (essentially, the commit
hash has to be used to give the release a version number).
A slightly better situation is where ``stable'' releases are tagged
in the RCS with a version number label.
In these cases one could check the history of certain pieces
of the code in the RCS manually; however it might be 
desirable to have automated tools for this. Literature search for
the latter did not return anything that does not require
actual installation of the software in question and testing its APIs.




\section{Conclusions and future work}
A wide range of tools to pick up the low-hanging fruit--- 
the commit history from the RCS and project communications---
and study it are already available and can be readily applied
to open-source VREs and to their components.
However, it is unclear whether any practically interesting conclusions
and recommendations can be derived from such studies---not
the least due to their extreme imperfections discussed above.

Better mathematical models to analyse OSS ought to be developed, perhaps
by treating them as large-scale discrete dynamical systems
\cite{Antoulas:2005:ALD:1088857,Minati:2011:MMS:2208175}---whether
this is feasible is open to discussion.
\printbibliography
\end{document}

%%% Local Variables:
%%% mode: latex
%%% TeX-master: t
%%% End:

%  LocalWords:  githubissuedescription newpage tableofcontents newpage printbibliography
