\documentclass{deliverablereport}

\usepackage[style=alphabetic,backend=bibtex]{biblatex}
\addbibresource{../../lib/kbibs/kwarcpubs.bib}
\addbibresource{../../lib/kbibs/extpubs.bib}
\addbibresource{../../lib/kbibs/kwarccrossrefs.bib}
\addbibresource{../../lib/kbibs/extcrossrefs.bib}
\addbibresource{../../lib/deliverables.bib}
%\addbibresource{../../lib/publications.bib}
\addbibresource{rest.bib}
% temporary fix due to http://tex.stackexchange.com/questions/311426/bibliography-error-use-of-blxbblverbaddi-doesnt-match-its-definition-ve
\makeatletter\def\blx@maxline{77}\makeatother

\deliverable{social-aspects}{social-datareport}
\deliverydate{02/27/2017}
\duedate{02/27/2017 (Month 18)}
\def\pn{OpenDreamKit}
\author{ }

\begin{document}
\maketitle
%  Work Package WP6 develops a novel, foundational, knowledge-based framework for
  interfacing existing open source mathematical software systems and knowledge bases into
  a mathematical VRE, where systems can delegate functionalities among each other
  seamlessly without losing semantics.

  The overall Math-in-the-Middle (MitM) Framework developed in WP6 over the last three
  years is described in D6.5; this Report complements it by describing the curated
  contents Math-in-the-Middle (MitM) Ontology which serves as a reference and pivotal
  point for translations between the various input languages of mathematical software
  systems and knowledge bases.

  In a nutshell, the MitM Ontology describes the mathematical objects, concepts, and their
  relations in a general, system-agnostic way in an OMDoc/MMT theory graph while the
  mathematical systems export API theories that describe the system interface language in
  terms of types, classes, constructors, and functions -- again in OMDoc/MMT. These two
  levels of descriptions are linked by OMDoc/MMT alignments that allow the translation of
  expressions between systems.

%%% Local Variables:
%%% mode: visual-line
%%% fill-column: 5000
%%% mode: latex 
%%% TeX-master: "report"
%%% End:

\strut\githubissuedescription
\newpage\tableofcontents\newpage

Large-scale and to a lesser extent medium-scale open-source software 
is as a rule a product of a collaborative effort spanning many years of
development, improvements, bug fixes, ports to new platforms,
and partial or even full rewrites. A number of interesting related questions
arise in this context.
\begin{enumerate}
\item What are available data, measurement parameters and tools?
\item Can one assess the usefulness of the project by estimating
how ``alive'' it is, i.e. how much it is changing over time?
\item Can one reliably range the contributors
by the effort put into the project?
\item Can one produce recommendations on the team size and composition
to ensure project's well-being?
\item Does the ``openness'' of the project matter?
\item Reliability, reproducibility, etc.
\end{enumerate}

\section{Data, parameters and tools}

The main sources of information about the history of a project are versions of
its source code and logs of various relevant communications, discussions, and
test results.  Before the wide acceptance of {\em revision control systems} 
(RCS) \cite{OSullivan:MakingSenseOfRCS} such as  CVS \cite{CVSWeb} and
Git \cite{ChaStr:pg14} the only readily available source code data came from
regular (often infrequent) public releases. Then it has become
more and more widespread, although not universal (cf. e.g. GAP \cite{gap},
which only in the past few years made its RCS 
public---and has not released earlier RCS data)
to keep the RCS {\em trees} holding {\em commits}---code changes
accompanied by comments---available online with read access for the public.

Communications on the project take basically three (not totally disjoint)
forms: mailing lists/bulletin
boards and tracker/code reviewing systems,
such as Trac \cite{wp7:trac}, Redmine, Github \cite{wp7:github},
Bitbucket, Gitlab, etc, and documentation
systems/wikis. The latter is open by nature, whereas for the first two
the prevailing trend, at least in the domain related to the
ODK themes, for these is the ever increased
openness of the development process.

The current prevailing form of the analysis of the source code is
based on analysing authorship, frequency, and other parameters of commits
in the RCS. Most of the tools are in one or another way related to 
Github and its APIs to access RCS trees and collect statistics,
see e.g. \cite{wp7:afronshapeoss}.
Communications are analysed using various text mining and analysis tools,
such as FOSS Heartbeat \cite{wp7:fossheartbeat};
these are not dissimilar to tools used to analyse {\em social networks}
such as Facebook.


\section{Alive or dead?}

\section{Ranging contributors}

\section{Team composition}

\section{Openness, licensing, etc}

\section{Reliability and reproducibility}

\printbibliography
\end{document}

%%% Local Variables:
%%% mode: latex
%%% TeX-master: t
%%% End:

%  LocalWords:  githubissuedescription newpage tableofcontents newpage printbibliography
