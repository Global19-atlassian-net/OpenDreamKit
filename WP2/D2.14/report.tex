\documentclass{deliverablereport}

\usepackage[style=alphabetic,backend=bibtex]{biblatex}
\usepackage{todonotes}
\addbibresource{report.bib}
\addbibresource{../../lib/publications.bib}

\usepackage{xparse}
\usepackage{etoolbox}
\usepackage{caption}

\deliverable{dissem}{ibook3c}
\duedate{31/07/2018 (M47)}
\deliverydate{31/07/2019}
\author{Marcin Kostur et al.}

\begin{document}
\maketitle
\githubissuedescription
\newpage
\tableofcontents
\newpage

\section{Notes}

\begin{itemize}
\item Textbook at https://github.com/fangohr/introduction-to-python-for-computational-science-and-engineering/blob/master/Readme.md
  \item consists of notebooks, one per chapter
\item can be executed online using Binder
\item tested with NBVal
\item lower barrier towards testing things
\item output as html and pdf available

\item attach TOC as pdf to deliverable

\item mention building and testing in Docker
\item translation into Turkish
\item DOI
\end{itemize}

\section{Introduction to Python for Computational Science and Engineering}

Contributors: Hans Fangohr, Thomas Kluyver, Marijan Beg, Min Ragan-Kelly

\subsection{Reducing barriers for learners using interactive textbooks}

The Jupyter notebook as a virtual research environment holds great
potential for the creation and use of interactive documents. In this
context, we investigate and prototype the use of such interactive
notebooks in the context of education at the university level.

There is a long history in academia to provide textbooks either as
the main point of reference for a given lecture course, or as an
additional "background reading" to provide more details which cannot
be covered by blackboard- or slide-centred lectures, typically due to
the lack of time available.

While providing potentially a wealth of information, such textbooks
are static, and require unusual skill to be exclusively learned
from. Instead, it is a common model to ask students to carry out
practical problem-solving exercises: this enforces engagement with the
material and supports deep learning of the subject.

For computational problems, there is often significant effort required
to set up an environment of software (such as Python with required
libraries or a symbolic mathematics package) and then to set up a
problem environment that allows the study of the topic under
investigation. For example, to solve a differential equation
numerically, the problem environment includes setting up functions
describing the ODE, boundary conditions, and a grid on which the
numerical solution should be obtained. Once this point is reached, the
student can start to explore -- for example -- the properties of a
numerical method being used to solve differential equations.

The \emph{interactive textbooks} developed here allow to improve the
learning experience by significantly reducing this barrier: both
setting up the software environment and setting up the problem
environment are reduced to the task of opening the interactive
document in a browser for which the teacher provides the URL, a short
wait while the virtual environment is created on the fly (using the
Binder service) and navigating to the point of interest in the text
book. Immediately, it is possible for the learner, to interactively
explore the topic of learning within a prepared learning and software
environment.

The technology can be transferred to the European Open Science Cloud
in fairly straightforward manner, as the communication protocal is https.

In the following, we detail the work on the interactive text books on
computational science and engineering (Sect.~)\ref{sec:computational-science-and-engineering})
and problems in physics with Sage \todo[inline]{Marcin/Viviane/Nicolas, please
udpate as required.}

\subsection{Interactive text book on Computational Science and Engineering}
\label{sec:computational-science-and-engineering}


The application of mathematics in science and engineering is the topic
of the textbook "Introduction to Computational Science and
Engineering".

\subsubsection{Context and overview}

The work is based on a text book that was available as a PDF file (and
generated from a \LaTeX{} file). In this deliverable, we have reviewed
the textbook and updated it from Python 2 to Python 3, added various
sections and a chapter on Pandas, but most importantly translated the
\LaTeX{} sources into Jupyter notebooks. Furthermore, we demonstrate and
use tools such as the \texttt{bookbook} and \texttt{nbconvert}
package to enable the automatic translation of the Jupyter notebook
chapters into a single PDF or a set of HTML pages, and which have been
supported through OpenDreamKit. \todo[inline]{Min, can we reference a
  Task/Deliverable/WorkPackage here?}

The new PDF is created from the Jupyter notebooks (using LaTeX as an
intermediate translation) and then by compiling the auto-generated
LaTeX sources to create a high quality PDF file. A LaTeX file with
custom style settings can be given as a template to the
\texttt{bookbook} package. The different chapters (each being one
notebook) are merged automatically, and get a joint table of contents.

From the same Jupyter notebook sources, a set of HTML files can be
created to allow more convenient online reading of the material. These
html files are organised into one HTML file per chapter (each being
created from one notebook), and an additional index file providing
links to all chapters.

The (automatic) translation of the Jupyter-notebook based textbook
into PDF is important to provide (at least) the same level of
publication quality outputs that can be expected from the more
traditional LaTeX based manuscript. The conversion to html is an added
bonus, and offers a way of reading the document that is more
appropriate for commonly used devices such as laptops, tables and
smart phones.

\subsubsection{Added value of Notebook based text book}

The additional value comes from the Jupyter Notebook based
nature of the chapters:

\begin{enumerate}
\item Students can download the notebooks, and inspect all
  computational steps that have created the results shown in the
  textbook. Assuming they have the relevant software installed,
  they can execute them on their own machine, modify,
  explore, understand, and extend the examples. As all computational
  steps are included in the notebooks, there is no guessing about
  assumptions, no code being executed before an example is introduced,
  or no reconstructions of sections labelled ``the required
  transformation of X is left as an exercise to the reader'' required:
  all steps are contained in the notebooks. This reduces the barrier
  towards learning.

\item Using the cloud hosted Binder environment, learners can open
  the book in an \emph{interactive virtual research environment} that
  has been created on demand just for them. While providing all the
  advantages outlined above, in this setup \emph{no software
    installation is required}.

  In more detail:

  For the particular example text book ``only'' the commonly used
  Anaconda Python distribution is sufficient on the learners laptop as
  a software dependency (because  only the standard scientific Python stack
  are required dependencies such as numpy, scipy, matplotlib, pandas,
  and the notebook itself).

  This required installation of the software environment can be
  avoided using the Binder project (\todo[inline]{Min, which
    Task/Deliv to cite? maybe D1.5?}). As the
  \texttt{MyBinder.org} instance can create a cloud-based Jupyter
  Notebook with a software specification (for example through a python
  \texttt{requirements.txt} file) on demand, every student can start
  their own notebook server on MyBinder, browse chapters, and execute
  chapter notebooks as they like to achieve better understanding. As
  all of this happens in the browser, there is no software
  installation required.
\end{enumerate}

\subsubsection{Good practice in software engineering for
  computational science}

We have used the following technologies and processes to improve the
quality and maintanability of the open source text book:
\begin{itemize}
\item The sources are available and publicly readable on Github.
\end{itemize}

\newpage\printbibliography

\end{document}

%%% Local Variables:
%%% mode: latex
%%% TeX-master: t
%%% End:
