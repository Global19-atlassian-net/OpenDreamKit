\hypertarget{status-report}{%
\section{Status report}\label{status-report}}

The objective of task T2.10 is to deliver an open-source
community-curated indexing tool for resources in computational
mathematics. The goal is to collect examples, tutorials, lessons,
exercises, related to a system or a field, under a unique hub, while
maintaining the quality of the content through community curation.

Such a need had been felt in all communities involved in OpenDreamKit,
and each had come with its own solutions. The joint development efforts
under OpenDreamKit, and in particular the unifying force of the common
Jupyer notebook infrastructure, have given a unique occasion to produce
a unified solution, applicable to all systems. This report introduces
\emph{planetaryum}\footnote{\url{https://github.com/OpenDreamKit/planetaryum}.},
the web toolkit for building curated Jupyter notebook collections.

Following the OpenDreamKit philosophy, \emph{planetaryum} is not a
single piece of software, but rather a toolkit, meant to power many
different flavors of curated collections. While it may not cover all
possible use cases, it is versatile enough to adapt to many needs of the
community.

\hypertarget{history}{%
\subsection{History}\label{history}}

The need for maintaining various types of community curated help
resources has long been felt in any community involved in OpenDreamKit.
Some interesting examples are:

\begin{itemize}
\item
  Most computer algebra systems maintain a section of their
  website\footnote{\url{http://www.gap-system.org/Doc/Examples/examples.html}.}\footnote{\url{http://www.sagemath.org/help.html}.}
  containing links to high quality resources on the web.
\item
  Some systems also have a wiki, e.g. \url{http://wiki.sagemath.org/}.
\item
  The French SageMath community also used to host a well curated wiki
  with pointers to many didactic resources. The wiki was taken down due
  maintenance difficulties.
\item
  SageMath maintains its own Q\&A website at
  \url{https://ask.sagemath.org}, whereas most other systems rely on
  generic hosted solution such as StackOverflow\footnote{\url{https://stackoverflow.com/}.}.
\item
  For a long time, SageMath had been hosting a public instance of a
  SageMath server at \url{http://sagenb.org/}, where anyone could
  publish SageMath notebooks (old format incompatible with Jupyter) for
  everyone to view. The server had to be taken down among maintenance
  and security issues.
\end{itemize}

More recently, the Jupyter community has provided the NBViewer
service\footnote{\url{http://nbviewer.jupyter.org/}.}. It is a static
notebook previewer service that takes as input a URL pointing to a
\texttt{.ipynb} file, and renders a static version of it. Importantly,
the service does not host notebooks, it only renders them (and
temporarily caches the rendered result).

Following the success of the Jupyter format, both the GitHub and
BitBucket code hosting services have started rendering static versions
of Jupyter notebooks without relying on NBViewer.

The availability of these services has spurred a proliferation of
collections of public notebooks hosted on code sharing services such as
GitHub, presented to the public through either NBViewer or the service
builtin preview. However, this practice has the major inconvenience of
making it hard to search, classify and rank notebooks.

When we sat down to plan for this deliverable, we wanted to provide a
solution to host, search and rank public resources, produced and curated
by the community itself.

We started by evaluating available solutions. The first that we studied
was Géant OER\footnote{\url{https://oer.geant.org/}.}, a metadata
aggregator for multimedia content. It quickly became apparent that its
focus on metadata and multimedia did not suit our needs.

We also evaluated the \emph{nbgallery}\footnote{\url{https://github.com/nbgallery/nbgallery}.}
software, a solution for hosting, indexing searching and ranking Jupyter
notebooks. Despite its potential, the application is rather unstable,
and support is limited, given that it is essentially an internal project
maintained by one person. Indeed, we didn't manage to install a fully
working instance. Plus, although its advanced features are quite
impressive, it does not cover all the use cases we were interested in.

Finally, we experimented with a custom developed application\footnote{\url{http://sageindex.lipn.univ-paris13.fr/}.},
whose development had already started in 2015 at a SageMath meeting. The
application is capable of indexing, mirroring, searching and ranking all
types of resources found on the web, with dedicated treatment for the
most relevant formats, such as PDF, Jupyter, SageMath, HTML, etc.
Unfortunately, when the application entered the alpha stage, it soon
became apparent that the hard part, rather than developing the
application, was getting the community to use it. It simply felt like
the community did not feel the need for such a generic tool, that was
essentially trying to (poorly) replicate the job of a web search engine.

\hypertarget{planetaryum}{%
\subsection{Planetaryum}\label{planetaryum}}

After our first failed attempts, planetaryum came as a new take on the
problem. We realized that it was essential to recenter our efforts on a
well defined type of document, rather than dispersing our users in a
format agnostic aggregator such as\footnote{\url{http://sageindex.lipn.univ-paris13.fr/}.}.
The generalization of Jupyter as a common document format for all
OpenDreamKit systems presents a unique occasion to host instructional
resources in a unified way.

Planetaryum also comes with the realization that not a single
application can fill all the user needs. We wanted to cover the teacher
hosting a gallery of a dozen notebooks on his course web page, as well
as the software community hub hosting thousands of user-contributed
notebooks.

\hypertarget{use-cases}{%
\subsubsection{Use cases}\label{use-cases}}

Planetaryum is a modular Python library that can be used to build many
different applications with a few lines of code. Some of the most
requested applications are already bundled in the library and shipped as
a command-line executable.

Here we present a few possible use cases for Planetaryum.

\begin{enumerate}
\def\labelenumi{\arabic{enumi}.}
\item
  \textbf{Static collection.} A (small) collection of notebooks can be
  used to generate a static website, based exclusively on HTML and
  JavaScript, and thus requiring very few resources for hosting. The
  collection is searchable by keywords, and the appearance is
  customizable. The website generation can be automatized through
  continuous integration tools, as in the documented example\footnote{\url{https://github.com/OpenDreamKit/planetaryum-example-static}.},
  where, from a Binder-ready GitHub repository containing notebooks, we
  automatically generate and host on GitHub pages a static view thanks
  to Travis CI.
\item
  \textbf{Medium sized collection, contributions via PR.} This model is
  suited for small to medium collections of notebooks where it is
  expected that the submission flow will be low and reserved to power
  users. It has the same advantages as the static collection, but at the
  same time it allows contributions, and can optionally be paired with a
  full-text search engine for better exploration.
\item
  \textbf{Large collection, user uploads.} This is a full fledged
  application, backed by a database and a full-text search engine. It
  features filtering, user voting, and potentially other advanced
  features such as recommendation. All the build and deploy steps are
  controlled from the planetaryum executable. It is very similar in
  spirit to \emph{nbgallery}\footnote{\url{https://github.com/nbgallery/nbgallery}.},
  but it is built with the same components as the other applications.
\end{enumerate}

\hypertarget{design}{%
\subsubsection{Design}\label{design}}

Planetaryum has a modular design, leading to many different types of
applications. Its main components are:

\begin{itemize}
\item
  \textbf{Readers} are responsible for reading a collection of notebooks
  from a medium (e.g., folder, git repository, \ldots{}),
\item
  \textbf{Extractors} are responsible for parsing and transforming the
  output of a reader to a data stream.
\item
  \textbf{Builders} take a data stream and produce an output (e.g., they
  populate a database or write files to disk); they can be chained to
  produce many effect at once (e.g., in a full stack application they
  both populate the database and write the front end files.
\item
  \textbf{Front ends} are client side (HTML, JavaScript) applications
  that take the outputs of a builder and produce a user interface. The
  library contains a few very basic default front ends, but they are
  really intended to be developed as separate applications, with their
  own build chain, invoked by a dedicated builder at build time.
\item
  \textbf{Apps} take all the above elements and link them together in a
  unique app with a well defined scope. Planetaryum provides a few
  default apps, but the user is free to write its own.
\item
  The \textbf{CLI} is a simple command-line interface that permits
  invoking the apps and passing parameters to them (through the command
  line, or through a configuration file).
\end{itemize}

\hypertarget{limitations}{%
\subsubsection{Limitations}\label{limitations}}

Despite its flexibility, Planetaryum has the obvious limitation of only
supporting Jupyter notebooks. Although this reduces its scope, we think
the choice was necessary to make a product that the user would find easy
to understand and attractive.

Other limitations, such as not supporting JavaScript-less browsers, are
purely technical and could be lifted if there is enough demand.

\hypertarget{conclusion}{%
\subsection{Conclusion}\label{conclusion}}

Planetaryum fulfills and surpasses the original goal of having a tool
for maintaining community-curated collections of resources on
mathematical software.

We have come to it through a long process of trial and error, that has
considerably delayed the deliverable. Because of this it is hard, for
the moment, to measure its impact, but we are optimistic on its adoption
by the concerned communities.
