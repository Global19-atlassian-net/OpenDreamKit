\begin{event}{Conference-school on Discrete Mathematics and Computer Science 2015}{2015DIMACOS}{University of Sidi Bel Abbès, Algeria -- 15-19 November 2015}{UB}{30}{https://www.univ-sba.dz/ldm/dimacos/}

\textbf{Main goals.} DIMACOS 2015 is a mathematical conference that took place in
Algeria. The first goal of ODK was to make an initiation on SageMath.
The second goal was to create a team of sage developers in Algeria.

\textbf{ODK implication.} A. Boussicault was sent by ODK to deliver
training sessions on
Sage Math. ODK paid the travel and accommodation of A. Boussicault.

\textbf{Event summary.}
This was a one-week event, with a two hour session per day.
The sessions were conducted and prepared by
an Algerian researcher (Professor A. Belahcene).
A Linux installfest on laptops took place every evening.
The evening sessions were beginning at 20h30 and ended at midnight.

The sessions were presented by A. Boussicault (University of Bordeaux) and
Z. Chemli (University of Paris-Est). The purpose was to present and make
mathematical calculus with Sage. The sessions covered also background
in Python and Combinatorics.

During that event, we discussed with professor H. Belbachir
(University USTHB in Alger) and professor I. Boudabbous (University of
SFAX in Tunisia) to plan another Sage event in the conference
"Combinatoire, Algèbre et Théorie des Nombres" in Monastir - Tunisia.


\textbf{Demographic.} 30 participants were present at the sessions.
DIMACOS being an international conference, the people present came from Lebanon, Algeria, Tunisia, Morocco, etc.

\textbf{Results and impact.}
This event allowed us to work with Imad Eddine Bousbaa, a PhD student.
He helped us during the Sage sessions.
It was the starting point of a collaboration that allowed us to recruit him in
the ODK project.

His recruitment is part of the will to build a team of Sage developers in Algeria.

We could use this event to prepare the next conference : "Combinatoire, Algèbre et Théorie des Nombres"
in Tunisia (event \ref{event-2016CATN}).


\end{event}
