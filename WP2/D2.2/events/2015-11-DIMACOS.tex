\begin{event}{Conference-school on Discrete Mathematics and Computer Science 2015}{2015DIMACOS}{University of Sidi Bel Abbès, Algeria -- 15-19 November 2015}{UB}{30}{https://www.univ-sba.dz/ldm/dimacos/}

\textbf{Main goals.} DIMACOS 2015 is a mathematical conference that took place in 
Algeria. The first goals of ODK was to make an initiation on SageMath. 
The second goal was to create a team of sage developer in Algeria.

\textbf{ODK implication.} A. Boussicault was sent by ODK to give a lesson on
Sage Math. ODK paid the travel and accommodation of A. Boussicault.

\textbf{Event summary.}
Each intervention was 2 hours long. There were one intervention by day during 
a week.
The lessons were conducted on computers and the computer were prepared by 
Algerian researchers (Professor A. Belahcene).
All the evening, we made Linux installfests on laptops. 
The evening sessions were beginning at 20h30 and ended at midnight.

The lessons were presented by A. Boussicault (University of Bordeaux) and 
Z. Chemli (University of Paris-Est). The purpose was to present and make 
mathematical calculus with Sage. We made also some python lessons and 
Combinatorics lessons.

During that event, we discussed with professor H. Belbachir 
(University USTHB in Alger) and professor I. Boudabbous 
(University of SFAX in Tunisia) to plan another Sage Math event in the 
conference : the conference "Combinatoire, Algèbre et Théorie des Nombres" 
in Monastir - Tunisia. 


\textbf{Demographic.} 30 particpants were present for the lessons. 
DIMACOS is an international conference. 
The people present came from Lebanon, Algeria, Tunisia, Morocco, etc.

\textbf{Results and impact.} 
This event allowed us to work with Imad Eddine Bousbaa, a PhD student. 
He helped us during the sage lessons.
It was the starting point of a collaboration that allowed us to recruit him in
the ODK project.

His recruitment is part of the will to mount team of Sage developer in Algeria.

We could use this event to prepare the next conference : "Combinatoire, Algèbre et Théorie des Nombres"
in Tunisia (event \ref{event-2016CATN}).


\end{event}
