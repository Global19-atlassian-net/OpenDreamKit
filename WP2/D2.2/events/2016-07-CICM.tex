\begin{event}{CICM 2016}{CICM2016}{July 25-29, Bialystok, Poland}{PS, JU}{around 50}{http://www.cicm-conference.org/2016}

\textbf{Main goals.} Since a decade, CICM collocates several workshops: Calculemus
(automated reasoning systems and Computer algebra), DML (Digital Math
Library), MKM (Mathematical Knowledge Management), AISC (Artificial
Intelligence and Symbolic Computation) with the deliberate strategy to
bring together a diverse crowd of people. In particular, the Tetrapod
workshop is meant to exchange ideas between people involved in the
four main areas of *mechanized mathematics*: computation, data,
knowledge management, and deduction (proof systems).

\textbf{ODK implication.} Five \ODK participants attended the conference, Michael Kohlhase
from \site{JU} as General chair together with Florian Rabe, Tom Wiesing, and Dennis Müller
from his group, and Nicolas Thiéry from \site{PS} as invited
speaker~\cite{Thierry:igcscac16}. They delivered several talks, including about the WP6y
paper~\cite{DehKohKon:iop16}, and an MMT tutorial~\cite{RabIanMue:ldm16,RabIanMue:adm16}.

\textbf{Results and impact.} CICM was a prime occasion to advertise
and brainstorm with the larger community about the WP6 advances, and
get very interesting feedback. For details, see this
\href{http://opendreamkit.org/activities/2016-08-01-CICM/}{blog post}.
\end{event}
