\begin{event}{Atelier PARI/GP 2016}{AtelierPARI2016}{Grenoble (FR),
2016-01-11 to 2016-01-25}{PS,UB,UV,UW}{36}{http://pari.math.u-bordeaux.fr/Events/PARI2016/}

\textbf{Main goals.}

The PARI/GP Ateliers were established in 2012 as a yearly meeting
between developers and users of the PARI/GP system.

The main goals are advertising new features and improvements,
discussing further developments, sharing best practices, and collaborative
code writing (hacking sessions, doc reviews, bug-squashing parties).

You can find the list of previous PARI Ateliers at
\url{http://pari.math.u-bordeaux.fr/ateliers.html}

\textbf{ODK implication.} 
%Describe how ODK was involved and give a rough estimation of cost for ODK

OpenDreamKit participants: B. Allombert, K. Belabas, J. Demeyer, J.-P. Flori,
L. de Feo, as well as Aurel Page from the Warwick group (LMFDB).

OpenDreamKit provided the main funding source for the workshop (accommodation,
subsistence and travel expenses), for about 15k\euro. ERC Starting Grant
ANTICS, and the LabEx PERSYVAL-Lab co-funded the event.

\textbf{Event summary.} 
%Give a summary of your event

The 6th Atelier PARI/GP took place in Grenoble (France) from january
11th to 15th.

There were 36 registered participants from 16 different institutions
(no registration fees).

A typical day of the workshop had introductory talks and tutorials
in the morning; afternoons allowed ample time for hacking sessions,
discussions and training.

The Atelier featured 10 morning talks on

\begin{itemize}
\item mathematical topics and implementation projects : modular forms,
    L-functions, polylogs \& multizeta values,

\item packages and interfaces : PARI Jupyter notebook, a number field database,
    an elliptic curve library for cryptography, CADO-NFS, GIAC/XCAS,
    parallel programming with GP2C.
\end{itemize}

Slides and videos for all talks are available at
\url{https://www.youtube.com/playlist?list=PL0E0n75oNCDnWuydCHepxxSRc4UbtQQ}

\textbf{Results and impact.} 
% What did you achieve with this event? (If ever it impacted 
% other ODK tasks and deliverables, mention it here)

The workshop was very productive and was particularly beneficial to WP4 (user
interfaces) and WP5 (high-performance computing):
\begin{itemize}
\item it was a major boost to PARI/GP development; feebackd received allowed
the release of PARI/GP-2.8 in august 2016, a major release after two years of
development; (T2.3, development workshops)

\item issues related to the new PARI Jupyter notebook (D4.4, D4.7) and
Sage/PARI interaction were ironed out during the meeting; discussions related
to PARI parallelisation engine (D5.10)

\item the PARI developers learnt about technologies and created resources for
online GP deployment during the meeting using the \emph{emscripten} compiler,
see e.g.~\url{http://pari.math.u-bordeaux.fr/gp.html}.
\end{itemize}

\end{event}
