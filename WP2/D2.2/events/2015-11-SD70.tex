\begin{event}{Sage Days 70}{sd70}{Berkeley (US California), 2015-11-08 to 2015-11-14}{PS}{16}{https://wiki.sagemath.org/days70}

\textbf{Main goals.} Gather developers from Sage, SageMathCloud and Jupyter together to learn the
inner machineries of the different projects and code together towards common goals.

\textbf{ODK implication.} This event was coorganized by ODK which
cofunded the participation of two ODK members and another European
associate.

\textbf{Event summary.} The event featured many interesting talks on the inner mechanics of
both SageMathCloud and Jupyter, in particular:
\begin{itemize}
\item \href{https://youtu.be/GOuy07Kift4}{How to contribute to SageMathCloud} by William Stein
\item \emph{The PARI Jupyter kernel} by Jeroen Demeyer
\item \emph{Jupyter Notebook development} by Jason Grout.
\end{itemize}
Lots of time was devoted to projects and code such as: installing a development version of SageMathCloud,
following tutorials on SageMathCloud development, working toward the integration of the Jupyter notebook
in Sage.

Furthermore a Jupyter interface for HPC-GAP was developed, and the Jupyter
interface for GAP was improved. A talk \emph{The current status of (HPC-)GAP}
was contributed.

\textbf{Results and impact.} 
This workshop was essential to some ODK planned tasks. This was especially related to WP3 and WP4. Here are some tasks that
were started during the Sage Days:
\begin{itemize}
\item \longtaskref{component-architecture}{extract-smc} Document and modularize SageMathCloud's codebase. This task was started during the workshop using the 
knowledge of the main developer of SageMathClod, William Stein.

\item \longtaskref{UI}{ipython-kernels} Uniform notebook interface for all interactive components. This is a major task of WP4. This workshop
was an occasion to share first hand information between Sage, GAP, and Jupyter developers.

\end{itemize}
The knowledge we gathered during presentations was relevant to all tasks including notebook interfaces and cloud
systems.

\end{event}
