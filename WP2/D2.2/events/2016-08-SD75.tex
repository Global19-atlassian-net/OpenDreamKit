\begin{event}{Sage Days 75: Coding theory}{sd75}{Cernay la Ville (France), 2016-08-22 to  2016-08-26}{PS, UV, UJF}{30}{https://wiki.sagemath.org/days75}

\textbf{Main goals.} The event was organized primarily by the Inria project
Actis, to celebrate the termination of its two year lifetime period. The purpose
of the project was a major redesign and implementation of the coding theory
features of SageMath. Hence the workshop gathered researchers from coding
theory, and related topics, including cryptography, group theory, combinatorics,
and linear algebra.
The goal was to expose the results of the project to the community ensure its
proper integration into the main frame of the software, and initiate new
projects, so that its development would carry over with, after the end of the
Actis engineer position

\textbf{ODK implication.} the event was co-organized by ODK (through Clément
Pernet, UJF) and Inria's Actis project. The event costed around 5000\euro (including 300\euro for ODK).
A short presentation about ODK was made during the conference to present the project to the participants.

\textbf{Event summary.} We started the event by some introduction presentations and tutorials so that
the participants would familiarize themselves with Sage. Then the time was shared between lectures
and coding sprints.

The full program can be found on the \href{https://wiki.sagemath.org/days75}{website}.

\end{event}
