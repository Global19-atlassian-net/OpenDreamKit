\begin{event}{Sage Days 75}{sd75}{Cernay la Ville (France), 2016-08-22 to  2016-08-26}{PS, UV, UJF}{30}{https://wiki.sagemath.org/days75}

\textbf{Main goals.} The event was organized primarily by the Inria project
Actis, to celebrate the termination of its two year lifetime period. The purpose
of the project was a major redesign and implementation of the coding theory
features of SageMath. Hence the workshop gathered researchers from coding
theory, and related topics, including cryptography, group theory, combinatorics,
and linear algebra.
The goal was to expose the results of the project to the community ensure its
proper integration into the main frame of the software, and initiate new
projects, so that its development would carry over with, after the end of the
Actis engineer position

\textbf{ODK implication.} the event was co-organized by ODK (through Clément
Pernet, UJF) and Inria's Actis project. The event costed around 5000eur (including 300eur for ODK).
A short presentation about ODK was made during the conference to present the project to the participants.

\textbf{Event summary.} We started the event by some introduction presentations and tutorials so that
the participants would familiarize themselves with Sage. Then the time was shared between lectures
and coding sprints.

The full program can be found on the website \url{https://wiki.sagemath.org/days75}. 

\textbf{Results and impact.} 
\begin{itemize}
\item \textbf{New comers got to use Sage for the first time} around one
third of the participants had zero or very little experience with Sage before the meeting. By the
end of the three days, everyone had a way to use Sage on their machines and had written a bit of code.
\item \textbf{New comers got to contribute to Sage}: a lecture was given on how to contribute to Sage
and groups were formed on different projects mixing more experienced people with new comers so
that the code that was written could end up being merged to the software. 
\item Formal thematic talks were given on the new coding theory component of Sage (David
  Lucas), exact linear algebra (Clément Pernet), Algebraic Coding Theory (Johan Rosenkilde) and Rank metric codes (Arpit
  Merchant), and other less formal, or lightning talks were organized upon
  request by participants.
\item \textbf{New contributions were made in the coding and linear algebra components of Sage}: we used the keyword
\textbf{days75} on the trac server of Sage to track the contributions that were submitted during the workshop.
 Altogether the participants
worked on 35 different tickets either reviewing existing ones, implementing, or
creating new tickets. 12 of them already got positive reviews and are on the process of being merged to the software.
\end{itemize}




\end{event}
