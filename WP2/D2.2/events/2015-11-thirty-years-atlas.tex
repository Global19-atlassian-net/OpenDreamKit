\begin{event}{Finite Simple Groups: Thirty Years of the Atlas and
    Beyond}{tyatlas}{Princeton (US California), 2015-11-02 to 2015-11-05}{SA}{94
    }{http://math.arizona.edu/~grouptheory/princeton/}

\textbf{Main goals.} Gather contributors to and users of the ``Atlas of Finite
Simple Groups'' and other mathematical databases to learn about past, presence,
and future.

\textbf{ODK implication.} ODK was not the main organizer of this event, it was used to fund
one project member (Markus Pfeiffer).

\textbf{Event summary.} The event featured talks by high-profile mathematicians,
such as John H. Conway, John Thompson, Michael Aschbacher, and many more.

Discussion sessions highlighted the need for mathematical knowledge
stored in databases. Some major examples that were discussed are
\begin{itemize}
  \item The ``ATLAS of Finite Group Representations - Version 3''
    http://brauer.maths.qmul.ac.uk/Atlas/v3/
  \item The ``Online Encyclopedia of Integer Sequences (OEIS)''
    http://oeis.org
  \item The ``Modular forms and L-functions database (LMFDB)''
    http://lmfdb.org
  \item The ``Small Groups Database'' (small)
\end{itemize}

Most of these databases share common issues such as \emph{reliability} of the
data, the \emph{reliability} and \emph{longevity} of the storage,
\emph{maintenance}, and \emph{managing contributions}.
    
\textbf{Results and impact.} 
The attendance of this conference shed light on how some mathematicians view
mathematical databases, and what issues they see. This is an important
contribution for WP6. It also contributed to the attendee's understanding of the
needs of our potential users (WP4).

\begin{itemize}
\item \longtaskref{dksbases}{data-assessment} Survey of existing DKS bases, Formulation of Requirements
\item \longtaskref{dksbases}{mws} Math Search Engine

\end{itemize}

Further discussions with GAP users and developers about HPC-GAP were a
side-effect of the attendance of this event. 

\end{event}