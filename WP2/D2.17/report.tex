\documentclass{deliverablereport}

\deliverable{dissem}{IntroODK}
\deliverydate{04/09/2019}
\duedate{31/08/2019 (M48)}
\author{Mike Croucher, Hans Fanghor and Nicolas M. Thiéry}

\begin{document}
\maketitle
\enlargethispage{.5cm}
% This will be the abstract, fetched from the github description
\githubissuedescription
\clearpage
\tableofcontents


\section{Dissemination and evaluation of activities}

\subsection{Dissemination events}

TODO: Check that this was reported on in the events deliverable and
remove

In 2017, we organised a comprehensive workshop on Computational
Mathematics With Jupyter in Edinburgh, targetting academics, researchers
and teachers at the same time. Topics delivered by OpenDreamKit members
included Jupyter, nbdime, nbval, Docker containers, JupyterHub. We
managed to engage external contributors, such as Christian
Lawson­Perfect from Newcastle reporting on the Numbas web-based
e-assessment system for mathematics and Mark Quinn discussing the use of
SageMathCloud for teaching purposes.

We contributed a workshop on ``Jupyter Notebooks for reproducible
research'' at the 2017 International Research Software Engineering
conference in the UK. While a lot of this work may appear as UK
-centric, this did have dissemination effects across Europe as the
emergence of the research software engineering idea and community was
driven in the UK; researchers across Europe were closely monitored and
participating in these events.

\subsection{Teaching with Jupyter}

Many -- if not most -- of the OpenDreamKit participants engaged actively
in using and testing Jupyter technologies in their daily teaching
duties.

Here are a few striking examples:

\begin{itemize}
\item
  OpenDreamKit experts on Jupyter delivered multiple courses (topic?) at
  the NGCM Summer School 2016 and 2017 at Southampton, which attracted
  many PhD students and some researchers from the UK, Europe and some
  from overseas.
\item
  Paris Sud participants used Jupyter in a variety of classes (graph
  theory, computational combinatorics, computational algebra,
  experimental mathematics, numerical analysis) throughout the math and
  computer science curriculum. An highlight is a 400 students
  introductory programming in C++ class. Having a uniform user interface
  across systems (C++, Python, SageMath, \ldots) was a major selling
  point by enabling students to be immediately productive in their new
  environment. The teaching was supported by the deployment of a local
  JupyterHub server and the occasion to experiment with various tools
  including the notebook converters nbsphinx and nbconvert, the C++
  interactive interpreter xeus-cling, Jupyter widgets, interactive web
  pages with ThebeLab, assignement handling and grading with nbgrader.
  This led to an invitation of the course leader to a workshop in
  Edinburgh to share experience and participate to an nbgrader coding
  sprints where several contributions were submitted and accepted.
\item
  Gent (TODO: Jeroen)
\item
  UVSQ (TODO: Luca)
\item
  Silesia (TODO: Marcin + refer to interactive book)
\item
  St Andrews (TODO: Alex)
\item
  Sheffield OpenDreamKit participants spearheaded a transformation in
  the way computation was taught at The University of Sheffield across
  multiple subjects and at many levels. The Department of Physics, for
  example, now teaches \emph{all} of its undergraduates how to program
  in Python using Jupyter notebooks and the CoCalc (Formerly
  SageMathCloud) cloud-based environemnt. Furthermore, many other
  physics modules at Sheffield now include some type of computation
  using the same technologies. Programming is no longer considered
  separately but as the integral part of modern science that it really
  is.

  Working with an Italian Marie-Curie fellow in Bioinformatics,
  Sheffield's OpenDreamKit participants developed a set of short
  postgraduate `Bioinformatics Awareness Days' which used OpenDreamKit
  technology to introduce Bioinformatics workflows to clinicians at
  Sheffield's Institute for Translational Neuroscience. This further led
  to an OpenDreamKit teaching tutorial being held at University of
  Naples Parthenope.

  Other subject areas where Sheffield's OpenDreamKit participants
  assisted in developing enhanced lecture material using OpenDreamKit
  technologies include Machine Learning, Mathematics, Computer Science,
  Biology and High Performance Computing.
\end{itemize}

\section{Teaching material}

Interactive lecture notes are an area where commercial vendors such as
MapleSoft and Wolfram Research are spending a lot of time and money
developing material. Within the Jupyter ecosystem it has become possible
to author interactive lecture notes and make them openly available
(e.g.~through BinderHub). We have created such interactive lecture
materials, used them in university education, and made them openly
available on the Internet. This includes four interactive textbooks (see
also D2.9 and D2.14), but also Software Carpentry lessons, course notes,
etc.

Anecdotal evidence and feedback from individual users (see D2.14) shows
that they are used outside the OpenDreamKit partners, in and out of
Europe, both by individual students and university lectures.

\subsection{Local consulting}

Across all the OpenDreamKit partner sites, OpenDreamKit staff and PIs
have engaged with colleagues, students and decision makers to advocate
the benefits of the Jupyter research environment; often effectively
serving as consultants for best practice computational mathematics and
science tools and workflows.

At Sheffield, taster seminar (1-2 hours) and follow-up short course (1-2
days) on Jupyter for lecturers and researcher were organised targeting
and engaging science and engineering disciplines beyond mathematics as
the Jupyter ecosystem of tools is of value for teaching and research in
any discipline having to work with computational and data based
research. These workshops were integrated into the research support
services of the Sheffield IT department, and integrated with the
research software engineering movement that originated in the UK over
the previous years.

The OpenDreamKit funding was essential in forming Sheffield's Research
Software Engineering (RSE) Group, one of the first such groups in the
UK. Providing training, documentation and support in the use of
OpenDreamKit technologies in both teaching and research helped the
Sheffield RSE group demonstrate to the University how vital RSE support
can be. This directly led to the fully-funded, diverse group that now
exists at Sheffield which has served as a model for many other such
groups around the UK, Europe and more, recently, the United States.

\section{Contributions}

At Southampton, OpenDreamKit developed and enhanced tools such as
Jupyter, nbval and nbdime were integrated into the taught Masters
programme of the national Centre for Doctoral Training in Next
Generation Computational Modelling (NGCM).

The integration of Sun Grid Engine and Project Jupyter, done as part of
OpenDreamKit (add deliverable XXX), has led to some educators
considering using the notebook to introduce various aspects of High
Performance Computing. The use of notebooks at Sheffield, Southampton
and other universities has grown significantly. Based on the
e-infrastructure work of OpenDreamKit and our dissemination activities,
the notebook is emerging as the de-facto standard in the symbolic
mathematics domain.

Our development and dissemination activities on nbval and nbdime (XXX
add deliverable) within OpenDreamKit had resulted in researchers
switching to using the notebook for all of their software documentation
and tutorials. The unique selling point of nbval is to make it possible
to ensure that notebook-based documentation will always work as
development of code progresses and the documentation is in danger of
becoming outdated: Using nbval allows the documentation to become part
of the formal testing and continuous integration framework.

\begin{itemize}
\tightlist
\item
  Tania's course template
\item
  Marcin's bookbook cookie-cutter
\end{itemize}

\section{Meta discussion}

\subsection{Evaluation and Adoption}

In this section, we build on all the gathered witnessing to reflect on
how each tool we have promoted appeared adequate for the users needs and
adopted by the community

\subsubsection{The Jupyter notebook}

Practice over hundreds of students in class or other learners at
dissemination workshops have shown that it takes no more than an hour of
guided instruction to become sufficiently acquainted with the basics of
the Jupyter notebook to be able to autonomously explore properly
structured collections of notebooks (e.g.~an interactive text-book, or a
collection of tutorials in a workshop). At this stage, the simple,
linear narrative structure of a notebook is a precious guide to the
reader. It does take practice however for the user to not get any more
confused by the potential disprecancy between the visual order and
execution order of cells. This can be mitigated by short notebooks, and
also the use of notebook extensions that enforce the execution order of
cells.

Properly authoring one's own notebooks also takes a lot of practice. The
simplicity of incrementally building code by testing code snippets and
aggregating them is a big strengh of the notebook. It's also a weakness,
as it encourages \ldots{} leads to bloated messed up notebooks.

TODO: nthiery

\subsubsection{Jupyter widgets}

\subsubsection{3D visualization}

\subsubsection{Computational systems}

\subsubsection{Converting tools}

nbconvert, sphinx, \ldots{}

\subsubsection{Binder}

\subsubsection{Version control nbdime}

\subsubsection{Validation with nbval}

\subsubsection{Installation with Conda}

\subsubsection{VRE access / deployment}

JupyterHub

at EOSC

\subsubsection{Class management}

There are various methods for managing classes using OpenDreamKit
technologies and OpenDreamKit participants have tried most of them. None
of them are perfect and many non-specialist lecturers require support in
choosing and using the various options which include:

\begin{itemize}
\tightlist
\item
  Commercial Cloud-based systems such as Microsoft Azure and CoCalc.
  These are very easy to set up and use but remove a lot of control from
  educators. Issues include updates occuring in the middle of exam
  sessions that modified user-interface behaviour and even computational
  results in some cases. For one computationally intense course, the
  amount of CPU power available in the cloud environment was
  insufficient for students to complete some project work which led to
  complications.
\item
  Bring your own laptop. Such sessions ensure that participants leave
  teaching sessions with a fully functioning computational environment
  but supporting diverse operating systems and hardware can be extremely
  challenging for those running the course.
\item
  Servers ran by University IT. Can provide a very controlled and stable
  environment but at the cost of teachers not being able to update in a
  timely fashion unless the servers are supported by dedicated Research
  Software Engineerign staff.
\end{itemize}

When using this type of technology across many subjects in a University,
it quickly becomes apparent that the only way to fully support lecturers
properly is to have on-site specialists that can provide the requsite
assistance. Such work forms the foundation of the Research Software
Engineering groups that have started to form all over the world.
OpenDreamKit participants formed the first demonstrations of such
partnerships

nbgrader is a tool for managing assignments in a class: authoring,
production and distribution of instructor and student version,
collection, semi-automatic grading, feedback. It is modular and each
piece can be used either from a graphical user interface (UI) within
Jupyter or from the commandline. This gives quite some flexibility to
integrate it into the local information system and workflow.

Experience shows that usage through the UI is straightforward, for
students and instructors alike, even novices. However, as of now,
setting it up requires some computer literacy, if not admin right on the
local system. The tricky part is to setup the \emph{exchange zone}
through which assignments are exchanged back and forth between
instructor and students. Support for multiple courses and
multi-instructor courses is also very recent. We expect the entry
barrier to decrease greatly with the ongoing development of an
\emph{exchange service}; once deployed locally at an institution and
integrated with the local information system, instructors will be able
to seamlessly setup new courses.

\subsubsection{Interactive documents}

TODO: reference to section 4 of D2.14

\subsubsection{EOSC}

All technologies developed in this work-package can be transferred to
the European Open Science Cloud in fairly straightforward manner, as
they are using open, standard, and modern web technologies.

\subsubsection{???}

\subsection{Reflection about ``small contributions'' making large  impact}

was it efficient?

\begin{itemize}
\tightlist
\item
  feedback loop
\item
  not easy to fund
\item
  not easy to find people who want to do this
\item
  the value of what we disseminated was for many participants dominated
  by learning about the basics of the tools rather than the more
  advanced extensions implemented through OpenDreamKit
\end{itemize}

\section{Stuff that could be added}

\subsection{Evaluation and contribution to marking systems}

\begin{itemize}
\tightlist
\item
  @nthiery: could write 2-3 subsubsections about use of nbgrader for Info
  111 + invitation to workshop + a few pull requests
\end{itemize}

\subsubsection{Somewhere: a brief story about Sheffield stoping and the project taking over the dissemination activities}

\end{document}

%%% Local Variables:
%%% mode: latex
%%% TeX-master: t
%%% End:

