\begin{event}{2018 Leeds: Introduction to reproducible workflows in Python}{leeds_repro_python}{University of Leeds, June 14th 2018}{LEEDS}{20}{1}{http://arc.leeds.ac.uk/training/spc-1-introduction-to-reproducible-workflows-in-python/}

\textbf{Main goals.} To introduce researchers at University of Leeds to good practice in reproducible research.

\textbf{ODK implication.} ODK member, Tania Allard, reused material developed by her for PyCon 2018 for this one day event.

\textbf{Event summary.} This is an introductory course to reproducible analysis workflows in Python. It is aimed at people with some experience in Python for data analysis or computational research (e.g. people already developing scripts or using Jupyter notebooks). By the end of the course the attendees will have learnt about best practices for reproducible scientific code development and should be able to implement these techniques to their day to day workflows.

Materials at \url{https://github.com/trallard/ReproduciblePython}

\textbf{Results and impact.} Around 20 researchers received training.

\end{event}
