\begin{event}{Atelier et \'Ecole MathExp 2018}{MathExp2018}{Saint-Flour (FR),
2018-05-21 to 2018-06-01}{UB}{36}{3}{https://mathexp2018.sciencesconf.org/}

\textbf{Main goals.}

The MathExp school and atelier organized in Saint Flour was a unique
opportunity for young mathematicians to learn about computer science
and the tools developed in the framework of \ODK. The event
was divided in two weeks. The first one focused on 4 courses and
introductory tutorials with SageMath and Jupyter. During the second
week the participants were asked to develop programs related to their
own research projects.

\textbf{ODK implication.} 
%Describe how ODK was involved and give a rough estimation of cost for ODK

\ODK organizer: V. Delecroix

\ODK provided the main funding source for the workshop (accommodation,
subsistence and some of the travel expenses) for about 40K\euro. There were
also inscription fees and the event was cofounded with the CNRS.

\textbf{Event summary.} 
%Give a summary of your event

The school and atelier MathExp took place in Saint-Flour (France)
from May 21st to June 1st.

There were 22 registered participants.

A typical day of the school consisted of 2 courses and tutorials
on computers. During the atelier, participants worked on their
own research projects asking for help when needed.

The School featured 4 courses on computer science topics directly
related to mathematical computations: probability (Ana Bušić (Paris, FR)),
linear programming (Xavier Goaoc (Marne-la-Vallée, FR)), formal computation
(Bruno Salvy (Lyon, FR)) and backtracking techniques (Michaël Rao (Lyon, FR)).

\textbf{Results and impact.} 
% What did you achieve with this event? (If ever it impacted 
% other ODK tasks and deliverables, mention it here)

We were delighted to achieve a gender equidistribution among the participants
(10 female and 12 males) which is not often in Mathematics and Computer
Science.

The school and workshop were very productive and beneficial to disseminate
the work achieved in the various work packages. In particular
all the work around SageMath and Jupyter.

\end{event}
