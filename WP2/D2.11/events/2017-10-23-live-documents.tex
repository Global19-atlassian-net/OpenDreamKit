\begin{event}{Live Structured Documents}{live-documents}{Location and date}{SR,PS,FAU}{11}{6}{https://www.eventbrite.com/e/opendreamkit-workshop-on-live-structured-documents-registration-37364670736}

\textbf{Main goals.} To facilitate development of live structured documents, using Jupyter infrastructure, such as interactive documentation and publication.

\textbf{ODK implication.}
The workshop was hosted at \ODK site, Simula Research Laboratory in Oslo, Norway. The only cost to \ODK was travel reimbursements to bring five participants to Oslo, Norway for three days.

\textbf{Event summary.} The event was a workshop gathering \ODK participants and others interested in interactive documents of various kinds.
Participants discussed available technologies and goals,
and worked together to build tools in this area,
especially collaborating to integrate across projects with different stakeholders.

\textbf{Results and impact.}
One of the main impacts of the workshop
was the creation of a new package, \href{https://github.com/minrk/thebelab}{thebelab},
which builds on \Jupyter technology
to add interactive code execution to any webpage,
with execution running on \href{https://mybinder.org}{mybinder.org} or a local \Jupyter server.
During the workshop, thebelab was put to use in \href{https://more-sagemath-tutorials.readthedocs.io/en/latest/}{documentation for
SageMath} and tested with documentation for both Singular and GAP.
This serves WP4 by adding the possibility of interactivity to documentation for projects in the \ODK community,
and benefiting from kernels for GAP and other \ODK projects developed in T4.1,
showing the value of integrating \ODK systems into a shared \Jupyter ecosystem.

\end{event}
