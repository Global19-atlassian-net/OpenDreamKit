\begin{event}{2017 International Research Software Engineering Conference}{SHF_RSECONF}{Manchester MOSI, 7th and 8th September 2018}{USH}{200}{3}{https://software-carpentry.org/blog/2018/01/rse-conf-repost.html}

\textbf{Main goals.} The second international Research Software Engineering (RSE) conference took place on the 7th and 8th of September 2017 at the Museum of Science and Industry (MOSI). There were over 200 attendees, 40 talks, 15 workshops, 3 keynote talks. This is the largest event in the Research Software Engineering calendar.

\textbf{ODK implication.} ODK members had significant involvement in this event. Sheffield's Tania Allard developed and delivered a workshop called 'Jupyter notebooks for reproducible research’ which demonstrated and taught how technologies developed in ODK significantly improved how computational research across all fields could be made more reproducible.  The workshop was one of the most popular workshops at the event and had to be delivered twice due to high demand. The only workshop with higher demand was Microsoft's introduction to Azure Cloud which included several hundred dollars of free cloud time per participant.

Workshop materials are freely available at https://github.com/trallard/JNB_reproducible

Tania also served as Diversity Chair on the RSE conference committee.

Fellow Sheffield ODK member, Mike Croucher, gave an invited keynote speech at the event called 'I, Research Software Engineer'. Slides at https://mikecroucher.github.io/RSE_2017_keynote_presentation/

\textbf{Event summary.} Research software engineers are the people who support computational research across all of academia. Influencing this group has the potential to influence how a large subset of computational research is performed. 

\textbf{Results and impact.} The event was incredibly successful for ODK since it introduced a significant proportion of the RSE community to ODK-developed tools.  It also led to invitations to deliver more workshops and talks at other prominent events.

\end{event}
