\begin{event}{JupyterHub and Binder coding sprint}{JupyterHub}{Université Paris Sud, March 7-8}{PS,SR,LL}{8}{4}{https://pad.unistra.fr/p/jupyterhub-dev}

  \textbf{Main goals.} This event was organized as a satellite of a
  Jupyter dissemination conference at École Polytechnique. It brought
  together developers of the Virtual Environments JupyterHub and
  BinderHub and DevOps working on deployments in Orsay and at EGI for
  a coding sprint. The goal was to improve those deployments, as a use
  case from which to learn and share procedures and best practices.

  \textbf{ODK implication.} Organization and funding of ODK participants; $\approx$ 1k\euro.

  \textbf{Results and impact.} Turnkey deployment instructions already
  existed for deploying JupyterHub and BinderHub on top of cloud
  infrastructure provided by e.g. Google Cloud, or Microsoft Azure.
  This workshop was a major step for the first deployment of a
  JupyterHub and BinderHub instance on top of an OpenStack cloud
  infrastructure. This is important as OpenStack is widely adopted in
  academia-run cloud infrastructures, yet raises some unique
  challenges due to its high customizability. Many notes were taken
  and shared. In addition a blog post summarized a brainstorm on the
  upcoming convergence between JupyterHub and BindherHub, toward
  providing versatile JupyterHub deployments that lets its user
  define, run, and share virtual environments equipped with an
  arbitrary software stack.

  \url{https://opendreamkit.org/2018/03/15/jupyterhub-binder-convergence/}
\end{event}
