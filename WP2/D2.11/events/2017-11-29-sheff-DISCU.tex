\begin{event}{Diversity and Inclusion in Scientific Computing}{SHF_DISCU}{New York, 29th and 30th November 2018}{USH}{50}{1}{https://pydata.org/nyc2017/diversity-inclusion/disc-unconference-2017/}

\textbf{Main goals.} NumFOCUS’s Diversity & Inclusion in Scientific Computing (“DISC”) Program strives to help create a more diverse community through initiatives and programming devoted to increasing participation by and inclusion of underrepresented people.

\textbf{ODK implication.} ODK Sheffield member, Tania Allard was invited to participate in this 2 day event.

\textbf{Event summary.} At the inaugural Diversity and Inclusion in Scientific Computing (DISC) Unconference in November 2017 in New York, one of the topics that came up was organizing inclusive and diverse events and conferences. How to organize such an event? Is there a checklist? Is it hard? Participants decided to build on work begun by the NumFOCUS DISC Committee creating a cookbook for organizing inclusive and diverse events, with the aim of encouraging and supporting such events.

\textbf{Results and impact.} The output and its creation are described at https://numfocus.org/blog/discover-cookbook

\end{event}
