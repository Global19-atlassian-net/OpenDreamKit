\begin{event}{Second CoDiMa Training School in Computational Discrete Mathematics}{codima2016}{ICMS, Edinburgh, Oct. 17 -- 21, 2017}{SA,USH,UW,PS}{38}{6}{https://www.codima.ac.uk/school2016/}

\textbf{Main goals.} This training school was organised for PhD students and researchers from
UK institutions interested in using computational techniques in the area of discrete mathematics.
The aim was to give an introduction into open source mathematical software packages in this area,
at the same time equipping attendees with modern skills that they need for using and developing
research software, and promoting best open science practices.

\textbf{ODK implication.} The school has been organised by Alexander Konovalov (USTAN) and supported
by the EPSRC-funded project CoDiMa (CCP in the area of Computational Discrete Mathematics). Presenters
included Alexander Konovalov, Steve Linton and Markus Pfeiffer (USTAN), Viviane Pons (UPSud),
Mike Croucher (USFD) and John Cremona (UWarwick). 

\textbf{Event summary.}

It started with the hands-on Software Carpentry workshop covering basic
concepts and tools, including working with the command line, version control and task automation,
continued with introductions to GAP and SageMath systems, and followed by the series of lectures
and exercise classes on a selection of topics in computational discrete mathematics. Alexander
Konovalov gave an introduction to GAP in the form of the Software Carpentry lesson. It was
followed by Steve Linton's talk ``How to use GAP effectively'' and Markus Pfeiffer's talks
``Advanced GAP programming'' and ``Pathways to impact : contributing to GAP''.
Viviane Pons gave a SageMath tutorial, and John Cremona presented LMFDB project. 
Other speakers were Christopher Jefferson (St Andrews) with 
``Debugging and profiling in GAP'' and 
``Building Efficient Algorithms on Permutation Groups with Stabilizer Chains''
and Wilf Wilson (St Andrews) with ``Semigroups in GAP: an introduction and tutorial''.
Both Markus Pfeiffer and Wilf Wilson used GAP Jupyter interface in their talks.
The closing activities were the talk ``Is your research software correct?'' by Mike Croucher 
and a panel discussion joined by the director of the Software Sustainability Institute Neil Chue Hong.

\textbf{Demographic.} There were 10 women among 26 learners attending the School.

\textbf{Results and impact.} This event contributed to building UK community
around ODK components, and to dissemination of ODK outputs in the UK.

\end{event}