\begin{event}{Bioinformatics Awareness Days}{SHF_BAD}{University of Sheffield, November 2017 }{USH}{30}{2}{https://bitsandchips.me/BAD_days/}

\textbf{Main goals.} The ecosystem of tools supported by the OpenDreamKit project are suitable for a wide range of scientific disciplines. The development of this workshop, in collaboration with a bioinformatics practitioner, was an experiment in how to apply \ODK supported technologies to the field of Bioinformatics.

\textbf{\ODK implication.} Workshop materials were developed by Tania Allard in collaboration with Statistician and Bioinformatics researcher, Dr Luisa Cutillo.  The materials were released online at \url{https://github.com/trallard/BAD_days}. The workshop was delivered by Dr Cutillo with support from Tania Allard.

\textbf{Event summary.} A group of researchers based within the Sheffield Institute for Translational Neuroscience (SITRAN, \url{http://sitran.org/}) were taught about Bioinformatics workflows using Jupyter notebooks with computation provided by the free Micosost Azure Notebook service.

\textbf{Results and impact.} The event demonstrated that \ODK supported technologies could be applied to the field of Bioinformatics and led to a new collaboration between Dr Cutillo and \ODK member Mike Croucher.

Following the success of this workshop, Dr Cutillo independently taught an introductory workshop on statistics using Jupyter notebooks on Azure at Parthenope University of Naples (Materials at \url{https://github.com/luisacutillo78/RbasicStats})

Dr Cutillo has since moved to University of Leeds where she will be teaching statistics to 200+ undergraduates. She plans to use \ODK developed technologies in collaboration with the Research Software Engineering group at Leeds.

The event required the development of a website that was linked to the Jupyter notebooks (\url{https://bitsandchips.me/BAD_days/}). The website caught the attention of Eleni Vasilaki, Head of Machine Learning at University of Sheffield who wanted to do something similar for her course on Adaptive Intelligence. We supported her in this endeavour and the result is at (\url{http://bitsandchips.me/COM3240_Adaptive_Intelligence/}).

In order to better support this, \ODK member Tania Allard, developed a Jekyll template for use by academics and researchers using Jupyter notebooks for course materials and dissemination. Such a template allows the creation of Jupyter notebooks based websites using Jekyll, which is the default static website framework supported by GitHub. It also allows for easy display of notebooks connected to cloud computing resources such as Microsoft Azure Notebooks (D2.17, T2.6).

This led to the development of a Python package: nbjekyll (\url{https://github.com/trallard/nbjekyll}) that complements the Jekyll template. This package converts Jupyter notebooks into .md files that can be readily usable by Jekyll (this uses nbformat for the conversion). It also uses \ODK-developed nbval to perform notebook validation and add custom headers indicating the last update date, version and test status of the notebook.

As well as being used internally at Sheffield, The nbjekyll package received some attention on twitter \url{https://twitter.com/jdblischak/status/1009800776305332224} and \url{https://twitter.com/walkingrandomly/status/1009414151716909057} receiving a total of 42 retweets and 80 'likes'

\end{event}
