\begin{event}{CICM2017 --- Conference on Intelligent Computer Mathematics}{CICM2017}{Edinburgh, UK, 17th-21st of July 2017}{FAU,JU}{45}{3}{https://cicm-conference.org/2017/cicm.php}

\textbf{Main goals.}
Digital and computational solutions are becoming the prevalent means for the generation, communication, processing, storage and curation of mathematical information.
Separate communities have developed to investigate and build computer based systems for computer algebra, automated deduction, and mathematical publishing as well as novel user interfaces.
While all of these systems excel in their own right, their integration can lead to synergies offering significant added value.
The Conference on Intelligent Computer Mathematics (CICM) offers a venue for discussing and developing solutions to the great challenges posed by the integration of these diverse areas. 

\textbf{ODK implication.}
Florian Rabe (JU) was track chair for Mathematical Knowledge Management.
One workshops was co-organized by Michael Kohlhase (FAU).
There were no costs to ODK other than travel costs for ODK members.

\textbf{Event summary.}
There were 17 formal paper presentations, 7 formal system presentations, 3 invited presentations, and 5 informal presentations.
Also, 2 workshops took place.

\textbf{Results and impact.}
The ODK participants presented the paper ``Classification of Alignments between Concepts of Formal Mathematical Systems'' that is integral to the MitM approach developed in ODK WP 6.
The authors included external collaborators in order to bundle system integration efforts.

Patrick Ion and Eric Weisstein presented ``The Special Function Concordance'', a development very closely related to the development of the MitM ontology of ODK WP 
\end{event}
