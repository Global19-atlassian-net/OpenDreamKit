\begin{event}{CICM2018 -- Workshop Computer Algebra in the Age of Types}{CAAT2018}{RISC Linz, Hagenberg, Austria, 17th of August 2018}{SA,FAU}{15}{5}{https://cicm-conference.org/2018/cicm.php?event=caat&menu=general}

\textbf{Main goals.} We want to promote the use of types to compose existing
constructive and formal systems, enable formal checkability and correctness,
enable development of domain specific tools, provide access to machine verified
proofs and efficient computations, enable effective automated testing.

\textbf{ODK implication.} The workshop was organised by Markus Pfeiffer (UStan), his
travel costs and conference fees were paid by ODK. The keynote lecture at the
workshop was given by Michael Kohlhase (FAU).

\textbf{Event summary.} Follwing an excellent keynote lecture given by Michael
Kohlhase about the Math-in-the-Middle approach advocated by ODK, we saw an
excellent usecase of types in Sebastian Posur's talk about categorical
computation. Two tutorials, one on the programming language Idris by its creator
Edwin Brady (University of St Andrews), and one by Dennis Müller (FAU) on the
formal system MMT showed off what types can do for programming and mathematics
today.
Some of the workshop participants went to dinner together after the event.

\textbf{Results and impact.} A lively discussion about the vision of using type
theory more in day-to-day mathematical programming, and for which purposes took
place during the event and also in the breaks and at dinner.
Some plans were discussed to use insights from the formal system MMT to drive
executable code in Idris.

\end{event}
