\begin{event}{PASCO 2017}{PASCO2017}{Kaiserslautern, Germany, July 23-24, 2017}{PS}{35}{1}{http://sigsam.org/PASCO/2017/}

\textbf{Main goals.}
The 8th International Workshop on Parallel Symbolic Computation (PASCO) is the
latest instance in a series of workshops dedicated to the promotion and
advancement of parallel algorithms and software in all areas of symbolic
mathematical computation. It is a two days event which is a satellite of
ISSAC, one of the main conference in symbolic computation.

\textbf{ODK implication.} Florent Hivert was invited keynote speaker. The cost
for ODK was therefore null.

\textbf{Event summary.}
Florent Hivert was invited to give a keynote talk presenting his work on WP5
T5.6. The talk was entitled \emph{High Performance Computing Experiments in
  Enumerative and Algebraic Combinatorics}. Here is the abstract:

In this talk, I will report on several experiments around large scale
enumerations in enumerative and algebraic combinatorics.  In a first part,
I'll present a small framework implemented in Sagemath allowing to perform
map/reduce like computations on large recursively defined sets. Though it
doesn't really qualify as HPC, it allowed to efficiently parallelize a dozen
of experiments ranging from Coxeter group and representation theory of monoids
to the combinatorial study of the C3 linearization algorithm used to compute
the method resolution order (MRO) in script language such as Python and Perl.

In a second part, I'll describe a methodology used to achieve large speedups
in several enumeration problems. Indeed, in many combinatorial structures
(permutations, partitions, monomials, young tableaux), the data can be encoded
as a small sequence of small integers that can often efficiently be handled by
a creative use of vector instructions. Through the challenging example of
numerical monoids, I will then report on how Cilkplus allows for a extremely
fast parallelization of the enumeration. Indeed, we have been able to
enumerate sets with more that $10^15$ elements on a single multicore machine.
This is joint work with Jean Fromentin.

\textbf{Results and impact.} The talk was a chance to disseminate ODK work in
a wider audience and to present the result on deliverable D5.1 and the ongoing
progress on the overall work package. The fact that it was an invited keynote
talk witnesses that the community is particularly interested and attentive on
the ODK progress on this matters.

\end{event}
