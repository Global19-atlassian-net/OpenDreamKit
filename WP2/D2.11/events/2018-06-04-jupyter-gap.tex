\begin{event}{GAP Days -- Jupyter in GAP and other CAS}{GAPDAYS2018}{St Andrews, United Kingdom, 4th to 8th June 2018}{SA}{13}{4}{https://www.gapdays.de/gap-jupyter-days2018}

\textbf{Main goals.} The aim of the workshop was to bring together people who are developing or using tools based around Jupyter, Thebelab, by OpenDreamKit.

\textbf{ODK implication.} Organised by Markus Pfeiffer at UStan. ODK Paid for
the workshop dinner, the accommodation for Nusa Zidaric and Pedro
Garcia-Sanchez, travel and accommodation for Nicolas Thiery.

\textbf{Event summary.} We met in a small group of developers and interested
clients of our Jupyter, thebelab, MyBinder and other web technology relevant for
virtual research environments, focused on GAP and the wider context of
OpenDreamKit. We explained how the GAP Jupyter Kernel works, developed some
prototypes with and improvements of the GAP Jupyter kernel.

We demonstrated the work of Manuel Martins (Markus Pfeiffer's PhD student),
Sebastian Gutsche, and Pedro A. García-Sánchez who are using GAP's Jupyter
kernel in research and teaching and to develop new ways of using them.

\textbf{Results and impact.} Teachers and researchers intending to of the GAP
Jupyter kernel and adjacent technologies in teaching and research, are now much
more comfortable using these technologies. The demonstration of the
possibilities left a good impression of the potential of this technology.

\end{event}
