\begin{event}{GAP Days -- Jupyter in GAP and other CAS}{GAPDAYS2018}{St Andrews, United Kingdom, 4th to 8th June 2018}{SA}{13}{4}{https://www.gapdays.de/gap-jupyter-days2018}

  \textbf{Main goals.} The aim of the workshop was to bring together
  developers and early adopters of Jupyter, Binder, Thebelab and other
  web technology relevant for Virtual Environments for research and
  teaching, with some focus on the GAP computational system.

Topics that were worked on included:

\begin{itemize}
\item The GAP JupyterKernel
\item Javascript-Visualizations using Jupyter
\item Using Thebe for interactive manuals
\item Developing teaching materials with Jupyter, and publishing them using MyBinder
\item Writing academic publications using MyBinder and Docker images, to make
  all computational results and all examples fully and easily reproducible.
\item Discussing demands on software and hardware infrastructure in day to day
  use.
\end{itemize}


\textbf{\ODK implication.} Organised by Markus Pfeiffer at UStan. \ODK paid for
the workshop dinner, the accommodation for Nusa Zidaric, Pedro
Garcia-Sanchez, and Sergio Siccha, travel and accommodation for Nicolas Thiéry.


\textbf{Event summary.} We met in a small group of developers and interested
clients of our Jupyter, ThebeLab, Binder and other web technology relevant for
virtual research environments, focused on GAP and the wider context of
OpenDreamKit. We explained how the GAP Jupyter Kernel works, developed some
prototypes with and improvements of the GAP Jupyter kernel.
%
We demonstrated the work of Manuel Martins (Markus Pfeiffer's PhD student),
Sebastian Gutsche, and Pedro A. García-Sánchez who are using GAP's Jupyter
kernel in research and teaching and to develop new ways of using them.

% The event consisted of the following talks
% \begin{itemize}
% \item \emph{Introduction to the GAP-Jupyter kernel} -- Markus Pfeiffer (University of St Andrews),
% \item \emph{Basic setup for Binder} -- Sebastian Gutsche (University of Siegen),
% \item \emph{ThebeLab demo} -- Nicolas M. Thiéry (Paris Sud),
% \item \emph{Experiencing with Jupyter GAP} -- Pedro Garcia-Sanchez (Universidad de Granada),
% \item \emph{FSR - Feedback Shift Registers} -- Nusa Zidaric (University of Waterloo).
% \end{itemize}

To structure programming work and discussions, we followed the well-established
practice of a stand-up meeting in the morning to coordinate work, and a meeting
in the afternoon to record achievements.
%
Discussions ranged from deepining the understanding of how Jupyter works, and
what needs the research and teaching community has to more technical topics
which reach into other workpackges such as improved integration between SageMath
and GAP.

\textbf{Results and impact.} Teachers and researchers intending to use the GAP
Jupyter kernel and adjacent technologies in teaching and research, are now much
more comfortable using these technologies. The demonstration of the
possibilities left a good impression of the potential of this technology.

% The workshop was very productive. The goal of a readily packaged JupyterKernel
% for GAP was achieved, an improved Docker deployment of GAP has been provided.
% Apart from work on WP4, we also achieved progress on WP3 and WP6 through
% discussions and indentification of synergies between the software demands of
% these workpackages.


\end{event}
