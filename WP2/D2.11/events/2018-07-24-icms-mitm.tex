\begin{event}{ICMS 2018 -- Session 14: Towards Composable Mathematical Software}{ICMS18-14}{South Bend, Notre Dame, 25th July 2018}{SA,PS,FAU}{25}{3}{http://icms-conference.org/2018/sessions/session14/}

\textbf{Main goals.} The aim of this session is to provide a forum for
developers and users of mathematical software with an interest in composablity
and interoperability of, and knowledge and data exchange between systems, to
share experiences, solutions, and a vision for the future.

\textbf{ODK implication.} The ODK participants Markus Pfeiffer, Nicolas Thiery,
and Florian Rabe organised this session, and invited the speakers.

Travel costs for Markus Pfeiffer were covered by ODK, as well as accommodation
costs for Sebastian Gutsche.

\textbf{Event summary.} In a single session we saw talks by Sebastian Gutsche
about integrating GAP and Julia, William Stein about the challenges with
SageMath integrating a wealth of software, Michael Kohlhase about the
Math-In-the-Middle approach advocated by ODK and Tim Daly about proving the
computer algebra system axiom sane.

The session was very well attended which points at the demand for this
technology. The ratio of talk submissions to attendees also hints at the fact
that not many people have the resources to address the challenges brought up by
the session topic.

\textbf{Results and impact.} The session enabled a lively discussion about the
challenges of interfacing mathematical software correctly and so achieved one of
its goals.

\end{event}
