\begin{event}{Introduction to HPC using the Jupyter notebook}{SHF_HPC}{University of Sheffield, 12th of April 2018}{USH}{15}{2}{https://github.com/RSE-Sheffield/hi-perf-ipynb}

\textbf{Main goals.} Previous ODK work provided us with Sun Grid Engine Support for Project Jupyter Hub (Deliverable 5.3) which gave users of High Performance Computing systems, such as the one at University of Sheffield, access to powerful supercomputer hardware via the notebook. This workshop acted to disseminate this work, encourage more people to use HPC via the newly developed interface and teach the basics of parallelisaton in Python.

\textbf{ODK implication.} Workshop materials were developed by Will Furnass (USH) and released online at \url{https://github.com/RSE-Sheffield/hi-perf-ipynb}. The workshop was delivered by Will Furnass and Mike Croucher at University of Sheffield.

\textbf{Event summary.} A group of researchers from a cross-section of subjects were taught the basics of parallelization and HPC using ODK developed technologies and tutorials. The event was publicized on twitter which lead to follow up queries about the underlying technology. 

\textbf{Results and impact.} The event established that the technology is scalable enough to be used in production for a typical class size at HPC training events. The ODK developed technology is now a permanent part of the HPC service at University of Sheffield with ongoing work continually released at \url{https://github.com/RSE-Sheffield/jupyterhub-gridengine-sharc}

\end{event}
