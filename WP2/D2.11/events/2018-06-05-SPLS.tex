\begin{event}{SPLS}{SPLS}{Heriot-Watt University, Edinburg, Scottland, June 5, 2018}{PS}{35}{1}{http://www.macs.hw.ac.uk/~rs46/spls-june-2018/}

\textbf{Main goals.}  The \emph{Scottish Programming Languages Seminar} is a
forum for discussion of all aspects of programming languages. They meet for a
day or afternoon once every few months, at some congenial location in
Scotland.

\textbf{\ODK implication.} Florent Hivert gave an invited talk. The cost for
\ODK was therefore null.

\textbf{Event summary.}  Florent Hivert was invited to give a keynote talk
presenting his work on WP5 T5.6. The talk was entitled \emph{ Multi-level
  parallelism for high performance combinatorics}. Here is the abstract:

In this talk, I will report on several experiments around large scale
enumerations in enumerative and algebraic combinatorics.  I'll describe a
methodology used to achieve large speedups in several enumeration
problems. Indeed, in many combinatorial structures (permutations, partitions,
monomials, young tableaux), the data can be encoded as a small sequence of
small integers that can often efficiently be handled by a creative use of
processor vector instructions. Through the challenging example of numerical
monoids, I will then report on how Cilkplus allows for a extremely fast
parallelization of the enumeration. Indeed, we have been able to enumerate
sets with more that $10^15$ elements on a single multicore machine.

\textbf{Results and impact.} The talk was a chance to disseminate \ODK work in
a wider audience and to present the result on deliverable D5.1 and the ongoing
progress on the overall work package. The fact that it was an invited keynote
talk witnesses that the community is particularly interested and attentive on
the \ODK progress on this matters.

\end{event}
