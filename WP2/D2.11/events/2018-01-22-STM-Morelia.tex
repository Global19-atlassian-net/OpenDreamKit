\begin{event}{Software tools for mathematics 2018-01 Morelia}%
{STM_2018_01_Morelia}{Morelia, Mexico, 2018-01-22--2018-01-26}%
{PS}{50}{1}{http://matmor.unam.mx/software-tools-math/}

\textbf{Main goals.} The goal of the event was for mathematicians
to improve their coding skills and knowledge of mathematical software.

\textbf{ODK implication.} Samuel Lelièvre (ODK member from UPSud) was one of
the organisers. The OpenDreamKit funds at Paris-Sud were used to fund travel
and stays of speakers, and lunch and coffee breaks for all participants.
Centro de Ciencias Matemáticas (CCM), the mathematics department at
UNAM Morelia, co-funded the event, paying travel and accommodation for
some participants.

\textbf{Event summary.} The event consisted in a two-day Software Carpentry
workshop (teaching participants the Unix shell, version control with Git,
and programming with Python) followed by three days on mathematical software
with mini-courses on CoCalc, GAP, Jupyter, PARI/GP, SageMath, YAGS, as well
as talks on other mathematical software and databases, and on mathematical
research using software. A problem session allowed participants to submit
mathematical problems they cared about and thought software might help with,
several of which were solved in the following days by other participants.

\textbf{Demographic.} Over 140 people registered: 36\%\ bachelor students,
10\%\ master students, 14\%\ PhD students, 6\%\ postdocs, 20\%\ professors
and researchers, 10\%\ other. Since we could only welcome on the order of
50 participants, we focused on PhD students, postdocs, professors and
researchers.

\textbf{Results and impact.}

During the Software Carpentry workshop, Tania Hernandez taught the Unix shell,
Nelly Selem taught version control with Git, Leticia Vega taught programming
with Python. These courses were taught in Spanish, although using supporting
materials in English. This may have been the first time a Software Carpentry
workshop was taught entirely in Spanish. An effort is underway to translate
the Software Caprentry and Data Carpentry lesson plans to Spanish.

During the mathematical software part, Alexander Hulpke taught GAP, Samuel
Lelièvre taught CoCalc, Jupyter and SageMath, Miguel Pizaña taught YAGS
(a GAP package for working with graphs), Miguel Raggi presented Discreture
(a library for enumerating combinatorial objects), Emmanuel Royer taught
PARI/GP, Adrián Soto presented TeXmacs and gave a talk on Rauzy fractals,
Janoš Vidali presented DiscreteZOO, Rafael Villarroel presented Emacs,
Russ Woodroofe gave two talks ("The story of a calculation" and "The story
of a figure"), Katja Berčič gave a talk on "Databases of theorems", and
Uziel Silva gave a presentation of Macaulay2, Greuel and Reveal.js.

Many participants told the organisers, orally or by email, that this workshop
was transformative for them; often they felt they had passed some confidence
threshold: whereas before the conference they were interested in mathematical
software but unsure how to install and use them, they were now confident how
to do that, and felt they had the necessary resources to learn more.

One of the participants who is part of the board of the Mexican Math Society
initially intended to visit briefly to check out our workshop briefly, and
ended up staying the whole week, staying at the install party during the free
afternoon, and got convinced of the importance of having a software component
in future mathematics conferences in his area. As a result, David Sanders
gave a course on "numerical methods for dynamical systems", based on Julia,
at the next national dynamical systems conference in Mexico in June 2018.

Another of the organisers, Katja Berčič, a Slovenian post-doc currently in
Morelia, Mexico, liked the format of this workshop so much that a new workshop
on the same format is planned for September 2018 in Koper, Slovenia.

% \begin{figure}[ht]
% \caption*{A great picture of my wonderful event}
% Did you take pictures? Please share!
% \end{figure}

\end{event}