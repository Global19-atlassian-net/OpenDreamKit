\begin{event}{Atelier PARI/GP 2018}{AtelierPARI2018}{Besan\c{c}on (FR),
2018-01-15 to 2018-01-19}{PS,UB,UV}{36}{6}{http://pari.math.u-bordeaux.fr/Events/PARI2018/}


\textbf{\ODK implication.}
%Describe how \ODK was involved and give a rough estimation of cost for \ODK

\ODK participants: B. Allombert, K. Belabas, V. Delecroix, J. Demeyer,
J.-P. Flori, L. de Feo.

\ODK provided the main funding source for the workshop (accommodation,
subsistence and travel expenses), for about 13k\euro. The Besan\c{c}on
institute of mathematics co-funded the event.

\textbf{Event summary.}
%Give a summary of your event

The 10th Atelier PARI/GP took place in Besan\c{c}on (France) from january
15h to 19th.

There were 38 registered participants from 19 different institutions
(no registration fees).

A typical day of the workshop had introductory talks and tutorials
in the morning; afternoons allowed ample time for hacking sessions,
discussions and training.

The Atelier featured 8 morning talks on mathematical topics and
implementation projects including 4 talks by \ODK members
\begin{itemize}
\item Karim Belabas ``Modular Forms''
\item Bill Allombert ``New GP features'', \texttt{lfun} and Artin
  $L$-functions.
\item Luca de Feo and Jean-Pierre Flori ``Algorithms for lattices of
  compatibly embedded finite fields''.
\end{itemize}

Slides for all talks are available at
\url{http://pari.math.u-bordeaux.fr/Events/PARI2018/}

\textbf{Results and impact.}
% What did you achieve with this event? (If ever it impacted
% other \ODK tasks and deliverables, mention it here)

The workshop was very productive and particularly beneficial to WP5
(high-performance computing), it provided final feedback on recent PARI/GP
  modules and interfaces that paved the way for the release of
  pari-2.10-alpha (2018/05) and pari-2.11-stable (2018/07, the first major
  release since 2016).

It also was a successful dissemination event: 14 participants had not come to
  a previous Atelier.
\end{event}
