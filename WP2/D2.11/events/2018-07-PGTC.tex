\begin{event}{GAP Tutorial at the 20th Postrgaduate Group Theory Conference}{PGTC2018}{St Andrews, Jul. 16 -- 20, 2018}{SA}{21}{1}{https://www.codima.ac.uk/pgtc2018/}

\textbf{Main goals.} The Postrgaduate Group Theory Conference (PGTC) is an annual meeting of
young researchers in group theory. The GAP Tutorial was organised as an optional satellite event.

\textbf{ODK implication.} Although ODK was not formally involved in this event, it was
organised by Alexander Konovalov, and was used to promote the project and its outcomes.

\textbf{Event summary.} The GAP Tutorial was intended for beginners without requiring
any prior knowledge of the GAP system. The first part on Monday July 16th covered
the Software Carpentry lesson ``Programming with GAP'' and was delivered in the live
coding style. The second part on Friday July 18th included a talk on debugging and
profiling in GAP by Christopher Jefferson (St Andrews) and a talk
``Using GAP Effectively: Some Tips and Pitfalls'' by Alexander Konovalov. The latter
included the demonstration of the GAP Jupyter interface as included in the GAP 4.9.2
release (June 2018). Using Jupyter notebooks for this talk contained in the repository
at \url{https://github.com/alex-konovalov/gap-teaching}, attendees were able to launch
them on Binder and try to work with GAP in Jupyter themselves. Several participants
received help with installing the latest release of GAP and configuring its Jupyter
kernel after the tutorial and left the event with fully usable working environment.

\end{event}