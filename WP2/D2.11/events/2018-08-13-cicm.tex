\begin{event}{CICM2018 --- Conference on Intelligent Computer Mathematics}{CICM2018}{RISC Linz, Hagenberg, Austria, 13th-17th of August 2018}{SA,FAU}{61}{6}{https://cicm-conference.org/2018/cicm.php}

\textbf{Main goals.}
Digital and computational solutions are becoming the prevalent means for the generation, communication, processing, storage and curation of mathematical information.
Separate communities have developed to investigate and build computer based systems for computer algebra, automated deduction, and mathematical publishing as well as novel user interfaces.
While all of these systems excel in their own right, their integration can lead to synergies offering significant added value.
The Conference on Intelligent Computer Mathematics (CICM) offers a venue for discussing and developing solutions to the great challenges posed by the integration of these diverse areas. 

\textbf{ODK implication.}
The program chair was Florian Rabe (FAU).
Workshops were co-organized by Michael Kohlhase (FAU) and Markus Pfeiffer (SA).
There were no costs to ODK other than travel costs for ODK members.

\textbf{Event summary.}
CICM featured 3 invited talks, 3 days of presentations (including a system demo session), a doctoral program, and 6 workshops.

\textbf{Results and impact.}
The ODK participants presented several papers that came out of ODK.
This triggered a number of discussions with researchers from adjacent fields in computer science as well as a few mathematicians.
Richard Markus was awarded the Best Demo Award for his 3-dimensional theory graph viewer --- a tool developed in ODK.

Bruno Buchberger gave an invited talk on the future of a global digital mathematical library, which prompted discussions on future grant proposals in this direction.

Bruce Miller and Adri Olde Daalhuis presented the DLMF, a library of special mathematical functions.
With them, Rabe and Kohlhase (FAU) designed an integration of the DLMF with the MitM ontology developed in ODK WP 6.

\end{event}
