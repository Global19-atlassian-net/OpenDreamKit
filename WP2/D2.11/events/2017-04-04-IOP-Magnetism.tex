\begin{event}{IOP Magnetism 2017 - Computational micromagnetics with JOOMMF workshop}{IOP2017}{University of York, UK, 04 April 2017}{XFEL}{30}{3}{http://magnetism2017.iopconfs.org/OOMMF}

\textbf{Main goals.} We introduced the basics of micromagnetics as well as taught the participants how to run OOMMF simulations using our Python interface - JOOMMF.

\textbf{\ODK implication.} JOOMMF was developed as a part of the \ODK project and three participants from the \ODK were present to deliver the workshop (Hans Fangohr, Marijan Beg, and Ryan A. Pepper). The workshop was co-funded by the conference organisers and the EPSRC CCP Computational Magnetism Network (EP/M022668/1) grant. No costs of the workshop were paid from the \ODK funds.

\textbf{Event summary.} In this workshop we provided a brief introduction to computational micromagnetics, Python, and the Jupyter notebook. We taught participants to use our Python interface to drive OOMMF micromagnetic simulation by guiding them through tutorials. At the beginning of the workshop, we provided a lecture style introduction, which was followed by practical exercises where attendees had an opportunity to carry out small micromagnetic calculations, modify given examples and ask more specific questions. This workshop was held together with Michael Donahue, NIST - one of the main developers of the OOMMF package.

\textbf{Demographic.} We had about 30 participants during the workshop, but due to the data protection regulations, the organisers did not allow us to have demographics information. During the workshop, 19 participants gave us their details with demographics: 13 males and 6 females.

\textbf{Results and impact.} During the workshop we received the feedback from the participants about our Python interface to OOMMF as well as gained experience which helped us to structure future workshops.

\end{event}
