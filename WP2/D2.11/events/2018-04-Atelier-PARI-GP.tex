\begin{event}{Atelier PARI/GP 2018b}{AtelierPARI2018b}{Roma (IT),
2018-04-16 to 2018-04-17}{UB}{36}{6}{http://pari.math.u-bordeaux.fr/Events/PARI2018b/}

\textbf{Main goals.}

This was a teaching and dissemination meeting, by invitation from the Roman
  Number Theory Association as a satellite event for their 4th
  mini-symposium.

\textbf{ODK implication.} 
%Describe how ODK was involved and give a rough estimation of cost for ODK

\ODK participants: B. Allombert and A. Page from Bordeaux.

\ODK funded travel and accomodation costs for the two instructors for about
  3k\euro. The ALGANT consortium, LIA LYSM (CNRS) and University Roma Tre
  co-funded the event.

\textbf{Event summary.} 
%Give a summary of your event

This Atelier PARI/GP took place in Roma (Italy) from april 16th to
17rd, it was followed by a 3-day international research conference on
  Number Theory. There were 30 participants for the Atelier.

The 2-day Atelier followed the same pattern as the preceding Oujda Atelier,
featuring a general introduction to PARI/GP and two 
  specialized courses (graduate level) in the mornings:
\begin{itemize}
\item Bill Allombert ``Elliptic curves over finite fields and number fields'',
\item Aurel Page ``Algebraic number theory''.
\end{itemize}
Afternoons were devoted to practice sessions.

Slides for all talks are available at
\url{http://pari.math.u-bordeaux.fr/Events/PARI2018b/}

\textbf{Results and impact.} 
% What did you achieve with this event? (If ever it impacted 
% other ODK tasks and deliverables, mention it here)

This was a successful teaching and dissemination event.
\end{event}
