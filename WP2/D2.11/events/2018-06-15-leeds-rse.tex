\begin{event}{Inauguration of UK Research Software Engineering Association}{Leeds_RSE}{United Kingdom, 15th of June 2018}{LEEDS}{8}{2}{https://twitter.com/sjh5000/status/1007639883165454336}

\textbf{Main goals.} The Research Software Engineering movement began in the UK in 2012 (See \url{https://doi.org/10.5281/zenodo.495360} for background and history) and aimed to support the people in academia who's careers focus on the computing that underpins research rather than the research itself. OpenDreamKit was an example of an EU-funded project that employed a large number of Research Software Engineers which served as a useful demonstrator for this emerging profession. In 2018, The UK Association of Research Software Engineers \url{https://rse.ac.uk/about/the-association/} became a formal society as recognised in UK Law. This was supported, in part, by the work of ODK-funded Tania Allard who was elected to the association committee in late 2017.

\textbf{ODK implication.} ODK was involved in this work via ODK-funded team members Tania Allard and Mike Croucher of University of Leeds.

\textbf{Event summary.} Following months of work, the committee signed the legal documents on 15th June 2018. The event was summarized on twitter at \url{https://twitter.com/sjh5000/status/1007639883165454336}

\textbf{Results and impact.} OpenDreamKit is part of the history of the development of this new profession. Initially as proof that Research Software Engineers could win funding in their own right to perform vital work for the research community and later as an integral part of the formation of the legal entity that supports the RSE community.  Research Software Engineers are changing the way that research is being performed for the better. Initially based only in the UK, the movement has since gone international \url{https://rse.ac.uk/community/international-rse-groups/}.

\end{event}
