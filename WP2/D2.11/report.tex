\documentclass{deliverablereport}

\usepackage[style=alphabetic,backend=bibtex]{biblatex}
\addbibresource{report.bib}
\addbibresource{../../lib/publications.bib}

\usepackage{xparse}
\usepackage{etoolbox}
\usepackage{caption}
 \ExplSyntaxOn

\newcounter{eventcounter}

\newenvironment{event}[7]{
\vspace{0.5cm}
\refstepcounter{eventcounter}
\label{event-#2}


\noindent\textbf{Event~\theeventcounter -~ #1}\newline % title

\noindent #3 \newline % location and date

\noindent ODK~partners~involved:~ \clist_map_inline:nn{#4}{\site{##1}~}\newline %partners

\ifx&#5&%
      % no participant #
\else
\noindent #5~participants 
\ifx&#6 &%
    % no odk participant
\else 
(including~#6~from within ODK)\newline
\fi
\fi

\ifx&#7&%
      % no website
\else
\noindent \url{#7}\newline
\fi



}{\begin{center}\noindent\rule{4cm}{0.4pt}\end{center}}

 \ExplSyntaxOff


\deliverable{dissem}{workshops-3}
\duedate{31/08/2018 (M36)}
\deliverydate{31/08/2018}
\author{Viviane Pons et al.}

\begin{document}
\enlargethispage{4ex}
\maketitle
\githubissuedescription
\tableofcontents
\newpage

% Other tasks and deliv impacted: T3.6, T4.1, T6.1, T6.3, T6.10, D6.2 WP5 D4.4 D4.7 T3.1 T4.4 T4.6

\section{Project meetings and development workshops}

We call a development workshop an event with a restricted number of participants
who meet to work on a specific task. These workshops are an inherent part
of \ODK development process as described in \taskref{dissem}{devel-workshops}:
 they bring together
developers from within and outside of \ODK and allow effective work
and discussions on many technical aspects. They also participate in building
and maintaining a community of developers inside \ODK and within the
open-source communities we belong to.

We list here all workshops which have been organized or co-organized by \ODK
as well as external workshops which have been attended by \ODK participants
with a significant impact on the project. We also include project meetings as they
participate to the same goal of bringing together participants of the project and
always include some development time.

Throughout year 2 and 3 of the project, we have had

\begin{event}{Event Name}{eventId}{Location and date}{Partners (use Partners id, separated with coma)}{# of participants}{url}

\textbf{Main goals.} Describe your event quickly

\textbf{ODK implication.} Describe how ODK was involved and give a rough estimation of cost for ODK

\textbf{Event summary.} Give a summary of your event

\textbf{Demographic.} Do you have demographic information? If so, please share!

\textbf{Results and impact.} What did you achieve with this event? (If ever it impacted 
other ODK tasks and deliv, mention it here)

\begin{figure}[ht]
\caption*{A great picture of my wonderful event}
Did you take pictures? Please share!
\end{figure}



\end{event}


\section{Dissemination and outreaching activities}

We describe here all activities related to \taskref{dissem}{dissemination}:
these are all events oriented towards dissemination, training, and outreach. This
includes events organized or co-organized by \ODK and also
participating in external events and many communication activities.

\subsection{Organization of Sage Days in established mathematical communities}

One goal of \ODK is to support local communities of researchers
and developers who contribute to the open-source softwares related to
the project. For Sage, this means supporting the organization of Sage-Days
workshops that arise from within all the different mathematical communities. The main 
goal of these workshops is mostly to improve the Sage coverage of some mathematical
area. They also play a major role in training and communication. The
impact for \ODK can be summarized this way:

\begin{itemize}
\item \textbf{Making ODK known to the end users}: by supporting Sage Days,
\ODK makes itself known to the Sage community and can
thus share the many developments of the project.

\item \textbf{Improving the overall quality of Sage}: by fostering researchers
in specific areas, Sage Days help bring interesting mathematics into
the software, which is beneficial for Sage and so \ODK.

\item \textbf{Training, bringing more user}: Sage Days are the perfect place
for new comers, especially students, to get their first experience with the software.

\item \textbf{Fostering a community}: Sage Days are helping making Sage a vibrant
community, which is vital for the success of \ODK.
\end{itemize}

\subsection{Training activities for Sage in developing countries}

As open-source software developers, we wish our products
to be accessible to as many people as possible. Even though we offer
 a free access, there is still a technical gap in many 
developing countries that 
often prevents schools and researchers to benefit from our softwares.
This is why we believe the role of \ODK is to foster 
a wider community that does not leave a part of the world behind. In 
this section, we describe training activities that have been conducted 
through \ODK in this regard.


\subsection{Communication and participation to external events}

Dissemination activities also include the participation of \ODK
members to many different conferences of various size and topics
in computer science, mathematics, physics, and more. The goal is
to reach potential end-users, build bridges between communities and stay aware 
of current development in the scientific community.

We list here major events and communication. We have also put in place
a blog on our website: \url{http://opendreamkit.org/activities/} to track
these activities.


\section{Upcoming events and plans for the future}


\newpage\printbibliography

{\footnotesize All referenced ODK talks are available as annexes}

\end{document}

%%% Local Variables:
%%% mode: latex
%%% TeX-master: t
%%% End:
