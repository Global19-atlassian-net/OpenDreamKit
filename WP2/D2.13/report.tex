\documentclass{deliverablereport}

\usepackage[style=alphabetic,backend=bibtex]{biblatex}
\addbibresource{report.bib}
\addbibresource{../../lib/publications.bib}

\usepackage{xparse}
\usepackage{etoolbox}
\usepackage{caption}

\deliverable{dissem}{oommfnb-vre-deliver}
\duedate{31/08/2018 (M48)}
\deliverydate{30/08/2019}
\author{Marijan Beg and Hans Fangohr}

\begin{document}
\maketitle
\githubissuedescription
\newpage
\tableofcontents
\newpage

\section{Introduction}

Object Oriented MicroMagnetic Framework (OOMMF) is a micromagnetic simulation tool, developed in 1990s at the National Institute of Science and Technology (NIST) by Michael Donahue and Don Porter. It is probably the most widely used and most trusted simulation tool in the micromagnetic scientific community. It was written in C++ and wrapped with Tcl, which is the language that must be used to configure simulations by the user. The computational workflow that must be done by the user in order to simulate a particular micromagnetic problem is as follows:

\begin{enumerate}
\item A configuration script (\texttt{.mif}) file is written in Tcl, so that all characteristics of the micromagnetic system are specified (geometry, Hamiltonian, dynamics equation, etc.).
\item OOMMF is run by providing a configuration \texttt{.mif} file via Terminal/Command Prompt. After the OOMMF run is complete, the vector fields are saved as \texttt{.omf} files, whereas scalar data is saved in an \texttt{.odt} file.
  \item If another run with different parameters is necessary steps 1 and 2 are repeated.
  \item Resulting files are opened and analysed by the user.
\end{enumerate}

There are several issues that are related to this particular type of workflow:
\begin{itemize}
\item It is very difficult for a user to automate the process of running many different simulations with different parameters.
\item It is difficult to keep a log of all steps performed in the entire micromagnetic study.
  \item Postprocessing and the analysis of results is performed outside OOMMF using techniques and scripts developed by the user.
\end{itemize}

All these issues compromise the reproducibility of the micromagnetic study because it is very difficult to convey the exact simulation procedure. The main goal of deliverable D2.13 in the \ODK project was to develop a Micromagnetic Virtual Research Environment (VRE) which would address this problem. More precisely, the goal is to develop a Python interface to OOMMF and integrate it into Jupyter notebook. Jupyter notebooks can contain: (i) simulation and data analysis code, (ii) code output, such as numerical and visualised data, and (iii) human-readable text. This way, all the necessary information required to make the study reproducible are contained in a single document which can be later shared, modified and re-executed. Another advantage is that, because micromagnetic simulations are exposed to Python, all the benefits of Python language can be employed. For example, running different simulations with different parameters can be achieved by simple looping over the parameter space, which does not require the assistance from the user in terms of modifying and running individual \texttt{.mif} configuration files. Another benefit is the use of Python's scientific stack for data analysis and visualisation. This way, it is not necessary for the user to ``reinvent the wheel'' by writing scripts for opening \texttt{.odt} and \texttt{.omf} files as well as for performing basic data analysis and visualisation operations.   

\section{Implementation details}

The resulting package is called Ubermag and it is separated into several different sub-packages:

\begin{itemize}
\item \texttt{discretisedfield} - reading, writing, analysis, and visualisation of scalar and vector fields. This package provides all the necessary functionality for reading resulting OOMMF vector and scalar field files and their analysis, such as sampling, iterating, computing averages and norms, and visualisation. For visualisation, we use two different approaches: (i) 2D visualisation of spacial slices of using \texttt{matplotlib} and (ii) 3D visualisation of both scalar and vector fields using \texttt{k3d}. The main benefit of this package is that all fields, after being created or read from files, are represented via \texttt{numpy} array, which enables exposing micromagnetic fields obtained from OOMMF to the Python scientific stack.
\item \texttt{ubermagtable} - reading and analysis of scalar tabular data. The main purpose of this package is to enable easy reading and analysis of OOMMF \texttt{.odt} files. This is achieved by converting the contents of an \texttt{.odt} file to \texttt{pandas} DataFrame object. Similar to exposing scalar and vector field data to Python's scientific stack in \texttt{discretisedfield}, in \texttt{ubermagtable}, this is achieved via \texttt{pandas} DataFrame, which enables easy analysis and visualisation of the tabular data.
\item \texttt{ubermagutil} - typesystem used across all Ubermag packages. This package contains the implementation of different Python descriptors used for imposing a typesystem for individual attributes in different classes.
\item \texttt{micromagneticmodel} - a domain specific language for defining a micromagnetic simulation. Details about this package are explained in Sec.~\ref{sec:micromagneticmodel}.
\item \texttt{oommfc} - OOMMF calculator. After the micromagnetic simulation is defined using \texttt{micromagneticmodel}, \texttt{oommfc} translates the model into OOMMF simulation and runs it. Finally all resulting simulation files are read by employing \texttt{discretisedfield} and \texttt{ubermagtable}. Details about this package are explained in Sec.~\ref{sec:oommfc}.
  \item \texttt{micromagneticdata} - data analysis convenience tools. This package contains code for the analysis of simulation results integrated in Jupyter notebook.
\end{itemize}

We decided to split the entire functionality of Ubermag into separate smaller packages in order to allow users to use them individually. For instance, \texttt{discretisedfield} package is a universal package - not restricted to micromagnetic - and can be used for any finite-difference field, such as in computational fluid dynamics. 

\subsection{\texttt{micromagneticmodel}}\label{sec:micromagneticmodel}

The motivation for the development of the domain specific language for the defining of micromagnetic problems comes from the fact that almost all micromagnetic problems can be defined in a unified way. More precisely, in order to uniquely define a micromagnetic problem, three main components must be defined:

\begin{enumerate}
\item Initial magnetisation. Magnetisation is a a vector field defined on a finite-difference mesh, so that one vector is associated to each cell.
\item Hamiltonian. Depending on what energies are present in the simulated sample, Hamiltonian can contain different energy terms. More precisely, the total energy of the system can be computed as a sum of different energy terms computed for the particular magnetisation configuration.
  \item Dynamics equation. This equation governs the magnetisation dynamics and, similar to Hamiltonian, it can contain different dynamics terms.
\end{enumerate}

Because there is a uniform way how a micromagnetic problem can be defined, we developed a domain specific language for defining such problems. A model, defined using this domain specific language, is not aware of the particular simulation tool that is going to perform the actual micromagnetic simulation, but it is only used to describe the problem.

\subsection{\texttt{oommfc}}\label{sec:oommfc}

\texttt{micromagneticmodel} described in the previous section does not contain any information on how the defined micromagnetic problem can be translated into the OOMMF configuration file and run in OOMMF. \texttt{oommfc} inherits all the functionality of \texttt{micromagneticmodel} and specifies how all necessary elements of the micromagnetic model can be translated into the OOMMF configuration file. OOMMFC runs the simulation and after it is completed, read the resulting files and updates the system object.

\subsection{Code hosting}

All the code developed as a part of this deliverable is hosted in different repositories on the code hosting service GitHub as a part of the Ubermag organization. All the code is open source and publicly available. In addition, a snapshot of the repositories has been made and all their contents are hosted in Zenodo.

\subsection{Testing}

All individual packages of Ubermag are tested by running different Python tests. There are three different types of tests:

\begin{itemize}
\item Testing of the package code. These tests are running individual functionalities of of different Ubermag packages as well as larger micromagnetic simulations, such as standard problems. These tests are used for computing the code coverage - the percentage of the code covered by tests.
\item Jupyter notebook tests. The majority of the Ubermag's functionality is documented using Jupyter notebook. in order to make sure that the documentation is up-to-date with the newest version of the source code, they are tested using \texttt{nbval}, which was developed as an earlier part of OpenDreamKit deliverable \delivref{UI}{jupyter-test}. More precisely, all individual code cells in a Jupyter notebook are run and their results are compared to the previous ones.
\item API documentation tests. As a part of different class and function implementation, doc-strings were included and later used for the API documentation. These strings usually contain small examples demonstrating the usage of individual features of the code. Similar to the testing of Jupyter notebook s used for documentation, we also test these examples.

  All three different types of tests are run on two different continuous integration platforms: Travis CI for Linux and AppVeyor for Windows operating system. These tests are triggered each time a new commit to the repository is made. The codecov service is employed to record the coverage of the tested code.
\end{itemize}

\subsection{Documentation}

Similar to testing, every package is individually documented. There are two different types of documentation: (i) API documentation, and (ii) examples of usage in Jupyter notebooks. API documentation, implemented via doc-strings, is generated by Sphinx. It contains the detailed explanation of all possible ways of using different functions and classes as well as small examples. On the other hand, Jupyter notebooks provide tutorial-like documentation which explains how different functionalities of a package can be used in real-world problems. The top documentation containing the tutorial on Ubermag, is a part of \texttt{ubermag} meta-package. All the documentation is generated as soon as a new commit is made to the repository and it is hosted on readthedocs. In addition, several different tutorials, in particular for the installation of Ubermag on different platforms, and recorded as videos and hosted on YouTube on ubermag channel. All documentation can also be interactively used via MyBinder.

\subsection{Installation}

One of the main challenges scientists face in using simulation tools is their installation. Installation procedures are usually complicated and too specific for a particular platform. Therefore, we wanted to make the installation of Ubermag as simple as possible.

All packages are built and made available for installation on PyPI. However, installing Ubermag using pip is not going to install OOMMF. therefore, users have to install OOMMF themselves and set the environment variable which points to the installation path. Alternatively, Docker can be installed and before the simulations are run, a Docker image will be pulled from Docker cloud, container created, simulations run inside the container, results extracted, and container destroyed. This approach allows the maximum flexibility ion terms of the platforms allowed.

Alternatively, Ubermag can be installed via conda on all three major operating systems. This approach installs not only Ubermag, but also OOMMF. Detailed installation instructions are a part of documentation and videos are created and hosted on YouTube.

\subsection{Ubermag in the cloud}

Apart from installing Ubermag on user's machine, Ubermag can be used in the cloud. We provide this service using MyBinder where, in the cloud Jupyter notebooks can be run. Running Ubermag in the cloud was particularly helpful when we delivered the workshop at various international conferences. More precisely, due to the data protection rules, we were not allowed to know the participants' constant information in advance so that we could contact them and provide them installation instruction they should follow before they arrive at the conference. Therefore, we had to deal with a large number of simultaneous installations on different machines before the workshop. This included downloading large installation files, such as Anaconda and long waiting times for the installation to complete. Therefore, we asked all participants who were not able to install Ubermag before the workshop to use Ubermag in the cloud, where we hosted all materials we intended to cover during the workshop.

\subsection{mumax3}

At the beginning of the project, our main focus was to develop a Python interface to OOMMF and integrate it into Jupyter notebooks. After we developed a domain specific language in \texttt{micromagneticmodel}, we decided to demonstrate its universality to other packages. Therefore, we chose to build another micromagnetic calculator for the mumax3 simulation tool. mumax3 is another finite-difference package widely used in the scientific community. Its main advantage is that it runs on GPU and it is usually much faster than OOMMF. Accordingly, mumax3 is a simulation tool of choice when the simulation speed is an important factor. In order to implement another micromagnetic calculator, based on mumax3, we organised a workshop at European XFEL and invited 5 participants. 3 participants were from the research group at the University of Southampton, where the OpenDreamKit project was previously hosted, before moving to European XFEL. Another 2 participants are the developers of mumax3 from the University of Ghent, Belgium. After a 4-day workshop we managed to implement all the necessary functionality to run micromagnetic standard problems.

\subsection{JOOMMF vs Ubermag}

After the implementation of mumax3c, we decided to rename the entire project from JOOMMF to Ubermag. This way, Ubermag provides the top layer functionality which can them use two different micromagnetic calculators in the background. If user wants to use only one computational backend then installing either \texttt{oommfc} or \texttt{mumax3c} would install all the necessary components. On the other hand, to install the entire Ubermag and all its computational backends, installing Ubermag meta-package is necessary.

\section{Dissemination}

During the project, we conducted several workshops at various international conferences. At these workshops we introduced the basics of micromagnetics and through examples demonstrated how micromagnetic simulations can be performed using Ubermag. All tutorials were followed by exercises which participants could solve. The workshops we did are:

\begin{enumerate}
  
\item Institute of Physics (IOP) Magnetism conference, 04 April 2017, York, UK.

  This workshop was held together with Michael Donahue, NIST, USA, who is the main developer and maintainer of OOMMF. In the first half of the workshop, Michael Donahue introduced to the participants the basics of micromagnetics as well as the main capabilities of the OOMMF simulation tool. In the second half, we introduced a Python interface to OOMMF by guiding the participants through several tutorials and letting them to complete the exercises. We had approximately 30 participants at this workshop. Unfortunately, the conference organisers did not allow us to have the details of participants or to take photos of the event due to the data protection regulations. Therefore, we had to ask the participants if they want to provide us their information personally. This workshop was co-funded with EPSRC CCP Computational Magnetism Network (EP/M022668/1)

\item Intermag 2017 conference, 24 April 2017, Dublin, Ireland

  This workshop was held at one of the major conferences in the field of magnetism research. At the workshop we had 50 registered participants, who had to register at the time of conference registration. The maximum number of allowed participants was determined by the conference organisers. The workshop was divided into two parts: (i) main workshop event and (ii) follow up session. At the main workshop event, we taught the participants the basics of micromagnetics and how they can use Ubermag in their every-day research. In the follow-up session, we talked to the participants of the main workshop event as well as to those who wanted to attend, but who could not register due to the limited number of spaces. In the follow-up session we were able to discuss the Ubermag capabilities in more detail as well as to answer any specialised questions current or potential users might have. In both sessions we were able to receive some feedback as well as the requests from users. This workshop was co-funded with EPSRC CCP Computational Magnetism Network (EP/M022668/1).
  
  \item 62nd Annual Conference on Magnetism and Magnetic Materials, 6-10 Nov 2017, Pittsburgh, PA, USA

    Similar to Intermag, this is one of the major conferences in the field of magnetism research. We were not able to organise the workshop jointly with the conference organisers. Therefore, we organised an informal workshop at the conference. More precisely, we invited all interested participants to come and see us at the conference using several micromagnetics oriented mailing lists. At the workshop we had both Ubermag users and researchers who could potentially start using Ubermag. We were able to introduce to all interested participants the main capabilities of OOMMF as well as the benefits of using our Python interface and Jupyter integration. During the workshop, we received feedback from the users as well as the feature requests.
    
  \item Advances in Magnetism 2017, 04-07 February 2018, La Thuile, Italy

    This event was organised as a tutorial session for all conference attendees by the conference organisers. This was structured as an invited talk. During the 30 min talk we explained the basics of micromagnetics as well as the benefits of Ubermag. In addition, we provided enough information for the participants to start using Ubermag on their own. At this event, we had more than 100 participants.

  \item International Conference on Magnetism, 14-20 July 2018, San Francisco, USA

    This workshop was a part of the official conference programme. Jointly with the conference organisers, we organised a workshop which was offered to all conference participants. No limitation on the number of available places was imposed by the organisers. The conference attendees were offered to register for the workshop at the time of conference registration. Similar to the the previous events, we were not allowed have the details of the participants. At the workshop, we had more than 70 participants. At the workshop we had both Ubermag users and researchers who could potentially start using Ubermag. The workshop was divided into two parts: (i) main workshop event and (ii) follow up session. At the main workshop event, we taught the participants the basics of micromagnetics and how they can use Ubermag in their every-day research. In the follow-up session, we talked to the participants of the main workshop event as well as to those who were not able to attend on the first day of the conference. In the follow-up session we were able to discuss the Ubermag capabilities in more detail as well as to answer any specialised questions current or potential users might have.
    
  \item Joint Magnetism and Magnetic Materials - Intermag conference, 14-18 Jan 2019, Washington DC, USA

    Similar to other MMM conferences, we could not organise the workshop jointly with the conference organisers. Therefore, we invited all interested participants to come and see us at the conference using several micromagnetics oriented mailing lists. At the workshop we had both Ubermag users and researchers who could potentially start using Ubermag. We were able to introduce to all interested participants the main capabilities of OOMMF as well as the benefits of using our Python interface and Jupyter integration. During the workshop, we received feedback from the users as well as the feature requests.
    
\end{enumerate}

Due to the restrictions imposed by the conference organisers, we were not able to take photos and get personal information from the participants. Therefore, we asked them to voluntarily do our online survey.

\section{Evaluation}

We evaluated Ubermag by asking users as well as the participants at different workshops to complete our online survey. Basic conclusions about our users/participants and their previous experiences we got from the survey are:

\begin{itemize}
  \item Majority of our users/participants are students (56.1\%), whereas 43.9\% of them are employed. Similarly, the vast majority of them are employed in academia (90.2\%) - much more than in industry and government labs. The age of our participants is 18-34 (78\%) and 35-54 (22\%). 
\item Firstly, we asked both current and potential Ubermag users whether they are active in the field of magnetism research. 92.7\% of participants answered they are. However, from another question if they have ever performed a micromagnetic simulation 65.9\% of participants answered that they have not. Accordingly, we concluded that the majority of participants at our workshops had no previous experience of running micromagnetic simulations and they attended our workshops not only to learn a new simulation tool, but to learn general micromagnetic simulation techniques. After we learned this information, we restructured our future workshops and taught users for the first half of the workshop the basics of micromagnetics. In the second half of the workshop we focused on Ubermag.
\item For those participants, who have performed micromagnetic simulations before, we asked them what simulation tool they have used. More than two thirds answered that they have used or are actively using OOMMF. This confirmed our initial prediction that OOMMF is the most widely used simulation tool at the moment of the project proposal. In the second place (37.5\%) was mumax3. This result supported our decision to demonstrate the universality of our domain based language for the description of micromagnetic problems by adding mumax3 as another back-end.
\item We wanted to know about what computational science tools they have used in the past and how familiar they are with the state-of-the art tools we used to develop our Python interface to OOMMF. The majority (more than two thirds - 68.3\%) of users are working on Windows operating system. In the second place was MacOS (31.7\%) and Linux (24.4\%). These results confirmed our idea of making the installation simple on all three major operating systems by building conda packages. All participants have heard previously about Python and 70.7\% of them have used it before. However, 60\% of all participants do not have enough experience in using Python and consider them to be beginners. Only 12.2\% of users think they are advanced users of Python. These findings confirmed several of our observations at the workshops. Although we built a a much simpler to use interface to OOMMF and integrated it into Jupyter notebook, for new users learning Ubermag is still a steep learning curve. This is partly because they are new to Python, but mostly due to the fact that they have no prior programming experience.
\item In terms of Jupyter notebooks, 63.2\% of our participants have heard about the Jupyter notebook, but only 26.8\% are actively using it. On the other hand, 36.6\% of users have never heard of it.
\item The questions regarding the use of more advanced tools in computational science, such as Docker and Anaconda, the vast majority have never heard of it.
  \item We asked users to give us an overview of what expectation do they have from a new simulation tool they want to employ in their every-day research. In the first place is the ease of use, which confirmed our initial predictions that it is necessary to have a simple, intuitive, and understandable user interface. This was what guided us in the development of Ubermag. Secondly, simple installation was marked as a second most important factor. This, in combination with the computational habits of our users motivated us to make a simple installation procedure based on Anaconda. Last but not least, documentation, visualisation capabilities, and speed were mentioned and some of the factors users consider before they decide to use a micromagnetic simulation tool.
\end{itemize}

In addition to general questions we asked users about their previous experiences in computational science and micromagnetics, we asked users to give a direct feedback on what they like and do not like about Ubermag. The vast majority of participants answered that they think Ubermag is user friendly and simple to use. Among the things they do not like, they say that documentation was not complete which comes from the fact that most of the workshops were held early on in the project and not all documentation was complete at that point. Secondly, users complained about the installation due to certain limitations they experienced on their machines.

\section{Administration}

\subsection{Move from the University of Southampton to European XFEL}

The OpenDreamKit project with principal investigator Hans Fangohr and postdoctoral researcher Marijan Beg moved from the University of Southampton (UK) to European XFEL (Germany) on 1st September 2017. The last day of OpenDreamKit at the University of Southampton was 31st August 2017. The move was initiated by the offer and acceptance of a new position for Hans Fangohr at European XFEL. There were no problems encountered in the relocation process. The facilities at European XFEL are suitable for the completion of OpenDreamKit and the administrative support is provided. Tasks started at Southampton will be continued at European XFEL. There are no parts of OpenDreamKit left at the University of Southampton.

\subsection{Sergii Mamedov}

In addition to Marijan beg, who was fully employed on the OpenDreamKit project, between Oct 2.18 and February 2019 Sergii Mamedov was employed.

%\newpage\printbibliography

\end{document}

%%% Local Variables:
%%% mode: latex
%%% TeX-master: t
%%% End:
