\documentclass{deliverablereport}

\usepackage[T1]{fontenc}
\usepackage[utf8]{inputenc}
\usepackage{color}
\hypersetup{colorlinks,linkcolor=red,citecolor=green,urlcolor=blue}


\deliverable{dissem}{techno}
\deliverydate{??/09/2016}
\duedate{31/08/2016 (M12)}
\author{Erik Bray, Luca De Feo, Viviane Pons, Tom Wiesing}

\begin{document}
\maketitle
\githubissuedescription
\tableofcontents
\newpage

\section{Introduction}
This deliverable reviews emerging technologies that did not exist, or
were not sufficiently visible at the time the \ODK proposal was
written. Its goal is to inform other Work Packages on technologies
that have a potential impact on the achievement of their tasks, and to
suggest corrective actions to be undertaken.  A secondary goal for
this review is to inform the general public about technologies related
to \ODK. Parts of this deliverable appeared, or will appear, on
\href{http://opendreamkit.org}{\ODK's weblog}.

\section{Emerging technologies external to \ODK}
\label{sec:external}

This section is about technologies being developed outside of \ODK.

\subsection{Anaconda}
\label{sec:anaconda}

Anaconda is a free open source Python-based distribution for
scientific computing, powered by the Conda package manager. Anaconda
is also the name of a family of derived solutions sold by Continuum
Analytics, however here we will only refer to the free source
distribution by the name of Anaconda.

Anaconda is already extremely popular, it is thus slightly misleading
to include it in a report on \emph{emerging technologies}. However we
mention it here because of its impact on \ODK and on WP3 in
particular.

One of the main goals of WP3 is to make \Sage and its components
easily packaged and distributed. This is implemented in
deliverables~\longdelivref{component-architecture}{sage-repository}
and~\longdelivref{component-architecture}{sage-distribution}. Because of
Anaconda's size and popularity, its packages are another obvious
target for \ODK. WP3 is already exploring the possibilities to make
\Sage and its components available as Anaconda packages.



\subsection{Docker for Mac and Windows}
\label{sec:docker}


\href{https://www.docker.com}{Docker} is an emerging technology for
packaging software in so-called containers that can run processes
isolated from the rest of the operating system. It is the midway point
between a complete virtual machine for a single system and running a
process with limited user rights.

Docker provides so-called software images that package software. These
images can then be executed inside a docker container. Furthermore,
Docker provides a service called
\href{https://hub.docker.com/}{DockerHub} that allows users to upload
and publish their own images. This allows developers to bundle their
software and distribute it easily. Furthermore, since each container
runs isolated from the rest of the systems, the developers do not have
to rely on any kind of other system configuration.

Within the \ODK project this allows us to bundle mathematical software
systems (such as \GAP, \Sage, etc) and distribute them in a manner
that is both accessible to users and easily maintainable for
developers. Furthermore the isolated aspect of docker containers
allows us to easily integrate multiple systems together without having
to make additional assumptions about the user's specific setup --- we
can just run all systems in one docker container. This way users can
install the entire mathematical software stack that the \ODK project
aims to provide easily.

Docker was originally a Linux only application --- it relied on a lot
of functionality provided by the Linux kernel. To make it available on
Windows and Mac the developers provide a virtual machine, called
\href{https://www.docker.com/products/docker-toolbox}{Docker Toolbox},
that runs a streamlined Linux system with Docker pre-installed. This
makes it possible for Windows and Mac users to run docker containers,
however it introduces an additional layer of abstraction that comes
with some disadvantages. The additional virtualisation slows down
docker containers and faces technical limitations when wanting to
integrate with the host system. It also requires users to install
virtualisation software before being able to run any kind of Docker
Image. Even though Docker Toolbox automatically installs
\href{https://www.virtualbox.org/}{VirtualBox}, this is a separate
application that adds load to users machines.

Recently Docker started to build native versions for Windows and
Mac. These versions do not rely on Linux functionality --- instead they
leverage functionality provided by Windows and OS X operating systems
natively. In terms of the \ODK project these are a big step in terms
of usability --- they make it significantly easier for users to run a
Docker container. Users can now install Docker just as they would
install any other software on their machine. Among speed and resource
advantages, these versions will make it easier for developers to
create docker-powered applications and Docker containers because of
better integration between host system and containers, for example the
file system of the physical machine can be mounted inside containers
more easily. As a side note, the Windows version of Docker only works
on Windows 10 Professional and Enterprise editions and requires some
manual configuration --- a setting in the BIOS has to be changed (for
more information see
\href{https://docs.docker.com/docker-for-windows/#/what-to-know-before-you-install}{Docker
  for Windows --- Getting Started Documentation}). This is much less
effort than was required previously, however it is not quite ready for
adoption by inexperienced users yet. The Docker developers have stated
clearly their intention to make the Docker experience in Windows and
OSX as easy and streamlined as it is for Linux. When the users will be
able to run Docker without the need for manual configuration or a
high-end edition of Windows, we expect many components of \ODK to be
avaible via Docker containers on Windows and OSX as easily as they are
now for Linux.

Docker for Windows and Mac has been in a private Beta since March 2016
and has recently become available as a public Beta. Interested
readers can find more information on the
\href{https://blog.docker.com/2016/06/docker-mac-windows-public-beta/}{Docker
  Blog}.

When this technology reaches maturity, it will impact
deliverables~\longdelivref{component-architecture}{virtual-machines}
and~\longdelivref{component-architecture}{portability-cygwin}.
\delivref{component-architecture}{virtual-machines} is already
delivered; the recommended action for its continuous maintenance is to
add Windows and Mac Docker containers to the ones already
distributed. The recommended action
for~\delivref{component-architecture}{portability-cygwin} is to
reconsider Cygwin as a platform for distributing one-click installs of
\Sage on Windows; this recommendation has already been enacted:
progress on an experimental Docker-based installer is being tracked at
\url{https://github.com/sagemath/docker-images/issues/1}.


\subsection{Windows Subsystem for Linux}
\label{sec:winix}

One of the goals of the \ODK project is to improve support for open
source mathematics software on a wider range of hardware platforms and
operating systems (see \taskref{component-architecture}{portability}).
Among the largest portability challenges is improving installation and
operation of such softwares on Microsoft Windows---still the dominant
OS in many user communities, especially on desktop and laptop
computers.  Despite there being many large communities of Windows
users, most open source software developers have traditionally
preferred UNIX-like software development environments.  The UNIX
environment differs in many significant ways from Windows, such that
support for Windows has often been neglected by those developers.


\subsubsection{Introducing Windows Subsystem for Linux}

In late March of 2016, at its annual developers' conference, Microsoft
announced a surprising new technology.  Dubbed
\href{https://msdn.microsoft.com/commandline/wsl/about}{Windows
  Subsystem for Linux} (WSL), this new feature premiering in the
Windows 10 ``Anniversary Update'' would add a Linux system call
compatibility layer to the Windows NT kernel, and a Windows-native
port of the popular ``bash'' shell.  And furthermore, in partnership
with Canonical, creators of the popular
\href{http://www.ubuntu.com/}{Ubuntu} Linux distribution, the WSL
supports Ubuntu's ``apt'' package repository, giving Windows users
access to a large swath of open source software built for Ubuntu, but
running directly on Windows.

In short, what this means, is that Windows users will now have a
Microsoft-supported Unix-like shell environment, and the ability to
run Linux-based software directly on Windows, without a virtual
machine.  This would have been unthinkable to most even a decade ago.


\subsubsection{Why porting UNIX software to Windows is hard}

Software that is compiled from languages like C and C++, often favored
by researchers, is generally built in such a way that the compiled
\emph{binaries} support a specific operating system.  Each OS has a
particular \emph{binary format}---the way the program is organized on
disk and copied into memory at runtime.  So any compiled software
built for that OS has to be arranged in the binary format for that OS
in order for the OS to know how to interpret and execute it.  It is
not typical for one OS to be able to understand binaries for another
OS.  For example, software built for Linux uses the
\href{https://en.wikipedia.org/wiki/Executable_and_Linkable_Format}{ELF}
binary format; normally if one tried to run a program built for Linux
on Windows, which only understands the
\href{https://en.wikipedia.org/wiki/Portable_Executable}{PE} format,
it will not be recognized as a valid executable.

An even deeper complication to writing portable software is the system
calls--- software run by users interacts with the operating system to
perform low-level operations such as writing to disk, or making
network connections, through special functions provided by the
operating system called ``system calls''.  Modern UNIX-like operating
systems follow, to an extent the
\href{https://en.wikipedia.org/wiki/POSIX}{POSIX standard} for system
calls, allowing them to be generally more interoperable.  Windows, on
the other hand, has its own system call defitions that are not
necessarily in one-to-one correspondence with POSIX system calls.  As
such, a program built for Linux has no idea how to communicate with a
Windows operating system.

This can be a problem even on higher-level interpreted languages like
Python.  Although code writing in Python abstracts away most operating
system differences, Python code \emph{can} still access OS-specific
features such as system calls, and this is sometimes necessary to
access more advanced OS features needed by some scientific software.
So Python code that uses Linux-specific features, for example, can
only run on a version of the Python \emph{interpreter} built for
Linux.

A third difficulty has to do with minor differences in user interface
standards.  For example, a common issue in Windows support is its
different standard for representing file paths.  While Windows paths
contain a ``drive letter'' and uses the backslash
(``\texttt{\textbackslash}'') to separate between folders (e.g.
\texttt{C:\textbackslash{}Windows\textbackslash{}cmd.exe}), UNIX-like
systems have no concept of a ``drive letter'', and use forward-slashes
(``\texttt{/}'') (e.g. \texttt{/bin/bash}).  Issues like this can cause
many small, but pervasive bugs when porting software between operating
systems.


\subsubsection{How WSL gets around it}

The Windows Subsystem for Linux does two main things:

\begin{enumerate}
\item It enables with Windows NT kernel to understand the ELF binary
  format, and \emph{translate} it, as closely as possible, to the
  binary format used by Windows.
\item It implements a sizeable subset of the POSIX system call
  standard on top of Windows.  Although Windows' own system calls do
  not map directly the POSIX, because Microsoft has access to how its
  underlying operating system is implemented, they are able to
  implement the POSIX interface on top of the lower-level details of
  their NT kernel.
\end{enumerate}

WSL also provides its own \emph{bash} shell---a command-line interface
favored by many users of Linux.  This provides a UNIX-like
command-line interface within Windows, also has an underlying system
for transparently translating things like file paths between the
Windows and UNIX formats.

The ultimate goal is to be able to take a program compiled and built
on a Linux system, copy it over to Windows, and allow it to run
without any modifications, with all the system-level translations
completely transparent to the user.  Targeting Linux software
\emph{specifically} makes this possible, because the system interfaces
it will use are well-specified \emph{predictable} in most cases.  This
is as opposed to running a virtual machine, in which an entire
separate operating system is run in order to run software on that OS,
and which needs to be able to run any arbitrary OS.

This is direct support for Linux software in Windows itself---there is
no virtualization.

This is also an improvement over previous efforts at supporting Linux
software on Windows, such as \href{https://www.cygwin.com/}{Cygwin}.
Because Cygwin is third-party software it cannot modify the Windows NT
kernel itself.  It does not support ELF binaries---to run software
with Cygwin it has to be \emph{recompiled} to the native PE binaries
understood by Windows.  It also does its best to provide emulation of
POSIX \emph{system calls}, but it has to do this by building them on
to of the NT system calls which, as noted above, is not a one-to-one
mapping.  WSL, on the other hand, provides support directly from the
operating system for POSIX and other Linux system calls.


\subsubsection{What it means for \ODK}

Because WSL allows binaries built for Linux to run directly on
Windows, it makes much of the enormous repository of software built
for Ubuntu (and potentially other Linux distributions) immediately
available to run on Windows.  No recompilation has to be performed or
anything (at least, that is the goal---as we'll see below it is still
not fully realized).

For example, Ubuntu's software repository already includes builds of
many of the packages that are central to \ODK, such as
\href{http://packages.ubuntu.com/trusty/gap}{\GAP},
\href{http://packages.ubuntu.com/trusty/pari-gp}{\Pari}, and some
smaller packages including many of the dependencies of \Sage.  \Sage
itself has an unofficial Ubuntu package---this has been found so far
to nominally ``work'' on WSL, but there have been found to be many
bugs.  That said, a great deal of other mathematical
software---especially that which is less dependent on OS-specific
features, should already work out of the box.

An additional potential advantage for WSL (indeed, one of the
project's goals as detailed in an Ars Technica
article\footnote{\url{http://arstechnica.com/information-technology/2016/04/why-microsoft-needed-to-make-windows-run-linux-software/}})
is to make the development tools and command-line interfaces favored
by UNIX-oriented developers available on Windows.  This makes it
possible, in principle, to develop software like \Sage the same way on
both Windows and Linux.

In some sense this could be an end-run around \ODK's goal of better
supporting Windows---Microsoft has already done the lion's share of
the work for us.  But there is more to be done, and it may not be an
end-all be-all solution.


\subsubsection{Caveats}

As mentioned in the previous section, while some \ODK software has
been found to work in WSL, it is not without issues.  Many bugs were
found in running \Sage on WSL (and even more when trying to compile
it).  This is not unexpected---the current release is marked
``\href{https://msdn.microsoft.com/en-us/commandline/wsl/install_guide}{beta}''
by Microsoft, and they fully acknowledge that it is buggy and
incomplete.

Second, Microsoft has made it clear in several
statements\footnote{\url{https://blogs.windows.com/buildingapps/2016/03/30/run-bash-on-ubuntu-on-windows/}}
that the WSL and ``Bash for Windows'' are to be considered tools for
developer convenience \emph{only}.  It is not intended for use in a
server infrastructure nor, presumably, as a means of
distributing/installing software for end-users (i.e.\ who are agnostic
about how the software is implemented).  Although one could take the
cynical view that this just Microsoft's way of protecting its own
server products, there are also some practical reasons for this:

\begin{enumerate}
\item As a developer tool, the WSL + Bash for Windows are not easy for
  casual users to install.  First, it is only available on Windows 10
  with the recent (as of writing) ``Anniversary Update''.  Not all
  users are on Windows 10 yet.  It also requires having an account on
  Microsoft's developer network, and for their Windows to be
  configured to ``developer mode'' in order to receive
  development-related updates, plus a few extra steps.  This can also
  involve some sizeable downloads.  This is not especially onerous for
  a developer, but is not a serious of steps that can or should be
  asked of the ``casual'' or first-time user just to install some
  software.
\item Despite having support directly in the kernel, the WSL is
  something of a walled garden.  It is not possible to run native
  Windows applications from within the Windows \emph{bash} prompt.
  Nor is it possible (in any transparent sense) to interact with Linux
  applications from native Windows applications.  This is probably
  required, on some level, to maintain a clean abstraction.
\end{enumerate}

Finally, it is not currently supported to run GUI applications on top
of WSL, in part because that requires a lot more than just system call
compatibility.  While not supported officially by Microsoft, some
hobbyists have made progress on it though, by integrating with
existing X server implementations for
Windows\footnote{\url{http://www.pcworld.com/article/3055403/windows/windows-10s-bash-shell-can-run-graphical-linux-applications-with-this-trick.html}}.
For many mathematical softwares this is a non-issue---they are text
based: numbers in; numbers out.  Additionally, graphical interfaces
for interactive research environments are increasingly moving to the
web (see for example
\href{http://opendreamkit.org/activities/2016-08-30_SMC}{\SMC}).  In
such cases the GUI elements have been moved out to the web browser and
the backend typically runs ``headlessly''---it has no reliance on the
system's desktop interface.


\subsubsection{Conclusion}

The Windows Subsystem for Linux represents a major step in the right
direction for Microsoft.  It shows that they are listening to the
needs of the broader software developer community (not just those who
work exclusively on Windows) and that they have some interest in
cooperating with the open source software community (this has also
been demonstrated in several other ways in recent years).

For the purposes of \ODK, this work will make \emph{development} of
open mathematical software more accessible to a wider community.
Although this may not improve accessiblity for casual end-users, many
users of open research software tend to become \emph{de facto}
developers as well, as the more they use the software the more
interested they become in modifying it for their own purposes.  Making
it possible for Windows users to do development on otherwise
UNIX-oriented software, without leaving their personal desktop
environments, is appealing.  Being able to compile one's own software
is also important for some highly optimized numerical software, which
tunes itself at compile time to the computer it is being built on,
sometimes with dramatic results.

In conclusion, we do not recommend any corrective action on \ODK's
deliverables. Although WSL does not yet provide a fully reliable
immediate solution for porting \ODK software to Windows, we encourage
partners involved in \taskref{component-architecture}{portability} to
keep an eye on it as it evolves.


\subsection{Cap'n proto}
\label{sec:capnproto}

\href{http://capnproto.org/}{Cap'n proto} is a multi-language
serialization protocol and toolkit, providing \emph{zero-cost
  encoding/decoding}. Cap'n proto is an open source project of
\href{http://sandstorm.io}{Sandstorm.io}.

Cap'n proto works by fixing a portable, efficient, memory layout for
its data structures. This way, data can be serialized and transferred
simply by coping the raw data in memory. Cap'n proto ships with an
official C++ implementation, and many contributed implementations in
other languages (notably C, Python, Java, ...). Each implementation
gives access to Cap'n proto data structures through the language's
native APIs, thus abstracting away all the protocol's complexity and
making data access very efficient.

Serialization is a core component of any complex system. It allows
communication \emph{inter-process}, \emph{inter-node}, and across
time. The specific design of Cap'n proto has the potential to allow
even \emph{inter-language} shared-memory communication inside the same
process, something that is usually done in a language-dependent and
non-portable way.

Because of its focus on efficiency, Cap'n proto has potential
applications in WP3 and WP5. We recommend the partners involved in
these work-packages to closely follow this technology.


\subsection{Binder}
\label{sec:binder}

\href{http://mybinder.com}{Binder} is a free cloud service that lets
users define a \emph{running environment} (e.g., by a Docker file, a
Python \texttt{requirements.txt} or a Conda environment), and obtain a
link to a cloud instance of the running environment, together with a
Jupyter frontend.

This is similar to the \href{http://tmpnb.org/}{tmpnb} service, except
that the repository owner defines the running environment.  The
computing power for the default instance is provided by The Freeman
Lab at HHMI Janelia Research Campus to support open science. However
Binder is open source and could be deployed elsewhere.

\paragraph{Relevance to \ODK} Binder is a promising approach to make
it as easy as possible for our users to share publicly their Jupyter
notebooks. \ODK could help by:

\begin{itemize}
\item Contributing ready-to-use environment descriptions for our
  favorite sofware (\GAP, \dots) Docker containers developed in
  \delivref{component-architecture}{virtual-machines} are probably a
  good starting point.
\item Finding infrastructure support (universities, EGI, \dots) to run
  more instances of Binder.
\end{itemize}


\section{GÉANT Open Education Resource project}

The goal of \longdelivref{dissem}{ils-tool} is to provide a (open source)
community-curated indexing tool for resources (documentation,
tutorials, courses, notebooks, \dots) related to mathematical software.

\href{https://oer.geant.org/}{eduOER} is a searchable,
metadata-driven, multilingual, indexing service for educational
multimedia content. The eduOER service is being developed by the
``Real Time Communications and Multimedia Management'' service
activity of the GN4-1 project partly funded by the European
Commission. Its alpha version was released in March 2016.

There is a clear overlap between the goals of
\delivref{dissem}{ils-tool} and eduOER, which could justify offloading
the contents of \delivref{dissem}{ils-tool} to eduOER. However, there
are also some major differences:

\begin{itemize}
\item{Role:} eduOER is an aggregator (it aggregates metadata from
  third-party repositories), \delivref{dissem}{ils-tool} is a
  repository (of URLs + metadata).
\item{Content:} eduOER is audio-video only (although support for other
  contents may be added in the future), \delivref{dissem}{ils-tool} is
  text-oriented (although any format is supported in principle).
\item{Metadata:} eduOER aggregates metadata from participating
  repositories, metadata in \delivref{dissem}{ils-tool} is
  user-generated.
\item{Search:} eduOER offers search on metadata,
  \delivref{dissem}{ils-tool} requires metadata and full-text
  search. However eduOER can perform full-text search if
  full-text-extraction is provided as metadata.
\item{Social:} eduOER has no social interaction (however the frontend
  component of eduOER has some planned social
  features). \delivref{dissem}{ils-tool} is community curated in the
  sense that entries are reviewed, commented and scored by humans.
\end{itemize}

We recommend that the partners involved in \delivref{dissem}{ils-tool}
keep surveying the state of eduOER, in view of a possible partial or
total adoption of the technology, or at least in view of
interoperability.

%%%%%%%%%%%%%%%%%%%%%%%%%%%%%%%%%%%%%%%%%%%%%%%%%%%%%%%%%%%%%%%%%%%%%%%%

\section{Emerging technologies internal to \ODK}
\label{sec:internal}

This section being about technologies developed internally by \ODK,
its goal is mainly to inform the general public. However, some task
leaders may get some useful insights on technologies they have only
been following from a distance.

\subsection{JupyterLab}
\label{sec:jupyterlab}

At the SciPy 2016 conference, Brian Granger and Jason Grout presented
the next generation of the Jupyter Notebook application: JupyterLab.
The presentation was followed by a post on Jupyter's
blog\footnote{\url{http://blog.jupyter.org/2016/07/14/jupyter-lab-alpha/}}.
JupyterLab is in pre-alpha stage, and is available on
GitHub\footnote{\url{https://github.com/jupyter/jupyterlab}}.

\subsubsection{What is JupyterLab?}

JupyterLab captures a lot of what has been learned from the usage
patterns of the Notebook application over the last 5 years and seeks
to build a clean and robust foundation that will offer not only an
improved user interface and experience, but also a flexible and
extensible environment for interactive computing.

Today's Notebook application includes not only support for Notebooks
but also a file manager, a text editor, a terminal emulator, a monitor
for running Jupyter processes, an IPython cluster manager and a pager
to display help.  

But the underlying code is not the cleanest to extend and providing a
more responsive and flexible UI atop it is difficult. JupyterLab is a
next-generation architecture to support all of the above tools, but
with a flexible and responsive UI that adapts easily to multiple
workflow needs, thanks to its user-controlled layout that ties
together many tools under a single roof.  The entire JupyterLab is
built as a collection of plugins that talk to kernels for code
execution and that can communicate with one another.

\subsubsection{JupyterLab in \ODK}

The way JupyterLab is being built enables building different
applications, such as making other non-notebook webpages (e.g.\
documentation) interactive. This fits the \ODK philosophy perfectly:
rather than building one unique VRE, JupyterLab ships a modular
environment to build VREs tailored to each user's needs.

JupyterLab has the potential to be the \emph{one-size-fits-all}
standard for graphical user interfaces delivered by \ODK. Given the
very large projected user base of JupyerLab, this will make adoption
\ODK products easier for end-users. WP4 (user interfaces) will be
especially involved in integrating JupyterLab into its demonstrators.

We strongly encourage all partners working on delivering a fully
integrated VRE, such as \SMC (see next section), to keep assessing the
maturity of JupyterLab, and its potential to replace their currently
planned UI.


\subsection{\SMC}
\label{sec:SMC}

Part of \ODK's mission is to work on user-interfaces for better
collaboration and also component architectures.  This is why the
\href{http://cloud.sagemath.com/}{\SMC} platform is of special
interest for us. The goal of
\taskref{component-architecture}{extract-smc} is even to have a deeper
look into its code base.

\subsubsection{What is \SMC?}

\SMC is an online platform which allows the creation of
\textbf{collaborative scientific projects} including many scientific
softwares and tools like \href{http://www.sagemath.org/}{\Sage},
\href{http://jupyter.org/}{Jupyter},
\href{https://www.scipy.org/}{SciPy},
\href{http://julialang.org/}{Julia},
\href{https://fr.wikipedia.org/wiki/LaTeX}{Latex}, and more.

Its codebase is
\href{https://github.com/sagemathinc/smc}{open-source}, distributed
under the GNU General Public License. The platform is run by a private
company (SageMath Inc.) created by William Stein who is also the
initiator of the \Sage software. The platform offers both free and
paying premium accounts.

\paragraph{Projects}

The main tool of the \SMC platform is the possibility to create
\textbf{projects} from which you can access the many features. A
single user can create as many projects as needed. Each project is an
\textbf{independant Linux virtual machine}. It thus comes with a full
file system and an \textbf{online terminal} that allows you to run
Linux commands. The storage of each project is limited by default but
can be extended on premium accounts. You can access the files through
the \SMC web interface or also through ssh.

One key feature is that each project can be \textbf{shared by multiple
  users}. This allows sharing access to the files and also
\textbf{real time editing} though the platform. Single files or
folders can also be made \textbf{public}. A link is then provided
which allows either viewing or downloading the files (even without a
\SMC account) and also an easy way to copy onto a different \SMC
project owned by the viewer.

\paragraph{Softwares}

When you create a \SMC project, your Linux virtual machine comes with
many softwares and tools especially useful for mathematicians and
scientists in general.  We list here the most important ones.

\begin{itemize}
\item \Sage and \Sage \textbf{worksheets}. As the name indicates, the
  platform was primarily developed as a replacement for the old \Sage
  notebook server to allow collaborative online work using \Sage. The
  \Sage software is of course installed by default on the virtual
  machine and one can run \Sage through the online terminal. The
  platform also offers its own \Sage \textbf{worksheet filetype} to
  edit and run \Sage code in a cell-type system (as in the Jupyter
  notebook or the old \Sage notebook) mixed with other cell types like
  text and HTML. This is used to create interactive worksheets that
  can be easily shared and copied.
\item \textbf{Jupyter}. \SMC includes a Jupyter notebook interface
  with many kernel options (Python 2, Python 3, Anaconda, \Sage, R,
  Julia, and more). On top of the usual interface, the \SMC Jupyter
  offers \textbf{real time} synchronization among multi users.
\item \textbf{Latex}. The common document preparation system Latex is
  installed on the virtual machine. It also offers a multi user editor
  with real time synchronization and a dual view of both the Latex
  source code and pdf output.
\end{itemize}

\subsubsection{Sharing and teaching with \SMC}

\paragraph{Accessibility}

The great advantage of \SMC is that it offers a \textbf{complete
  scientific environment} without the usual setting up hassle. It
makes the different softwares very easy to access independently of the
user personal system as long as there is an access to a good Internet
connection. As an example, a mathematician can share a demo of code (in
a Jupyter or a \Sage notebook) that could be used directly by its
collaborators. Of course, the Internet access is itself a limit. For
example, the system is not well suited for developing countries where
bandwidth is sometimes limited.

\paragraph{Teaching}

When teaching is concerned, the sharing facilities of \SMC come very
useful.  Moreover, the platform offers a course managing system. The
principle is as follows: the teacher has acces to a ``main project''
containing the class material; every student has its own project which
is shared with the teacher. The course management system allows for
automatic actions like:

\begin{itemize}
\item Create all the student projects where the teacher is
  automatically added as a collaborator.
\item Create assignments by copying some material from the main
  project to the students projects.
\item Collecting, grading, and returning assignments by copying back
  and forth between the students projects and the main project.
\end{itemize}

An \emph{assignment} is just a folder. It can have multiple content
depending on the class.  Of course, the system is especially
interesting when the assignment is given within an \textbf{interactive
  worksheet} and can then be achieved by the student directly on the
interface. \SMC then becomes a very good interface to initiate
students to the many scientific softwares it offers.

\subsubsection{\SMC and \ODK}

The many features of \SMC make it a very interesting project for \ODK
to look at. Indeed, it offers one of the leading technologies for
scientists in terms of cloud project management, teaching and sharing
facilities. It also has some limits which we would like to address
through our project:

\begin{itemize}
\item \textbf{Accessibility}. As previously mentioned, the cloud based
  interface can not be easily accessed in places where the Internet
  connection is not good enough. One solution would be to have clear
  easy-to-follow instructions on how to install a \SMC platform in a
  local institution. This is taken care of in
  \longdelivref{component-architecture}{smc-documentation}.

\item \textbf{Interoperability and file formats}. At the moment, the
  \SMC platform offers two file formats for interactive worksheet: the
  Jupyter one and a home-made \Sage worksheet one. It is not possible
  to run the \Sage worksheets elsewhere than on the
  platform. Especially, there is no way to run a \Sage worksheet on a
  local \Sage installation. It is not yet clear what a long term
  unified worksheet solution would be and it is part of the \ODK
  project to work on this question. The technical choices made for the
  \Sage worksheets are interesting to investigate in this regard, as
  well as local \SMC installations (see
  \delivref{component-architecture}{personal-smc}), file conversions
  and so on.
\end{itemize}


\end{document}

%%% Local Variables:
%%% mode: latex
%%% TeX-master: t
%%% End:
