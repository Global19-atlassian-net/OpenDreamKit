\documentclass{deliverablereport}

\deliverable{dissem}{press-release-2}
\deliverydate{XX/YY/201Z}
\duedate{31/08/2019 (M48)}
\author{Izabela Faguet, Viviane Pons, Nicolas M. Thiéry}

\begin{document}
\maketitle
% This will be the abstract, fetched from the github description
\githubissuedescription

% write the report here

\section Executive summary

Press releases were included in the Workpackage 2 and are an important 
part of our communication stategy. With the press releases we plan to call
attention of multiple audiences about our research in a way understandable
by non-specialists and address the EU perspective of research and innovation 
funding, by considering some aspects such as the European transnational 
cooperation in the consortium and the scientific excellence with a better use
of results to the scientific community.

During the first months of the project, we have covered 6 press releases and we
planned to do the same at the end to promote the actions and the results by providing
targeted information to broad audience ( including the media and the public).
First we will submit a general communication about the project on our website presenting 
the results of the assessment made by the referees and the Project Officers during our 
final review meeting.

\section{Collaborative process}

The press releases will be the result of a joint effort of the partners who are actively
involved in the promotion of the press releases; their collective effort will result in
at least 12 releases in general press in the high education, research area and local press.
We also followed the bottom up aprroach even to prepare the article that was written completely
in the open. % insérer schéma

\subsection{Press releasing process}

 1. The coordinator have already contacted the Press Office of UPSud to inform about the topic
 of the project that was already approved by the Communication officer. 
 2.The UPsud communication department was in charge of preparing a general text that will be sent
 to the partners and that could be adapted at their convenience. We received the article proposal 
 and some changes were made. it has been translated in English and sent to the partners 
 for distribution to their respective Press offices. 
% insérer une annexe avec l'article et insérer un lien de renvoi
 3.The approved release will then be sent to relevant journalists contacts of each institution 
 and uploaded to the newswire services. The press releases will be published by end of november. 
 Each partner still needs to negociate with their respective communication departments the 
 definitive deadline for publication.



\subsection{other Communicative material}

 \item blogposts about the project results
 The press releases are a general dissemination about the project, and addressed to 
 a non specialist audience. We also plan to submit a second type of communicative released 
 information that will present more specificall the results achieved in the project.
 We plan to publish a blogpost that will be in the form of the summary for publication
 provided for each periodic report but more longer and  extensive, it will present the 
 major achievements and milestones of the project. It will be released and published in 
 the project website and delivered to mass media. This article will be written by Upsud 
 with the technical contribution of the ODK partners. 
 
 
 \European Commission's channels
 
The Coordinator has contacted the communication officer  who will support the dissemination
of the results of the project ad will promote them through the EC's communication channels. 
The communication officer will publish an article and a presentation of the project in the 
newsletter ” Digital Excellence & Science Infrastructure “ of the European Commission, 
on the European Commission website “https://ec.europa.eu/digital-single-market/” to reach 
a wider audience, potentiating it outreach. 
HORIZON, the EU Research & Innovation e-magazine, will also be used to inform about 
the benefits and progress that ODK will generate in Europe, informing about 
the open debates created and the results.

\end{document}

%%% Local Variables:
%%% mode: latex
%%% TeX-master: t
%%% End:

