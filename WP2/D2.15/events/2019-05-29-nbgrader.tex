\begin{event}{nbgrader hackathon workshop in Edinburgh}{nbgrader}{Edinburgh, May
    29th-311st}{PS}{10}{1}{https://blog.jupyter.org/https-blog-jupyter-org-university-of-edinburgh-jupyter-community-nbgrader-hackathon-2eff14df298a}

  \textbf{Event summary.} Within the Jupyter Community Workshop series
  funded by Bloomberg, the University of Edinburgh hosted a three-day
  event. Its core aspect was a hackathon focused on adding
  improvements, fixes and extra documentation for the course
  management tool \texttt{nbgrader}. Alongside this it held an
  afternoon of talks highlighting how Jupyter can be used in education
  at varying levels.

  \textbf{ODK implication.} ODK participant Nicolas M. Thiéry was
  invited to participate to the hackathon and deliver a talk on
  Jupyter for teaching introductory programming.

  \textbf{Results and impact.} The hackathon resulted in many
  improvements to nbgrader, including some contributions by Nicolas,
  that make it easier to integrate in a variety of environments.

  Nicolas used the occasion to visit St Andrews, for brainstorms on
  upcoming deliverable reports and collaboration on Sage-GAP
  interfacing. This visit also led to the invitation of Nathan Carter
  to Sage Days 101 for collaboration on mathematical visualization and
  live data browsing in Jupyter.
\end{event}
