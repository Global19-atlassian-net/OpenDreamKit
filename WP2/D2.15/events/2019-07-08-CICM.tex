\begin{event}{CICM2019: Conference on Intelligent Computer Mathematics, in Prague}{CICM2019}{Prague, Czech Republic, 08th-12th of July 2019}{FAU}{65}{7}{https://cicm-conference.org/2019/cicm.php}

\textbf{Main goals.}
Digital and computational solutions are becoming the prevalent means for the generation, communication, processing, storage and curation of mathematical information.
Separate communities have developed to investigate and build computer based systems for computer algebra, automated deduction, and mathematical publishing as well as novel user interfaces.
While all of these systems excel in their own right, their integration can lead to synergies offering significant added value.
The Conference on Intelligent Computer Mathematics (CICM) offers a venue for discussing and developing solutions to the great challenges posed by the integration of these diverse areas.

\textbf{\ODK implication.}
Florian Rabe (FAU) was a member of the steering committee.
A workshop was co-organized by Dennis M\"uller (FAU).
There were no costs to \ODK other than travel costs for \ODK members.

\textbf{Event summary.}
CICM featured 3 invited talks, 5 days of presentations, a doctoral program, 3 workshops, and 1 tutorial.

\textbf{Results and impact.}
\ODK members presented about 10 talks at the conference, various workshops, and the doctoral programme, describing various \ODK results.
This triggered a number of discussions with researchers from adjacent fields in computer science as well as a few mathematicians.

Florian Rabe and Yasmine Sharoda (who visited FAU for one month during \ODK) were awarded the Best Paper Award for his paper on diagram operators --- a new method for organizing large libraries of mathematics that was motivated by \ODK.

Sylvain Corlay (QuantStack) gave an invited talk on ``Jupyter: From IPython to the Lingua Franca of Scientific Computing''.
Jupyter being central to \ODK, this talk mentioned various technologies related to \ODK results.

Makarius Wenzel gave an invited talk on ``Interaction with Formal Mathematical Documents in Isabelle/PIDE'', in which he also presented some of the technologies for which he was subcontracted by \ODK.
\end{event}
