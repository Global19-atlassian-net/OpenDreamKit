\begin{event}{CICM2019 -- Workshop on Large Mathematics Libraries}{LML2019}{Prague, Czech Republic, 08th-12th of July 2019}{FAU}{35}{7}{https://cicm-conference.org/2019/cicm.php?event=lml&menu=general}

\textbf{Main goals.}
Large formal and semiformal mathematics libraries are needed to support mathematics research, mathematics education, rigorous software development, and formal proof development.
This workshop explored methods for designing, constructing, and maintaining large mathematics libraries as well as for finding, comparing, and applying the knowledge residing in these libraries. 

\textbf{\ODK implication.}
The workshop was co-organised by Dennis M\"uller (FAU).
Florian Rabe (FAU) was a member of the programme committee.
Michael Kohlhase (FAU) gave a talk presenting \ODK results.

\textbf{Event summary.}
The workshop featured two invited talks by Makarius Wenzel and Claudio Sacerdoti Coen on making large libraries (Isabelle resp. Coq) available for system integration.
It also included contributed talks on Logipedia by the Dedukti group, which closely collaborates with \ODK members, and mathematical datasets.

\textbf{Results and impact.}
The main result of the workshop was to strengthen the awareness of and to support the community for managing large formal libraries.
This has been a central topic in WP 6 of \ODK, and the discussions allowed presenting \ODK results.

In particular, the talks be Wenzel and Sacerdoti Coen presented results that were developed in collaboration with \ODK members in WP 6.
\end{event}
