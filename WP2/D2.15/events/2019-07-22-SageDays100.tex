\begin{event}{Sage Days 100, University of Bonn}{SD100}{Bonn (Germany), July 22 -- July 27, 2019}{PS}{25}{6}{https://opendreamkit.org/2019/07/22/SageDays100/}

\textbf{Main goals.} The Sage Days 100 workshop gathered a variety of public
(from master student with no prior knowledge
of Sage, to senior lecturer with 10 years of contribution to 
\Sage source code). The goals were to introduce \ODK softwares
to beginners, introduce good practice software development for
most advanced users and have programmers present their specialized
packages.

\textbf{\ODK implication.} The event was organized by Vincent Delecroix.
The \ODK member Samuel Leli\`evre participated in the event and
provided individual help to participants. The on-site facilities and
logistics were funded by the University of Bonn. \ODK funded the
travel and lodging of 11 of the participants that came from across
Europe (France, Germany and United Kingdom).

\textbf{Event summary.} The week was organized so that half of the
time was left for people to work on their own project while advanced
users can answer questions. During the tutorial sessions we had
a mix of specialized package presentation, learning development tool
(control version system, testing, debugging, etc). We also had a brief
experience with pair programming where an advanced programmer was
paired with a beginner to solve a challenge.


\textbf{Demographics.} Out of approximately 25 participants approximtely
half of them where from Germany and the other half from neighbouring
countries (Switzerland, France, United Kingdom). We had an almost parity
of male and female participants which is rare in software development
workshops in general. The level of studies spread from one sixth master students,
one sixth PhD students, one sixth postdocs and one half of senior participants
(professor or associate professor, retired engineer).


\textbf{Results and impact.} One of the primary goal of the workshop
was to introduce \Sage to students. I believe that we were successful
given the feedback we obtained from participants. All of the participants
answered to have learn something.

\end{event}
