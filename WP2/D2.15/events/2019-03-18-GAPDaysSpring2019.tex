\begin{event}{GAP Days Spring 2019, Martin Luther University of Halle-Wittenberg}
  {GAPDaysSpring2019}
  {Martin Luther University of Halle-Wittenberg (Germany), 18--22 Mar 2019}
  {SA,PS}
  {22}
  {2}
  {https://www.gapdays.de/gapdays2019-spring/}
  
\textbf{Main goals.} This was a meeting for experienced GAP developers and users
to discuss and contribute to the GAP project.  The overarching theme was
algorithms for permutation groups.

\textbf{\ODK implication.} Michael Torpey and Markus Pfeiffer attended, and
worked on software while learning more about GAP and permutation groups.

\ODK participants: M.~Torpey and M.~Pfeiffer

\textbf{Event summary.} This was a meeting for experienced developers and users
of GAP to discuss and influence the future development of GAP.  There was also
an overarching theme of algorithms for permutation groups, with discussions and
exercises relating to stabiliser chains and the Schreier--Sims algorithm.
There were also various coding sprints for the GAP system and various GAP
packages.

\textbf{Results and impact.} M.~Torpey learned more about the core GAP system
and permutation group algorithms.  He also made progress with work on
PackageManager, which he was able to advertise to others at the meeting.

\end{event}
