\begin{event}{Sage Gap Days 101}{SageGapDays101}{Cernay-la-ville (FR), 2019-06-17 to 2019-06-21}{PS}{20}{8}{https://wiki.sagemath.org/days101}

\textbf{Main goals.} This developer meeting was the third in a row
with a focus on \ODK tasks related to packaging, portability and
documentation tools for GAP and SageMath.


\textbf{Event summary.} An intensive week of brainstorms and coding
sprints, structured by regular planning and debriefing sessions, and
interspersed with a few presentations:
\begin{itemize}
\item Sage Combinat Widgets and Francy, Sage Explorer by O. Benassy
\item Semantic aware Sage interface to GAP by Nicolas M. Thiéry
\item Jupyter Viz by professor Nathan Carter from St Andrews
  university,
\item Cppyy by Julian Rüth, PhD.
\item RISE by Tomer Bauer, PhD student from Bar-Ilan University, in
  Israel
\end{itemize}

\textbf{\ODK implication.} This event was organized and funded by \ODK
(Paris Sud). Accommodation and meals for all participants were covered
by ODK funding sources. OpenDreamKit participants covered their own
travel expenses.

\textbf{Demographic.} Eight \ODK participants (N. Thiéry, V.
Delecroix, F. Rabe, V. Klein, S. Lelièvre, L. de Feo, O. Benassy, E.M.
Bray) from four sites, together with twelve other participants from
nine different institutions

% \begin{figure}[ht]
%   \includegraphics[width=.75\textwidth]{2019-06-15-SageGapDays101.jpg}
%   \caption*{Sage GAP Days 101}
% \end{figure}

\textbf{Results and impact.}

The workshop was the occasion of a major advance for packaging
SageMath and dependencies with Conda. A first version of SageMath had
been packaged two years ago in good parts by independent conda expert
Isuru Fernando. However the packages had not been updated since due to
a series of hurdles. Thanks to the coming of Isuru and brainstorms
with local experts of SageMath and GAP, many of the hurdles have been
lifted, leading to the packaging of the latest version of SageMath and
paving the way for more regular updates in the future.

In addition, a lot of work was targeted at continuous integration,
Python 3 support, docker packaging and interfaces between systems.

Finally the workshop was the occasion for a joint coding sprint
between ODK participants and Nathan Carter, around visualization
(Francy, Jupyter Viz) and live browsing of mathematical data (Sage
Explorer, Group Explorer). This led to the sharing of much vision and
know how, and opened the door for potential future convergences
between the related projects.

\end{event}
