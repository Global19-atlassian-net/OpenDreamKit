\begin{event}{Free and Practical Software for Algebraic Combinatorics 2019}{Sage days 105}{University of Ljubljana, Slovenia,
8th-12th of July 2019}{UPsud}{58}{2}{https://wiki.sagemath.org/fpsac19}

\textbf{Main goals.} The aim of this workshop  was  to gather people  interested in SageMath, from newcomers to contributors.

\textbf{ODK implication.} Nicolas Thiery  was one of the organizers.
There was no registration fee. OpenDreamKit funded meals and nights taken by the participants 
at the local Youth Hostel (Dijaski dom Vic, Gerbičeva ulica 51a, Ljubljana). 

\textbf{Event summary.} The main purpose of the plenary talks was to pave the path for the implementation
of new features in Sage and related software, to ignite, inspire and fuel brainstorms and 
coding sprints with other participants. Short demos were made to give an overview of the
ecosystem to all participants. A combination of free tutorials were also presented to point participants to 
resources to explore according to their pace and taste, with support from instructors, and  guided tutorials/longer presentations
in separate rooms. Participants were encouraged to skip the parts that are 
irrelevant to them (e.g. tutorials on material they already master) to engage into parallel collaborative 
activities such as coding sprints. The workshop featured a total of 16 tutorial sessions, 8 demos and 6 presentations. 

\textbf{Demographics.} Out of 56 participants, 19 were PDH students and 37 Researches/ Professors, 15 from european countries 
( Germany, France, Slovenia, Portugal, Netherlands, Austria, Iceland), 21 from North America (US, the Carribbean, Canada) 
1 from Australia and 19 from developing countries (Isreal, Korea, Turkey, India, China, Nigeria, Chile).
 We had 15 female participants to this workshop. Although it appears unsatisfactory with respect to the last four years 
 this gender gap in the software development area tends to shrink. 

\textbf{Results and impact.}

\textbf{Results and impact.} Nicolas Thiéry gave a presentation about "Best practice 
for computer exploration", a tutorial "Live online notebooks with Binder" and made two demos 
"Object oriented programming in Sage" and "writing Python packages".

Odk member, Odile benassy also made a presentation "free/libre software is good and what you can do about it" and a 
demo "Sage-Combinat-Widgets, Francy"  (\url{https://hackmd.io/8E5Tky5cSsq1tD73QTVfcQ#Some-best-practices-for-computer-exploration}),
