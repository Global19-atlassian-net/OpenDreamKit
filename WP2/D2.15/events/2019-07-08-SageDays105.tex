\begin{event}{Sage Days 105: Free and Practical Software for Algebraic Combinatorics}{Sage days 105}{University of Ljubljana, Slovenia,
8th-12th of July 2019}{UPsud}{58}{2}{https://wiki.sagemath.org/fpsac19}

\textbf{Main goals.} This workshop was organized as satellite event to
the main yearly international conference on algebraic combinatorics
FPSAC (Formal Power Series and Algebraic Combinatorics). It aimed at
gathering researchers from this community interested in computation
for training, community building, and coding sprints.

\textbf{ODK implication.} Nicolas M. Thiéry  was one of the organizers.
There was no registration fee. OpenDreamKit funded meals and nights taken by the participants 
at the local Youth Hostel (Dijaski dom Vic, Gerbičeva ulica 51a, Ljubljana). 

\textbf{Event summary.} The workshop featured a total of 16 tutorial
sessions, 8 demos and 6 presentations, including the following
presentations by ODK participants: ``Best practice for computer
exploration'', ``Live online notebooks with Binder'', ``Object
oriented programming in Sage'', ``Authoring SageMath packages'',
``free/libre software is good and what you can do about it'',
``Sage-Combinat-Widgets, Francy''.

The first plenary talk was about inspiring the next generation,
notably women, about the impact using and developing computational
software can have on conducting one's research in algebraic
combinatorics. The main purpose of the remaining plenary talks was to
pave the path for the implementation of new features in SageMath and
related software, to ignite, inspire and fuel brainstorms and coding
sprints with other participants. Training took the form of short demos
to give an overview of the OpenDreamKit toolkit, tutorials that were
presented to point participants to resources to explore according to
their pace and taste, with continuous support from instructors, and
guided tutorials/longer presentations in separate rooms. Participants
were encouraged to skip the parts that were irrelevant to them (e.g.
tutorials on material they already master) to engage into parallel
collaborative activities such as coding sprints.

\textbf{Demographics.} Out of 56 participants, 19 were PhD students
and 37 researchers or professors, 15 from european countries 
(Germany, France, Slovenia, Portugal, Netherlands, Austria, Iceland), 21 from North America (US, the Carribbean, Canada) 
1 from Australia and 19 from all over the world (Israel, Korea, Turkey, India, China, Nigeria, Chile).
We had 15 female participants to this workshop. Although still far
from satisfactory with respect to the last four years
this gender gap in the software development area tends to shrink, with
several bright young women taking on lead roles.

\textbf{Results and impact.} More than one fifth of the FPSAC
participants decided to participate to this event; they actively
engaged into the collaborative atmosphere; a lot of expertise was
shared, not counting new software contributions. One of the
participant wrote back ``Thanks again for a very useful Sage Days. I
was happy that the Dynamics package got finished ... Its fantastic how
you’ve shepherded this community over many years now, and the
mathematical community is much better for it.''. Another decided to
organize a similar satellite event to FPSAC'20.

\end{event}
