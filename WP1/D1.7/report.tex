\documentclass{deliverablereport}
\usepackage[show]{ed}
\usepackage{paralist}
\def\dmh{\texttt{data.mathub.info}\xspace}

\deliverable{management}{imp2}
\duedate{31/08/2019 (M45)}
\deliverydate{31/08/2019}

\author{Nicolas M. Thiéry et al.}

\begin{document}
\enlargethispage{4ex}
\maketitle
\githubissuedescription
\tableofcontents\newpage


\section{Context}

OpenDreamKit (Open Digital Research Environment Toolkit, ODK for
short) is a four-year Horizon 2020 European Research Infrastructure
project (\#676541). It provides substantial funding to the open source
computational mathematics ecosystem, and in particular popular tools
such as \Linbox, \MPIR, \Sage, \GAP, \PariGP, \LMFDB, \Singular,
\MathHub, and the \Jupyter interactive computing environment.

From this ecosystem, ODK will eventually deliver a flexible toolkit
enabling research groups to set up Virtual Research Environments
(VRE), customised to meet the varied needs of research projects in
pure mathematics and applications, and supporting the full research
life-cycle from exploration, through proof and publication, to
archival and sharing of data and code.

The primary end-users are researchers in (pure) mathematics; however,
thanks to the modular design the work will benefit a wide range of
end-users, including teachers, engineers, in academia, public
institutions, or enterprises.

An unusual feature of ODK is that it emerged from an ecosystem of
community-developed software where the prevalent development model is
``by users for users''. And indeed, many of the ODK participants are
themselves end-users: academics wearing several hats -- developer,
researcher, teacher, in mathematics or computer science, etc. -- who
gathered to develop the computational tools they needed for their daily
research. ODK itself was born from two dual movements,
end-users pushing innovation and developers pulling innovation,

\TODO{HF: pull and push mixed up?}
often
led the same people wearing both hats. This, combined with long
dissemination experience from the said people, is a strong asset for
delivering that actually benefits a wide range of end-users.

Reciprocally, ODK works hand in hand with the communities; some
innovations brought to the toolkit are partially or entirely
accomplished outside of the ODK project. This is not a
problem in itself as the communities and the project share the same
values and objectives.

End-users will benefit from innovations brought by ODK in the
multiple pieces of software forming the opensource VRE. In the case of
ODK the innovation will be the unification of opensource
tools with overlapping functionality, the simplification of the tools
for end-users without coding expertise, and the development of
user-friendly interfaces.

The following document will first explain the innovations that
ODK will bring to end-users, then explain the processes
enabling the implementation of the said innovations, and finally will
explain how ODK will target its end-users.


\section{Innovations ODK is bringing to the research community}

ODK has several cross-cutting objectives, each of which
brings innovation to the research community:
\begin{itemize}
\item{Develop and standardise math software and data for VRE: WP3, WP4, WP5, WP6}
\item{Develop core VRE components: WP3, WP4, WP5, WP6}
\item{Bring together communities: WP2, WP3}
\item{Update a range of software: WP3, WP5}
\item{Foster a sustainable ecosystem: WP3, WP4, WP5, WP6}
\item{Explore social aspects: WP7}
\item{Identify and extend ontologies: WP6}
\item{Effectiveness of the VRE: WP2, WP7}
\item{Effective dissemination: WP2, WP7}
\end{itemize}

We explore two typical areas of innovations in more detail:
\begin{description}
\item[Best practice and tools for correct and reproducible research]
  In the excellent talk
  \href{https://mikecroucher.github.io/MLPM_talk/}{``Is your research
    software correct''}, Mike Croucher highlights crucial best
  practices for the use of software in research, including open code
  and data sharing, automation, use of high level languages, software
  training, version control, pair programming, literate computing, and thorough
  testing. A lot of work in ODK relates to disseminating this set of
  best practices (\longWPref{dissem}), and enabling it through
  appropriate technology (\longWPref{UI}).  Just to cite a few
  examples, \longdelivref{UI}{jupyter-collab}, and
  \longdelivref{UI}{jupyter-test} enable respectively version control
  and testing in the \Jupyter literate computing technology, while
  Mike's talk is and will be delivered in several of ODK's many
  training events (see \longdelivref{dissem}{workshops-1} for the list
  of training events in year item). \TODO{add subsequent years, or
    remove this}
  \item[Multisystem architecture] Modern research increasingly
    requires  the combination of multiple
    computational, database, and user
  interface components. We explore novel ways to combine software
  (\longWPref{component-architecture}, \longWPref{UI}), while taking
  advantage of parallel features (\longWPref{hpc}), sharing data and
  semantic in a sound way (\longWPref{dksbases}), and fostering
  collaboration between systems (\longWPref{social-aspects}).
\end{description}

More details about the innovations ODK is developing and implementing
will be available in the second version of this document at Month 45,
when the VRE is much more mature.


\section{Implementation processes of the innovations}

The success of the software in the ecosystem ODK builds upon owes much
to the organization of the work process of the innovation. Their
communities have, over the years if not decades, developed and
accumulated a strong expertise in the social engineering aspects of
community software development, pulling general ideas from the open
source movement (e.g. public development, early releases, ...), and
adapting them to their specific contexts. They are heavy users of the
usual collaborative software development tools and best practice such
as mailing lists, wikis, collaborative editing pads, online chat
rooms, version control, issue tracker, continuous integration,
regression testing, code reviews, coding sprints, etc.

We describe below some striking aspects of the implementation process
in some of the software systems involved in ODK.

\subsection{Implementation process of \Sage}

All the development happens in the open (public mailing lists, bug
tracker, ...). There is no specific sustainable leader in the
development of \Sage. Its "community" of developers bases its work on
the consensus of the group and on the availability of its members to
tackle issues and work on the software development. If a decision
doesn't create a consensus, there can be a vote but that seldom
happens. When a development is brought to the software, it is reviewed
by at least one other member of the community, more often than not
with others looking over the shoulder.

Experience has shown that the success of \Sage, and in particular the
biggest achievements and best innovations, owe much to a long track of
focused week-long workshops, called Sage Days (about ten per year
since 2005), where developers get together and form small groups for
focused coding sprints. This method is actually close to the two-pizza
team rule from Jeff Bezoes, the founder and CEO of Amazon. According
to Mr Bezoes, beyond 6-7 people, the more you add people to a group,
the less the group is agile and innovative because too much effort is
put into communication and management.

An example of such workshop is
\href{https://wiki.sagemath.org/days77/}{Sage Days 77}, organised by
Nicolas Thiéry in April 4-8 2016, in a guest house far from urban
civilization and with a solid internet connection. About 15 people
joined this workshop throughout the week, and split into three or four
constantly self-reorganizing teams. Concerning the impact, proper
packaging and distribution has been a recurrent issue for \Sage and is
a major task for \ODK
(\longtaskref{component-architecture}{mod-packaging}). Major
brainstorms occurred during the week to clarify the needs, isolate the
core difficulties, and explore potential approaches to tackle
them. The outcome was posted on the
\href{https://wiki.sagemath.org/days77/packaging}{Sage Wiki}, to be
shared and further edited by the community. This fostered tighter
collaboration between the packaging efforts for various Linux
distribution, and triggered major progress on the Debian packaging
side. Several small workshops such as this one are to be organised all
project long to speed up the software development process.

\subsection{Implementation process of \Jupyter}

The \Jupyter project is driven by the \href{https://jupyter.org/about.html}{Jupyter steering council}, as described in the \href{https://github.com/jupyter/governance}{Jupyter governance documents}. This council is composed of 15 members, one being Benjamin Ragan-Kelley who is leader of the Work Package 4 (User Interfaces),  at least two being regular participants to ODK events.

The larger scale mission of the project is decided by the steering council. In most cases, the wider \Jupyter community decides what should be done by contributing proposals on an individual basis. Most proposals come in the form of pull requests on GitHub, but larger proposals can be discussed as \href{https://github.com/jupyter/enhancement-proposals}{enhancement proposals} beforehand, and must be approved by the steering council. In general, decisions for additions are handled by the maintainers of existing packages, who are longstanding members of the community or delegates thereof (such as those hired under the ODK grant). If there is conflict, decisions can be resolved by the steering council.

Concerning the developments reviews, the \Jupyter community does it publicly. The maintainers of each project take on the bulk of the review responsibility. If there are conflicts, they can be brought to the steering council for resolution.


\subsection{Implementation process of \Singular}

Wolfram Decker and Hans Schönemann from the University of Kaiserslautern are considered to be the leaders of the \Singular software.
There is no specific process to decide how the software should evolve. It can be individual decisions of a single person in need of a service for their research, or decision taken by vote by the core of \Singular developers during a meeting.
Nevertheless the new code is essentially reviewed by Hans Schönemann.

\subsection{Implementation process of \GAP}

Individuals who have helped or been helping in the development, the
maintenance, the advisory and in the support for users are referred as
\href{https://www.gap-system.org/Contacts/People/people.html}{the \GAP
  Group}.  Furthermore, the \GAP council was formed in 1995, which is
currently consisting of 19 senior mathematicians and computer
scientists with Prof.~Leonard Soicher as
chair. \href{https://www.gap-system.org/Contacts/People/Council/council.html}{The
  \GAP council} is not a representative body but more resembles an
editorial board, in which the Council chairman acts as editor in
chief, and the other members as "associate editors," after
the fashion of some journals.

In \GAP, most issues concerning development are decided after
discussion among the developers or the support team. Most of the
discussions happen in \href{https://github.com/gap-system}{GitHub
  issues and pull requests}, primarily in a
\href{https://github.com/gap-system/gap}{former
  repository}.\TODO{why ``former''?}. Larger
proposals can be discussed in the
\href{http://mail.gap-system.org/mailman/listinfo/gap}{Open \GAP
  development mailing list}. Sometimes discussions may fail to reach a
conclusion, either because the issues are too complex or because
people simply can’t agree. In this case a smaller core group will
discuss and make decisions which have to be in the general interests
of \GAP.

For \GAP packages, the decisions are taken by their authors and
maintainers. Many of currently 145 packages redistributed with \GAP
adopt an \href{http://gap-packages.github.io/}{open development model}
which facilitates interaction between them and participation of the
other \GAP developers in technical discussions.  For the
core \GAP system, reviewing and evaluation is mostly done via public
code review on GitHub and regression tests. For \GAP packages
redistributed with \GAP, there are regression tests and checklists to
use when a new package is submitted for the redistribution with \GAP.

Finally, \GAP features a longstanding formal package refereeing process
overseen by the \GAP Council. This process, rather unique and exemplary
among academic software, aims at promoting high quality contributions
by rewarding authors with credit similar to that of a published paper.

\section{The end-users targeted at by innovations}

\TODO{@fanghor: reread this section and see if we want to update anything}

As it was previously said, the originality of the ODK VRE is its "by users for users" model. Indeed, ODK's participants have many hats: they are mathematicians, computer scientists, developers, researchers, professors, end-users and sometimes all of this at the same time.

But since ODK is promoting open source, it is naturally aiming at reaching out as many end-users as possible. In order to do that ODK comprises two boards which can help reach out end-users outside the developer communities.

\subsection{Boards within ODK}

\subsubsection{Quality-Review board}

The Quality Review board (QRB) is composed of four members: Hans
Fanghor (chair), Alexander Konovalov, Konrad Hinsen and Mike
Croucher. They all have a long track of developing opensource research
software and disseminating to end-users. Some of their objectives are
to:

\begin{itemize}
\item{Assess the quality of the software engineering aspects}
\item{Identify best practice}
\item{Improve future work within ODK and disseminate knowledge to a wider audience, i.e. potential end-users.}
\end{itemize}

Thus, the work of the QRB can facilitate access of the end-user to the
innovations brought by ODK by making sure best practices are
respected and the software development is of high quality. However the
board specifically turned towards the end-user needs is the Advisory
board (AB).

\subsubsection{The Advisory board and its end-user group}

The AB is composed of seven members (Lorena Barba, Jacques Carette,
Istvan Csabai, Françoise Genova, Konrad Hinsen, William Stein and Paul
Zimmermann), and industry/end-users and academic representatives who
understand broad 21st century needs for computational mathematics.
The AB is to give an independent opinion on scientific and innovation
matters, in order to guarantee:

\begin{itemize}
\item Quality implementation of the project,
\item Efficient innovation management,
\item Project sustainability.
\end{itemize}

Beside, the Board includes a small End-user Group (3 to 4 persons out
of 7) which will be connected to an informal community of
end-users. They will control the project execution from the point of
view of the end-user needs and requirements, making sure that the
outcome of the project indeed matches those needs.

\subsection{End-users targeted by ODK innovations}

Matching user needs is not sufficient. One additional challenge is to
promote the VRE to potential end-users so that it actually gets put to
use, in Europe and beyond, in established developer communities and
across new users, in math and other relevant research fields. This
challenge is being tackled by the WorkPackage 2 "Community building,
training, dissemination, exploitation and outreach".


\subsubsection{A worldwide promotion}

As it was previously said, ODK is developing and promoting software that are opensource. The universal and cost-free distribution nature of opensource software allows ODK to target every country and continent notwithstanding any level of infrastructure development, economic performance or lack of a solid institutional academic network.
In the first year of the project from Sept. 2015 to Sept 2016, 14 events (workshops, schools, etc.) were (co-)organised by ODK. Some of them were of course planned in Europe and North America, but others were planned in Africa, South America and Middle East. In the next years ODK will continue at the same pace to (co-)organise events in the same geographical regions.
However in some areas where the internet connection does not allow for massive cloud usage, one must find alternative solutions such as distributing the required install files using USB sticks  rather than online repositories. This simple trick enables dozens of undergraduates, master students, PhD students, postdoctorates, teachers and professors following the same given school to start working on a SageMath or Jupyter tutorial at the same time.

Because of the opensource nature of the ODK software, everything can infinitely be replicated and shared without any constraint. Therefore in addition and following the events organised by the project, ODK is counting on a snowball effect for the reaching out of end-users.

\subsubsection{Established communities}

Since ODK participants are all part of at least one of the well established opensource software communities, the communication on ODK's achievements comes naturally and easily. Furthermore, many workshops are organised all year-long, whereas they are (co)financed by the project or not, during which the developer communities join together their efforts to improve the software from the ODK toolkit.

\subsubsection{Research fellows outside of established communities}

Furthermore ODK was introduced at external events in order to reach out potential end-users outside of the regular opensource software developper communities.
All these activities can be tracked in \longdelivref{dissem}{workshops-1}.


All the pieces of software improved and developed by the project such as \Sage, \Singular, etc. were presented during these schools and workshops targeted at students and potential new users. Of course this software being originally developed for end-users who are potentially very skilled in computing and software development, one of the goals is to make it easy  for relatively unskilled end-users to start using our opensource toolkit.
For example, one of the main objectives in the development of \Sage in the attraction of new end-users is to develop the portability of \Sage (and therefore its dependencies) on Windows. With such a tool, the toolkit will enlarge its target to all Windows users in need of software like \Sage but without the time or knowledge to set it up themselves on their computers.

Three major conferences of about 60 to 90 participants are planned within the frame of ODK. One out of three was already jointly organised with the Collaborative Computational Project (CoDiMa) at the ICMS (International Center for Mathematical Sciences) of Edinburgh on the 10-16 January 2017 with about 50 to 60 participants. The topic was: Computational Mathematics with Jupyter. This conference aimed at enabling users and developers of \GAP, \Singular, \Sage and \Jupyter to meet. Talks were given and coding hackathons were organised. The event is currently being evaluated by its organisers.
The two other conferences will be organised each in 2018 and 2019.

%TODO: Tutorials/Moocs?

\section{On the choice and impact of open licenses}


\TODO{@stevelinton: brief story about the SmallGroups library license}
\TODO{@stevelinton: brief story about the GAP license discussion?}

The management of intellectual property (IP) developed over several
decades
by a collaborative, distributed community by researchers with,
especially initially, little or no input from IP professionals
presents some unique challenges. In the early years of the \GAP
system, for instance, no reliable record was kept of the authorship of
all the components of the system or of the employment status of the
authors, and no transfers of IP rights were obtained. This means that
many individuals and institutions undoubtedly own the IP of small
fragments of the system, in many cases unwittingly.
This does not create a problem for continuing
distribution under the current license (since any contributor
implictly grants us permission to redistribute their
extensions), but makes it very difficult to make any changes to the
license. On two occasions when it has been necessary to move to
similar (but more modern or widely understood licenses) we have simply
made a good faith effort to contact all stakeholders that we are aware
of, and proceeded on the basis that no objections were received.

While unsatisfactiory in some respects, this arrangement has actually
worked well to date. Extensions to \GAP have typically been
implementations of new algorithms and/or extensions of the system to
new mathematical areas, which were developed and expected to be used,
in much the same way as the rest of the system, and the existing
license terms have been appropriate. In the context of the wider \ODK
ecosystem, including for instance datasets, on-line services and smart
documents, this situation is less clear, and we are working to build
consensus in the \GAP community around some new policies.

One example is the \href{???}{The Library of Small Groups}. The main
component of this is a dataset which lists and assigns unique IDs to
all 423~million isomorphism types of group of
order up to 2000 and not 1024. The
authors of this work were reluctant to release it under the GPL
because of the risk to their professional reputations if someone
released an ``enhanced'' version which contained mathematical errors,
or altered the IDs so as to create confusion. After extensive
discussion with the authors, an alternative free license -- the
\href{???}{``artistic''} license was found which provided greater
protections in their areas of concern, while being compatible with
incorporation into \ODK tools.

Another example is the question of what constitutes a ``derived
work'', a fundamental question since free software licensing is based
wholly on copyright rather than patents as the basis for its IP. After
discussion, the \GAP community has taken the view that programs and
packages written in the \GAP language and smart documents based on
the \GAP Jupyter kernel or similar technology (although requiring a copy of
\GAP to be used) are not derived works, and we make no attempt to control
their redistribution or use  provided the included copy of \GAP is
properly sourced and acknowledged.

\TODO{@defeo / @embray: brief story about the impact of the openssl
  license mess and how it affect(ed) sage and our efforts at promoting
  packaging for modularization }

\TODO{on the graphviz license and impact on its integration in SageMath?}

\TODO{Some discussion about GPL versus BSD license: math software tend
  to be GPL whereas more general purpose tools such as Jupyter tend to
  be BSD; why, including the relative importance of facilitating their
  use and integration by industrial stakeholders?}

\TODO{@nchauvat: would you have some reflections to share about which
  licenses are more handy for companies?}

\section{Sustainability}

In this section, we review the different kind of outcomes of
OpenDreamKit, and assess their respective sustainability.

\subsection{Contributions to existing software}

\TODO{setup the stage by giving some indications on the volume of
  ODK's contributions: which software, number of lines contributed by
  Erik+Jeroen+Dima as measured by GitHub, ...}

\TODO{}

\subsection{Implementation of new packages}

\ODK has created several new software packages,
including pypersist, nbval, nbdime \TODO{, and others}.
There is a sustainability risk in creating new software during a finite project period,
as the end of the project can end funded support of the software.
In all cases, these projects do not rely on infrastructure beyond free,
public code hosting services such as GitHub to continue to exist in their current form.
The sustainability is only vulnerable in that it requires human effort to continue further development and support.
In most cases, \ODK members will continue as maintainers of the software through their work after \ODK,
but not always.
The typical sustainability model for open source and community software
is that software with sufficient interest and activity will develop
a community of maintainers beyond the original authors,
able to support the software as maintainers.
\ODK contributors may also continue to work on the software
as community volunteers, or funded via other means.
Some metrics for how well this is achieved is measuring contributions
by individuals outside \ODK on a given project.

For WP4, the nbdime package was created as an official part of the Jupyter project,
owned by the jupyter organization on GitHub.
This gives all members of the Jupyter team access to the repository to continue its maintenance.
Jupyter is a large collaboration of many organizations, and not dependent on \ODK.

As of August, 2019, nbdime is installed from the Python Package Index (PyPI) on average 3,000 times per week.
In the last twelve months, there have been 61 contributors to code and discussion, compared with 45 in the twelve months prior,
indicating a growth in the community surrounding nbdime.
In terms of code committed, contributions are still dominated by \ODK members,
who will continue as maintainers of the project.

nbval is in a similar situation,
owned by the Computational Modelling Group at University of Southampton / European XFEL,
led by \ODK member Hans Fangohr.
nbval is installed on average 3,500 times per week from PyPI.
nbval has 25 contributors to code and discussion in the last twelve months,
compared with 16 in the twelve months prior,
and continues to be maintained by the Computational Modelling Group after the end of \ODK.


\TODO{2/3 paragraphs giving an indication on the volume of ODK's
  contributions, e.g. some typical new packages like pypersist, nbval,
  ..., with assessment of their sustainability: will they remain
  available? with they be maintained? will the authors stick around?
  will they be adopted by the community?}

\subsection{Prototypes}

\TODO{@kohlhase: some paragraphs giving an indication on the volume of
  prototypes that were produced by OpenDreamKit (e.g. the various bits
  of the MitM infrastructure); some assesement of the importance (or
  not!) of having them functional / maintained in the long run, for
  reproducibility on the one hand and adoption by community on the
  other hand.}

\subsection{Infrastructure setup}

\TODO{@embray / @alex-konovolov: 2/3 paragraphs giving an indication
  on the volume of ODK's contribution toward setting up infrastructure
  for continuous integration / artifact building / continuous
  deployement, with some assessment of the manpower that will be
  required to maintain it in the long run, and words on what we hope
  will happen?}

\begin{newpart}{MK: please re-read}
\subsection{Dataset Production, Hosting, and Dissemination}

One of the main topics of \WPref{dksbases} has been studying the semantics of mathematical
datasets as VRE components, using the OEIS, FindStat, and LMFDB as case studies. In the
course of this endeavor \pn
\begin{compactenum}
\item has signiticantly improved the structure, maintainability, and interoperability of
  the LMFDB and
\item developed a new MitM-interoperable model for deep FAIR mathematical datasets, which
  was implemented in the \dmh system. 
\end{compactenum}
LMFDB is a large-scale management system for matheamtical datasets in number theory, which
is carried by a considerable and thriving research community independent of the \pn
project, which will ensure sustainability of the effort.
\pn has contributed the design and implementation of an online inventory for the LMFDB datasets, 
this inventory being itself stored as part of the LMFDB for ease of maintenance and for access by a variety of user interfaces.
Building on this, ODK researchers have designed a new API for the LMFDB, with a Sage front-end still under development.




\dmh is part of the MathHub.info system, which is a central research and development
product of the KWARC group at FAU Erlangen-N\"urnberg and development will be continued
under this rubric.
The \pn project has motivated the system archtecture, funded the development of the
initial prototype, and triggered the inclusion of the first five data sets.
The Math Data Workshop in August 2019 in Cernay has brought together a first community of
enthusisasts which -- so we hope will carry the further development and use of the system
beyond the FAU group.
Indeed, the fledgling has its own slack channel and three new members and five additional
data sets are already in the pipeline. 
\end{newpart}

\subsection{ODK's web site}

\ODK's website at opendreamkit.org is hosted for free on the GitHub-pages service.
There is no ongoing cost to host the website beyond the low cost of registration of the the opendreamkit.org domain.
The \ODK website will remain as an archive of project activities and does not need further updating.
The site is sustainable at no cost or effort as long as the GitHub service continues to exist,
which is likely as it is a large platform with a long track record, currently owned by a large
If GitHub terminates its service at some point,
the repository can be migrated to other similar services with minimal effort,
so it is sustainable even beyond the expected long life of GitHub.

\TODO{how is opendreamkit.org registered? Will it be renewed?}

\subsection{ODK's infrastructure}

\TODO{A few words on the ODK infrastructure (GitHub repo, ...), its
  long term availability and potential use for inspiration for other
  projects, social analysis, ...}

\subsection{Training and promotional material}

\TODO{@nthiery, @alex-konovolov: a few paragraphs about our comics,
  videos, training material, ...}

\end{document}

%%% Local Variables:
%%% mode: latex
%%% mode: visual-line
%%% fill-column: 5000
%%% TeX-master: t
%%% End:

%  LocalWords:  maketitle githubissuedescription newpage newcommand xspace Jupyter dissem
%  LocalWords:  tableofcontents visualizations composability itemize analyzed taskref hpc
%  LocalWords:  dissemination-of-oommf-nb-virtual-environment taskref dissem taskref pn
%  LocalWords:  dissemination-of-oommf-nb-workshops dissem ibook taskref taskref taskref
%  LocalWords:  oommf-python-interface oommf-py-ipython-attributes taskref oommf-nb-ve
%  LocalWords:  oommf-tutorial-and-documentation taskref oommf-nb-evaluation taskrefs
%  LocalWords:  delivref pythran-typing sage-paral-tree subsubsection organized Dagstuhl
%  LocalWords:  co-organized organization modularization ipython-kernels nbdime Pythran
%  LocalWords:  jupyter-collab ystok WPref dksbases compactitem emph WPtref DehKohKon
%  LocalWords:  iop16 textbf tasktref lfmverif triformal formalized biformal ossp09 Dima
%  LocalWords:  hline Marijan Pilorget Pierrick Kruppa Dehaye Dehaye's Dehaye's Alnaes
%  LocalWords:  Konovalov Hinsen github printbibliography
