\subsubsection{WorkPackage 2:  Community Building, Training, Dissemination, Exploitation, and Outreach}
\label{dissem}
%Explain, task per task, the work carried out in WP during the reporting period giving details of the work carried out by each beneficiary involved.

\TODO{Update for Reporting Period 2}

%%%%%%%%%%%%%%%%%%%%%%%%%%%%%%%%%%%%%%%%%%%%%%%%%%%%%%%%%%%%%%%%%%%%%%%%%%%%%%
\paragraph{Overview}

  As planned in \longtaskref{dissem}{dissemination-communication} and
  \longtaskref{dissem}{dissemination}, 22 meetings, developer and training workshops have already
  been organized and co-organized by \ODK (including 14 have for year 1 described in \longdelivref{dissem}{workshops-1}) , and complemented by many
  presentations and activities in external events. This includes the first Women in Sage workshop in Europe and the
  first major dissemination event of the project (Computational Mathematics with \Jupyter in ICMS). This testifies
  of the vibrant activity and energy of the OpenDreamKit participants.

  More specifically, we have targeted two specific communities:

\begin{compactitem}
\item The Mathematics community through specific Sage Days.
\item The micro-magnetic community through specific workshops presenting
the \Jupyter and Python interface to the widely used OOMMF \longtaskref{dissem}{dissemination-of-oommf-nb-virtual-environment} (5 workshops so far).
\end{compactitem}


  \ODK\ is also working on its visibility and communication strategy with an active website keeping track of all
  project activities including conferences, workshops, talks, blogposts, press releases, reports, etc.


%%%%%%%%%%%%%%%%%%%%%%%%%%%%%%%%%%%%%%%%%%%%%%%%%%%%%%%%%%%%%%%%%%%%%%%%%%%%%%
\paragraph{Tasks}

\subparagraph{\longtaskref{dissem}{dissemination-communication}}

The \longdelivref{dissem}{press-release-1} was delivered and a page for the \href{https://github.com/OpenDreamKit/OpenDreamKit/blob/master/Communication/eInfra-Booklet/ODK.md}{E-infrastructure booklet} was written jointly by \ODK members. The website has been
entirely rebuilt to better promote  and centralize project activities. At this date,
we have 58 posts on our website which correspond to blogposts (14), conferences and workshops (11),
talks (14), and other project communication. The website also keeps track of press releases
and deliverable reports. We have recently added an open-source web analytics system (Piwik)
to track some basic informations about our visitors.

\subparagraph{\longtaskref{dissem}{training-portal}}

Training is a core and transversal aspect of our project. It is carried out
through interventions and events as we discuss in \longtaskref{dissem}{dissemination} but also
by writing documentation, tutorials (\delivref{dissem}{short-course}), blogposts, etc. All
this communication is centralized on our website. We have created a specific
page listing the different software programmes which could eventually host training material. Furthermore,
each programme has its own tag which links to all related project activities.

\subparagraph{\longtaskref{dissem}{devel-workshops}}
\label{dissem@devel-workshops}

Development workshops are a key aspect of OpenDreamKit development model. The aim of these workshops is to bring together developers from the different communities to design and implement some
of the wanted features. As reported in \longdelivref{dissem}{workshops-1}, we have organized
or co-organized 5 of these workshops during year 1 of the project. Since then, 2 more have happened. The thematics varies
for each event: packaging and portability, Sage and Jupyter, Sage-GAP days, PARI/GP, knowledge representation, etc. To this,
we can add 4 project meetings which we always turn into an occasion for more coding sprints and development
discussions.

\subparagraph{\longtaskref{dissem}{tech-review}}

This task has been started during our initial KickOff meeting where we organized a session of short
talk presenting many different technologies (\Jupyter, \cocalc, Docker, \Sphinx, \Cython, \Pythran, and
many more). Some of them were further developed into \delivref{dissem}{techno}. The goal is to
keep up to date with most recent breakthroughs related to our work in OpenDreamKit. When we felt
it was relevant, we turned part of the document into blogposts on our website gathered under a
specific tag. At this date, we have published 4 articles and plan on publishing more throughout the
project.

\subparagraph{\longtaskref{dissem}{dissemination}}

Dissemination is a key aspect of the success of OpenDreamKit. Indeed, our development is carried
out to help and support mathematical communities. One of the goals is to bring
more users and more developers to the different projects we are involved in. The events
that took place during Year 1 have been reported in \longdelivref{dissem}{workshops-1}, since then, still
more happened.

\begin{compactitem}
\item \textbf{Organization of Sage Days in established mathematical communities.} Sage Days have long been
part of the SageMath tradition. By organizing and supporting Sage Days, OpenDreamKit can stay close
the mathematical community, understand its needs, gather more users and developers, and improve
the overall quality of the software. We have been involved in 5 different such events since the beginning
of the project.
\item \textbf{Training activities in developing countries.} OpenDreamKit has a long term plan of fostering
a SageMath community in the Mediterranean area where 3 different events where organized (Algeria, Lebanon, and Tunisia)
and some more are planned in Morocco. We were also present at ECCO 2016 in Columbia and have been invited
to come back for the next conference in 2018.
\item \textbf{Women in Sage.} OpenDreamKit is concerned with the gender gap in mathematic software development.
We have organized the first Women in Sage conference in Europe, inviting 20 women to participate to a week of Sage development. We plan at least one other such event during the project. You can read our \href{http://opendreamkit.org/2017/04/06/WomenInSage/}{report} on our website.
\item \textbf{Computational Mathematics with Jupyter.} This was the first event of the 3 main dissemination conferences
planned throughout the project. It gathered a large mathematics community in ICMS, Edinburgh, in collaboration with the CoDiMa project.
\item \textbf{Jupyter Day in Orsay.} This one day training event in Orsay was fully booked shortly after it was announced, testifying of the growing interest of the community for the Jupyter project. It featured demos and talks from experts of the field as well as an afternoon of tutorials.
\item \textbf{Other training and communication activities.} The participants of OpenDreamKit are very active
in spreading their knowledge and the project's news to the different open source
and mathematical communities they belong to. We count 11 talks on the
website. We have been present to many major events: PyCon, EuroScyPy, CICM, ISSAC, CoDiMa school and more.
\end{compactitem}

\subparagraph{\longtaskref{dissem}{project-intro}}

At the occasion of the KickOff Meeting, Viviane Pons presented \cocalc
(then called \SMC): an online solution for collaborative
work on OpenDreamKit software such as \Sage and \Jupyter. In particular, it offers a basic course management system that appeared to be a very good solution for some of the challenges described in this task. Following this talk, Michael Croucher implemented this solution for many courses at Sheffield university. This system is also used at Paris-Sud. As a result, the first goal of the task was achieved when we delivered \delivref{dissem}{short-course} on how to use \cocalc for teaching with OpenDreamKit technologies.

We are also looking at nbgrader: at Southampton, the Jupyter Notebook and nbgrader were used to support teaching of large (ca. 500 students) and more specialized engineering design teaching modules for students in the first year of their engineering degree programs.

\subparagraph{\longtaskref{dissem}{dissemination-of-oommf-nb-virtual-environment}}
\label{dissem@dissemination-of-oommf-nb-virtual-environment}

We created a GitHub organisation named JOOMMF, where the micro-magnetic VRE code is publicly hosted (\href{https://github.com/joommf}{JOOMMF repo}). For each JOOMMF package we use \href{https://travis-ci.org/joommf/discretisedfield}{continuous integration on Travis CI} where we perform tests and monitor the test coverage, which we then make available on \href{https://codecov.io/}{Codecov}. Documentation for each package consists of APIs (automatically generated from the code) and different tutorials created in Jupyter notebooks. Both of them are tested on Travis CI. Documentation is built and made publicly available on \href{http://discretisedfield.readthedocs.io}{Read the Docs}. After every major milestone, we upload each package to the Python Package Index repository. We encourage the early use of our software and invite for feedback for which we provide several different communication channels: Google group (joommf-news), \href{https://gitter.im/joommf/}{Gitter channel}, \href{https://github.com/joommf/help}{GitHub help repository}, \href{https://twitter.com/joommf}{Twitter account}, and a \href{http://joommf.github.io/}{website}.

\subparagraph{\longtaskref{dissem}{dissemination-of-oommf-nb-workshops}}
\label{dissem@dissemination-of-oommf-nb-workshops}

We had several workshops and tutorials so far where we demonstrated the use of our Micromagnetic VRE, received feedback and feature requests from the community:

\begin{compactitem}
\item Two workshops at the 61st Annual Conference on Magnetism and Magnetic Materials in New Orleans,
    LA, USA (2nd and 3rd November 2016).
\item Tutorial at the Deutsche Physikalische Gesellschaft Fruehjahrstagung (Spring Meeting) of the Condensed Matter Section in Dresden, Germany on 19th March 2017.
\item Workshop at the Institute of Physics Magnetism 2017 conference in York, UK on 5th April 2017.
\item Workshop at the IEEE International Magnetics Conference - Intermag 2017, Dublin, Ireland. 24th April 2017.
\end{compactitem}

\subparagraph{\longtaskref{dissem}{ibook}}


The first book we committed to deliver is \emph{Linear Algebra (lectures for physicists)} \delivref{dissem}{ibook2}. It is under active development \href{https://github.com/Hadriamit/iODKbook2}{as you can see on its git repo}. Our main issue is now to complete the English version (it already exists in polish). We chose to use Sphinx and the Sagecell plug-in. The remaining books are at earlier stage. They will include some ideas and teaching materials which are developed at the moment.

Moreover, at Southampton, we have started to create interactive notebooks that support teaching of mathematics and computing to engineers. In particular, we have studied the feasibility of converting existing teaching materials (in LaTeX) containing code snippets into interactive notebooks. This is work in progress. We worked together with the Simula team in directing the design of the (NBVAL) software developed in \taskref{UI}{notebook-verification} of the OpenDreamKit project (\delivref{UI}{adcomp}), which addresses one of the questions outlined in this task: "How can we facilitate automatic testing of all code examples, plots, etc ?".

\subparagraph{\longtaskref{dissem}{index-librorum-salvificorum}}

We have started to list relevant resources related to the OpenDreamKit project and started discussions
on the best course of action to follow. We are looking closely at the \href{https://oer.geant.org/}{GEANT}
project to compare it with the \href{http://sageindex.lipn.univ-paris13.fr/}{SageIndex} and understand their
common goals and main differences.

%%% Local Variables:
%%% mode: latex
%%% TeX-master: "report"
%%% End:

%  LocalWords:  subsubsection dissem longtaskref organized co-organized longdelivref emph
%  LocalWords:  Jupyter compactitem dissemination-of-oommf-nb-virtual-environment Piwik
%  LocalWords:  delivref centralized cocalc Cython Pythran textbf organizing EuroScyPy
%  LocalWords:  nbgrader nbgrader specialized Codecov joommf-news Micromagnetic Intermag
%  LocalWords:  dissemination-of-oommf-nb-workshops Fruehjahrstagung Sagecell taskref
%  LocalWords:  adcomp index-librorum-salvificorum
