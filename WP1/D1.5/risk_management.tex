\subsection{Risk management}
\subsubsection{Recruitment of highly qualified staff}
Recruitment of highly qualified staff was planned to be a high risk
when the Proposal was written. And unfortunately it turned out we were
right. In such a field as computer science and software development,
potential candidates who are likely to be fairly young considering
only temporary positions are offered, are very scarce. Furthermore
they need to make a choice between public and private bodies which are
very attractive, and the choice between pure development and research.
Because of this difficulty to recruit in the past year, there have
been slight changes in the workplan, which do
not put the project results at risk.

The following people were hired in the past year or are in the hiring pipeline for next year\\
\begin{tabular}{|l|c|r|r|r|}\hline
  NAME&GENDER&PARTNER&POSITION&HIRING DATE\\\hline
  Theresa Pollinger & F & \site{FAU} & Junior Researcher & October 2017\\
  Tom Wiesing & M & \site{FAU}  & Junior Researcher & September 2017\\
  PD. Dr. Florian Rabe & M & \site{FAU}/\site{PS} & PostDoc & December 2017\\
  Dr. Katja Ber\v{c}i\v{c} & F & \site{FAU} & PostDoc &  November 2018\\
\hline
\end{tabular}\\
~\\
Dr. Florian Rabe is a joint appointment and splits his time and research between \site{FAU} and \site{PS}, which reflects
the close cooperation and cross-fertilization of methods in \WPref{dksbases}. 

\ODK partners had to face some Human Resources issues; this mostly
concerned Reporting Period 1 where most of the hiring occurred, with
reduced effects on Reporting Period 2:
\begin{itemize}
\item{\site{PS}:}
  Thanks to an early start in the recruitment process, and despite
  some difficulties in attracting experienced candidates for a part
  time position, the project manager position (24PM) was filled by
  Benoît Pilorget shortly after the start of the project. Unfortunately at month 36 the project
finds itself without a Project manager since the departure of B. Pilorget.

  The recruitment of \site{PS}'s first Research Engineer (48PM) was
  delayed by four months because the top ranked candidate for this
  position, Erik Bray, was originating from the US and needed time to
  arrange for his moving; there were also some administrative delays
  (visa, ...).

  The second Research Engineer position (36PM) was more problematic
  for internal administrative reasons. The top ranked candidate,
  Jeroen Demeyer, had the perfect profile; however for family reasons,
  he wished to work most of the time from Ghent in Belgium. After eight
  months investigating an administrative solution to hire him at
  \site{PS}, and a temporary four month solution, it was decided with
  OpenDreamKit's Steering Committee and Project Officer to instead add
  Ghent's university as new partner, hire Jeroen Demeyer there, with an
  adequate budget transfer and amendment to the Grant Agreement.

  Those delays have induced late start on several tasks, and costed
  much management time. However the excellence of the recruitment,
  well confirmed by the results obtained so far, was worth it and soon
  compensated for the late start.

  In addition to this, a three year PhD position was open to work on
  WP6, starting from Month 12. By lack of suitable candidate, this
  position was converted into a PostDoc position; this position was
  filled in half by Florian Rabe in Spring 2018. A research software
  engineer, Odile Benassy was hired in June 2018 until the end of the
  project using the remaining PMs.\\

\item{CNRS:} Because the research engineer offer (48PM) was still not
  filled in the Summer 2016, the CNRS decided to divide the position
  in two full positions of 24 PM each. As a result, a candidate was
  selected for one of the two positions and began his work in Fall
  2016. Thanks to the PM division, there should be no delay in any
  task or
  deliverable. \\

\item{JacobsUni:} Michael Kohlase, lead PI for Jacobs University, has
  moved on 01/09/2016 to Friedrich-Alexander-Universität
  Erlangen-Nürnberg, and most of his team will follow him. The necessary changes have been
  implemented in a grant agreement amendment in 2017.\\

\item{UJF:} The original tentative candidate for UJF's Research
  Engineer position (12PM, planned to start on Month 1), Pierrick
  Brunet, finally declined the position to accept an alternative
  permanent offer. The position was filled by another candidate in
  Autumn 2016. This induced a delay of Deliverable \textbf{D5.2} from Month 12
  to Month 18, without impact on other tasks.\\

\item{UNIKL:} UNIKL had to split the 12 PM planned for a software developer into 2 shorter positions (Anders Jensen and Alexander Kruppa) in order to deliver the planned work on time. Indeed the few qualified persons for this job were not able to accept this 12 months position during the timelapse planned within the project.\\

\item{USFD:} The University of Sheffield has also been struggling in
  the hiring process of a postdoc (36PM), and a move of the PI to the
  University of Leeds.\\

\item{Southampton:} Southampton faced administrative difficulties in
  the recruitment of Marijan Beg (38PM) as a post-doc, due to
  Marijan's Croatian nationality and recent changes in the relevant
  legislation. His recruitment was delayed by four months, and
  therefore some tasks and deliverables were postponed. We will
  compensate for the delay by putting more staff effort in at later
  stages in the project. We don’t expect any delay nor implication on
  the main tasks of OpenDreamKit.

\item{UVSQ:} Nicolas Gama was on a long-term leave until September
  2017. This did not affect the project in any way.\\

\item{UZH:} The University of Zürich partner is only composed of one
  person, Paul-Olivier Dehaye, who does not enjoy a permanent position
  there. There have been worries that Mr Dehaye's contract with his
  university might end earlier than planned within OpenDreamKit. But
  thanks to the action of the OpenDreamKit steering committee, Mr
  Dehaye has been technically rehired by UZH as a scientific
  consultant until the end of the planned implication.\\

\item{Simula:} Everything is fine concerning temporary staff
  recruitment on the Simula side, however we have had to endure the
  hazards of human ressources with Hans-Peter Langtanger (the PI when
  the Grant was signed) being on a long-term sick leave, and with
  Martin Alnaes replacing him as PI having a paternity leave. However
  Benjamin Ragan-Kelley has stepped in to lead the
  Simula contribution in the meantime and all planned tasks are on time.\\
\end{itemize}


Altogether, this confirmed that the recruitment of highly
qualified staff is indeed a risky endeavour, which induced delays on
several deliverables. However the planned mitigation measures --
taking into account the pool of potential candidates in the design of
the positions, aggressive advertisement, weak coupling between tasks
-- worked adequately: with appropriate reshuffling of the work plan,
this did not impact the overall progress of the project.

\subsubsection{Different groups not forming effective team}

As expected, this risk was tamed by the existence of many preexisting
collaborations between the partners and of ``joint itches to scratch
together'' (to use a common open source software metaphor). The
organization of many joint workshops (for example the Sage-GAP
workshop, the Atelier Pari attended by SageMath developers, the WP6
workshops) helped bootstrap joint activities through brainstorms and
coding sprints. Upcoming workshops are planned on Year~2 to strengthen
collaborations with the social aspects team in Oxford and the Singular
team in Kaiserslautern.


\subsubsection{Implementing infrastructure that does not match the needs of end-users}

The consortium is keeping in their minds the end-user needs. Since
OpenDreamKit is improving already existent software which have their
own users, their needs are naturally met. However Key performance
Indicators will evaluate the effects of OpenDreamKit on these
software. KPIs, indicated in the Proposal, will be launched this
Autumn with the help of the end-user group which was merged with the
Advisory Board. Constant links between the accomplished work and the
end-user needs should be made in WP2 deliverables and also in WP7
deliverables when relevant.  Open tracking of KPIs evolution can be
found on
\href{https://github.com/OpenDreamKit/OpenDreamKit/labels/KPI}{GitHub}.

\subsubsection{Lack of predictability for tasks that are pursued jointly
  with the community}

As planned, we are regularly shifting manpower around to adapt for the
variability of the involvement of the community in the different
tasks. For example, the SageMath Jupyter kernel of
\longdelivref{UI}{ipython-kernels-basic} was mostly implemented by the
community which allowed to focus on other tasks such as the long term
task~\longdelivref{component-architecture}{portability-cygwin}.  On
the other hand many other deliverables were implemented with very
little help from the community.

\subsubsection{Reliance on external software components}

There is not much to report on this front yet: none of the external
software component we rely on have failed us. Quite on the contrary,
critical software like \Jupyter have continued to blossom. Besides the
high modularity of the design means few components are critical to the
overall success of the project.

%%% Local Variables:
%%% mode: latex
%%% TeX-master: "report"
%%% End:

%  LocalWords:  hline cross-fertilization WPref dksbases Pilorget Pierrick textbf Kruppa
%  LocalWords:  Dehaye Dehaye's Dehaye Alnaes organization Jupyter longdelivref
%  LocalWords:  ipython-kernels-basic portability-cygwin
