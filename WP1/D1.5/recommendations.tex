\subsection{Follow-up of recommendations}

We are extremely grateful for the very constructive comments and
recommendations that were provided during the review itself and in the
formal report. Most recommendations were implemented right away at the
occasion of the Work Plan Revision process that followed the second
reporting period and involved the reviewers and advisory board. In the
sequel we explain how we took the recommendations into account.

\newenvironment{recommendation}[1]
{\noindent{\textbf{#1:}} \begingroup\it}
{\endgroup}

\begin{recommendation}{Recommendation 1} The deliverable D4.7 needs to
  be presented or reasons for not presenting it needs to be detailed
  in the progress report.
\end{recommendation}

Taken from our Work Plan Revision proposal:
This was in fact a communication glitch, which we clarified since with
the project officer. The deadline of M14 for
\delivref{UI}{ipython-kernels} was a typo in the original proposal: it
should have been M24, 12 months after the related deliverable
\longdelivref{UI}{ipython-kernels-basic}. After consulting our former
project officer back in Fall 2016, this typo was fixed in the second
amendment to the grant agreement. We oversaw that at the time of the
review this agreement was not yet effective and available to the
reviewers. We will make sure to mention any such change in the report
next time.

\begin{recommendation}{Recommendation 2} The final Progress Report
  must be presented and it should report on all deviations including
  cancelled or amalgamated deliverables.
\end{recommendation}

We followed the advice from our Project Officer: \emph{``The review
report says that you should amend the current progress report on all
deviations etc but as we have already started to process the report
and cost, you don’t need to do that anymore for the past period.
However, in the future please report on all changes/deviations in the
progress report to be clear.''}

In addition we made sure that changes and deviations for Reporting
Period 2 were reported in this document.

\begin{recommendation}{Recommendation 3}
  The workplan of WP7 should be assessed critically and the draft
  proposal of the revised plan submitted to the Commission as soon as
  possible and latest by 1 July 2017.
\end{recommendation}

This was implemented in the Work Plan Revisions.

\begin{recommendation}{Recommendation 4}
  The deliverables presentation should be clearer concerning their key
  content. They should include a clear ``executive'' summary, state
  the purpose and target audience of the deliverable and include clear
  conclusions. The full title of the deliverable should be used
  instead of the often used form ``Report on Dn.m''. Consolidation of
  the deliverables should be considered to reduce their number.
\end{recommendation}

Deliverables were consolidated at the occasion of the Work Plan
Revision, without affecting the content and work time line, beside
enabling a bit of flexibility to adapt to a quickly evolving landscape
(where e.g. some actions become more pressing and others less).
Together with the refactoring of \WPref{social-aspects} this
consolidation reduced the number of remaining deliverables from 55 to
below 40.

We tried hard to follow the presentation recommendations for all newly
submitted deliverables.

\begin{recommendation}{Recommendation 5}
  Provide clearer and more relevant work-package specific milestones
  to allow for effective monitoring of the project's progress.
\end{recommendation}

This was implemented in the Work Plan Revisions.

\begin{recommendation}{Recommendation 6}
  The web presence should be made more attractive for broader
  dissemination and it is recommended to create separate externally
  facing websites. The internal project website could be separated from
  the external one, as their purpose and target audiences are not the
  same. The social media presence and blogging should be made more
  vivid and attractive.
\end{recommendation}

At the occasions of the Work Plan Revisions, we proposed a plan to
improve our web presence, which we implemented during Reporting Period
2, with continued efforts on the content since then, including
interview videos and an upcoming motion design explainer video. To
this end, we used help from master students in communications, and two
specialized companies. For details, see
\longlocaltaskref{dissem}{dissemination-communication}.

\begin{recommendation}{Recommendation 7}
  The KPIs should be refined and alternatives suggested and the
  currently very generic milestones should be revised to be more
  specific and appropriate for project monitoring purposes.
\end{recommendation}

Alternative KPI's were chosen at the occasion of the Work Plan
Revisions.

\begin{recommendation}{Recommendation 8}
  Regarding the WP2, it is recommended to deploy some additional
  resources to improve the externally facing website as well as
  improving the internal site.
\end{recommendation}

See Recommendation 6 above.

\begin{recommendation}{Recommendation 9}
  WP3 work to be integrated into the proposed revisions of the
  website. Efforts need to be re-allocated as SMC developers have
  already done the work that was described to be in D3.4.
\end{recommendation}

This was implemented in the Work Plan Revisions.

\begin{recommendation}{Recommendation 10}
  Regarding WP5, make contacts with HPC community in order to
  ascertain current state- of-the-art. The work in this WP needs to be
  nearer the leading edge.
\end{recommendation}

The following text is updated from our Work Revision Proposal.

We would like to clarify the context of high performance \emph{mathematical} computing,
which is subject of \WPref{hpc}, and its relation to high
performance \emph{scientific} computing.

High performance computing is mostly driven by scientific computing and its
applications. As a consequence, it focuses on numerical computations using
approximate floating point arithmetic and parallel numerical linear algebra. Decades of
efforts in research and development in this field have produced a mature  set of
software and algorithmic practices and even impacted the design of most of nowadays computers.

On the other hand, computational mathematics, at the core of \ODK activity, differs from scientific
computing primarily on the type of arithmetic being used. There is in fact a
large variety of arithmetics to be supported depending on the application: finite
  fields of small, medium and large cardinalities, multiprecision integer and
  rationals, polynomials over these domains, etc. Obviously all these
  arithmetics need to be exact which defeats a direct use of floating point
  arithmetic. Another challenge is the deep interplay between these arithmetics
  which are often composed in high stacks of algebraic structures: e.g. tensor
  algebras built on top of algebras built on top of combinatorics and
  fractions, the latter being built on polynomials built on arithmetic.

Consequently, the experience of the numerical scientific computing community can not
be blindly exploited in the development of high performance mathematical
computing.
Driven by emerging applications, such as experimental mathematics, crytanalysis,
discrete optimization, and bioinformatics, high performance mathematical computing has
become an active field in computer algebra over the last two decades, but is
comparatively at an earlier stage of development, given the smaller community
and the broader scope to be adressed.
However interactions with the numerical scientific computing community
have always been intense and several crucial innovations
resulted from a convergence between the two fields: floating point arithmetic
can be used under control for finite field linear algebra, sparse solvers used
in cryptography are adapted from iterative numerical methods, etc.

In the first evaluation period of the project, most efforts have been put
towards tasks specific to mathematical computing: improving some core computational kernels, for polynomial and finite
field arithmetic (\delivref{hpc}{MPIRsuperoptimiser}  and \delivref{hpc}{FFT}),
parallelization of recursive tree explorations~\delivref{hpc}{sage-HPCcombi}, etc. This could explain a
feeling that these contributions are not in line with mainstream numerical high
performance computing research.
Achievements during second reporting period, which will be presented
at the formal review, include significant contributions to parallel
exact linear algebra, which is much closer in nature to numerical HPC.

Participants of the \WPref{hpc} are (and have been for a long time) designing
leading edge software for high performance mathematical computing and are in
continuous interaction with the mainstream numerical HPC community.
More precisely, we identify that the major forthcoming interactions will be in
the following aspects:
\begin{description}
\item[Parallel runtime for task based parallelization] Parallel runtime systems
  are becoming a key component to properly harness the always growing number
  parallel cores. Shifting from low level thread management to higher level
  parallel programming languages and runtimes has become a hot topic in numerical
  HPC. Convergence in this area is already happening, with for instance our
  participation to a french national workshop ``Journée Runtime'' where we could exchange with the
  numerical HPC community on our experience using XKaapi, Cilk and OpenMP for the task
  based parallelization of exact linear algebra. During the 11th e-Concertation
  Meeting, we also started an interaction with the research group at the BSC
  (Barcelona Supercomputing Center) regarding their runtime OmpSS.
\item[Automated SIMD optimization.] SIMD is the lowest level of parallelization
  available inside each processor. Automating the optimization of such low level code is a common target
  between communities of exact and numerical computation. 
  Our project's contribution in  \delivref{hpc}{MPIRsuperoptimiser} is a major
  step in this direction. Further work  involve interactions with the developpers of the BLIS project
  (numerical linear algebra) in order to promote and improve this superoptimzer.
\end{description}



%% We therefore acknowledge the importance of keeping and increasing our interactions with the
%% mainstream HPC community and of keeping us up to date with the state of the art
%% practices in the domain.


%% One message we take back home here is that we really need
%% to better communicate about this work package and its rationale. Indeed,
%% The \WPref{hpc} participants are (and have been for a long time) in
%% continuous contact with the mainstream numerical  HPC community, trying hard to
%% reuse or adapt its know-how and technologies. However, in particular
%% due to the wide variety of data structures and algorithms, the
%% challenges arising in HPC for pure mathematics are often of rather
%% different nature, calling for custom solutions.

%% Here is some evidence:
%% \begin{itemize}
%% \item There is a dedicated yearly conference on the topic, PASCO.
%%   Incidentally, it turns out that PASCO is hosted by one of our site
%%   (Kaiserslautern); Florent Hivert from Orsay is invited speaker and
%%   will present, among other things, his work around
%%   \delivref{hpc}{sage-paral-tree}. Other \ODK
%%   participants from \ODK will be presenting work or attending too.
%% \item Several participants of this work package have taken part in the
%%   past to joint research projects with established members of the HPC
%%   community (Grenoble, Kaiserslautern).
%% \item Enabling HPC in GAP by itself was a major undertaking of the
%%   previous SCIEnce project (Symbolic Computation Infrastructure for
%%   Europe FP6 eRII3-CT-026133, 2006–2011).
%% \end{itemize}

% We are planning to write a blog post about the
% interplay and differences between high performance mathematical software and
% mainstream HPC.

We nevertheless doubled our efforts to deepen our contacts with the
mainstream HPC community, and very much appreciate the contacts that
were and will be supplied by the reviewers.


\begin{recommendation}{Recommendation 11}
  The activities of the WP7 should be assessed critically and a
  revised work plan for this WP should be presented that is of greater
  relevance to the aims and goals of the project. If a satisfactory
  resolution of the issues is not reached then this WP should be
  dropped and the effort re-assigned elsewhere within the project.
\end{recommendation}

See Recommendation 3 above.

%%% Local Variables:
%%% mode: latex
%%% TeX-master: "report"
%%% End:
