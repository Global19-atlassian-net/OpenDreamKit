
\section{Key Performance Indicators (KPIs)}

The following Key Performance Indicators (KPIs) show how OpenDreamKit addresses the specific impacts listed in the work
programme. KPIs were thought through by the members of OpenDreamKit so that they are meaningful, reusable, realistic and easily measurable.
the following qualitative and quantitative indicators are divided into the four aims of OpenDreamKit. If quantitative indicators are more
useful for reporting and internal evaluation, qualitative indicators will give content for futher dissemination and communication purposes,
for example through the project website.


\subsection{KPIs for Aim 1}

\begin{recommendation}{Aim 1}
  Improve the productivity of researchers in pure mathematics and
  applications by promoting collaborations based on mathematical
  software, data, and knowledge.
\end{recommendation}

\subsubsection{Success story}

\begin{enumerate}
\item \textbf{How OpenDreamKit supported the RSE revolution}\\
  In a \href{https://opendreamkit.org/2018/10/29/ODK-RSE/}{blog post}
  on October 2018, OpenDreamKit fellow Mike Croucher describes How
  OpenDreamKit supported the RSE revolution, saying that "By 2015,
  there were a small number of central ‘Research Software Engineering
  Groups’ within UK Universities with his group at Sheffield being
  among the first. OpenDreamKit was one of the first projects they won
  that demonstrated that funders would support RSEs on major grants –
  this improved credibility of the new role a great deal and helped
  secure its future at Sheffield."

  \noindent
  Mike Croucher is a Research Software Engineer at the university
  of Sheffield, passionate about improving the quality of research
  software. He enables researchers to ask larger and more complex
  research questions by improving the software they develop. Along
  with the Software Sustainability Institute, the UK Research Software
  Engineering Association and the EU-funded OpenDreamKit project,
  Mike Croucher actively campaign to improve the career prospects
  of the talented people who underpin a huge amount of computational
  research….
\end{enumerate}

\subsection{KPIs for Aim 2 and adoption of \ODK's technologies}

\begin{recommendation}{Aim 2}
  Make it easy for teams of researchers of any size to set up custom,
  collaborative Virtual Research Environments tailored to their
  specific needs, resources and workflows. The VRE should support the
  entire life-cycle of computational work in mathematical research,
  from initial exploration to publication teaching and outreach;
\end{recommendation}

%\paragraph*{Success story: \ODK based VRE deployments} %  and generally speaking adoption of \ODK's components.

\subsubsection{Success stories}
\begin{enumerate}
\item \textbf{Jupyter's ACM Software System award}\\
  In June 2018,
  \href{https://blog.jupyter.org/jupyter-receives-the-acm-software-system-award-d433b0dfe3a2}{Jupyter
    was awarded prestigious 2017 ACM Software System}. Award previous
  winners include: UNIX, TCP/IP, the Web, TeX, Java, GCC, LLVM.

\item \textbf{Partnership with EGI}\\
  there is an ongoing collaboration between EGI (main stakeholder of
  the EOSC) and OpenDreamKit to deploy JupyterHub and BinderHub-based
  EGI services. Proofs of concepts for both have been
  deployed.% by Enol Fernandez from EGI.
  Both parties are very satisfied with the collaboration and want to
  strengthen it. A Technology Provider Agreement was signed between
  EGI and Simula, on behalf of OpenDreamKit/Jupyter developers, and a
  joint applications submitted to the EOSC call INFRAEOSC-02-2019.
%https://github.com/OpenDreamKit/OpenDreamKit.github.io/blob/7cbee70c3bff6d0778733be931c927981df87e65/meetings/2018-10-28-Luxembourg/SteeringCommitteeMeeting/ProgressReports/ParisSud.md
%https://github.com/OpenDreamKit/OpenDreamKit/blob/88ee2544e31e661c4eac7f39fa7333f24b62ab2a/WP4/D4.8/report.tex

\item \textbf{Use of Jupyter components by Logilab's Simulagora}\\
  %
  One of the flagship product of Logilab is Simulagor, an
  industry-grade VRE for collaborative computational simulation; it
  eases the deployment of micro web applications to leverage the
  access to complex simulations to end users. Since Logilab joined
  OpenDreamKit, it was able to outsource several of the core
  components of Simulagora to replace them by analog standard Jupyter
  components, bringing in more interactivity.

%As part of the Open Dream Kit project, Logilab is working on the creation of tools for mesh data visualization and analysis in a web application. The goal was to create widgets to use in Jupyter notebook (formerly IPython) for 3D visualization and analysis. They want to create a graphical user interface in order to enable users to intuitively compute multiple effects on their meshes.

%Simulagora.com, a virtual research environment has been under heavy development since the summer 2017 and a new version is online since March 2018. This showcases the virtual desktops available from a web browser and collaboration workflows based on “tools” that can be described as micro web applications that require very little development skills to set up, but make it easy to make available complex simulations to users.


%Use of ODK's computational components in collaborative workspaces: A group of mathematical researchers with access to common computational resources, such as a shared lab computer or cloud servers,has already been able to deploy a prototype VRE with JUPYTERHUB, integrating OpenDreamKit components. The Jupyter kernels for mathematical software developed as part of OpenDreamKit make computational mathematical components accessible in a JUPYTER environment, enabling a Jupyter-based deployment of the relevant tools for the researchers. The process of working on notebooks is greatly improved by review tools developed as part of WP4, enabling researchers to collaborate to some degree in a shared computational environment.

\item \textbf{Multi-Site involvement of Researchers (Mobility of Researchers)}\\
  The excellent collaboration between the \ODK partners has led to an
  increased mobility of researchers. Rehires between partners and
  joint appointments are a good measure of this:
  \begin{compactitem}
  \item PD. Dr. Florian Rabe (Joint appointment UPSud/FAU)
  \item Felix Schmoll Summer Internship (From JacU to St.Andrews)
  \item Prof. Nathan Carter (Bentley Univ.) in St. Andrews (Sabbatical)
  \end{compactitem}
\item \textbf{Heavy interest by the theorem proving community about MitM Ontology}
  \begin{compactitem}
  \item Logipedia (\url {http://logipedia.science}) adopts the MitM principle of integrating (logical) systems by aligning concepts.
  \item First ODK-external MitM ``user'' for the next months: Andrea Thevis, Saarbrücken
  \end{compactitem}
\item \textbf{Blogs about how to deploy Jupyter-based VREs}
  \begin{itemize}
  \item Luca De Feo: \href{https://opendreamkit.org/2018/10/17/jupyterhub-docker/}{Deploying a containerized JupyterHub server with Docker}
  \item Nicolas Thiéry: \href{https://opendreamkit.org/2018/03/15/jupyterhub-binder-convergence/}{Toward versatile JupyterHub deployments, with the Binder and JupyterHub convergence}
  \item Loic gouarin: \href{https://blog.jupyter.org/how-to-deploy-jupyterhub-with-kubernetes-on-openstack-f8f6120d4b1}{Deploying JupyterHub with Kubernetes on OpenStack}
  \end{itemize}
\end{enumerate}

\subsubsection{Quantitative metrics}
\begin{enumerate}
\item \textbf{\ODK based VRE deployments}\\
  We are
  \href{https://github.com/OpenDreamKit/OpenDreamKit/issues/174}{tracking}
  collaborations with various institutions and projects to deploy
  instances of JupyterHub and CoCalc (formerly SageMathCloud). There
  are \textbf{seven major OpenDreamKit based VRE deployments} so far:
  five at the scale of academic institutions (universities of Zurich,
  Paris Sud / Paris Saclay, Sheffield, UVSQ, Gent), and two at the
  scale of research networks (EGI, Mathrice).
  % TODO: as the number increase, separate between local, national, european

  % one on EGI infrastructure (EGI is a
  % federation of publicly funded computing and storage resource
  % providers for research and innovation) and one by the Mathrice
  % Group. which brings together computer specialists from mathematics
  % laboratories, and the Computing group, which conducts exchanges
  % between experts in scientific computing, was also involved in a
  % deployment. There is also a growing interest in planning a
  % JupyterHub deployment at the Einstein Institute of Mathematics,
  % part
  % of the Hebrew University of Jerusalem, in deploying a local Cocalc
  % instance at UPSud and in Integrating Sage in the tmpnb.org's
  % temporary notebook server.
\item \textbf{Systems and languages integrated in Jupyter}\\
  At the begining of ODK, there were 49 Jupyter kernels (languages or
  systems that could be used in Jupyter). There are now 117 of them, 6
  of which were contributed or significatively improved by ODK.
\item \textbf{Usage of Jupyter notebooks}\\
  Millions of Jupyter notebooks are already online with over 3 million
  on GitHub alone. Thousands of them are using Jupyter kernels
  (co)developped by ODK (SageMath: 6199, Xeus-cling C++: 684, GAP: 63,
  Singular: 8, PARI/GP: 3, MMT: 1).

\item \textbf{Download statistics for SageMath's Windows installer}\\
  OpenDreamKit invested 6 months worth of work in a native Windows
  application for SageMath which was delivered in mid 2017. In 2018,
  it counts for 44\% of SageMath downloads.

  % are for the Windows platform, a number that is essentially stable
  % since 2015 (showing that the native port has not been a source of
  % problems to Windows users).

\item \textbf{Download statistics via platform-specific distribution
    channels}\\
  At the beginning of \ODK, not all of ODK's computational systems
  were available in the standard package repositories of major Linux
  distributions (e.g., Debian, Ubuntu, Fedora, \dots). Following major
  efforts from \ODK and the community this was resolved in 2017. with
  the availability of SageMath on Debian/Ubuntu.

  \noindent
  The ``popularity contests''\footnote{Public statistics on package
    use, based on voluntary submissions by users.} organized by the
  various Linux distributions give widely different numbers,
  reflecting the varying interests of the communities tied to each
  distribution. Because Jupyter is used in many research areas outside
  of mathematics, it is interesting to compare the percentage of ODK
  users to that of Jupyter users. For example, in 2018, in the Arch
  Linux distribution 50\% of Jupyter users are also ODK users. This
  percentage drops to 10\% on Debian and Ubuntu, which is unsurprising
  given that these distributions are much more popular in the server
  ecosystem.

  \noindent
  It is also (mildly) interesting to compare the number of
  \emph{pulls} on DockerHub for Jupyter and for ODK's containers
  (available for GAP and SageMath). In 2018, Jupyter's
  \texttt{scipy-notebook} container has more than 1 million pulls,
  whereas ODK components have between 4 and 10 thousands.

  \noindent
  This numbers should be taken with a grain of salt, since they are
  automatically incremented by continuous integration/testing systems
  deployed by the projects themselves.

  \noindent
  ODK plans to make its components available on CondaForge. The first
  (experimental) packages for SageMath currently have 6K downloads
  (compare to 380K downloads for Jupyter), mostly coming from beta
  testers and continuous integration/testing.
\end{enumerate}

\subsection{KPIs for Aim 3}

\begin{recommendation}{Aim 3}
  Identify and promote best practices in computational mathematical research including: making results easily reproducible; producing
  reusable and easily accessible software; sharing data in a semantically sound way; exploiting and supporting the growing
  ecosystem of computational tools.
\end{recommendation}

\subsubsection{Success stories}

\begin{enumerate}
%\item Best practice and tools for correct and reproducible research
\item \textbf{Mike Croucher's talk ``Is your research software correct''}\\
  This excellent talk highlights crucial best practice whenever
  software is used in research, including open code and data sharing,
  automation, use of high level languages, software training, version
  control, pair programming, literate computing, or testing. A lot of
  the work in ODK relates to disseminating this set of best practice
  (\longWPref{dissem}), and enabling it through appropriate technology
  (\longWPref{UI}). Just to cite a few examples,
  \longdelivref{UI}{jupyter-collab}, and
  \longdelivref{UI}{jupyter-test} enable respectively version control
  and testing in the \Jupyter literate computing technology, while
  Mike's talk is and will be delivered in several of ODK's many
  training events.

\item \textbf{RSE Conference Workshop: ``Reproducible research with Jupyter''}
  (\href{https://opendreamkit.org/2018/03/07/opendreamkit-at-the-rse-conference/}{blog post})\\
  OpenDreamKit member, Tania Allard, ran a hands-on workshop on
  Jupyter notebooks for reproducible research at the UK RSE
  Conference. This workshop focused on the use of Jupyter notebooks as
  a means to disseminate reproducible analysis workflows and how this
  can be leveraged using tools such as nbdime and nbval. Both nbdime
  and nbval were developed by members of the OpenDreamKit project as a
  response to the growing popularity of the Jupyter notebooks and the
  lack of native integration between these technologies and existing
  version control and validation/testing tools. An exceptional win was
  that this workshop was, in fact, \textbf{one of the most popular events of
  the conference} and we were asked to run it twice as it was massively
  oversubscribed. This reflects, on one hand, the popularity of
  Jupyter notebooks due to the boom of literate programming and its
  focus on human-readable code. Allowing researchers to share their
  findings and the code they used along the way in a compelling
  narrative. On the other hand, it demonstrates the importance of
  reproducible science and the need for tools that help RSE and
  researchers to achieve this goal, which aligns perfectly with the
  goals of OpenDreamKit.
  %

\item \textbf{Best presentation: 3D at the rescue for visualizing large mathematical ontologies}
  \href{https://opendreamkit.org/2018/08/20/tgview3d.md}{blog post}\\
  The Math-in-the-Middle (MitM) ontology and the system API theories
  in the MitM paradigm are big theory graphs with thousands of nodes
  and edges. Understanding and interacting with such large and complex
  objects is very difficult.The FAU group has conducted research into
  whether virtual reality technologies are helpful for this task. We
  have presented a first working prototype at the Conference on
  Intelligent Computer Mathematics CICM 2018 and the author: Richard
  Marcus - a master's student at FAU has received a prize for best
  presentation.
\end{enumerate}

%Some research paper that showcases a range of best practices supported by ODK work (paper written collaboratively on e.g. github, software distributed as e.g. SageMath package, live demo and logbooks on binder, nbdime for collaboration, ...).

\subsubsection{Use Case blog posts}

As part of our training and dissemination activities, we have authored
a series of Use Case blog posts: examples of applications that have
been made possible through the OpenDreamKit project. For each one, we
provide a brief tutorial on how to tackle it OpenDreamKit supported
tools, link to some examples, and suggest best practices. The most
important ones are illustrated by an explainer cartoon authored by
Juliette Belin, graphic designer working at Logilab.

\begin{enumerate}
%4 use cases: %see:https://opendreamkit.org/tag/use-case
\item \textbf{Publishing reproducible logbooks}
  \href{https://opendreamkit.org/2017/11/02/use-case-publishing-reproducible-notebooks}{blog post}
\item \textbf{Live online slides with SageMath, Jupyter notebooks, RISE and Binder}
  \href{https://opendreamkit.org/2018/07/23/live-online-slides-with-sagemath-jupyter-rise-binder/}{blog post}
\item \textbf{WP6 Math-in-the-Middle Integration Use Case}
  \href{https://opendreamkit.org/2017/10/15/WP6-Usecase/}{blog post}\\
  Two papers published at MACIS-2017
\item \textbf{Mixing Data and Computation to explore mathematical data sets: Knowledge to the rescue with LMFDB + SageMath + Pari + MitM}
  \href{https://opendreamkit.org/2018/05/16/lmfdb-usecase/}{blog post}
% \item https://opendreamkit.org/2018/03/15/jupyterhub-binder-convergence/
\end{enumerate}

\subsubsection{Quantitative metrics}

\begin{enumerate}
\item \textbf{GAP packages: activity and adoption of best practices}\\
  GAP has a powerful extension system that allows users to share their
  research code through an official channel. Thanks to GAP's
  continuous integration/testing, extension developers have been
  encouraged and helped to keep their code up to date. The \emph{code
    coverage}\footnote{Percentage of code covered by tests} of GAP has
  gone up from 69\% for the 4.9 release, to 75\% for the 4.10 release.
  In 2018, 50\% of GAP extension packages have been updated in the
  past year.
\item \textbf{SageMath packages on PyPI}\\
  Following the lead of GAP, SageMath has been advocating, with the
  support of OpenDreamKit, the Python package repository PyPI for
  users to share their research code. This channel is still in its
  infancy. In 2015 there were a handful of packages on PyPI for
  SageMath; in 2018 the number grew to 80.
\item \textbf{Systems made interoperable with the Math-in-the-Middle architecture}\\
  % More generally: metrics on the scale of the Math-in-the-Middle
  % architecture; e.g. number of API CDs generated and number of
  % alignements the current state of play is that we have initial
  % exports of system interface ontologies for three systems:

  The Math-in-the-Middle interoperability architectures is a
  foundational new way of making open-API systems interoperable
  (see~\cite{ODK-D6.5} for an overview). Thanks to a fully functional
  prototype integrating GAP, SAGE, SINGULAR, and LMFDB via the SCSCP
  Protocol, users can run calculations involving any combination of
  those systems from any of them.

  The following numbers highlight the state of play as of M36 (details in~\cite{ODK-D6.8}):
  \begin{compactdesc}
  \item[MitM-connected Systems] four (GAP, Sage, LMFDB, Singular)(See D6.5);
    % Upcoming: OEIS, FindStat
  \item[Formal MitM Ontology] 55 files, 2600 LoF, 360 commits (See D6.8);
  \item[Informal MitM Ontology] 815 theories, 1700 concepts in
    English, German,(Romanian, Chinese);
  \item[MitM System API Theories (GAP, Sage, LMFDB, Singular)] 1.000+;
    Theories, 22.000 Concepts;
  \item[SageMath Ontology] 512 CDs with 2800 entries;
  \item[GAP Ontology] 218 CDs with 2996 entries.
  \end{compactdesc}

%In the course of the deliberations in the WP6 workshops we saw a shift from the development of computational foundations and verification towards API/Interface function specifications to enable semantic system interoperability via the Math-in-the-Middle Ontology. Consequently, emphasis has changed to the generation of API Content Dictionaries (API CDs) for GAP, LMFDB and SAGE. We have a prototypical set of GAP and SAGE Content Dictionaries in OMDoc/MMT form (GAP: 218 CDs, 2996 entries; SAGE: 512 CDs, 2800 entries overall). The computational foundations exist but are rather more simple than originally anticipated. Much of the functionality has been offloaded to the SCSCP standard – remote procedure call with OpenMath representations of the mathematical objects – developed in the SCIENCE Project. As a direct consequence of the work in OpenDreamKit the OpenMath Society has promoted the SCSCP protocol into as an OpenMath Standard.Conversely, the GAP and SAGE CDs are rather more elaborated than anticipated in the proposal, and thus form a viable basis for alignment with the MitM Ontology.
% https://github.com/OpenDreamKit/OpenDreamKit/blob/88ee2544e31e661c4eac7f39fa7333f24b62ab2a/terminations/report/wp6.tex
%https://github.com/OpenDreamKit/OpenDreamKit.github.io/blob/333b21524f753dd49e441b4b1157ded68999eeff/meetings/2018-02-01-SteeringCommitteeMeeting/ProgressReports/Zurich.md
\end{enumerate}

\subsection{KPIs for Aim 4}

\begin{recommendation}{Aim 4}
  Maximise sustainability and impact in mathematics, neighbouring
  fields, and scientific computing.
\end{recommendation}

\subsubsection{OpenDreamKit's web site statistics}

\begin{itemize}
  % 7 activities on the website
\item 6 video interviews and explainer material totaling 400 views
\item 6 press releases
\item 31 blogs (28 blogs and 3 technical blogposts)
\item 471 followers
\item 6500 visits in 2018
\end{itemize}

\subsubsection{Workshops, conferences, events}

% Statistics on workshops organized and conference presentations
% delivered as part of our dissemination activities, including
% estimates of number of attendees and what (if anything) happened as
% a follow-up.

An important part of the success of the ODK project is linked to its
ability to foster a community in the spirit of the open source
projects it is built on. Part of this relies on the organization and
participation to scientific and development events of many different
scales and objectives.

Over the last four years OpenDreamKit has organized or coorganized 66
events:
\begin{itemize}
\item 21 development workshops and project meetings
\item 26 training workshops or sessions adding 1100 trainees
\item 13 community building workshops
\item 6 research workshops
\end{itemize}
In addition, it participated to 19 external events.

\subsubsection{Diversity success stories}

\begin{enumerate}
\item \textbf{Women in \Sage workshops} \href{https://opendreamkit.org/2017/04/06/WomenInSage/}{blog post}\\
  The under-representation of women in the scientific world is even
  more visible if we intersect science with software development. As
  we know, we have many talented women in our community.

  Last January, Viviane Pons, Jessica Striker and Jennifer
  Balakrishnan organized the first WomenInSage event in Europe with
  OpenDreamKit. 20 women spent a week together coding and learning in
  a rented house in the Paris area. We took advantage of the diverse
  knowledge background of our group to work together and learn from
  each other. It was an occasion for many "first times" among
  participants who had very little experience with Sage.

  This sparked the organization of another Women in Sage workshop in
  Montreal in June 2018. OpenDreamKit is organizing a new one in 2019
  in Crete.

\item \textbf{Women in computing}\\
  In partnership with CodeFirstGirls, OpenDreamKit developed training
  materials and provided training for over 130 women in the last 12
  months at Sheffield and Manchester

\item \textbf{ODK RSE Tania Allard was invited by NumFocus to
    participate in the Diversity and Inclusion in Scientific Computing
    unconference}

\item \textbf{ODK RSE Tania Allard diversity chair for the 2017
    International Research Software Engineering conference.}

\item \textbf{Training workshops in developing countries }\\
  8 training workshops were organised in developing countries
  including Algeria, Liban, Tunisia, Colombia, Morocco and Mexico, and
  attended by about 451 trainees.
\end{enumerate}

\subsubsection{Adoption of ODK technologies for teaching}

We collect some success stories and statistics on courses and
departments we worked with directly and an estimate of how many
students this subsequently affected.

\begin{enumerate}
\item \textbf{African Institute for Mathematical sciences}\\
  In early spring 2017, Prof.~Dr.~W.~Decker and Prof.~Dr.~G.~Pfister
  gave a three-week course on computational algebraic geometry at the
  African Institute for Mathematical sciences (Cape Town, South
  Africa) with lectures and computer lab sessions. The course was
  attended by about 50 students from all over Africa. In the lab
  sessions, the students learned how to experiment with the computer
  algebra system Singular. It proved extremely valuable that the
  students could run Singular in the Jupyter notebook.

\item \textbf{PGTC "Software tools for mathematics" workshop 2018}\\
  Alexander Konovolov use GAP Jupyter interface in teaching
  Reproducible GAP experiments on Binder.
  % Here https://github.com/alex-konovalov
  % /gap-teaching, you can find a collection of GAP Jupyter notebooks. It uses the Docker container with the latest public release of GAP, which
  % is maintained in a separate repository at https://github.com/gap-system/gap-docker.

\item \textbf{University of Sheffield}\\
  Cocalc was adopted for teaching in UK after ODK's Kickoff: Support
  was given to a number of lecturers in Sheffield to migrate to
  Jupyter and CoCalc (formerly SageMathCloud) but also to those that
  had already been using CoCalc and Jupyter notebooks for their
  courses (D2.17, T2.6):
  \begin{itemize}
  \item Courses in physics: 500 students yearly.
  \item Courses inBiomedical Sciences
  \item Courses inComputer Science
  \item Courses inBioinformatics
  \item Courses inMaterials Science
    % #277) https://github.com/OpenDreamKit/OpenDreamKit/issues/277
  \end{itemize}

  In addition, a previously generated CoCalc tutorial was extended by
  adding tutorial sections for students having courses in CoCalc as
  well as with a hands-on tutorial for lecturers to get started. The
  material can be found as a website at
  \url{https://tutorial.cocalc.com/}.

\item \textbf{Université Paris Sud}\\
  Since 2017, Jupyter is used at Université Paris Sud for teaching C++
  to over 400 students.This was initiated in particular by \ODK
  participants Loïc Gouarin, Viviane Pons, and Nicolas M.\ Thiéry. The
  mix of narrative documents and interactive programming fostered
  active participation from the students while our web-based
  deployment made it easier for them to work from home.The course
  material is available from
  \url{http://Nicolas.Thiery.name/Enseignement/Info111}.

  In addition, ODK participants were involved in the following
  Jupyter-based courses:
  \begin{itemize}
  \item Projet Math-info (Bachelor). Tools: \Sage. 40 students yearly.
  \item Algèbre et Calcul Formel (Master). Tools: \Sage. 25 students yearly.
  \item Algorithmique (Engineers). 20 students yearly.
  \item Algorithmique (Master). 20 students yearly.
  \item Combinatorics (Master). 20 students yearly.
  \end{itemize}

\item \textbf{University of Granada}\\
  The \GAP Jupyter kernel was used by Pedro Garcia-Sanchez to teach a
  master course in mathematical software at the University of Granada.
  See
  \url{https://github.com/pedritomelenas/Software-Matematicas-GAP}.
  Pedro has taken on the technology, and is now involved in the
  development of interactive visualization widgets for discrete maths
  (package Francy; see
  also~\longdelivref{UI}{ipython-advanced-interacts}) in particular
  for use in other courses.

\item \textbf{University of Ghent}
  \begin{itemize}
  \item Courses in mathematics (Bachelor, Master). Tools: CoCalc,
    Jupyter, SageMath. 31 students yearly.
  \item Courses in chemistry (Master). Tools: SageMath. 37 students
    yearly.
  \end{itemize}

\item \textbf{FAU Erlangen-Nürnberg}
  \begin{itemize}
  \item Logic-based Natural Language Semantics. Tools: MMT, domain
    modeling and reasoning system. 5-10 students yearly.
  \item Knowledge Representation for Mathematics and Technology.
    Tools: MMT, domain modeling and reasoning system.
    5-10 students yearly.
 \end{itemize}

\item \textbf{Université Versailles Saint Quentin}
  \begin{itemize}
  \item Algèbre Commutative et Effective (M). Tools: Jupyterhub,
    SageMath, Planetaryum. 22 students yearly.
  \item Algorithmique et Programmation C (M). Tools: JupyterHub,
    JupyterLab, C. 22 students yearly
  \item Introduction au calcul formel (M). Tools: Jupyterhub,
    SageMath. 20 students yearly
  \item Algorithmique (M). Tools: Jupyterhub, JupyterLab, Python. 20
    students yearly.
  \item Database (M). Tools: JupyterHub, Apache Spark. 60 students yearly.
 \end{itemize}

\item \textbf{ Bioinformatics Awareness Days}\\
An event that took place in The Sheffield Institute for Translational Neuroscience on November 2017 where were given training activities
about bioinformatics workflows using Jupyter notebooks with computation provided by the free Micosost Azure Notebook service.The event
demonstrated that \ODK supported technologies could be applied to the field of Bioinformatics and led to a new collaboration between Dr
Cutillo and \ODK member Mike Croucher.

Following the success of this workshop, Dr Cutillo independently taught an introductory workshop on statistics using Jupyter notebooks
on Azure at Parthenope University of Naples (Materials at \url{https://github.com/luisacutillo78/RbasicStats}) r Cutillo has since moved
to University of Leeds where she will be teaching statistics to 200+ undergraduates. She plans to use \ODK developed technologies
in collaboration with the Research Software Engineering group at Leeds.

The event required also the development of a website that was linked to the Jupyter notebooks (\url{https://bitsandchips.me/BAD_days/}). The
website caught the attention of Eleni Vasilaki, Head of Machine Learning at University of Sheffield who wanted to do something similar for
her course on Adaptive Intelligence. We supported her in this endeavour and the result is at (\url{http://bitsandchips.me
/COM3240_Adaptive_Intelligence/}).

In order to better support this, \ODK member Tania Allard, developed a Jekyll template for use by academics and researchers using Jupyter
notebooks for course materials and dissemination. This led to the development of a Python package: nbjekyll (\url{https://github.com
/trallard/nbjekyll}) that complements the Jekyll template.As well as being used internally at Sheffield, The nbjekyll package received some
attention on twitter \url{https://twitter.com/jdblischak/status/1009800776305332224} and \url{https://twitter.com/walkingrandomly/status
/1009414151716909057} receiving a total of 42 retweets and 80 'likes'% ( à remettre à jour).
\end{enumerate}

\subsubsection{Impact of some tools developed by OpenDreamKit}

% TODO: Stories about the impact of the Micromagnetics VRE;
% TODO: D4.13 (Sphinx) might have impact beyond ODK. For example, Simon King
%       is interested in things done for D4.13 for his
%       \software{p\_group\_cohomology} package. It might also lead to a PEP,
%       which by itself counts as impact.

%Vidar Fauske PRESENTATION filling in for Benjamin Ragan-Kelley during Second OpenDreamKit Review, on October 30, 2018 in Luxembourg
%https://opendreamkit.org/meetings/2018-10-28-Luxembourg/ProjectReview/wp4.pdf
%TODO: Could you please add more narrative text or more context? the information below appears in your presentation but listed that way this is too technical.

\begin{enumerate}
\item \textbf{nbdime}: Tool for diffing Jupyter notebooks\\
  GitHub statistics: 855 stars on github, 64 contributors (36 in 12
  months prior), 611 comments, 239 new issues (241 closed). Delivered
  in the first period (D4.6), \software{nbdime} has been met with
  enthusiasm and widely adopted. This led to further developments,
  including integration in the Jupyter Notebook,
  Jupyter Lab, and the version control git through extensions.
\item \textbf{Thebelab}: Jupyter-based javascript library for integrating live code in static web pages\\
  GitHub statistics: 44 stars, 15 contributors, 151 new issues (118 closed), 323 comments.\\
\item \textbf{K3D-Jupyter}: 3D visualisation in the Jupyter notebook.\\
  GitHub statistics: 48 stars, 14 contributors, 140 new issues (129 closed), 303 comments.
\end{enumerate}

%%% Local Variables:
%%% mode: latex
%%% TeX-master: "report"
%%% End:
