
\section{KPIs}

\begin{introduction}
 After some internal discussion and project meetings, we revamp our KPIs according 
to the oral written recommendations of the reviewers. We also finalized the classification
of the KPIs and added more metrics. We report here on the current state of our KPIs.
\end{introduction}

\subsubsection{KPIs for Aim 1}

\begin{Aim 1}
  Improve the productivity of researchers in pure mathematics and
  applications by promoting collaborations based on mathematical
  software, data, and knowledge.
\end{recommendation}

Qualitative metrics: success stories, reported as blog posts.

TODO: review our existing blogs
story of cocalc adoption for teaching in UK after ODK's Kickoff
stories of Mike (Michael Croucher)
getting Katja involved in OpenDreamKit (Katja, Michael, Samuel)
"Software tools for mathematics" workshops (Katja, Samuel)
Reproducible GAP experiments on Binder (AlexK) to write about https://github.com/alex-konovalov/gap-teaching
AlexK used GAP Jupyter interface in teaching at PGTC 2018 and "Software tools for mathematics" workshop
Wanted: blog post about Jupyter in GAP and other CAS https://www.gapdays.de/gap-jupyter-days2018/

\subsubsection{KPIs for Aim 2 / adoption of \ODK's technologies}

\begin{Aim 2}
  Make it easy for teams of researchers of any size to set up custom,
  collaborative Virtual Research Environments tailored to their
  specific needs, resources and workflows, ...
\end{recommendation}

Qualitative metrics:
\begin{itemize}
\item Success stories about \ODK based VRE deployments (university-wide, Mathrice, EGI, Micromagnetics VRE, ...), 
and generally speaking adoption of \ODK's components.
\item added metrics:

  the goal of this issue is to track collaborations with various institutions and projects
  to deploy instances of JupyterHub and CoCalc (formerly SageMathCloud), and collect estimates of
  the difficulties involved in such deployments in various use cases to:
  - help plan future deployments
  - seek what could be done to ease deployment

  ## Running:
  - [x] Local CoCalc instance at Universität Zürich.
      Deployed September 2015 - February 2016.
      People involved: @pdehaye, @williamstein
  - [x] [Instance of JupyterHub](https://jupyter.math.cnrs.fr/hub/) deployed by the [Mathrice group](http://mathrice.fr/)
      Host Infrastructure: France Grille's LAL cloud
      Users: members of math labs in France
      Main use case: casual use 
      People involved: the Mathrice group, @nthiery

  - [x] Local [JupyterHub instance at Université Paris Sud / Paris Saclay](http://jupytercloud.lal.in2p3.fr/)
      Host Infrastructure: France Grille's LAL cloud
      Users: personnel and students of UPSud / Paris Saclay
      Main use case: use in classroom (Python, Sage, C++), casual use
      People involved: @VivianePons, @nthiery, @gouarin 
  - [x] JupyterHub instance deployed on USheffield's HPC system
      http://docs.iceberg.shef.ac.uk/en/latest/using-iceberg/accessing/jupyterhub.html
      People involved: @mikecroucher
  - [x] JupyterHub instance(s) deployed at UVSQ
      https://opendreamkit.org/2018/10/17/jupyterhub-docker/
      Main use case: use in classroom (Sage, Python, C, Apache Spark), casual use
      People involved: @defeo 
  - [x] [Gallery of JupyterHub instances](https://jupyterhub.readthedocs.io/en/latest/gallery-jhub-deployments.html)
  - [x] JupyterHub and Binder instances deployed on EGI infrastructure; see #205.
  - [x] Easy deployment of live GAP/SageMath/... notebooks with [mybinder](mybinder.org), 
  thanks to the Docker containers (#58);potential alternatives: [Debian packaging](https://wiki.debian.org/DebianScience/Sage) 
  and [Conda packaging](https://wiki.sagemath.org/Conda).
    People involved: @nthiery, ...
  - [x] Local instance of CoCalc (using the Docker container) at the University of Gent
    Main use case: teaching for mathematics students

  ## Investigated or under investigation:
  - [ ] Interest in a JupyterHub deployment at the [Einstein Institute of Mathematics](http://math.huji.ac.il), part of the Hebrew University of Jerusalem
      Related: [Sage Days 79](https://wiki.sagemath.org/days79)
  - [ ] Local Cocalc instance at UPSud
      Host infrastructure: UPSud's cloud.
      Users: personnel and students of UPSud
      Main use case: use in classroom (Python, Sage, C++), casual use
      People involved: @VivianePons, @nthiery, @gouarin 
      For now, priority has been given to the above JupyterHub instance
  - [ ] Integration of Sage in the tmpnb.org's temporary notebook server
      People involved: @rgbkrk, @nthiery
      Status: some experiments run during a sprint at Pycon'15. Now that Sage's
      Jupyter kernel is well integrated in the stable version of Sage, it's just a question
      of installing Sage in tmpnb's docker container.
      This is superseded by [binder](http://mybinder.org); see above.
\end{itemize}

Quantitative metrics:
\begin{description}
\item List of known \ODK based VRE deployments, as tracked on
  \url{https://github.com/OpenDreamKit/OpenDreamKit/issues/174}
\item Number of installs of \ODK's components via platform-specific
  distribution channels: Debian popcon, Arch statistics, installer
  downloads, \dots
%\item[WP3] Decrease in set-up time for dissemination workshops based
%  on \ODK components. This is hard to measure objectively, better to
%  gather user stories.
\end{description}

GAP packages code coverage: 69% (4.9) → 75% (4.10).
Freshness of GAP packages: 50% released in the last year.
SageMath on Windows: 44% (happy) Windows users.
Packaging (not counting alt. methods, such as Conda):
Arch: 50% of Jupyter users are also ODK users;
Debian: 10% of Jupyter users are also ODK users.
Medium sized VRE deployments: 20h of work, 244 LOCs.
Docker Hub: 4K-10K pulls of ODK images.

\subsubsection{KPIs for Aim 3}

\begin{recommendation}{Aim 3}
  Identify and promote best practices in computational mathematical
  research including: making results easily reproducible; producing
  reusable and easily accessible software; sharing data in a
  semantically sound way; exploiting and supporting the growing
  ecosystem of computational tools.
\end{recommendation}

Qualitative metrics: Success stories
\begin{itemize}
\item Mike Croucher's talk ``Is your research software correct''
\item Some research paper that showcases a range of best practices
  supported by ODK work (paper written collaboratively on e.g. github,
  software distributed as e.g. SageMath package, live demo and
  logbooks on binder, nbdime for collaboration, ...).
\item ...
\end{itemize}

OpenDreamKit bet on Jupyter notebooks: It has paid off!
Millions of notebooks online (over 3M on GitHub alone)
June 2018, Jupyter awarded prestigious 2017 ACM Software System Award
Previous winners include: UNIX, TCP/IP, the Web, TeX, Java, GCC, LLVM

Background: Jupyter notebooks
Document with code, prose, maths, visualization
It also has support for rich, interactive UI components:
Web-based interactive computing environment
Language-agnostic protocol for computation
Extensible VRE built around Jupyter
Notebook Interfaces
Key areas:
Improve working in notebooks
Improve working with notebooks

Quantitative metrics:
\begin{itemize}
\item Number of PyPI hosted packages for \Sage, and similarly for
  other components.
\item Number of additional systems made interoperable with the
  Math-in-the-Middle architecture, on top of the three for the Month
  36 prototype
  % Michael: I am expecting about 2 (OEIS and Findstat) any more than that would be a success.

\item Some metrics on the scale of the Math-in-the-Middle architecture; e.g. number of API CDs generated and number of
alignments (very technical).
\item More metrics required.
\end{itemize}

KPIs and Deliverables for WP6
MitM-connected Systems: four (GAP, Sage, LMFDB, Singular)(See D6.5)
Formal MitM Ontology: 55 files, 2600 LoF, 360 commits (See D6.8)
Informal MitM Ontology: 815 theories, 1700 concepts in English, German,(Romanian, Chinese)
MitM System API Theories (GAP, Sage, LMFDB, Singular): 1.000+ Theories, 22.000 Concepts.
Multi-Site involvement of Researchers (Mobility of Researchers)
PD. Dr. Florian Rabe (Joint appointment UPSud/FAU)
Felix Schmoll Summer Internship (From JacU to St.Andrews)
Prof. Nathan Carter (Bentley Univ.) in St. Andrews (Sabbatical)
Heavy interest by the theorem proving community about MitM Ontology
Logipedia (http://logipedia.science) adopts the MitM principle of integrating (logical) systems by aligning concepts.
First ODK-external MitM “user” for the next months: Andrea Thevis, Saarbrücken 

Conclusion: What are we doing in WP6 in terms of a VRE
SageMath/CoCalc and WP6 approach (Math-in-the-Middle; MitM) are both attempts at making a VRE Toolkit.
SageMath/CoCalc is very successful, because integration is lightweight:
It makes no assumption on the meaning of math objects exchanged.
Restricts itself to master-slave integration of systems into SageMath.
But there are safety, extensibility, and flexibility issues!
MitM tries to take the high road (make possible by OpenDreamKit)
Safety: by semantic (i.e. context-aware) objects passed.
Extensibility: any open-API system (i.e. with API CDs) can play.
Flexibility: full peer-to-peer possibilities.
(future: service discovery)
But we have to develop a whole new framework!(Review 1;Proof of Concept)
Review Period2: State of WP6 (MitM) Integration
Developed mathematical use-cases (what do researchers want to do)
Extended middleware, grown MitM ontology, collected alignments
Jupyter integration into MathHub.info
Plan for Review Period 3: Extend to external use-cases users  (scale and deploy publicly)
overall pattern: design – prototype – scale

Highlight: OpenDreamKit kernels
Now 117 Jupyter kernels (49 when ODK started), 6 contributed to by ODK.
Further improved kernels from first reporting period (GAP, PARI, Singular). 
Delivered as D4.7, and MMT as part of WP6
Also contributed to kernels: cling (C++), SageMath
KPI: ODK Kernels on GitHub
Notebooks found on GitHub using each kernel (code):
SageMath: 6199
Xeus-cling: 684
GAP: 63
Singular: 8
PARI/GP: 3
MMT: 1
Highlight: nbdime
Further development on the nbdime project, delivered in the first period 
(D4.6). Has been met with enthusiasm, adoption in the community.
Jupyter Notebook and Jupyter Lab extensions, git integration.
Highlight: real-time collaboration
D4.15 prototype and plan for live collaboration in JupyterLab.
Optimistic about good integration during the final year of ODK.
Highlight: 3D visualization in Jupyter notebooks
D4.12: Jupyter extension for 3D vis, demonstrated with fluid-dynamics
Packages:
k3d-jupyter
ipydatawidgets
ipyscales○unray
Improved distribution
Highlight: Dynamic documentation and exploration system
Delivered D4.16 as two new packages, 
built on Jupyter widgets for interactively 
exploring objects in Sage
○Sage Combinat Widgets
○Sage Explorer
Interactive Documents
Key areas:
Active Documents
Interactive Documentation
Highlight: Active Documents Workshop
Workshop on live documents hosted in Oslo. Resulted in new package: 
thebelab, for embedding execution on any page, integrating tools from 
JupyterLab and MyBinder.org, demonstrating value of coordination.
Highlight: MathHub notebook integration
MathHub.info is a portal for active mathematical documents. As part of D4.11, a 
notebook integration with MathHub was added. This allows:
Authoring MathHub documents as a Notebook
Interactively exploring existing MathHub documents as a Notebook.
Highlight: Sage documentation
Refactorisation of SageMath’s Sphinx documentation system as part of D4.13
Improve Sphinx support for Cython projects.
○Enabled building proper documentation for fpylll, CyPari2, CySignals.
To completely enable Cython documentation out of the box, Python needs to 
be fixed. For this, we submitted PEP (Python Enhancement Proposal) 580.
Highlight: PARI bindings
Improved PARI/GP bindings delivered as D4.10
CyPari2 used to be part of SageMath, but it was made a separate package in 
D4.10 (see also D4.1). This ties into WP3.
KPI: JupyterHub deployments
Local CoCalc instance at Universität Zürich.
Deployed September 2015 - February 2016.
People involved: @pdehaye, @williamstein
Instance of JupyterHub
 deployed by the 
Mathrice group
Host Infrastructure: France Grille's LAL cloud
Users: members of math labs in France
Main use case: casual use
 Local 
JupyterHub instance at Université Paris Sud / Paris Saclay
Host Infrastructure: France Grille's LAL cloud
Users: personnel and students of UPSud / Paris Saclay
Main use case: use in classroom (Python, Sage, C++), casual use
People involved: @VivianePons, @nthiery, @gouarin
JupyterHub instance deployed on USheffield's HPC system
People involved: @mikecroucher
JupyterHub instance(s) deployed at UVSQ
Main use case: use in classroom (Sage, Python, C, Apache Spark), 
casual use
People involved: @defeo
JupyterHub and Binder instances deployed on EGI 
infrastructure
Easy deployment of live GAP/SageMath/... notebooks with 
mybinder, thanks to the Docker containers (#58); 
potential alternatives: Debian packaging and Conda packaging.
People involved: @nthiery, ...
Local instance of CoCalc (using the Docker container) at the University of Gent
Main use case: teaching for mathematics students
Deployed at jupyter.mathhub.info
With MMT kernel
People involved: @tkw1536
Highlight: Simulagora
Logilab VRE deployment for application development and deployment.
Can use JupyterLab for application development, which can then be deployed 
with a simplified parameters form input.

KPI: Usage/impact statistics (since last reporting period)

Nbdime: 855 stars on github, 64 contributors (36 in 12 months prior), 611 
comments, 239 new issues (241 closed).
Thebelab: 44 stars, 15 contributors, 151 new issues (118 closed), 323 
comments.
K3D-Jupyter: 48 stars, 14 contributors, 140 new issues (129 closed), 303 
comments.


\subsubsection{KPIs for Aim 4 / Dissemination / Impact}

\begin{recommendation}{Aim 4}
  Maximise sustainability and impact in mathematics, neighbouring
  fields, and scientific computing.
\end{recommendation}

Qualitative metrics: success stories. For example:
\begin{itemize}
\item Women in \Sage workshops;
\item Use of \Jupyter, \cocalc, ... for teaching, at Sheffield, at the
  African Institute for Mathematical Sciences, ...;
\item Stories about the impact of the Micromagnetics VRE;
\item Impact of nbdime, 3D widgets, Thebelab, ...
% D4.13 (Sphinx) might have impact beyond ODK. For example, Simon King
%is interested in things done for D4.13 for his
%\software{p\_group\_cohomology} package. It might also lead to a PEP,
%which by itself counts as impact.
\end{itemize}

Quantitative metrics:
\begin{itemize}
\item Statistics on workshops organized and conference presentations
  delivered as part of our dissemination activities, including
  estimates of number of attendees and what (if anything) happened as
  a follow-up.
\item Number of courses and departments we worked with directly and an
  estimate of how many students this subsequently affected.

  For example:
  \begin{itemize}
  \item Department of physics, University of Sheffield. Estimated
    number of new users: 500 undergraduate students
  \item Department of Biomedical Sciences, University of Sheffield.
    BMS353. 2015: N students. 2016: Y students
  \end{itemize}
\end{itemize}

We have been systematically collecting these metrics since the
beginning of \ODK, as part of our dissemination reports
\delivref{dissem}{workshops-1}, \delivref{dissem}{workshops-2},
\delivref{dissem}{workshops-3}, \delivref{dissem}{workshops-4}.


https://github.com/OpenDreamKit/OpenDreamKit.github.io/blob/master/meetings/2018-10-28-Luxembourg/ProjectReview/WP2.md
