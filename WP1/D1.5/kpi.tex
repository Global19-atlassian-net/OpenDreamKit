
\section{KPIs}

\begin{introduction}The following Key Performance Indicators (KPI) show how OpenDreamKit addresses the specific impacts listed in the work 
programme. KPIs were thought through by the members of OpenDreamKit so that they are meaningful, reusable, realistic and easily measurable. 
the following qualitative and quantitative indicators are divided into the four aims of OpenDreamKit. If quantitative indicators are more 
useful for reporting and internal evaluation, qualitative indicators will give content for futher dissemination and communication purposes, 
for example through the project website.
\end{introduction}


\subsubsection{KPIs for Aim 1}

\begin{recommendation}{Aim 1} Improve the productivity of researchers in pure mathematics and applications by promoting collaborations based 
on mathematical software, data, and knowledge.\end{recommendation}


\begin{enumerate}
\item Quantitative metrics: Web site Statistics
\begin{itemize}
%7 activities on the website 
\item 31 blogs (28 blogs and 3 technical blogposts)
\item 471 followers
\item 6 press releases
\item 6500 visits in 2018
\item 6 video interviews and explainer material
%Statistiques sur le site web Open dream kit:( à revoir)
\end{itemize}

\item Qualitative metrics:
\begin{description}
\item [Story of Michael Croucher:] Michael is a Research Software Engineer at the university of Sheffield, passionate about improving the 
quality of research software. He enables researchers to ask larger and more complex research questions by improving the software they 
develop. Along with the Software Sustainability Institute, the UK Research Software Engineering Association and the EU-funded OpenDreamKit 
project, Michael Croucher actively campaign to improve the career prospects of the talented people who underpin a huge amount of 
computational research…. He explained in a blogpost on october 2018 " How OpenDreamKit supported the RSE revolution" saying that "By 2015, 
there were a small number of central ‘Research Software Engineering Groups’ within UK Universities with his group at Sheffield  being among 
the first. OpenDreamKit was one of the first projects they won that demonstrated that funders would support RSEs on major grants – this 
improved credibility of the new role a great deal and helped secure its future at Sheffield." 
\end{description}
\end{enumerate}

\subsubsection{KPIs for Aim 2 / adoption of \ODK's technologies}

\begin{recommendation}{Aim 2}Make it easy for teams of researchers of any size to set up custom, collaborative Virtual Research Environments 
tailored to their specific needs, resources and workflows. The VRE should support the entire life-cycle of computational work in 
mathematical research, from initial exploration to publication teaching and outreach;
\end{recommendation}

\begin{enumerate}
\item Qualitative metrics:
\begin{itemize}
\item Success stories about \ODK based VRE deployments, and generally speaking adoption of \ODK's components.
\begin{itemize}
\item June 2018, Jupyter was awarded prestigious 2017 ACM Software System Award
Previous winners include: UNIX, TCP/IP, the Web, TeX, Java, GCC, LLVM
%ttps://blog.jupyter.org/jupyter-receives-the-acm-software-system-award-d433b0dfe3a2

\item Partnership with EGI (main stakeholder of the EOSC): there is an ongoing collaboration between EGI and OpenDreamKit to deploy
JupyterHub and BinderHub-based EGI services. Proofs of concepts for both have been deployed by Enol Fernandez from EGI. Both parties
are very satisfied with the collaboration and want to strengthen it. A Technology Provider Agreement was signed between EGI and
Simula, on behalfF of OpenDreamKit/Jupyter developers. Joint applications to upcoming EOSC calls, and in particular INFRAEOSC-02-2019
%https://github.com/OpenDreamKit/OpenDreamKit.github.io/blob/7cbee70c3bff6d0778733be931c927981df87e65/meetings/2018-10-28-Luxembourg/SteeringCommitteeMeeting/ProgressReports/ParisSud.md
%ttps://github.com/OpenDreamKit/OpenDreamKit/blob/88ee2544e31e661c4eac7f39fa7333f24b62ab2a/WP4/D4.8/report.tex
  
\item Use of Jupyter components by Logilab
Simulagora
Logilab VRE deployment for application development and deployment
Can use JupyterLab for application development, which can then be deployed with a simplified parameters form input. % à détailler

%Use of ODK's computational components in collaborative workspaces: A group of mathematical researchers with access to common computational resources, such as a shared lab computer or cloud servers,has already been able to deploy a prototype VRE with JUPYTERHUB, integrating OpenDreamKit components. The Jupyter kernels for mathematical software developed as part of OpenDreamKit make computational mathematical components accessible in a JUPYTER environment, enabling a Jupyter-based deployment of the relevant tools for the researchers. The process of working on notebooks is greatly improved by review tools developed as part of WP4, enabling researchers to collaborate to some degree in a shared computational environment.


\item
%Michael Kohlhase PRESENTATION during Second OpenDreamKit Review, on October 30, 2018 in Luxembourg
%TODO: more narrative text? more context?
%https://opendreamkit.org/meetings/2018-10-28-Luxembourg/ProjectReview/WP6.pdf
\item Multi-Site involvement of Researchers (Mobility of Researchers)
PD. Dr. Florian Rabe (Joint appointment UPSud/FAU)
Felix Schmoll Summer Internship (From JacU to St.Andrews)
Prof. Nathan Carter (Bentley Univ.) in St. Andrews (Sabbatical)
\item Heavy interest by the theorem proving community about MitM Ontology
Logipedia (\url {http://logipedia.science}) adopts the MitM principle of integrating (logical) systems by aligning concepts.
First ODK-external MitM “user” for the next months: Andrea Thevis, Saarbrücken 
\end{itemize}

\item Blogs about how to deploy VRE:
\begin{itemize}
\item Luca De Feo: \href{https://opendreamkit.org/2018/10/17/jupyterhub-docker/}{Deploying a containerized JupyterHub server with Docker} 
\item Nicolas Thiéry: \href{https://opendreamkit.org/2018/03/15/jupyterhub-binder-convergence/}{Toward versatile JupyterHub deployments, with the Binder and JupyterHub convergence}
\item Loic gouarin: \href{https://blog.jupyter.org/how-to-deploy-jupyterhub-with-kubernetes-on-openstack-f8f6120d4b1}{Deploying JupyterHub with Kubernetes on OpenStack} 
  \end{itemize}
    \end{itemize}
    
\item Quantitative metrics:
\begin{itemize}
\item[List of known \ODK based VRE deployments, as tracked on \url{https://github.com/OpenDreamKit/OpenDreamKit/issues/174}:] We have made 
collaborations with various institutions and projects to deploy instances of JupyterHub and CoCalc (formerly SageMathCloud),Collecting 
estimates of the difficulties involved in such deployments in various use cases will help us to plan future 
deployments and seek what could be done to ease deployment.

\begin{itemize} % nombre seulement et rajouter lien sur github, pour la recherche et l'enseignement, plus separer en unievrsité et mathrice, sites nationaux
\item 7 Jupyter deployments:%https://github.com/OpenDreamKit/OpenDreamKit/issues/174
\begin{itemize}
\item Local CoCalc instance at Universität Zürich. Deployed September 2015 - February 2016.

\item [Instance of JupyterHub](https://jupyter.math.cnrs.fr/hub/) deployed by the [Mathrice group](http://mathrice.fr/)
Host Infrastructure: France Grille's LAL cloud
Users: members of math labs in France
Main use case: casual use 

\item [JupyterHub instance at Université Paris Sud / Paris Saclay](http://jupytercloud.lal.in2p3.fr/)
Host Infrastructure: France Grille's LAL cloud
Users: personnel and students of UPSud / Paris Saclay
Main use case: use in classroom (Python, Sage, C++), casual use
 
\item JupyterHub instance deployed on USheffield's HPC system ttp://docs.iceberg.shef.ac.uk/en/latest/using-iceberg/accessing/jupyterhub.html
 
\item JupyterHub instance(s) deployed at UVSQ %https://opendreamkit.org/2018/10/17/jupyterhub-docker/
Main use case: use in classroom (Sage, Python, C, Apache Spark), casual use
     
\item [Gallery of JupyterHub instances](https://jupyterhub.readthedocs.io/en/latest/gallery-jhub-deployments.html)
JupyterHub and Binder instances deployed on EGI infrastructure; see #205.
Easy deployment of live GAP/SageMath/... notebooks with [mybinder](mybinder.org),thanks to the Docker containers (#58);potential 
alternatives: [Debian packaging](https://wiki.debian.org/DebianScience/Sage) and [Conda packaging](https://wiki.sagemath.org/Conda).
 
\item Local instance of CoCalc (using the Docker container) at the University of Gent
Main use case: teaching for mathematics students
Deployed at jupyter.mathhub.info With MMT kernel 
  \end{itemize}
  
\item 3 interests in Jupyter deployments:
\item JupyterHub deployment at the [Einstein Institute of Mathematics](http://math.huji.ac.il), part of the Hebrew University of Jerusalem
Related: [Sage Days 79](https://wiki.sagemath.org/days79)

\item Local Cocalc instance at UPSud
Host infrastructure: UPSud's cloud.
Users: personnel and students of UPSud
Main use case: use in classroom (Python, Sage, C++), casual use
For now, priority has been given to the above JupyterHub instance

\item Integration of Sage in the tmpnb.org's temporary notebook server
People involved: @rgbkrk, @nthiery
Status: some experiments run during a sprint at Pycon'15. Now that Sage's Jupyter kernel is well integrated in the stable version of Sage, it's just a question of installing Sage in tmpnb's docker container. This is superseded by [binder](http://mybinder.org).
\end{itemize}
\end{itemize}

\item [Number of installs of \ODK's components via platform-specific distribution channels: Debian popcon, Arch statistics, installer
  downloads]

%Vidar Fauske PRESENTATION filling in for Benjamin Ragan-Kelley during Second OpenDreamKit Review, on October 30, 2018 in Luxembourg
%https://opendreamkit.org/meetings/2018-10-28-Luxembourg/ProjectReview/wp4.pdf

%TODO: Could you please read and verify the information written below and maybe add additional info? 
\item At the begining of ODK, there were 49 Jupyter kernels (languages or systems that could be used in Jupyter). There are now 117 of them, 6 of which were contributed or significatively improved by ODK.
\item Millions of Jupyter notebooks are already online with over 3 million on GitHub alone. Thousands of them are using Jupyter kernels (co)developped by ODK (SageMath: 6199, Xeus-cling C++: 684, GAP: 63, Singular: 8, PARI/GP: 3, MMT: 1).

%LUCA DE FEO PRESENTATION during Second OpenDreamKit Review, on October 30, 2018 in Luxembourg
%https://github.com/OpenDreamKit/OpenDreamKit.github.io/blob/af29108dd216972485a17a3444caf6eee569033a/meetings/2018-10-28-Luxembourg/ProjectReview/WP3.ipynb
%TODO: Could you please synthesize the information ( more narrative text, more context) that appears in your presentation:
\item ODK’s computational components available on major platforms” .
User story: users shall be able to easily install ODK’s computational components on the three major platforms (Windows, Mac, Linux)
via their standard distribution channels.
There are Packages available for all components on Debian, Ubuntu, Fedora, Arch, Gentoo, ...Experimental Conda Forge packages  
Arch Based upon voluntary reports from ~30k users 
Debian Based on ~200K voluntary submissions 
Ubuntu, based on ~2.8M voluntary submissions 
Docker hub 
10k pulls for sagemath-jupyter
5.7k pulls for sagemath
3.8K pulls for gap-docker


  \end{itemize}
  \end{itemize}
 \end{enumerate}     
 
\subsubsection{KPIs for Aim 3}

\begin{recommendation}{Aim 3}
  Identify and promote best practices in computational mathematical research including: making results easily reproducible; producing
  reusable and easily accessible software; sharing data in a semantically sound way; exploiting and supporting the growing
  ecosystem of computational tools.
\end{recommendation}

\begin{enumerate}
\item Qualitative metrics: Success stories
\begin{itemize}
\item Best practice and tools for correct and reproducible research
 \begin{itemize}       
\item Mike Croucher's talk ``Is your research software correct''
 The excellent talk \href{https://mikecroucher.github.io/MLPM_talk/}{``Is your research software correct''} by Mike 
 Croucher, highlights crucial best practice whenever software is used in research, including open code and data sharing, 
 automation, use of high level languages, software training, version control, pair programming, literate computing, or testing. A 
 lot of the work in ODK relates to disseminating this set of best practice (\longWPref{dissem}), and enabling it through appropriate 
 technology (\longWPref{UI}).  Just to cite a few examples, \longdelivref{UI}{jupyter-collab}, and \longdelivref{UI}{jupyter-test} 
 enable respectively version control and testing in the \Jupyter literate computing technology, while Mike's talk is and will be 
 delivered in several of ODK's many training events.
         
  \item Success of Tania's Workshop on Jupyter notebooks for reproducible research
  OpenDreamKit member, Tania Allard, ran a hands-on workshop on Jupyter notebooks for reproducible research. This workshop focused on 
  the use of Jupyter notebooks as a means to disseminate reproducible analysis workflows and how this can be leveraged using tools 
  such as nbdime and nbval. Both nbdime and nbval were developed by members of the OpenDreamKit project as a response to the growing 
  popularity of the Jupyter notebooks and the lack of native integration between these technologies and existing version control and 
  validation/testing tools.
  An exceptional win was that this workshop was, in fact, one of the most popular events of the conference and we were asked to run 
  it twice as it was massively oversubscribed. This reflects, on one hand, the popularity of Jupyter notebooks due to the boom of 
  literate programming and its focus on human-readable code. Allowing researchers to share their findings and the code they used 
  along the way in a compelling narrative. On the other hand, it demonstrates the importance of reproducible science and the need for 
  tools that help RSE and researchers to achieve this goal, which aligns perfectly with the goals of OpenDreamKit.
  %ttps://github.com/OpenDreamKit/OpenDreamKit.github.io/blob/6faf6eb2f1532f342f86c8da633078067ca40c85/_posts/2018-03-07-         opendreamkit-at-the-rse-conference.md

 \item The Math-in-the-Middle (MitM) ontology and the system API theories in the MitM paradigm are big theory graphs with thousands of 
nodes and edges. Understanding and interacting with such large and complex objects is very difficult.The FAU group has conducted
research into whether virtual reality technologies are helpful for this task. We have presented a first working prototype at the
Conference on Intelligent Computer Mathematics CICM 2018 and the author: Richard Marcus - a master's student at FAU has received a
prize for best presentation.https://github.com/OpenDreamKit/OpenDreamKit.github.io/blob/1f175efc40146570fafa28123bed6b1198c41b00
/_posts/2018-08-20-tgview3d.md  
\end{itemize}
 
  
%Some research paper that showcases a range of best practices supported by ODK work (paper written collaboratively on e.g. github, software distributed as e.g. SageMath package, live demo and logbooks on binder, nbdime for collaboration, ...).

\item A serie of Use cases: examples of work that have been made possible through the OpenDreamKit project.The general structure is to 
describe a work requirements what is required followed by a solution using OpenDreamKit supported tools. Where appropriate, we provide links 
to related examples, and provide more details.

\begin{itemize}     
4 use cases: %see:https://opendreamkit.org/tag/use-case
\item Nicolas M. Thiéry: Publishing reproducible logbooks 
\item Nicolas M. Thiéry: Live online slides with SageMath, Jupyter notebooks, RISE and Binder
\item Michael Kohlhase: WP6 Math-in-the-Middle Integration Use Case to be Published at MACIS-2017 (two papers) 
\item Michael Kohlhase: Mixing Data and Computation to explore mathematical data sets: Knowledge to the rescue with LMFDB + SageMath + Pari + MitM 
   %https://opendreamkit.org/2017/11/02/use-case-publishing-reproducible-notebooks/
   %https://opendreamkit.org/2018/03/15/jupyterhub-binder-convergence/
\end{itemize}        
\end{itemize} 
\item Qantitative metrics:
    \begin{itemize}
    \item Number of PyPI hosted packages for \Sage, and similarly for other components.
     https://github.com/OpenDreamKit/OpenDreamKit/blob/88ee2544e31e661c4eac7f39fa7333f24b62ab2a/WP3/D3.3/report.tex
  
%LUCA DE FEO PRESENTATION during Second OpenDreamKit Review, on October 30, 2018 in Luxembourg
%https://github.com/OpenDreamKit/OpenDreamKit.github.io/blob/af29108dd216972485a17a3444caf6eee569033a/meetings/2018-10-28-Luxembourg/ProjectReview/WP3.ipynb
%TODO more narrative text, more context: Could you please synthesize the information that appears in your presentation:
  Conda packages
      From Conda Forge:
      Jupyter 380K downloads: https://anaconda.org/conda-forge/jupyter
      Sage 6K downloads: https://anaconda.org/conda-forge/sage

      Sage on PyPI
      80 packages in PyPI, not all Sage-related.
      Make them easily discoverable, document workflows.


    GAP packages code coverage: 69% (4.9) → 75% (4.10).
    Freshness of GAP packages: 50% released in the last year.
    SageMath on Windows: 44% (happy) Windows users.
    Packaging (not counting alt. methods, such as Conda):
    Arch: 50% of Jupyter users are also ODK users;
    Debian: 10% of Jupyter users are also ODK users.
    Medium sized VRE deployments: 20h of work, 244 LOCs.
    Docker Hub: 4K-10K pulls of ODK images.
    

    \item Number of additional systems made interoperable with the Math-in-the-Middle architecture, on top of the three for the Month 36 prototype. % Michael: I am expecting about 2 (OEIS and Findstat) any more than that would be a success.conv: {dissem: PU, due: '24', issue: '138', label: D6.4, lead: ZH, nature: DEC,


%Michael Kohlhase PRESENTATION during Second OpenDreamKit Review, on October 30, 2018 in Luxembourg
%https://opendreamkit.org/meetings/2018-10-28-Luxembourg/ProjectReview/WP6.pdf
%TODO: Could you please add more narrative text? more context? the info below appears in your presentation.

MitM-connected Systems: four (GAP, Sage, LMFDB, Singular)(See D6.5)
Formal MitM Ontology: 55 files, 2600 LoF, 360 commits (See D6.8)
Informal MitM Ontology: 815 theories, 1700 concepts in English, German,(Romanian, Chinese)
MitM System API Theories (GAP, Sage, LMFDB, Singular): 1.000+ Theories, 22.000 Concepts.


%https://github.com/OpenDreamKit/OpenDreamKit/issues/264 First Math-In-The-Middle-based interoperability prototype Thanks to a fully functional prototype integrating of at least the systems GAP, SAGE, SINGULAR, and LMFDB via the SCSCP Protocol, and users shall be able to run calculations involving any combination of those systems from any of them. This prototype will be the basis for integration work for additional systems and the user interface from WP4.(#264) TODO: regarder s'il a été ajouté dans la justification du Milestone correspondant

 \item Some metrics on the scale of the Math-in-the-Middle architecture; e.g. number of API CDs generated and number of alignements
 the current state of play is that we have initial exports of system interface ontologies for three systems:
 \begin{itemize}
 \item SageMath 512 CDs with 2800 entries
 \item GAP 218 CDs with 2996 entries.
  \end{itemize}

%In the course of the deliberations in the WP6 workshops we saw a shift from the development of computational foundations and verification towards API/Interface function specifications to enable semantic system interoperability via the Math-in-the-Middle Ontology. Consequently, emphasis has changed to the generation of API Content Dictionaries (API CDs) for GAP, LMFDB and SAGE. We have a prototypical set of GAP and SAGE Content Dictionaries in OMDoc/MMT form (GAP: 218 CDs, 2996 entries; SAGE: 512 CDs, 2800 entries overall). The computational foundations exist but are rather more simple than originally anticipated. Much of the functionality has been offloaded to the SCSCP standard – remote procedure call with OpenMath representations of the mathematical objects – developed in the SCIENCE Project. As a direct consequence of the work in OpenDreamKit the OpenMath Society has promoted the SCSCP protocol into as an OpenMath Standard.Conversely, the GAP and SAGE CDs are rather more elaborated than anticipated in the proposal, and thus form a viable basis for alignment with the MitM Ontology.
% https://github.com/OpenDreamKit/OpenDreamKit/blob/88ee2544e31e661c4eac7f39fa7333f24b62ab2a/terminations/report/wp6.tex
%https://github.com/OpenDreamKit/OpenDreamKit.github.io/blob/333b21524f753dd49e441b4b1157ded68999eeff/meetings/2018-02-01-SteeringCommitteeMeeting/ProgressReports/Zurich.md

\end{itemize}
\end{enumerate}

\subsubsection{KPIs for Aim 4 / Dissemination / Impact}
\begin{recommendation}{Aim 4}
  Maximise sustainability and impact in mathematics, neighbouring fields, and scientific computing.
 \end{recommendation}

An important part of the success of the ODK project is linked to its ability to foster a community in the spirit of the open source projects 
it is built on. Part of this relies on the organization and participation to scientific and development events of many different scales and 
objectives. We gave a first overview of the actions taken to this day. This includes:

 \begin{itemize}
    \item Organization of development workshops
    \item Organization of dissemination workshops in different thematic linked to ODK
    \item Training intervention in external events
    \item Communication interventions and participation to external events
    \item Actions taken to foster a community in developing countries
    \end{itemize}

\begin{enumerate}
\item Qualitative metrics:
\begin{itemize}
\item Women in \Sage workshops

 The under-representation of women in the scientific world is even more visible if we intersect science with software
  development. As we know, we have many talented women in our community, and we have organised some events targeted at women in the
  spirit of the "Women in Sage" days that happened many times in the US already. %e are planning to have one more on 2019......
\begin{itemize}
 \item Women in Sage Events:
    Organised by OpenDreamKit in Paris January 2017
    Another event (independent of ODK) in Montreal in June 2018
    Next ODK one: Spring/Summer 2019 in Crete

 \item Women in computing
    Developed training materials and provided training for over 130 women in the last 12 months at Sheffield and Manchester in partnership 
    with CodeFirstGirls.
    Tania Allard participated in the Diversity and Inclusion in Scientific Computing unconference by direct invite of NumFOCUS
    Tania Allard was diversity chair for the 2017 International Research Software Engineering conference.

 \item Women in sage workshops
    Last January, Viviane Pons, Jessica Striker and Jennifer Balakrishnan organized the first WomenInSage event in Europe with OpenDreamKit.      
    20 women spent a week together coding and learning in a rented house in the Paris area.
    We took advantage of the diverse knowledge background of our group to work together and learn from each other. It was an occasion for 
    many "first times" among participants who had very little experience with Sage:

\begin{itemize}
    \item 5 participants installed a source version of Sage for the first time (so that they could edit the source).
    \item3 used git for the first time.
    \item5 used git within Sage for the first time.
    \item11 got their first Trac account .
    \item5 got their first contribution to a Sage ticket.
    \item8 are in the process of getting their first code integrated to Sage.
\end{itemize}

We worked on 14 tickets during the week, 6 of those which have been merged since the conference. All participants said they had learned new 
things and it would impact their careers. %https://github.com/OpenDreamKit/OpenDreamKit.github.io/blob/6faf6eb2f1532f342f86c8da633078067ca40c85/_posts/2017-04-06-WomenInSage.md
\end{itemize}

\item Diversity in ODK workshops
8 training workshops were organised in developing countries including Algeria, Liban, Tunisia, Colombia, Morocco and Mexico, and attended by about 451 trainees.

\item Adoption of ODK technologies for teaching
\begin{itemize}
\item University of Sheffield
Cocalc was adopted for teaching in UK after ODK's Kickoff: Support was given to a number of lecturers in Sheffield to migrate to Jupyter and 
CoCalc (formerly SageMathCloud) but also to those that had already been using CoCalc and Jupyter notebooks for their courses. These included 
lecturers from Computer Science, Physics, Biomedical Science, Bioinformatics, and Materials Science. (D2.17, T2.6) A previously generated 
CoCalc tutorial was extended by adding tutorial sections for students having courses in CoCalc as well as with a hands-on tutorial for 
lecturers to get started. The material can be found as a website at https://tutorial.cocalc.com/ 

\item PGTC 
Reproducible GAP experiments on Binder: AlexK %lien avec ODK Qui est-ce? 
used GAP Jupyter interface in teaching at PGTC 2018 and "Software tools for mathematics" workshop. Here https://github.com/alex-konovalov
/gap-teaching, you can find a collection of GAP Jupyter notebooks. It uses the Docker container with the latest public release of GAP, which 
is maintained in a separate repository at https://github.com/gap-system/gap-docker.

\item UPSUD
Since 2017, Jupyter is used at Université Paris Sud for teaching C++ to over 300 students.This was initiated in particular by
     \ODK participants Loïc Gouarin, Viviane Pons, and Nicolas M.\ Thiéry. The mix of narrative documents and interactive programming 
     fostered active participation from the students while our web-based deployment made it easier for them to work from home.The course 
     material is available from \url{http://Nicolas.Thiery.name/Enseignement/Info111}.
     
 \item African Institute for Mathematical sciences
 In early spring 2017, Prof.~Dr.~W.~Decker and Prof.~Dr.~G.~Pfister gave a three-week course on computational algebraic geometry 
     at the African Institute for Mathematical sciences (Cape Town, South Africa) with lectures and computer lab sessions.
     The course was attended by about 50 students from all over Africa. In the lab sessions, the students learned how to experiment with the
     computer algebra system Singular. It proved extremely valuable that the students could run Singular in the Jupyter notebook.
     
  \item University of Granada
     
     The \GAP Jupyter kernel was used by Pedro Garcia-Sanchez to teach a master course in mathematical software at the University of
     Granada. See  \url{https://github.com/pedritomelenas/Software-Matematicas-GAP}. Pedro has taken on the technology, and is now involved
     in the   development of interactive visualization widgets for discrete maths (package Francy; see also~\longdelivref{UI}{ipython
     advanced-interacts}) in particular for use in other courses.
     
  \end{itemize}

\item success of the Bioinformatics Awareness Days
An event that took place in The Sheffield Institute for Translational Neuroscience on November 2017 where were given training activities 
about bioinformatics workflows using Jupyter notebooks with computation provided by the free Micosost Azure Notebook service.The event 
demonstrated that \ODK supported technologies could be applied to the field of Bioinformatics and led to a new collaboration between Dr 
Cutillo and \ODK member Mike Croucher.

Following the success of this workshop, Dr Cutillo independently taught an introductory workshop on statistics using Jupyter notebooks 
on Azure at Parthenope University of Naples (Materials at \url{https://github.com/luisacutillo78/RbasicStats}) r Cutillo has since moved 
to University of Leeds where she will be teaching statistics to 200+ undergraduates. She plans to use \ODK developed technologies 
in collaboration with the Research Software Engineering group at Leeds.

The event required also the development of a website that was linked to the Jupyter notebooks (\url{https://bitsandchips.me/BAD_days/}). The 
website caught the attention of Eleni Vasilaki, Head of Machine Learning at University of Sheffield who wanted to do something similar for 
her course on Adaptive Intelligence. We supported her in this endeavour and the result is at (\url{http://bitsandchips.me
/COM3240_Adaptive_Intelligence/}).

In order to better support this, \ODK member Tania Allard, developed a Jekyll template for use by academics and researchers using Jupyter 
notebooks for course materials and dissemination. This led to the development of a Python package: nbjekyll (\url{https://github.com
/trallard/nbjekyll}) that complements the Jekyll template.As well as being used internally at Sheffield, The nbjekyll package received some 
attention on twitter \url{https://twitter.com/jdblischak/status/1009800776305332224} and \url{https://twitter.com/walkingrandomly/status
/1009414151716909057} receiving a total of 42 retweets and 80 'likes'% ( à remettre à jour).
  
\item Stories about the impact of the Micromagnetics VRE;
\item Impact of nbdime, 3D widgets, Thebelab, ...
% D4.13 (Sphinx) might have impact beyond ODK. For example, Simon King
%is interested in things done for D4.13 for his
%\software{p\_group\_cohomology} package. It might also lead to a PEP,
%which by itself counts as impact.

%Vidar Fauske PRESENTATION filling in for Benjamin Ragan-Kelley during Second OpenDreamKit Review, on October 30, 2018 in Luxembourg
%https://opendreamkit.org/meetings/2018-10-28-Luxembourg/ProjectReview/wp4.pdf
%TODO: Could you please add more narrative text or more context? the information below appears in your presentation but listed that way this is too technical.

Usage/impact statistics: Further development on the nbdime project, delivered in the first period (D4.6). Has been met with enthusiasm, adoption in the community. Jupyter Notebook and Jupyter Lab extensions, git integration. 
Nbdime: 855 stars on github, 64 contributors (36 in 12 months prior), 611 comments, 239 new issues (241 closed).
Thebelab: 44 stars, 15 contributors, 151 new issues (118 closed), 323 comments.
K3D-Jupyter: 48 stars, 14 contributors, 140 new issues (129 closed), 303 comments.
\end{itemize}

\item Quantitative metrics:
\begin{itemize}
\item Statistics on workshops organized and conference presentations delivered as part of our dissemination activities, including estimates of number of attendees and what (if anything) happened as a follow-up.

\begin{itemize} 
item/Over the last four years OpenDreamKit has been involved in 66 events, 31 organized/co-organized by UPSud.
\begin{itemize} 
\item 21 development workshops and project meetings, among these: 14 organized or co-organized by UPSud
\item 26 training workshops or sessions,  7 org./co-org. by UPSud adding 1100 trainees
\item 13 community building workshops, 8 org./co-org. by UPSud
\item 6 research workshops, 2 org./co-org. by UPSud
\end{itemize} 
\item And 19 external events ( 9 organized/co-organized by UPSud).
\end{itemize}

\item Number of courses and departments we worked with directly and an estimate of how many students this subsequently affected.

\begin{itemize}
  \item University of Ghent
  \begin{itemize}
  \item Department of Mathematics (Bachelor, Master). Courses: CoCalc, Jupyter, SageMath. Estimated number of new users: 31 students
  \item Department of Chemistry (Master). Courses: SageMath. Estimated number of new users: 37 students
   \end{itemize}
   
  \item Paris Sud University 
  \begin{itemize}
   \item Department of Algèbre et Calcul Formel. Courses: Jupyter, SageMath. Estimated number of new users: 25 students yearly.
   \item Department of Introduction à l'Informatique. Courses: Jupyter, C++. Estimated number of new users: 400 students yearly.
   \item Department of Algorithmique, ingénieurs. Courses: Jupyter. Estimated number of new users: 20 students yearly.
   \item Department of Algorithmique M1. Courses: Jupyter. Estimated number of new users: 20 students yearly.
   \item Department of Combinatorics M2. Courses: Jupyter. Estimated number of new users: 20 students yearly.
   \item Department of Projet Math-info L1. Courses: SageMath, Jupyter. Estimated number of new users: 40 students yearly.
     \end{itemize}
     
  \item FAU Erlangen-Nürnberg
    \begin{itemize}
   \item Department of Logic-based Natural Language Semantics. Courses: MMT is the logic, domain modeling and reasoning system. Estimated 
 number of new users: 5-10 yearly. 
   \item Department of Knowledge Representation for Mathematics and Technology. Courses: MMT is the logic, domain modeling and reasoning 
  system. Estimated number of new users: 5-10 yearly. 
 \end{itemize}
 
 \item  UVSQ
  \begin{itemize}
  \item Department of Algèbre Commutative et Effective (M). Courses: Jupyterhub, SageMath, Planetaryum. Estimated number of new users: 22 
  students.
  \item Department of Algorithmique et Programmation C (M). Courses: JupyterHub, JupyterLab, C. Estimated number of new users: 22 students.
  \item Department of Introduction au calcul formel (M). Courses: Jupyterhub, SageMath. Estimated number of new users: 18 students.
  \item Department of Algorithmique (M). Courses: Jupyterhub, JupyterLab, Python. Estimated number of users: 18 students.
  \item Some database stuff (M). Courses: JupyterHub, Apach Spark. Estimated number of users: 60 students. 
 \end{itemize}
 
\item  Sheffield
  \begin{itemize}
 \item Department of physics. Courses: .................. . Estimated number of new users: 500 students.
 \item Department of Biomedical Sciences. Courses: .............................. . Estimated number of new users: .... students.
%#277) https://github.com/OpenDreamKit/OpenDreamKit/issues/277
\end{itemize}

\end{itemize}
\end{itemize}
\end{itemize}
\end{enumerate}
        





















