\subsubsection{Work Package 1: Project Management}

%Explain, task per task, the work carried out in WP during the reporting period giving details of the work carried out by each beneficiary involved.

%%%%%%%%%%%%%%%%%%%%%%%%%%%%%%%%%%%%%%%%%%%%%%%%%%%%%%%%%%%%%%%%%%%%%%%%%%%%%%
\paragraph{Overview}

The general objectives of Work Package 1 are:

\begin{enumerate}
\item{Meeting the objectives of the project within the agreed budget and timeframe and carrying out control of the milestones and deliverables}
\item{Ensure all the risks jeopardising the success of the projects are managed and that the final results are of good quality}
\item{Ensuring the innovation process within the project is fully aligned with the objectives set up in the Grant agreement}
\end{enumerate}

\WPref{management} has been divided into three tasks. In the
following, we highlight the key achievement in this work package, when
necessary report specifically on these individual tasks.


\begin{itemize}
\item \site{PS} lead the 2nd, 3rd and 4th amendment to the consortium,
  in particular to cater for the moving to other institutions of
  permanent and/or non-permanent researchers who are key personnel for
  the success of \ODK. A collateral effect is the addition of the
  sites \site{Leeds}, \site{FAU}, \site{XFEL}, and termination of the
  sites \site{USH}, \site{USO}, \site{JU}, \site{ZH} since no relevant
  staff for \ODK remained at these institution.
\item Following the recommendations from the reviewers after the first
  reporting period, \site{PS} also lead the work plan revision process
  that were implemented in the 3rd amendment.
\item \site{PS} organized the first project review in Brussels with
  the Funding Agency, and regular project meetings, including three
  steering committee meetings in Brussels (March 2017), online
  (February 2018), and at \site{XFEL} (June 2018).
\item \site{PS} made sure that all the milestones and deliverables of
  the Second Reporting Period were achieved and reported on within its
  timeframe.
\item Concerning the communication, \site{PS} has been maintaining the
  internal communication tools that were described in
  \longdelivref{management}{infrastructure}. As for external
  communication the website for the project has been continuously
  updated with new content, and virtually all work in progress is
  openly accessible on the Internet to external experts and
  contributors (for example through open source software repositories
  on Github). A new version of the website was released in June 2018.
  Its end-user friendly interface and content makes it a tool not only
  for internal communication but very much for dissemination and
  progress tracking by the reviewers and the community.
\item Concerning the future of \ODK and of its infrastructure toolkit,
  the coordinator took action to access information and get involved
  in the development of the European Open Science Cloud that is
  currently promoted by the European Commission. We took advantage of
  the EOSC stakeholder forum on 28-29 November and the 2017 edition of
  the DI4R (Digital Infrastructures for Research) in Brussels. During
  these events we gathered information on the potential of EOSC and
  how \ODK could fit in there. Furthermore we initiated a partnership
  with EGI -- a key participant to EOSC -- to deploy \ODK based
  infrastructure.
\item The continued interaction with an Advisory Board and a Quality
  Review Board to control the quality and the relevance of the
  software development relative to the end-user needs. See below.
\item \site{PS} produced a second version of the Data Management Plan
  (\delivref{management}{data-plan2}).
\item Finally \site{PS} kept an eye on recruitment: the strategies we
  used (tailoring of the positions according to the known pool of
  potential candidates, in particular among previous related projects,
  strong advertisement, ...) seem to have paid off, and we are really
  happy with the top notch quality of our recruits. However, despite
  many steps to foster women applications to apply (e.g. through
  reaching personally toward potential candidates or including women
  in the committees), we had almost no female candidate, and none made
  it to the short list. This is alas unsurprising in the very tight
  segment of experienced research software engineers for mathematics
  on temporary positions which is highly gender imbalanced; this is
  nevertheless a failure.
\end{itemize}


%%%%%%%%%%%%%%%%%%%%%%%%%%%%%%%%%%%%%%%%%%%%%%%%%%%%%%%%%%%%%%%%%%%%%%%%%%%%%%
\paragraph{Tasks}

\subparagraph{\longtaskref{management}{project-finance-management}}

As planned in WP1, \site{PS} has been coordinating \ODK and took care
of the budget management together with the administration body, the
D.A.R.I. (Direction des Activités de Recherche et de l'Innovation) and
its finance service. This included prefinancing and funds transfer to
cater for the moving of personnel across sites and from old sites to
new sites.


\subparagraph{\longtaskref{management}{project-quality-management}}


\begin{enumerate}
\item Continued success in the recruitment of highly qualified staff.
\end{enumerate}

The Quality Assurance Plan is described in detail in
\longdelivref{management}{ipr}. We will describe the main points
below. \site{PS} had launched during RP1 a Quality Review Board which
is chaired by Hans Fangohr. The four members of the board have a track
record of caring about the quality of software in computational
science. This board is responsible for ensuring key deliverables do
reach their original goal and that best practice is followed in the
writing process as well as in the innovation production process. The
board meets after the end of each Reporting Period (RP), and before
the Review following that RP. More details about the Quality Review
Board are given in Section~\ref{section.QAP}, including a summary of
the recommendations it emitted after the first reporting period.

% More information on
% \longtaskref{management}{project-quality-management} can be found in
% Section 4 of this document: Quality assurance plan.


The other structure supporting \ODK to ensure the quality of the
infrastructure it produce is the Advisory Board. It is composed of
seven members, some of which specifically represent End Users:

\begin{itemize}
\item{Lorena Barba from the George Washington University}
\item{Jacques Carette from the McMaster University}
\item{Istvan Csabai from the Eötvös University Budapest}
\item{Françoise Genova from the Observatoire de Strasbourg}
\item{Konrad Hinsen from the Centre de Biophysique Moléculaire}
\item{William Stein, CEO of SageMath Inc.}
\item{Paul Zimmermann from the INRIA}
\end{itemize}

This Advisory Board is composed of Academics and/or software
developers from different backgrounds, countries and communities. It
is a strong asset to understand the needs of a variety of end-user
profiles. After the first reporting period, \site{PS} led the review
by the Advisory Board of our Technical Report and later Work Plan
Revision Proposal. This feedback, complementing that of the reviewers,
was very helpful to orient the final work plan revisions.

\site{PS} has also been managing risks. In \delivref{management}{ipr}
all potential risks had been assessed by the Coordinator at Month 12.
Here is a brief update on Risk 1 concerning the recruitment of highly
qualified staff. This risk has been globally well managed thanks to a
flexible workplan enabling adjustments in the timing of some tasks or
deliverables, and thanks to legal actions taken by the Coordinator to
allow key personnel, permanent or not, to remain in the Consortium
even though their positions changed. The addition of the three
partners is representative of these actions. The assessment for the
other risks remain valid at Month 36, and we refer to
\delivref{management}{ipr} for details.

\subparagraph{\longtaskref{management}{project-innovation-management}}

\longdelivref{management}{imp1} was produced at month 18, with a focus on:

\begin{itemize}
\item{The open source aspect of the innovation produced within \ODK}
\item{The various implementation processes the project is dealing with}
\item{The strategy to match end-users needs with the promoted VRE}.
\end{itemize}

The second version of the Innovation Management Plan in RP3 will add content
to explain all the innovations that the VRE is bringing to end-users.
However the open source approach and the "by users for users"
development process will not change.

One of the assessed risks for
\ODK is to have different groups not forming effective teams. Put in
other words, having developers of the different pieces of software
working solely for the benefit of the programme they were initially
working on and for. This risk was continuously tackled by the
Coordinator in order to reach the final goals of the VRE which are the
unification of open source tools with overlapping functionality, the
simplification of the tools for end-users without coding expertise,
and the development of user-friendly interfaces. For this, the
Scientific Coordinator is for example wilfully pushing for joint
actions and workshops.

The evolution of the social organization of software development
communities with a long history is by nature a slow process;
nevertheless, ever since the beginning of the project, collaborative
efforts across communities are progressively becoming a standard
practice, and not only for \ODK participants. More information on
joint workshops can be found in the next section.

%%% Local Variables:
%%% mode: latex
%%% TeX-master: "report"
%%% End:

%  LocalWords:  subsubsection longtaskref delivref organized Bougeret ipr Csabai
