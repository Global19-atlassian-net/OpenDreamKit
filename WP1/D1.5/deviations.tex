\section{Deviations from Annex 1 (if applicable)}
  % Explain the reasons for deviations from the DoA, the consequences and the proposed
  % corrective actions

There was no major deviation from Annex 1. All deliverables due for M18 were delivered
  within the timeframe of the 1st Reporting Period, and all milestones in this period were
  reached.  Slight modifications were brought to \WPref{hpc} and \WPref{dksbases} and were
  included in the AMD-676541-13.

\subsection{Tasks}
% Include explanations for tasks not fully implemented, critical objectives not fully
% achieved and/or not being on schedule.  Explain also the impact on other tasks on the
% available resources and the planning

No deviation from the tasks. All workplan is on time at the end of the Reporting
Period.

However, there were four deliverables that
were handed in late. We will explain the situation and implications in
each case. Therefore, administratively speaking, Milestone 2
(Implementations), originally due on Month 24, was only reached in
Month 36, though with little, if any, consequences on the project as a
whole.

\subsubsection{\protect\delivtref{dissem}{ils-tool}}
This deliverable, a tool for publishing and organizing curated
collections of Jupyter notebooks, was delivered in month 36, a delay
of 12 months after the initially planned schedule.

The initial plan had been to build upon an already existing prototype,
whose development had started at a \Sage meeting back in 2015. The
original tool was specific to \Sage, but broader in scope and not tied
to Jupyter. Given the constantly evolving context, it quickly became
apparent that this tool did not properly address the community needs.
We thus took on evaluating other available open source solutions,
which considerably delayed the deliverable.

As we were not able to find an appropriate solution to build upon, we
finally decided to bootstrap a new project, called
\emph{planetaryum}. Once the development of \emph{planetaryum}
started, we were able to complete the version 0.1 in the planned
time frame.

Since other solutions were available before \emph{planetaryum}, albeit
less powerful, the delay in the deliverable did not impact other tasks
in the project.


\subsubsection{\protect\delivtref{UI}{ipython-kernels}}

This deliverable of full-featured kernels for GAP, Pari, Singular, etc.  has been
delivered in month 36 after a delay of 12 months.  The initial plan was for delivery in
Month 24, but was delayed to ensure a high quality of the delivered software, as more
time-sensitive resources were directed away from this task during months 12-24.  The delay
had no impact beyond the deliverable itself, as no other tasks relied strongly on this
deliverable being ready, and the result is a much stronger collection of Jupyter kernels
for mathematical software.

\subsubsection{\protect\delivtref{UI}{sage-sphinx}}

This deliverable is delivered in month 36, a delay of 12 months after the initial schedule
of month 24.  Progress was slower than planned, due to the nature of coordinating with
large software collaborations.  Additionally, work was shifted to other tasks during early
stages, resulting in the delay of this deliverable.  There have been no negative
consequences of the delay, as its delivery was not a prerequisite for other tasks.  As a
result of the delay, we have delivered much greater work than initially planned, including
significant improvements to the Sphinx documentation system itself used by projects all
over the world, and an Enhancement Proposal to improve the Python language itself,
ensuring wide impact for this work.


\subsubsection{\protect\delivtref{dksbases}{psfoundation}}
This deliverable report was delayed, as we found it useful to extend the scope of the work
reported from just the format of the interface theories and aligments -- these are
extensively discussed in the report as well -- to a full account of the Math-in-the-Middle
interoperability paradigm for \pn and discuss two full-scale use cases. It just made more
sense to deliver this report together with D6.8 (the resources for the use cases) for
Milestone M9 \emph{First Math-In-The-Middle-based interoperability prototype} in month 36,
in particular, since the delay of D6.5 did not delay the research and development in WP6
(after all, an earlier version of much of the content of D6.5 has been pre-published
as~\cite{WieKohRab:vtuimkb17,KohMuePfe:kbimss17} near the original deadline of D6.5 and
was therefore available to the \pn partners). 

This way D6.5 can serve as as reference for opening the MitM paradigm to outside users. We
are currently working on a high-visibility Journal publication based on D6.5 and D6.8
(presumably Journal of Symbolic Computation). 

\subsection{Use of resources}
% Include explanations on deviations of the use of resources between actual and planned
% use of resources in Annex 1, especially related to person-months per work package.

All changes of use of resources were included in the two amendments previously cited and were
due to modifications in the personnel. Those adjustments were due to the change of positions
of some key \ODK participants and expected difficulties in hiring planned
staff. The work plan has been updated accordingly, with no foreseeable
impact on the achievement of tasks, deliverables, and milestones.

Another minor deviation in the proposed use of resource was that FAU hired students to do
some routine jobs (simple formalizations, and the creation of alignments in WP6 and the
creation of example documents in WP4) that did not require the attention of a mature
researchers. As the pay grade of student assistant is roughly 1/4 of that of full
researchers, this action was cost-effective. An unplanned effect was that the reported
person months went up considerably, exceeding the planned amount, without incurring
additional cost. 

%UOXF claimed effort in WP2 where they are not active.

\subsubsection{Unforeseen subcontracting (if applicable)}

Not applicable.
% Specify in this section: a) the work (the tasks) performed by a subcontractor which
  % may cover only a limited part of the project; b) explanation of the circumstances
  % which caused the need for a subcontract, taking into account the specific
  % characteristics of the project; c) the confirmation that the subcontractor has been
  % selected ensuring the best value for money or, if appropriate, the lowest price and
  % avoiding any conflict of interests

\subsubsection{Unforeseen use of in kind contribution from third party against payment or
  free of charges (if applicable)}

 Not applicable. 
 % Specify in this section: d) the identity of the third party; e) the resources made
  % available by the third party respectively against payment or free of charges f)
  % explanation of the circumstances which caused the need for using these resources for
  % carrying out the work.
