\documentclass{deliverablereport}

\deliverable{management}{ipr2}
\duedate{31/08/2018 (M36)}
\deliverydate{Unknown}

\usepackage[style=alphabetic,backend=bibtex]{biblatex}
\addbibresource{../../lib/kbibs/kwarcpubs.bib}
\addbibresource{../../lib/kbibs/extpubs.bib}
\addbibresource{../../lib/kbibs/kwarccrossrefs.bib}
\addbibresource{../../lib/kbibs/extcrossrefs.bib}
\addbibresource{../../lib/deliverables.bib}
\addbibresource{../../lib/publications.bib}
% temporary fix due to http://tex.stackexchange.com/questions/311426/bibliography-error-use-of-blxbblverbaddi-doesnt-match-its-definition-ve
\makeatletter\def\blx@maxline{77}\makeatother

\makeatletter
%%%%%%%%%%%%%%%%%%%%%%%%%%%%%%%%%%%%%%%%%%%%%%%%%%%%%%%%%%%%%%%%%%%%%%%%%%%%%%
% Styling: adapt amsart's subsubsection macro to put a newline after the title
%%%%%%%%%%%%%%%%%%%%%%%%%%%%%%%%%%%%%%%%%%%%%%%%%%%%%%%%%%%%%%%%%%%%%%%%%%%%%%
\renewcommand\subsubsection{\@startsection{subsubsection}{2}%
  \z@{.5\linespacing\@plus.7\linespacing}{.1\linespacing}%
  {\normalfont\bfseries}}

% Variant of taskref that links to the section on the task in this file
\newcommand\localtaskref[2]{\hyperref[#1@#2]{\csname task@#1@#2@label\endcsname}}
\newcommand\longmilestoneref[1]{\textbf{\csname mile@#1@label\endcsname}:
  ``\csname mile@#1@title\endcsname''
  (month \csname mile@#1@month\endcsname)}
\makeatother

%\usepackage{todonotes}
\author{Nicolas M. Thiéry, Benoît Pilorget, et al.}

\begin{document}
\enlargethispage{4ex}
\maketitle
\githubissuedescription
\tableofcontents\newpage

\section{Progress on the project}

In this section, we give a general overview of the progress of the
project in the years 2017 and 2018. We start by recalling some context of \ODK's
approach that is important to understand and evaluate the
progress. Then we describe the general state, and in more detail
the progress of the work packages.

\subsection{Some context: \ODK's approach}
\ODK's approach to delivering a Virtual Research Environment (VRE) for
mathematics is not to build a monolithic one-size-fits-all VRE, but
rather a toolkit from which it is easy to set up VRE's that are
customised to specific needs by combining the appropriate components
(collaborative workspaces, user interfaces, computational software,
databases, \dots) on top of available physical resources (from
personal laptops to cloud infrastructure). This approach --- chosen by
design --- allows to flexibly put together lean computational
environments and tools for particular research challenges. These tools
provide the required functionality but due to the component based
approach carry no unnecessary bloat that would reduce effectiveness in
terms of installation process, size, computation time, and
reproducibility.

Most of the components preexist as an ecosystem of open source
software, developed by well established communities of developers. For
example, for interactive computing and data analysis, OpenDreamKit
promotes Jupyter, a web-based general purpose flexible notebook
interface\footnote{a notebook is a document that contains live code,
  equations, visualizations and explanatory text} that targets all
areas of science. A number of Virtual Research Environment already
exist, e.g.\ powered by SageMathCloud or JupyterHub.

Hence most of the work in \ODK\ is to foster this ecosystem, improving
the components themselves and their composability. The technical is
distributed over the work packages:
\begin{itemize}
\item Component architecture (WP3):
  \begin{itemize}
  \item ease of deployment: modularity, packaging, portability,
    distribution, for individual components and combinations thereof.
  \item sustainability of the ecosystem: improving the development workflows.
  \end{itemize}
\item User Interfaces (WP4): enable Jupyter as uniform notebook
  interface, and further improve it; foster the collaboration between
  SageMathCloud and JupyterHub; generally speaking investigate
  collaborative, reproducible, and active documents.
\item Performance (WP5): make the most of available hardware
  (multi-core, HPC, cloud), for individual computational components and
  combinations thereof.
\item Data/Knowledge/Software (WP6): enable rich and robust
  interaction between computational components, data bases, knowledge
  bases, users through explicit common semantic spaces, a language to
  express them, and tools to leverage them.
\end{itemize}
These technical work packages are supported by the following two activities:
\begin{itemize}
\item Community building and dissemination (WP2): developer and
  training workshops, conferences, teaching material with focus on
  making the created value accessible to a wide, varied and growing user community.
\item Studies of Social Aspects (WP7): analysis of user needs and
  research into collaborative and open software development in
  mathematics and science.
\end{itemize}

As a result of \ODK's approach, the work programme for \ODK\ consists
of a large array of loosely coupled tasks, each being useful in its
own right, and none being absolutely critical.

The second and third years confirmed that this is a strong feature of \ODK's
approach. Indeed, as analysed in the proposal, this kind of project is
subject to the following risks:
\begin{itemize}
\item Recruitment of qualified personnel;
\item Different groups not forming effective team;
\item Implementing infrastructure that does not match the needs of end-users;
\item Lack of predictability for tasks that are pursued jointly with
  the community;
\item Reliance on external software components.
\end{itemize}
Together with ambitious software challenges, this makes the accurate
prediction of workload and precise timeline of work packages
difficult, especially over a period of four years in a field of
rapidly evolving technologies.

The loose coupling allows some flexibility,
permitting to modify the tasks schedule and human resources allocation,
with little influence on the general aims and objectives.

\subsection{General progress}

Intensive work has now started on almost all fronts of the
project\footnote{status reports delivered at the St Andrews project
  meeting (January 2016) and at the Bremen's project meeting (June
  2016) helped the Coordinator to track the progress}. A few tasks
(and the corresponding deliverables) have been postponed by a couple
months due to recruitment delays. This concerns mostly the
micro-magnetic VRE demonstrator
(\taskref{dissem}{dissemination-of-oommf-nb-virtual-environment},
\taskref{dissem}{dissemination-of-oommf-nb-workshops},
\taskref{dissem}{ibook},
\taskref{component-architecture}{oommf-python-interface},
\taskref{UI}{oommf-py-ipython-attributes},
\taskref{UI}{oommf-tutorial-and-documentation},
\taskref{UI}{oommf-nb-ve},
\taskref{social-aspects}{oommf-nb-evaluation}), where recruitment
at Southampton initially proceeded at expected speed but eventually
experienced delays of several months outside our control due to unusually high demand on the UK Home
Office which had to process work permit paperwork for
the successful candidate (attributed to a high number of immigration applications in the run up to the
Brexit referendum). Some deliverables got
delayed as well by a couple months due to unexpected technical
difficulties or misplanning (e.g. \delivref{hpc}{pythran-typing},
\delivref{hpc}{sage-paral-tree}, \delivref{UI}{pari-python-lib1}). All
these delays have been included in the amendment of the Grant
Agreement, which was necessary to include UGent in the
consortium. On the other hand, we are happy to report below on very
strong recruitment (see Section 3.1), as well as unexpectedly rapid
progress on portability and packaging aspects. Also WP6
(Data/Knowledge/Software) has witnessed a particularly strong and
early uptake, with active involvement of many of the participants and
promising outcomes.

All in all, \ODK\ is running according to its plan, and its first
outcomes are already benefiting the mathematical community and
beyond. September 2016 will see the start of Key Performance
Indicators.  These KPIs, which will be more precisely and
realistically defined then, will give results for the 1st Reporting
Period (RP1) at month 18.  This way we will hopefully be able to see the
evolution of the impact OpenDreamKit has had between the RP1 and RP2,
at month 36.

\subsection{Explanation of work carried out per Objective}
%List the specific  objectives  for  the  project  as  described  in  section  1.1  of  Part B   and describe  the  work  carried  out  during  the  reporting  period  towards  the  achievement  of  each listed objective. Provide clear and measurable details.
For reference, let us recall the aims of \ODK.
\begin{compactenum}[\bf {Aim} 1\rm]
\item \label{aim:collaboration} Improve the productivity of
  researchers in pure mathematics and applications by promoting
  collaborations based on mathematical \textbf{software},
  \textbf{data}, and \textbf{knowledge}.
\item \label{aim:vre} Make it easy for teams of researchers of any
  size to set up custom, collaborative \emph{Virtual Research
    Environments} tailored to their specific needs, resources and
  workflows. The \VREs should support the entire life-cycle of
  computational work in mathematical research, from initial
  exploration to publication, teaching and outreach.
  % and bridge the gaps between
  % code, published results, and educational material.
\item \label{aim:sharing} Identify and promote best practices in
  computational mathematical research including: making results easily
  reproducible; producing reusable and easily accessible
  software; sharing data in a semantically sound way; exploiting and
  supporting the growing ecosystem of computational tools.
\item \label{aim:impact} Maximise sustainability and impact in
  mathematics, neighbouring fields, and scientific computing.
\end{compactenum}

Those aims are backed up in our proposal by nine objectives; we now
highlight our main contributions during this reporting period toward
achieving each of them.

\begin{compactenum}[\bf {Obj} 1\rm]
\item\label{objective:framework} \textbf{Virtual Research Environment Kit:} ``\emph{To develop and standardise an architecture
    allowing combination of mathematical, data and software components with off-the-shelf
    computing infrastructure to produce specialised \VREs for different communities.}''

  \ednote{@defeo, @minrk: update for RP3: work carried out for Objective 1: Virtual Research Environment Kit}
  This objective is by nature multilevel; achievements include:
  \begin{itemize}
  \item Collaborative workspaces: major \JupyterHub developments,
    see~\longlocaltaskref{UI}{notebook-collab};
  \item User interface level: enabling \Jupyter as uniform interface for all computational
    components; see \longlocaltaskref{UI}{ipython-kernels}.
  \item Interfaces between computational or database components:
    \begin{itemize}
    \item \emph{short term}: refactoring of existing ad-hoc interfaces, see \longlocaltaskref{UI}{pari-python};
    \item \emph{long term}: investigation of patterns to share data, ontologies, and semantics uniformly across components, see \longlocaltaskref{component-architecture}{interface-systems}, and Section~\ref{dksbases} about \WPref{dksbases}, where we report on the ``Math-in-the-Middle'' (MitM) paradigm for semantic system integration and non-trivial mathematical use cases. In RP3, we have added interoperability of mathematical data sets to the mix. 
  \end{itemize}
\end{itemize}

\item\label{objectives:core} \textbf{Core Components:}
  ``\emph{To develop open source core components
  for \VREs where existing software is not suitable. These components
  will support a variety of platforms, including standard cloud
  computing and clusters. This primarily addresses Aim~\ref{aim:vre},
  thereby contributing to Aim \ref{aim:collaboration}
  and~\ref{aim:sharing}.}''
  \ednote{@defeo, @minrk: update for RP3: work carried out for Objective 2: Core Components
    Presumably we need not list nbval, nbdiff, planetaryum anymore}
  At this stage, it has been possible to implement most of the required developments within
  existing components or extensions thereof. New software components includes the tools
  nbmerge, nbdiff and nbval (see \delivref{UI}{jupyter-test} and
  \delivref{UI}{jupyter-collab}), and planetaryum (see \delivref{dissem}{ils-tool}). For the
  Math-in-the-Middle paradigm for semantic system interoperability we have developed
  knowledge-based Mediator based on the MMT system.
  In RP3 we have concentrated on mathematical data sets. We have added a data aspect to \textsf{MathHub.info} which allows dataset authors to semantically  describe data sets and then generate data database schemata, management functionality, and user interfaces from that. 

\item \label{objective:community}
  \textbf{Community Building across Disciplines:}
  ``\emph{To bring together research
    communities (e.g. users of \Jupyter, \Sage, \Singular, and \GAP) to
    symbiotically exploit overlaps in tool creation building efforts,
    avoid duplication of effort in different disciplines, and share best
    practice. This supports Aims~\ref{aim:collaboration},
    \ref{aim:sharing} and~\ref{aim:impact}.}''
  \ednote{@defeo,@minrk, @ClementPernet: update for RP3: key outcomes of dev workshops
    Maybe: conda packaging, continuous integration}

  We have organized or co-organized a dozen users or developers
  workshops (see~\longlocaltaskref{dissem}{devel-workshops}) which brought
  together several communities. Some key outcomes include:
  \begin{itemize}
  \item Enabling \Jupyter as uniform interface for all computational components; see \longlocaltaskref{UI}{ipython-kernels}.
  \item Sharing best practices for development, packaging, building containers (see~\longlocaltaskref{component-architecture}{mod-packaging}), and continuous integration (see~\longlocaltaskref{component-architecture}{portability});
  \item A smooth collaboration between \JupyterHub, \SMC, and \Simulagora; see~\longlocaltaskref{component-architecture}{extract-smc}, \longlocaltaskref{component-architecture}{simulagora-dev} and Section~\ref{infrastructures};
  \item Work on interfaces between systems; see \longlocaltaskref{component-architecture}{interface-systems}, \longlocaltaskref{UI}{mathhub}, and \longlocaltaskref{UI}{pari-python};
    % \item Steps toward \longlocaltaskref{UI}{sage-sphinx}
  \item Sharing of best practices and tools for authoring live structured documents (see~\longlocaltaskref{UI}{structdocs});
  \item Sharing of best practices when using VRE's like \cocalc or \Jupyter for research and education;
  \item Collaboration on interactive visualization \longlocaltaskref{UI}{vis3d}, \longlocaltaskref{UI}{cfd-vis}, \longlocaltaskref{UI}{dynamic-inspect}.
  \item Jump-starting a community on semantically described, interoperable mathematical data sets around \url{data.mathhub.info}. A Math Data workshop (8 Days) in Cernay included external mathematicians and will be continued by external partners in 2020. 
  \end{itemize}

  \ednote{maybe: mention dissemination workshops as well?}

\item \label{objective:updates}
  \textbf{Updates to Mathematical Software Components:}
  ``\emph{Update a range of existing open source
  mathematical software systems for seamless deployment and efficient
  execution within the VRE architecture of objective~\ref{objective:framework}.
  This fulfils part of Aim~\ref{aim:vre}.}''

  \ednote{@defeo, @ClementPernet: update for RP3: work carried out for Objective 4: Updates to Mathematical Software Components}

  Achievements include:
  \begin{itemize}
  \item Continuous efforts of development, release and integration within \Sage
    have been put for
    \begin{itemize}
    \item  the linear algebra computational kernels of LinBox,
      fflas-ffpack and Givaro (Deliverable~\longdelivref{hpc}{LinBox-algo})
    \item the PARI library for computational number theory
      (Deliverable~\longdelivref{hpc}{pari-hpc2} still ongoing)
    \item the GAP software for computational group theory
      (Deliverable~\longdelivref{hpc}{GAP-HPC-report})
    \end{itemize}
  \item Packaging efforts: docker containers (delivered and regularly
    updated), Debian and Conda packages (beta); see
    \longlocaltaskref{component-architecture}{mod-packaging}.
  \item Continued efforts on portability of \Sage and its dependencies
    (see \longlocaltaskref{component-architecture}{portability}, in
    particular \delivref{component-architecture}{portability-cygwin}).
  \item Improved continuous integration and development workflow;
    (see~\longlocaltaskref{component-architecture}{workflow}), and
    \longdelivref{component-architecture}{multiplatform-buildbot}.
  \item Integration of all the relevant mathematical software in the
    uniform \Jupyter user interface, in particular for integration in
    the VRE framework (delivered, ongoing); see
    \longlocaltaskref{UI}{ipython-kernels}.
  \item Ongoing work in \WPref{hpc} to better support HPC in the
    individual mathematical software system and combinations thereof;
    see Section~\ref{hpc}.
  \item The \Sage and \GAP systems have been extended by a persistent memoization package, which allows to cache computational results and even share them between any system that implementes the memoization format; see \longdelivref{dksbases}{persistent-memoization} for details. 
  \item Ongoing work on the MMT system which forms the basis of the \WPref{dksbases}: Work on \taskref{dksbases}{isabelle} has led to a tight integration with the Isabelle theorem prover and a complete revamp of the code for indexing theory morphisms (crucial for the MitM-based integration of VRE components). 
  \end{itemize}

\item \label{objective:sustainable}
  \textbf{A Sustainable Ecosystem of Software Components:}
  ``\emph{Ensure that our ecosystem of
  interoperable open source components is \emph{sustainable} by
  promoting collaborative software development and outsourcing
  development to larger communities whenever suitable. This fulfils
  part of Aims~\ref{aim:sharing} and~\ref{aim:impact}.}''

  \ednote{@defeo: update for RP3: work carried out for Objective 5: A Sustainable Ecosystem of Software Components
    Maybe: Python 3? phasing out of the Sage notebook? Jeroen's PEP to ease introspection and doctools?
  }

  Achievements include:
  \begin{itemize}
  \item Continued work on outsourcing the computational system user
    interfaces by migrating to \Jupyter; see \longlocaltaskref{UI}{ipython-kernels};
  \item Refactoring \Sage's documentation build system to contribute many local developments
    upstream (\Sphinx) \longlocaltaskref{UI}{sage-sphinx};
  \item Outsourcing and contributing upstream as \Python bindings the existing \Sage
    bindings for many computational systems; see \longlocaltaskref{UI}{pari-python}.
  \end{itemize}

\begingroup
\color{gray}
\item \label{objective:social}
  \textbf{Engineering Social Interactions in Open Source \VRE:}

  This objective was the social science research side of
  \WPref{social-aspects}. Following the work plan revisions after
  Reporting Period 1, the manpower originally allocated to this
  objective was reallocated to other objectives. There thus was no new
  achievements in Reporting Period 2 and 3.

\endgroup

\item \label{objective:data}
  \textbf{Next Generation Mathematical Databases:}
  ``\emph{Identify and extend ontologies and
  standards to facilitate safe and efficient storage, reuse,
  interoperation and sharing of rich mathematical data whilst taking
  account of provenance and citability. This fulfills parts of
  Aims~\ref{aim:vre} and~\ref{aim:sharing}.}''

  This objective is at the core of \WPref{dksbases}; see Section~\ref{dksbases} for details.
  In the first two reporting periods \WPref{dksbases} has developed the Math-in-the-Middle ontology that acts as the pivot point mediating between system languages in the MitM interoperability framework.
  This work has been reported in deliverables \delivref{dksbases}{psfoundation} and \delivref{dksbases}{lfmverif}.

  In the third reporting period the focus of \WPref{dksbases} has been on instantiating the FAIR principles for mathematics (we call the result \textbf{deep FAIR}) and turning mathematical datasets into deep FAIR VRE components.
  This work has been reported in \delivref{dksbases}{nbad-search}.
  The \dmh system, which implements deep FAIR datasets from scratch meets exactly the objectives stated above -- but the system is still very young and needs to attract a critical mass of datasets and community.
  The LMFDB system which has both has been retrofitted with aspects of deep FAIR in \pn, and is much more interoperable than at the start of \pn.
  In parallel, and somewhat dual (lightweight/ad-hoc persistent data caching for mathematical software systems), is the work on \tasktref{dksbases}{data-memo}. Here we have developed a  data memorization format and corresponding memoization packages for \Python (for \Sage) and \GAP.
  These have the potential to lead (by collecting computation results on the side) to informal data sets, which can be semantified later. 

\item \label{objective:demo}
  \textbf{Collaborative Research Environments that Transcend Domains:}
  ``\emph{Demonstrate the effectiveness of Virtual
    Research Environments built on top of \ODK components for a
    number of real-world use cases that traverse domains. This addresses
    part of Aim~\ref{aim:vre} and through documenting best practices in
    reproducible demonstrator documents Aim~\ref{aim:sharing}.}''

  \ednote{@fangohr: update for RP3: work carried out for Objective 8: Collaborative Research Environments that Transcend Domains
    Ubermag's tasks and deliverable
    Shall we mention here other VRE's to illustrate the breath of applications?
    Maybe the interactive text books which cover computer science, biology, physics?
    }

  Most of the work toward this objective is by nature planned for the last period of the \pn
  project. Nevertheless, work has started e.g.  toward the OOMMF demonstrator; see
  \longlocaltaskref{dissem}{dissemination-of-oommf-nb-virtual-environment}
  \longlocaltaskref{dissem}{dissemination-of-oommf-nb-workshops},
  \longlocaltaskref{component-architecture}{oommf-python-interface}.

%Long term sustainability
\item \label{objective:disseminate}
  \textbf{Training and Dissemination:}
  ``\emph{Promote and disseminate
    \ODK to the scientific community by active communication,
    workshop organisation, and training in the spirit of open-source
    software. This addresses Aim~\ref{aim:impact}.}''

  \ednote{@Izabela: update number of meetings and workshops}
  This objective is at the core of \WPref{dissem}, with in particular
  more than 30 meetings, developer, training, and community building
  workshops organized during the third reporting period. See
  Section~\ref{dissem} and \longdelivref{dissem}{workshops-4} for
  details.
\end{compactenum}

%%% Local Variables:
%%% mode: latex
%%% mode: visual-line
%%% fill-column: 5000
%%% TeX-master: "report"
%%% End:

%  LocalWords:  compactenum textbf aim:vre emph JupyterHub longtaskref notebook-collab Simulagora mathhub visualization cfd-vis fflas-ffpack Givaro LinBox-algo multiplatform-buildbot nbad-search dmh psfoundation lfmverif
%  LocalWords:  extract-smc Jupyter localtaskref ipython-kernels dksbases WPref dksbases
%  LocalWords:  nbmerge nbdiff nbval delivref delivref jupyter-collab dissem organized
%  LocalWords:  cocalc portability-cygwin hpc taskref citability fulfills
%  LocalWords:  dissemination-of-oommf-nb-virtual-environment oommf-python-interface
%  LocalWords:  dissemination-of-oommf-nb-workshops


\subsection{Achievements and ongoing progress in workpackages}

\subsubsection{Work Package 1: Project Management}

%Explain, task per task, the work carried out in WP during the reporting period giving details of the work carried out by each beneficiary involved.

%%%%%%%%%%%%%%%%%%%%%%%%%%%%%%%%%%%%%%%%%%%%%%%%%%%%%%%%%%%%%%%%%%%%%%%%%%%%%%
\paragraph{Overview}

As in the previous reporting periods, \site{PS} coordinated \ODK
in close collaboration with the other beneficiaries to ensure that:
\begin{enumerate}
\item{the objectives of the project were met within the agreed budget
    and the timeframe specified by milestones and deliverables;}
\item{all the risks jeopardising the success of the project are managed and that the final results are of high quality;}
\item{the innovation process within the project is fully aligned with the objectives set up in the Grant agreement.}
\end{enumerate}

%%%%%%%%%%%%%%%%%%%%%%%%%%%%%%%%%%%%%%%%%%%%%%%%%%%%%%%%%%%%%%%%%%%%%%%%%%%%%%
\paragraph{Tasks}

\subparagraph{\longtaskref{management}{project-finance-management}}

\begin{itemize}
\item \site{PS} took care of the budget management together with the
  administration body, the D.A.R.I. (Direction des Activités de
  Recherche et de l'Innovation) and its finance service. This included
  prefinancing, interim payments, funds transfer to cater for the moving of personnel
  across sites and from old sites to new sites, and the coordination
  of financial reports.
\item In earlier reporting periods, \site{PS} led four amendment
  processes to the Grant Agreement, to manage work plan revisions and
  to reallocate staff and all remaining resources from the four
  terminated beneficiaries \site{ZH}, \site{USH}, \site{JU},
  \site{USO} to the added beneficiaries \site{UG}, \site{FAU},
  \site{XFEL}, \site{LEEDS}.

  \noindent
  During Reporting Period 3, \site{PS} led a fifth amendment upon
  the request of \site{FAU} to add a subcontractor to conduct a new
  task \longtaskref{dksbases}{isabelle}. Following the suggestions of
  the Project officer, \site{PS} used the occasion to formalize budget
  transfers between beneficiaries to optimize the use of remaining
  resources to achieve the project aims; this was notably required to
  exploit resources left at LEEDS following the early departure of all
  its personnel.

\item In earlier reporting periods,
  \site{FAU} had organized an interim review in Bremen on
  June 2016, and \site{PS} had organized the first
  project review in Brussels on April 2017, and steering committee
  meetings in Orsay (September 2015), St Andrews (January 2016),
  Edinburgh (January 2017), Brussels (March 2017), online (February
  2018), and at \site{XFEL} (June 2018).

  \noindent
  During Reporting Period 3, \site{PS} organized the second project
  review in Luxembourg on October 2018 and steering committee meetings
  in Luxembourg (October 2018) and Marseille (February 2019). It also
  organized a one week "report writing sprint" in Cernay (August 2019)
  to collectively write the project reports. The final review meeting
  will take place in Luxembourg on October 30, 2019, and will gather
  20 \ODK participants to present the project final results.

\item As in earlier reporting periods, \site{PS} ensured that all the
  milestones and deliverables of Reporting Period 3 were achieved
  within its timeframe, and reported on in a timely manner.

\item As in earlier reporting periods, \site{PS} maintained the
  internal and external communication tools that were described in
  \longdelivref{management}{infrastructure}. The project website was
  continuously updated with new content, and virtually all work in
  progress is openly accessible on the Internet to external experts
  and contributors (for example through open source software
  repositories on Github).
  % A new version of the website was released in June 2018.
  % Its end-user friendly interface and content makes it a tool not only
  % for internal communication but very much for dissemination and
  % progress tracking by the reviewers and the community.

\item Concerning the future of \ODK and of its infrastructure toolkit,
  the consortium kept accessing information and getting involved in
  the development of the European Open Science Cloud that is currently
  promoted by the European Commission. The project manager
  participated to the following events:
  \begin{itemize}
  \item \href{https://www.eosc-hub.eu/events/eosc-hub-week-2019}{EOSC
      Hub week}, April 10-12 of 2019, Prague, Czech Republic.
  \item
    \href{https://www.eosc-hub.eu/events/building-open-science-europe-road-ahead-eosc-community}{Building
      Open Science in Europe: The road ahead for the EOSC community
      and the EU Member States}, June 20th of 2019, Tallinn, Estonia.
  \item ICT Proposers' Day 2019, September 19-20th of 2019, Helsinky, Finland.
  \end{itemize}

  \noindent
  In addition, two spin-off proposals were submitted on January 29th
  to the H2020 European E-Infrastructure call INFRAEOSC-02-2019:
  \begin{itemize}
  \item \href{https://github.com/bossee-project/proposal}{BOSSEE}:
    Building Open Science Services on European E-Infrastructure, with
    a focus on Jupyter and applications;
  \item \href{https://opendreamkit.org/2019/01/29/FAIRmat/}{FAIRMAT}:
    FAIR Mathematical Data for the European Open Science Cloud.
  \end{itemize}
  Both were prepared with the same open strategy as OpenDreamKit, and
  involved new combinations of OpenDreamKit and external
  beneficiaries. None was accepted but the respective consortia are
  determined to resubmit them or variants thereof at the earliest
  opportunities.

  % We took advantage of
  % the EOSC stakeholder forum on 28-29 November and the 2017 edition of
  % the DI4R (Digital Infrastructures for Research) in Brussels. During
  % these events we gathered information on the potential of EOSC and
  % how \ODK could fit in there. Furthermore we initiated a partnership
  % with EGI -- a key participant to EOSC -- to deploy \ODK based
  % infrastructure.


\end{itemize}

\subparagraph{\longtaskref{management}{project-quality-management}}

We recall that the Quality Assurance Plan is described in detail in
\longdelivref{management}{ipr}.

\begin{itemize}
% \item Continued success in the recruitment of highly qualified staff.
% \end{enumerate}
\item As in the previous reporting periods, \site{PS} organized the
  interaction with the Advisory Board, composed of seven members, some
  of which specifically represent End Users:
  % The other structure supporting \ODK to ensure the quality of the
  % infrastructure it produce is the Advisory Board. It is
  \begin{itemize}
  \item{Lorena Barba from the George Washington University}
  \item{Jacques Carette from the McMaster University}
  \item{Istvan Csabai from the Eötvös University Budapest}
  \item{Françoise Genova from the Observatoire de Strasbourg}
  \item{Konrad Hinsen from the Centre de Biophysique Moléculaire}
  \item{William Stein, CEO of SageMath Inc.}
  \item{Paul Zimmermann from the INRIA}
  \end{itemize}
  This Advisory Board is composed of academics and/or software
  developers from different backgrounds, countries and communities. It
  is a strong asset to understand the needs of a variety of end-user
  profiles.

\item As in the previous reporting periods, The Quality Review Board
  monitored the quality and the relevance of the software development
  relative to the end-user needs. This board -- chaired by Hans
  Fangohr and composed of four members with a track record of caring
  about the quality of software in computational science -- is
  responsible for ensuring key deliverables do reach their original
  goal and that best practice is followed in the writing process as
  well as in the innovation production process. It met after the end
  of each Reporting Period (RP), and before the Review following that
  RP. More details are given in Section~\ref{section.QAP}.

% More information on
% \longtaskref{management}{project-quality-management} can be found in
% Section 4 of this document: Quality assurance plan.

\item \site{PS} has also been managing risks. Up to the Leeds
  situation, the assessment we present in
  Section~\ref{section.risk_management} has only marginally deviated
  from earlier assessments at Month 12 (\delivref{management}{ipr})
  and Month 36 (\delivref{management}{ipr2}).
\end{itemize}
\subparagraph{\longtaskref{management}{project-innovation-management}}

\begin{itemize}
\item At month 18, \site{PS} had with the help of the consortium a
  first version of Innovation Management Plan
  (\delivref{management}{imp1}), with focuses on:
  \begin{itemize}
  \item{The open source aspect of the innovation produced within \ODK;}
  \item{The various implementation processes the project is dealing with;}
  \item{The strategy to match end-users needs with the promoted VREs}.
  \end{itemize}

  \noindent
  During Reporting Period 3, \site{PS} produced a second version of
  the Innovation Management Plan (\delivref{management}{data-plan2}).
  % to assure that research activities meet the required milestones and
  % produce outputs fully aligned with the project objectives.
  It confirms the elements of strategy of the previous plan, and adds
  two additional sections: one on the choice and impact of open
  licenses in the context of OpenDreamKit, and one on the different
  types of outcome of the project and their respective sustainability.
  This second version also includes minor updates to the earlier
  sections, notably to reflect the work plan revisions that occurred
  after Reporting Period 1.
\end{itemize}

%%% Local Variables:
%%% mode: latex
%%% TeX-master: "report"
%%% End:

%  LocalWords:  subsubsection longtaskref delivref organized Bougeret ipr Csabai

\subsubsection{WorkPackage 2:  Community Building, Training, Dissemination, Exploitation, and Outreach}
\label{dissem}
%Explain, task per task, the work carried out in WP during the reporting period giving details of the work carried out by each beneficiary involved.

\ednote{@nthiery, @IzabelaFaguet, @VivianePons: update WP2 for RP3
  - Update T2.1, notably production of comics and videos
  - update the figures in the overview, T2.3, and T2.5; highlight e.g. CIRM
  - ...?
}

%(T2.1, T2.3,T2.5)

During the last period of the project, We  put considerable effort into communicating the outcomes of our work. The major focus of ODK dissemination framework was to ensure that the project’s outcomes  are widely  disseminated  to  the  appropriate  target communities,  and that those communities could contribute to the development process of the further improved mathematical software systems in the spirit of open source; To this end, during the last year of the project, we participated in many dissemination and communication activities through community building workshops.The material we developed for presentation at all our events were made publicly available.

The aim of  task 2.3 was to organize community building development workshops all throughout the project, to bring together developers from the different communities to design and implement some key aspects of OpenDreamKit such as user interface, and documentation and to ensure cross compatibility. For year 4, we have been part of 7 development meetings that gathered not only participants of ODK but also members of the different communities involved. Those were aimed at a specific software componeents(Sage, GAP, SageMathCloud, IPython, Singular, etc.) and to improve joint developments. It fostered collaboration between scientists and developers from different backgrounds to build tools that are needed by all. These workshops were the first step to disseminate our work improving it. 

The second one was to communicate our activities and make them known though the organization of training workshop where anyone interested were welcomed to learn about ODK and see a demonstration of what ODK components can do. One of the main purposes of our disseminative role was to reach these users while fostering diversity. It is in this spirit that we also organized Sage Days to  promote our tools and bring more users and developers from the scientific world. The aim of the T2.5 task was precisely to gather and train more users and foster scientific development around ODK. During Year 4,  one of our main dissemination event was the CIRM conference in Marseille  that welcomed around 50 participants and had a big impact on the scientific community. It brought together users and developers of most ODK software components and consisted of keynote talks and tutorials with a focus on the development of best practices. 
Another important dissemination event was the Sage Days 105, organized as a satellite event to the main yearly international conference  on algebraic combinatorics also involving aroung 50 participants. It featured several tutorials; demos and presentations including best practices. In addition to these events, were organized  our fifth targeted at women event on April 2019 to reduce gender gap in mathematics software developments.  

The impact of development and training workshops was the awareness rising of project results and of the possibilities to strenghten our collaborative open source development model.

Horizontal  activities  were also  implemented  towards  increasing  the  outreach  of  the project  results  and  improving  the  visibility  of  ODK-Eu funded project.

Our communication activities include:
1.	the project’s website
The website for the project has been continuously updated with new content, and virtually all work in progress is openly accessible on the Internet to external experts and.  The   website   is   the   primary   communication   tool   for   dissemination   and communication.  For  this  reason,  it  became  a  repository  for  a  wide  type  of information and communication material and a long-term dissemination and communication tool.

2. Research and Innovation
Micromagnetism Workshop was organized by XFEL on June 2019, to develop a micromagnetic calculator for driving mumax  micromagnetic simulation tool, it was developed as a part of OpenDreamKit project and the advantage is that it allows to run micromagnetic simulations on GPU. Thus could enable us to reach a much larger target audience in the future.

3. Social Media, blogs 
Social media and blogs are good means of outreach to the public and the presence of the project on social networking platform; It has been established from the early stages of the project and used throughout the entire project life, to promote its improvements and results permitting a two-way exchange of information. These tools were used to update on new technical results, and events that might be of interest for our targeted communities, and also to help to share the demos experience, facilitate adoption of project results by the users, to support best practices.  Social media were used also to strengthen the project’s community online and to raise awareness of the project results, as well as  their use and applicability.

4) Press Releases,  comics and  explainer Video
Press Releases were considered an important dissemination and communication tool at the start of the project and will also be at the end. During the very first year, the project has covered six press releases with a general communication about the project.  To promote ODK innovative method and highlight its results to a general public, we plan to submit press releases at the beginning of November, after the Final review meeting. The procedure for the press release production and distribution is still under revision. The text proposal was made available to all the partners  inviting them to finalize its publication through their press offices. The final press releases will be published in French,  the Coordinator’s and 3 partners main language but also translated in English for the others beneficiaries to enable its publication in local media. 
We also plan to send this article proposal to our European communication officer to publish it in the EC newsletter and submit it for publication in the Horizon magazine.  These Press releases will  be  addressed  to   the general press in the high education, research area but also in local press ,  to audiences that do not require a detailed knowledge of the work carried out.  

Also to raise interest of the General public on the project topic and its impact, our communication strategy was accompanied   by   audio-visually   enhanced   materials targeted at non-specialist general public:
-	we authored (with the help of experts) several explainer comics and life and motion-design videos that have been reused in a variety of contexts to promote Binder. % more info needed
-	In  order  to increase  the  visibility  and  public  acknowledgement  of  the  ODK-EU  project,  we created a 2 minutes motion graphic explainer video with Pix Videos, based on the sketches created by Juliette Belin from Logilab.% was published on Youtube?



%%%%%%%%%%%%%%%%%%%%%%%%%%%%%%%%%%%%%%%%%%%%%%%%%%%%%%%%%%%%%%%%%%%%%%%%%%%%%%
\paragraph{Overview}

  We have continued the work started in period one, especially on \longlocaltaskref{dissem}{dissemination-communication} and
  \longlocaltaskref{dissem}{dissemination}, organizing or participating in more than 50 events throughout the last two years.
  In particular, we have intensified our efforts in \textbf{training} the community to use the tools developed through \ODK.
  Indeed we have had 14 workshops or events directly organized by \ODK on subjects such as Jupyter, JOOMF (see \longlocaltaskref{dissem}{dissemination-of-oommf-nb-virtual-environment}), PARI/GP, and more. On top of that, we were also part of 5 SageDays.

  One of our priorities is to increase the diversity of the open-source community in science. We have been organizing
  and supporting initiatives to support women developers and scientists such as: Women in Sage, Code First: Girl, and
  PyLadies. We have also organized specific events in developing countries: Colombia, Mexico, and Morocco.

  Finally, this period was also an occasion to improve our online communication, following the advice of our last review.
  Indeed, we collaborated with students in communication to design a new website. We have also conducted interviews to explain
  the key points of the project and we are working on some multimedia content.


%%%%%%%%%%%%%%%%%%%%%%%%%%%%%%%%%%%%%%%%%%%%%%%%%%%%%%%%%%%%%%%%%%%%%%%%%%%%%%
\paragraph{Tasks}

\subparagraph{\longtaskref{dissem}{dissemination-communication}}
\label{dissem@dissemination-communication}

We have followed the advice from our reviewers and have worked at a better communication:
\begin{itemize}
\item We have worked in collaboration with a master degree in web communication and design. The
\ODK website was the year project of a team of three students who delivered different reports to us, some
new communication ideas and a whole new website design. We implemented the new design ourselves and
organized a small workshop to think as a group about the website organization and content.
\item To reach a wider audience, we decided to use different media for communication. We hired an interviewer
and created short videos to share with our views on project's goals and achievement with our communities. These
video are in final stage of edition and will be released shortly and made available on the website.

\item We have worked with (motion) graphic designers to create
  infographic content for our project. The infography gives a clear
  and fast way to understand the tools we are developing. A first
  sketch illustrating one of our use cases will be posted on the web
  site by the review. More sketches will be posted later this fall and
  will serve as a basis for an explainer video to be produced by
  PixVideos over the winter.
\end{itemize}

\subparagraph{\longtaskref{dissem}{training-portal}}

Training is a core and transversal aspect of our project. It is carried out
through interventions and events as we discuss in \longlocaltaskref{dissem}{dissemination}. These
past two years especially, there is been a very strong effort from the \ODK team to organize
training events and workshops. We are also working on creating better content on our webpage
to help users understand in what aspect of their work can \ODK help. This is why we have create the
``Use Cases'' section.

\subparagraph{\longtaskref{dissem}{devel-workshops}}
\label{dissem@devel-workshops}

Development workshops are a key aspect of OpenDreamKit development model. The aim of these workshops is to bring together developers from the different communities to design and implement some
of the wanted features. As reported in \longdelivref{dissem}{workshops-3}, we have organized
or co-organized 12 of these workshops throughout years 2 and 3 of the project. The thematics varies
for each event: PARI/GP, Linbox, and many cross-thematic events such as GAP-Sage and GAP-Jupyter days,
live structured documents, low level libraries, and more.

\subparagraph{\longtaskref{dissem}{tech-review}}

This task has been started during period one, especially through \delivref{dissem}{techno}. We continue
to keep track of new technologies and report by writing blog posts on our website.


\subparagraph{\longtaskref{dissem}{dissemination}}
\label{dissem@dissemination}

Dissemination is a key aspect of the success of OpenDreamKit. Indeed, our development is carried
out to help and support mathematical communities. One of the goals is to bring
more users and more developers to the different projects we are involved in. The events
that took place during Years 2 and 3 have been reported in \longdelivref{dissem}{workshops-3}.

\begin{compactitem}
\item \textbf{Training workshops and events.} This has been the most important aspect of this task
for the past two years. We have been organizing 14 events covering subjects such as: Jupyter, JOOMMF,
Bioinformatics, HPC, PARI/GP, GAP, experimental mathematics, web data, reproducible workflows.
\item \textbf{Organization of Sage Days in established mathematical communities.} Sage Days have long been
part of the SageMath tradition. By organizing and supporting Sage Days, OpenDreamKit can stay close
the mathematical community, understand its needs, gather more users and developers, and improve
the overall quality of the software. We have been involved in 5 different such events since the beginning
of the project.
\item \textbf{Training activities in developing countries.} \ODK has been present in Colombia for the second time at
the conference ECCO. We also organized SageDays workshops in Mexico and Algeria as well as a PARI/GP event in Morocco.
\item \textbf{Women in \ODK.} Following the organization of Women in Sage in January 2017, two female \ODK participants
have given time and energy to this specific topic. Viviane Pons organized the Women in Sage event, she has also been
an organizer of the \textit{Pyladies Paris} chapter for the last two years. Another Women in Sage is planned for summer 2019. Tania
Allard was a research software engineer for \ODK until July 2018, she worked at the \textit{Code First: Girl} chapter in
Sheffield, and was also invited to the \textit{Diversity and Inclusion in Scientific Computing} event in 2018.
\end{compactitem}

\subparagraph{\longtaskref{dissem}{project-intro}}

\ednote{@mikecroucher, @trallard, @fangohr: proofread brief overview of work done on T2.6: Introduce OpenDreamKit to Researchers and Teachers}

Training and disseminating to Researchers and Teachers is at the heart
of OpenDreamKit and the participants doubled up their efforts during
the last reporting period. This included the organization of training
events (see \longtaskref{dissem}{dissemination} above), but also many
more evaluation and dissemination activities: teaching with
OpenDreamKit technology (thereby training students and other
instructors alike), local consulting, contributing course material and
templates, etc. This is reported on in
\longdelivref{dissem}{IntroODK}, together with some reflection on the
lessons learned at the occasion of these activities: adoption,
adequateness for the needs, best practice.

It should be noted that Sheffield (now Leeds) has been the lead on
this task until its participants got compelling opportunities in the
industry in Fall 2018. This did not reduce the overall dissemination
activities of the project: indeed, the freed resources were
redistributed to other participants that were eager to organize more
activities than originally planned. There was some impact however:
with continued leadership some more of the lessons learned at the
occasion of those activities could have been formally collated, when
currently many are in the state of shared folklore. Luckily this
information is still spreading in the community through many channels:
informal discussions, blog posts, mailing lists, etc.


\subparagraph{\longtaskref{dissem}{dissemination-of-oommf-nb-virtual-environment}}
\label{dissem@dissemination-of-oommf-nb-virtual-environment}

\ednote{@fangohr: update for RP3: brief overview of work done on T2.7, T2.8 micromagnetism VRE}

This task was mostly carried out during the first period. The Ubermag
(previously called JOOMMF) project is working and available on GitHub
(\href{https://github.com/ubermag}{Ubermag repo}). For each Ubermag
package we use continuous integration on both Travis CI and AppVeyor,
where we perform tests and monitor the test coverage, which we then
make available on \href{https://codecov.io/}{Codecov}. Documentation
for each package consists of APIs (automatically generated from the
code) and different tutorials created in Jupyter notebooks. Both of
them are tested on Travis CI. Documentation is built and made publicly
available on \href{http://discretisedfield.readthedocs.io}{Read the
  Docs}. After every major milestone, we upload each package to the
Python Package Index repository and build a Conda package, which can
later be easily installed on different operating systems. We encourage
the early use of our software and invite for feedback for which we
provide several different communication channels. Ubermag can also be
used in the cloud as a Virtual Research Environment, by using Binder
services.

\subparagraph{\longtaskref{dissem}{dissemination-of-oommf-nb-workshops}}
\label{dissem@dissemination-of-oommf-nb-workshops}

We had several workshops and tutorials during major events where we demonstrated the use of our Micromagnetic VRE, received feedback and feature requests from the community:

\begin{compactitem}
\item IOP Magnetism in April 2017, univ. of York.
\item Intermag in April 2017, Dublin.
\item MMM in November 2017, Pittsburgh.
\item Advances in Magnetism in February 2018, Italy.
\end{compactitem}

\subparagraph{\longtaskref{dissem}{ibook}}

In \longdelivref{dissem}{ibook3c} we report on the delivery of two new
open interactive textbooks. Together with the two books delivered
during RP2 (\longdelivref{dissem}{ibook1}), this was the occasions to
explore various approaches to exploit OpenDreamKit technology for
authoring textbooks. In the deliverable report, we reflect on their
respective merits and suggest some best practice.

\subparagraph{\longtaskref{dissem}{index-librorum-salvificorum}} Not
applicable for this period.  The web toolkit \textit{planetaryum}
(\delivref{dissem}{ils-tool}) has been delivered in the 2nd reporting
period, closing the task.

  
%%% Local Variables:
%%% mode: latex
%%% TeX-master: "report"
%%% End:

%  LocalWords:  subsubsection dissem longtaskref organized co-organized longdelivref emph
%  LocalWords:  Jupyter compactitem dissemination-of-oommf-nb-virtual-environment Piwik
%  LocalWords:  delivref centralized cocalc Cython Pythran textbf organizing EuroScyPy
%  LocalWords:  nbgrader nbgrader specialized Codecov joommf-news Micromagnetic Intermag
%  LocalWords:  dissemination-of-oommf-nb-workshops Fruehjahrstagung Sagecell taskref
%  LocalWords:  adcomp index-librorum-salvificorum

\subsection{WorkPackage 3:  Component Architecture}
%Explain, task per task, the work carried out in WP during the reporting period giving details of the work carried out by each beneficiary involved.

%%%%%%%%%%%%%%%%%%%%%%%%%%%%%%%%%%%%%%%%%%%%%%%%%%%%%%%%%%%%%%%%%%%%%%%%%%%%%%
\paragraph{Overview}

This Work Package focuses on the structure of the components that make
up a mathematical software and their interactions. Such components can
be separate modules inside a unique software, or separate softwares
interacting through library calls and/or through APIs.

The latest reporting period has focused mainly on improving
development workflows and user experience, in particular targeting
notoriously ``difficult'' platforms such as Windows.

%%%%%%%%%%%%%%%%%%%%%%%%%%%%%%%%%%%%%%%%%%%%%%%%%%%%%%%%%%%%%%%%%%%%%%%%%%%%%%
\paragraph{Milestones} Helping end users perform computations on
whatever hardware they possess is one of the major goals of
OpenDreamKit, and of WP3 in particular. The only milestone involving
WP3 is

\subparagraph{\longmilestoneref{component-architecture-distribution}}

\emph{“User story: users shall be able to easily install ODK's
    computational components on the three major platforms (Windows,
    Mac, Linux) via their standard distribution channels.”}

  With the completion of
  \longdelivref{component-architecture}{portability-cygwin}, all
  OpenDreamKit components now run on Windows. Packages for the major
  Linux distributions (Debian, Ubuntu, Fedora, Arch, ...) have also
  been available for at least a year, thanks to the efforts of the
  community\footnote{Note that the role of OpenDreamKit is to
    facilitate packaging for Linux distributions, by simplifying
    dependency management and build chains, and keeping up to date
    with dependencies. It is not OpenDreamKit's goal to directly take
    the lead on packaging for the dozens of available distributions,
    as this would not be sustainable. We will keep monitoring the
    status of Linux packages, prioritizing more popular distributions
    such as Ubuntu, and continue our efforts to make our components
    easy to package.}. MacOS binaries are regularly released, albeit
  with the usual hiccups typical of the Apple ecosystem.

  The bulk of the milestone is thus completed, ahead of schedule,
  although there is still work to do, the most important items on the
  agenda being better continuous integration, and support for Python 3
  in SageMath\footnote{Python 3 support is vital for SageMath going
    into the year 2020, when Python 2 will be officially
    deprecated.}. We will focus on these items for the remaining year.
  
%%%%%%%%%%%%%%%%%%%%%%%%%%%%%%%%%%%%%%%%%%%%%%%%%%%%%%%%%%%%%%%%%%%%%%%%%%%%%%
\paragraph{Tasks}

  \paragraph{\longtaskref{component-architecture}{portability}}
  \label{component-architecture@portability}
  The first task of this workpackage is to improve the portability of
  computational components.

  The most challenging target is the Windows platform, and indeed
  SageMath has not had native Window support for years. With the
  completion of
  \longdelivref{component-architecture}{portability-cygwin}, we are
  happy to announce Windows support for SageMath: since version 8.0,
  released in July 2017, a one-click installer based on Cygwin is the
  recommended way to install SageMath on Windows. With this
  deliverable, we achieved Windows support for 100\% of OpenDreamKit's
  components.

  In support of developing and maintaining OpenDreamKit's software on
  all platforms, we have also worked on infrastructures for continuous
  integration. No unique solution was found that could accommodate the
  needs of every project, however, through sharing information and
  experience returns, each of the software projects inside
  OpenDreamKit has managed to put in place the infrastructure better
  suited for its needs, leveraging various popular technologies such
  as Docker, Jenkins, etc. These efforts have been reported in
  \longdelivref{component-architecture}{multiplatform-buildbot}.

  \paragraph{\longtaskref{component-architecture}{interface-systems}}
  \label{component-architecture@interface-systems}
  In this task we investigate patterns to share data, ontologies,
  and semantics across computational systems, possibly connected
  remotely.

  The work concerning this work package was essentially completed in
  Year 1, through
  \href{http://www.symbolic-computing.org/science/index.php/SCSCP}{Symbolic
    Computation Software Composability Protocol (SCSCP)}. All
  subsequent planned work has been moved to WP6.

  
  \paragraph{\longtaskref{component-architecture}{mod-packaging}}
  \label{component-architecture@mod-packaging}
  In this task we investigate best practices for composing, sharing
  and interfacing computational components and data for connected
  mathematical systems.

  The main deliverable in this task is
  \delivref{component-architecture}{sage-distribution}, due in month
  48. Thanks to the joint efforts of OpenDreamKit and of the
  community, SageMath is now available as a Debian package, and
  recently also as a Conda package.

  This task is progressing as planned, and we expect to successfully
  complete it next year.

  \paragraph{\longtaskref{component-architecture}{simulagora-dev}}
  The goal of this task is to deliver every six months a new Simulagora
  VM image containing all the software components released over the
  period.

  To this date, five OpenDreamKit VMs have been released in
  Simulagora. The latest version, released in March 2018,
  showcases virtual desktops available from a web browser and
  collaboration workflows based on ``tools'' that can be described as
  micro web applications that require very little development skills
  to set up, but make it easy to make available complex simulations to
  users.
  
  \paragraph{\longtaskref{component-architecture}{component-for-HPC}}
  Not applicable for this period.

  \paragraph{\longtaskref{component-architecture}{extract-smc}}
  \label{component-architecture@extract-smc}
  Recall \cocalc used to be called \SMC at the beginning of this project.
  This task has been terminated early due to the cancellation of
  \longdelivref{component-architecture}{personal-smc}, achieved by the
  \cocalc developers \emph{before the start of \ODK}.

  The resources planned for this task were diverted to other work
  packages.
  
  \paragraph{\longtaskref{component-architecture}{workflow}}
  This task seeks new ways of accepting contributions to mathematical
  software in a scalable way.

  Deliverable \longdelivref{component-architecture}{smc-trac} was
  considerably reshaped to take into account the recent developments
  in the ecosystem. This caused a one year delay in the delivery.

  Thanks to the work done, the entry barrier for developing SageMath
  has been considerably lowered. It is now possible for users with a
  GitHub or GitLab account to contribute to SageMath without having to
  go through a manual (and slow) registration process, and editing
  documentation is now easier then ever, even for the inexperienced
  user.

  With the delivery of
  \delivref{component-architecture}{smc-trac}, this task is now
  complete.

  \paragraph{\longtaskref{component-architecture}{oommf-python-interface}}
  \label{component-architecture@oommf-python-interface}
  Not applicable for this period.

  
%%% Local Variables:
%%% mode: latex
%%% TeX-master: "report"
%%% End:

%  LocalWords:  subsubsection longmilestoneref emph longdelivref portability-cygwin
%  LocalWords:  prioritizing longtaskref multiplatform-buildbot Composability delivref
%  LocalWords:  simulagora-dev Simulagora extract-smc cocalc personal-smc smc-trac
%  LocalWords:  oommf-python-interface

\subsubsection{WorkPackage 4: User Interfaces}
%Explain, task per task, the work carried out in WP during the reporting period giving details of the work carried out by each beneficiary involved.

%%%%%%%%%%%%%%%%%%%%%%%%%%%%%%%%%%%%%%%%%%%%%%%%%%%%%%%%%%%%%%%%%%%%%%%%%%%%%%
\paragraph{Overview}

The objective of WorkPackage 4 is to provide modern, robust, and flexible user interfaces for
computation, supporting real-time sharing, integration with collaborative problem-solving,
multilingual documents, paper writing and publication, links to databases, etc. This work is focused primarily around the \Jupyter project, in the form of:

\begin{itemize}
    \item Enhancing existing \Jupyter tools (\localtaskref{UI}{notebook-collab})
    \item Building new tools in the \Jupyter ecosystem (\localtaskref{UI}{notebook-verification}, \localtaskref{UI}{notebook-collab}, \localtaskref{UI}{vis3d})
    \item Improving the use of \ODK components in \Jupyter and \Sage environments (\localtaskref{UI}{ipython-kernels}, \localtaskref{UI}{sage-sphinx}, \localtaskref{UI}{dynamic-inspect}, \localtaskref{UI}{pari-python})
    \item Demonstrating effectiveness of WorkPackage 4 results in specific scientific applications (\localtaskref{UI}{cfd-vis}, \localtaskref{UI}{oommf-py-ipython-attributes}, \localtaskref{UI}{oommf-nb-ve}, \localtaskref{UI}{oommf-tutorial-and-documentation})
    \item Work on Active Documents, which have some goals in common with \Jupyter notebooks (\localtaskref{UI}{structdocs}, \localtaskref{UI}{mathhub})
\end{itemize}

All deliverables for WorkPackage 4 have been delivered and highly successful in previous reporting periods.
There are no new deliverables in Reporting Period 3.
However, the work of software is never really complete.
Work has continued on some tasks to further improve,
mature, and maintain the results of WorkPackage 4
toward sustainability and to best serve \ODK objectives
based on feedback from \ODK and the wider user community.

%%%%%%%%%%%%%%%%%%%%%%%%%%%%%%%%%%%%%%%%%%%%%%%%%%%%%%%%%%%%%%%%%%%%%%%%%%%%%%
\subparagraph{Milestones}

\subparagraph{\longmilestoneref{UI-vre}}

\emph{“The prototype VRE shall be extended with improved ease of deployment, new
  functionality such as interactive 3D visualization and real-time
  collaboration, enabling researchers to collaborate productively in a shared
  computational environment. Finally, integrating notebooks and semantic
  knowledge into a publication / knowledge system enable a continuous process
  of leveraging \ODK components from research to publication.”}


The \Jupyter-based prototype for this has been previously delivered in \longmilestoneref{UI-vre-prototype},
and is extended in \longtaskref{UI}{notebook-collab} to more mature functionality.

WorkPackage 4 has resulted in a number of useful pieces of software
for mathematical researchers,
sometimes creating new software,
improving existing software,
or establishing new or improved connections between two existing systems.

Combining the above, Milestone~\longmilestoneref{UI-vre} has
been reached:
from the obtained toolkit, we can produce a \Jupyter-based VRE,
integrating \ODK components.
The Jupyter kernels delivered in \localtaskref{UI}{ipython-kernels}
enable access to a broader collection of mathematical software.
The interactive utility of software such as \Pari is improved in \localtaskref{UI}{pari-python},
and general interactivity and exploration of mathematical objects in \Sage is improved in \localtaskref{UI}{dynamic-inspect}.
The scope of what classes of work can be made interactive is increased
by the development of interactive three-dimensional visualization tools in \localtaskref{UI}{vis3d}.
Further, the process of collaboration on notebook documents is improved by \localtaskref{UI}{notebook-collab}
and prototype support for live collaboration with \localtaskref{UI}{notebook-collab}.
By focusing on \Jupyter as our User Interface of choice,
all of these tools can be combined in a single VRE,
hosted in the cloud or and made accessible to any researcher,
building on the Docker images created in \longdelivref{component-architecture}{virtual-machines}.

The work in this final reporting period has focused on stabilising and maturing the software delivered in previous periods.

%%%%%%%%%%%%%%%%%%%%%%%%%%%%%%%%%%%%%%%%%%%%%%%%%%%%%%%%%%%%%%%%%%%%%%%%%%%%%%
\paragraph{Tasks}

\subparagraph{\longtaskref{UI}{ipython-kernels}}
\label{UI@ipython-kernels}

All deliverables for this task have been delivered in previous reporting periods.

Kernels for \ODK components \GAP, \Pari, \Sage, and \Singular,
had been delivered in the form of \delivref{UI}{ipython-kernels-basic}
in RP1 and \longdelivref{UI}{ipython-kernels} in RP2.
Work has continued to develop these kernels in this reporting period
to bring them to further maturity and sustainability.

\smallskip
\subparagraph{\longtaskref{UI}{notebook-collab}}
\label{UI@notebook-collab}

All deliverables for this task have been delivered in previous reporting periods.

Prototype components and plan for \delivref{UI}{jupyter-live-collab} had been delivered in RP2.
This has been developed to further complete prototypes of real-time collaboration in JupyterLab in collaboration with the \Jupyter community.
We are optimistic about its completion and adoption in JupyterLab in the near future.
Real-time collaboration has proven to be the largest and most challenging
effort in WP4,
both in terms of technical effort and in community engagement.
The reason being that real-time collaboration needs extensive work
in development in the core of JupyterLab itself,
which required collaboration and coordination with the JupyterLab community for assembling plans and implementation,
aligning with other goals of the JupyterLab project,
including development of new features in the phosphorjs framework on with JupyterLab is based,
and a complete refactor of the JupyterLab data model.
This work has involved participation in workshops and meetings,
as well as addition of \ODK team members to the core JupyterLab team.
As of August 2019, real-time collaboration has been implemented in JupyterLab in a \texttt{datastore} branch on the official jupyterlab repository on GitHub,
and is expected to arrive in a public release of JupyterLab soon.

In addition, further releases of \texttt{nbdime} from
\delivref{UI}{jupyter-collab} have been made
to better support asynchronous collaboration.

This work furthers \ODK objective 5 of promoting sustainable software in math and science.


\smallskip
\subparagraph{\longtaskref{UI}{notebook-verification}}
\label{UI@notebook-verification}

All deliverables for this task have been delivered in previous reporting periods.

\longdelivref{UI}{jupyter-test} was delivered in the form of a new Python package, \texttt{nbval},
which enables testing and verification of existing notebooks via a plugin to the Python testing
framework \textbf{pytest}.
In this reporting period, nbval has received further activity and contributions and new releases.
nbval integrates with nbdime from \delivref{UI}{jupyter-collab} to deliver
testable, reproducible notebooks via traditional software development testing practices.
This work furthers \ODK objective 5 of promoting sustainable software in math and science.

\smallskip
\subparagraph{\longtaskref{UI}{sage-sphinx}}
\label{UI@sage-sphinx}

%%% Updated for RP3 by Jeroen Demeyer %%%
Even though this reporting period contains no explicit deliverables
for this task, significant foundation work was carried out which we
now describe. Documentation tools such as Sphinx rely on introspection
to harvest the documentation out the sources. For performance, a large
fraction of the SageMath sources is however written in Cython
(compiled Python) which, until recently, had an incompatible and
limited introspection API. This forced SageMath and other projects to
maintain bespoke and fragile Sphinx extensions to harvest their
documentation.

Tackling this required to dig deep into the system and design,
implement, and get accepted a change to Python itself: PEP (Python
Enhancement Proposal) 590. PEP 590 makes available Python's fast
calling protocol to custom code, thereby enabling full support for
introspection and documentation to Python functions implemented in C
-- e.g. Cython functions --, with no performance loss. This has been
implemented in the upcoming Python~3.8 and Cython~3.0 releases. We
expect not only Cython and therefore SageMath to benefit from this,
but also other similar projects such as Pythran or Numba.

\smallskip
\subparagraph{\longtaskref{UI}{dynamic-inspect}} Due M36 (\delivref{UI}{ipython-advanced-interacts})
\label{UI@dynamic-inspect}

All deliverables for this task have been delivered in previous reporting periods.

As planned in \delivref{UI}{ipython-advanced-interacts}, \ODK
packages \emph{Sage-Combinat-Widgets} and \emph{Sage-Explorer} were
further developed during RP3.
%
%In versions 0.5.0 to 0.7.6,
\emph{Sage-Combinat-Widgets} has gained in
flexibility and has been applied to a range of new mathematical
objects. User interfaces features like feedback have been enhanced,
and documentation has been augmented and gained a tutorial.
%
%With version 0.5.0,
\emph{Sage-Explorer} has gone through a complete new design and reengineering process,
at the same time for better modularity in the code and for better ergonomics.
%
Finally, the \emph{Francy} Jupyter-based graph visualisation library
was generalized to support \Python -- and therefore \SageMath -- in
addition to \GAP.
%
All three benefited from feedback, if not contributions, from end-users.

% Both build on the robust
% foundation of Jupyter Widgets, and explore what it can bring to
% interactive mathematics. The former focuses on interactive
% visualization and edition of mathematical objects, taking
% combinatorics and discrete math as use case. The latter, which uses
% the former as building block, provides rich, detailed, and efficient
% interactive exploration of objects, their properties and
% interrelations. Both are
% \href{https://github.com/sagemath/sage-explorer}{demonstrated online}
% via the Binder service.


\smallskip
\subparagraph{\longtaskref{UI}{structdocs}}
\label{UI@structdocs}

All deliverables for this task have been delivered in previous reporting periods.

% Active structured documents are a common need with many use cases, and has many potential
% solutions.  Requirements and venues for collaborations were explored through discussions
% between participants, in particular at the occasion of
% \href{https://wiki.sagemath.org/days77/}{Sage Days 77} workshop (see the
% \href{https://wiki.sagemath.org/days77/live-structured-documents}{notes}), and the ODK
% meeting in Bremen. The findings were reported in \longdelivref{UI}{adstex}.

% In \longdelivref{UI}{adcomp}, We have presented a general framework for in-situ computation in active documents. This is
% a contribution towards using mathematical documents -- the traditional form mathematicians
% interact with mathematical knowledge and computations -- as a user interface for a
% mathematical virtual research environments. This is also a step towards integrating the
% two main UI frameworks under investigation in the \ODK project: \Jupyter notebooks and
% active documents -- see~\longdelivref{UI}{adstex} -- at a conceptual level. The system is
% prototypical at the moment, but can already be embedded into active documents via a
% Javascript framework and is ready for use in the \ODK project. The user interface and \SCSCP
% connections are quite fresh and need substantial testing and optimizations.

% \ODK hosted a workshop on live structured documents in October 2017,
% which resulted in the development of \href{https://github.com/minrk/thebelab}{thebelab} software for interactive computing on any website,
% enabling interactivity in traditional web-based documentation,
% and further development of the \MathHub facilities for evaluation in structured documents.

During RP3, we developed the JupyterLab extension
\href{https://gitlab.com/logilab/jupyterhub-training}{JupyterLabTraining}
dedicated to teaching programming, e.g. in Python or Sage.
It provides an environment where learners can autonomously do a
series of exercises in order to learn a new programming language. Each exercise is
an independant Jupyter notebook containing the questions, a cell where the learner will
write her code, a hidden cell containing automated tests, and a button to run these tests
and check the code that has been written answers the questions. The left panel shows
the list of all the exercises; they can be sorted by topic (keyword), complexity or
learning track. Thanks to this environment, each learner can do the exercises at his
own pace and choose the exercises that focus on his own points of interest. The
learning process is thus much more efficient for each person.

We also developed further the
\href{https://github.com/minrk/thebelab}{thebelab} software for
interactive computing in traditional web-based documentation.

\smallskip
\subparagraph{\longtaskref{UI}{mathhub}}
\label{UI@mathhub}

All deliverables for this task have been delivered in previous reporting periods.

One of the most prominent features of a virtual research environment (VRE) is a unified user interface. The \ODK approach is to create a mathematical VRE by integrating various pre-existing mathematical software systems. There are two approaches that can serve as a basis for the \ODK UI: computational notebooks and active documents. The former allows for mathematical text around the computation cells of a read-eval-print loop of a mathematical software system and the latter makes semantically annotated documents active.

\MathHub is a portal for active mathematical documents ranging from formal libraries of theorem provers to informal – but rigorous – mathematical documents lightly marked up by preserving LaTeX markup.

As the authoring, maintenance, and curation of theory-structured mathematical ontologies and the transfer of mathematical knowledge via active documents are an important part of the \ODK VRE toolkit, the editing facilities in \MathHub play a great role for the project,
as delivered in \longdelivref{UI}{mathhub-editing}.

\smallskip
\subparagraph{\longtaskref{UI}{vis3d}}
\label{UI@vis3d}

All deliverables for this task have been delivered in previous reporting periods.

The software developed for this task has been delivered in earlier reporting periods.
Packages such as ipyvolume and k3d-jupyter have received further development,
improved compatibility with JupyterLab,
and developed toward maturity and stability,
with growing community adoption.
Several contributions have been made to JupyterLab and
the \Jupyter ecosystem to further support similar work,
benefiting a wide user community.

\smallskip
\subparagraph{\longtaskref{UI}{cfd-vis}} % M12-36
\label{UI@cfd-vis}

No work to report in this period.


\smallskip
\subparagraph{\longtaskref{UI}{Sage-display}} % M24, no deliverables

No work to report in this period.

\smallskip
\subparagraph{\longtaskref{UI}{oommf-py-ipython-attributes}} % M13-19
\label{UI@oommf-py-ipython-attributes}

The micromagnetic virtual research environment is hosted in the
\Jupyter Notebook. The computational backend is the existing \OOMMF
(Object Oriented MicroMagnetic Framework) simulation tool, which is
accessible through the new Python interface that has been created as
part of \ODK
(\localtaskref{component-architecture}{oommf-python-interface}). The
\Jupyter Notebook allows us to integrate the micromagnetic model
specification, the execution of the simulation, and the postprocessing
and data representation within a single executable document; providing
a new computational research environment for micromagnetic simulation
that uses the most widely used simulation code. We have enhanced this
environment further by exploiting that the notebook allows objects to
represent themselves in different ways within the notebook. For
example, Python objects that represent mathematical equations in the
micromagnetic VRE appear rendered as \LaTeX{} in the notebook. It
allows users to interactively compose and explore computational
models, and to be able to inspect what they have put together in the
language of the scientist (i.e. through equations) rather than through
the language of the computer (i.e. code). The addition of this
representation options does not stop the code from being valid \Python
that can be run outside the notebook. We have also provided a
graphical representation of the mesh and discretisation cell as the
appropriate representation of a finite difference mesh to further
assist the effective communication between code and science user and
graphical representation of vector field objects.  We have used
dissemination workshops to seek feedback from users and to refine
interface.

\smallskip
\subparagraph{\longtaskref{UI}{pari-python}}
\label{UI@pari-python}

No work to report in this period.

% \ednote{@jdemeyer, @videlec: proofread/update report on T4.12: Pari bindings}

% There has been a great deal of progress delivering improved \Pari.
% This work has resulted in benefits to the wider Python and \Sage communities
% via substantial contributions to the \Sage codebase,
% the benefits of which go well beyond this deliverable,
% being used by projects outside \ODK.

% The end results of this first state of the work are the packages
% \href{https://github.com/sagemath/cysignals}{cysignals} and
% \href{https://github.com/defeo/cypari2}{CyPari2}, both installable
% in a pure \Python environment via the standard tool
% \texttt{pip}. Starting from version 8.0, installation via \texttt{pip}
% is \Sage's default way of providing the \Pari interface.

% \longdelivref{UI}{pari-python-lib2} has been delivered, further improving the \Pari packages
% by adding new features, in particular to the Python interface to \Pari.
% \emph{cypari2} has gained the ability produce high-resolution SVG plots.
% It now also supports the dynamic array type from PARI/GP, \verb/t_LIST/.
% The source code of cypari2 is automatically generated.
% This automatic generation has been greatly improved
% and can be re-used outside cypari2 for any Python package that wants to interface efficiently with PARI.
% The cypari2 documentation is also greatly improved,
% as a direct result of improvements to the Sphinx documentation system
% in \localtaskref{UI}{sage-sphinx}.

\smallskip
\subparagraph{\longtaskref{UI}{oommf-tutorial-and-documentation}
  has been merged into
  \longlocaltaskref{dissem}{dissemination-of-oommf-nb-virtual-environment}
}
\label{UI@oommf-tutorial-and-documentation}

\smallskip
\subparagraph{\longtaskref{UI}{oommf-nb-ve}
  has been merged into
  \longlocaltaskref{dissem}{dissemination-of-oommf-nb-virtual-environment}
}
\label{UI@oommf-nb-ve}

%%% Local Variables:
%%% mode: latex
%%% TeX-master: "report"
%%% End:

%  LocalWords:  subsubsection Jupyter taskref notebook-collab ipython-kernels cfd-vis
%  LocalWords:  oommf-py-ipython-attributes oommf-nb-ve oommf-tutorial-and-documentation
%  LocalWords:  mathhub longmilestoneref emph visualization longdelivref UI-vre delivref
%  LocalWords:  jupyter-live-collab ipython-kernel-sage jupyter-collab texttt nbdime
%  LocalWords:  nbval textbf pytest Cython-generated ipython-advanced-interacts adstex
%  LocalWords:  adcomp optimizations thebelab ipyvolume pythreejs threejs ipyscales unray
%  LocalWords:  ipydatawidgets micromagnetic oommf-python-interface cysignals cypari2
%  LocalWords:  dissem dissemination-of-oommf-nb-virtual-environment

\subsubsection{WorkPackage 5: High Performance Mathematical Computing}
  \label{hpc}
%Explain, task per task, the work carried out in WP during the reporting period giving details of the work carried out by each beneficiary involved.


  %%%%%%%%%%%%%%%%%%%%%%%%%%%%%%%%%%%%%%%%%%%%%%%%%%%%%%%%%%%%%%%%%%%%%%%%%%%%%% 
  \paragraph{Overview}

  Workpackage 5 is about the development of high performance computing tools in
  mathematical virtual research environments. It is addressed at the level
  of each kernel library composing the computational tools of the project (\Pari,
  \GAP, \Linbox, \MPIR, \Sage, \Singular, ...), and also at the level of interfacing and exposing
  core parallel features to higher level programming interfaces.

  Key results obtained over the period for WorkPackage 5 are the following:
  %% Only list deliverables produced in the reporting period
  \begin{compactitem}
  %% \item A closer integration of \Linbox in \Sage with improved reliability and
  %%   computing efficiency.
  \item A full-featured parallelisation engine, supporting POSIX threads and
  MPI, for PARI in production release of the software
  \item Release of GAP-4.9 allowing compilation  in HPC-GAP compatibility mode.
  %    \item A new super-optimizer for vectorized assembly code and its
  %  exploitation to improve the performances of the MPIR code.
  \item A new symmetric matrix factorization algorithm over finite fields, and
  its high-performance implementation in the \texttt{fflas-ffpack} library.
  \item Major redesign the the polynomial arithmetic used in Singular delivering
  state of the art efficiency.
\end{compactitem}

%%%%%%%%%%%%%%%%%%%%%%%%%%%%%%%%%%%%%%%%%%%%%%%%%%%%%%%%%%%%%%%%%%%%%%%%%%%%%%
\subparagraph{Milestones}

\subparagraph{\longmilestoneref{hpc-prototype}}

\emph{“User story: Astrid wants to run compute intensive routines
    involving both dense linear algebra and combinatorics. She has
    access through a JupyterHub-based VRE to a high end multi-core
    machine which includes a vanilla \Sage installation. She
    automatically benefits from the HPC features of the underlying
    specialized libraries (\Linbox, ...). This is a proof of concept
    of the overall framework to integrate the HPC advances of
    specialized libraries into a general purpose VRE.
    %
    It will prepare the final integration of a broader set of such
    parallel features for the end of the project.”}

With Deliverable~\delivref{hpc}{LinBox-algo}, we developped, released and integrated in the
\Sage the LinBox library and its core dependencies: fflas-ffpack and givaro.
When installing the latest \Sage release on a multithreaded multicore server, it
only takes one configure option to let fflas-ffpack use a multi-threaded BLAS
and therefore expose its parallel speed-up to the end-user of Sage. This feature
is compliant with the use of a higher level of parallelism, through process
workstealing queues that \textit{Astrid} may be using in her combinatorics code, as those
exposed in \delivref{hpc}{sage-HPCcombi}. Now that this first proof of concept has been
successfully achieved, we are working in exposing the more advanced parallel
routines of fflas-ffpack into \Sage, following~\delivref{hpc}{LinBox-DSL}. It
should in particular make Gaussian elimination and related routines enjoy a
better scaling with respect to available CPU cores.

%%%%%%%%%%%%%%%%%%%%%%%%%%%%%%%%%%%%%%%%%%%%%%%%%%%%%%%%%%%%%%%%%%%%%%%%%%%%%%
\paragraph{Tasks}

\subparagraph{\longtaskref{hpc}{hpc-pari}}
Deliverable~\longdelivref{hpc}{pari-hpc1} has been merged
 with \longdelivref{hpc}{pari-hpc2} in the revised workplan.

Deliverable~\delivref{hpc}{pari-hpc2} is on time or even ahead of schedule.
The generic engine for Deliverable~\delivref{hpc}{pari-hpc1} was written and
finalized in 2015 and 2016 and is in production since PARI-2.9 (released
11/2016), it supports sequential evaluation (no parallelism), POSIX threads and
MPI within the same code base.
Development work since then is targeted at using the engine wherever it makes
sense in the code base. The current situation (PARI-2.11, released 07/2018)
includes:
\begin{itemize}
\item fast (near linear time) Chinese remaindering;
\item fast linear algebra over Q and cyclotomic fields, a critical component of
the new "Modular Forms" package;
\item polynomial resultant in $\mathbb{Z}[X]$ via Chinese remainders;
\item computation of classical modular polynomials for about 20 classical
invariants (j, Weber functions, small eta quotients...);
\item discrete logarithm over finite fields (prime fields and
$\mathbb{F}_{p^e}$ for word-sized prime $p$) - polynomial resultant in
$\mathbb{Z}[X] \times \mathbb{Z}[X,Y]$ via 
Chinese remainders / evaluation;
\item APRCL primality proof.
\end{itemize}

More work is underway regarding the development of  ECPP, integer
factorization engines, class group and units computations.

Compared to the pre-existing sequential code, the parallel version adds about 15 extra lines of
C on average (conveniently encapsulated in a separate "worker" object). As usual
in "embarassingly parallel" code, the measured speedup is essentially linear in
the number of available cores. The resulting GP binary has been successfully
crash-tested by a panel of users during the 2017 PARI/GP Atelier in Lyon:
basically, substituting the \texttt{Configure} instruction by \texttt{Configure --mt=pthreads}
before compilation transparently yields the expected speedups for the target
functions on participants dual-core laptops. 


\subparagraph{\longtaskref{hpc}{hpc-gap}}
\label{hpc@hpc-gap}

No deliverable is due for the evaluation period but steady progress was made on
Deliverable~\longdelivref{hpc}{GAP-HPC-report}. Over this period, eight releases were cut
incorporating contributions to Deliverable~\longdelivref{component-architecture}{hpc-configure}.

Another major direction of efforts is the HPC-GAP integration:
HPC-GAP is a fork of \GAP initiated during the \scienceproject project, which
enables multithreaded calculations. Now that HPC-GAP has reached
maturity, it's critical for its widespread use and long term
maintainability to merge it back into \GAP's master branch.
The first step towards this long-standing goal, which is at the core of
Task~\localtaskref{hpc}{hpc-gap}, is the major release of \GAP 4.9, published in 2018 and allowing compilation in HPC-GAP
compatibility mode. It comes together with the new manual book called ``HPC-GAP Reference Manual'',
which is also available online at \url{https://www.gap-system.org/Manuals/doc/hpc/chap0.html}.
HPC-GAP integration required a major
refactorization of \GAP's build system
mainly developed by our external collaborator Max Horn (Giessen) into \GAP's master
branch during the Sage-GAP-Days-85 we organized in March 2017. 
An overview of the most important changes introduced in GAP 4.9.1,
with links to corresponding GitHub entries,
can be found in the \GAP manual: \url{https://www.gap-system.org/Manuals/doc/changes/chap2.html}.
Many \GAP packages also have been updated to work in \GAP 4.9 and make use of its new features.

\subparagraph{\longtaskref{hpc}{hpc-linbox}}
  \label{hpc@hpc-linbox}

The deliverable~\longdelivref{hpc}{LinBox-distributed} due for this reporting period is
delivered on time.

A first focus was made on distributed computing, with an MPI parallelization of a Chinese remainder based
algorihtm. The prototypical implementation first produced was then cleanly integrated in the mainstream code of the
library. Its performance show a very nice scaling with the number of compute nodes on a 256 cores cluster. Although this
approach is best suited for large scale parallelization, its generates a suboptimal workload which become a major
concern on large instances.

An alternative approach based on $p$-adic lifting has an  asymptotically smaller workload, but is intrinsically more
sequential, and therefore less suited for large scale parallelization. 
A major contribution in this task is a new algorithm combining $p$-adic lifting and Chinese remaindering in order to
expose more parallelism without sacrifying the gain in the workload.
We also provide a full-featured  implementation of this new algorithm in \Linbox, which shows delivers both high
efficiency in sequential, and scales well up to 16 cores on a multicore server.

Lastly, we also introduced a support for GPUs in the \texttt{fflasffpack} library and show how matrix product over a finite
field benefit from these accelerators.

All these software improvements are closely integrated in the mainstream code of the \texttt{fflasffpack} and \Linbox libraries.

  \subparagraph{\longtaskref{hpc}{hpc-singular}}
  \label{hpc@hpc-singular}

  The only deliverable under consideration for this reporting period
  is~\longdelivref{hpc}{singular-polyarith}

Multivariate polynomials are represented in Singular using the sdmp format. While this data structure is generally amenable to parallelization, the implementation and some of the algorithms in Singular are not. Since the last update much work has been invested in updating the algorithms and data structures and making Singular polynomial arithmetic competitive with other systems. This work has been done in the Singular submodule Flint, whose code is available at \url{https://github.com/wbhart/flint2}. We also now support polynomial exponents of unlimited size with the three basic monomial orderings of lex, deglex, and degrevlex. Suggestions by colleagues in the HPC community including Bernard Parisse, Michael Monagan, Roman Pearce, and Micka\"el Gastineau have been invaluable.

The serial implementations of the operations of multiplication, division and GCD are complete in both the dense and sparse cases and the performance is competitive with other systems. The parallel implementation of multiplication is also complete with competitive performance as well. We are on track to have division and GCD parallelized and delivered on time.

We have also been able to parallelize polynomial root clustering, which is a
major engine in Singular that benefits from fine-grained
parallelization. Performance improvements continue to be made by Remi Imbach
(now at NYU following his ODK contract). The fully working implementation is
at \url{https://github.com/rimbach/Ccluster}.




  \subparagraph{\longtaskref{hpc}{hpc-mpir}}
  \label{hpc@hpc-mpir}

  Not applicable for this period.
  
  \subparagraph{\longtaskref{hpc}{hpc-combi}}
  \label{hpc@hpc-combi}

  Not applicable for this period.


  \subparagraph{\longtaskref{hpc}{pythran}}
  \label{hpc@hpc-pythran}
  Not applicable for this period.

%%   The goal of this task is to make Pythran easily integratable in large-scale
%%   project, taking into account native dependencies, compilation time, memory
%%   footprint, speed and size of compiled binaries as well as multi-platform
%%   support. Integration with \software{cython} is a possible mean to achieve
%%   this goal.

%%   Two projects have been selected for this task: \software{scipy} and
%%   \software{scikit-image}. These projects are relevant for \software{pythran}
%%   because they have many small to medium kernels that can benefit from
%%   compilation. Even though \software{pythran} has not been selected as a scipy
%%   backend, the exchanges with the community have led to a great deal of
%%   improvements of which all \software{Pythran} users take advantage.  The
%%   \software{scikit-image} community is still examining the possibility of using
%%   \software{pythran} as an acceleration mean.

%%   Integration of  \software{pythran} as a \software{cython} backend for
%%   \software{numpy} has improved in various aspects: better error detection,
%%   more supported expression patterns and improved performance for the compiled
%%   expressions.

%% Deliverable~\longdelivref{hpc}{sage-HPCcombi} is shared with
%% Task~\longlocaltaskref{hpc}{hpc-combi}, the status of which we reported on above.

  \subparagraph{\longtaskref{hpc}{hpc-jupyter}}
  \label{hpc@hpc-jupyter}
  
%% It is common for academic High Performance Computing (HPC) clusters to make
%% use of schedulers based on Sun Grid Engine with Son of Grid Engine as one of
%% the most popular. It is used, for example, on the institutional HPC systems
%% in the Universities of Sheffield and Manchester in the United Kingdom. It is also used
%% on the regional N8 HPC facility, a system shared by the eight most research
%% intensive universities in the North of England.
  Not applicable for this period.

%  LocalWords:  subsubsection hpc compactitem super-optimizer vectorized factorization
%  LocalWords:  texttt fflas-ffpack longmilestoneref emph JupyterHub-based specialized
%  LocalWords:  longtaskref longdelivref delivref finalized mathbb embarassingly taskref
%  LocalWords:  scienceproject refactorization organized LinBox-algo DumPerSul:fcrpmgbd16
%  LocalWords:  Pernet:cqm16,PerSto:tsegqm17 DumKalTho:lticmpdsm16,DumLucPer:cftearp17
%  LocalWords:  Cython Hongguang singular-polyarith sdmp parallelization deglex degrevlex
%  LocalWords:  Monagan Micka parallelized parallelize Imbach hpc-mpir sage-paral-tree
%  LocalWords:  Cilk libsemigroup optimized pythran integratable scipy scikit-image numpy
%  LocalWords:  sage-HPCcombi hpc-jupyter

\subsubsection{WorkPackage 6:  Data/Knowledge/Software-Bases}\label{dksbases}
%Explain, task per task, the work carried out in WP during the reporting period giving details of the work carried out by each beneficiary involved.

%%%%%%%%%%%%%%%%%%%%%%%%%%%%%%%%%%%%%%%%%%%%%%%%%%%%%%%%%%%%%%%%%%%%%%%%%%%%%%
\paragraph{Overview}

In a series of workshops (September 2015 in Paris, January 2016 in St. Andrews, June 2016 in Bremen, and July 2016 in Bia{\l}ystok, 2017 in Orsay, 2018 in Cernay, 2019 in Cernay), the participants working on \WPref{dksbases} met and discussed the topic of integrating the \pn systems into a mathematical VRE toolkit.
Additionally, Florian Rabe was employed at both FAU and UPSud throughout 2018 and 2019 to deepen the integration.

Key results of the first two reporting periods were
\begin{compactitem}[\bf R1.]
\item the observation that \emph{knowledge-aware interoperability of software and database-systems is the most critical objective} for \WPref{dksbases} in the \pn project.
\item the consensus that this can be achieved by \emph{aligning the mathematical knowledge underlying the various systems},
\item the existing integration of mathematical computation systems in the Sage and Jupyter systems must be complemented with a similar integration of mathematical databases.
\end{compactitem}
This requires explicitly representing the three aspects of math VREs -- Data (D), Knowledge (K), and Software (S) -- and basing computational services and inter-system communication on a joint \DKS-base.
These results are engrained in the ``Math-in-the-Middle'' (MitM) paradigm~\cite{DehKohKon:iop16}, which gives a representational basis for specification-based interoperability of mathematical software systems -- so that they can be integrated in a VRE toolkit.
In the MitM paradigm, the mathematical knowledge underlying the VREs (K) and the interface for each system (S) are represented as modular theory graphs in the OMDoc/MMT format.
For the data aspect (D) we have extended the concept of OMDoc/MMT theories to ``virtual theories'' that allow the practical management of possibly infinite theories, see~\cite{ODK-D6.5} for details.

Through the concerted effort of the WP6 participants, we have been able to implement this design and instantiate it with generate theory graphs for the \GAP and \Sage systems and integrating the \LMFDB (see~\cite{ODK-D6.5}.
Based on this, we were able to generically integrate \GAP, \Sage, and \LMFDB via the standardised SCSCP protocol~\cite{HorRoz:ossp09}. This case study shows the feasibility of the design. 

\begin{wrapfigure}r{6cm}\vspace*{-1em}
\documentclass{standalone}
\usepackage{tikzinput}
\begin{document}
\providecommand\myscale{4.5}
\begin{tikzpicture}[scale=\myscale]
  \node (center) at (0,.15) {Organization};
  \node (left) at (.2,-.3) {Computation};
  \node (right) at (.4,0) {Tabulation};
  \node (back) at (-.5,0) {Inference};
  \node (up) at (0,.5) {Narration};

  \draw[very thick] (center) -- (left);
  \draw[very thick] (center) -- (right);
  \draw[very thick] (center) -- (back);
  \draw[very thick] (center) -- (up);
  \draw[dotted] (left) -- (right) -- (back) -- (left);
  \draw[dotted] (up) -- (left);
  \draw[dotted] (up) -- (right);
  \draw[dotted] (up) -- (back);
\end{tikzpicture}
\end{document}
%%% Local Variables: 
%%% mode: latex
%%% TeX-master: t
%%% End: 
\vspace*{.5em}
\caption{Five Aspects of Math VREs, a Tetrapod Structure}\label{fig:tetrapod}\vspace*{-1.5em}
\end{wrapfigure}
In the \textbf{third reporting period}, the focus was on the \textbf{representation and curation of mathematical data}, building on the earlier work. We have refined the original notion of \DKS-bases from the grant proposal into a tetrapodal structure which captures the four primary aspects of ``doing Maths'' that have to be supported in a VRE toolkit: narration (papers and textbooks), computation (algorithms and software), inference (theorems and proofs), and tabulation (database schemas and datasets).
These are joined via a fifth modular organization aspect -- see the introduction  of \cite{ODK-D6.10} and \cite{CarFarKohRab:bmobb19} for a discussion.

We have taken up the general discussion of research data, the FAIR principles, and have adapted them to the case of mathematical research data. The outcome of this was the observation that -- even though mathematics deals with ideal and abstract objects -- it is often possible to describe these objects concisely by representing them as database structures in a way that complements they their formal symbolic descriptions. The codecs from the ``virtual theories'' approach developed in \WPref{dksbases} tie these two representations together: they link the database level of mathematical data sets with the MitM ontology -- and from there via the interface theory graphs interface it to the mathematical software systems in \pn.

We undertook three larger case studies to bring this about:
\begin{itemize}
\item developing the system data.mathhub.info for managing mathematical data sets MitM-style and equipping it with a search UI; see the report on \taskref{dksbases}{data-LMFDB} below,
\item exporting a the Isabelle knowledge base (via a subcontract), and equipping it with a semantic search facility; see the report on \taskref{dksbases}{isabelle} below,
\item extending the formula search capabilities developed in the first reporting period to Jupyter notebooks; see the report on
 \taskref{dksbases}{mws} below.
\end{itemize}
This wraps up and integrates the work in \WPref{dksbases} combining aspects of Data (D) (now captured by tabulation), Knowledge (K) (now captured by organization and inference), and Software (S) (now captured by computation).
Importantly, the joint system addresses semantically the central aspects of all four FAIR requirements for the open sharing of research data. 
For a joint and integrated final report on this, see ~\cite{ODK-D6.10}.


%%%%%%%%%%%%%%%%%%%%%%%%%%%%%%%%%%%%%%%%%%%%%%%%%%%%%%%%%%%%%%%%%%%%%%%%%%%%%% 
\paragraph{Milestones}

% month 36
\subparagraph{\longmilestoneref{dksbases-interop1}}
This milestone was addressed in the second reporting period.
\medskip
%\emph{“User story: thanks to a fully functional prototype integrating of at least the systems \GAP, \Sage, \Singular, and \LMFDB via the \SCSCP Protocol, end users shall be able to run calculations involving any combination of those systems from any of them.
%  This prototype will be the basis for integration work for additional systems and the user interface from WP4.”}
%\medskip
%
%Workpackage \textbf{WP6} is fully on track with this milestone.
%After first integration and DKS prototypes (the MitM VRE middleware  framework) became available in late fall 2017 (see~\cite{KohMuePfe:kbimss17,WieKohRab:vtuimkb17}) we were able to develop more sophisticated -- and mathematically more realistic/relevant -- use cases~\cite{CreLow:mdcmds18} and generalize those parts of the framework that had been overly specific to the first use cases.
%This involved non-trivial investments in all parts of the framework, as well as the system API theory generation systems and (in particular) the MitM ontology. 

% month 42
\subparagraph{\longmilestoneref{dksbases-interop2}}
\emph{“The goal of this milestone was to take into account all the operational experiences with the first prototype and add more systems and integrate some of the UI components from WP4.
  The experiences with the preparation of this prototype allow us to estimate the joining costs of adding a system to the OpenDreamKit VRE toolkit, which is an important measure of the flexibility of the Math-In-the-Middle approach.”}

The state of the MitM VRE middleware is sufficiently mature that most of the functionality can be configured by writing domain and system knowledge in form of OMDoc/MMT theories, but without requiring extensions of the system (e.g., changes to the programming of the VRE systems or the MMT mediator).
This means that additional computational systems can be added at the cost of generating system API theories, extending the MitM ontology, and supplying alignments.

In the \emph{third review period} we have concentrated on mathematical data sources.
Our analysis of systems in \taskref{dksbases}{data-OEIS}, \taskref{dksbases}{data-findstat}, and even more \taskref{dksbases}{data-LMFDB} has revealed that to achieve deep and meaningful FAIRness of mathematical research data -- in particular of interoperability (I) -- we have to semantically model the mathematical objects in math-aware representations (as described above).
As a consequence, we have to integrate mathematical data systematically into the \pn VRE toolkit, giving rise to the refined tetrapodal model described above. In particular, this integration must consider individual datasets rather than dataset-related systems like OEIS, FindStat, or LMFDB.
The \dmh system was expressly designed to do this: we can just specify a dataset in the MDDL language (see~\cite{BerKohRab:tumdi19,ODK-D6.10}) and then generate a database infrastructure including user interface and import facility from it, resulting in a MitM-Interoperable VRE component for that dataset.
We have tested and evaluated this setup on a special Math Data Workshop~\cite{ODK-WDM19} using five data sets supplied by one internal and two external mathematicians.
This showed that -- after a learning period of a day and with some help from \pn knowledge engineers, external users can MitM-integrate datasets by specifying their structure and semantics and supplying them in a specification-conforming format in under a day of work.
We anticipate that creating the alignments (see \cite{ODK-D6.5}) that relate the mathematical background of these datasets to algorithms in computational MitM systems like \GAP, \Sage, and \Singular is of the same order of magnitude. 


%%%%%%%%%%%%%%%%%%%%%%%%%%%%%%%%%%%%%%%%%%%%%%%%%%%%%%%%%%%%%%%%%%%%%%%%%%%%%% 
\paragraph{Tasks}
\medskip

\subparagraph{\longtaskref{dksbases}{data-assessment}}
\label{dksbases@data-assessment}
This task was addressed in the first reporting period.
\medskip

\subparagraph{\longtaskref{dksbases}{data-triform}}
\label{dksbases@data-triform}
This task was addressed in the first reporting period.
%For this task we have specified and implemented the concept of virtual theories that can contain large -- theoretically even infinite -- numbers of declarations and objects (e.g. 3.5M declarations in the LMFDB data base for elliptic functions) in OMDoc/MMT.
%Virtual theories are characterized by the fact that they are too large to keep in main memory of the MMT System and have to be partially and lazily imported from an external data store.
%We have reported on the design in \longdelivref{dksbases}{design}, on a first implementation on the international conference (MACIS 2017)~\cite{WieKohRab:vtuimkb17}, and finally on an extended use-case in \LMFDB in \longdelivref{dksbases}{psfoundation}. 
\medskip

\subparagraph{\longtaskref{dksbases}{data-design}}
\label{dksbases@data-design}
This task was addressed in the first reporting period.
%This task was directly addressed in the \WPref{dksbases} workshops in the first year and has led to the design and implementation in \delivref{dksbases}{design}. A first implementation has been presented on the international conference (MACIS 2017)~\cite{WieKohRab:vtuimkb17}, and finally on an extended use-case in \LMFDB in \longdelivref{dksbases}{psfoundation}.
% \medskip

\subparagraph{\longtaskref{dksbases}{data-foundationCAS}}
\label{dksbases@data-foundationCAS}
This task was addressed in the first reporting period. 
%In the course of the deliberations in the \WPref{dksbases} workshops we saw a shift from the development of computational foundations and verification towards API/Interface function specifications to enable semantic system interoperability via the Math-in-the-Middle (MitM) Ontology.
%Consequently, emphasis has changed to the generation of system API theories for \GAP, \Sage, \Singular, and \LMFDB, which act as OpenMath content dictionaries.
%The computational foundations exist but are rather more simple than originally anticipated.
%Much of the functionality has been offloaded to the SCSCP standard -- remote procedure call with OpenMath representations of the mathematical objects -- developed in the SCIENCE Project.
%As a direct consequence of the work in \pn the OpenMath Society has promoted the \SCSCP protocol into as an OpenMath Standard.
%
%Conversely, the \GAP and \Sage CDs are rather more elaborated than anticipated in the proposal, and thus form a viable basis for alignment with the MitM Ontology.
%
%The MitM integration paradigm is the result of our research and development on the computer algebra foundations in this task has been presented on the international conference MACIS 2017~\cite{KohMuePfe:kbimss17} and is described in deliverable \longdelivref{dksbases}{psfoundation}, which presents an advanced CAS integration use case. 
%The MitM ontology and the system API theories have been developed to the point, where the data model is fully developed and the contents cover the use cases corresponding to this task and \longlocaltaskref{dksbases}{data-design} are surveyed in \longdelivref{dksbases}{lfmverif}.
\medskip

\subparagraph{\longtaskref{dksbases}{research-categories}}
\label{dksbases@data-research-categories}
This task was addressed in the second reporting period. In this period, the infrastructure has matured and been extended with a data component \dmh.

\subparagraph{\longtaskref{dksbases}{data-OEIS}}
\label{dksbases@data-OEIS}
This task was addressed in the first reporting period.
%For the OEIS case study we have parsed the OEIS data and converted it into OMDoc/MMT theories (ca. 260,000).
%The main problem solved here was to parse the formula section (generating functions, relations between sequences, \ldots): they are represented in a human-oriented ASCII syntax, which is highly irregular, ill-separated from surrounding text, and interpunctuation.
%Nonetheless we managed to recover ca. 90\% of the formulae and
%\begin{compactenum}[\em i\rm)]
%\item generate ca. 100,000 new relations between sequences and
%\item provide a package of ca. 50,000 generating functions to Sage (which can be used
%  e.g. in the FindStat database).
%\end{compactenum}
%We use this theory set to test the functionalities of ``virtual theory graphs'' (one step up from the ``virtual theories'' developed in \localtaskref{dksbases}{data-design}).
\medskip

\subparagraph{\longtaskref{dksbases}{data-findstat}}
\label{dksbases@data-findstat}
This task was addressed in the second reporting period.
%We have seen that the \LMFDB already shows all the complexities needed to develop full-coverage DKS functionality for the \pn VRE toolkit.
%On the other hand our survey shows that our DKS design (OMDoc/MMT virtual theories) is sufficient for covering the FindStat use case as well.
%Therefore we have delayed this taks to the last year of the \pn project, when the system API theories for \Sage and OEIS have matured. With the declarative design of the virtual theories, task \localtaskref{dksbases}{data-findstat} becomes a matter of writing down the schema theories system API theories for FindStat and defining the requisite codecs. We expect this to be a matter of one of two weeks of joint development of the FAU team together with UPSud. 
\medskip

\subparagraph{\longtaskref{dksbases}{data-LMFDB}}
\label{dksbases@data-LMFDB}
Work on this task had already started in the second report period. There we had used the concept of virtual theories developed in \localtaskref{dksbases}{data-triform} to MitM-integrate (parts of) the LMFDB and make it interoperable as a VRE component.
During this process, it became apparent that to achieve meaningful interoperability (the I in FAIR), we have to model the mathematical datasets as described above.
This prompted us to refocus on an integration into the \pn VRE toolkit at the dataset level, not at the level of systems like OEIS, FindStat, or LMFDB.
Indeed, LMFDB is itself a collection of over 80 datasets and the work on the semantics and representation of datasets catalyzed a large inventory of datasets in LMFDB.
At the start of the \pn project, LMFDB had been growing organically based on the schema-less MongoDB database, and no-one had an complete overview on the details and extent of the content.
The Warwick group led a move to inventory all the data sets, and to (manually) recover their specifications at the mathematical and data base level (schema information), which has in turn facilitated the recent move from MongoDB to PostGreSQL (independent of \pn).
Another fruit of the \pn work was an vastly improved and more semantic API for LMFDB (see \url{http://www.lmfdb.org/api2/}) that has recently come online.
LMFDB’s earlier API was just a very thin HTTP wrapper over the database core of LMFDB.
API2 adds full SQL querying support and first steps towards ``semantic/mathematic'' queries.
A \Sage interface based on API2 is currently under development.

While the LMFDB has ``retrofitted'' a more semantic treatment of datasets on LMFDB, the FAU group has explored building a dataset hosting system and user interface based on the MitM model from scratch.
The \dmh system leverages the MitM ontology for specifying the mathematical objects in a dataset and extends MitM with a set of MDDL (Math Dataset Description Language) specifications that link the mathematical specifications with database schemata.
The MDDL specifications are implemented in a set of codecs that implement the transformations between the layers, encapsulate the translations of mathematical queries to SQL queries in the underlying database generated from MDDL, and provide user interface widgets for the corresponding mathematical types.
We have tested this setup on five external data sets -- we did not want to duplicate work with LMFDB -- and so far the \dmh design seems to scale.
When the set of codecs collected in \dmh and made available for reuse by other datasets reaches a point of saturation, we expect joining the costs for \dmh to become restricted to the dataset-immanent information.
\medskip

\subparagraph{\longtaskref{dksbases}{data-memo}}
\label{dksbases@data-memo}
We have developed persistent memoization modules for Sage and Gap that can use both local and remote data stores.
Both use the same format so they can share the same data stores.\ednote{to be finished by sites US,PS,UW}

We report on this task in detail in \delivref{dksbases}{persistent-memoization}.
\medskip

\subparagraph{\longtaskref{dksbases}{mws}}
This task runs over the whole length of the \pn project. The third reporting period with its refined model of the semantic level of mathematical VRE components has given us a new view on the \taskref{dksbases}{mws} as well.

Generally, search is one of the FAIR principles of research data (F), and adapting it to mathematical data/knowledge/software has been one of the central topics in \WPref{dksbases} (and \WPref{UI}). In the grant proposal we had concentrated on formula- and full-text search, but our deeper understanding of the categories of mathematical data (see ~\cite{ODK-D6.10}) developed during the \pn project showed that for mathematical semantic search we need to take the kind of data into account better.
\begin{compactitem}
\item For \textbf{symbolic data} (organization and inference in Figure~\ref{fig:tetrapod}), formula search as reported on in \cite{ODK-D6.1} is sufficient as long as the context of all formulas is included in the search index. For this, we have pioneered an export of the knowledge base of the Isabelle system (and others) into symbolic OMDoc formulae and RDF triples that they can be searched via SPARQL queries. See our description of the new \taskref{dksbases}{isabelle} below for details.
\item In \textbf{narrative data}, we need full-text search capabilities; this has been addressed in the first reporting period -- see \cite{ODK-D6.1} again.
\item For \textbf{tabulated data}, we need a mathematical query language in which users can express information needs at the mathematical level, and which can be compiled into e.g. SQL queries at the database level.
  This was initiated in the second reporting period.
   In the third reporting period, it has greatly matured, been integrated into the new \dmh system, and extended by a user interface generation system.
\item For \textbf{computational data}, i.e. mathematical software, there are two options: we can search source code, or we can search the mathematical artefacts in the programs.
  As source code search is already provided by repository hosting systems like GitHub or GitLab we have concentrated on the latter.
  Concretely, we built a system to harvest mathematical formulae from Jupyter notebooks, index them in the MathWebSearch engine, and provide a specialized user interface for searching them.
  The main technical development has been to make the MathWebSearch engine -- which had been mostly experimental -- more deployable and manageable so that it can be used as a VRE component and so that it can be integrated into the \pn VRE toolkit without developing instance-specific code. 
\end{compactitem}
We have reported on all the aspects of this task in detail in \cite{ODK-D6.10}.
\medskip

\subparagraph{\longtaskref{dksbases}{isabelle}}
For many decades, the development of a universal database of all mathematical knowledge, as envisioned, e.g., in the QED manifesto \cite{qed}, has been a major driving force of computer mathematics.
Today a variety of such libraries are available.
These are most prominently developed in proof assistants such as Coq \cite{coq} or Isabelle \cite{isabelle} and are treasure troves of detailed mathematical knowledge.
Within \pn, we have developed interface standards, specifically OMDoc for symbolic and ULO for relational knowledge, that allow maintainers of formal libraries to make their content available to outside systems.

In this task (which has been added in the last amendment of the grant agreement), we have exported the large Isabelle knowledge base as both OMDoc/MMT and ULO format
Concretely, we have built an exporter from the Isabelle Theorem prover library (Archive of Formal Proof) to both MMT and RDF data.
This exporter is now part of the latest releases of both Isabelle and MMT, and the exported data is available online.

We report on this task in detail in \delivref{dksbases}{nbad-search}.


%%% Local Variables:
%%% mode: latex 
%%% mode: visual-line
%%% fill-column: 5000
%%% TeX-master: "report"
%%% End:

%  LocalWords:  subsubsection dksbases ystok WPref compactitem emph DehKohKon:iop16 textbf taskref longdelivref lfmverif triformal formalized biformal HorRoz:ossp09 medskip longmilestoneref dksbases-interop1 dksbases-interop2 characterized WieKohRab:vtuimkb17 psfoundation delivref KohMuePfe:kbimss17 regularized synchronized ldots interpunctuation compactenum mws KohMuePfe:kbimss17,WieKohRab:vtuimkb17 CreLow:mdcmds18 jupyter-import Jupyter MitM-based Jupyter Cernay Cernay tetrapodal organization CarFarKohRab:bmobb19 losslessly dmh BerKohRab:tumdi19,ODK-D6.10 MitM-integrate oldpart dksbases@data-findstat ednote nbad-search newpart specialized catalyzed


\section{Risk management}
\subsection{Recruitment of highly qualified staff}
Recruitment of highly qualified staff was planned to be a high risk
when the Proposal was written. And unfortunately it turned out we were
right. In such a field as computer science and software development,
potential candidates who are likely to be fairly young considering
only temporary positions are offered, are very scarce. Furthermore
they need to make a choice between public and private bodies which are
very attractive, and the choice between pure development and research.
Because of this difficulty to recruit in the past year, there have
been slight changes in the workplan, which do
not put the project results at risk.

The following people were hired in the past year:\\


\begin{tabular}{|l|c|r|r|r|}
\hline
NAME&GENDER&PARTNER&POSITION&HIRING DATE\\
\hline
Benoît PILORGET&M&\site{PS}&Project manager&17-09-2015\\
Jeroen DEMEYER&M&\site{PS}&Research engineer&01-03-2016\\
Erik BRAY&M&\site{PS}&Research engineer&01-01-2016\\
Christian MAEDER&M&JacobsUni&Senior researcher&01-01-2016\\
Tom WIESING&M&JacobsUni&Junior researcher&01-09-2015\\
Xu HE&M&JacobsUni&Junior Researcher&01-09-2015\\
Alexander BEST&M&UNIKL&Research engineer&01-02-2016\\
Anders JENSEN&M&UNIKL&Postdoc&01-11-2015\\
Alexander KRUPPA&M&UNIKL&Postdoc&01-08-2016\\
Jan AKSAMIT&M&USlaski&Technical staff&01-10-2015\\
Marijan BEG&M&Southampton&Research fellow&01-05-2016\\
B. RAGAN-KELLEY&M&Simula&Postdoc&01-09-2015\\
V.T. FAUSKE&M&Simula&Postdoc fellow&02-05-2016\\
\hline
\end{tabular}\\
~\\
 \ODK partners had to face some Human Resources issues in the past year:
\begin{itemize}
\item{\site{PS}:}
  Thanks to an early start in the recruitment process, and despite
  some difficulties in attracting experienced candidates for a part
  time position, the project manager position (24PM) was filled by
  Benoît Pilorget shortly after the start of the project. Unfortunately at month 36 the project
finds itself without a Project manager since the departure of B. Pilorget.

  The recruitment of \site{PS}'s first Research Engineer (48PM) was
  delayed by four months because the top ranked candidate for this
  position, Erik Bray, was originating from the US and needed time to
  arrange for his moving; there were also some administrative delays
  (visa, ...).

  The second Research Engineer position (36PM) was more problematic
  for internal administrative reasons. The top ranked candidate,
  Jeroen Demeyer, had the perfect profile; however for family reasons,
  he wished to work most of the time from Ghent in Belgium. After eight
  months investigating an administrative solution to hire him at
  \site{PS}, and a temporary four month solution, it was decided with
  OpenDreamKit's Steering Committee and Project Officer to instead add
  Ghent's university as new partner, hire Jeroen Demeyer there, with an
  adequate budget transfer and amendment to the Grant Agreement.

  Those delays have induced late start on several tasks, and costed
  much management time. However the excellence of the recruitment, well
  confirmed by the results obtained so far, was worth it and will soon
  compensate for the late start.

  In addition to this, a three year PhD position was open to work on
  WP6, starting from Month 12. By lack of suitable candidate, this
  position will be converted into a two year PostDoc position,
  presumably starting at Month 24. Active advertising has started and
  there are some tentative candidates. The relevant deliverables being
  due late in
  the project, no delay is to be expected from this change.\\

\item{CNRS:} Because the research engineer offer (48PM) was still not
  filled in the Summer 2016, the CNRS decided to divide the position
  in two full positions of 24 PM each. As a result, a candidate was
  already selected for one of the two positions and should begin his
  work this Fall 2016.  Thanks to the PM division, there should be no
  delay in any task or
  deliverable. \\

\item{JacobsUni:} Michael Kohlase, lead PI for Jacobs University, has
  moved on 01/09/2016 to Friedrich-Alexander-Universität
  Erlangen-Nürnberg, and most of his team will follow him. Since he is
  a critical asset for OpenDreamKit, a Grant Agreement amendment will
  be submitted in Winter 2016-2017 to update the consortium accordingly.\\

\item{UJF:} The original tentative candidate for UJF's Research
  Engineer position (12PM, planned to start on Month 1), Pierrick
  Brunet, finally declined the position to accept an alternative
  permanent offer. The position will be filled by another candidate in
  Autumn 2016. This induced a delay of Deliverable \textbf{D5.2} from Month 12
  to Month 18, without impact on other tasks.\\

\item{UNIKL:} UNIKL had to split the 12 PM planned for a software developer into 2 shorter positions (Anders Jensen and Alexander Kruppa) in order to deliver the planned work on time. Indeed the few qualified persons for this job were not able to accept this 12 months position during the timelapse planned within the project.\\

\item{USFD:} The University of Sheffield has also been struggling in the the hiring process of a postdoc (36PM). The position should be filled this Autumn.\\

\item{Southampton:} Southampton faced administrative difficulties in the recruitment of Marijan Beg (38PM) as a post-doc, due to the Croatian nationality of Mr Beg. His recruitment was delayed of four months, and therefore some tasks and deliverables, planned to be borne around the end of the project, were postponed of four months. However no serious delay nor implication on the main tasks of OpenDreamKit followed these difficulties.\\

\item{UVSQ:} Nicolas Gama is currently on a long-term leave until September 2017. This will not affect the project in any way.\\

\item{UZH:} The University of Zürich partner is only composed of one person, Paul-Olivier Dehaye, who does not enjoy a permanent position there. There have been worries that Mr Dehaye's contract with his university might end earlier than planned within OpenDreamKit. But thanks to the action of the OpenDreamKit steering committee, Mr Dehaye has been technically rehired by UZH as a scientific consultant for as long as the project needs.\\

\item{Simula:} Everything is fine concerning temporary staff
  recruitment on the Simula side, however we have had to endure the
  hazards of human ressources with Hans-Peter Langtanger (the PI when
  the Grant was signed) being on a long-term sick leave, and with
  Martin Alnaes replacing him as PI currently on a paternity
  leave. However Benjamin Ragan-Kelley has stepped in to lead the
  Simula contribution in the meantime and all planned tasks are on time.\\
\end{itemize}


Altogether, this first year confirmed that the recruitment of highly
qualified staff is indeed a risky endeavour, which induced delays on
several deliverables. However the planned mitigation measures --
taking into account the pool of potential candidates in the design of
the positions, aggressive advertisement, weak coupling between tasks
-- worked adequately: with appropriate reshuffling of the work plan,
we don't expect an impact on the overall progress of the project.

\subsection{Different groups not forming effective team}

As expected, this risk was tamed by the existence of many preexisting
collaborations between the partners and of ``joint itches to scratch
together'' (to use a common open source software metaphor). The
organization of many joint workshops (for example the Sage-GAP
workshop, the Atelier Pari attended by SageMath developers, the WP6
workshops) helped bootstrap joint activities through brainstorms and
coding sprints. Upcoming workshops are planned on Year~2 to strengthen
collaborations with the social aspects team in Oxford and the Singular
team in Kaiserslautern.


\subsection{Implementing infrastructure that does not match the needs of end-users}

The consortium is keeping in their minds the end-user needs. Since
OpenDreamKit is improving already existent software which have their
own users, their needs are naturally met. However Key performance
Indicators will evaluate the effects of OpenDreamKit on these
software. KPIs, indicated in the Proposal, will be launched this
Autumn with the help of the end-user group which was merged with the
Advisory Board. Constant links between the accomplished work and the
end-user needs should be made in WP2 deliverables and also in WP7
deliverables when relevant.  Open tracking of KPIs evolution can be
found on
\href{https://github.com/OpenDreamKit/OpenDreamKit/labels/KPI}{GitHub}.

\subsection{Lack of predictability for tasks that are pursued jointly
  with the community}

As planned, we are regularly shifting manpower around to adapt for the
variability of the involvement of the community in the different
tasks. For example, the SageMath Jupyter kernel of
\longdelivref{UI}{ipython-kernels-basic} was mostly implemented by the
community which allowed to focus on other tasks such as the long term
task~\longdelivref{component-architecture}{portability-cygwin}.  On
the other hand many other deliverables were implemented with very
little help from the community.

\subsection{Reliance on external software components}

There is not much to report on this front yet: none of the external
software component we rely on have failed us. Quite on the contrary,
critical software like \Jupyter have continued to blossom. Besides the
high modularity of the design means few components are critical to the
overall success of the project.

\section{Quality assurance plan}

\subsection{Deliverables quality: Quality Review Board}

The Quality Review Board is the Consortium Body that fosters best
possible quality in the delivered work of the project.
All four members of the board
have a research interest in the quality of software in computational
science, and use and share their experience to benefit the quality of
the work.

The board is chaired by Hans Fangohr, from the University of
Southampton and European XFEL GmbH. He is supported in this task by
Mike Croucher from the University of Leeds, Alexander Konovalov from
the University of St Andrews, and by Konrad Hinsen from the Centre de
Biophysique Moléculaire with whom a Non-Disclosure Agreement was
signed.

The members engage with European initiatives working towards
improvement of the software quality in research, in particular in
computational and data science; both as voluntary activities and key
of their professional roles. Mike Croucher is the head of research
computing at Leeds, well known through his outreach blog, Alexander Konovalov, an EPSRC
Research Software Engineering Fellow, is a driving
force in the Software Sustainability Institute, Konrad Hinsen has
founded and is editing the ReScience Journal for reproducible Science,
and Hans Fangohr is founder and director of the UK's only centre for
doctoral training in computational modelling, a fellow of the Software
Sustainability Institute, was chairing the EPSRC's national scientific
advisory committee on high performance computing, and is leading big
data analysis infrastructure development at the European XFEL research
facility.

The quality review board has reviewed deliverables at the end of
reporting period 1, identified good practice - both in terms of
software engineering content but also presentation of the work -,
produced a report, and shared the findings with all members in the
project to improve the quality of the remaining deliverables. A
summary is included in Sec.~\ref{sec:summ-recomm-deliv}. The
board has stuck to its no-blame culture in its reporting.

A similar process is underway for the end of reporting period 2.


\subsection{Good practice}

The quality review board notes that there is no firmly established
view on what best practice establishes, and that (i) we expect our
best practice checklist to grow and change, and (ii) that due to the
variety of possible outputs not all categories will be appropriate for
every item under review. Here is a summary of good practice, with
particular focus on software and computational projects.

\subsubsection{Software engineering}
\label{sec:org2e9824e}

Use of
\begin{itemize}
\item[{$\square$}] version control
\item[{$\square$}] tests
\item[{$\square$}] automated tests
\item[{$\square$}] continuous integration
\item[{$\square$}] automatic building of releases
\end{itemize}

\subsubsection{Dissemmination}
\label{sec:org1f65c9b}
\begin{itemize}
\item[{$\square$}] Host code publicly (Github, \ldots{})
\item[{$\square$}] Reference Manual (APIs)
\item[{$\square$}] Tutorial (for beginning users)
\item[{$\square$}] Examples
\item[{$\square$}] Offer live interactive online demos (for example
  through Binder)
\item[{$\square$}] Support mechanisms (email/forum/gitter/github issues/\ldots{})
\item[{$\square$}] How to cite the output?
\item[{$\square$}] Installation mechanism
\item[{$\square$}] High level description of tool/activity accessible to non-experts
\item[{$\square$}] URLs/Blog/etc to and from  OpenDreamKit project
\item[{$\square$}] Grant acknowledgements
\item[{$\square$}] Open Source license
\item[{$\square$}] Workshop
\item[{$\square$}] Engaging users
\end{itemize}

\subsubsection{Pathways to impact}
\label{sec:orgc218a3a}
\begin{itemize}
\item[{$\square$}] Does the software address the need\TODO{s?} of the users?
\item[{$\square$}] Workshops to gather feedback
\end{itemize}

\subsection{Summary of recommendations for deliverable reports}
\label{sec:summ-recomm-deliv}
For reports that are well written, the quality review board found good
software engineering practices. However, for some deliverables the
reports were more difficult to assess. To address this, the following
guidance has been developed:

\begin{itemize}
\item Context setting
  \begin{itemize}
  \item what is the problem that is being addressed?
  \item Why should we care about the deliverable?
  \item what is the thing that has been created?
  \end{itemize}

\item Stating the obvious: for example if people (outside the project)
  are excited about it

\item Provide introduction to topics, even for people not too familiar
  with the field
\item Comment on other good (software) practices you may use without
  thinking about it (version control, testing, continuous integration,
  distribution)

\item Comment on the testing that has been done, even if not
  automatic. If there is a prototype / demonstrator, explain it in
  more detail.
\item Work relating to sustainability, should be mentioned, even if
  implicit (for example growing a community through workshops will
  help to make the project more sustainable).

\item Anything with impact should be mentioned (contribution / uptake
  to Software carpentry, other computational science and software
  projects, github, users, . . . )

\item Have an executive summary on the
  first page (just half a page).
  \begin{itemize}
  \item why does this deliverable exist? (context)
  \item have you achieved everything you set out to do?
  \item what would be / are the next steps?
  \end{itemize}

\end{itemize}

%\ODK consortium about the deliverables content and layout.

% While the primary focus of the board is on the OpenDreamKit project
% and the software it develops,
% some of the lessons may be more widely applicable and be made publicly
% available.





% % original

% The content form of deliverables due by then had to meet the
% expectations of the Project Officer and of Reviewers. Following this
% experience, we have concluded that deliverables should be written in
% Latex using a style file created for this purpose. For deliverables
% that are not reports by themselves, it's appropriate to have a
% relatively short report with a link to the github issue, and a copy of
% the description of this issue. In all cases, the report shall be
% self-contained. Deliverables are indeed evaluated based upon their
% versions submitted on the EU portal without retrieving other
% resources. Links have no legal value, since there is no guarantee that
% the referenced material will remain unchanged.  Partners who have a
% deliverable due at month 12 (August 2016) have been following these
% tips. The feedback of the Official review will help the Quality review
% Board in ensuring the quality of reports meets the needs. This will be
% up to the Quality Review Board to meet after the 1st reporting period
% and to decide if the quality of deliverables is acceptable.  They will
% aim at identifying good practice and weaknesses, and to share the
% lessons with the project to improve any future project work. The board
% will focus on selected deliverables and investigate those in detail
% rather than attempting a superficial inspection of all deliverables.


\subsection{Infrastructure quality: End-user group}


It was decided by the Steering Committee during the
\href{http://opendreamkit.org/meetings/2015-09-02-Kickoff/management_structure/}{kick-off
  meeting} to slightly modify the management structure by having only
one gender-friendly Advisory Board composed of 6 people (as agreed a
few months later at the
\href{http://opendreamkit.org/meetings/2016-06-27-Bremen/minutes/}{Bremen
  meeting}), some of which to be end-users.

Members of the board are: Jacques Carette from the McMaster University, Istvan Csabai from the Eötvös University Budapest,
Françoise Genova from the Observatoire de Strasbourg, Konrad Hinsen from the Centre de Biophysique Moléculaire,
William Stein who is CEO of SageMath, Inc. (SME), and Paul Zimmermann from INRIA.

\printbibliography

\end{document}

%%% Local Variables:
%%% mode: latex
%%% TeX-master: t
%%% End:

%  LocalWords:  maketitle githubissuedescription newpage newcommand xspace Jupyter dissem
%  LocalWords:  tableofcontents visualizations composability itemize analyzed taskref hpc
%  LocalWords:  dissemination-of-oommf-nb-virtual-environment taskref dissem taskref pn
%  LocalWords:  dissemination-of-oommf-nb-workshops dissem ibook taskref taskref taskref
%  LocalWords:  oommf-python-interface oommf-py-ipython-attributes taskref oommf-nb-ve
%  LocalWords:  oommf-tutorial-and-documentation taskref oommf-nb-evaluation taskrefs
%  LocalWords:  delivref pythran-typing sage-paral-tree subsubsection organized Dagstuhl
%  LocalWords:  co-organized organization modularization ipython-kernels nbdime Pythran
%  LocalWords:  jupyter-collab ystok WPref dksbases compactitem emph WPtref DehKohKon
%  LocalWords:  iop16 textbf tasktref lfmverif triformal formalized biformal ossp09 Dima
%  LocalWords:  hline Marijan Pilorget Pierrick Kruppa Dehaye Dehaye's Dehaye's Alnaes
%  LocalWords:  Konovalov Hinsen github printbibliography
