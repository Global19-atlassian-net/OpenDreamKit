\subsection{Objectives}
%List the specific  objectives  for  the  project  as  described  in  section  1.1  of  Part B   and describe  the  work  carried  out  during  the  reporting  period  towards  the  achievement  of  each listed objective. Provide clear and measurable details.

\TODO{Update for Reporting Period 2}

For reference, let us recall the aims of \ODK.
\begin{compactenum}[\textbf{Aim} 1:]
\item \label{aim:collaboration} Improve the productivity of
  researchers in pure mathematics and applications by promoting
  collaborations based on mathematical \textbf{software},
  \textbf{data}, and \textbf{knowledge}.
\item \label{aim:vre} Make it easy for teams of researchers of any
  size to set up custom, collaborative \emph{Virtual Research
    Environments} tailored to their specific needs, resources and
  workflows. The \VREs should support the entire life-cycle of
  computational work in mathematical research, from initial
  exploration to publication, teaching and outreach.
  % and bridge the gaps between
  % code, published results, and educational material.
\item \label{aim:sharing} Identify and promote best practices in
  computational mathematical research including: making results easily
  reproducible; producing reusable and easily accessible
  software; sharing data in a semantically sound way; exploiting and
  supporting the growing ecosystem of computational tools.
\item \label{aim:impact} Maximise sustainability and impact in
  mathematics, neighbouring fields, and scientific computing.
\end{compactenum}

Those aims are backed up in our proposal by nine objectives; we now
highlight our main contributions during this reporting period toward
achieving each of them.

\begin{compactenum}[\textbf{Objective} 1:]
\item\label{objective:framework} ``To develop and standardise an
  architecture allowing combination of mathematical, data and software
  components with off-the-shelf computing infrastructure to produce
  specialised \VREs for different communities.''

  This objective is by nature multilevel; achievements include:
  \begin{itemize}
  \item Collaborative workspaces: major \JupyterHub developments,
    see~\longtaskref{UI}{notebook-collab}; study and documentation of
    the \SMC architecture, see
    \longtaskref{component-architecture}{extract-smc};
  \item User interface level: enabling \Jupyter as uniform interface
    for all computational components; see
    \localtaskref{UI}{ipython-kernels}.
  \item Interfaces between computational or database components: short
    term: refactoring of existing ad-hoc interfaces, see
    \localtaskref{UI}{pari-python}; long term: investigation of
    patterns to share data, ontologies, and semantics uniformly across
    components, see
    \localtaskref{component-architecture}{interface-systems}, and
    Section~\ref{dksbases} about \WPref{dksbases}.
  \end{itemize}

\item\label{objectives:core} ``To develop open source core components
  for \VREs where existing software is not suitable. These components
  will support a variety of platforms, including standard cloud
  computing and clusters. This primarily addresses Aim~\ref{aim:vre},
  thereby contributing to Aim \ref{aim:collaboration}
  and~\ref{aim:sharing}.''

  At this stage, it has been possible to implement most of the
  required developments within existing components or extensions
  thereof. New software components includes the tools nbmerge, nbdiff and nbval
  (see \delivref{UI}{jupyter-test} and \delivref{UI}{jupyter-collab}),
  and planetaryum (see \delivref{dissem}{ils-tool}).

\item \label{objective:community} ``To bring together research
  communities (e.g. users of \Jupyter, \Sage, \Singular, and \GAP) to
  symbiotically exploit overlaps in tool creation building efforts,
  avoid duplication of effort in different disciplines, and share best
  practice. This supports Aims~\ref{aim:collaboration},
  \ref{aim:sharing} and~\ref{aim:impact}.''

  We have organized or coorganized a dozen users or developers
  workshops (see~\localtaskref{dissem}{devel-workshops}) which brought
  together several communities. Some key outcomes include:
  \begin{itemize}
  \item Enabling \Jupyter as uniform interface for all computational
    components; see \localtaskref{UI}{ipython-kernels}.
  \item Sharing best practices for packaging and building
    containers;
    see~\localtaskref{component-architecture}{mod-packaging};
  \item A smooth collaboration between \JupyterHub and \SMC;
    see~\localtaskref{component-architecture}{extract-smc} and
    Section~\ref{infrastructures};
  \item Work on interfaces between systems; see
    \localtaskref{component-architecture}{interface-systems} and
    \localtaskref{UI}{pari-python};
  %\item Steps toward \localtaskref{UI}{sage-sphinx}
  \item Sharing best practices when using VRE's like \SMC or \Jupyter
    for research and education.
  \end{itemize}

\item \label{objective:updates} ``Update a range of existing open source
  mathematical software systems for seamless deployment and efficient
  execution within the VRE architecture of objective~\ref{objective:framework}.
  This fulfills part of Aim~\ref{aim:vre}.''

  Achievements include:
  \begin{itemize}
  \item Packaging efforts: Docker containers (delivered and
    regularly updated), Debian and Conda packages (beta); see
    \localtaskref{component-architecture}{mod-packaging}.
  \item Portability of \Sage and its dependencies on Windows; see
    \localtaskref{component-architecture}{portability}, in particular
    \delivref{component-architecture}{portability-cygwin}.
  \item Integration of all the relevant mathematical software in the
    uniform \Jupyter user interface, in particular for integration in
    the VRE framework (delivered, ongoing); see
    \localtaskref{UI}{ipython-kernels}.
  \item Ongoing work in \WPref{hpc} to better support HPC in the
    individual mathematical software system and combinations thereof;
    see Section~\ref{hpc}.
  \end{itemize}

\item \label{objective:sustainable} ``Ensure that our ecosystem of
  interoperable open source components is \emph{sustainable} by
  promoting collaborative software development and outsourcing
  development to larger communities whenever suitable. This fulfills
  part of Aims~\ref{aim:sharing} and~\ref{aim:impact}.''

  Achievements include:
  \begin{itemize}
  \item Outsourcing of the \Sage user interface by migrating to
    \Jupyter (delivered); see \localtaskref{UI}{ipython-kernels};
  \item Refactoring \Sage's documentation build system to contribute
    many local developments upstream (\Sphinx)
    \localtaskref{UI}{sage-sphinx} (ongoing);
  \item Outsourcing and contributing upstream as \Python bindings the
    existing \Sage bindings for many computational systems (delivered,
    ongoing); see \taskref{UI}{pari-python}.
  \end{itemize}

\item \label{objective:social} ``Promote collaborative mathematics and
  science by exploring the social phenomena that underpin these
  endeavours: how do researchers collaborate in Mathematics and
  Computational Sciences?  What can be the role of \VREs?  How can
  collaborators within a VRE be credited and incentivised? This
  addresses parts of Aims~\ref{aim:sharing}, \ref{aim:collaboration},
  and~\ref{aim:vre}.''

  This objective is at the core of \WPref{social-aspects}; see
  Section~\ref{social-aspects} for details.
  Achievements include:
  \begin{itemize}
  \item Methodology, data, and tools needed to assess
    development models of academic open-source systems have been surveyed in
    \delivref{social-aspects}{social-datareport}.
  \item Game-theoretic aspects of ways to incentivise
  collaboration are studied---ongoing, also see \cite{Pavlou:2016:MCI:2936924.2936934}.
  \item Implementing in \Sage constructions from the database of
  strongly regular graphs by Andries Brouwer, see paper \cite{2016arXiv160100181C}.
  \end{itemize}


\item \label{objective:data} ``Identify and extend ontologies and
  standards to facilitate safe and efficient storage, reuse,
  interoperation and sharing of rich mathematical data whilst taking
  account of provenance and citability. This fulfills parts of
  Aims~\ref{aim:vre} and~\ref{aim:sharing}.''

  This objective is at the core of \WPref{dksbases}; see
  Section~\ref{dksbases} for details. 

%\TODO{Michael: flesh up}

\item \label{objective:demo} ``Demonstrate the effectiveness of Virtual
  Research Environments built on top of \ODK components for a
  number of real-world use cases that traverse domains. This addresses
  part of Aim~\ref{aim:vre} and through documenting best practices in
  reproducible demonstrator documents Aim~\ref{aim:sharing}.''

  Most of the work toward this objective is by nature planned for
  later in the project execution. Nevertheless, work has started e.g.
  toward the OOMMF demonstrator; see
  \localtaskref{dissem}{dissemination-of-oommf-nb-virtual-environment}
  \localtaskref{dissem}{dissemination-of-oommf-nb-workshops},
  \localtaskref{component-architecture}{oommf-python-interface}.

%Long term sustainability
\item \label{objective:disseminate} ``Promote and disseminate
  \ODK to the scientific community by active communication,
  workshop organisation, and training in the spirit of open-source
  software. This addresses Aim~\ref{aim:impact}.''

  This objective is at the core of \WPref{dissem}, with in particular
  22 meetings, developer and training workshops organized during the
  first reporting period. See Section~\ref{dissem} for details.
\end{compactenum}
