  \subsection{Impact}
  % Include in this section whether the information on section 2.1 of the DoA (how your
  % project will contribute to the expected impacts) is still relevant or needs to be
  % updated. Include further details in the latter case

  All the information of section 2.1 of the DoA is still relevant.

  There is for now no
  change to bring to Key Performance Indicators. The evolution of the measures between
  Month 18 and Month 36 allowed the Coordinator to assess that selected KPI were
  appropriate.

\subsection{Infrastructures}
% If access to research infrastructures has been provided under the grant please include
% access provision activities

\label{infrastructures}

Per design, \ODK focuses on delivering ``a flexible toolkit enabling
research groups to set up Virtual Research Environments''. As such,
there is no e-infrastructure deployed and managed by \ODK. Instead,
there are many e-infrastructures that use the software developed or
contributed to by \ODK, and we regularly help with new or updated
deployments.

Some of the typical content of this section (e.g. Selection Panel,
...) is therefore irrelevant for \ODK, and we simply provide some
informal information and figures on the main existing deployments and
their typical public, together with some assessment of the impact we
had on them.

\begin{itemize}
\item{cloud.sagemath.org} With 500k accounts worldwide and 30k active
  projects both for research and education, \SMC is the largest
  Virtual Research Environment based on the ecosystem \ODK contributes
  to. Predating \ODK, it benefits back from most of our actions. \ODK
  has been contributing to a healthy collaboration/competition
  relation between \JupyterHub and \SMC, with the competition
  occurring only at the level of specific individual components and
  both teams learning from each other.

\item{JupyterHub @ EGI} We have partnered with EGI, and helped them
  deploy an experimental instance of JupyterHub on their
  infrastructure. This service is now in their catalog and available
  to all academics in Europe:
  \url{https://jupyterhub.fedcloud-tf.fedcloud.eu/}. We are jointly
  seeking for ways to make this service sustainable in the long run,
  in particular as part of the EOSC.

\item{jupyter.math.cnrs.fr} We have helped setup this \JupyterHub
  service, deployed by the French CNRS for the benefit of the
  personnel of all math labs in France. This service includes all the
  \ODK computational components.

\item{mybinder.org} Binder is a web service that makes it easy for any
  user to publish live notebooks based on an arbitrary reproducible
  executable environments. It thus fosters dissemination and
  reproducible research. The current main instance
  (\url{http://mybinder.org/}) is often overloaded by the demand,
  proving that it has identified just the right service for a critical
  need.

  Our work on packaging
  \localtaskref{component-architecture}{mod-packaging} and \Jupyter
  integration \localtaskref{UI}{ipython-kernels} enabled the easy
  definition of executable environments for Binder and beyond that
  include \ODK's computational math software.

  Within our EGI partnership, we have started exploring ways to
  contribute additional computing resources to the main mybinder
  instance by setting a new one for the EC community:
  \url{http://binderhub.fedcloud-tf.fedcloud.eu/}.
  For the current status, see
  \url{https://github.com/OpenDreamKit/OpenDreamKit/issues/205}.

  We are also supporting the convergence between Binder and
  JupyterHub, to bring the flexibility of Binder computing
  environments to JupyterHub; see:
  \url{https://opendreamkit.org/2018/03/15/jupyterhub-binder-convergence/}.

\item{JupyterHub @ \site{Sheffield}}
  \href{http://docs.iceberg.shef.ac.uk/en/latest/using-iceberg/accessing/jupyterhub.html}{\JupyterHub}
  instance deployed on USheffield's HPC system.

\item{JupyterHub @ \site{UPSud}, \site{UVSQ}, ...} We have helped with
  the definition and deployment of those several University wide
  instances, exercising them with large classes (e.g. 400 students at
  \site{UPSud}. Lessons learned at the occasion have been shared
  through blog posts such as:
  \begin{itemize}
  \item \url{https://opendreamkit.org/2018/10/17/jupyterhub-docker/}
  \item \url{https://blog.jupyter.org/how-to-deploy-jupyterhub-with-kubernetes-on-openstack-f8f6120d4b1}
  \end{itemize}
\end{itemize}

Many more institutions are deploying JupyterHub instances. We are
keeping track of the instances we are aware of at
\url{https://github.com/OpenDreamKit/OpenDreamKit/issues/174}.

%%% Local Variables:
%%% mode: latex

%%% TeX-master: "report"
%%% End:
