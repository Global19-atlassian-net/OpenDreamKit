\documentclass{deliverablereport}

\deliverable{management}{ipr}
\issue{19}
\duedate{31/08/2016 (M12)}
\deliverydate{31/08/2016}
\author{Nicolas Thiéry \& Benoît Pilorget}

\usepackage{lscape} % for landscape
\usepackage{comments}
% %\usepackage[final]{comments}
\usepackage{verbatim}
\usepackage{listings}
\usepackage{supertabular,array}
\makeatletter
\newcommand\arraybslash{\let\\\@arraycr}
\makeatother
% \setlength\tabcolsep{1mm}
% \renewcommand\arraystretch{1.3}
%% Related Projects
\newcommand{\scienceproject}{\mbox{\textsc{SCIEnce}}\xspace}
\newcommand{\OOMMF}{OOMMF\xspace}
\newcommand{\OOMMFNB}{OOMMF-NB\xspace}
\newcommand{\ODK}{OpenDreamKit\xspace}
\newcommand{\VRE}{VRE\xspace}
\newcommand{\VREs}{VRE\xspace}
\newcommand{\software}[1]{\textsc{#1}\xspace}
\newcommand{\GAP}{\software{GAP}}
\newcommand{\HPCGAP}{\software{HPC-GAP}}
\newcommand{\libGAP}{\software{libGAP}}
\newcommand{\Singular}{\software{Singular}}
\newcommand{\Sage}{\software{Sage}}
\newcommand{\SageCombinat}{\software{Sage-Combinat}}
\newcommand{\MuPADCombinat}{\software{MuPAD-Combinat}}
\newcommand{\Sphinx}{\software{Sphinx}}
\newcommand{\SCSCP}{\software{SCSCP}}
\newcommand{\JavaScript}{\software{JavaScript}}
\newcommand{\Python}{\software{Python}}
\newcommand{\IPython}{\software{IPython}}
\newcommand{\Jupyter}{\software{Jupyter}}
\newcommand{\JupyterHub}{\software{JupyterHub}}
\newcommand{\Cython}{\software{Cython}}
\newcommand{\Pythran}{\software{Pythran}}
\newcommand{\Numpy}{\software{Numpy}}
\newcommand{\Pari}{\software{PARI}}
\newcommand{\PariGP}{\software{PARI/GP}}
\newcommand{\libpari}{\software{libpari}}
\newcommand{\GP}{\software{GP}}
\newcommand{\GPtoC}{\software{GP2C}}
\newcommand{\Linbox}{\software{LinBox}}
\newcommand{\LMFDB}{\software{LMFDB}}
\newcommand{\OpenEdX}{\software{OpenEdX}}
\newcommand{\Linux}{\software{Linux}}
\newcommand{\LATEX}{\software{\LaTeX}}
\newcommand{\SMC}{\software{SageMathCloud}}
\newcommand{\Simulagora}{\software{Simulagora}}
\newcommand{\KANT}{\software{KANT}}
\newcommand{\Magma}{\software{Magma}}
\newcommand{\Mathematica}{\software{Mathematica}}
\newcommand{\Maple}{\software{Maple}}
\newcommand{\Matlab}{\software{Matlab}}
\newcommand{\MuPAD}{\software{MuPAD}}
\newcommand{\MPIR}{\software{MPIR}}
\newcommand{\Arxiv}{\software{arXiv}}
\newcommand{\Givaro}{\software{Givaro}}
\newcommand{\fflas}{\software{fflas}}
\newcommand{\MathHub}{\software{MathHub}}
\newcommand{\FindStat}{\software{FindStat}}
\newcommand{\GitHub}{\software{GitHub}}
\newcommand{\git}{\software{git}}
\newcommand\DKS{\ensuremath{\mathcal{DKS}}\xspace}
\newcommand{\FLINT}{\software{FLINT}}

%%% Local Variables: 
%%% mode: latex
%%% TeX-master: "proposal"
%%% End: 

\newcommand{\TODO}[1]{{\color{red}{TODO}: #1}}
\begin{document}
\maketitle
\githubissuedescription
\newpage

\newcommand{\ODK}{OpenDreamKit\xspace}
\tableofcontents\newpage

\section{Introduction}




\section{Progress on the project}

In this section, we give a general view on the progress of the
project. We start by recalling some bits of context about \ODK's
approach that are important to understand and evaluate the
progress. Then we describe the general state, and enter in more detail
into the achieved and in progress tasks of the work packages.

\subsection{Some context: \ODK's approach}

Recall that, by design, \ODK's approach to deliver a Virtual Research
Environment (VRE) for mathematics is not to build a monolithic
one-size-fits-all VRE, but rather a toolkit from which it is easy to
setup VRE's customized to specific needs by combining together
components (collaborative workspaces, user interfaces, computational
software, databases, ...) on top of available physical resources (from
personal laptop to cloud infrastructure).

Most of the components preexist as an ecosystem of open source
software, developed by well established communities of developers. For
example, for interactive computing and data analysis, OpenDreamKit
promotes Jupyter, a web-based general purpose flexible notebook
interface\footnote{a notebook is a document that contains live code,
  equations, visualizations and explanatory text} that targets all
areas of science. A number of Virtual Research Environment already
exist, e.g. powered by SageMathCloud or JupyterHub.

Hence most of the work in \ODK is to foster this ecosystem, improving
the components themselves and their composability:
\begin{itemize}
\item Component architecture (WP3):
  \begin{itemize}
  \item ease of deployment: modularity, packaging, portability,
    distribution, for individual components and combinations thereof.
  \item sustainability of the ecosystem: improving the development workflows.
  \end{itemize}
\item User Interfaces (WP4): enable Jupyter as uniform notebook
  interface, and further improve it; foster the collaboration between
  SageMathCloud and JupyterHub; generally speaking investigate
  collaborative, reproducible, and active documents.
\item Performance (WP5): make the most of available hardware
  (multicore, HPC, cloud), for individual computational components and
  combinations thereof.
\item Data/Knowledge/Software (WP6): enable rich and robust
  interaction between computational components, data bases, knowledge
  bases, users through explicit common semantic spaces, a language to
  express them, and tools to leverage them.
\end{itemize}
This work is backed up by
\begin{itemize}
\item Community building and dissemination (WP2): developer and
  training workshops, conferences, teaching material, ...
\item A study of Social aspects (WP7): analysis of user needs and
  research on collaborative software development in mathematics
\end{itemize}

As a result of \ODK's approach, the work programme for \ODK consists
of a large array of loosely coupled tasks, each being useful in its
own right, and none being absolutely critical.

This first year confirmed that this is a strong feature of \ODK's
approach. Indeed, as analyzed in the proposal, this kind of project is
subject to the following risks:
\begin{itemize}
\item Recruitment of qualified personnel
\item Lack of predictability for tasks that are pursued jointly with
  the community
\end{itemize}
Together with challenging software aspects and rapidly evolving
technologies, this makes the prediction of workload and timeline for
tasks guess work at best, especially over a period of four years. The
loose coupling gives much flexibility, allowing to reshuffle tasks
schedule and human resources allocation, with little influence on the
general plan.

\subsection{General progress}

\TODO{check links below}

Intensive work has now started on almost all fronts of the project. A
few tasks (and the corresponding deliverables) have been postponed by
a couple months due to recruitment delays. This concerns mostly the
micromagnetic VRE demonstrator (\taskref{dissem}{dissemination-of-oommf-nb-virtual-environment}, \taskref{dissem}{dissemination-of-oommf-nb-workshops}, \taskref{dissem}{ibook}, \taskref{component-architecture}{oommf-python-interface}, \taskref{UI}{oommf-py-ipython-attributes}, \taskref{UI}{oommf-tutorial-and-documentation}, \taskref{UI}{oommf-nb-ve}, \taskref{social-aspects}{oommf-nb-evaluation})\TODO{use taskrefs}. Some deliverables got delayed as well
by a couple months due to unexpected technical difficulties or
misplanning (e.g. \delivref{hpc}{pythran-typing},
\delivref{hpc}{sage-paral-tree}, \delivref{UI}{pari-python-lib1}). On
the other hand, we are happy to report below on very strong
recruitment (see Section~\ref{Recruitment of highly qualified staf}), as well as unexpectedly rapid
progress on portability and packaging aspects. Also WP6
(Data/Knowledge/Software) has witnessed a particularly strong and
early uptake, with active involvement of many of the participants and
promising outcomes.

All in all, \ODK is running according to its plan, and its first
outcome are already benefiting the mathematical community and beyond. September 2016 will see the start of Key Performance Indicators. These KPIs, which will be more precisely and realistically defined then, will give results for the 1st Reporting Period (RP1) at month 18. This way we will be able to see the changes between the RP1 and RP2, at month 36.


\subsection{Achievements and ongoing progress in workpackages}

\subsubsection{WP1}

Anything to say here?

\subsubsection{WP2: Community building and dissemination}

As planned in \taskref{dissem}{dissemination-communication} and
\taskref{dissem}{dissemination}, 10 meetings, developer and training
workshops have been organized and coorganized by \ODK during year 1,
as well as many interventions and activities in external events.
Many more are being organized, including the first Women in Sage
workshop in Europe and three major training conferences (tentatively
at CIRM, Dagstuhl, and ICMS); \ODK and \ODK related work is regularly
presented at conferences (see the report for
\delivref{UI}{workshops-1}).


\subsubsection{WP3: Component architecture}

The first task of this workpackage is to improve the portability of
computational components
\taskref{component-architecture}{portability}. A particular challenge
is the portability of \Sage (and therefore all its dependencies)
on Windows, which has remained elusive for a decade, despite many
efforts of the community. We are happy to report that, in particular
thanks to months of intensive and expert work by our recruit in Paris
Sud Erik Bray, this challenge is about to be tackled, almost one year
before the expected delivery time.

A focused workshop in March (Sage Days 77) also triggered much work
and progress on the packaging side, both by \ODK participants and the
community. There is now good hope to have proper packages for SageMath
(and its dependencies) on the Debian distribution in the coming
months, a feature that has been desperately longed for over a decade.
The workshop was also the occasion to clarify the modularization,
packaging, and distribution needs and challenges\TODO{link to
https://wiki.sagemath.org/days77/packaging}.

\TODO{
- brief notes on other ongoing tasks?
- T3.2: ref to ongoing work in WP6
}
\subsection{WP4: User interfaces}

The first task for this workpackage is to enable the use of Jupyter as
uniform notebook interface for the relevant computational components
\taskref{UI}{ipython-kernels}. This is well under way for most
components. Progress was particularly fast for \Sage thanks to a
very active involvement of the community; this will enable, in the
coming months, a systematic transition from the legacy \Sage
notebook system to \Jupyter; this is a particularly important
achievement: beside all the benefits of a uniform and actively
developed interface for the user, outsourcing the maintenance of the
notebook interface will save the \Sage community much needed
resources.

A new \Jupyter package, nbdime, was created for \taskref{UI}{jupyter-collab},
enabling easier collaboration on notebooks via version control systems such as git.
This project was presented at SciPy US in July and EuroSciPy in August,
and has been met with enthusiasm from the scientific Python community
for its prospect of solving a longstanding difficulty in working with notebooks.

\TODO{Luca: a focus on one or two other tasks and or brief notes on:
- Jupyter / Jupyter hub improvements
- Discussions about structured documents, reproducibility, ...
- Work on Sphinx documentation started in collaboration with Sphinx
  dev and Sage Days 77 (TODO: mv
  https://www.lri.fr/etherpad/p/sage-days77-documentation to the
  sage-days77 wiki, and link from here)
}
\subsection{WP5: HPC}

TODO{Clément:
- Active development of Pythran
- mention upcoming workshops?
- mention Grenoble's active involvement starting now}

\subsection{WP6: }

\TODO{Michael: you must of some 2-3 paragraphs about what we have
been doing so far, right?}

\subsection{WP7: }

\TODO{Dima: describe briefly the current plan (upcoming workshops)
and evolutions in Oxford with the arrival of many people working on
related areas}

\section{Risk management}
\subsection{Recruitment of highly qualified staff}
 ~\\~\\Recruitment of highly qualified staff was planned to be a high risk when the Poposal was written. And unfortunatly it turned out we were right. In such a field as computer science and software development, potential candidates who are likely to be fairly young considering only temporary positions are offered, are very scarce. Furthermore they need to make a choice between public and provate bodies which are very attractive, and the choice between pure development and research.
Because of this difficulty to recruit in the past year, there have been slight changes in the workplan. Changes which have, of course, not put the project results at risk.

The following people were hired in the past year:\\


\begin{tabular}{|l|c|r|r|r|r|}
\hline
NAME&GENDER&PARTNER&POSITION&HIRING DATE\\
\hline
Benoît PILORGET&M&UPSud&Project manager&17-09-2015\\
Jeroen DEMEYER&M&UPSud&Research engineer&01-03-2016\\
Erik BRAY&M&UPSud&Research engineer&01-01-2016\\
Christian MAEDER&M&JacobsUni&Senior researcher&01-01-2016\\
Tom WIESING&M&JacobsUni&Junior researcher&01-09-2015\\
Xu HE&M&JacobsUni&Junior Researcher&01-09-2015\\
Alexander BEST&M&UNIKL&Research engineer&01-02-2016\\
Anders JENSEN&M&UNIKL&Postdoc&01-11-2015\\
Alexander KRUPPA&M&UNIKL&Postdoc&01-08/2016\\
Marijan BEG&M&Southampton&Research fellow&01-05-2016&\\
B. RAGAN-KELLEY&M&Simula&?&01-09-2015\\
V.T. FAUSKE&M&Simula&Postdoc fellow&02-05-2016\\
\hline
\end{tabular}\\
\begin{itemize}
\item{UPSud:}
  Thanks to an early start in the recruitment process, and despite
  some difficulties in attracting experienced candidates for a part
  time position, the project manager position (24PM) was filled by
  Benoît Pilorget shortly after the start of the project.

  The recruitment of UPSud's first Research Engineer (48PM) was
  delayed by four months because the top ranked candidate for this
  position, Erik Bray, was originating from the US and needed time to
  arrange for his moving; there were also some administrative delays
  (visa, ...).

  The second Research Engineer position (36PM) was more problematic
  for internal administrative reasons. The top ranked candidate,
  Jeroen Demeyer, had the perfect profile; however for family reasons,
  he wished to work most of the time from Gent in Belgium. After eight
  months investigating an administrative solution to hire him at
  UPSud, and a temporary four month solution, it was decided with
  OpenDreamKit's Steering Committee and Project Officer to instead add
  Gent's university as new partner, hire Jeroen there, with an
  adequate budget transfer and amendment to the Grant Agreement.

  Those delays have induced late start on several tasks, and costed
  much management time. However the excellent of the recruitment, well
  confirmed by the results obtained so far, was worth it and will soon
  compensate for the late start.

  In addition to the A three year PhD position was planned to work on WP6, starting from
  Month 12. By lack of suitable candidate, this position will be
  converted into a two year PostDoc position, presumably starting at
  Month 24. Active advertising has started and there are some
  tentative candidates. The relevant deliverables being due late in
  the project, no delay is to be expected from this change.

\item{CNRS:}The research engineer position (48PM) at the CNRS is not yet filled. The partner has unfortunately failed to recruit someone on time. The position is still open. However this should not postpone any deliverable.

\item{JacobsUni:} Michael Kohlase, lead PI for Jacobs University, has
  moved on 01/09/2016 to Friedrich-Alexander-Universität
  Erlangen-Nürnberg, and most of his team will follow him. Since he is
  an critical asset for OpenDreamKit, a Grant Agreement amendment will
  be submitted in Fall 2016 to update the consortium accordingly.

\item{UJF:} The original tentative candidate for UJF's Research
  Engineer position (12PM, planned to start on Month 1), Pierrick
  Brunet, finally declined the position to accept an alternative
  permanent offer. The position will be filled by another candidate in
  Autumn 2016. This induced a delay of Deliverable D5.2 from Month 12
  to Month 18, without impact on other tasks.

\item{UNIKL:} UNIKL had to split the 12 PM planned for a software developer into 2 shorter positions (Anders Jensen and Alexander Kruppa) in order to deliver the planned work on time. Indeed the few qualified persons for this job were not able to accept this 12 months position during the timelapse planned within the project.\\

\item{USFD:} The University of Sheffield has also been struggling in the the hiring process of a postdoc (36PM). The position should be filled this Autumn.\\

\item{Southampton:} Southampton faced administrative difficulties in the recruitment of Marijan Beg (38PM) as a post-doc, due to the Croatian nationality of Mr Beg. His recruitment was delayed of four months, and therefore some tasks and deliverables, planned to be bone around the end of the project, were postponed of a four months. However no serious delay nor implication on the main tasks of OpenDreamKit followed these difficulties.

\item{UZH:} The University of Zürich partner is only composed of one person, Paul-Olivier Dehaye, who does not enjoy a permanent position there. There has been worries that Mr Dehaye's contract ends at the University earlier than planned within OpenDreamKit. But thanks to the action of the OpenDreamKit steering committee, Mr Dehaye's position should be renewed for as long as the project needs.\\

\item{Simula:} Everything is fine concerning temporary staff recruitment on the Simula side, however we have had to endure the hazards of Human ressources with Hans-Peter Langtanger (the PI when the Grant was signed) being on a long-term sick leave, and with Martin Alnaes replacing him as PI currently on a paternity leave. However Benjamin Ragan-Kelley has been perfect  substitute leader of the Simula partner and all planned tasks are on time.
\end{itemize}


Altogether, this first year confirmed that the recruitment of highly
qualified staff is indeed a risky endeavor, which induced delays on
several deliverables. However the planned mitigation measures --
taking into account the pool of potential candidates in the design of
the positions, aggressive advertisement, weak coupling between tasks
-- worked adequately: with appropriate reshuffling of the work plan,
we don't expect an impact on the overall progress of the project.

\subsection{Different groups not forming effective team}
NT
\subsection{Implementing infrastructure that does not match the needs of end-users}

The consortium is keeping in their minds the end-user needs. Since OpenDreamKit is improving already existant softwares which have their own users, their needs are naturally met. However Key performance Indicators will evaluate the effects of OpenDreamKit on these softwares. KPIs, indicated in the Proposal, will be launched this Autumn with the help of the end-user group which was merged with the Advisory Board. Constant links between the accomplished work and the end-user needs should be made in WP2 deliverables and also in WP7 deliverables when relevant.
Open tracking of KPIs evolution can be found on [Github](https://github.com/OpenDreamKit/OpenDreamKit/labels/KPI).

\subsection{Lack of predictability for tasks that are pursued jointly with the community}
NT
\subsection{Reliance on external software components}
NT

\section{Quality assurance plan}

\subsection{Deliverables quality: Quality Review Board}
 ~\\~\\

 The Quality Review Board is the Consortium Body that fosters best
 possible quality in the deliverables. The body is chaired by Hans
 Fangohr, from the University of Southampton. He is supported in this
 task by Mike Croucher from the University of Sheffield, Alexander
 Konovalov from the University of St Andrews, and from Konrad Hinsen
 from the Centre de Biophysique Moléculaire.

 All board members have a track record of caring about the quality in
 software for computational science, including Mike Croucher's
 outreach and blogs, Alexander Knovalov's engagement with the Software
 Sustainability Institute, Konrad Hinsen's founding and editorship of
 the ReScience Journal, and Hans Fangohr's creation and directorship
 of the UK's only centre for doctoral training in computational
 modelling.

  The quality review board meets after each reporting period, the first
 one for OpenDreamKit ending at month 18 (February 2017), to review
 completed deliverables with focus on quality. The board will choose
 and focus on selected deliverables and review these in greater detail
 rather than attempting a superficial inspection of all deliverables.

 Seeking for continual improvement of the project's processes, the
 board will look for weaknesses, strengths and best-practice used in
 the creation of the deliverables, seeking further information from
 authors of the deliverables. The board will subsequently share their
 findings with the aim of increasing quality of future deliverables
 where possible. The quality review board embraces a no-blame culture
 to foster open exchanges and most-effective use and exploitation of
 their findings in achieving and sustaining high quality outcomes.

 While the primary focus of the board is on the OpenDreamKit project,
 some of the lessons may be more widely applicable and be made
 publicly available.





% % original

% The content form of deliverables due by then had to meet the
% expectations of the Project Officer and of Reviewers. Following this
% experience, we have concluded that deliverables should be written in
% Latex using a style file created for this purpose. For deliverables
% that are not reports by themselves, it's appropriate to have a
% relatively short report with a link to the github issue, and a copy of
% the description of this issue. In all cases, the report shall be
% self-contained. Deliverables are indeed evaluated based upon their
% versions submitted on the EU portal without retrieving other
% resources. Links have no legal value, since there is no guarantee that
% the referenced material will remain unchanged.  Partners who have a
% deliverable due at month 12 (August 2016) have been following these
% tips. The feedback of the Official review will help the Quality review
% Board in ensuring the quality of reports meets the needs. This will be
% up to the Quality Review Board to meet after the 1st reporting period
% and to decide if the quality of deliverables is acceptable.  They will
% aim at identifying good practice and weaknesses, and to share the
% lessons with the project to improve any future project work. The board
% will focus on selected deliverables and investigate those in detail
% rather than attempting a superficial inspection of all deliverables.


\subsection{Infrastructure quality: End-user group}

SL



\end{document}
