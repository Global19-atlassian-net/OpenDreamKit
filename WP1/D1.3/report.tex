\documentclass{../../Proposal/LaTeX-proposal/deliverablereport}


\deliverable{management}{ipr}
\issue{19}
\deliverydate{01/09/2016}
\duedate{01/09/2016 (M12)}
\author{Nicolas Thiéry & Benoît Pilorget}

\usepackage{lscape} % for landscape
\usepackage{comments}
% %\usepackage[final]{comments}
\usepackage{verbatim}
\usepackage{listings}
\usepackage{supertabular,array}
\makeatletter
\newcommand\arraybslash{\let\\\@arraycr}
\makeatother
% \setlength\tabcolsep{1mm}
% \renewcommand\arraystretch{1.3}
%% Related Projects
\newcommand{\scienceproject}{\mbox{\textsc{SCIEnce}}\xspace}
\newcommand{\OOMMF}{OOMMF\xspace}
\newcommand{\OOMMFNB}{OOMMF-NB\xspace}
\newcommand{\ODK}{OpenDreamKit\xspace}
\newcommand{\VRE}{VRE\xspace}
\newcommand{\VREs}{VRE\xspace}
\newcommand{\software}[1]{\textsc{#1}\xspace}
\newcommand{\GAP}{\software{GAP}}
\newcommand{\HPCGAP}{\software{HPC-GAP}}
\newcommand{\libGAP}{\software{libGAP}}
\newcommand{\Singular}{\software{Singular}}
\newcommand{\Sage}{\software{Sage}}
\newcommand{\SageCombinat}{\software{Sage-Combinat}}
\newcommand{\MuPADCombinat}{\software{MuPAD-Combinat}}
\newcommand{\Sphinx}{\software{Sphinx}}
\newcommand{\SCSCP}{\software{SCSCP}}
\newcommand{\JavaScript}{\software{JavaScript}}
\newcommand{\Python}{\software{Python}}
\newcommand{\IPython}{\software{IPython}}
\newcommand{\Jupyter}{\software{Jupyter}}
\newcommand{\JupyterHub}{\software{JupyterHub}}
\newcommand{\Cython}{\software{Cython}}
\newcommand{\Pythran}{\software{Pythran}}
\newcommand{\Numpy}{\software{Numpy}}
\newcommand{\Pari}{\software{PARI}}
\newcommand{\PariGP}{\software{PARI/GP}}
\newcommand{\libpari}{\software{libpari}}
\newcommand{\GP}{\software{GP}}
\newcommand{\GPtoC}{\software{GP2C}}
\newcommand{\Linbox}{\software{LinBox}}
\newcommand{\LMFDB}{\software{LMFDB}}
\newcommand{\OpenEdX}{\software{OpenEdX}}
\newcommand{\Linux}{\software{Linux}}
\newcommand{\LATEX}{\software{\LaTeX}}
\newcommand{\SMC}{\software{SageMathCloud}}
\newcommand{\Simulagora}{\software{Simulagora}}
\newcommand{\KANT}{\software{KANT}}
\newcommand{\Magma}{\software{Magma}}
\newcommand{\Mathematica}{\software{Mathematica}}
\newcommand{\Maple}{\software{Maple}}
\newcommand{\Matlab}{\software{Matlab}}
\newcommand{\MuPAD}{\software{MuPAD}}
\newcommand{\MPIR}{\software{MPIR}}
\newcommand{\Arxiv}{\software{arXiv}}
\newcommand{\Givaro}{\software{Givaro}}
\newcommand{\fflas}{\software{fflas}}
\newcommand{\MathHub}{\software{MathHub}}
\newcommand{\FindStat}{\software{FindStat}}
\newcommand{\GitHub}{\software{GitHub}}
\newcommand{\git}{\software{git}}
\newcommand\DKS{\ensuremath{\mathcal{DKS}}\xspace}
\newcommand{\FLINT}{\software{FLINT}}

%%% Local Variables: 
%%% mode: latex
%%% TeX-master: "proposal"
%%% End: 


\begin{document}
\maketitle
\newpage

\tableofcontents\newpage

\section{Introduction}

\section{Progress on the project}
\subsection{General progress}
\subsection{Focus on tasks to be implemented by now} 
Benoît and Nicolas (cf status reports)

\section{Risk management}
\subsection{Recruitment of highly qualified staff}
\subsection{Different groups not forming effective team}
\subsection{Implementing infrastructure that does not match the needs of end users}
\subsection{Lack of predictability for tasks that are pursued jointly with the community}
\subsection{Reliance on external software components}

\section{Quality assurance plan}
\subsection{Deliverables quality: Quality Review Board}

The Quality Review Board is the Consortium Body that must ensure the quality of deliverables.
The body is chaired by Hans Fangohr, from the University of Southampton. He is helped in this task by Ursula Martin, from the University of Oxford, and from Konrad Hinsen from the Centre de Biophysique Moléculaire. This board must meet after each reporting period, the first one for OpenDreamKit ending at month 18 (February 2017). 
For the moment, deliverables have mostly been discussed in the frame of the Interim Check Review which took place in Bremen on the 28th of June 2016.

The content form of deliverables due by then had to meet the expectations of the Project Officer and of Reviewers. Following this experience, we have concluded that deliverables should be written in Latex using a style file created for this purpose. For deliverables that are not reports by themselves, it's appropriate to have a relatively short report with a link to the github issue, and a copy of the description of this issue. In all cases, the report shall be self-contained. Deliverables are indeed evaluated based upon their versions submitted on the EU portal without retrieving other resources. Links have no legal value, since there is no guarantee that the referenced material will remain unchanged.
Partners who have a deliverable due at month 12 (August 2016) have been following these tips. The feedback of the Official review will help the Quality review Board in ensuring the quality of reports meets the needs. This will be up to the Quality Review Board to meet after the 1st reporting period and to decide if the quality of deliverables is acceptable.
They will aim aat identifying good practice and weaknesses, and to share the lessons with the project to improve any future project work. The board will focus on selected deliverables and investigate those in detail rather than attempting a superficial inspection of all deliverables.


\subsection{Infrastructure quality: End-user group}





\end{document}


