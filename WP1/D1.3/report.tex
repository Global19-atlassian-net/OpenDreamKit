\documentclass{../../Proposal/LaTeX-proposal/deliverablereport}


\deliverable{management}{ipr}
\issue{19}
\deliverydate{01/09/2016}
\duedate{01/09/2016 (M12)}
\author{Nicolas Thiéry \& Benoît Pilorget}

\usepackage{lscape} % for landscape
\usepackage{comments}
% %\usepackage[final]{comments}
\usepackage{verbatim}
\usepackage{listings}
\usepackage{supertabular,array}
\makeatletter
\newcommand\arraybslash{\let\\\@arraycr}
\makeatother
% \setlength\tabcolsep{1mm}
% \renewcommand\arraystretch{1.3}
%% Related Projects
\newcommand{\scienceproject}{\mbox{\textsc{SCIEnce}}\xspace}
\newcommand{\OOMMF}{OOMMF\xspace}
\newcommand{\OOMMFNB}{OOMMF-NB\xspace}
\newcommand{\ODK}{OpenDreamKit\xspace}
\newcommand{\VRE}{VRE\xspace}
\newcommand{\VREs}{VRE\xspace}
\newcommand{\software}[1]{\textsc{#1}\xspace}
\newcommand{\GAP}{\software{GAP}}
\newcommand{\HPCGAP}{\software{HPC-GAP}}
\newcommand{\libGAP}{\software{libGAP}}
\newcommand{\Singular}{\software{Singular}}
\newcommand{\Sage}{\software{Sage}}
\newcommand{\SageCombinat}{\software{Sage-Combinat}}
\newcommand{\MuPADCombinat}{\software{MuPAD-Combinat}}
\newcommand{\Sphinx}{\software{Sphinx}}
\newcommand{\SCSCP}{\software{SCSCP}}
\newcommand{\JavaScript}{\software{JavaScript}}
\newcommand{\Python}{\software{Python}}
\newcommand{\IPython}{\software{IPython}}
\newcommand{\Jupyter}{\software{Jupyter}}
\newcommand{\JupyterHub}{\software{JupyterHub}}
\newcommand{\Cython}{\software{Cython}}
\newcommand{\Pythran}{\software{Pythran}}
\newcommand{\Numpy}{\software{Numpy}}
\newcommand{\Pari}{\software{PARI}}
\newcommand{\PariGP}{\software{PARI/GP}}
\newcommand{\libpari}{\software{libpari}}
\newcommand{\GP}{\software{GP}}
\newcommand{\GPtoC}{\software{GP2C}}
\newcommand{\Linbox}{\software{LinBox}}
\newcommand{\LMFDB}{\software{LMFDB}}
\newcommand{\OpenEdX}{\software{OpenEdX}}
\newcommand{\Linux}{\software{Linux}}
\newcommand{\LATEX}{\software{\LaTeX}}
\newcommand{\SMC}{\software{SageMathCloud}}
\newcommand{\Simulagora}{\software{Simulagora}}
\newcommand{\KANT}{\software{KANT}}
\newcommand{\Magma}{\software{Magma}}
\newcommand{\Mathematica}{\software{Mathematica}}
\newcommand{\Maple}{\software{Maple}}
\newcommand{\Matlab}{\software{Matlab}}
\newcommand{\MuPAD}{\software{MuPAD}}
\newcommand{\MPIR}{\software{MPIR}}
\newcommand{\Arxiv}{\software{arXiv}}
\newcommand{\Givaro}{\software{Givaro}}
\newcommand{\fflas}{\software{fflas}}
\newcommand{\MathHub}{\software{MathHub}}
\newcommand{\FindStat}{\software{FindStat}}
\newcommand{\GitHub}{\software{GitHub}}
\newcommand{\git}{\software{git}}
\newcommand\DKS{\ensuremath{\mathcal{DKS}}\xspace}
\newcommand{\FLINT}{\software{FLINT}}

%%% Local Variables: 
%%% mode: latex
%%% TeX-master: "proposal"
%%% End: 


\begin{document}
\maketitle
\newpage

\tableofcontents\newpage

\section{Introduction}

\section{Progress on the project}

\subsection{General progress}

NT

\subsection{Focus on tasks to be implemented by now} 

NT

\section{Risk management}
\subsection{Recruitment of highly qualified staff}

Recruitment of highly qualified staff was planned to be a high risk when the Poposal was written. And unfortunatly it turned out we were right. In such a field as computer science and software development, potential candidates who are likely to be fairly young considering only temporary positions are offered, are very scarce. Furthermore they need to make a choice between public and provate bodies which are very attractive, and the choice between pure development and research.
Because of this difficulty to recruit in the past year, there have been slight changes in the workplan. Changes which have, of course, not put the project results at risk

The following people were hired in the past year


\begin{tabular}{|l|c|r|r|r|r|}
\hline
NAME&GENDER&PARTNER&POSITION&HIRING DATE\\
\hline
Benoît PILORGET&M&UPSud&Project manager&17-09-2015\\
Jeroen DEMEYER&M&UPSud&Research engineer&01-03-2016\\
Erik BRAY&M&UPSud&Research engineer&01-01-2016\\
Christian MAEDER&M&JacobsUni&Senior researcher&01-01-2016\\
Tom WIESING&M&JacobsUni&Junior researcher&01-09-2015\\
Xu HE&M&JacobsUni&Junior Researcher&01-09-2015\\
Alexander BEST&M&UNIKL&Research engineer&01-02-2016\\
Anders JENSEN&M&UNIKL&Postdoc&01-11-2015\\
Alexander KRUPPA&M&UNIKL&Postdoc&01-08/2016\\
Marijan BEG&M&Southampton&Research fellow&01-05-2016&\\
B. RAGAN-KELLEY&M&Simula&?&01-09-2015\\
V.T. FAUSKE&M&Simula&Postdoc fellow&02-05-2016\\
\hline
\end{tabular}
\begin{itemize}
\item{UPSud:} there were some delay in the recruitment of a project manager (24PM) due to the fact the parti-tim position was less attractive for experienced persons. The recruitment of Erik Bray (36PM) as software developer took also longer than planned because his moving from the USA to France did not go as smoothly as planned. However his skills and results have proven more than once he is the right person at the right place.
The second developer position (36PM) was more problematic for intern administrative reasons to UPSud. Jeroen Demeyer was selected  for the position among other candidates because he had the perfect profile for the job asked from him. However and despise he has the Belgian nationality, since he wished to commute from Belgium to Paris, the UPSud administration refused to offer him a contract longer than 4 months and preferred including a new partner in the OpenDreamKit consortium  and to move the person-months planned for Joeroen Demeyer and all the related budget to that new partner. The OpenDreamKit Steering Committee willing to keep Mr Demeyer on within the project, an amendment to the Grant Agreement was signed between the Coordinator, the Commission, and the University of Gent by whom Jeroen Demeyer is now employed. ++++ SAY SOMETHING ABOUT PHD (36PM) TURNED TO POSTDOC++++

\item{CNRS:}The research engineer position (48PM) at the CNRS is not yet filled. The partner has unfortunately failed to recruit someone on time. The position is still open. However this should not postpone any deliverable.

\item{JacobsUni:} Michael Kohlase, the PI of the Jacobs Univeristy site, will be change position on the 01/09/2016 to move to the Friedrich-Alexander-Universität Erlangen-Nürnberg. Since most of his team follows him and since he is an important asset to the success of OpenDreamKit, there will be a new amendment to the Grant Agreement in the Autumn 2016. More explanation concerning the terms of the amendment will be given then.

\item{UJF:}The UJF had planned to recruit an engineer(12PM) named Pierrick Brunnet who was supposed to begin his work on the 1st of September 2015 for the UJF and OpenDreamKit. However Mr Brunnet decided to decline the offer in order to accept a permanent position. The engineer position will be filled in Autumn 2016 but the deliverable D5.2 had to be delayed from month 12 to month 18. This delay did not have any imapct on other tasks.

\item{UNIKL:} UNIKL had to split the 12 PM planned for a software developer into 2 shorter positions (Anders Jensen and Alexander Kruppa) in order to deliver the planned work on time. Indeed the few qualified persons for this job were not able to accept this 12 months position during the timelapse planned within the project.

\item{USFD:} The University of Sheffield has also been struggling in the the hiring process of a postdoc (36PM). The position should be filled this Autumn.

\item{Southampton:} Southampton faced administrative difficulties in the recruitment of Marijan Beg (38PM) as a post-doc, due to the Croatian nationality of Mr Beg. His recruitment was delayed of four months, and therefore some tasks and deliverables, planned to be bone around the end of the project, were postponed of a four months. However no serious delay nor implication on the main tasks of OpenDreamKit followed these difficulties.
\item{UZH:} The University of Zürich partner is only composed of one person, Paul-Olivier Dehaye, who does not enjoy a permanent position there. There has been worries that Mr Dehaye's contract ends at the University earlier than planned within OpenDreamKit. But thanks to the action of the OpenDreamKit steering committee, Mr Dehaye's position should be renewed for as long as the project needs.
\item{Simula:} Everything is fine concerning temporary staff recruitment on the Simula side, however we have had to endure the hazards of Human ressources with Hans-Peter Langtanger (the PI when the Grant was signed) being on a long-term sick leave, and with Martin Alnaes replacing him as PI currently on a paternity leave. However Benjamin Ragan-Kelley has been perfect  substitute leader of the Simula partner and all planned tasks are on time.
\end{itemize}

In spite of Human Resources issues, and even though some small workplan reshufflings were made, all tasks and deliverables will be accomplished within the reporting period they were planned.

\subsection{Different groups not forming effective team}
NT
\subsection{Implementing infrastructure that does not match the needs of end-users}

The consortium is keeping in their minds the end-user needs. Since OpenDreamKit is improving already existant softwares which have their own users, their needs are naturally met. However Key performance Indicators will evaluate the effects of OpenDreamKit on these softwares. KPIs, indicated in the Proposal, will be launched this Autumn with the help of the end-user group which was merged with the Advisory Board. Constant links between the accomplished work and the end-user needs should be made in WP2 deliverables and also in WP7 deliverables when relevant.
Open tracking of KPIs evolution can be found on [Github](https://github.com/OpenDreamKit/OpenDreamKit/labels/KPI).

\subsection{Lack of predictability for tasks that are pursued jointly with the community}
NT
\subsection{Reliance on external software components}
NT

\section{Quality assurance plan}

\subsection{Deliverables quality: Quality Review Board}


The Quality Review Board is the Consortium Body that must ensure the quality of deliverables.
The body is chaired by Hans Fangohr, from the University of Southampton. He is helped in this task by Ursula Martin, from the University of Oxford, and from Konrad Hinsen from the Centre de Biophysique Moléculaire. This board must meet after each reporting period, the first one for OpenDreamKit ending at month 18 (February 2017). 
For the moment, deliverables have mostly been discussed in the frame of the Interim Check Review which took place in Bremen on the 28th of June 2016.


The content form of deliverables due by then had to meet the expectations of the Project Officer and of Reviewers. Following this experience, we have concluded that deliverables should be written in Latex using a style file created for this purpose. For deliverables that are not reports by themselves, it's appropriate to have a relatively short report with a link to the github issue, and a copy of the description of this issue. In all cases, the report shall be self-contained. Deliverables are indeed evaluated based upon their versions submitted on the EU portal without retrieving other resources. Links have no legal value, since there is no guarantee that the referenced material will remain unchanged.
Partners who have a deliverable due at month 12 (August 2016) have been following these tips. The feedback of the Official review will help the Quality review Board in ensuring the quality of reports meets the needs. This will be up to the Quality Review Board to meet after the 1st reporting period and to decide if the quality of deliverables is acceptable.
They will aim at identifying good practice and weaknesses, and to share the lessons with the project to improve any future project work. The board will focus on selected deliverables and investigate those in detail rather than attempting a superficial inspection of all deliverables.


\subsection{Infrastructure quality: End-user group} 

SL



\end{document}


