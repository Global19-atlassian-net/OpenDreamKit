\documentclass{deliverablereport}

\deliverable{management}{ipr}
\duedate{31/08/2016 (M12)}
\deliverydate{??/09/2016}

\usepackage[style=alphabetic,backend=bibtex]{biblatex}
\addbibresource{../../lib/kbibs/kwarcpubs.bib}
\addbibresource{../../lib/kbibs/extpubs.bib}
\addbibresource{../../lib/kbibs/kwarccrossrefs.bib}
\addbibresource{../../lib/kbibs/extcrossrefs.bib}
\addbibresource{../../lib/deliverables.bib}
% temporary fix due to http://tex.stackexchange.com/questions/311426/bibliography-error-use-of-blxbblverbaddi-doesnt-match-its-definition-ve
\makeatletter\def\blx@maxline{77}\makeatother

\author{Nicolas M. Thiéry, Benoît Pilorget, et al.}

\begin{document}
\enlargethispage{4ex}
\maketitle
\githubissuedescription
\tableofcontents\newpage






\section{Progress on the project}

In this section, we give a general view on the progress of the
project. We start by recalling some bits of context about \ODK's
approach that are important to understand and evaluate the
progress. Then we describe the general state, and enter in more detail
into the achieved and in progress tasks of the work packages.

\subsection{Some context: \ODK's approach}
Recall that, by design, \ODK's approach to deliver a Virtual Research
Environment (VRE) for mathematics is not to build a monolithic
one-size-fits-all VRE, but rather a toolkit from which it is easy to
setup VRE's customized to specific needs by combining together
components (collaborative workspaces, user interfaces, computational
software, databases, ...) on top of available physical resources (from
personal laptop to cloud infrastructure).

Most of the components preexist as an ecosystem of open source
software, developed by well established communities of developers. For
example, for interactive computing and data analysis, OpenDreamKit
promotes Jupyter, a web-based general purpose flexible notebook
interface\footnote{a notebook is a document that contains live code,
  equations, visualizations and explanatory text} that targets all
areas of science. A number of Virtual Research Environment already
exist, e.g. powered by SageMathCloud or JupyterHub.

Hence most of the work in \ODK is to foster this ecosystem, improving
the components themselves and their composability:
\begin{itemize}
\item Component architecture (WP3):
  \begin{itemize}
  \item ease of deployment: modularity, packaging, portability,
    distribution, for individual components and combinations thereof.
  \item sustainability of the ecosystem: improving the development workflows.
  \end{itemize}
\item User Interfaces (WP4): enable Jupyter as uniform notebook
  interface, and further improve it; foster the collaboration between
  SageMathCloud and JupyterHub; generally speaking investigate
  collaborative, reproducible, and active documents.
\item Performance (WP5): make the most of available hardware
  (multicore, HPC, cloud), for individual computational components and
  combinations thereof.
\item Data/Knowledge/Software (WP6): enable rich and robust
  interaction between computational components, data bases, knowledge
  bases, users through explicit common semantic spaces, a language to
  express them, and tools to leverage them.
\end{itemize}
This work is backed up by
\begin{itemize}
\item Community building and dissemination (WP2): developer and
  training workshops, conferences, teaching material.
\item A study of Social aspects (WP7): analysis of user needs and
  research on collaborative software development in mathematics.
\end{itemize}

As a result of \ODK's approach, the work programme for \ODK consists
of a large array of loosely coupled tasks, each being useful in its
own right, and none being absolutely critical.

This first year confirmed that this is a strong feature of \ODK's
approach. Indeed, as analyzed in the proposal, this kind of project is
subject to the following risks:
\begin{itemize}
\item Recruitment of qualified personnel;
\item Different groups not forming effective team;
\item Implementing infrastructure that does not match the needs of end-users;
\item Lack of predictability for tasks that are pursued jointly with
  the community;
\item Reliance on external software components.
\end{itemize}
Together with challenging software aspects and rapidly evolving
technologies, this makes the prediction of workload and timeline for
tasks guess work at best, especially over a period of four years. The
loose coupling gives much flexibility, allowing to reshuffle tasks
schedule and human resources allocation, with little influence on the
general plan.

\subsection{General progress}

Intensive work has now started on almost all fronts of the project\footnote{status reports delivered at the St Andrews project meeting (January 2016) and at the Bremen's project meeting (June 2016) helped the Coordinator to track the progress}. A
few tasks (and the corresponding deliverables) have been postponed by
a couple months due to recruitment delays. This concerns mostly the
micro-magnetic VRE demonstrator (\taskref{dissem}{dissemination-of-oommf-nb-virtual-environment}, \taskref{dissem}{dissemination-of-oommf-nb-workshops},
 \taskref{dissem}{ibook}, \taskref{component-architecture}{oommf-python-interface}, \taskref{UI}{oommf-py-ipython-attributes}, \taskref{UI}{oommf-tutorial-and-documentation},
 \taskref{UI}{oommf-nb-ve}, \taskref{social-aspects}{oommf-nb-evaluation}). Some deliverables got delayed as well
by a couple months due to unexpected technical difficulties or
misplanning (e.g. \delivref{hpc}{pythran-typing},
\delivref{hpc}{sage-paral-tree}, \delivref{UI}{pari-python-lib1}). All these delays have been included in the amendment of the Grant Agreement
, amendment which was necessary to include UGent in the consortium. On
the other hand, we are happy to report below on very strong
recruitment (see Section 3.1), as well as unexpectedly rapid
progress on portability and packaging aspects. Also WP6
(Data/Knowledge/Software) has witnessed a particularly strong and
early uptake, with active involvement of many of the participants and
promising outcomes.

All in all, \ODK is running according to its plan, and its first
outcome are already benefiting the mathematical community and beyond. September 2016 will see the start of Key Performance Indicators. 
These KPIs, which will be more precisely and realistically defined then, will give results for the 1st Reporting Period (RP1) at month 18. 
This way we will be able to see the evolution of the impact OpenDreamKit has had between the RP1 and RP2, at month 36.


\subsection{Achievements and ongoing progress in workpackages}

\subsubsection{WP1: Project management}

As planned in WP1, \site{PS} has been coordinating \ODK. 
Most of the management effort for year 1 has been made in \longtaskref{management}{project-finance-management} (see \delivref{management}{infrastructure} and \delivref{management}{data-plan1}). 
A consortium agreement was signed between partners, stating precise rules about topics such as: responsibilities,
 governance, access to results and the background included. Ugent has recently agreed to sign this Consortium Agreement without any modification to it.
 Concerning communication, the website for the project has been 
continuously updated with new content. However, the
steering committee agreed that today's website was more built for an internal use and not towards the outside. 
\ODK entering its second year, a new version of the website
 is currently on the way and should be ready for Autumn 2016. This version shall be thought as a long-term dissemination and communication tool.
\\Mailing lists created for \ODK are still in use, though the ones turned towards dissemination and communication outside of the 
consortium would deserve to be developed now
the project is entering a more mature part of its life. 
During year 1, a kick-off meeting was organised in Orsay, and two progress meetings which allowed partners to give status reports 
and the steering committee to meet were organised.
 The first progress meeting was organised in St Andrews (January 2016) and the second one was located in Bremen (June 2016).
The latter was also sided with the interim project review, planned at month 9, where deliverables due by then were presented to 
the Project Officer and Reviewers.
\ODK was granted the grade 3 out of 4 for this interim review: "Good progress (the project has achieved most of its objectives and technical goals for the 
period with relatively minor deviations)".

Furthermore, the first reporting period lasting 18 months, \site{PS} administration decided to organise an internal and interim 
breakdown of costs. This exercise aimed at raising potential questions from partners and to make sure partners do follow
the EC rules for the eligibility of costs.

More information on \longtaskref{management}{project-quality-management} can be found in Section 4 of this document:
 Quality assurance plan.
Also, if some work has been done on \longtaskref{management}{project-innovation-management}, a full report on the matter will
 be available at month 18.



\subsubsection{WP2: Community building and dissemination}

As planned in \longtaskref{dissem}{dissemination-communication} and
\longtaskref{dissem}{dissemination}, 14 meetings, developer and training
workshops have been organized and co-organized by \ODK during year 1,
and complemented by many presentations and activities in external events.
Many more are being prepared, including the first Women in Sage
workshop in Europe and three major training conferences (tentatively
at CIRM, Dagstuhl, and ICMS); \ODK and \ODK related work is regularly
presented at conferences (see the report for
\delivref{dissem}{workshops-1}).

\ODK is also working on its communication and visibility. The \longdelivref{dissem}{press-release-1}
 was delivered and a page fro the \href{https://github.com/OpenDreamKit/OpenDreamKit/blob/master/Communication/eInfra-Booklet/ODK.md}{E-infrastructure booklet} was written
jointly by \ODK members. After
one year, we have a clearer understanding of what is needed by the project.
We are working on a new organization for the website where day to day activities
would be more visible through our blog. Posts include reports on conferences, workshops,
new features and emerging technologies (as part of D2.2).


\subsubsection{WP3: Component architecture}

The first task of this workpackage is to improve the portability of
computational components
\longtaskref{component-architecture}{portability}. A particular challenge
is the portability of \Sage (and therefore all its dependencies)
on Windows, which has remained elusive for a decade, despite many
efforts of the community. We are happy to report that, in particular
thanks to months of intensive and expert work by our recruit Erik Bray at \site{PS}
, this challenge is about to be tackled, almost one year
before the expected delivery time.

Task \taskref{component-architecture}{interface-systems} on interfaces
between mathematical systems is progressing as expected. Experimental
work on a semantic interface between \GAP and \Sage
(\delivref{component-architecture}{semantic-interface-sage-gap}, due
on month 36) has started during the joint GAP-Sage days, and a working
prototype is already available. The current prototype uses \emph{ad
  hoc} language mechanisms to transfer the semantics from one system
to the other; these mechanisms will be replaced with a generic API
once the MitM approach developed in WP6 will be mature
enough. Meanwhile, a purely technical piece of the puzzle has been
already achieved by D3.3, which brings support for the SCSCP protocol
to the Python ecosystem (and thus to \Sage and its subsystems). This
is instrumental for supporting the MitM approach.

After \delivref{component-architecture}{virtual-machines} was delivered, a focused workshop
 in March (Sage Days 77) also triggered much work
and progress on the packaging side
(\taskref{component-architecture}{mod-packaging}), both by \ODK
participants and the community. There is now good hope to have proper
packages for \Sage (and its dependencies) on the Debian
distribution in the coming months, a feature that has been desperately
longed for over a decade.  The workshop was also the occasion to
clarify the modularization, packaging, and distribution needs and
challenges. Internal notes on the progress made have been taken in the
\href{https://wiki.sagemath.org/days77/packaging}{\Sage wiki}, and a
\href{https://groups.google.com/forum/#!forum/sage-packaging}{mailing
  list specifically dedicated to packaging \Sage} has been created.

% \TODO{
% - brief notes on other ongoing tasks? (Luca: I don't know, I don't see any delivs in 3.4 and 3.8, and 3.6 has been delayed)
% }
\subsubsection{WP4: User interfaces}

The first task for this workpackage is to enable the use of Jupyter as
uniform notebook interface for the relevant computational components
\taskref{UI}{ipython-kernels}. This is well under way for most
components. Progress was particularly fast for \Sage thanks to a
very active involvement of the community; this will enable, in the
coming months, a systematic transition from the legacy \Sage
notebook system to \Jupyter; this is a particularly important
achievement: beside all the benefits of a uniform and actively
developed interface for the user, outsourcing the maintenance of the
notebook interface will save the \Sage community much needed
resources.

A new \Jupyter package, nbdime, was created for \delivref{UI}{jupyter-collab}
enabling easier collaboration on notebooks via version control systems such as git.
This project was presented at SciPy US in July and EuroSciPy in August,
and has been met with enthusiasm from the scientific Python community
for its prospect of solving a longstanding difficulty in working with notebooks.

\TODO{
- @minrk: Jupyter / Jupyter hub improvements
- @minrk?: Short discussion about reproducibility, ...
}

Active structured documents are a common need with many use cases, and
as many potential solutions. Requirements and venues for
collaborations were explored through discussions between participants,
in particular at the occasion of
\href{https://wiki.sagemath.org/days77/}{Sage Days 77} workshop (see
the
\href{https://wiki.sagemath.org/days77/live-structured-documents}{notes}),
and June's ODK meeting in Bremen. The findings were reported in
\delivref{UI}{adstex}. Sage Days 77 was also the occasion to bootstrap the long term work of
\href{https://wiki.sagemath.org/days77/documentation}{refactoring the
  Sage documentation build system} (\delivref{UI}{sage-sphinx}) in
collaboration with a Sphinx developer.

One deliverable, \delivref{UI}{pari-python-lib1}, was delayed by a
couple months due to unforeseen technical difficulties, but with no
impact on the rest of the project.

\subsubsection{WP5: High Performance Mathematical Computing}

After \delivref{hpc}{sage-paral-tree} was delivered, \longtaskref{hpc}{pythran} is making good progress towards the development of Pythran and its
interaction with~\Sage.
More precisely, Pythran's typing system (\delivref{hpc}{pythran-typing}) has already been improved, as described
in a progress note\footnote{\url{http://serge-sans-paille.github.io/pythran-stories/identifier-binding-computation.html}},
yet it still requires an aditional effort and the delivery date of
\delivref{hpc}{pythran-typing} has been postoned from month 9 to month 12. It
has no impact on related tasks or
deliverables. The start of Deliverable~\delivref{hpc}{pythran-sage} has been
delayed from Month 12 to Month 18 due to the difficulty of hiring an engineer for the task. It is now
making good progress.

Deliverable~\delivref{hpc}{MPIRsuperoptimiser} (writting a super-optimizer for modern CPU
architectures for MPIR) is making progress but has hit a major blocker: it
requires to use a precise clock cycle counter, for which a kernel module has been
proposed in this deliverable. However a bug in the Linux kernel seem to
automatically disable these counters. It has been reported upstream but the long
delay to have a patch incorporated into the kernel will impact the delivery of
this deliverable. There is no dependency to this deliverable, and the problem is
now well understood and its solution is underway. See
\url{https://github.com/OpenDreamKit/OpenDreamKit/issues/118} for further details.

As mentioned in the initial proposal, work on \taskref{hpc}{hpc-linbox} (LinBox) is just
starting in September 2016, as the members of this group had already too many PM
involved in other projects.

A first workshop on HPC will be organized in March 2017 in Grenoble, France. A
smaller workshop may possibly be organized in December 2016, to gather
participants involved in the development and Pythran.


\subsubsection{WP6: Data/Knowledge/Software-Bases }

In a series of workshops (September 2015 in Paris, January 2016 in St. Andrews, June 2016
in Bremen, and July 2016 in Bia{\l}ystok) the participants working on \WPref{dksbases} met
and discussed the topic of integrating the \pn systems into a mathematical VRE toolkit.
Key results were
\begin{compactitem}[\bf D1.]
\item the observation that \emph{knowledge-aware interoperability of software and
    database-systems is the most critical objective} for \WPtref{dksbases} in the \pn
  project.
\item the consensus that this can be achieved by \emph{aligning the mathematical knowledge
    underlying the various systems}.
\end{compactitem}
This requires explicitly representing the three aspects of math VREs -- Data (D),
Knowledge (K), and Software (S) -- and basing computational services and inter-system
communication on a joint \DKS-base. These results are engrained in the
``Math-in-the-Middle'' (MitM) paradigm~\cite{DehKohKon:iop16}, which gives a
representational basis for specification-based interoperability of mathematical software
systems -- so that they can be integrated in a VRE toolkit. In the MitM paradigm, the
mathematical knowledge underlying the VREs (K) and the the interface of the for each
system (S) are represented as modular theory graphs in the OMDoc/MMT format. For the data
aspect (D) we have extended the concept of OMDoc/MMT theories to ``virtual theories'' that
allow the practical management of possibly infinite theories, see~\cite{ODK-D6.2} for
details. 

A side effect of the \textbf{D1.} is that the verification aspects anticipated in the
proposal are non-critical to the \pn project. In particular the value of the exemplary
verification of an LMFDB algorithm in \taskref{dksbases}{data-LMFDB} and
deliverable~\delivref{dksbases}{lfmverif} seems highly questionable.

Correspondingly we have refined the notion of ``triformal theories'' coined in the
proposal into the concept of ``\DKS theory graphs'', which can be formalized and
implemented without the extension of OMDoc/MMT for ``biformal theories'' anticipated in
the proposal.

Through the concerted effort of the WP6 participants, we have been able to implement
this design into prototypical \DKS base patterned after the MitM paradigm with virtual
theories, generating interface theory graphs for the \GAP and \Sage systems and
integrating the \LMFDB system via the MitM codec architecture described
in~\cite{ODK-D6.2}. Based on this, we were able to generically integrate \GAP, \Sage, and
\LMFDB via the standardised SCSCP protocol~\cite{HorRoz:ossp09} -- essentially remote
procedure calls with OpenMath Objects. This case study shows the feasibility of the
initial design of \DKS-bases; further investigations and the integration of additional
systems will determine the practicability.

\subsubsection{WP7: Social aspects}

\TODO{Dima: describe briefly the current plan (upcoming workshops)
and evolutions in Oxford with the arrival of many people working on
related areas}

\section{Risk management}
\subsection{Recruitment of highly qualified staff}
 Recruitment of highly qualified staff was planned to be a high risk when the Proposal was written. And unfortunately it turned out we were right. In such a field as computer science and software development, potential candidates who are likely to be fairly young considering only temporary positions are offered, are very scarce. Furthermore they need to make a choice between public and private bodies which are very attractive, and the choice between pure development and research.
Because of this difficulty to recruit in the past year, there have been slight changes in the workplan. Changes which have, of course, not put the project results at risk.

The following people were hired in the past year:\\


\begin{tabular}{|l|c|r|r|r|r|}
\hline
NAME&GENDER&PARTNER&POSITION&HIRING DATE\\
\hline
Benoît PILORGET&M&\site{PS}&Project manager&17-09-2015\\
Jeroen DEMEYER&M&\site{PS}&Research engineer&01-03-2016\\
Erik BRAY&M&\site{PS}&Research engineer&01-01-2016\\
Christian MAEDER&M&JacobsUni&Senior researcher&01-01-2016\\
Tom WIESING&M&JacobsUni&Junior researcher&01-09-2015\\
Xu HE&M&JacobsUni&Junior Researcher&01-09-2015\\
Alexander BEST&M&UNIKL&Research engineer&01-02-2016\\
Anders JENSEN&M&UNIKL&Postdoc&01-11-2015\\
Alexander KRUPPA&M&UNIKL&Postdoc&01-08-2016\\
Jan AKSAMIT&M&USlaski&Technical staff&01-10-2015\\
Marijan BEG&M&Southampton&Research fellow&01-05-2016&\\
B. RAGAN-KELLEY&M&Simula&Postdoc&01-09-2015\\
V.T. FAUSKE&M&Simula&Postdoc fellow&02-05-2016\\
\hline
\end{tabular}\\
~\\
 \ODK partners had to face some Human resources issues in the past year:
\begin{itemize}
\item{\site{PS}:}
  Thanks to an early start in the recruitment process, and despite
  some difficulties in attracting experienced candidates for a part
  time position, the project manager position (24PM) was filled by
  Benoît Pilorget shortly after the start of the project.

  The recruitment of \site{PS}'s first Research Engineer (48PM) was
  delayed by four months because the top ranked candidate for this
  position, Erik Bray, was originating from the US and needed time to
  arrange for his moving; there were also some administrative delays
  (visa, ...).

  The second Research Engineer position (36PM) was more problematic
  for internal administrative reasons. The top ranked candidate,
  Jeroen Demeyer, had the perfect profile; however for family reasons,
  he wished to work most of the time from Gent in Belgium. After eight
  months investigating an administrative solution to hire him at
  \site{PS}, and a temporary four month solution, it was decided with
  OpenDreamKit's Steering Committee and Project Officer to instead add
  Gent's university as new partner, hire Jeroen Demeyer there, with an
  adequate budget transfer and amendment to the Grant Agreement.

  Those delays have induced late start on several tasks, and costed
  much management time. However the excellence of the recruitment, well
  confirmed by the results obtained so far, was worth it and will soon
  compensate for the late start.

  In addition to this, a three year PhD position was open to work on WP6, starting from
  Month 12. By lack of suitable candidate, this position will be
  converted into a two year PostDoc position, presumably starting at
  Month 24. Active advertising has started and there are some
  tentative candidates. The relevant deliverables being due late in
  the project, no delay is to be expected from this change.\\

\item{CNRS:} Because the research engineer offer (48PM) was still not filled
  in the Summer 2016, the CNRS decided to divide the position in two full
  positions of 24 PM each. As a result, a candidate was already selected for
  one of the two positions and should begin his work this Tall 2016.
  Thanks to the PM division, there should be no delay in any task or
  deliverable. \\

\item{JacobsUni:} Michael Kohlase, lead PI for Jacobs University, has
  moved on 01/09/2016 to Friedrich-Alexander-Universität
  Erlangen-Nürnberg, and most of his team will follow him. Since he is
  a critical asset for OpenDreamKit, a Grant Agreement amendment will
  be submitted in Fall 2016 to update the consortium accordingly.\\

\item{UJF:} The original tentative candidate for UJF's Research
  Engineer position (12PM, planned to start on Month 1), Pierrick
  Brunet, finally declined the position to accept an alternative
  permanent offer. The position will be filled by another candidate in
  Autumn 2016. This induced a delay of Deliverable D5.2 from Month 12
  to Month 18, without impact on other tasks.\\

\item{UNIKL:} UNIKL had to split the 12 PM planned for a software developer into 2 shorter positions (Anders Jensen and Alexander Kruppa) in order to deliver the planned work on time. Indeed the few qualified persons for this job were not able to accept this 12 months position during the timelapse planned within the project.\\

\item{USFD:} The University of Sheffield has also been struggling in the the hiring process of a postdoc (36PM). The position should be filled this Autumn.\\

\item{Southampton:} Southampton faced administrative difficulties in the recruitment of Marijan Beg (38PM) as a post-doc, due to the Croatian nationality of Mr Beg. His recruitment was delayed of four months, and therefore some tasks and deliverables, planned to be borne around the end of the project, were postponed of four months. However no serious delay nor implication on the main tasks of OpenDreamKit followed these difficulties.\\

\item{UVSQ:} Nicolas Gama is currently on a long-term leave until September 2017. This will not affect the project in any way.\\

\item{UZH:} The University of Zürich partner is only composed of one person, Paul-Olivier Dehaye, who does not enjoy a permanent position there. There have been worries that Mr Dehaye's contract with his university might end earlier than planned within OpenDreamKit. But thanks to the action of the OpenDreamKit steering committee, Mr Dehaye's position should be renewed for as long as the project needs.\\

\item{Simula:} Everything is fine concerning temporary staff recruitment on the Simula side, however we have had to endure the hazards of human ressources with Hans-Peter Langtanger (the PI when the Grant was signed) being on a long-term sick leave, and with Martin Alnaes replacing him as PI currently on a paternity leave. However Benjamin Ragan-Kelley has been perfect  substitute leader of the Simula partner and all planned tasks are on time.\\
\end{itemize}


Altogether, this first year confirmed that the recruitment of highly
qualified staff is indeed a risky endeavour, which induced delays on
several deliverables. However the planned mitigation measures --
taking into account the pool of potential candidates in the design of
the positions, aggressive advertisement, weak coupling between tasks
-- worked adequately: with appropriate reshuffling of the work plan,
we don't expect an impact on the overall progress of the project.

\subsection{Different groups not forming effective team}

As expected, this risk was tamed by the existence of many preexisting
collaborations between the partners and of ``joint itches to scratch
together'' (to use a common open source software metaphor). The
organization of many joint workshops (for example the Sage-GAP
workshop, the Atelier Pari attended by SageMath developers, the WP6
workshops) helped bootstrap joint activities through brainstorms and
coding sprints. Upcoming workshops are planned on Year 2 to strengthen
collaborations with the social aspects team in Oxford and the Singular
team in Kaiserslautern.


\subsection{Implementing infrastructure that does not match the needs of end-users}

The consortium is keeping in their minds the end-user needs. Since OpenDreamKit is improving already existent software which have their own users, their needs are naturally met. However Key performance Indicators will evaluate the effects of OpenDreamKit on these software. KPIs, indicated in the Proposal, will be launched this Autumn with the help of the end-user group which was merged with the Advisory Board. Constant links between the accomplished work and the end-user needs should be made in WP2 deliverables and also in WP7 deliverables when relevant.
Open tracking of KPIs evolution can be found on \href{https://github.com/OpenDreamKit/OpenDreamKit/labels/KPI}{GitHub}.

\subsection{Lack of predictability for tasks that are pursued jointly
  with the community}

As planned, we are regularly shifting manpower around to adapt for the
variability of the involvement of the community in the different
tasks. For example, the SageMath Jupyter kernel of
\delivref{UI}{ipython-kernels-basic} was mostly implemented by the
community which allowed to focus on other tasks like the long term
task like~\delivref{component-architecture}{portability-cygwin}.  On
the other hand many other deliverables were implemented with very
little help from the community.

\subsection{Reliance on external software components}

There is not much to report on this front yet: none of the external
software component we rely on have failed us. Quite on the contrary,
critical software like \Jupyter have continued to blossom. Besides the
high modularity of the design means few components are critical to the
overall success of the project.

\section{Quality assurance plan}

\subsection{Deliverables quality: Quality Review Board}
 

 The Quality Review Board is the Consortium Body that fosters best
 possible quality in the deliverables. The body is chaired by Hans
 Fangohr, from the University of Southampton. He is supported in this
 task by Mike Croucher from the University of Sheffield, Alexander
 Konovalov from the University of St Andrews, and by Konrad Hinsen
 from the Centre de Biophysique Moléculaire.

 All board members have a track record of caring about the quality in
 software for computational science, including Mike Croucher's
 outreach and blogs, Alexander Konovalov's engagement with the Software
 Sustainability Institute, Konrad Hinsen's founding and editorship of
 the ReScience Journal, and Hans Fangohr's creation and directorship
 of the UK's only centre for doctoral training in computational
 modelling.

  The quality review board meets after each reporting period, the first
 one for OpenDreamKit ending at month 18 (February 2017), to review
 completed deliverables with focus on quality. The board will choose
 and focus on selected deliverables and review these in greater detail
 rather than attempting a superficial inspection of all deliverables.

 Seeking for continual improvement of the project's processes, the
 board will look for weaknesses, strengths and best-practice used in
 the creation of the deliverables, seeking further information from
 authors of the deliverables. The board will subsequently share their
 findings with the aim of increasing quality of future deliverables
 where possible. The quality review board embraces a no-blame culture
 to foster open exchanges and most-effective use and exploitation of
 their findings in achieving and sustaining high quality outcomes. For their duty,
 board members will benefit from the feedback of the Interim Review (June 2016, Bremen)
where reviewers have advised the \ODK consortium about the deliverables
 content and layout.

 While the primary focus of the board is on the OpenDreamKit project,
 some of the lessons may be more widely applicable and be made
 publicly available.





% % original

% The content form of deliverables due by then had to meet the
% expectations of the Project Officer and of Reviewers. Following this
% experience, we have concluded that deliverables should be written in
% Latex using a style file created for this purpose. For deliverables
% that are not reports by themselves, it's appropriate to have a
% relatively short report with a link to the github issue, and a copy of
% the description of this issue. In all cases, the report shall be
% self-contained. Deliverables are indeed evaluated based upon their
% versions submitted on the EU portal without retrieving other
% resources. Links have no legal value, since there is no guarantee that
% the referenced material will remain unchanged.  Partners who have a
% deliverable due at month 12 (August 2016) have been following these
% tips. The feedback of the Official review will help the Quality review
% Board in ensuring the quality of reports meets the needs. This will be
% up to the Quality Review Board to meet after the 1st reporting period
% and to decide if the quality of deliverables is acceptable.  They will
% aim at identifying good practice and weaknesses, and to share the
% lessons with the project to improve any future project work. The board
% will focus on selected deliverables and investigate those in detail
% rather than attempting a superficial inspection of all deliverables.


\subsection{Infrastructure quality: End-user group}


It was decided by the Steering Committee during the
\href{http://opendreamkit.org/meetings/2015-09-02-Kickoff/management_structure/}{kick-off meeting}
to slightly modify the management structure by having only one
gender-friendly Advisory Board composed of 7 people (as agreed a few
months later at the
\href{http://opendreamkit.org/meetings/2016-06-27-Bremen/minutes/}{Bremen meeting}),
some of which to be end-users.  The end user group was to be replaced
by an informal community, modelled by a public and open mailing list.

Unfortunately, potential Advisory Board Members have not yet been selected. Names have already been raised and
accepted by the consortium as potential members, and the Scientific Coordinator will personally take care of this in September 2016. Concerning the end-user group,
open mailing lists have not succeeded in meeting their public, neither within \ODK nor outside the consortium.
This issue should be soon cleared thanks to the new ODK website to be launched in Autumn 2016. This website will be designed to
raise potential end-users awareness on the project. The different activities and results of \ODK will be made more visible thanks to blog posts, which could be sent out
via a newsletter or via the already existing mailing lists.



\printbibliography

\end{document}

%%% Local Variables:
%%% mode: latex
%%% TeX-master: t
%%% End:

%  LocalWords:  maketitle githubissuedescription newpage newcommand xspace Jupyter dissem
%  LocalWords:  tableofcontents visualizations composability itemize analyzed taskref hpc
%  LocalWords:  dissemination-of-oommf-nb-virtual-environment taskref dissem taskref pn
%  LocalWords:  dissemination-of-oommf-nb-workshops dissem ibook taskref taskref taskref
%  LocalWords:  oommf-python-interface oommf-py-ipython-attributes taskref oommf-nb-ve
%  LocalWords:  oommf-tutorial-and-documentation taskref oommf-nb-evaluation taskrefs
%  LocalWords:  delivref pythran-typing sage-paral-tree subsubsection organized Dagstuhl
%  LocalWords:  co-organized organization modularization ipython-kernels nbdime Pythran
%  LocalWords:  jupyter-collab ystok WPref dksbases compactitem emph WPtref DehKohKon
%  LocalWords:  iop16 textbf tasktref lfmverif triformal formalized biformal ossp09 Dima
%  LocalWords:  hline Marijan Pilorget Pierrick Kruppa Dehaye Dehaye's Dehaye's Alnaes
%  LocalWords:  Konovalov Hinsen github printbibliography
