\documentclass{deliverablereport}

\deliverable{management}{tickets}
\duedate{28/02/2017 (M18)}
\deliverydate{??/??/2017}

\usepackage[style=alphabetic,backend=bibtex]{biblatex}
\addbibresource{../../lib/kbibs/kwarcpubs.bib}
\addbibresource{../../lib/kbibs/extpubs.bib}
\addbibresource{../../lib/kbibs/kwarccrossrefs.bib}
\addbibresource{../../lib/kbibs/extcrossrefs.bib}
\addbibresource{../../lib/deliverables.bib}
% temporary fix due to http://tex.stackexchange.com/questions/311426/bibliography-error-use-of-blxbblverbaddi-doesnt-match-its-definition-ve
\makeatletter\def\blx@maxline{77}\makeatother

\author{Nicolas M. Thiéry, Benoît Pilorget}

\begin{document}
\enlargethispage{4ex}
\maketitle
\githubissuedescription
\tableofcontents\newpage


\section{Introduction}

OpenDreamKit (Open Digital Research Environment Toolkit) is a four-year Horizon 2020 European Research Infrastructure project (\#676541) that will run for four years. It provides substantial funding to the open source computational mathematics ecosystem, and in particular popular tools such as LinBox, MPIR, SageMath, GAP, PARI/GP, LMFDB, Singular, MathHub, and the IPython/Jupyter interactive computing environment.

From this ecosystem, OpenDreamKit will eventually deliver a flexible toolkit enabling research groups to set up Virtual Research Environments (VRE), customised to meet the varied needs of research projects in pure mathematics and applications, and supporting the full research life-cycle from exploration, through proof and publication, to archival and sharing of data and code. 
This means the potential end-user, i.e the person that the software and/or tool is designed for, is a person in need of a toolkit to optimise their work process. This person can be a researcher (whether academic or not), a professor from any field related to mathematics or using mathematics, a public institution etc. As OpenDreamKit is promoting an opensource VRE there is no discrimination as to whom will be allowed to use the latter. It is therefore possible that Small and Medium Enterprises (SME) also become end-users.

However potential end-users are more likely to be people from the scientific community, the main reason for that being the developers of the VRE come from it. Most of the OpenDreamKit participants are indeed academics who often have several hats -developer, mathematician, computing scientist etc.- and who are willing to develop the tools they need for themselves to simplify their work. For that reason OpenDreamKit is unique in the fact the software developers are at the same time end-users as well, with a "by users for users model". The software developers/end-users are also connected outside of OpenDreamKit to the opensource software communities that they are improving and developing. Therefore some innovations brought to the toolkit are partially or entirely accomplished outside of the OpenDreamKit project. This is not a problem in itself as the communities and the project share the same values and objectives.
That being said, targeted end-users comprise the software developer communities but not only, the VRE will also benefit to end-users who are not experts in the field and who simply need tools to enhance their results and improve their processes.

Once the OpenDreamKit project ends, end-users will benefit from innovations in the multiple pieces of software forming the opensource VRE. In the case of OpenDreamKit the innovation will be the unification of opensource tools with overlapping functionality, the simplification of the tools for end-users without coding expertise, and the development of user-friendly interfaces.

The following document will first explain the innovations that OpenDreamKit will bring to end-users, then explain the processes enabling the implementation of the said innovations, and finally will explain how OpenDreamKit will target its end-users.


\section{Innovations OpenDreamKit is bringing to the research community}

OpenDreamKit has several transversal objectives:

\begin{itemize}
\item{Develop and standardise math soft and data for VRE: WP3, WP4, WP5, WP6}
\item{Develop core VRE components: WP3, WP4, WP5, WP6}
\item{Bring together communities: WP2, WP3}
\item{Update a range of softwares: WP3, WP5}
\item{Foster a sustainable ecosystem: WP3, WP4, WP5, WP6}
\item{Explore social aspects: WP7}
\item{Identify and extend ontologies: WP6}
\item{Effectiveness of the VRE: WP2, WP7}
\item{Effective dissemination: WP2, WP7}


\end{itemize}


\section{Implementation processes of the innovations}

The OpenDreamKit project was born by two movements: end-users pushing innovation and developers pulling innovation. But this dual push and pull movement was triggered by the same people since developers and end-users are for the OpenDreamKit VRE, until now, the same people.

Since the goals of OpenDreamKit are well known and supported by the developer communities, what needs be organised is the work process of the innovation. Basically, experience proves that the biggest achievements and best innovations are made when less than a dozen developers join together for a week in a a house, preferably far from urban civilisation, with a solid internet connection. This method which was proven to be very efficient for opensource software communities, is actually close to the two-pizza team rule from Jeff Bezoes who is the founder and CEO of Amazon. According to Mr Bezoes, the more you add people to a group, the less the group is agile and innovative because too much effort is put into communication and management. Following this rule a team should be composed of maximum 6-7 people to be effective .
Now, opensource development workshops don't follow such strict rules as the Silicon Valley companies but they do tend to gather a small amount of people. An example of which is the regular Cernay workshop organised by Nicolas Thiéry around \Sage, the last one being the Sage Days 77 in April 4-8 2016. About 15 people joined this workshop throughout the week. Concerning the impact, proper packaging and distribution has been a recurrent issue for \Sage and is a major task for \ODK (\longtaskref{component-architecture}{mod-packaging}). Major brainstorms occurred during the week to clarify the needs, isolate the core difficulties, and explore potential approaches to tackle them. The outcome was posted on the \href{https://wiki.sagemath.org/days77/packaging}{Sage Wiki}, to be shared and further edited by the community. This fostered tighter collaboration between the packaging efforts for various Linux distribution, and triggered major progress on the Debian packaging side.Several small workshops such as this one are to be organised all project long to speed up the software development process.

That being said, each piece of software involved in OpenDreamKit has its own implementation process, described below.

\subsection{Implementation process of \Sage}

There is no specific sustainable leader in the development of \Sage. Its "community" of developers bases its work on the consensus of the group and on the availability of its members to tackle issues and work on the software development. If a decision doesn't create a consensus, there can be a vote but that seldom happens. 
When a development is brought to the software, it is reviewed by at least one other member of the community.

\subsection{Implementation process of \Jupyter}

The Jupyter project is driven by the \href{https://jupyter.org/about.html}{Jupyter steering council}, as described in the \href{https://github.com/jupyter/governance}{Jupyter governance documents}. This council is composed of 15 members, one being Benhamin Ragan-Kelly who is leader of the Work Package 5 (User Interfaces),  at least two being regular participants to OpenDreamKit events. 

The larger scale mission of the project is decided by the steering council. In most cases, the wider Jupyter community decides what should be done by contributing proposals on an individual basis. Most proposals come in the form of pull requests on GitHub, but larger proposals can be discussed as \href{https://github.com/jupyter/enhancement-proposals}{enhancement proposals} beforehand, and must be approved by the steering council. In general, decisions for additions are handled by the maintainers of existing packages, who are longstanding members of the community or delegates thereof (such as those hired under the OpenDreamKit grant). If there is conflict, decisions can be resolved by the steering council.

Concerning the developments reviews, the Jupyter community does it publicly. The maintainers of each project take on the bulk of the review responsibility. If there are conflicts, they can be brought to the steering council for resolution.


\subsection{Implementation process of \Singular}

Wolfram Decker and Hans Schönemann from the University of Kaiserslautern are considered to be the leaders of the \Singular software. 
There is no specific process to decide how the software should evolve. It can be individual decisions of a single person in need of a service for their research, or decision taken by vote by the core of \Singular developers during a meeting.
Nevertheless the new code is essentially reviewed by Hans Schönemann.

\subsection{Implementation process of GAP}



In addition to this process, one can add that the communities in question use various collaborative tools such as mailing lists, version controln editing pads, wikis etc.

\section{The end-users targeted at by innovations}

As it was previously said, the originality of the OpenDreamKit VRE is its "by users for users" model. Indeed, OpenDreamKit's participants have many hats: they are mathematicians, computer scientists, developers, researchers, professors, end-users and sometimes all of this at the same time. 

But since OpenDreamKit is promoting opensource, it is naturally aiming at reaching out as many end-users as possible. In order to do that OpenDreamKit comprises two boards which can help reach out end-users outside the developer communities.

\subsection{Boards within OpenDreamKit}

\subsubsection{Quality-Review board}

The Quality Review board (QRB) is composed of four members: Hans Fanghor (chair), Alexander Konovalov, Konrad Hinsen and Mike Croucher. They are all turned towards the end-user needs and towards the development of opensource research software.
Some of their objectives are to:

\begin{itemize}
\item{Emphasis the quality of the software engineering aspects}
\item{Identify best practice}
\item{Improve future work within OpenDreamKit and disseminate knowledge to a wider audience, i.e. potential end-users.}
\end{itemize}
  
Thus, the work of the QRB can facilitate access of the end-user to the innovations brought by OpenDreamKit by making sure best practice are respected and the software development is of high quality. But the board specifically turned towards the end-user needs is the Advisory board (AB).

\subsubsection{The Advisory board and its end-user group}

Indeed, the AB which is composed of seven members (Lorena Barba, Jacques Carette, Istvan Csabai, Françoise Genova, Konrad Hinsen, William Stein and Paul Zimmermann) comprises an End-user group among its members. Was convened for this AB an international mix of industry/end-users and academic representatives who understand broad 21st century needs for computational mathematics.
The AB is to give an independent opinion on scientific and innovation matters, in order to guarantee:

\begin{itemize}
\item{Quality implementation of the project}
\item{Efficient innovation management}
\item{Project sustainability.}
\end{itemize}

The Board will include a small End-user Group (3 to 4 persons out of 7) that will control the project execution from the point of view of the end-user needs and requirements. The End-User Group will be connected to an informal community of end-users. This will help tackle the risk that users will not use the VRE.

\subsection{End-users targeted by OpenDreamKit innovations}

Since the biggest risk for the VRE being developped by OpenDreamKit Participants is that end-users do not use the VRE, the challenge is to promote the VRE to potential end-users: in Europe and beyond, in established developper communities, in any research field that could benefit from the VRE. This is the challenge that is being tackled by the WorkPackage 2 "Community building, training, dissemination, exploitation and outreach".


\subsubsection{A worldwide promotion}

As it was previously said, OpenDreamKit is developing and promoting software that are opensource. The universal and cost-free distribution nature of opensource software allows OpenDreamKit to target every country and continent notwithstanding any level of infrastructure development, economic performance or lack of a solid institutional academic network.
In the first year of the project from Sept. 2015 to Sept 2016, 14 events (workshops, schools, etc.) were (co-)organised by OpenDreamKit. Some of them were of course planned in Europe and North America, but others were planned in Africa, South America and Middle East. In the next years OpenDreamKit will continue at the same pace to (co-)organise events in the same geographical regions.
However in some areas where the internet connection does not allow for massive cloud usage, one must find alternative solutions such as distributing the required install files using USB sticks  rather than online repositories. This simple trick enables dozens of undergraduates, master students, PhD students, postdoctorates, teachers and professors following the same given school to start working on a SageMath or Jupyter tutorial at the same time.

Because of the opensource nature of the OpenDreamKit software, everything can infinitely be replicated and shared without any constraint. Therefore in addition and following the events organised by the project, OpenDreamKit is counting on a snowball effect for the reaching out of end-users.



\subsubsection{Established communities}

Since OpenDreamKit participants are all part of at least one of the well established opensource software communities, the communication on OpenDreamKit's achievements comes naturally and easily. Furthermore, many workshops are organised all year-long, whereas they are (co)financed by the project or not, during which the developer communities join together their efforts to improve the software from the OpenDreamKit toolkit. 

\subsubsection{Research fellows outside of established communities}

Furthermore OpenDreamKit was introduced at external events in order to reach out potential end-users outside of the regular opensource software developper communities.
All these activities can be tracked in \longdelivref{dissem}{workshops-1}. 


All the pieces of software improved and developed by the project such as \Sage, \Singular, etc. were presented during these schools and workshops targeted at students and potential new users. Of course this software being originally developed for end-users who are potentially very skilled in computing and software development, one of the goals is to make it easy  for relatively unskilled end-users to start using our opensource toolkit. 
For example, one of the main objectives in the development of \Sage in the attraction of new end-users is to develop the portability of \Sage (and therefore its dependencies) on Windows. With such a tool, the toolkit will enlarge its target to all Windows users in need of software like \Sage but without the time or knowledge to set it up themselves on their computers.

Three major conferences of about 60 to 90 participants are planned within the frame of OpenDreamKit. One out of three was already jointly organised with the Collaborative Computational Project (CoDiMa) at the ICMS (International Center for Mathematical Sciences) of Edinburgh on the 10-16 January 2017with about 50 to 60 participants. The topic was: Computational Mathematics with Jupyter. This conference aimed at enabling users and developers of GAP, \Singular, \Sage and \Jupyter to meet. Talks were given and coding hackathons were organised. The event is currently being evaluated by its organisers. 
The two other conferences will be organised each in 2018 and 2019.

%TODO: Tutorials/Moocs?


\end{document}

%%% Local Variables:
%%% mode: latex
%%% TeX-master: t
%%% End:

%  LocalWords:  maketitle githubissuedescription newpage newcommand xspace Jupyter dissem
%  LocalWords:  tableofcontents visualizations composability itemize analyzed taskref hpc
%  LocalWords:  dissemination-of-oommf-nb-virtual-environment taskref dissem taskref pn
%  LocalWords:  dissemination-of-oommf-nb-workshops dissem ibook taskref taskref taskref
%  LocalWords:  oommf-python-interface oommf-py-ipython-attributes taskref oommf-nb-ve
%  LocalWords:  oommf-tutorial-and-documentation taskref oommf-nb-evaluation taskrefs
%  LocalWords:  delivref pythran-typing sage-paral-tree subsubsection organized Dagstuhl
%  LocalWords:  co-organized organization modularization ipython-kernels nbdime Pythran
%  LocalWords:  jupyter-collab ystok WPref dksbases compactitem emph WPtref DehKohKon
%  LocalWords:  iop16 textbf tasktref lfmverif triformal formalized biformal ossp09 Dima
%  LocalWords:  hline Marijan Pilorget Pierrick Kruppa Dehaye Dehaye's Dehaye's Alnaes
%  LocalWords:  Konovalov Hinsen github printbibliography
