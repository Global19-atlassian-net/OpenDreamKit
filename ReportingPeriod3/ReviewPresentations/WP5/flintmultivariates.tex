%\frame{
%\frametitle{Main achievements}

%\begin{itemize}

%% \item Multivariate polynomials are in the \textsc{Flint} library over the coefficient rings $\mathbb{Z}$, $\mathbb{Q}$, $\mathbb{F}_p$, $\mathbb{F}_{p^n}$ for $p < 2^{64}$. The monomial orderings \emph{lex}, \emph{deglex}, and \emph{degrevlex} are currently supported. The computer algebra system \textsc{Singular} now uses \textsc{Flint} for multiplication, exact division, and gcd over $\mathbb{Q}$ and $\mathbb{F}_p$ for $p>500$.

%% \item The following operations are parallelized over $\mathbb{Z}$, $\mathbb{Q}$ and $\mathbb{F}_p$.
%% \begin{itemize}
%% \item multiplication
%% \item divisibility testing
%% \item greatest common divisor
%% \end{itemize}

%% \item The single-core algorithms are an improvement over the algorithms previously used in \textsc{Singular}.

%% \item The parallelism on large problems is 80\% to 90\% efficient on 8 threads and 70\% to 80\% efficient on 16 threads.

%% \item The library itself is well-tested and offers a consistent interface that allows it to be used for polynomial arithmetic in other computer algebra systems as well.

%\end{itemize}

%}

%%%%%%%%%%%%%%%%%%%%%%%%%%%%%%%%%%%%
\begin{frame}
\frametitle{Main achievements}

%sizes are the term counts in the inputs and output.

%The apparent wall at 32 threads over $\mathbb{Z}$ is not real and is an anomalous reading probably due to the fact that the 32 core server was running 2 other intensive jobs at the time.
  \begin{block}{Multivariate Polynomial \only<1,2>{Multiplication}\only<3,4>{GCD} in \texttt{FLINT}}
    \begin{columns}
      \begin{column}   {.6\textwidth}
%    \hspace{-.8em}
        \vspace{1em}
        \noindent
        \only<1>{\input{sparsePM_plot1}}
        \only<2>{\input{sparsePM_plot2}}
        \only<3>{\input{sparseGCD_plot1}}
        \only<4>{\input{sparseGCD_plot2}}
      \end{column}
      \begin{column}   {.38\textwidth}
        \begin{itemize}
     \item over $\mathbb{Z}$, $\mathbb{Q}$, $\mathbb{F}_p$, $\mathbb{F}_{p^n}$ for $p < 2^{64}$.
     \item Lex, degLex and degGrevLex ordering supported
       \item Sequential alg: improves \texttt{Singular}'s
       \item Close to linear scaling
       \item \texttt{Singular} now relies on \texttt{FLINT}
        \end{itemize}
      \end{column}
    \end{columns}
  \end{block}

\end{frame}


%%%%%%%%%%%%%%%%%%%%%%%%%%%%%%%%%%%%
\frame{
\frametitle{New Perspectives and directions}


%% \item Though the single-core gains to arithmetic in \textsc{Singular} are real, the gains afforded by the thread-level parallelization of these operations are only applicable when the arithmetic operations are large.
%% \begin{itemize}
%% \item The process of splitting up the basic arithmetic task into independent tasks and combining the results is non-trivial and costs overhead.
%% \item Decent scaling of the multiplication to $16$ or $32$ threads requires huge polynomials.
%% \end{itemize}

%% \item Calculations that perform many arithmetic operations on small polynomials cannot expect to gain from our parallelization.
%% \begin{itemize}
%% \item These kind of calculations must and usually can benefit from parallelization at a higher level.
%% \end{itemize}
\begin{columns}
  \begin{column}{.53\textwidth}
\begin{block}{Parallelizing memory intensive kernels}

  \begin{itemize}
  \item Overhead of split and combine
  \item Only amortized for large instances
%  \item Scaling is limited to a few cores on small instances
  \item Parallelization  at a coarser grain
\end{itemize}
  \end{block}
\end{column}

\begin{column}{.45\textwidth}
\begin{block} {Data placement}<2->
  \begin{itemize}
  \item Heavy use of dynamic allocation (\texttt{malloc})
  \item Causes performances fickle
  \item To be used with parsimony
  \item Investigate dedicated  allocators
  \end{itemize}
\end{block}

\end{column}
\end{columns}

\begin{block} <3-> {New directions}
  Application driven:
  \begin{itemize}
  \item More orderings: block, weighted
  \item Factorization: (harnessing most T5.4 contributions)
  \end{itemize}
  
\end{block}
}

%%%%%%%%%%%%%%%%%%%%%%%%%%%%%%%%%%%%%%%%%%%%%%%%%%%%%%%%%%
%% \frame{
%% \frametitle{New Directions}

%% \begin{block}  {Support for more monomial orderings}
%% \begin{itemize}
%% \item Besides Lex, GradLex, RevGradLex,
%% \item block and weighted orderings required by applications in  \textsc{Singular}.
%% \item Code modularity of \textsc{Flint}: limited effort to extend it
%% \end{itemize}
%% \end{block}
%% \begin{block}{Factorization}
%% \begin{itemize}
%% \item Polynomial factorization is an important operation in many algorithms used in algebraic geometry.
%% \item Much of the infrastructure developed in \textsc{Flint} as part of ODK is directly applicable to factorization algorithms.
%% \item This would further improve the performance of \textsc{Singular}'s factorization, which is already quite good.
%% \end{itemize}

%% \end{block}

%% }


%%%%%%%%%%%%%%%%%%%%%%%%%%%%%%%%%%%%%%%%%%%%%%%%%%%%%%%%%%%%%
%%% Local Variables:
%%% mode: latex
%%% TeX-master: "WP5_pres.tex"
%%% End:
