\subsection{Follow-up to comments of the financial officers}

\subsubsection{\site{PS}}

\begin{EUcomment}
  The partner has claimed unforeseen equipment costs and unforeseen
  costs for an implementation contract. A justification should be
  included in the Technical Report.
\end{EUcomment}

Following a recommendation of the Project Officer and the Referees at
the occasion of the First Formal Review, we contracted with external
communication professional to coproduce visual content for our web
site, to better explain and disseminate the project. This included
comics, coproduced with graphic designer Juliette Belin, and high
quality motion graphics video, coproduced with the PixVideo company.
This expense was not foreseen in the Annex 1.

CoCalc.com (Collaborative Calculation) is a Virtual Environment for
collaborative computation, with strong ties with OpenDreamKit. As
planned in Annex 1, we have been using it extensively at Paris-Sud to
assess its features and applicability for Research and Education
purposes. During the first year of OpenDreamKit, we realized that the
free version was too limited for our purposes -- notably to support
and observe its in-class use by colleagues. We therefore decided to
upgrade to the pro version for the remaining duration of the project.
This expense was not foreseen in the Annex 1.

\begin{EUcomment}
  Overspending by 97.22\% (8.75PM) in WP2 and underspending by 46.32\% (11.58PM) in WP6.
  Please explain these deviations in the Technical Report.
\end{EUcomment}
Generally speaking \site{PS} -- WP2 lead -- was very active in
community building and dissemination activities. Following
\site{LEEDS} early termination, \site{PS} took over some of the WP2
work originally planned at \site{LEEDS} (WP2), notably using the freed
financial resources to organize many more dissemination events.

Following the success of WP6's Math-in-the-Middle (MitM) approach for
high level interoperability between systems, \site{PS} got engaged in
a stronger than originally planned collaboration with \site{FAU} and
\site{SA}, notably for MitM integration between GAP and SageMath.

\subsubsection{CNRS}
\begin{EUcomment}
  The partner is underspending significantly in WP1 and WP4 and
  overspending by 96.43\% in WP5. Please explain these deviations in
  the Technical Report.
\end{EUcomment}
The underspending in WP1 is due to an initial overestimation of the
cost of Management.

Similarly to what was already mentioned in the second technical report, the
underspending in WP4 and overspending in WP5 is due to a reallocation of
efforts due to a change of priorities and the specific competences of the
engineer. More specifically, we spend a lot of work on helping the transition
of SageMath from using Python2 to using Python3 which will concretely happen
in the next release before the end of the year.

\subsubsection{Université de Bordeaux (Third party of beneficiary CNRS)}

\begin{EUcomment}
  The partner has claimed effort in WP1 and WP4 in which they are not
  active. These costs are not eligible unless properly explained in
  the Technical Report.
\end{EUcomment}
Efforts have been claimed in WP1 to take into account participation in
the write up of deliverable reports and
the participation of Bill Allombert in the last review meeting.

Efforts have been claimed in WP4 as UB integrated better visualization tools in
PARI/GP (e.g. parallel computation for plotting). This was not planned in the
Proposal but turned out to be a highly desirable feature.

\subsubsection{\site{UJF}}
%CFS for 407,695.18 euro direct costs; and 509,618.98 euro total is NOT SIGNED; the indicated cost of the certificate refers to the amount of the direct costs, instead shall be zero, please correct it; the attached auditors guidelines refer to another project: Grant Agreement number: 681044 –RESSTORE – H2020-PHC-2014-2015/H2020-PHC-2015-single-stage_RTD, please use those for OpenDreamKit Project and they shall be signed by the auditor The ToR signed by both parties is missing.ALREADY DEALT WITH BY THE ADMIN

\begin{EUcomment}
  Please confirm that the adjustment amount of 10,859.29€ from RP1 is
  related to costs and not effort.
\end{EUcomment}
This adjustment is related to effort: the central services of UGA
forgot to include the time sheets for Jean-Guillaume Dumas for years
2015 and 2017 in their statements for RP1.

\begin{EUcomment}
  No explanation provided for other direct costs declared in RP2.
\end{EUcomment}
These costs correspond to expenses already engaged but which were still not being paid by the university at
the time of RP2 evaluation.

\begin{EUcomment}
  Significant underspending in the use of the budget for other direct
  costs. In addition to less face-to-face meetings and travel costs
  resp, there must be other reason(s) for this deviation.
\end{EUcomment}

The under-spending in direct costs is indeed mostly due to smaller
conference travel expenses (44\%) and workshop organization costs
(39.3\%). The original budget had been set according to the general
travel budget guidelines of OpenDreamKit which turned out to largely
overestimate the actual local needs. Factors included the rather
central location of UGA and the limited interaction that the locally
hired engineers needed with other sites. Cl\'ement Pernet also had to
limit his long distance travels to only once per year, to accommodate
with the birth of his two children during the project.

Other minor factors included the increase in the deprecation duration for the
servers bought (due to the change from UJF to UGA regulations) and the absence of auditing costs.

More precisely, the under-spending of roughly 55000 euros splits in the following way:

\begin{itemize}
\item travels: 24310
\item workshop organization: 21623
\item deprecation for servers: 5282
\item Audit costs: 4000
\end{itemize}

\subsubsection{\site{UK}}

\begin{EUcomment}
  The partner is overspending in WP1, WP2 and WP4 by 100\%. A
  justification shall be provided in the Technical Report.
\end{EUcomment}
We believe what is meant here is underspending. For these work packages, in the OpenDreamKit Proposal, 
the \site{KL} project effort in person‑months was set to be 2 PM each. In WP1 and WP2, the relevant tasks were taken care 
of in full by Professor Decker, the lead PI. However, this does not appear in our cost statement since, as written
in the OpenDreamKit proposal,  Professor Decker’s activities within OpenDreamKit (6 PM), including the related 
overhead, were covered by \site{KL} (and were therefore not part of the requested funding). To conclude, there is 
no \site{KL} underspending in WP1 and WP2. With regard to WP4, our contribution to the deliverables was the 
implementation of a Jupyter interface for Singular, as listed as a part of task T1 of WP4. As it turned out, 
such an interface has been created by other researchers in a more general context, and made available to 
Singular and the open source community. To conclude, our contribution to the deliverables in WP4 arrived
in due time, but there was no need to spend OpenDreamKit funds here.

\begin{EUcomment}
  Significant underspending in the use of the budget for other direct
  costs. In addition to less face-toface meetings and travel costs
  resp, there must be other reason(s) for this deviation.
\end{EUcomment}
Indeed, there is a significant underspending here, but this does not mean, that there were less face-toface
meetings than originally planned. To the contrary: 1) International travelling to conferences was typically 
by invitation, with reimbursement covered from the inviting side. 2) Leading largescale international conferences 
took place at TU Kaiserslautern and provided further opportunities to meet, without spending travel money from our 
side (ANTS 2016, ISSAC 2017, and PASCO 2017). 3) The participation in project meetings could often be achieved 
via modern video equipment. Similarly for discussions with researchers from other sites. 4) Due to independent 
developments, the group at Kaiserslautern has significantly increased its size over the past years. 
The considerable additional expertise has drastically reduced both the need to travel to other research centers 
and the need to invite external experts. 5) Within the computer algebra community, it became clear over the past 
years that originally independent developer teams, creating systems for applications in different mathematical 
areas, should join forces, for at least two reasons, the exchange of ideas to solve to  technical problems, and
the development of computational tools for applications across the boundaries of the different areas. As a result, 
quite a number of developer workshops took place in a much larger context, within and without Europe, into 
which in particular problems concerning OpenDreamKit could be embedded. To conclude, there was a significant 
progress in bringing different groups of researchers together, which led to quite a number of success stories, 
but less OpenDreamKit funds were needed.

\subsubsection{\site{UO}}
\begin{EUcomment}
  The partner has declared costs in WP2 and WP4 in which they are not
  active. These costs are not eligible unless properly described in
  the Technical Report.
\end{EUcomment}
Regarding WP2: it was clearly an oversight during the grant preparation, as dissimination and 
outreach was a task where more participants took part than originally planned. In particular Oxford 
was asked to present ODK tools and technologies at a number of international and UK-wide events,
and agreed to such requests, as it felt it is a very important part of the project. This resulted 
in contributing to \longtaskref{dissem}{dissemination} in WP2. As well, it participated in reviewing
new technology, as in \longtaskref{dissem}{tech-review} in WP2.

Regarding WP4: as WP7 was cancelled, the focus of Oxford shifted to other WPs, where its expertise and 
skills were useful and beneficial, such as \longtaskref{UI}{sage-sphinx} and \longtaskref{UI}{vis3d} in WP4.

%for USlaski: Waiting for final submission.

\subsubsection{\site{SA}}
%CFS to be produced.
\begin{EUcomment}
  Significant underspending in the use of the budget for other direct
  costs. In addition to less face-to face meetings and travel costs
  resp, there must be other reason(s) for this deviation.
\end{EUcomment}
Many project-related meetings and events were combined or colocated
with events to which we were travelling for other purposes, supported
by other funding especially CCP-CoDiMa a Uk project which supported
much of our UK travel. Some events we did organise were much less
costly than expected due to lower costs for rooms etc. It also appears
that the  estimation rule for travel costs recommended by the EU and
used in the grant application over estimates the cost of mathematical
conferences and workshops very considerably. For the UK portions of
our costs, the change in exchange rates was also a factor.

Family considerations (especially for Prof.~Linton in 2018 when his
wife was in hospital) also limited travel.

\subsubsection{\site{UV}}
\begin{EUcomment}
  The partner is overspending in WP2 by 137.50\% and has declared
  costs in WP6 in which they are not active. These costs are not
  eligible unless properly described in the Technical Report.
\end{EUcomment}
UVSQ hasn't participated to WP6 during RP3. The participation of UVSQ
to WP6 during RP1 and RP2 has been addressed in the respective technical reports.

Concerning the overspending in WP2 (2.75PM over the course of 4
years), Luca De Feo turned out to be involved in dissemination more
than what was originally planned (on average 2 more
conferences/workshop per year). It should be noted that the format of
some of these conference (Atelier PARI/GP, Sage days, ...) mixes
training sessions with development sessions, it is thus not easy to
delimit the time spent disseminating from the time spent working on
development tasks, e.g., on WP3 and WP4.
Realistically, the overspending is closer to 70\% than to 137\%.

%unless the EU wants researchers to invest in stopwatches;

\subsubsection{\site{UW}}
\begin{EUcomment}
  The partner has claimed effort in WP5 in which they are not active.
  These costs are not eligible unless properly described in the
  Technical Report.
\end{EUcomment}
UWarwich has mistankenly declared costs in WP5, the FS declaration has been corrected and effort has been claimend only in WP1 and WP6. 

\begin{EUcomment}
  Significant underspending in the use of the budget for other direct
  costs. In addition to less face-to-face meetings and travel costs
  resp, there must be other reason(s) for this deviation.
\end{EUcomment}
The largest component of the underspend was the sum of 12k€ which was originally intended for partial support of a workshop joint with our other project, LMFDB.  In the end the LMFDB funds were sufficient and we did not need to call upon this.


\subsubsection{\site{ZH}}
\begin{EUcomment}
  significant underspending in the use of the budget for other direct
  costs. In addition to less face-to-face meetings and travel costs
  resp, there must be other reason(s) for this deviation. The actual
  average personnel costs are significantly higher than planned,
  please clarify the reasons for this deviations (besides the currency
  exchange rate fluctuation).
\end{EUcomment}

At the beginning of the project, Paul-Olivier Dehaye -- unique
OpenDreamKit participant at UZH -- held a temporary position at UZH.
As stated in the proposal, the original plan was for Paul Olivier
Dehaye to spread his involvement (12PM WP6 + 1PM admin) over the whole
duration of the project, under the assumption that Paul-Olivier's
position would be renewed by UZH. This however did not materialize,
and Paul-Olivier's contract at UZH expired at the end of year 1. The
consortium then decided with Paul-Olivier to focus his involvement in
ODK during year 2, staying at UZH under full-time ODK funding.

This had two consequences:
\begin{itemize}
\item Paul-Olivier used travel money during two years instead of four
  years as originally planned. This explains the underspending in the
  other direct costs.
\item Part of the negotiation with UZH to extend Paul-Olivier's stay
  was to recruit him during year 2 as Assistant Professor, a position
  that came with minimal salary requirements that were above the
  budgeted ODK money. This explains the higher than planned average
  personnel costs. This additional expense was taken from the leftover
  direct costs.
\end{itemize}

\subsubsection{\site{LL}}
%CFS to be produced. 
\begin{EUcomment}
  The partner is underspending in WP5 by 95.83\% and has a declared a
  higher monthly average rate (7,889.82 vs. 5,937.75). The actual
  average personnel costs are significantly higher than planned,
  please clarify the reasons for this deviations.
\end{EUcomment}
For the WP5, Logilab subcontracted its work to Serge Guelton (12 PM) and thus  did not charge regular direct personnel costs for this work package. Nevertheless, the partner has declared the subcontracting costs in line (d).

\begin{EUcomment}
  Significant underspending in the use of the budget for other direct
  costs. In addition to less face-to-face meetings and travel costs
  resp, there must be other reason(s) for this deviation.
\end{EUcomment}

\subsubsection{\site{SR}}
\begin{EUcomment}
  The partner should explain the underspending in WP2 and the
  overspending in WP4 in the Technical Report. The partner has also
  declared a much lower monthly rate (6,607.88 vs. 10,085.28). Please
  explain in the Technical Report.
\end{EUcomment}
The first project-related reason for the lower monthly rate was shifting person-months from senior staff with high salaries
to postdoctoral and research engineer staff.
This was caused by a combination of senior staff departing Simula, including one death very early in the project,
and the presence of experienced personnel at lower salary rates.
Additionally, exchange rate forecasts in the budget systematically overestimated the cost of kroner in Euros,
resulting in person months and direct costs being significantly less costly than in the original budget.
These two factors account for the difference in the monthly rate.

Regarding person months:

Simula reported efforts to disseminate WP4 results via workshops and conferences under WP4 instead of WP2.
2 PMs of effort from Simula reported as WP4 should be reported as WP2 instead,
corresponding to these dissemination efforts.

The result of all of these factors was a cost-effective expenditure of person months,
allowing additional effort on WP4 without incurring any additional cost.
Since the work of software that people actually use is never complete,
we were able to use the unspent budget on effort to deliver more mature results in WP4,
actively responding to the user community reactions to work delivered over the course of the project.
The result was improved impact and sustainability of work delivered in Work Package 4,
without any increase in cost.

\begin{EUcomment}
  Significant underspending in the use of the budget for other direct
  costs. In addition to less face-to face meetings and travel costs
  resp, there must be other reason(s) for this deviation
\end{EUcomment}
Organization of the Jupyter Workshop at Simula proved significantly
less costly to Simula, due to
\begin{itemize}
    \item Fewer participants needed travel support than planned (ODK participants used their own budgets, rather than being reimbursed through Simula)
    \item Hosting the event at Simula instead of an external venue, making the venue cost zero
\end{itemize}

No open access publication fees were required because Simula did not submit any articles for publication in journals with open access fees.
WP4 efforts were disseminated through conferences and events which do not have publication fees.

Travel costs were significantly reduced by a number of factors:
\begin{itemize}
  \item On some occasions, hosting institutions and venues paid for travel, so project funds were not required for all trips
  \item Costs were billed in Norwegian kroner, reducing billed totals at the end of reporting periods due to exchange rates
  \item Parental leave and visa challenges resulted in requiring more remote participation
    than travel for parts of RP2 and RP3
  \item The growing concern that frequent academic airline travel is a significant contributor to climate change,
    increasing pressure to collaborate remotely rather than via in-person gatherings
\end{itemize}



%for UGent: Waiting for final submission.

%for XFEL: Waiting for final submission.

% for JacobsUni:
\subsubsection{\site{JU}}

\begin{EUcomment}
  The actual average personnel costs for Jacobs University Bremen GGMBH are significantly lower than planned. Please
  clarify the reasons for this deviations ( besides the currency exchange rate fluctuation).
\end{EUcomment}

The actual average personnel costs of JacobsUniv were significantly
lower than planned. This is a consequence of the fact that personnel
costs significantly depend on the seniority of the employee. We have
been able to hire junior researchers, which were cheaper than the
conservative estimate in the proposal.

\subsubsection{\site{FAU}}

\begin{EUcomment}
in the FS we need a confirmation that the adjusted amount is related to costs not effort. 
The partner should explain the 76.94\% overspending in terms of effort in WP6. The partner is
also using a much lower average rate. These deviations should be explained in the technical report.
\end{EUcomment}

We confirm that the adjusted amount is related to costs not efoort in the FS.
FAU overspent in terms of effort by 76.94\% in WP6 and correspondingly used a lower monthly
average rate. The reason for this is that FAU hired students to do some routine jobs
(simple formalizations, mathematical data curation, interface work for MathHub.info) that
did not require the attention of a mature researchers. As the pay grade of student
assistant is roughly 1/4 of that of full researchers, this action was cost-effective. An
unplanned effect was that the reported person months went up considerably, exceeding the
planned amount, without incurring additional cost.


\subsubsection{\site{LEEDS}}
%Waiting for final submission.
\begin{EUcomment}
  Significant underspending in the use of the budget for other direct
  costs. In addition to less face-to-face meetings and travel costs
  resp, there must be other reason(s) for this deviation.
\end{EUcomment}

Due to all of the \site{LEEDS} participants -- including the PI
himself -- being progressively hired by the industry, all activities
stopped before or early on in the Reporting Period 3. Hence no
other direct costs were claimed for Reporting Period 3.

\begin{EUcomment}
  The actual average personnel costs are significantly higher than
  planned, please clarify the reasons for this deviations (besides the
  currency exchange rate fluctuation).
\end{EUcomment}

As already mentioned, all activities at Leeds unexpectedly stopped at
the end of Reporting Period 2. Following the recommendations of the
referees and project officers during the second formal review, we
sought for opportunities to transfer some of the leftover resources to
other sites to make the best use of them for activities there. This was formalized in
Grant Agreement 5. However the amendment only formalized transfers of
budgets, not of PM's. By way of consequence the Grant Agreement still
planned 22PMs at Leeds for a personnel budget of 25970€, making for a
completely unrealistic average personnel costs of 1.1k€/PM.

The actual average cost for Leeds has been of 3.71k€/PM (9k€ /
2.42PM), while the originally planned average personnel costs for
Leeds before Grant Agreement 5 was of 6.4 k€/PM (141k€ / 22 PM). The
discrepancy is unsurprising: indeed the use of resources at Leeds are
marginal compared to the overall use of resources for Sheffield +
Leeds; the relevant metric is the average costs for Sheffield + Leeds
over the whole duration of the project.

%The following changes are required for the technical part:
%– The deviations explained in the Technical Report include the changes that have been accepted in the amendment. What has been accepted in the last amendment is what should be considered as final figures. Only deviations that are below/ above those should be explained. After the Amendment, Subcontracting is considered foreseen; Task is foreseen as well.
%I. Technical Report
%a. Content: the format of the report deviates from the H2020 template. Please respect the numbering of the chapters as in the template (e.g. chapter 5.1 shall be Tasks; 5.2 Use of resources; 5.2.1 Unforeseen subcontracting; 5.2.2 Unforeseen use of in kind contribution from third party against payment or free of charges, etc.)
%b. Use of resources:
%i. Overall, UNIVERSITE GRENOBLE ALPES, TECHNISCHE UNIVERSITAET KAISERSLAUTERN, THE UNIVERSITY COURT OF THE UNIVERSITY OF ST ANDREWS, THE UNIVERSITY OF WARWICK, UNIVERSITAT ZURICH, LOGILAB, SIMULA RESEARCH LABORATORY AS, and UNIVERSITY OF LEEDS register significant underspending in the use of the budget for other direct costs. In addition to less face-to face meetings and travel costs resp, there must be other reason(s) for this deviation, which need(s) to be addressed in the relevant chapter of the Technical Report.
%ii. The actual average personnel costs for JACOBS UNIVERSITY BREMEN GGMBH, SIMULA RESEARCH LABORATORY AS and FRIEDRICHALEXANDERUNIVERSITAET ERLANGEN NUERNBERG are significantly lower than planned; and for UNIVERSITAT ZURICH, LOGILAB and UNIVERSITY OF LEEDS are significantly higher than planned. Please clarify the reasons for this deviation (besides the currency exchange rate fluctuation).

%%% Local Variables:
%%% mode: latex
%%% TeX-master: "report"
%%% End:
