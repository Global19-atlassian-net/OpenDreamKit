\subsubsection{WorkPackage 2:  Community Building, Training, Dissemination, Exploitation, and Outreach}
\label{dissem}
%Explain, task per task, the work carried out in WP during the reporting period giving details of the work carried out by each beneficiary involved.

\ednote{@nthiery, @IzabelaFaguet, @VivianePons: update WP2 for RP3
  - Update T2.1, notably production of comics and videos
  - update the figures in the overview, T2.3, and T2.5; highlight e.g. CIRM
  - ...?
}


%%%%%%%%%%%%%%%%%%%%%%%%%%%%%%%%%%%%%%%%%%%%%%%%%%%%%%%%%%%%%%%%%%%%%%%%%%%%%%
\paragraph{Overview}

  We have continued the work started in period one, especially on \longlocaltaskref{dissem}{dissemination-communication} and
  \longlocaltaskref{dissem}{dissemination}, organizing or participating in more than 60 events throughout the last two years.
  In particular, we have intensified our efforts in \textbf{training} the community to use the tools developed through \ODK.
  Indeed we have had 36 workshops or events directly organized by \ODK on subjects such as Jupyter, JOOMF (see \longlocaltaskref{dissem}{dissemination-of-oommf-nb-virtual-environment}), PARI/GP, and more. On top of that, we were also part of 5 SageDays.

  One of our priorities is to increase the diversity of the open-source community in science. We have been organizing
  and supporting initiatives to support women developers and scientists such as: Women in Sage, Code First: Girl, and
  PyLadies. We continued to organize such events targeted at women and communities from developing countries: Colombia, Mexico, Morocco, Nigeria.

  Finally, this third period was also an occasion to improve our online communication,continuously updating the website with new content to share demos, facilitate adoption of project results by the users, and to support best practices.
  Following the advice of our last review, %  recommendations after the RP2 review, have they been followed or not?
%greater attention must be paid to acknowledgement of EU funding in all areas ( give examples)
% include the name of the project in Software carpentry, related projects???
% develop a comic explaining the MitM approach
% to disseminate the adoption by Logipedia of the MitM principle of intagrating (logical) systems by aligning concepts
%some guidelines for using the different tools provided by ODK were recommended by the EC in the result of the review letter
% some guidelines for using the different hardware architectures were recommended
  
  Indeed, we collaborated with students in communication to design a new website. We have also conducted interviews to explain
  the key points of the project and we are working on some multimedia content.% publication of our explainer video made by Pix Videos cie ?


%%%%%%%%%%%%%%%%%%%%%%%%%%%%%%%%%%%%%%%%%%%%%%%%%%%%%%%%%%%%%%%%%%%%%%%%%%%%%%
\paragraph{Tasks}

\subparagraph{\longtaskref{dissem}{dissemination-communication}}
\label{dissem@dissemination-communication}

We have followed the advice from our reviewers and have worked at a better communication:
\begin{itemize}
\item We have worked in collaboration with a master degree in web communication and design. The
\ODK website was the year project of a team of three students who delivered different reports to us, some
new communication ideas and a whole new website design. We implemented the new design ourselves and
organized a small workshop to think as a group about the website organization and content.
\item To reach a wider audience, we decided to use different media for communication. We hired an interviewer
and created short videos to share with our views on project's goals and achievement with our communities. These
video are in final stage of edition and will be released shortly and made available on the website.

\item We have worked with (motion) graphic designers to create
  infographic content for our project. The infography gives a clear
  and fast way to understand the tools we are developing. A first
  sketch illustrating one of our use cases will be posted on the web
  site by the review. More sketches will be posted later this fall and
  will serve as a basis for an explainer video to be produced by
  PixVideos over the winter.
\end{itemize}

%Press Releases,  comics and  explainer Video. Press Releases were considered an important dissemination and communication tool at the start of the project and will also be at the end. During the very first year, the project has covered six press releases with a general communication about the project.To promote ODK innovative method and highlight its results to a general public, we plan to submit press releases after our Final project review, at the beginning of November. The procedure for the press release production and distribution is still under revision. The text proposal was made available to all the partners  inviting them to finalize its publication through their press offices. The final press releases will be published in French,  the Coordinator’s and 3 partners main language but also translated in English for the others beneficiaries to enable its publication in local media. We also plan to send this article proposal to our European communication officer to publish it in the EC newsletter and submit it for publication in the Horizon magazine.  These Press releases will  be  addressed  to   the general press in the high education, research area but also in local press , also to audiences that do not require a detailed knowledge of the work carried out.  Also to raise interest of the General public on the project topic and its impact, our communication strategy was accompanied   by   audio-visually   enhanced   materials targeted at non-specialist general public:-	we authored (with the help of experts) several explainer comics and life and motion-design videos that have been reused in a variety of contexts to promote Binder.......-	In  order  to increase  the  visibility  and  public  acknowledgement  of  the  ODK-EU  project,  we created a 2 minutes motion graphic explainer video with Pix Videos, based on the sketches created by Juliette Belin from Logilab.


\subparagraph{\longtaskref{dissem}{training-portal}}

Training is a core and transversal aspect of our project. It is carried out
through interventions and events as we discuss in \longlocaltaskref{dissem}{dissemination}. These
past two years especially, there is been a very strong effort from the \ODK team to organize
training events and workshops. 

We are also working on creating better content on our webpage
to help users understand in what aspect of their work can \ODK help. This is why we have create the
``Use Cases'' section.
%The website for the project has been continuously updated with new content, and virtually all work in progress was openly accessible. It  became  a  repository  for  a  wide  type  of information and communication material. It was used to update on new technical results, and events that might be of interest for our targeted communities, and also to help to share the demos experience, facilitate adoption of project results by the users, to support best practices...
\subparagraph{\longtaskref{dissem}{devel-workshops}}
\label{dissem@devel-workshops}

Development workshops are a key aspect of OpenDreamKit development model. The aim of these workshops is to bring together developers from the different communities to design and implement some
of the wanted features%such as user interface, and documentation and to ensure cross compatibility. 
As reported in \longdelivref{dissem}{workshops-3}, we have organized
or co-organized 16  of these workshops throughout years 3 and 4 of the project. The thematics varies
for each event: PARI/GP, Linbox, and many cross-thematic events such as GAP-Sage and GAP-Jupyter days,
live structured documents, low level libraries, and more.%Those were aimed at a specific software componeents(Sage, GAP, SageMathCloud, IPython, Singular, etc.) and to improve joint developments. It fostered collaboration between scientists and developers from different backgrounds to build tools that are needed by all. These workshops were essential in order to disseminate our work while improving it. 

\subparagraph{\longtaskref{dissem}{tech-review}}

This task has been started during period one, especially through \delivref{dissem}{techno}. We continue
to keep track of new technologies and report by writing blog posts on our website.


\subparagraph{\longtaskref{dissem}{dissemination}}
\label{dissem@dissemination}

Dissemination is a key aspect of the success of OpenDreamKit. Indeed, our development is carried
out to help and support mathematical communities. One of the goals is to bring
more users and more developers to the different projects we are involved in. The events
that took place during Years 3 and 4 have been reported in \longdelivref{dissem}{workshops-3}.% add the list for year 4

\begin{compactitem}
\item \textbf{Training workshops and events.} This has been the most important aspect of this task
for the past two years. We have been organizing 30 events covering subjects such as: Jupyter, JOOMMF,
Bioinformatics, HPC, PARI/GP, GAP, experimental mathematics, web data, reproducible workflows.
\item \textbf{Organization of Sage Days in established mathematical communities.} Sage Days have long been
part of the SageMath tradition. By organizing and supporting Sage Days, OpenDreamKit can stay close
the mathematical community, understand its needs, gather more users and developers, and improve
the overall quality of the software. We have been involved in 15 such events since the beginning
of the project.
\item \textbf{Training activities in developing countries.} \ODK has been present in Colombia for the second time at
the conference ECCO. We also organized SageDays workshops in Mexico, Nigeria and PARI/GP event in Morocco.
\item \textbf{Women in \ODK.} The organization of Women in Sage started in January 2017, two female \ODK participants
have given time and energy to this specific topic. Viviane Pons has continued the organization of the \textit{Pyladies Paris} workshops for the past two years. Following the success of the last Women in Sage event, she organized a second one in Crete on April 2019, 20 women ( 15 PhD master students) received one-week training in SageMath and good practice for sharing code . 
\end{compactitem}
%% During Year 3 and 4, in order to strongly communicate our activities and make them known we organized 30 training workshop where anyone interested were welcomed to learn about ODK and see a demonstration of what ODK components can do. One of the main purposes of our disseminative role was to reach users and foster diversity. It is in this spirit that we also organized 15 Sage Days to  promote our tools and bring more users and developers from the scientific world. by achieving the T2.5 task we succeed to gather and train more than 1300 users and foster the scientific development around ODK. During Year 4,  one of our main dissemination event was the CIRM conference, in Marseille  that welcomed around 50 participants and had a big impact on the scientific community. It brought together users and developers of most ODK software componenets and consisted of keynote talks and  tutorials with a focus on the development of best practices. Another important dissemination event was the Sage Days 105, organized as a satellite event to the main yearly international conference  on algebraic combinatorics involving 50 participants. It featured several tutorial demos and presentations by ODK participants including best practices. In the same spirit, were also organized organized  our fifth targeted at women event on April 2019 to reduce gender gap in mathematics software developments. All the material we developed for presentation at all events organised throughout the project were made publicly available.The impact of development and training workshops was the awareness rising of project results and of the possibilities to strenghten our collaborative open source development model. 
\subparagraph{\longtaskref{dissem}{project-intro}}

\ednote{@mikecroucher, @trallard, @fangohr: proofread brief overview of work done on T2.6: Introduce OpenDreamKit to Researchers and Teachers}

Training and disseminating to Researchers and Teachers is at the heart
of OpenDreamKit and the participants doubled up their efforts during
the last reporting period. This included the organization of training
events (see \longtaskref{dissem}{dissemination} above), but also many
more evaluation and dissemination activities: teaching with
OpenDreamKit technology (thereby training students and other
instructors alike), local consulting, contributing course material and
templates, etc. This is reported on in
\longdelivref{dissem}{IntroODK}, together with some reflection on the
lessons learned at the occasion of these activities: adoption,
adequateness for the needs, best practice.

It should be noted that Sheffield (now Leeds) has been the lead on
this task until its participants got compelling opportunities in the
industry in Fall 2018. This did not reduce the overall dissemination
activities of the project: indeed, the freed resources were
redistributed to other participants that were eager to organize more
activities than originally planned. There was some impact however:
with continued leadership some more of the lessons learned at the
occasion of those activities could have been formally collated, when
currently many are in the state of shared folklore. Luckily this
information is still spreading in the community through many channels:
informal discussions, blog posts, mailing lists, etc.


\subparagraph{\longtaskref{dissem}{dissemination-of-oommf-nb-virtual-environment}}
\label{dissem@dissemination-of-oommf-nb-virtual-environment}

\ednote{@fangohr: update for RP3: brief overview of work done on T2.7, T2.8 micromagnetism VRE}

This task was mostly carried out during the first period. The Ubermag
(previously called JOOMMF) project is working and available on GitHub
(\href{https://github.com/ubermag}{Ubermag repo}). For each Ubermag
package we use continuous integration on both Travis CI and AppVeyor,
where we perform tests and monitor the test coverage, which we then
make available on \href{https://codecov.io/}{Codecov}. Documentation
for each package consists of APIs (automatically generated from the
code) and different tutorials created in Jupyter notebooks. Both of
them are tested on Travis CI. Documentation is built and made publicly
available on \href{http://discretisedfield.readthedocs.io}{Read the
  Docs}. After every major milestone, we upload each package to the
Python Package Index repository and build a Conda package, which can
later be easily installed on different operating systems. We encourage
the early use of our software and invite for feedback for which we
provide several different communication channels. Ubermag can also be
used in the cloud as a Virtual Research Environment, by using Binder
services.

\subparagraph{\longtaskref{dissem}{dissemination-of-oommf-nb-workshops}}
\label{dissem@dissemination-of-oommf-nb-workshops}

We had several workshops and tutorials during major events where we demonstrated the use of our Micromagnetic VRE, received feedback and feature requests from the community:

\begin{compactitem}
\item IOP Magnetism in April 2017, univ. of York.
\item Intermag in April 2017, Dublin.
\item MMM in November 2017, Pittsburgh.
\item Advances in Magnetism in February 2018, Italy.
\end{compactitem}

\subparagraph{\longtaskref{dissem}{ibook}}
\label{dissem@ibook}

In \longdelivref{dissem}{ibook3c} we report on the delivery of two new
open interactive textbooks. Together with the two books delivered
during RP2 (\longdelivref{dissem}{ibook1}), this was the occasions to
explore various approaches to exploit OpenDreamKit technology for
authoring textbooks. In the deliverable report, we reflect on their
respective merits and suggest some best practice.

\subparagraph{\longtaskref{dissem}{index-librorum-salvificorum}} Not
applicable for this period.  The web toolkit \textit{planetaryum}
(\delivref{dissem}{ils-tool}) has been delivered in the 2nd reporting
period, closing the task.

  
%%% Local Variables:
%%% mode: latex
%%% TeX-master: "report"
%%% End:

%  LocalWords:  subsubsection dissem longtaskref organized co-organized longdelivref emph
%  LocalWords:  Jupyter compactitem dissemination-of-oommf-nb-virtual-environment Piwik
%  LocalWords:  delivref centralized cocalc Cython Pythran textbf organizing EuroScyPy
%  LocalWords:  nbgrader nbgrader specialized Codecov joommf-news Micromagnetic Intermag
%  LocalWords:  dissemination-of-oommf-nb-workshops Fruehjahrstagung Sagecell taskref
%  LocalWords:  adcomp index-librorum-salvificorum
