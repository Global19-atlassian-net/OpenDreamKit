\subsubsection{WorkPackage 2:  Community Building, Training, Dissemination, Exploitation, and Outreach}
\label{dissem}
%Explain, task per task, the work carried out in WP during the reporting period giving details of the work carried out by each beneficiary involved.

%%%%%%%%%%%%%%%%%%%%%%%%%%%%%%%%%%%%%%%%%%%%%%%%%%%%%%%%%%%%%%%%%%%%%%%%%%%%%%
\paragraph{Overview}

We continued the line of work from the previous reporting periods,
especially on \longlocaltaskref{dissem}{dissemination-communication}
and \longlocaltaskref{dissem}{dissemination}: we organized or
participated in about 30 events throughout the last year, including
the large dissemination conference \emph{Free Computational
  Mathematics} at CIRM in February 2019. Over the whole project, this
accumulates to 110 events.

We have also pursued our effort towards greater diversity in the
open-source community, organizing the first ever \Sage workshop in
Nigeria as well as our second Women in Sage event in Archanes, Crete.

About our communication, we continued our work to reach the community
through multimedia. We produced comics with Juliette Belin from
Logilab and short motion graphics videos with the Pix Videos company
to explain the project and common use cases. This self-explanatory
visual material reinforces our website to keep informing the community
about the project long after it is officially over.

%%%%%%%%%%%%%%%%%%%%%%%%%%%%%%%%%%%%%%%%%%%%%%%%%%%%%%%%%%%%%%%%%%%%%%%%%%%%%%
\paragraph{Tasks}

\subparagraph{\longtaskref{dissem}{dissemination-communication}}
\label{dissem@dissemination-communication}

 Press Releases were considered an important dissemination and communication tool at the start of the project and will also be at the end. During the  first year, the project has covered six press releases describing the general goals of the project. To promote \ODK innovative method and highlight its results to the general public, we plan to submit press releases at the beginning of November, after the Final review meeting. The procedure for the press release production and distribution is still under revision. The text proposal was made available to all the partners  inviting them to finalize its publication through their press offices. The final press releases will be published in French, the Coordinator’s and 3 partners main language, but also translated in English for the others beneficiaries to enable its publication in local media. We also plan to send this article proposal to our European communication officer to publish it in the EC newsletter and submit it for publication in the Horizon magazine.  These Press releases will  be  addressed  to   the general press in the high education, research area but also in local press, to audiences that do not require a detailed knowledge of the work carried out.

 Beside that, to raise interest of the scientific community on the project topic and its impact, our communication strategy was accompanied by audio-visually enhanced materials targeted at non-specialist. With the help of Pix Videos, we created several explainer comics and life motion-design videos based on the sketches by Juliette Belin from Logilab. These multimedia creations describe different common use-cases of tools either directly developed by \ODK or that can be used in conjunction with our software. As an example, it describes how to use Binder (an external tool) and \Jupyter (developed by \ODK) in the context of scientific collaboration. We plan to use those videos and comics to promote open source software now that the project itself is ending.



\subparagraph{\longtaskref{dissem}{training-portal}}


The website for the project has been continuously updated with new content, and virtually all work in progress was openly accessible. It  became  a  repository  for  a  wide  type  of information and communication material. It was used to update on new technical results, and events that might be of interest for our targeted communities, and also to help to share the demos experience, facilitate adoption of project results by the users, to support best practices, \emph{etc.}.
Some of our ``Use Cases`` were illustrated by comics designed by
Juliette Belin, giving a quick overview on how to use some of \ODK
tools.
\ednote{@nthiery%The website for the project has been continuously updated with new content, and virtually all work in progress was openly accessible. It  also  became  a  repository  for  a  wide  type  of information and disseminative material.  In line with our philosophy for software, we still believe in the benefits of sharing the material and maximising the value of the financial investment into the project. In the central ODK training portal that has been created during the first reporting period, we continued to a made available training materials hosted on the project websites.... give examples for RP3
\subparagraph{\longtaskref{dissem}{devel-workshops}}
\label{dissem@devel-workshops}

Development workshops are a key aspect of OpenDreamKit development model. The aim of these workshops is to bring together developers from the different communities to design and implement some
of the wanted features such as user interface, and documentation and to ensure cross compatibility.
As reported in \longdelivref{dissem}{workshops-4}, we have organized
or co-organized 7  of these workshops throughout year 4 of the project. The thematic varies
for each event: \PariGP, \Linbox, Data, and cross-thematic events such as \GAP-\Sage and \GAP-\Singular days. They were aimed at a specific software components to improve joint developments. It fostered collaboration between scientists and developers from different backgrounds to build tools that are needed by all. These workshops were essential in order to disseminate our work while improving it.

\subparagraph{\longtaskref{dissem}{tech-review}}

By nature, most of the work on this task occurred in the earlier
reporting periods, especially through \longdelivref{dissem}{techno}.
Of course, we continued keeping track of new technologies and writing
about them on our website.

\subparagraph{\longtaskref{dissem}{dissemination}}
\label{dissem@dissemination}

During Reporting Period 3, we organized 10 more training workshops on
various components of \ODK such as \Sage, \Jupyter, \GAP, Ubermag and
more, to disseminate them to the scientific community. Other the four
years of OpenDreamKit, this accumulate to 45 training events and about
1800 attendees.


One of our main dissemination event was the CIRM conference \emph{Free Computational Mathematics}, in Marseille which was aimed at the general scientific community.  It was an occasion to showcase  many of the tools developed and supported by \ODK and to promote the spirit of collaboration, free software, and best practices. We had 58 participants mixing different level of expertise, from newcomers to advanced developers, and different software communities (\GAP, \Jupyter, \Linbox,  \MPIR, \PariGP, \Sage, \Singular). Another important dissemination event was the Sage Days 105, organized as a satellite event to the main yearly international conference  on algebraic combinatorics involving 50 participants. It featured several tutorial demos and presentations by \ODK participants including best practices.

We pursued our effort towards better diversity in the open source software community. We widely advertised our
  \emph{Free Computational Mathematics} conference to reach a large audience and provided funding to many attendees.
  This allowed in particular the attendance of three researchers from University of Ibadan, Nigeria. They were very enthusiastic
  about the conference and it was then decided to organize a \Sage workshop directly in Ibadan for the benefit of Nigerian and
  West-African mathematical community. This event happened in July 2019 and welcomed 80 participants, mostly from Nigeria and
  neighboring countries.
  See also Section~\ref{diversity_success_stories} for some details.

  We also had another \emph{Women in Sage} event, following the one we organized in Paris in 2017. The event was co-organized
  by Viviane Pons from \ODK and Eleni Tzanaki, who attended the 2017 Women in Sage workshop. It was held in the village of Archanes
  in Crete. We welcomed 22 women from 8 different countries, many of them \Sage beginners. The organization of the event was also
  an occasion for a series of \Sage lectures at the mathematics department of University of Crete which initiated the inclusion of \Sage
  in the students curriculum.


All the material we developed for presentation at all events organized throughout the project were made publicly available. The impact of development and training workshops was the awareness rising of project results and of the possibilities to strengthen our collaborative open source development model.
\smallskip
\subparagraph{\longtaskref{dissem}{project-intro}}

Training and disseminating to Researchers and Teachers is at the heart
of OpenDreamKit and the participants doubled up their efforts during
the last reporting period. This included the organization of training
events (see \longtaskref{dissem}{dissemination} above), but also many
more evaluation and dissemination activities: teaching with
OpenDreamKit technology (thereby training students and other
instructors alike), local consulting, contributing course material,
templates and utilities. This is reported on in
\longdelivref{dissem}{IntroODK}, together with some reflection on the
lessons learned at the occasion of these activities: adoption,
adequateness for the needs, best practice.

It should be noted that Sheffield (now Leeds) has been the lead on
this task until its participants got compelling opportunities in the
industry in Fall 2018. This did not reduce the overall dissemination
activities of the project: indeed, the freed resources were
redistributed to other participants that were eager to organize more
activities than originally planned. There was some impact however:
with continued leadership some more of the lessons learned at the
occasion of those activities could have been formally collated, when
currently many are in the state of shared folklore. Luckily this
information is still spreading in the community through many channels:
informal discussions, blog posts, mailing lists, etc.

\smallskip
\subparagraph{\longtaskref{dissem}{dissemination-of-oommf-nb-virtual-environment}}
\label{dissem@dissemination-of-oommf-nb-virtual-environment}

This task was mostly carried out during the first reporting period. The Ubermag
(previously called JOOMMF) project is working and available on GitHub
(\href{https://github.com/ubermag}{Ubermag repo}). For each Ubermag
package we use continuous integration on both Travis CI and AppVeyor,
where we perform tests and monitor the test coverage, which we then
make available on \href{https://codecov.io/}{Codecov}. Documentation
for each package consists of APIs (automatically generated from the
code) and different tutorials created in Jupyter notebooks. Both of
them are tested on Travis CI. Documentation is built and made publicly
available on \href{http://discretisedfield.readthedocs.io}{Read the
  Docs}. After every major milestone, we upload each package to the
Python Package Index repository and build a Conda package, which can
later be easily installed on different operating systems. We encourage
the early use of our software and invite for feedback for which we
provide several different communication channels. Ubermag can also be
used in the cloud as a Virtual Research Environment, by using Binder
services.

\smallskip
\subparagraph{\longtaskref{dissem}{dissemination-of-oommf-nb-workshops}}
\label{dissem@dissemination-of-oommf-nb-workshops}

We had several workshops and tutorials during major events where we demonstrated the use of our Micromagnetic VRE, received feedback and feature requests from the community:

\begin{compactitem}
\item IOP Magnetism in April 2017, univ. of York.
\item Intermag in April 2017, Dublin.
\item MMM in November 2017, Pittsburgh.
\item Advances in Magnetism in February 2018, Italy.
\end{compactitem}

\smallskip
\subparagraph{\longtaskref{dissem}{ibook}}
\label{dissem@ibook}

In \longdelivref{dissem}{ibook3c} we report on the delivery of two new
open interactive textbooks. Together with the two books delivered
during RP2 (\longdelivref{dissem}{ibook1}), this was the occasions to
explore various approaches to exploit OpenDreamKit technology for
authoring textbooks. In the deliverable report, we reflect on their
respective merits and suggest some best practice.

\smallskip
\subparagraph{\longtaskref{dissem}{index-librorum-salvificorum}} Not
applicable for this period.  The web toolkit \textit{planetaryum}
(\delivref{dissem}{ils-tool}) has been delivered in the 2nd reporting
period, closing the task.


%%% Local Variables:
%%% mode: latex
%%% TeX-master: "report"
%%% End:

%  LocalWords:  subsubsection dissem longtaskref organized co-organized longdelivref emph
%  LocalWords:  Jupyter compactitem dissemination-of-oommf-nb-virtual-environment Piwik
%  LocalWords:  delivref centralized cocalc Cython Pythran textbf organizing EuroScyPy
%  LocalWords:  nbgrader nbgrader specialized Codecov joommf-news Micromagnetic Intermag
%  LocalWords:  dissemination-of-oommf-nb-workshops Fruehjahrstagung Sagecell taskref
%  LocalWords:  adcomp index-librorum-salvificorum
