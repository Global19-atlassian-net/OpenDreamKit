\subsection{Quality assurance plan}
\label{section.QAP}

\subsubsection{Deliverables quality: Quality Review Board}

The Quality Review Board is the Consortium Body that fosters best
possible quality in the delivered work of the project.
All four members of the board
have a research interest in the quality of software in computational
science, and use and share their experience to benefit the quality of
the work.

The board was chaired by Hans Fangohr, from the University of
Southampton and European XFEL GmbH (Germany). He is supported in this task by
Mike Croucher from the University of Leeds and now
Numerical Algorithm Group (UK), Alexander Konovalov from
the University of St Andrews (UK), and by Konrad Hinsen from the Centre de
Biophysique Moléculaire (France) with whom a Non-Disclosure Agreement was
signed.

These board members engage with European initiatives working towards
improvement of the software quality in research, in particular in
computational and data science; both as voluntary activities and key
of their professional roles. Mike Croucher was the head of research
computing at Leeds, and is well known through his outreach blog; Alexander Konovalov
is a fellow of the Software Sustainability Institute and an active
member of the Software Carpentry community; Konrad Hinsen has
founded and is editing the ReScience Journal for reproducible Science,
and Hans Fangohr is the founder of the UK's only centre for
doctoral training in computational modelling with focus on software
engineering training for scientists, a fellow of the Software
Sustainability Institute, was chairing the EPSRC's national scientific
advisory committee on high performance computing, is heading
data analysis infrastructure development at the European XFEL research
facility, and leading the data analysis work package in the Photon and
Neutron Open Science Cloud H2020 project that works towards
implementation of the European Open Science Cloud.

The quality review board has reviewed deliverables after the reporting
period 1 and 2, identified good practice - both in terms of software
engineering content but also presentation of the work -, produced
reports, and shared the findings with all members in the project to
improve the quality of the remaining deliverables. An improvement of
the quality of deliverable reports at the end of reporting period 2 in
comparison to reporting period 1 was noted. The reports are available
on request. The board has stuck to its no-blame
culture in its reporting, but has pointed out deliverable reports of
very high quality.



\subsubsection{Infrastructure quality: End-user group}


It was decided by the Steering Committee during the
\href{http://opendreamkit.org/meetings/2015-09-02-Kickoff/management_structure/}{kick-off
  meeting} to slightly modify the management structure by having only
one gender-friendly Advisory Board composed of 6 people (as agreed a
few months later at the
\href{http://opendreamkit.org/meetings/2016-06-27-Bremen/minutes/}{Bremen
  meeting}), some of which to be end-users.

Members of the board are: Jacques Carette from the McMaster University, Istvan Csabai from the Eötvös University Budapest,
Françoise Genova from the Observatoire de Strasbourg, Konrad Hinsen from the Centre de Biophysique Moléculaire,
William Stein who is CEO of SageMath, Inc. (SME), and Paul Zimmermann from INRIA.

%%% Local Variables:
%%% mode: latex
%%% TeX-master: "report"
%%% End:

%  LocalWords:  subsubsection Dissemmination ldots sec:orgc218a3a
%  LocalWords:  github Csabai
