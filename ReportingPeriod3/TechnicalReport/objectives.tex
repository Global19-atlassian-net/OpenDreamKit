\subsection{Explanation of work carried out per Objective}
%List the specific  objectives  for  the  project  as  described  in  section  1.1  of  Part B   and describe  the  work  carried  out  during  the  reporting  period  towards  the  achievement  of  each listed objective. Provide clear and measurable details.
For reference, let us recall the aims of \ODK.
\begin{compactenum}[\bf {A}1\rm:]
\item \label{aim:collaboration} Improve the productivity of
  researchers in pure mathematics and applications by promoting
  collaborations based on mathematical \textbf{software},
  \textbf{data}, and \textbf{knowledge}.
\item \label{aim:vre} Make it easy for teams of researchers of any
  size to set up custom, collaborative \emph{Virtual Research
    Environments} tailored to their specific needs, resources and
  workflows. The \VREs should support the entire life-cycle of
  computational work in mathematical research, from initial
  exploration to publication, teaching and outreach.
  % and bridge the gaps between
  % code, published results, and educational material.
\item \label{aim:sharing} Identify and promote best practices in
  computational mathematical research including: making results easily
  reproducible; producing reusable and easily accessible
  software; sharing data in a semantically sound way; exploiting and
  supporting the growing ecosystem of computational tools.
\item \label{aim:impact} Maximise sustainability and impact in
  mathematics, neighbouring fields, and scientific computing.
\end{compactenum}

Those aims are backed up in our proposal by nine objectives; we now
highlight our main contributions during this reporting period toward
achieving each of them.

\begin{compactenum}[\bf O1\rm:]
\item\label{objective:framework} ``\emph{To develop and standardise an architecture
    allowing combination of mathematical, data and software components with off-the-shelf
    computing infrastructure to produce specialised \VREs for different communities.}''

  This objective is by nature multilevel; achievements include:
  \begin{itemize}
  \item Collaborative workspaces: major \JupyterHub developments,
    see~\longlocaltaskref{UI}{notebook-collab}; study and documentation of the \SMC
    architecture, see \longlocaltaskref{component-architecture}{extract-smc};
  \item User interface level: enabling \Jupyter as uniform interface for all computational
    components; see \longlocaltaskref{UI}{ipython-kernels}.
  \item Interfaces between computational or database components: short term: refactoring
    of existing ad-hoc interfaces, see \longlocaltaskref{UI}{pari-python}; long term:
    investigation of patterns to share data, ontologies, and semantics uniformly across
    components, see \longlocaltaskref{component-architecture}{interface-systems}, and
    Section~\ref{dksbases} about \WPref{dksbases}, where we report on the
    ``Math-in-the-Middle'' (MitM) paradigm for semantic system integration and non-trivial
    mathematical use cases. 
  \end{itemize}

\item\label{objectives:core} ``\emph{To develop open source core components
  for \VREs where existing software is not suitable. These components
  will support a variety of platforms, including standard cloud
  computing and clusters. This primarily addresses Aim~\ref{aim:vre},
  thereby contributing to Aim \ref{aim:collaboration}
  and~\ref{aim:sharing}.}''

At this stage, it has been possible to implement most of the required developments within
existing components or extensions thereof. New software components includes the tools
nbmerge, nbdiff and nbval (see \delivref{UI}{jupyter-test} and
\delivref{UI}{jupyter-collab}), and planetaryum (see \delivref{dissem}{ils-tool}). For the
Math-in-the-Middle paradigm for semantic system interoperability we have developed
knowledge-based Mediator based on the MMT system. 

\item \label{objective:community} ``\emph{To bring together research
  communities (e.g. users of \Jupyter, \Sage, \Singular, and \GAP) to
  symbiotically exploit overlaps in tool creation building efforts,
  avoid duplication of effort in different disciplines, and share best
  practice. This supports Aims~\ref{aim:collaboration},
  \ref{aim:sharing} and~\ref{aim:impact}.}''

  We have organized or co-organized a dozen users or developers
  workshops (see~\longlocaltaskref{dissem}{devel-workshops}) which brought
  together several communities. Some key outcomes include:
  \begin{itemize}
  \item Enabling \Jupyter as uniform interface for all computational
    components; see \longlocaltaskref{UI}{ipython-kernels}.
  \item Sharing best practices for development, packaging, building
    containers
    (see~\longlocaltaskref{component-architecture}{mod-packaging}),
    and continuous integration
    (see~\longlocaltaskref{component-architecture}{portability});
  \item A smooth collaboration between \JupyterHub, \SMC, and \Simulagora;
    see~\longlocaltaskref{component-architecture}{extract-smc},
    \longlocaltaskref{component-architecture}{simulagora-dev} and
    Section~\ref{infrastructures};
  \item Work on interfaces between systems; see
    \longlocaltaskref{component-architecture}{interface-systems}, \longlocaltaskref{UI}{mathhub},
    and \longlocaltaskref{UI}{pari-python};
    % \item Steps toward \longlocaltaskref{UI}{sage-sphinx}
  \item Sharing of best practices and tools for authoring live structured
    documents (see~\longlocaltaskref{UI}{structdocs});
  \item Sharing of best practices when using VRE's like \cocalc or \Jupyter for research and
    education;
  \item Collaboration on interactive visualization
    \longlocaltaskref{UI}{vis3d}, \longlocaltaskref{UI}{cfd-vis},
    \longlocaltaskref{UI}{dynamic-inspect}.
  \end{itemize}

\item \label{objective:updates} ``\emph{Update a range of existing open source
  mathematical software systems for seamless deployment and efficient
  execution within the VRE architecture of objective~\ref{objective:framework}.
  This fulfils part of Aim~\ref{aim:vre}.}''

Achievements include:
\begin{itemize}
  \item Continuous efforts of development, release and integration within \Sage
    have been put for
    \begin{itemize}
    \item  the linear algebra computational kernels of LinBox,
    fflas-ffpack and Givaro (Deliverable~\longdelivref{hpc}{LinBox-algo})
    \item the PARI library for computational number theory
    (Deliverable~\longdelivref{hpc}{pari-hpc2} still ongoing)
    \item the GAP software for computational group theory
    (Deliverable~\longdelivref{hpc}{GAP-HPC-report} still ongoing)
  \end{itemize}
\item Packaging efforts: docker containers (delivered and regularly
  updated), Debian and Conda packages (beta); see
  \longlocaltaskref{component-architecture}{mod-packaging}.
\item Continued efforts on portability of \Sage and its dependencies
  (see \longlocaltaskref{component-architecture}{portability}, in
  particular \delivref{component-architecture}{portability-cygwin}).
\item Improved continuous integration and development workflow;
  (see~\longlocaltaskref{component-architecture}{workflow}), and
  \longdelivref{component-architecture}{multiplatform-buildbot}.
\item Integration of all the relevant mathematical software in the
  uniform \Jupyter user interface, in particular for integration in
  the VRE framework (delivered, ongoing); see
  \longlocaltaskref{UI}{ipython-kernels}.
\item Ongoing work in \WPref{hpc} to better support HPC in the
  individual mathematical software system and combinations thereof;
  see Section~\ref{hpc}.
\end{itemize}

\item \label{objective:sustainable} ``\emph{Ensure that our ecosystem of
  interoperable open source components is \emph{sustainable} by
  promoting collaborative software development and outsourcing
  development to larger communities whenever suitable. This fulfils
  part of Aims~\ref{aim:sharing} and~\ref{aim:impact}.}''

Achievements include:
\begin{itemize}
\item Continued work on outsourcing the computational system user
  interfaces by migrating to \Jupyter; see \longlocaltaskref{UI}{ipython-kernels};
\item Refactoring \Sage's documentation build system to contribute many local developments
  upstream (\Sphinx) \longlocaltaskref{UI}{sage-sphinx};
\item Outsourcing and contributing upstream as \Python bindings the existing \Sage
  bindings for many computational systems; see \longlocaltaskref{UI}{pari-python}.
\end{itemize}

\item \label{objective:social} ``\emph{Promote collaborative mathematics and
  science by exploring the social phenomena that underpin these
  endeavours: how do researchers collaborate in Mathematics and
  Computational Sciences?  What can be the role of \VREs?  How can
  collaborators within a VRE be credited and incentivised? This
  addresses parts of Aims~\ref{aim:sharing}, \ref{aim:collaboration},
  and~\ref{aim:vre}.}''

This objective was the social science research side of
\WPref{social-aspects}. Following the work plan revisions after
Reporting Period 1, the manpower originally allocated to this
objective was reallocated to other objectives. There thus was no new
achievements in Reporting Period 2.

\item \label{objective:data} ``\emph{Identify and extend ontologies and
  standards to facilitate safe and efficient storage, reuse,
  interoperation and sharing of rich mathematical data whilst taking
  account of provenance and citability. This fulfills parts of
  Aims~\ref{aim:vre} and~\ref{aim:sharing}.}''
 
This objective is at the core of \WPref{dksbases}; see Section~\ref{dksbases} for
details. In WP6 we have developed the Math-in-the-Middle ontology that acts as the pivot
point mediating between system languages in the MitM interoperability framework. The work
has been reported in D6.8.

\item \label{objective:demo} ``\emph{Demonstrate the effectiveness of Virtual
  Research Environments built on top of \ODK components for a
  number of real-world use cases that traverse domains. This addresses
  part of Aim~\ref{aim:vre} and through documenting best practices in
  reproducible demonstrator documents Aim~\ref{aim:sharing}.}''

Most of the work toward this objective is by nature planned for the last period of the \pn
project. Nevertheless, work has started e.g.  toward the OOMMF demonstrator; see
\longlocaltaskref{dissem}{dissemination-of-oommf-nb-virtual-environment}
\longlocaltaskref{dissem}{dissemination-of-oommf-nb-workshops},
\longlocaltaskref{component-architecture}{oommf-python-interface}.

%Long term sustainability
\item \label{objective:disseminate} ``\emph{Promote and disseminate
  \ODK to the scientific community by active communication,
  workshop organisation, and training in the spirit of open-source
  software. This addresses Aim~\ref{aim:impact}.}''

This objective is at the core of \WPref{dissem}, with in particular
more than 30 meetings, developer, training, and community building
workshops organized during the second reporting period. See
Section~\ref{dissem} and \longdelivref{dissem}{workshops-3} for
details.
\end{compactenum}

%%% Local Variables:
%%% mode: latex
%%% TeX-master: "report"
%%% End:

%  LocalWords:  compactenum textbf aim:vre emph JupyterHub longtaskref notebook-collab
%  LocalWords:  extract-smc Jupyter localtaskref ipython-kernels dksbases WPref dksbases
%  LocalWords:  nbmerge nbdiff nbval delivref delivref jupyter-collab dissem organized
%  LocalWords:  cocalc portability-cygwin hpc taskref citability fulfills
%  LocalWords:  dissemination-of-oommf-nb-virtual-environment oommf-python-interface
%  LocalWords:  dissemination-of-oommf-nb-workshops
