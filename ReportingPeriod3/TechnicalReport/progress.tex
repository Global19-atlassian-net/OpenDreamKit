\section{Explanation of the work carried out by the beneficiaries and Overview of the progress}

In this section, we give a general overview of the progress of the
project during the second reporting period, ranging from March 2017 to
August 2018. We start by recalling some context of \ODK's approach
that is important to understand and evaluate the progress. Then we
describe the general state, and in more detail the progress of the
work packages.

\subsubsection*{Some context: \ODK's approach}
\ODK's approach to delivering a Virtual Research Environment (VRE) for
mathematics is not to build a monolithic one-size-fits-all VRE, but
rather a toolkit from which it is easy to set up VRE's that are
customised to specific needs by combining the appropriate components
(collaborative workspaces, user interfaces, computational software,
databases, \dots) on top of available physical resources (from
personal laptops to cloud infrastructure). This approach --- chosen by
design --- allows to flexibly put together lean computational
environments and tools for particular research challenges. These tools
provide the required functionality but due to the component based
approach carry no unnecessary bloat that would reduce effectiveness in
terms of installation process, size, computation time, and
reproducibility.

Most of the components preexist as an ecosystem of open source
software, developed by well established communities of developers. For
example, for interactive computing and data analysis, OpenDreamKit
promotes Jupyter, a web-based general purpose flexible notebook
interface\footnote{a notebook is a document that contains live code,
  equations, visualizations and explanatory text} that targets all
areas of science. A number of Virtual Research Environment already
exist, e.g.\ powered by \cocalc (formerly \SMC) or \JupyterHub.

Hence most of the work in \ODK\ is to foster this ecosystem, improving
the components themselves and their composability. The technical work is
distributed over the work packages:
\begin{itemize}
\item \emph{Component Architecture} (\textbf{WP3}): ease of
  deployment: modularity, packaging, portability, distribution, for
  individual components and combinations thereof. sustainability of
  the ecosystem: improving the development workflows.
\item \emph{User Interfaces} (\textbf{WP4}): enable Jupyter as uniform notebook
  interface, and further improve it; foster the collaboration between
  \cocalc and JupyterHub; generally speaking investigate
  collaborative, reproducible, and active documents.
\item \emph{Performance} (\textbf{WP5}): make the most of available hardware
  (multi-core, HPC, cloud), for individual computational components and
  combinations thereof.
\item \emph{Data/Knowledge/Software} (\textbf{WP6}): enable rich and robust
  interaction between computational components, data bases, knowledge
  bases, and users through explicit common semantic spaces, a language to
  express them, and tools to leverage them.
\end{itemize}
These technical work packages are supported by%
\footnote{The project originally had another work package on
  \emph{Studies of Social Aspects} (\textbf{WP7}); following the
  formal review for reporting period 1, it was decided in agreement
  with the reviewers and advisory board to shut down the work package
  after reporting period 1, moving some of its tasks to other work
  packages, and redirecting the man power for the others to more
  central tasks.}
\begin{itemize}
\item \emph{Community Building and Dissemination} (\textbf{WP2}): developer and
  training workshops, conferences, teaching material with focus on
  making the created value accessible to a wide, varied and growing user community.
\end{itemize}

As a result of \ODK's approach, the work programme for \ODK\ consists
of a large array of loosely coupled tasks, each being useful in its
own right, and none being absolutely critical.

The first and second reporting period confirmed that this is a strong
feature of \ODK's approach. Indeed, as analysed in the proposal, this
kind of project is subject to the following risks:
\begin{enumerate}
\item Recruitment of qualified personnel;
\item Different groups not forming an effective team;
\item Implementing infrastructure that does not match the needs of end-users;
\item Lack of predictability for tasks that are pursued jointly with
  the community;
\item Reliance on external software components.
\end{enumerate}
Together with ambitious software challenges, this makes the accurate
prediction of workload and precise timeline of work packages
difficult, especially over a period of four years in a field of
rapidly evolving technologies.

And indeed the project actually faced each of the risks above,
especially 1, 4, and 5. However, thanks to the flexibility enabled by
the loose coupling, those risks could be mitigated by adapting the
tasks schedule and human resources allocation, with little influence
on the general aims and objectives.

% \subsubsection*{General progress}

% Intensive work has now started on almost all fronts of the project\footnote{status reports
%   delivered at the St Andrews project meeting (January 2016) and at the Bremen's project
%   meeting (June 2016) helped the Coordinator to track the progress}.\ednote{This footnote
%   seems outdated, and should at least be extended} A few tasks (and the corresponding
% deliverables) have been postponed by a couple months due to recruitment delays. This
% concerns mostly the micro-magnetic VRE demonstrator
% (\taskref{dissem}{dissemination-of-oommf-nb-virtual-environment},
% \taskref{dissem}{dissemination-of-oommf-nb-workshops}, \taskref{dissem}{ibook},
% \taskref{component-architecture}{oommf-python-interface},
% \taskref{UI}{oommf-py-ipython-attributes}, \taskref{UI}{oommf-tutorial-and-documentation},
% \taskref{UI}{oommf-nb-ve}, \taskref{social-aspects}{oommf-nb-evaluation}),\ednote{MK: is
%   this taks still up to date?} where recruitment at Southampton initially proceeded at
% expected speed but eventually experienced delays of several months outside our control due
% to unusually high demand on the UK Home Office which had to process work permit paperwork
% for the successful candidate (attributed to a high number of immigration applications in
% the run up to the Brexit referendum). Some deliverables got delayed as well by a couple
% months due to unexpected technical difficulties or misplanning
% (e.g. \delivref{hpc}{pythran-typing}, \delivref{hpc}{sage-paral-tree},
% \delivref{UI}{pari-python-lib1}). All these delays have been included in the amendment of
% the Grant Agreement, which was necessary to include UGent\ednote{MK: do we need to mention
%   FAU here as well?} in the consortium. On the other hand, we are happy to report below on
% very strong recruitment (see Section 3.1), as well as unexpectedly rapid progress on
% portability and packaging aspects. Also \WPtref{dksbases} has witnessed a particularly
% strong and early uptake, with active involvement of many of the participants and promising
% outcomes. This has led to a refinement of research/development goals and a more
% foundational outlook on the semantic integration middleware in this work package.

% \begin{oldpart}{MK: this is hopelessly outdated}
%   All in all, \ODK\ is running according to its plan, and its first outcomes are already
%   benefiting the mathematical community and beyond. September 2016 will see the start of
%   Key Performance Indicators.  These KPIs, which will be more precisely and realistically
%   defined then, will give results for the 1st Reporting Period (RP1) at month 18.  This
%   way we will hopefully be able to see the evolution of the impact OpenDreamKit has had
%   between the RP1 and RP2, at month 36.
% \end{oldpart}
%%% Local Variables:
%%% mode: latex
%%% TeX-master: "report"
%%% End:

%  LocalWords:  Jupyter visualizations cocalc composability emph textbf taskref dissem
%  LocalWords:  dissemination-of-oommf-nb-virtual-environment oommf-python-interface hpc
%  LocalWords:  dissemination-of-oommf-nb-workshops oommf-py-ipython-attributes delivref
%  LocalWords:  oommf-tutorial-and-documentation oommf-nb-ve oommf-nb-evaluation ednote
%  LocalWords:  pythran-typing sage-paral-tree oldpart
