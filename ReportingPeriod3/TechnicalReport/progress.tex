\section{Explanation of the work carried out by the beneficiaries and Overview of the progress}

In this section, we give a general overview of the progress of the
project during the third reporting period, ranging from September 2018
to August 2019. For new readers, we start by recalling some context of
\ODK's approach that is important to understand and evaluate the
progress; except for the last paragraphs, this piece of text is
unchanged from the preview reporting periods. Then we provide a brief
overview of the work carried out by objective of the project; finally
we detail the progress work package by work package.

\subsubsection*{Some context: \ODK's approach}
\label{section.context}
\ODK's approach to delivering a Virtual Research Environment (VRE) for
mathematics is not to build a monolithic one-size-fits-all VRE, but
rather a toolkit from which it is easy to set up VRE's that are
customised to specific needs by combining the appropriate components
(collaborative workspaces, user interfaces, computational software,
databases, \dots) on top of available physical resources (from
personal laptops to cloud infrastructure). This approach --- chosen by
design --- allows users to flexibly put together lean computational
environments and tools for particular research challenges. These tools
provide the required functionality but due to the component based
approach carry no unnecessary bloat that would reduce effectiveness in
terms of installation process, size, computation time, and
reproducibility.

Most of the components preexist as an ecosystem of open source
software, developed by well established communities of developers. For
example, for interactive computing and data analysis, OpenDreamKit
promotes Jupyter, a web-based general purpose flexible notebook
interface\footnote{a notebook is a document that contains live code,
  equations, visualizations and explanatory text} that targets all
areas of science. A number of Virtual Research Environment already
exist, e.g.\ powered by \cocalc (formerly \SMC) or \JupyterHub.

Hence most of the work in \ODK\ is to foster this ecosystem, improving
the components themselves and their composability. The technical work is
distributed over the work packages:
\begin{itemize}
\item \emph{Component Architecture} (\textbf{WP3}): ease of
  deployment: modularity, packaging, portability, distribution, for
  individual components and combinations thereof. sustainability of
  the ecosystem: improving the development workflows.
\item \emph{User Interfaces} (\textbf{WP4}): enable Jupyter as uniform notebook
  interface, and further improve it; foster the collaboration between
  \cocalc and JupyterHub; generally speaking investigate
  collaborative, reproducible, and active documents.
\item \emph{Performance} (\textbf{WP5}): make the most of available hardware
  (multi-core, HPC, cloud), for individual computational components and
  combinations thereof.
\item \emph{Data/Knowledge/Software} (\textbf{WP6}): enable rich and robust
  interaction between computational components, data bases, knowledge
  bases, and users through explicit common semantic spaces, a language to
  express them, and tools to leverage them.
\end{itemize}
These technical work packages are supported by%
\footnote{The project originally had another work package on
  \emph{Studies of Social Aspects} (\textbf{WP7}); following the
  formal review for reporting period 1, it was decided in agreement
  with the reviewers and advisory board to shut down the work package
  after reporting period 1, moving some of its tasks to other work
  packages, and redirecting the man power for the others to more
  central tasks.}
\begin{itemize}
\item \emph{Community Building and Dissemination} (\textbf{WP2}): developer and
  training workshops, conferences, teaching material with focus on
  making the created value accessible to a wide, varied and growing user community.
\end{itemize}

As a result of \ODK's approach, the work programme for \ODK\ consists
of a large array of loosely coupled tasks, each being useful in its
own right, and none being absolutely critical.

All three reporting periods confirmed that this is a strong
feature of \ODK's approach. Indeed, as analysed in the proposal, this
kind of project is subject to the following risks:
\begin{enumerate}
\item Recruitment of qualified personnel;
\item Different groups not forming an effective team;
\item Implementing infrastructure that does not match the needs of end-users;
\item Lack of predictability for tasks that are pursued jointly with
  the community;
\item Reliance on external software components.
\end{enumerate}
Together with ambitious software challenges, this made the accurate
prediction of workload and precise timeline of work packages
difficult, especially over a period of four years in a field of
rapidly evolving technologies.

And indeed the project actually faced each of the risks above,
especially 1, 4, and 5.
The toughest situation we encountered was the volatility of the personnel at
Sheffield, where the PIs and hired personnel all left for industry at
various stage of the project, some after an intermediate move to
Leeds. However, thanks to the flexibility enabled by
the loose coupling, those risks could be mitigated by adapting the
tasks schedule and human resources allocation, with little influence
on the general aims and objectives.

% The toughest situation we faced was the volatility of the personnel at
% Sheffield, where the PIs and hired personnel all left for industry at
% various stage of the project, some after an intermediate move to
% Leeds. Luckily, most of the work planned for this site was on
% dissemination tasks

OpenDreamKit's approach has one downside: it impedes formal
evaluation, as much for us to assess the adequacy and impact of our
tools, as for our reviewers to assess the depth and value of our
contribution. Indeed, we do not have a main well-defined product and
we capture a very diverse range of end-users and use-cases; hence
quantitative methods of evaluation of the adequacy for end users, like
satisfaction surveys, are rather elusive. In addition, when tools are
jointly developed with the community, how should one attribute the
merit of their success to the project or to the community?

In practice, this certainly added complexity to our reporting efforts
and to our reviewers efforts. It is our \emph{belief} however that
this did not impact the work itself. Indeed, co-design is intrinsic to
the by-users for-users development model of the ecosystem, most of the
participants were also end-users themselves, and we maintained deep
contact with the user community notably through our continuous
dissemination actions. Therefore informal evaluation through first
hand experience or witnessing was largely sufficient to inform the
design and execution of the project.

%  LocalWords:  Jupyter visualizations cocalc composability emph textbf taskref dissem
%  LocalWords:  dissemination-of-oommf-nb-virtual-environment oommf-python-interface hpc
%  LocalWords:  dissemination-of-oommf-nb-workshops oommf-py-ipython-attributes delivref
%  LocalWords:  oommf-tutorial-and-documentation oommf-nb-ve oommf-nb-evaluation 
%  LocalWords:  pythran-typing sage-paral-tree oldpart subsubsection WPtref dksbases
