\section{Deviations from Annex 1}
  % Explain the reasons for deviations from the DoA, the consequences and the proposed
  % corrective actions
\label{section.deviations}

Generally speaking, there was no major deviation to Annex 1. The
deliverables for RP3 were completed on time, and we reached the
objectives and milestones of the project. On the financial side, the
project is claiming 7.16M€\footnote{Numbers in this section are
  indicative, and subject to small adjustments once financial
  statements are final. Nevertheless, their values are precise enough
  for the large grain discussions carried out here.}, that is 94\% of
the maximum grant amount.

Per se, this is a good news: the project was cost effective! Let's
take however the perspective of our communities of users and
developers: running a European project like this one is a major time
investment, in terms of preparation, management, and reporting. For
that same investment, we could have, in principle, taken advantage of
.46M€ of additional resources -- about seven years of Research
Software Engineers -- to support our aims.

Therefore, before discussing the details of the deviations for RP3, we
reflect in the next paragraphs about the deeper causes of this
underspending. First, some facts. At the end RP1, we had claimed
1.95M€, an underspending of 68\% assuming a uniform spending of the
total grant amount. To some point, this was expected: it takes time to
launch a project of this scale and reach full speed, notably due to
the inherently slow process of hiring high profile personnel.
Unforeseen administrative burden (visas, ...) induced additional
delays. But all in all, this shifted work load later into the project,
without impacting the total use of resources.

By the end of RP2, we had claimed 4.7M€. Assuming that the project had
been running at continuous full speed for RP2 and would continue to do
so in RP3, this gave a projection of 6.5M€ of claimed expenses for the
full duration of the project, that is an underspending of .85\%.

We identified several causes. First the consortium as a whole took
active steps -- sometime creative ones -- to make the most of the
granted resources, notably when organizing meetings and conferences.
This included e.g. using video-conference whenever appropriate.
Second, it is standard policy for
beneficiaries to make conservatively high estimates of their budget to
mitigate internal and external uncertainties in the implementation of
the action.
This is particularly true for British
beneficiaries that face fluctuations of the conversion rate between the
\ednote{reuse the explicit exchange rate from the commented out text below}
British pound and the euro. Brexit caused the exchange rate to
actually drop. Third, in addition to a major nationwide grant of the
DFG which, coordinated at \site{UK}, was running until 2016 (SPP
1489), \site{UK} became the speaker university of a more recent major
grant of the DFG (SFB/TRR 195); important events that \site{UK} had
originally planned to fund on \ODK ended up being funded by the DFG
grants.
Finally, there was the situation at Sheffield/Leeds: not only did they have
recurrent difficulties hiring and keeping staff, but even their PIs
were being hired by the industry for high profile jobs. In fact, we
learned shortly after the review for RP2 that the last remaining PI
was hired himself at NAG, leading Leeds to completely stop its
activities, with an unspent budget of 0.26M€.

Encouraged by the reviewers, we explored venues to shift workload and
resources among partners in order to optimize the use of the remaining
resources toward the objectives of the project. This led to budget
transfers, enabling the organization of many additional dissemination
activities by other beneficiaries to make up for Leeds activity, and
to the addition of a new task in WP6. This was formalized by a 5th
amendment to the grant agreement. These mitigation measures, combined
with \site{UK} contributing financially to our major dissemination
event at CIRM, and with some earlier planned recruitment, enabled the
exploitation of another 0.66M€. It was however too late, 10 months to
the end of the project, for launching additional recruitment to
exploit the remaining 0.46M€ budget.

What lessons did we learn?

Foreseeing events like Brexit is obviously beyond reach. What is
needed is agility. For us, this agility was enabled by loosely coupled
tasks, few critical tasks, and no critical tasks with tiny bus factor.
And by the continuous support from our reviewers and project officers
to explore paths to make the best use of our resources to achieve our
aims.

However enacting the agility necessitates to detect financial
deviations as early as possible to take mitigation measures. With
OpenDreamKit's administrative setup, we had very detailed expenses
information at three points in time: M19, M39 (it should have been
M37), and M49. But we had absolutely no intermediate data point.To the  
coordinator, it felt like steering a cargo ship in the fog, with
enormous stretches of silence between two sonar pings. Even just with
some gross estimates at M30, we could have acted.

Future projects should explore ways to collect such gross estimates
from time to time, without adding to the administrative burden. Having
regular informal financial reports? Fostering continuous financial
reporting within each site, through more automation of the expense
collection and forecast?

Future projects should also include in their risk management not only
the difficulty of recruiting highly qualified personnel, but also of
keeping such personnel, whether permanent or temporary. As seen in the
risk management section~\ref{section.risk_management}, for a group of this size
the hazards of life will take their toll.

\subsection{Use of resources}

\subsubsection{Budget}

Note: the percentages given below are computed w.r.t. the total
planned budget (max grant amount spent) as of the beginning of
reporting period 3, before the 5th amendment to the grant agreement.
For sites that moved, the numbers are consolidated together.

\begin{itemize}
\item As explained above, \site{Leeds} early shut down released about
  250k€ of budget for use in other sites.
\item \site{PS} used 115\% of its budget, overspending by 150k€ (105k€
  formalized in the 5th amendment) for the organization of many
  additional events, a higher involvement in several other tasks, and
  also to claim costs for the much higher involvement of the
  coordinator than was originally planned (about 20PM instead of
  12PM).
\item \site{UB} used 110\% of its budget, overspending by 79k€ (40k€
  formalized in the 5th amendment) for additional dissemination
  activities and higher involvement than originally planned.
\item \site{UV} used 115\% of its budget, overspending by 18k€ for
  higher involvement than originally planned, notably in
  dissemination.
\item \site{UJF} used 93\% of its budget, underspending by 38k€.
\item \site{LL} used 96\% of its budget, underspending by 16k€.
\item Due to the low exchange rate -- and to the difficulty of hiring
  and keeping personnel which was exacerbated by the uncertainties of
  Brexit -- British sites \site{UO}, \site{SA}, and \site{UW} used
  respectively only 90\%, 82\%, and 91\% of their budget
  (underspending by 30k€, 157k€, and 19k€), with only minor deviations
  to their number of PMs. This also contributed to
  \site{USO}+\site{XFEL} using only 92\% of its budget (underspending
  by -37k€) and contributed to \site{USH}+\site{LEEDS} using only 56\%
  of its budget (underspending by 250k€).
\item As formalized in the 5th amendment, \site{JU}+\site{FAU} used
  103\% of its budget, overspending by 18k€ for
  \taskref{dksbases}{isabelle}, in the form of a subcontract.
\item As explained earlier, \site{UK} used 76\% of its budget,
  underspending by 135k€.
\item \site{UG} used 103\% of its budget, overspending by 9.5k€ (16k€
  formalized in the 5th amendment) due to evolution in their monthly
  personnel costs.
\item \site{US} used 88\% of its budget, underspending by 20k€.
\item \site{SR} used 97\% of its budget, underspending by 15k€.
\item \site{ZH} used 95\% of its budget, underspending by 8k€, before
  shutting down in RP2.
\end{itemize}


\subsubsection{Person-Months}

Note: the percentages given below are computed w.r.t. the total actual
person/months per participant  (with respect to the estimations made 
at the beginning of the project).

\ednote{Site leaders: please include an explanation if there is a
  large deviation in your site; you can find elements of language
  below in the commented out explanation of deviations from earlier
  reporting periods}

\begin{itemize}
\item \site{UPSud} declared 109,9\% of its P/M, overspending of 12,37 P/M
with respect to the amount specified in the GA.
\item \site{CNRS} declared 98,3\% of its P/M, underspending of 1,07 P/M (63 P/M
  formalized in the GA) 
\item \site{UB} declared 121,7\% of its P/M, overspending of 4,34 P/M (20 P/M
  formalized in the GA).
\item \site{UGA} declared 97\% of its P/M, underspending of 1,82 P/M ( 60 P/M 
formalized in the GA against 58,18 in the financial statements).
\item \site{UNIKL} declared 84,5\% of its P/M, underspending of 10,25 P/M ( 66 in the
 GA, against 55,75 in the financial statements).
\item \site{UOXF} declared 105,6\% of its P/M, overspending of 1,61 P/M ( 29 P/M 
in the GA against 30,61 in the FS).
\item \site{USlaski} declared 99,4\% of its P/M, underspending of 0,20 P/M ( 36 P/M 
in the GA against 35,8 in the FS).
\item \site{USTAN} declared 109\% of its P/M, overspending of 7,35 P/M (82 in the GA,
 89,35 PM in the FS)
\item \site{UVSQ} declared 100,9\% of its P/M, overspending of 0,13 P/M.
\item \site{WARWICK} declared 90\% of its P/M, underspending of 2,7 P/M.
\item \site{Logilab} declared 76\% of its P/M, underspending of 11,5 P/M (48 P/M
formalized in the GA, 36,5 in actual financial statements)
\item \site{Simula} declared 167,4\% of its P/M, overspending of 21,58 P/M ( 32 P/M
formalized in the GA against 53,58 in the actual FS).
\item \site{Ugent} declared 94,7\% of its P/M, underspending of 1,62 P/M.
\item \site{XFEL} declared 107,2\% of its P/M, overspending of 1,95 P/M.
\item \site{FAU} declared 160,8\% of its P/M, overspending of 26,16 P/M ( 43 P/M
formalized in the GA against 69,16 in the actual FS).
\item \site{LEEDS} declared 11\% of its P/M, underspending of 19,55 P/M.( 21,97 P/M
formalized in the GA against 2,42 in the actual FS)

\end{itemize}

% \subsection{Deviations section of the periodic report 1 and 2}

% \item{Use of resources Reporting period 1}
% All changes of use of resources were included in the two amendments previously cited and were
% due to modifications in the personnel. Those adjustments were due to the change of positions
% of some key \ODK participants and expected difficulties in hiring planned
% staff. The work plan has been updated accordingly, with no foreseeable
% impact on the achievement of tasks, deliverables, and milestones.

% \item{Use of resources Reporting period 2}
% All changes of use of resources were included in the two amendments previously cited and were
% due to modifications in the personnel. Those adjustments were due to the change of positions
% of some key \ODK participants and expected difficulties in hiring planned
% staff. The work plan has been updated accordingly, with no foreseeable
% impact on the achievement of tasks, deliverables, and milestones.

% Another minor deviation in the proposed use of resource was that FAU hired students to do
% some routine jobs (simple formalizations, and the creation of alignments in WP6 and the
% creation of example documents in WP4) that did not require the attention of a mature
% researchers. As the pay grade of student assistant is roughly 1/4 of that of full
% researchers, this action was cost-effective. An unplanned effect was that the reported
% person months went up considerably, exceeding the planned amount, without incurring
% additional cost. 

% Due to cancellation of WP7, the efforts of UOXF were redirected to other WPs, including
% WP2, where they were originally not active. Within WP2, UOXF delivered mini-courses on various ODK components,
% such as GAP and \Sage.

% A major deviation from the work plan for CNRS was an overspending of
% PM in \WPref{hpc} and underspending in \WPref{UI}. This reallocation
% is due to a change of priorities in the tasks assigned to CNRS and the
% specific competences of the hired engineer. In particular we decided
% to drop \taskref{UI}{Sage-display}: this task is generally useful in
% the long term, but low priority; since the writing of the proposal, it
% appeared that its implementation would require much larger amounts of
% code refactoring and backward incompatibilities than originally
% expected; in addition the landscape of visualization libraries (in
% particular in javascript) has been evolving at a very fast pace which
% is likely to quickly; hence the refactoring will be more efficiently
% achieved in a later project, once the landscape will have stabilized.

% Taking this decision enabled us to reinvest the freed resources
% (26PMs) on more critical and urgent tasks, as listed below. Additional
% efforts are made to provide native Windows support for the \ODK
% components, in particular the cysignals library (6PM); this was
% requested by many users and developers and will enable the native use
% of PARI/GP from Python via cypari on Windows. Some extra efforts are
% invested in the fine grain parallelization of PARI/GP (4PM). Finally,
% most efforts are dedicated to modularity and Python3 compatibility for
% \Sage (16PM).

% A major deviation for USTAN is the reduced cost of manpower due to exchange rate changes. In the application
% we used an expected rate of \euro 1 = \pounds 0.69. In our most recent financial report we used an actual
% rate of \euro 1 = \pounds 0.88. The monthly rates paid in pounds have been in line with our budget as had the 
% effort in terms of man months, but the cost of this in Euros has fallen dramatically. We anticipate a modest increase in 
% effort beyond the original budget over the final reporting period, but expect to fall well within the finanicial budget.

% UVSQ claiming effort in WP6 is due to Luca De Feo spending 5 hours on WP6 from April 24th to 28th, 2017. He helped Michael 
% Kolhase's team (from FAU) to prepare the WP6 presentations. The benefits of his collaboration are here:
% \url{https://link.springer.com/chapter/10.1007%2F978-3-319-72453-9_14}.

\subsubsection{Unforeseen subcontracting (if applicable)}

\end{itemize}

\subsection{Tasks}

\subsubsection{New Task \tasktref{dksbases}{isabelle}}

The 5th amendment to the grant agreement introduced the new task \taskref{dksbases}{isabelle}.
During the course of the project, it became evident that we should also test the practical coverage of the trifunctional modules, by
transforming an existing, high-profile library of formalized mathematics (in contrast to computational mathematics) into OMDoc/MMT.
We introduced the new task to this end.

It conducted a case study on the Isabelle library.
This task was carried out in collaboration with Dr. Makarius Wenzel, the main developer of Isabelle on a subcontract (as explained below).

\subsection{Deliverables}

\subsubsection{Changed the title of \delivtref{dksbases}{nbad-search}}
The title of this deliverable was originally \emph{Full-text search
  (Formulae + Keywords) in OpenDreamKit}, but in the last grant
proposal amendment the scope was broadened to a report on the
remaining \WPref{dksbases} activities and achievements -- also to
account for the new task \taskref{dksbases}{isabelle}. As the focus of
the last reporting period was on integrating mathematical data, the title was changed to better account for this.

\subsubsection{New Task \tasktref{dksbases}{isabelle}}
The report on this new task is included in Deliverable \delivtref{dksbases}{nbad-search}.

\subsection{Unforeseen subcontracting (if applicable)}
FAU has subcontracted the Isabelle case study (\taskref{dksbases}{isabelle}) to Dr. Makarius Wenzel's company \url{sketis.net}.

Note that the task as a whole was unforeseen --- once the task was added, the decision to subcontract was rather obvious:
Dr. Wenzel is the main developer of Isabelle who has spent the last $\sim 10$ years building the technological prerequisites for such a case study.
There is no alternative way of conducting such a case study since developing the necessary expertise in-house would have been prohibitively expensive (if possible at all).
Because Dr. Wenzel left academia years ago and started his own company
specializing on Isabelle development, the above-mentioned amendment to
the grant agreement included a subcontract for his company.

%%% Local Variables:
%%% mode: latex
%%% mode: visual-line
%%% TeX-master: "report"
%%% End:


%  LocalWords:  WPref hpc WPref dksbases subsubsection delivtref dissem organizing emph
%  LocalWords:  Jupyter emph ipython-kernels psfoundation formalizations WPs taskref
%  LocalWords:  WieKohRab:vtuimkb17,KohMuePfe:kbimss17 visualization stabilized cysignals
%  LocalWords:  cypari parallelization
