\section{Deviations from Annex 1}
  % Explain the reasons for deviations from the DoA, the consequences and the proposed
  % corrective actions
\label{section.deviations}

Generally speaking, there was no major deviation to Annex 1. The
deliverables for RP3 were completed on time, and we reached the
objectives and milestones of the project. On the financial side, the
project is claiming 7.19M€\footnote{Numbers in this section are
  indicative, and subject to small adjustments once financial
  statements are final. Nevertheless, their values are precise enough
  for the large grain discussions carried out here.}, that is 94\% of
the maximum grant amount.

Per se, this is a good news: the project was cost effective! Let's
take however the perspective of our communities of users and
developers: running a European project like this one is a major time
investment, in terms of preparation, management, and reporting. For
that same investment, we could have, in principle, taken advantage of
0.46M€ of additional resources -- about seven years of Research
Software Engineers -- to support our aims.

Therefore, before discussing the details of the deviations for RP3, we
reflect in the next paragraphs about the deeper causes of this
underspending. First, some facts. At the end RP1, we had claimed
1.95M€, an underspending of 68\% assuming a uniform spending of the
total grant amount. To some point, this was expected: it takes time to
launch a project of this scale and reach full speed, notably due to
the inherently slow process of hiring high profile personnel.
Unforeseen administrative burden (visas, ...) induced additional
delays. But all in all, this shifted work load later into the project,
without impacting the total use of resources.

By the end of RP2, we had claimed 4.7M€. Assuming that the project had
been running at continuous full speed for RP2 and would continue to do
so in RP3, this gave a projection of 6.5M€ of claimed expenses for the
full duration of the project, that is an underspending of .85\%.

We identified several causes. First the consortium as a whole took
active steps -- sometime creative ones -- to make the most of the
granted resources, notably when organizing meetings and conferences.
This included e.g. using video-conference whenever appropriate.
Second, it is standard policy for
beneficiaries to make conservatively high estimates of their budget to
mitigate internal and external uncertainties in the implementation of
the action.
This is particularly true for British
beneficiaries that face fluctuations of the conversion rate between the
\ednote{reuse the explicit exchange rate from the commented out text below}
British pound and the euro. Brexit caused the exchange rate to
actually drop. Third, in addition to a major nationwide grant of the
DFG which, coordinated at \site{UK}, was running until 2016 (SPP
1489), \site{UK} became the speaker university of a more recent major
grant of the DFG (SFB/TRR 195); important events that \site{UK} had
originally planned to fund on \ODK ended up being funded by the DFG
grants.
Finally, there was the situation at Sheffield/Leeds: not only did they have
recurrent difficulties hiring and keeping staff, but even their PIs
were being hired by the industry for high profile jobs. In fact, we
learned shortly after the review for RP2 that the last remaining PI
was hired himself at NAG, leading Leeds to completely stop its
activities, with an unspent budget of 0.26M€.

Encouraged by the reviewers, we explored venues to shift workload and
resources among partners in order to optimize the use of the remaining
resources toward the objectives of the project. This led to budget
transfers, enabling the organization of many additional dissemination
activities by other beneficiaries to make up for Leeds activity, and
to the addition of a new task in WP6. This was formalized by a 5th
amendment to the grant agreement. These mitigation measures, combined
with \site{UK} contributing financially to our major dissemination
event at CIRM, and with some earlier planned recruitment, enabled the
exploitation of another 0.66M€. It was however too late, 10 months to
the end of the project, for launching additional recruitment to
exploit the remaining 0.46M€ budget.

What lessons did we learn?

Foreseeing events like Brexit is obviously beyond reach. What is
needed is agility. For us, this agility was enabled by loosely coupled
tasks, few critical tasks, and no critical tasks with tiny bus factor.
And by the continuous support from our reviewers and project officers
to explore paths to make the best use of our resources to achieve our
aims.

However enacting the agility necessitates to detect financial
deviations as early as possible to take mitigation measures. With
OpenDreamKit's administrative setup, we had very detailed expenses
information at three points in time: M19, M39 (it should have been
M37), and M49. But we had absolutely no intermediate data point.To the  
coordinator, it felt like steering a cargo ship in the fog, with
enormous stretches of silence between two sonar pings. Even just with
some gross estimates at M30, we could have acted.

Future projects should explore ways to collect such gross estimates
from time to time, without adding to the administrative burden. Having
regular informal financial reports? Fostering continuous financial
reporting within each site, through more automation of the expense
collection and forecast?

Future projects should also include in their risk management not only
the difficulty of recruiting highly qualified personnel, but also of
keeping such personnel, whether permanent or temporary. As seen in the
risk management section~\ref{section.risk_management}, for a group of this size
the hazards of life will take their toll.

\subsection{Tasks}

\subsubsection{New Task \tasktref{dksbases}{isabelle}}

The 5th amendment to the grant agreement introduced the new task \taskref{dksbases}{isabelle}.
During the course of the project, it became evident that we should also test the practical coverage of the trifunctional modules, by
transforming an existing, high-profile library of formalized mathematics (in contrast to computational mathematics) into OMDoc/MMT.
We introduced the new task to this end.

It conducted a case study on the Isabelle library.
This task was carried out in collaboration with Dr. Makarius Wenzel, the main developer of Isabelle on a subcontract (as explained below).

The report on this new task is included in Deliverable
\delivtref{dksbases}{nbad-search} (see below).

\subsubsection{Changed the title of \delivtref{dksbases}{nbad-search}}

The title of this deliverable was originally \emph{Full-text search
  (Formulae + Keywords) in OpenDreamKit}, but in the last grant
proposal amendment the scope was broadened to a report on the
remaining \WPref{dksbases} activities and achievements -- also to
account for the new task \taskref{dksbases}{isabelle}. As the focus of
the last reporting period was on integrating mathematical data, the title was changed to better account for this.

\subsubsection{Late deliverables}

\longdelivref{dksbases}{persistent-memoization} (due 28/02/2019;
delivered 02/04/2019; updated 29/08/2019): in retrospect, we should
have chosen the end of project as delivery date. Indeed, the work was
implemented in due time and the original version of the report
submitted on time. However, at the end of the project seven months
later, much follow up activity had happened and we found interesting
to provide an updated view to the reviewers, notably on the user
adoption. Hence, after approval by the project officer, we resubmitted
an updated version.

\longdelivref{management}{imp2} (due 31/05/2019; delivered
06/09/2019): in retrospect, we should have chosen the end of project
as delivery date; indeed this deliverable was in no critical path and
was an update to the earlier \delivref{management}{imp1}. On the other
hand, the write up required some brainstorms and interactions with
many members of the consortium. It was therefore deemed more time
efficient to take advantage of the end-of-project report writing
workshop that was organized in late August.

\longdelivref{dissem}{press-release-2} (due 31/08/2019; delivered
16/12/2019): following the unexpected early termination of activities
in Leeds, a lot of the dissemination work originally planned at Leeds
was taken over by other sites. Together with the maternity leave of
the Work Package 2 leader, this added the leadership and reporting
burden of several deliverables (e.g.
\longdelivref{dissem}{workshops-4}, \longdelivref{dissem}{IntroODK},
\longdelivref{dissem}{press-release-2}) to the coordinators hands. It
was decided that the priority should go to the former two, as they had
the most scientific content. Besides, the press releases were best
published after the final review to include its outcome and reporting
about them then allowed for reviewing the whole process of their
production and dissemination.

\subsection{Use of resources}

\subsubsection{Unforeseen subcontracting (if applicable)}
FAU has subcontracted the Isabelle case study (\taskref{dksbases}{isabelle}) to Dr. Makarius Wenzel's company \url{sketis.net}.

Note that the task as a whole was unforeseen --- once the task was added, the decision to subcontract was rather obvious:
Dr. Wenzel is the main developer of Isabelle who has spent the last $\sim 10$ years building the technological prerequisites for such a case study.
There is no alternative way of conducting such a case study since developing the necessary expertise in-house would have been prohibitively expensive (if possible at all).
Because Dr. Wenzel left academia years ago and started his own company
specializing on Isabelle development, the above-mentioned amendment to
the grant agreement included a subcontract for his company.

\subsubsection{Unforeseen use of in kind contribution from third party against payment or free of charges}

\subsubsection{Budget}

Note: the percentages given below are computed w.r.t. the total
planned budget (max grant amount spent) as of the beginning of
reporting period 3, after the 5th amendment to the grant agreement.
For sites that moved, the numbers are consolidated together.

\begin{itemize}
\item As explained above, \site{Leeds} early shut down released about
  250k€ of budget for use in other sites.
\item \site{PS} used 102\% of its budget, overspending by 19k€ ( for the organization of many
  additional events, a higher involvement in several other tasks, and
  also to claim costs for the much higher involvement of the
  coordinator than was originally planned (about 20PM instead of
  12PM).
\item \site{UB} used 123\% of its budget, overspending by 56k€ (for additional dissemination
  activities and higher involvement than originally planned.
\item \site{UV} used 115\% of its budget, overspending by 18k€ for
  higher involvement than originally planned, notably in
  dissemination.
\item \site{LL} used 96\% of its budget, underspending by 16k€.
\item Due to the low exchange rate -- and to the difficulty of hiring
  and keeping personnel which was exacerbated by the uncertainties of
  Brexit -- British sites \site{UO}, \site{SA}, and \site{UW} used
  respectively only 92\%, 83\%, and 93\% of their budget
  (underspending by 24k€, 154k€, and 16k€), with only minor deviations
  to their number of PMs. This also contributed to
  \site{USO}+\site{XFEL} using only 92\% of its budget (underspending
  by -37k€) and contributed to \site{USH}+\site{LEEDS} using only 56\%
  of its budget (underspending by 250k€).
\item As formalized in the 5th amendment, \site{JU}+\site{FAU} used
  98\% of its budget, underspending by 10k€ for
  \taskref{dksbases}{isabelle}, in the form of a subcontract.\\
  This is consistent with amendment V of the grant agreement.
\item As explained earlier, \site{UK} used 76\% of its budget,
  underspending by 135k€.
\item \site{UG} used 94\% of its budget, underspending by 33k€ 
due to evolution in their monthly personnel costs.
\item \site{US} used 100\% of its budget.
\item \site{SR} used 100\% of its budget.
\item \site{ZH} used 95\% of its budget, underspending by 8k€, before
  shutting down in RP2.
\end{itemize}


\subsubsection{Person-Months}

Note: the percentages given below are computed w.r.t. the total actual
person/months per participant  (with respect to the estimations made 
at the beginning of the project).

% \ednote{Site leaders: please include an explanation if there is a
%   large deviation in your site; you can find elements of language
%   below in the commented out explanation of deviations from earlier
%   reporting periods}

\begin{itemize}
\item \site{PS} declared 110\% of its PMs, overspending of 12 PMs
with respect to the amount specified in the GA.
\item \site{CNRS} declared 98\% of its PMs, underspending of 1,1 PMs (63 PMs
  formalized in the GA)
  \ednote{It's unclear which site this refers to.}
\item \site{UB} declared 122\% of its PMs, overspending of 4,3 PMs (20 PMs
  formalized in the GA).
\item \site{UJF} declared 97\% of its PMs, underspending of 1,8 PMs ( 60 PMs 
formalized in the GA against 58,2 in the financial statements).
\item \site{UK} declared 84\% of its PMs, underspending of 10,2 PMs ( 66 in the
 GA, against 55,7 in the financial statements).
\item \site{UO} declared 106\% of its PMs, overspending of 1,6 PMs ( 29 PMs 
in the GA against 30,6 in the FS).
\item \site{US} declared 99\% of its PMs, underspending of 0,20 PMs ( 36 PMs 
in the GA against 35,8 in the FS).
\item \site{SA} declared 109\% of its PMs, overspending of 7,35 PMs (82 in the GA,
 89,35 PMs in the FS)
\item \site{UV} declared 101\% of its PMs, overspending of 0,13 PMs.
\item \site{UW} declared 90\% of its PMs, underspending of 2,7 PMs.
\item \site{LL} declared 76\% of its PMs, underspending of 11,5 PMs (48 PMs
formalized in the GA, 36,5 in actual financial statements)
\item \site{SR} declared 167\% of its PMs, overspending of 21,6 PMs ( 32 PMs
formalized in the GA against 53,6 in the actual FS).
\item \site{UG} declared 94\% of its PMs, underspending of 1,6 PMs.
\item \site{XFEL} declared 107\% of its PMs, overspending of 1,9 PMs.
\item \site{FAU} declared 161\% of its PMs, overspending of 26,2 PM ( 43 PMs
formalized in the GA against 69,2 in the actual FS).\\
The reason for this is that \site{FAU} was able to hire junior researchers for some of the more routine parts of the work.
This decision was cost-effective for \pn, but according to the EU guidelines these junior researcher months were counted as full PMs even though they only worked part-time (on average about 10 hours/week).
This led to an artificially inflated number of PMs. \site{FAU} did not overspend in personnel costs.
\item \site{LEEDS} declared 11\% of its PMs, underspending of 19,55 PMs.( 22 PMs
formalized in the GA against 2,4 in the actual FS)

\end{itemize}

% \subsection{Deviations section of the periodic report 1 and 2}

% \item{Use of resources Reporting period 1}
% All changes of use of resources were included in the two amendments previously cited and were
% due to modifications in the personnel. Those adjustments were due to the change of positions
% of some key \ODK participants and expected difficulties in hiring planned
% staff. The work plan has been updated accordingly, with no foreseeable
% impact on the achievement of tasks, deliverables, and milestones.

% \item{Use of resources Reporting period 2}
% All changes of use of resources were included in the two amendments previously cited and were
% due to modifications in the personnel. Those adjustments were due to the change of positions
% of some key \ODK participants and expected difficulties in hiring planned
% staff. The work plan has been updated accordingly, with no foreseeable
% impact on the achievement of tasks, deliverables, and milestones.

Another minor deviation in the proposed use of resource was that FAU hired students to do
some routine jobs (simple formalizations, mathematical data curation, interface work for
MathHub.info) that did not require the attention of a mature researchers. As the pay grade
of student assistant is roughly 1/4 of that of full researchers, this action was
cost-effective. An unplanned effect was that the reported person months went up
considerably, exceeding the planned amount, without incurring additional cost.

Simula was within budget for costs, but with an increase in person months.
The first project-related reason was shifting person-months from senior staff with high salaries
to postdoctoral and research engineer staff.
This was caused by a combination of senior staff departing Simula
and the presence of experienced personnel at lower salary rates.
Additionally, exchange rate fluctuations between Norwegian kroner and Euros contributed
to person months being significantly less costly than in the original budget.
The result was a cost-effective expenditure of person months,
delivering more mature results in Work Package 4,
without incurring any additional cost.

% Due to cancellation of WP7, the efforts of UOXF were redirected to other WPs, including
% WP2, where they were originally not active. Within WP2, UOXF delivered mini-courses on various ODK components,
% such as GAP and \Sage.

% A major deviation from the work plan for CNRS was an overspending of
% PM in \WPref{hpc} and underspending in \WPref{UI}. This reallocation
% is due to a change of priorities in the tasks assigned to CNRS and the
% specific competences of the hired engineer. In particular we decided
% to drop \taskref{UI}{Sage-display}: this task is generally useful in
% the long term, but low priority; since the writing of the proposal, it
% appeared that its implementation would require much larger amounts of
% code refactoring and backward incompatibilities than originally
% expected; in addition the landscape of visualization libraries (in
% particular in javascript) has been evolving at a very fast pace which
% is likely to quickly; hence the refactoring will be more efficiently
% achieved in a later project, once the landscape will have stabilized.

% Taking this decision enabled us to reinvest the freed resources
% (26PMs) on more critical and urgent tasks, as listed below. Additional
% efforts are made to provide native Windows support for the \ODK
% components, in particular the cysignals library (6PM); this was
% requested by many users and developers and will enable the native use
% of PARI/GP from Python via cypari on Windows. Some extra efforts are
% invested in the fine grain parallelization of PARI/GP (4PM). Finally,
% most efforts are dedicated to modularity and Python3 compatibility for
% \Sage (16PM).

% A major deviation for USTAN is the reduced cost of manpower due to exchange rate changes. In the application
% we used an expected rate of \euro 1 = \pounds 0.69. In our most recent financial report we used an actual
% rate of \euro 1 = \pounds 0.88. The monthly rates paid in pounds have been in line with our budget as had the 
% effort in terms of man months, but the cost of this in Euros has fallen dramatically. We anticipate a modest increase in 
% effort beyond the original budget over the final reporting period, but expect to fall well within the finanicial budget.

% UVSQ claiming effort in WP6 is due to Luca De Feo spending 5 hours on WP6 from April 24th to 28th, 2017. He helped Michael 
% Kolhase's team (from FAU) to prepare the WP6 presentations. The benefits of his collaboration are here:
% \url{https://link.springer.com/chapter/10.1007%2F978-3-319-72453-9_14}.

\subsection{Follow-up to comments of the financial officers}

\subsubsection{\site{PS}}

\begin{EUcomment}
  The partner has claimed unforeseen equipment costs and unforeseen
  costs for an implementation contract. A justification should be
  included in the Technical Report.
\end{EUcomment}

Following a recommendation of the Project Officer and the Referees at
the occasion of the First Formal Review, we contracted with external
communication professional to coproduce visual content for our web
site, to better explain and disseminate the project. This included
comics, coproduced with graphic designer Juliette Belin, and high
quality motion graphics video, coproduced with the PixVideo company.
This expense was not foreseen in the Annex 1.

CoCalc.com (Collaborative Calculation) is a Virtual Environment for
collaborative computation, with strong ties with OpenDreamKit. As
planned in Annex 1, we have been using it extensively at Paris-Sud to
assess its features and applicability for Research and Education
purposes. During the first year of OpenDreamKit, we realized that the
free version was too limited for our purposes -- notably to support
and observe its in-class use by colleagues. We therefore decided to
upgrade to the pro version for the remaining duration of the project.
This expense was not foreseen in the Annex 1.

\begin{EUcomment}
  Overspending by 97.22\% (8.75PM) in WP2 and underspending by 46.32\% (11.58PM) in WP6.
  Please explain these deviations in the Technical Report.
\end{EUcomment}
Generally speaking \site{PS} -- WP2 lead -- was very active in
community building and dissemination activities. Following
\site{LEEDS} early termination, \site{PS} took over some of the WP2
work originally planned at \site{LEEDS} (WP2), notably using the freed
financial resources to organize many more dissemination events.

Following the success of WP6's Math-in-the-Middle (MitM) approach for
high level interoperability between systems, \site{PS} got engaged in
a stronger than originally planned collaboration with \site{FAU} and
\site{SA}, notably for MitM integration between GAP and SageMath.

\subsubsection{CNRS}
\begin{EUcomment}
  The partner is underspending significantly in WP1 and WP4 and
  overspending by 96.43\% in WP5. Please explain these deviations in
  the Technical Report.
\end{EUcomment}
The underspending in WP1 is due to an initial overestimation of the
cost of Management.

Similarly to what was already mentioned in the second technical report, the
underspending in WP4 and overspending in WP5 is due to a reallocation of
efforts due to a change of priorities and the specific competences of the
engineer. More specifically, we spend a lot of work on helping the transition
of SageMath from using Python2 to using Python3 which will concretely happen
in the next release before the end of the year.

\subsubsection{Université de Bordeaux (Third party of beneficiary CNRS)}

\begin{EUcomment}
  The partner has claimed effort in WP1 and WP4 in which they are not
  active. These costs are not eligible unless properly explained in
  the Technical Report.
\end{EUcomment}
Efforts have been claimed in WP1 to take into account participation in
the write up of deliverable reports and
the participation of Bill Allombert in the last review meeting.

Efforts have been claimed in WP4 as UB integrated better visualization tools in
PARI/GP (e.g. parallel computation for plotting). This was not planned in the
Proposal but turned out to be a highly desirable feature.

\subsubsection{\site{UJF}}
%CFS for 407,695.18 euro direct costs; and 509,618.98 euro total is NOT SIGNED; the indicated cost of the certificate refers to the amount of the direct costs, instead shall be zero, please correct it; the attached auditors guidelines refer to another project: Grant Agreement number: 681044 –RESSTORE – H2020-PHC-2014-2015/H2020-PHC-2015-single-stage_RTD, please use those for OpenDreamKit Project and they shall be signed by the auditor The ToR signed by both parties is missing.ALREADY DEALT WITH BY THE ADMIN

\begin{EUcomment}
  Please confirm that the adjustment amount of 10,859.29€ from RP1 is
  related to costs and not effort.
\end{EUcomment}
This adjustment is related to effort: the central services of UGA
forgot to include the time sheets for Jean-Guillaume Dumas for years
2015 and 2017 in their statements for RP1.

\begin{EUcomment}
  No explanation provided for other direct costs declared in RP2.
\end{EUcomment}
These costs correspond to expenses already engaged but which were still not being paid by the university at
the time of RP2 evaluation.

\begin{EUcomment}
  Significant underspending in the use of the budget for other direct
  costs. In addition to less face-to-face meetings and travel costs
  resp, there must be other reason(s) for this deviation.
\end{EUcomment}

The under-spending in direct costs is indeed mostly due to smaller
conference travel expenses (44\%) and workshop organization costs
(39.3\%). The original budget had been set according to the general
travel budget guidelines of OpenDreamKit which turned out to largely
overestimate the actual local needs. Factors included the rather
central location of UGA and the limited interaction that the locally
hired engineers needed with other sites. Cl\'ement Pernet also had to
limit his long distance travels to only once per year, to accommodate
with the birth of his two children during the project.

Other minor factors included the increase in the deprecation duration for the
servers bought (due to the change from UJF to UGA regulations) and the absence of auditing costs.

More precisely, the under-spending of roughly 55000 euros splits in the following way:

\begin{itemize}
\item travels: 24310
\item workshop organization: 21623
\item deprecation for servers: 5282
\item Audit costs: 4000
\end{itemize}


\begin{EUcomment}
The adjustment related to RP2 needs explanation both for the reduction of direct personnel costs and for adding 14933,35 euro in other direct costs.
\end{EUcomment}

This adjustment in personnel costs was necessary due to changes in the methodology of calculating the hourly rate of the engineer HongGuang ZH, involved in ODK (WP5). The adjusted amount is related to costs not efFort in the FS.

 Explanation for the addition of 14 933.35 euros in other direct costs:
UGA forgot to report 5 140.14 of Depreciations expenses in RP2 : 
 4 378,02 : multi core large scale server for the development and experimentation of parallel codes (D5.12 ,D5.14 and D3.11)
 419,82: Laptop used in  development workshops and during presentations at conferences ( WP3, WP5) 
 342,30: Small scale Skylake architecture Server for the development and experimentation of the vectorisation (D5.12 D5.14-WP5, D3.11- WP3)

 9793.21 Euros of travel costs incurred during period 2 but not declared in the financial statement :
- 1757.65: travel expenses  for collaborative works between Vincent Neiger and Clement Pernet (ODK member) first meeting on February 14-17, 2018 in Grenoble (Université Grenobles Alpes) and second one on March 19-24, 2018 in Copenhagen (Technical University of Denmark.)
- 98.80 : Clément Pernet (ODK member) for attending the Structured Matrix Days, on May 14-15, in Lyon (France)
- 1482.69 : Jean Guillaume Dumas (ODK member) for attending the Fun with Algorithms Conference on June 12-17 2018, Pisa, Italy
- 1027.94 : travel expenses for 2 ODK members (C. Pernet-103.25, J.-G. Dumas-500,49) and 3 participants ( David Lucas - 76.25 euros, Cyril Bouvier-76.25 euros, Pierre Karpman 76.25 euros, Pascal Giorgi-195.45 euros) to the LinBox Days June 20-22 2018 in Grignan –development workshop organized by UGA(France)
- 2550.72: Clément Pernet (ODK member) for attending the ISSAC , July 16-19 2018, New York, USA,
- 2875,41 : Jean Guillaume Dumas (ODK member) for attending the the ISSAC , July 16-19 2018, New York, USA

\subsubsection{\site{UK}}

\begin{EUcomment}
  The partner is overspending in WP1, WP2 and WP4 by 100\%. A
  justification shall be provided in the Technical Report.
\end{EUcomment}
We believe what is meant here is underspending. For these work packages, in the OpenDreamKit Proposal, 
the \site{KL} project effort in person‑months was set to be 2 PM each. In WP1 and WP2, the relevant tasks were taken care 
of in full by Professor Decker, the lead PI. However, this does not appear in our cost statement since, as written
in the OpenDreamKit proposal,  Professor Decker’s activities within OpenDreamKit (6 PM), including the related 
overhead, were covered by \site{KL} (and were therefore not part of the requested funding). To conclude, there is 
no \site{KL} underspending in WP1 and WP2. With regard to WP4, our contribution to the deliverables was the 
implementation of a Jupyter interface for Singular, as listed as a part of task T1 of WP4. As it turned out, 
such an interface has been created by other researchers in a more general context, and made available to 
Singular and the open source community. To conclude, our contribution to the deliverables in WP4 arrived
in due time, but there was no need to spend OpenDreamKit funds here.

\begin{EUcomment}
  Significant underspending in the use of the budget for other direct
  costs. In addition to less face-toface meetings and travel costs
  resp, there must be other reason(s) for this deviation.
\end{EUcomment}
Indeed, there is a significant underspending here, but this does not mean, that there were less face-toface
meetings than originally planned. To the contrary: 1) International travelling to conferences was typically 
by invitation, with reimbursement covered from the inviting side. 2) Leading largescale international conferences 
took place at TU Kaiserslautern and provided further opportunities to meet, without spending travel money from our 
side (ANTS 2016, ISSAC 2017, and PASCO 2017). 3) The participation in project meetings could often be achieved 
via modern video equipment. Similarly for discussions with researchers from other sites. 4) Due to independent 
developments, the group at Kaiserslautern has significantly increased its size over the past years. 
The considerable additional expertise has drastically reduced both the need to travel to other research centers 
and the need to invite external experts. 5) Within the computer algebra community, it became clear over the past 
years that originally independent developer teams, creating systems for applications in different mathematical 
areas, should join forces, for at least two reasons, the exchange of ideas to solve to  technical problems, and
the development of computational tools for applications across the boundaries of the different areas. As a result, 
quite a number of developer workshops took place in a much larger context, within and without Europe, into 
which in particular problems concerning OpenDreamKit could be embedded. To conclude, there was a significant 
progress in bringing different groups of researchers together, which led to quite a number of success stories, 
but less OpenDreamKit funds were needed.

\subsubsection{\site{UO}}
\begin{EUcomment}
  The partner has declared costs in WP2 and WP4 in which they are not
  active. These costs are not eligible unless properly described in
  the Technical Report.
\end{EUcomment}
Regarding WP2: it was clearly an oversight during the grant preparation, as dissemination and 
outreach was a task where more participants took part than originally planned. In particular Oxford 
was asked to present ODK tools and technologies at a number of international and UK-wide events,
and agreed to such requests, as it felt it is a very important part of the project. This resulted 
in contributing to \longtaskref{dissem}{dissemination} in WP2. As well, it participated in reviewing
new technology, as in \longtaskref{dissem}{tech-review} in WP2.

Regarding WP4: as WP7 was cancelled, the focus of Oxford shifted to other WPs, where its expertise and 
skills were useful and beneficial, such as \longtaskref{UI}{sage-sphinx} and \longtaskref{UI}{vis3d} in WP4.

%for USlaski: Waiting for final submission.

\subsubsection{\site{SA}}
%CFS to be produced.
\begin{EUcomment}
  Significant underspending in the use of the budget for other direct
  costs. In addition to less face-to face meetings and travel costs
  resp, there must be other reason(s) for this deviation.
\end{EUcomment}
Many project-related meetings and events were combined or colocated
with events to which we were travelling for other purposes, supported
by other funding especially CCP-CoDiMa a Uk project which supported
much of our UK travel. Some events we did organise were much less
costly than expected due to lower costs for rooms etc. It also appears
that the  estimation rule for travel costs recommended by the EU and
used in the grant application over estimates the cost of mathematical
conferences and workshops very considerably. For the UK portions of
our costs, the change in exchange rates was also a factor.

Family considerations (especially for Prof.~Linton in 2018 when his
wife was in hospital) also limited travel.

\subsubsection{\site{UV}}
\begin{EUcomment}
  The partner is overspending in WP2 by 137.50\% and has declared
  costs in WP6 in which they are not active. These costs are not
  eligible unless properly described in the Technical Report.
\end{EUcomment}
UVSQ hasn't participated to WP6 during RP3. The participation of UVSQ
to WP6 during RP1 and RP2 has been addressed in the respective technical reports.

Concerning the overspending in WP2 (2.75PM over the course of 4
years), Luca De Feo turned out to be involved in dissemination more
than what was originally planned (on average 2 more
conferences/workshop per year). It should be noted that the format of
some of these conference (Atelier PARI/GP, Sage days, ...) mixes
training sessions with development sessions, it is thus not easy to
delimit the time spent disseminating from the time spent working on
development tasks, e.g., on WP3 and WP4.
Realistically, the overspending is closer to 70\% than to 137\%.

%unless the EU wants researchers to invest in stopwatches;

\subsubsection{\site{UW}}
\begin{EUcomment}
  The partner has claimed effort in WP5 in which they are not active.
  These costs are not eligible unless properly described in the
  Technical Report.
\end{EUcomment}
UWarwich has mistankenly declared costs in WP5, the FS declaration has been corrected and effort has been claimend only in WP1 and WP6. 

\begin{EUcomment}
  Significant underspending in the use of the budget for other direct
  costs. In addition to less face-to-face meetings and travel costs
  resp, there must be other reason(s) for this deviation.
\end{EUcomment}
The largest component of the underspend was the sum of 12k€ which was originally intended for partial support of a workshop joint with our other project, LMFDB.  In the end the LMFDB funds were sufficient and we did not need to call upon this.


\subsubsection{\site{ZH}}
\begin{EUcomment}
  significant underspending in the use of the budget for other direct
  costs. In addition to less face-to-face meetings and travel costs
  resp, there must be other reason(s) for this deviation. The actual
  average personnel costs are significantly higher than planned,
  please clarify the reasons for this deviations (besides the currency
  exchange rate fluctuation).
\end{EUcomment}

At the beginning of the project, Paul-Olivier Dehaye -- unique
OpenDreamKit participant at UZH -- held a temporary position at UZH.
As stated in the proposal, the original plan was for Paul Olivier
Dehaye to spread his involvement (12PM WP6 + 1PM admin) over the whole
duration of the project, under the assumption that Paul-Olivier's
position would be renewed by UZH. This however did not materialize,
and Paul-Olivier's contract at UZH expired at the end of year 1. The
consortium then decided with Paul-Olivier to focus his involvement in
ODK during year 2, staying at UZH under full-time ODK funding.

This had two consequences:
\begin{itemize}
\item Paul-Olivier used travel money during two years instead of four
  years as originally planned. This explains the underspending in the
  other direct costs.
\item Part of the negotiation with UZH to extend Paul-Olivier's stay
  was to recruit him during year 2 as Assistant Professor, a position
  that came with minimal salary requirements that were above the
  budgeted ODK money. This explains the higher than planned average
  personnel costs. This additional expense was taken from the leftover
  direct costs.
\end{itemize}

\subsubsection{\site{LL}}
%CFS to be produced. 
\begin{EUcomment}
  The partner is underspending in WP5 by 95.83\% and has a declared a
  higher monthly average rate (7,889.82 vs. 5,937.75). The actual
  average personnel costs are significantly higher than planned,
  please clarify the reasons for this deviations.
\end{EUcomment}
For WP5, as stated in the project proposal, Logilab has subcontracted
most of the work to Serge Guelton (12 PM). Therefore, the costs are not
charged as regular direct personnel costs but have been declared in the
appropriate line (line d, subcontracting costs).

\begin{EUcomment}
  Significant underspending in the use of the budget for other direct
  costs. In addition to less face-to-face meetings and travel costs
  resp, there must be other reason(s) for this deviation.
\end{EUcomment}
The other direct costs had been planned following the general project
guidelines for meetings and travel. In retrospect, these general
guidelines overestimated the specific needs for Logilab, notably due
to subcontracting. For example, in the initial proposal, Serge Guelton
(subcontractor for most of the work in WP5) intended to participate to
the SciPy conference (major world conference, held in the USA, around
the SciPy library and the scientific Python usages). Mr Guelton did
not have the opportunity to do so. Nevertheless, a lot of work was
done during the project to begin the integration of the developments
conducted in OpenDreamKit in significant Python projects such as
Cython. This work ensures the visibility and the sustainability of
OpenDreamKit achievements.

\subsubsection{\site{SR}}
\begin{EUcomment}
  The partner should explain the underspending in WP2 and the
  overspending in WP4 in the Technical Report. The partner has also
  declared a much lower monthly rate (6,607.88 vs. 10,085.28). Please
  explain in the Technical Report.
\end{EUcomment}
The first project-related reason for the lower monthly rate was shifting person-months from senior staff with high salaries
to postdoctoral and research engineer staff.
This was caused by a combination of senior staff departing Simula, including one death very early in the project,
and the presence of experienced personnel at lower salary rates.
Additionally, exchange rate forecasts in the budget systematically overestimated the cost of kroner in Euros,
resulting in person months and direct costs being significantly less costly than in the original budget.
These two factors account for the difference in the monthly rate.

Regarding person months:

Simula reported efforts to disseminate WP4 results via workshops and conferences under WP4 instead of WP2.
2 PMs of effort from Simula reported as WP4 should be reported as WP2 instead,
corresponding to these dissemination efforts.

The result of all of these factors was a cost-effective expenditure of person months,
allowing additional effort on WP4 without incurring any additional cost.
Since the work of software that people actually use is never complete,
we were able to use the unspent budget on effort to deliver more mature results in WP4,
actively responding to the user community reactions to work delivered over the course of the project.
The result was improved impact and sustainability of work delivered in Work Package 4,
without any increase in cost.

\begin{EUcomment}
  Significant underspending in the use of the budget for other direct
  costs. In addition to less face-to face meetings and travel costs
  resp, there must be other reason(s) for this deviation
\end{EUcomment}
Organization of the Jupyter Workshop at Simula proved significantly
less costly to Simula, due to
\begin{itemize}
    \item Fewer participants needed travel support than planned (ODK participants used their own budgets, rather than being reimbursed through Simula)
    \item Hosting the event at Simula instead of an external venue, making the venue cost zero
\end{itemize}

No open access publication fees were required because Simula did not submit any articles for publication in journals with open access fees.
WP4 efforts were disseminated through conferences and events which do not have publication fees.

Travel costs were significantly reduced by a number of factors:
\begin{itemize}
  \item On some occasions, hosting institutions and venues paid for travel, so project funds were not required for all trips
  \item Costs were billed in Norwegian kroner, reducing billed totals at the end of reporting periods due to exchange rates
  \item Parental leave and visa challenges resulted in requiring more remote participation
    than travel for parts of RP2 and RP3
  \item The growing concern that frequent academic airline travel is a significant contributor to climate change,
    increasing pressure to collaborate remotely rather than via in-person gatherings
\end{itemize}

\subsubsection{\site{UGent}}
\begin{EUcomment}
An adjustment of 98.17 euro for direct personnel costs as actual costs is not explained 
\end{EUcomment}
UGent made a small correction to the personnel cost calculation of Jeroen Demeyer of 2018. Mr. Demeyer took some paternity leave in August-September and had not added the costs of those days.

\subsubsection{\site{XFEL}}

\begin{EUcomment}
The adjustment of minus 93.78 euro for direct personnel costs declared as actual costs, related to Period 2, needs explanation in the 
Financial Statement“
\end{EUcomment}

The Adjustment was made, because in Project Period 2, it was only possible to estimate the Christmas pay, which is paid as part of the salary in November each year by European XFEL. Unfortunately the estimation was 93.78 Euro too high , therefore the adjustment was made. This has not influence on the number of person month worked for the project in Project Period 2. For information: The personnel costs are calculated on a monthly bases and European XFEL uses the standard amount of  annual productive hours (1720 hours/year).

We wanted to write a comment in the Participant Portal, but since the change had not influence on the number of person month there was no possibility to do so in the Participant Portal.

% for JacobsUni:
\subsubsection{\site{JU}}

\begin{EUcomment}
  The actual average personnel costs for Jacobs University Bremen GGMBH are significantly lower than planned. Please
  clarify the reasons for this deviations ( besides the currency exchange rate fluctuation).
\end{EUcomment}

The actual average personnel costs of JacobsUniv were significantly
lower than planned. This is a consequence of the fact that personnel
costs significantly depend on the seniority of the employee. We have
been able to hire junior researchers, which were cheaper than the
conservative estimate in the proposal.

\subsubsection{\site{FAU}}

\begin{EUcomment}
in the FS we need a confirmation that the adjusted amount is related to costs not effort. 
The partner should explain the 76.94\% overspending in terms of effort in WP6. The partner is
also using a much lower average rate. These deviations should be explained in the technical report.
\end{EUcomment}

We confirm that the adjusted amount is related to costs not effort in the FS.
FAU overspent in terms of effort by 76.94\% in WP6 and correspondingly used a lower monthly
average rate. The reason for this is that FAU hired students to do some routine jobs
(simple formalizations, mathematical data curation, interface work for MathHub.info) that
did not require the attention of a mature researchers. As the pay grade of student
assistant is roughly 1/4 of that of full researchers, this action was cost-effective. An
unplanned effect was that the reported person months went up considerably, exceeding the
planned amount, without incurring additional cost.


\subsubsection{\site{LEEDS}}
%Waiting for final submission.
\begin{EUcomment}
  Significant underspending in the use of the budget for other direct
  costs. In addition to less face-to-face meetings and travel costs
  resp, there must be other reason(s) for this deviation.
\end{EUcomment}

Due to all of the \site{LEEDS} participants -- including the PI
himself -- being progressively hired by the industry, all activities
stopped before or early on in the Reporting Period 3. Hence no
other direct costs were claimed for Reporting Period 3.

\begin{EUcomment}
  The actual average personnel costs are significantly higher than
  planned, please clarify the reasons for this deviations (besides the
  currency exchange rate fluctuation).
\end{EUcomment}

As already mentioned, all activities at Leeds unexpectedly stopped at
the end of Reporting Period 2. Following the recommendations of the
referees and project officers during the second formal review, we
sought for opportunities to transfer some of the leftover resources to
other sites to make the best use of them for activities there. This was formalized in
Grant Agreement 5. However the amendment only formalized transfers of
budgets, not of PM's. By way of consequence the Grant Agreement still
planned 22PMs at Leeds for a personnel budget of 25970€, making for a
completely unrealistic average personnel costs of 1.1k€/PM.

The actual average cost for Leeds has been of 3.71k€/PM (9k€ /
2.42PM), while the originally planned average personnel costs for
Leeds before Grant Agreement 5 was of 6.4 k€/PM (141k€ / 22 PM). The
discrepancy is unsurprising: indeed the use of resources at Leeds are
marginal compared to the overall use of resources for Sheffield +
Leeds; the relevant metric is the average costs for Sheffield + Leeds
over the whole duration of the project.

%The following changes are required for the technical part:
%– The deviations explained in the Technical Report include the changes that have been accepted in the amendment. What has been accepted in the last amendment is what should be considered as final figures. Only deviations that are below/ above those should be explained. After the Amendment, Subcontracting is considered foreseen; Task is foreseen as well.
%I. Technical Report
%a. Content: the format of the report deviates from the H2020 template. Please respect the numbering of the chapters as in the template (e.g. chapter 5.1 shall be Tasks; 5.2 Use of resources; 5.2.1 Unforeseen subcontracting; 5.2.2 Unforeseen use of in kind contribution from third party against payment or free of charges, etc.)
%b. Use of resources:
%i. Overall, UNIVERSITE GRENOBLE ALPES, TECHNISCHE UNIVERSITAET KAISERSLAUTERN, THE UNIVERSITY COURT OF THE UNIVERSITY OF ST ANDREWS, THE UNIVERSITY OF WARWICK, UNIVERSITAT ZURICH, LOGILAB, SIMULA RESEARCH LABORATORY AS, and UNIVERSITY OF LEEDS register significant underspending in the use of the budget for other direct costs. In addition to less face-to face meetings and travel costs resp, there must be other reason(s) for this deviation, which need(s) to be addressed in the relevant chapter of the Technical Report.
%ii. The actual average personnel costs for JACOBS UNIVERSITY BREMEN GGMBH, SIMULA RESEARCH LABORATORY AS and FRIEDRICHALEXANDERUNIVERSITAET ERLANGEN NUERNBERG are significantly lower than planned; and for UNIVERSITAT ZURICH, LOGILAB and UNIVERSITY OF LEEDS are significantly higher than planned. Please clarify the reasons for this deviation (besides the currency exchange rate fluctuation).

%%% Local Variables:
%%% mode: latex
%%% TeX-master: "report"
%%% End:


%%% Local Variables:
%%% mode: latex
%%% mode: visual-line
%%% TeX-master: "report"
%%% End:


%  LocalWords:  WPref hpc WPref dksbases subsubsection delivtref dissem organizing emph
%  LocalWords:  Jupyter emph ipython-kernels psfoundation formalizations WPs taskref
%  LocalWords:  WieKohRab:vtuimkb17,KohMuePfe:kbimss17 visualization stabilized cysignals
%  LocalWords:  cypari parallelization
