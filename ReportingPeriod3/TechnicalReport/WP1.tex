\subsubsection{Work Package 1: Project Management}

%Explain, task per task, the work carried out in WP during the reporting period giving details of the work carried out by each beneficiary involved.

%%%%%%%%%%%%%%%%%%%%%%%%%%%%%%%%%%%%%%%%%%%%%%%%%%%%%%%%%%%%%%%%%%%%%%%%%%%%%%
\paragraph{Overview}

As in the previous reporting periods, \site{PS} coordinated \ODK
in close collaboration with the other beneficiaries to ensure that:
\begin{enumerate}
\item{the objectives of the project were met within the agreed budget
    and the timeframe specified by milestones and deliverables;}
\item{all the risks jeopardising the success of the project are managed and that the final results are of high quality;}
\item{the innovation process within the project is fully aligned with the objectives set up in the Grant agreement.}
\end{enumerate}

%%%%%%%%%%%%%%%%%%%%%%%%%%%%%%%%%%%%%%%%%%%%%%%%%%%%%%%%%%%%%%%%%%%%%%%%%%%%%%
\paragraph{Tasks}

\subparagraph{\longtaskref{management}{project-finance-management}}

\begin{itemize}
\item \site{PS} took care of the budget management together with the
  administration body, the D.A.R.I. (Direction des Activités de
  Recherche et de l'Innovation) and its finance service. This included
  prefinancing, funds transfer to cater for the moving of personnel
  across sites and from old sites to new sites, and the coordination
  of financial reports.
\item In earlier reporting periods, \site{PS} led four amendment
  processes to the Grant Agreement, to manage work plan revisions and
  to reallocate staff and all remaining resources from the four
  terminated beneficiaries \site{ZH}, \site{USH}, \site{JU},
  \site{USO} to the added beneficiaries \site{UG}, \site{FAU},
  \site{XFEL}.

  \noindent
  During Reporting Period 3, \site{PS} led a fifth amendment upon
  the request of \site{FAU} to add a subcontractor to conduct a new
  task \longtaskref{dksbases}{isabelle}. Following the suggestions of
  the Project officer, \site{PS} used the occasion to formalize budget
  transfers between beneficiaries to optimize the use of remaining
  resources to achieve the project aims; this was notably required to
  exploit resources left at LEEDS following the early departure of all
  its personnel.

\item In earlier reporting periods, \site{PS} had organized the first
  project review in Brussels on April 2017, and steering committee
  meetings in Orsay (September 2015), St Andrews (January 2016),
  Edinburgh (January 2017), Brussels (March 2017), online (February
  2018), and at \site{XFEL} (June 2018).

  \noindent
  During Reporting Period 3, \site{PS} organized the second project
  review in Luxembourg on October 2018 and steering committee meetings
  in Luxembourg (October 2018) and Marseille (February 2019). It also
  organized a one week "report writing sprint" in Cernay (August 2019)
  to collectively write the project reports. The final review meeting
  will take place in Luxembourg on October 30, 2019, and will gather
  20 \ODK participants to present the project final results.

\item As in earlier reporting periods, \site{PS} ensured that all the
  milestones and deliverables of Reporting Period 3 were achieved
  within its timeframe, and reported on in a timely manner.

\item As in earlier reporting periods, \site{PS} maintained the
  internal and external communication tools that were described in
  \longdelivref{management}{infrastructure}. The project website was
  continuously updated with new content, and virtually all work in
  progress is openly accessible on the Internet to external experts
  and contributors (for example through open source software
  repositories on Github).
  % A new version of the website was released in June 2018.
  % Its end-user friendly interface and content makes it a tool not only
  % for internal communication but very much for dissemination and
  % progress tracking by the reviewers and the community.

\item Concerning the future of \ODK and of its infrastructure toolkit,
  the consortium kept accessing information and getting involved in
  the development of the European Open Science Cloud that is currently
  promoted by the European Commission. The project manager
  participated to the following events:
  \begin{itemize}
  \item \href{https://www.eosc-hub.eu/events/eosc-hub-week-2019}{EOSC
      Hub week}, April 10-12 of 2019, Prague, Czech Republic.
  \item
    \href{https://www.eosc-hub.eu/events/building-open-science-europe-road-ahead-eosc-community}{Building
      Open Science in Europe: The road ahead for the EOSC community
      and the EU Member States}, June 20th of 2019, Tallinn, Estonia.
  \item ICT Proposers' Day 2019, September 19-20th of 2019, Helsinky, Finland.
  \end{itemize}

  \noindent
  In addition, two spin-off proposals were submitted on January 29th
  to the H2020 European E-Infrastructure call INFRAEOSC-02-2019:
  \begin{itemize}
  \item \href{https://github.com/bossee-project/proposal}{BOSSEE}:
    Building Open Science Services on European E-Infrastructure, with
    a focus on Jupyter and applications;
  \item \href{https://opendreamkit.org/2019/01/29/FAIRmat/}{FAIRMAT}:
    FAIR Mathematical Data for the European Open Science Cloud.
  \end{itemize}
  Both were prepared with the same open strategy as OpenDreamKit, and
  involved new combinations of OpenDreamKit and external
  beneficiaries. None was accepted but the respective consortia are
  determined to resubmit them or variants thereof at the earliest
  opportunities.

  % We took advantage of
  % the EOSC stakeholder forum on 28-29 November and the 2017 edition of
  % the DI4R (Digital Infrastructures for Research) in Brussels. During
  % these events we gathered information on the potential of EOSC and
  % how \ODK could fit in there. Furthermore we initiated a partnership
  % with EGI -- a key participant to EOSC -- to deploy \ODK based
  % infrastructure.


\end{itemize}

\subparagraph{\longtaskref{management}{project-quality-management}}

We recall that the Quality Assurance Plan is described in detail in
\longdelivref{management}{ipr}.

\begin{itemize}
% \item Continued success in the recruitment of highly qualified staff.
% \end{enumerate}
\item As in the previous reporting periods, \site{PS} organized the
  interaction with the Advisory Board, composed of seven members, some
  of which specifically represent End Users:
  % The other structure supporting \ODK to ensure the quality of the
  % infrastructure it produce is the Advisory Board. It is
  \begin{itemize}
  \item{Lorena Barba from the George Washington University}
  \item{Jacques Carette from the McMaster University}
  \item{Istvan Csabai from the Eötvös University Budapest}
  \item{Françoise Genova from the Observatoire de Strasbourg}
  \item{Konrad Hinsen from the Centre de Biophysique Moléculaire}
  \item{William Stein, CEO of SageMath Inc.}
  \item{Paul Zimmermann from the INRIA}
  \end{itemize}
  This Advisory Board is composed of academics and/or software
  developers from different backgrounds, countries and communities. It
  is a strong asset to understand the needs of a variety of end-user
  profiles.

\item As in the previous reporting periods, The Quality Review Board
  monitored the quality and the relevance of the software development
  relative to the end-user needs. This board -- chaired by Hans
  Fangohr and composed of four members with a track record of caring
  about the quality of software in computational science -- is
  responsible for ensuring key deliverables do reach their original
  goal and that best practice is followed in the writing process as
  well as in the innovation production process. It met after the end
  of each Reporting Period (RP), and before the Review following that
  RP. More details are given in Section~\ref{section.QAP}.

% More information on
% \longtaskref{management}{project-quality-management} can be found in
% Section 4 of this document: Quality assurance plan.

\item \site{PS} has also been managing risks. Up to the Leeds
  situation, the assessment we present in
  Section~\ref{section.risk_management} has only marginally deviated
  from earlier assessments at Month 12 (\delivref{management}{ipr})
  and Month 36 (\delivref{management}{ipr2}).
\end{itemize}
\subparagraph{\longtaskref{management}{project-innovation-management}}

\begin{itemize}
\item At month 18, \site{PS} had with the help of the consortium a
  first version of Innovation Management Plan
  (\delivref{management}{imp1}), with focuses on:
  \begin{itemize}
  \item{The open source aspect of the innovation produced within \ODK;}
  \item{The various implementation processes the project is dealing with;}
  \item{The strategy to match end-users needs with the promoted VREs}.
  \end{itemize}

  \noindent
  During Reporting Period 3, \site{PS} produced a second version of
  the Innovation Management Plan (\delivref{management}{data-plan2}).
  % to assure that research activities meet the required milestones and
  % produce outputs fully aligned with the project objectives.
  It confirms the elements of strategy of the previous plan, and adds
  two additional sections: one on the choice and impact of open
  licenses in the context of OpenDreamKit, and one on the different
  types of outcome of the project and their respective sustainability.
  This second version also includes minor updates to the earlier
  sections, notably to reflect the work plan revisions that occurred
  after Reporting Period 1.
\end{itemize}

%%% Local Variables:
%%% mode: latex
%%% TeX-master: "report"
%%% End:

%  LocalWords:  subsubsection longtaskref delivref organized Bougeret ipr Csabai
