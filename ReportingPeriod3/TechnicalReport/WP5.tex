\subsubsection{WorkPackage 5: High Performance Mathematical Computing}
  \label{hpc}
%Explain, task per task, the work carried out in WP during the reporting period giving details of the work carried out by each beneficiary involved.


  %%%%%%%%%%%%%%%%%%%%%%%%%%%%%%%%%%%%%%%%%%%%%%%%%%%%%%%%%%%%%%%%%%%%%%%%%%%%%% 
  \paragraph{Overview}

  This work package is about better exploiting modern parallel
  computer architectures in computational mathematics software,
  notably when deployed within a Virtual Research Environment. It is
  addressed at the level of individual computational components
  (\Pari, \GAP, \Linbox, \MPIR, \Sage, \Singular, ...), and also at
  the level of interfacing and exposing core parallel features to
  higher level programming interfaces.

  Key results obtained over the reporting period are the following:
  %% Only list deliverables produced in the reporting period
  \begin{compactitem}
  \item An MPI based distributed  rational linear system solver based on Chinese remainder algorithm in \Linbox.
  \item A new hybrid algorithm for rational linear system solving based on $p$-adic lifting and Chinese remainder
    algorithm, and its parallel implementation in \Linbox for multi-core servers.
  %% \item A closer integration of \Linbox in \Sage with improved reliability and
  %%   computing efficiency.
  \item A full-featured parallelisation engine, supporting POSIX threads and
    MPI, for \PariGP in production release of the software.
  \item The release of GAP-4.9 allowing compilation  in HPC-GAP compatibility mode.
  %    \item A new super-optimizer for vectorized assembly code and its
  %  exploitation to improve the performances of the MPIR code.
  \item A major redesign of the polynomial arithmetic used in Singular,
    delivering state of the art efficiency.
  \end{compactitem}
  In addition, we investigated how to exploit parallelism when
  combining computational software; see
  \longlocaltaskref{component-architecture}{component-for-HPC} and
  the following milestone.

%%%%%%%%%%%%%%%%%%%%%%%%%%%%%%%%%%%%%%%%%%%%%%%%%%%%%%%%%%%%%%%%%%%%%%%%%%%%%%
\paragraph{Milestones}

\subparagraph{\longmilestoneref{hpc-prototype}}

\emph{“User story: Astrid wants to run compute intensive routines
  involving both dense linear algebra and combinatorics. She has
  access through a JupyterHub-based VRE to a high end multi-core
  machine which includes a vanilla \Sage installation. She
  automatically benefits from the HPC features of the underlying
  specialized libraries (\Linbox, ...). This is a proof of concept
  of the overall framework to integrate the HPC advances of
  specialized libraries into a general purpose VRE.
  % 
  It will prepare the final integration of a broader set of such
  parallel features for the end of the project.”}

This milestone was already adressed at the second review meeting.
Nonetheless, we now report on further developments bringing much of
this proof of concept to production.

The  solution then proposed was limited to the use of a multi-threaded BLAS and required a minor configuration
change at the install time of the \Sage software. \longdelivref{component-architecture}{hpc-configure} made this process
more flexible to the user, while offering a broader range of parallel routines.
Indeed, the Singleton class \texttt{Parallelism} now provides the user with total control over the number of threads to
be used for the linear algebra computations.
The following session demonstrates the gain in parallelizing the product of a random $8000\times 8000$ matrix over
$\mathbb{Z}/65521\mathbb{Z}$ with itself using 16 cores, using a
vanilla Sage 8.9 installation:
\begin{lstlisting}
pernet@dahu34:~/soft/sage$ ./sage
SageMath version 8.9 Release Date: 2019-09-29
sage: a=random_matrix(GF(65521),8000)
sage: Parallelism()
Number of processes for parallelization:
 - linbox computations: 1
 - tensor computations: 1
sage: time b=a*a
CPU times: user 17.5 s, sys: 1.04 s, total: 18.5 s
Wall time: 18.5 s
sage: Parallelism().set("linbox",16)
sage: Parallelism()
Number of processes for parallelization:
 - linbox computations: 16
 - tensor computations: 1
sage: time b=a*a
CPU times: user 28.9 s, sys: 4.85 s, total: 33.8 s
Wall time: 2.41 s
\end{lstlisting}

Moreover, as announced in the previous technical report, the parallel
code exposed to the user now includes many more routines, especially
those related to exact Gaussian elimination provided by the
\fflasffpack library: \texttt{Det}, \texttt{Rank}, \texttt{Solve},
\texttt{Echelon Forms}, etc.

%% With Deliverable~\delivref{hpc}{LinBox-algo}, we developed, released and integrated in the
%% \Sage the LinBox library and its core dependencies: fflas-ffpack and givaro.
%% When installing the latest \Sage release on a multithreaded multicore server, it
%% only takes one configure option to let fflas-ffpack use a multi-threaded BLAS
%% and therefore expose its parallel speed-up to the end-user of Sage. This feature
%% is compliant with the use of a higher level of parallelism, through process
%% workstealing queues that \textit{Astrid} may be using in her combinatorics code, as those
%% exposed in \delivref{hpc}{sage-HPCcombi}. Now that this first proof of concept has been
%% successfully achieved, we are working in exposing the more advanced parallel
%% routines of fflas-ffpack into \Sage, following~\delivref{hpc}{LinBox-DSL}. It
%% should in particular make Gaussian elimination and related routines enjoy a
%% better scaling with respect to available CPU cores.

%%%%%%%%%%%%%%%%%%%%%%%%%%%%%%%%%%%%%%%%%%%%%%%%%%%%%%%%%%%%%%%%%%%%%%%%%%%%%%
\paragraph{Tasks}

\medskip
\subparagraph{\longtaskref{hpc}{hpc-pari}}
Deliverable~\longdelivref{hpc}{pari-hpc1} was merged with
\longdelivref{hpc}{pari-hpc2} in the revised workplan. The deliverable
\delivref{hpc}{pari-hpc2} is released on time.

The release 2.12 of the \PariGP suite features a MultiThread engine,
used transparently in all tools from the suite: the \Pari library, the
command line interface \texttt{gp} and the GP2C compiler. Written in 2015 and
2016, the engine supports sequential evaluation (no parallelism), POSIX
threads and MPI within the same code base. It is now being progressively
used wherever it makes sense in the code base and this is by nature work
in progress. In \Pari-2.12, the MT engine is a central component of
\begin{itemize}
\item fast (near linear time) Chinese remaindering;
\item fast linear algebra over $\mathbb{Q}$ and cyclotomic fields,
  a critical component of the new "Modular Forms" package;
\item polynomial resultants in
  $\mathbb{Z}[X] \times \mathbb{Z}[X,Y]$ (via fast Chinese remainders and
    evaluation / interpolation), a basic tool for algebraic number theory;
\item computation of classical modular polynomials for about 20 classical
invariants (j, Weber functions, small eta quotients\dots);
\item discrete logarithm over finite fields (prime fields and
$\mathbb{F}_{p^e}$ for word-sized prime $p$) ;
\item Adleman-Pomerance-Rumely-Cohen-Lenstra primality proof;
\item Fourier coefficients of $L$-functions (Hasse-Weil and Artin
  $L$-functions);
\item hi-resolution plot of mathematical functions using parallel evaluation.
\end{itemize}

The \texttt{master} branch on the public development server includes further
\begin{itemize}
  \item values of complex $L$-functions (via parallel computation of
    Meijer $G$-functions);
  \item a new thread-safe version of the Multiple Polynomial Quadratic Sieve
    (MPQS) integer factoring algorithm, ready to be parallelized;
\end{itemize}
The parallel-enabled components of the \PariGP suite have been advertised 
(including tutorial sessions) and tested by participants during PARI/GP
workshops in Grenoble (2016), Lyon (2017), Besançon (2018) and Bordeaux
(2019).

\medskip
\subparagraph{\longtaskref{hpc}{hpc-gap}}
\label{hpc@hpc-gap}

Deliverable~\longdelivref{hpc}{GAP-HPC-report} was completed at the end of this period,
reporting all of the developments in the \GAP system during the project which are relevant
to, or provide essential context for, this workpackage.

During reporting period 3 the main areas of effort were:
\begin{itemize}
\item Follow-up work to the integration of HPC-GAP into the main codebase reported in
  the previous period. This work improves the robustness of the system and dramatically reduced the
  differences between the two versions of the source code.
\item
  Development of the \GAP interface to the meataxe64 high performance linear algebra library (to which
  we also contributed significant development effort). This system targets large calculations over
  small finite fields on multi-core shared memory computers.

  These calculations are the key kernel underpinning higher level
  computations in representation theory (eg modular character tables)
  as well as computations in matrix groups and a wide range of other problems.
  
  The interface makes almost all of the
  capabilities of meataxe64 callable from \GAP, something which is not only a quantum leap in
  performance for \GAP in this critical area, but also allows easy prototyping in \GAP of new
  algorithms for meataxe64.  This library can make use of multiple cores whether or not it is being
  called from \HPCGAP. A full set of benchmarks are included in D5.15, but as a highlight, two
  dense random $320\,000\times 320\,000$ matrices over $GF(2)$ can be multiplied in just over 1000
  seconds on a 64 core AMD ``bulldozer'' system.
  \item Release of ``libGAP'' a general-purpose C API for \GAP. This allows any program, including in particular HPC code, to call on the
    functionality of \GAP efficiently and without the need to run a separate \GAP process. This makes
    fine grained interaction possible.

    To interface with \GAP, SageMath formerly used a bespoke
    implementation of ``libGAP``, requiring heavy patching of \GAP.
    Having the functionality available upstream reduced considerably
    the maintenance burden for SageMath developers and packagers
    alike.

  \item Development and release of a new linguistic reflection API in \GAP, allowing \GAP programs to
    access and modify the executable representation of their own functions at run-time. This will be
    the basis of future automatic parallelisation and optimisation tools.
  \item Very much improved profiling tools
  \item Release of the new package ``ferret'' which achieves world-leading performance in partition
    backtrack, a critical, and notoriusly challenging computational kernel
  \item Extensive developments in our testing and release infrastructure with the overall goal of
    ensuring that \GAP users have easy access to a reliable, up-to-date and mutually compatible set of
    versions of the large suite of packages redistributed with \GAP.
\end{itemize}

\medskip
\subparagraph{\longtaskref{hpc}{hpc-linbox}}
  \label{hpc@hpc-linbox}

During this reporting period, we delivered~\longdelivref{hpc}{LinBox-distributed}.

A first focus was made on distributed computing, with an MPI parallelization of a Chinese remainder based
algorithm. The first proof-of-concept implementation was then cleanly integrated in the mainstream code of the
library. Its performance shows a very nice scaling with the number of compute nodes on a 256 cores cluster.

Although this approach is best suited for parallelization on a large
number of nodes, its total computational complexity ($O{\tilde \ }(n^4)$) becomes
a major concern on large instances.
%
The usual alternative approach based on $p$-adic lifting has a better total computational complexity ($O{\tilde \ }(n^3)$), but is intrinsically more
sequential, and therefore less suited for large scale parallelization.
A major contribution in this task is a new algorithm combining $p$-adic lifting and Chinese remaindering in order to
expose more parallelism without sacrifying the gain in complexity.
We also provide a full-featured  implementation of this new algorithm in \Linbox, which delivers high
sequential efficiency and nevertheless scales well up to 16 cores.
This compromise is a good fit for personal computers or the typical
lab-wide computational server researchers have access to.

Lastly, we introduced support for GPUs in the \fflasffpack library and showed how matrix product over a finite
field benefit from these accelerators.

All these software improvements are closely integrated in the mainstream code of the \fflasffpack and \Linbox libraries.

\medskip
\subparagraph{\longtaskref{hpc}{hpc-singular}}
  \label{hpc@hpc-singular}

  The only deliverable under consideration for this reporting period
  is~\longdelivref{hpc}{singular-polyarith}

Multivariate polynomials are represented in Singular using the sdmp format. While this data structure is generally amenable to parallelization, the implementation and some of the algorithms in Singular were not. Much work has been invested in updating the algorithms and data structures and making Singular polynomial arithmetic competitive with other systems. This work has been done in the Singular submodule Flint, whose code is available at \url{https://github.com/wbhart/flint2}.

We now support polynomial exponents of unlimited size with the three basic monomial orderings of lex, deglex, and degrevlex over the integers mod p and rationals. Continued engagement by colleagues in the HPC community including Bernard Parisse, Michael Monagan, Roman Pearce, and Micka\"el Gastineau has been invaluable.

Parallel and serial implementations of the operations of multiplication, division and GCD are complete and perform well in both the dense and sparse cases. The performance is more than competitive with all other systems we are aware of, both on a single core and on multiple cores.

Basic arithmetic in Singular now benefits directly from the implementation and researchers are already working on leveraging the new implementation in other areas, e.g. Gr\"{o}bner bases over rational functions, Gr\"{o}bner bases with a bottleneck on multivariate arithmetic and polynomial factorisation. Early indications are that all of these are going to experience a huge improvement for many real-world research applications.

Other systems such as the new Oscar computer algebra system already benefit directly from the new ODK implementation. 

Sage will automatically benefit directly at the next update of the Singular version in Sage.

\medskip
  \subparagraph{\longtaskref{hpc}{hpc-mpir}}
  \label{hpc@hpc-mpir}

  Not applicable for this period.
  
  \subparagraph{\longtaskref{hpc}{hpc-combi}}
  \label{hpc@hpc-combi}

  Not applicable for this period.


  \subparagraph{\longtaskref{hpc}{pythran}}
  \label{hpc@hpc-pythran}
  Not applicable for this period.

%%   The goal of this task is to make Pythran easily integratable in large-scale
%%   project, taking into account native dependencies, compilation time, memory
%%   footprint, speed and size of compiled binaries as well as multi-platform
%%   support. Integration with \software{cython} is a possible mean to achieve
%%   this goal.

%%   Two projects have been selected for this task: \software{scipy} and
%%   \software{scikit-image}. These projects are relevant for \software{pythran}
%%   because they have many small to medium kernels that can benefit from
%%   compilation. Even though \software{pythran} has not been selected as a scipy
%%   backend, the exchanges with the community have led to a great deal of
%%   improvements of which all \software{Pythran} users take advantage.  The
%%   \software{scikit-image} community is still examining the possibility of using
%%   \software{pythran} as an acceleration mean.

%%   Integration of  \software{pythran} as a \software{cython} backend for
%%   \software{numpy} has improved in various aspects: better error detection,
%%   more supported expression patterns and improved performance for the compiled
%%   expressions.

%% Deliverable~\longdelivref{hpc}{sage-HPCcombi} is shared with
%% Task~\longlocaltaskref{hpc}{hpc-combi}, the status of which we reported on above.

  \subparagraph{\longtaskref{hpc}{hpc-jupyter}}
  \label{hpc@hpc-jupyter}
  
%% It is common for academic High Performance Computing (HPC) clusters to make
%% use of schedulers based on Sun Grid Engine with Son of Grid Engine as one of
%% the most popular. It is used, for example, on the institutional HPC systems
%% in the Universities of Sheffield and Manchester in the United Kingdom. It is also used
%% on the regional N8 HPC facility, a system shared by the eight most research
%% intensive universities in the North of England.
  Not applicable for this period.

%%% Local Variables:
%%% mode: latex
%%% mode: visual-line
%%% TeX-master: "report"
%%% End:


%  LocalWords:  subsubsection hpc compactitem super-optimizer vectorized factorization
%  LocalWords:  texttt fflas-ffpack longmilestoneref emph JupyterHub-based specialized
%  LocalWords:  longtaskref longdelivref delivref finalized mathbb embarassingly taskref
%  LocalWords:  scienceproject refactorization organized LinBox-algo DumPerSul:fcrpmgbd16
%  LocalWords:  Pernet:cqm16,PerSto:tsegqm17 DumKalTho:lticmpdsm16,DumLucPer:cftearp17
%  LocalWords:  Cython Hongguang singular-polyarith sdmp parallelization deglex degrevlex
%  LocalWords:  Monagan Micka parallelized parallelize Imbach hpc-mpir sage-paral-tree
%  LocalWords:  Cilk libsemigroup optimized pythran integratable scipy scikit-image numpy
%  LocalWords:  sage-HPCcombi hpc-jupyter
