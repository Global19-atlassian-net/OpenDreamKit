\subsection{Objectives}
%List the specific  objectives  for  the  project  as  described  in  section  1.1  of  Part B   and describe  the  work  carried  out  during  the  reporting  period  towards  the  achievement  of  each listed objective. Provide clear and measurable details.

For reference, let us recall the aims of \ODK.
\begin{compactenum}[\textbf{Aim} 1:]
\item \label{aim:collaboration} Improve the productivity of
  researchers in pure mathematics and applications by promoting
  collaborations based on mathematical \textbf{software},
  \textbf{data}, and \textbf{knowledge}.
\item \label{aim:vre} Make it easy for teams of researchers of any
  size to set up custom, collaborative \emph{Virtual Research
    Environments} tailored to their specific needs, resources and
  workflows. The \VREs should support the entire life-cycle of
  computational work in mathematical research, from initial
  exploration to publication, teaching and outreach.
  % and bridge the gaps between
  % code, published results, and educational material.
\item \label{aim:sharing} Identify and promote best practices in
  computational mathematical research including: making results easily
  reproducible; producing reusable and easily accessible
  software; sharing data in a semantically sound way; exploiting and
  supporting the growing ecosystem of computational tools.
\item \label{aim:impact} Maximise sustainability and impact in
  mathematics, neighbouring fields, and scientific computing.
\end{compactenum}

Those aims were backed up in our proposal by nine objectives. 


As explained in the preamble, all of our non-administrative work was concentrated on Work Package 6 (Data/Knowledge/Software-Bases). We now list the nine objectives, together with the relevance of Work Package 6 therein, if any (we will give more details on Work Package 6 deliverables later). 

\begin{compactenum}[\textbf{Objective} 1:]
\item\label{objective:framework} ``To develop and standardise an
  architecture allowing combination of mathematical, data and software
  components with off-the-shelf computing infrastructure to produce
  specialised \VREs for different communities.''
  
Work Package 6 is most relevant in standardizing the architecture, or actually in formulating a framework wherein this architecture can be expressed. 

\item\label{objectives:core} ``To develop open source core components
  for \VREs where existing software is not suitable. These components
  will support a variety of platforms, including standard cloud
  computing and clusters. This primarily addresses Aim~\ref{aim:vre},
  thereby contributing to Aim \ref{aim:collaboration}
  and~\ref{aim:sharing}.''

\item \label{objective:community} ``To bring together research
  communities (e.g. users of \Jupyter, \Sage, \Singular, and \GAP) to
  symbiotically exploit overlaps in tool creation building efforts,
  avoid duplication of effort in different disciplines, and share best
  practice. This supports Aims~\ref{aim:collaboration},
  \ref{aim:sharing} and~\ref{aim:impact}.''

Work Package 6 has required getting \Sage, \GAP, \LMFDB and \FindStat developers to work together. 

\item \label{objective:updates} ``Update a range of existing open source
  mathematical software systems for seamless deployment and efficient
  execution within the VRE architecture of objective~\ref{objective:framework}.
  This fulfills part of Aim~\ref{aim:vre}.''


\item \label{objective:sustainable} ``Ensure that our ecosystem of
  interoperable open source components is \emph{sustainable} by
  promoting collaborative software development and outsourcing
  development to larger communities whenever suitable. This fulfills
  part of Aims~\ref{aim:sharing} and~\ref{aim:impact}.''

\item \label{objective:social} ``Promote collaborative mathematics and
  science by exploring the social phenomena that underpin these
  endeavours: how do researchers collaborate in Mathematics and
  Computational Sciences?  What can be the role of \VREs?  How can
  collaborators within a VRE be credited and incentivised? This
  addresses parts of Aims~\ref{aim:sharing}, \ref{aim:collaboration},
  and~\ref{aim:vre}.''



\item \label{objective:data} ``Identify and extend ontologies and
  standards to facilitate safe and efficient storage, reuse,
  interoperation and sharing of rich mathematical data whilst taking
  account of provenance and citability. This fulfills parts of
  Aims~\ref{aim:vre} and~\ref{aim:sharing}.''

  This objective is at the core of \WPref{dksbases}; see
  Section~\ref{dksbases} for details. 


\item \label{objective:demo} ``Demonstrate the effectiveness of Virtual
  Research Environments built on top of \ODK components for a
  number of real-world use cases that traverse domains. This addresses
  part of Aim~\ref{aim:vre} and through documenting best practices in
  reproducible demonstrator documents Aim~\ref{aim:sharing}.''

  
%Long term sustainability
\item \label{objective:disseminate} ``Promote and disseminate
  \ODK to the scientific community by active communication,
  workshop organisation, and training in the spirit of open-source
  software. This addresses Aim~\ref{aim:impact}.''
\end{compactenum}




