\subsubsection{Work Package 1: Project Management}

%Explain, task per task, the work carried out in WP during the reporting period giving details of the work carried out by each beneficiary involved.

The general objectives of Work Package 1 are:

\begin{itemize}
\item{Meeting the objectives of the project within the agreed budget and timeframe and carrying out control of the milestones and deliverables}
\item{Ensure all the risks jeopardising the success of the projects are managed and that the final results are of good quality}
\item{Ensuring the innovation process within the project is fully aligned with the objectives set up in the Grant agreement}
\end{itemize}

\WPref{management} has been divided into three tasks. In the following, progress is reported with respect to these individual tasks.
Key results of WP1 are the following:

\begin{itemize}
\item{A Consortium Agreement signed by all partners}
\item{A kickoff meeting and three progress meetings organised}
\item{A successful interim review at month 9 with the grade 3/4}
\item{All milestones have been reached and deliverables achieved within the 1st Reporting Period timeframe}
\item{The setting up of a new version of the \ODK website, with a more end-user friendly interface}
\item{Success in the recruitment of highly qualified staff}
\item{Many successful workshops open to different communities organisedl}
\item{An Advisory Board and Quality Review Board set up to control the quality and the relevance of the software development relative to the end-user needs.}
\end{itemize}

Concerning the recruitment: the strategies we used (tayloring of the
positions according to the known pool of potential candidates, in
particular among previous related projects, strong advertisement, ...)
seem to have paid off, and we are really happy with the top notch
quality of our recruits. However, despite many steps to foster women
applications to apply (e.g. through reaching personally toward
potential candidates or including women in the committees), we had
almost no female candidate, and none made it to the short list. This
is alas unsurprising in the very tight segment of experienced research
software engineers for mathematics on temporary positions which is
highly gender imbalanced; this is nevertheless a failure.

\paragraph{\longtaskref{management}{project-finance-management}}

A consortium agreement was signed
between partners, stating precise rules about topics such as:
responsibilities, governance, access to results and the background
included.  This consortium agreement respects the state of mind of the opensource software communities and does not plan to commercially exploit the Intellectual Property produced in the frame of \ODK.

During the 1st Reporting Period, a kick-off meeting was organised in Orsay, followed by
3 progress meetings at which partners presented status reports, and
the steering committee got together.  The first progress meeting was
organised in St Andrews (January 2016), the second one was located
in Bremen (June 2016), and the last one in Edinburgh (January 2017). The second meeting coincided with the interim project
review, planned at month 9, where deliverables due by then were
presented to the Project Officer and Reviewers. The \ODK\ project was
granted the grade 3 out of 4 for this interim review: ``Good progress
(the project has achieved most of its objectives and technical goals
for the period with relatively minor deviations)''.

As planned in WP1, \site{PS} has been coordinating \ODK.  The \site{PS} relevant administration body, the D.A.R.I. (Direction des Activités de Recherche et de l'Innovation) and its finance service took care of the budget repartition in November-December 2015. The D.A.R.I., with Florence Bougeret, is also leading the Financial Statement of this reporting period. Due to the length of the first reporting period (18 months), the \site{PS} administration had decided to organise an internal and interim
breakdown of costs at the middle of the Reporting Period. This exercise aimed at raising potential questions
from partners early on and to make sure partners do follow the EC
rules for the eligibility of costs. The coordinator is therefore confident that all partners will be able to declare their costs for this Reporting Period.

\site{PS} also lead the amendment number AMD-676541-5 which added to the consortium \site{UG}, and is currently leading the amendment number AMD-676541-13 for the addition of two new sites: Friedrich-Alexander Universit\"{a}t Erlangen-NÃŒrnberg and the new laboratory European XFEL. The addition of these three new partners is due to the moving of key permanent and/or non-permanent researchers who are key personnel for the success of \ODK. A collateral effect is the termination of \site{USO}'s participation to the consortium since no relevant staff for \ODK remained in this institution.


Concerning the communication, intern communication tools are described
in \longdelivref{management}{infrastructure}. As for external
communication the website for the project has been continuously
updated with new content, and virtually all work in progress is openly
accessible on the Internet to external experts and contributors (for
example through open source software on Github). A new version of the
website was released on the 15/03/2017. Its end-user friendly
interface and content makes it a tool not only for internal
communication but very much for dissemination and progress tracking by
the reviewers and the community.

Furthermore \longdelivref{management}{data-plan1} gave a first version of the management of data produced \ODK.

\paragraph{\longtaskref{management}{project-quality-management}}
The Quality Assurance Plan is described in detail in \longdelivref{management}{ipr}. We will describe the main points below.
\site{PS} launched a Quality Review Board which is chaired by Hans Fangohr. The four members of the board have a track record of caring about the quality of software in computational science. This board is responsible for ensuring key deliverables do reach their original goal and that best practice is followed in the writing process as well as in the innovation production process.
The board will meet after the end of each Reporting Period (RP), and before the Review following that RP.

The other structure supporting \ODK to ensure the quality of the
infrastructure is the End-user group that is composed of some members
of the Advisory Board. It is composed of seven members:

\begin{itemize}
\item{Lorena Barba from the George Washington University}
\item{Jacques Carette from the McMaster University}
\item{Istvan Csabai from the Eötvös University Budapest}
\item{Françoise Genova from the Observatoire de Strasbourg}
\item{Konrad Hinsen from the Centre de Biophysique Moléculaire}
\item{William Stein, CEO of SageMath Inc.}
\item{Paul Zimmermann from the INRIA}
\end{itemize}

This Advisory Board being composed of Academics and/or software
developers from different backgrounds, countries and communities, it
will be a strong asset to understand the needs of a variety of
end-user profiles. This Technical Report for the first Reporting
Period will be the first occasion to ask for their feedback on the
potential of the VRE and our strategy to promote its use around the
world. According to the Consortium Agreement, all Advisory Board
members have signed a lightweight Non-Disclosure Agreement with the
consortium.

\site{PS} has also been managing risks. In \delivref{management}{ipr}
all potential risks were assessed by the Coordinator at Month 12. Here
is a brief update on Risk 1 concerning the recruitment of highly
qualified staff. This risk has been globally well managed thanks to a
flexible workplan enabling adjustments in the timing of some tasks or
deliverables, and thanks to legal actions taken by the Coordinator to
allow key personnel, permanent or not, to remain in the Consortium
even though their positions changed. The addition of the three
partners is representative of these actions. The assessment for the
other risks remain valid at Month 18, and we refer to
\delivref{management}{ipr} for details.

\paragraph{\longtaskref{management}{project-innovation-management}}
\longdelivref{management}{imp1} was produced at month 18 and is mainly focused on: 

\begin{itemize}
\item{The open source aspect of the innovation produced within \ODK}
\item{The various implementation processes the project is dealing with}
\item{The strategy to match end-users needs with the promoted VRE}.
\end{itemize}

  The second version of the Innovation Management Plan will add content to explain all the
  innovations that the VRE is bringing to end-users. However the open source approach and
  the "by users for users" development process will not change.  One
  of the assessed risks for \ODK is to have different groups not forming effective teams. Put
  in other words, having developers of the different pieces of software working solely for
  the benefit of the programme they were initially working on and for. This risk is
  tackled by the Coordinator in order to reach the final goals of the VRE which are the
  unification of open source tools with overlapping functionality, the simplification of
  the tools for end-users without coding expertise, and the development of user-friendly
  interfaces. For this, the Scientific Coordinator is for example willfully pushing
  for joint actions and
  workshops. Even if it takes time to bend some of the old implementation processes and coding habits,
  more and actions are taken by \ODK participants from different communities to work
  together. More information on joint workshops can be found in the section below.

\clearpage

