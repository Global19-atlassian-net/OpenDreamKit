\subsubsection{WorkPackage 4: User Interfaces}
%Explain, task per task, the work carried out in WP during the reporting period giving details of the work carried out by each beneficiary involved.

The objective of WorkPackage 4 is to provide modern, robust, and flexible user interfaces for
computation, supporting real-time sharing, integration with collaborative problem-solving,
multilingual documents, paper writing and publication, links to databases, etc. This work is focused primarily around the \Jupyter project, in the form of:

\begin{itemize}
    \item Enhancing existing \Jupyter tools (\taskref{UI}{notebook-collab})
    \item Building new tools in the \Jupyter ecosystem (\taskref{UI}{notebook-verification}, \taskref{UI}{notebook-collab}, \taskref{UI}{vis3d})
    \item Improving the use of \ODK components in \Jupyter and \Sage environments (\taskref{UI}{ipython-kernels}, \taskref{UI}{sage-sphinx}, \taskref{UI}{dynamic-inspect}, \taskref{UI}{pari-python})
    \item Demonstrating effectiveness of WorkPackage 4 results in specific scientific applications (\taskref{UI}{cfd-vis}, \taskref{UI}{oommf-py-ipython-attributes}, \taskref{UI}{oommf-nb-ve}, \taskref{UI}{oommf-tutorial-and-documentation})
    \item Work on Active Documents, which have some goals in common with \Jupyter notebooks (\taskref{UI}{structdocs}, \taskref{UI}{mathhub})
\end{itemize}

Progress across WorkPackage 4 has been highly successful thus far.
Several new software packages have been created,
and existing projects in the \Sage and \Jupyter communities have been improved toward sustainability to serve \ODK objectives.

\paragraph{\longtaskref{UI}{notebook-collab}}

\longdelivref{UI}{jupyter-collab} has been delivered in the form of a new \Jupyter package, nbdime,
enabling easier collaboration on notebooks via version control systems such as Git. This project
was presented at the major Scientific Python conferences SciPy US in July 2016 and EuroSciPy in August 2016,
and has been met with enthusiasm from the scientific Python community for its prospect of solving a
longstanding difficulty in working with notebooks.

The \JupyterHub package has received significant updates and further development, specifically a
\emph{Services extension point}, which enables shared workspaces for collaboration, a step on the path
toward real-time collaboration for \delivref{UI}{jupyter-live-collab}.
  \ednote{FR: M0-24: this should be completed by now; \color{red} MK: Jacobs was never part of this,
    delete this? }


\paragraph{\longtaskref{UI}{structdocs}}

Active structured documents are a common need with many use cases, and has many potential solutions.
Requirements and venues for collaborations were explored through discussions between participants,
in particular at the occasion of \href{https://wiki.sagemath.org/days77/}{Sage Days 77} workshop
(see the \href{https://wiki.sagemath.org/days77/live-structured-documents}{notes}), and June's ODK
meeting in Bremen. The findings were reported in \longdelivref{UI}{adstex}.

In \longdelivref{UI}{adcomp}, We have presented a general framework for in-situ computation in active documents. This is
a contribution towards using mathematical documents -- the traditional form mathematicians
interact with mathematical knowledge and computations -- as a user interface for a
mathematical virtual research environments. This is also a step towards integrating the
two main UI frameworks under investigation in the \ODK project: \Jupyter notebooks and
active documents -- see~\delivref{UI}{adstex} -- at a conceptual level. The system is
prototypical at the moment, but can already be embedded into active documents via a
Javascript framework and is ready for use in the \ODK project. The user interface and \SCSCP
connections are quite fresh and need substantial testing and optimizations.

\textcolor{red}{This }

  \ednote{M0-24: this should be completed by now}


\paragraph{\longtaskref{UI}{mathhub}}

One of the most prominent features of a virtual research environment (VRE) is a unified user interface. The \ODK approach is to create a mathematical VRE by integrating various pre-existing mathematical software systems. There are two approaches that can serve as a basis for the \ODK UI: computational notebooks and active documents. The former allows for mathematical text around the computation cells of a real-eval-print loop of a mathematical software system and the latter makes semantically annotated documents active.

\MathHub is a portal for active mathematical documents ranging from formal libraries of theorem provers to informal but rigorous mathematical documents lightly marked up by preserving LaTeX markup.

As the authoring, maintenance, and curation of theory-structured mathematical ontologies and the transfer of mathematical knowledge via active documents are an important part of the \ODK VRE toolkit, the editing facilities in \MathHub play a great role for the project,
as delivered in \longdelivref{UI}{mathhub-editing}.
  \ednote{M12-46: just extend the progress}

\paragraph{\longtaskref{UI}{oommf-nb-ve}} % M25-28

  \ednote{M25-28: this should be completed by now}


\clearpage
